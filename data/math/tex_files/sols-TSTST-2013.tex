\documentclass[11pt]{scrartcl}
\usepackage[sexy]{evan}
\ihead{\footnotesize\textbf{\thetitle}}
\ohead{\footnotesize\theauthor}
\begin{document}

\title{TSTST 2013 Solution Notes}
\subtitle{Lincoln, Nebraska}
\date{\today}

\maketitle
\begin{abstract}
This is a compilation of solutions
for the 2013 TSTST.
Some of the solutions are my own work,
but many are from the official solutions provided by the organizers
(for which they hold any copyrights),
and others were found by users on the Art of Problem Solving forums.

These notes will tend to be a bit more advanced and terse than the ``official''
solutions from the organizers.
In particular, if a theorem or technique is not known to beginners
but is still considered ``standard'', then I often prefer to
use this theory anyways, rather than try to work around or conceal it.
For example, in geometry problems I typically use directed angles
without further comment, rather than awkwardly work around configuration issues.
Similarly, sentences like ``let $\mathbb{R}$ denote the set of real numbers''
are typically omitted entirely.

Corrections and comments are welcome!
\end{abstract}
\tableofcontents
\newpage

\addtocounter{section}{-1}
\section{Problems}
\begin{enumerate}[\bfseries 1.]
\item %% Problem 1
Let $ABC$ be a triangle and $D$, $E$, $F$ be the midpoints of arcs $BC$, $CA$, $AB$ on the circumcircle.
Line $\ell_a$ passes through the feet of the perpendiculars
from $A$ to $\ol{DB}$ and $\ol{DC}$.
Line $m_a$ passes through the feet of the perpendiculars from $D$ to $\ol{AB}$ and $\ol{AC}$.
Let $A_1$ denote the intersection of lines $\ell_a$ and $m_a$.
Define points $B_1$ and $C_1$ similarly.
Prove that triangles $DEF$ and $A_1B_1C_1$ are similar to each other.

\item %% Problem 2
A finite sequence of integers $a_1, a_2, \dots, a_n$ is called
\emph{regular} if there exists a real number $x$ satisfying
\[ \left\lfloor kx \right\rfloor = a_k \quad \text{for } 1 \le k \le n. \]
Given a regular sequence $a_1, a_2, \dots, a_n$, for $1 \le k \le n$ we say that
the term $a_k$ is \emph{forced} if the following condition is satisfied:
the sequence \[ a_1, \; a_2, \; \dots, \; a_{k-1}, \; b \]
is regular if and only if $b = a_k$.

Find the maximum possible number of forced terms
in a regular sequence with $1000$ terms.

\item %% Problem 3
Divide the plane into an infinite square grid by drawing
all the lines $x=m$ and $y=n$ for $m,n \in \ZZ$.
Next, if a square's upper-right corner has both coordinates even, color it black;
otherwise, color it white (in this way, exactly $1/4$ of the squares are black
and no two black squares are adjacent).
Let $r$ and $s$ be odd integers,
and let $(x,y)$ be a point in the interior of any white square
such that $rx-sy$ is irrational.
Shoot a laser out of this point with slope $r/s$;
lasers pass through white squares and reflect off black squares.
Prove that the path of this laser will from a closed loop.

\item %% Problem 4
Circle $\omega$, centered at $X$, is internally tangent to circle $\Omega$,
centered at $Y$, at $T$.
Let $P$ and $S$ be variable points on $\Omega$ and $\omega$,
respectively, such that line $PS$ is tangent to $\omega$ (at $S$).
Determine the locus of $O$ -- the circumcenter of triangle $PST$.

\item %% Problem 5
Let $p$ be a prime.
Prove that in a complete graph with $1000p$ vertices
whose edges are labelled with integers,
one can find a cycle whose sum of labels is divisible by $p$.

\item %% Problem 6
Let $\NN$ be the set of positive integers.
Find all functions $f \colon \NN \to \NN$ that satisfy the equation
\[ f^{abc-a}(abc) + f^{abc-b}(abc) + f^{abc-c}(abc) = a + b + c \]
for all $a,b,c \ge 2$. (Here $f^k$ means $f$ applied $k$ times.)

\item %% Problem 7
A country has $n$ cities, labelled $1,2,3,\dots,n$.
It wants to build exactly $n-1$ roads between certain pairs of cities
so that every city is reachable from every other city via some sequence of roads.
However, it is not permitted to put roads between pairs of cities
that have labels differing by exactly $1$,
and it is also not permitted to put a road between cities $1$ and $n$.
Let $T_n$ be the total number of possible ways to build these roads.
\begin{enumerate}[(a)]
  \ii For all odd $n$, prove that $T_n$ is divisible by $n$.
  \ii For all even $n$, prove that $T_n$ is divisible by $n/2$.
\end{enumerate}

\item %% Problem 8
Define a function $f \colon \NN \to \NN$ by $f(1) = 1$,
$f(n+1) = f(n) + 2^{f(n)}$ for every positive integer $n$.
Prove that $f(1)$, $f(2)$, \dots, $f(3^{2013})$
leave distinct remainders when divided by $3^{2013}$.

\item %% Problem 9
Let $r$ be a rational number in the interval $[-1,1]$
and let $\theta = \cos^{-1} r$.
Call a subset $S$ of the plane good if $S$ is unchanged
upon rotation by $\theta$ around any point of $S$
(in both clockwise and counterclockwise directions).
Determine all values of $r$ satisfying the following property:
The midpoint of any two points in a good set also lies in the set.

\end{enumerate}
\pagebreak

\section{Solutions to Day 1}
\subsection{TSTST 2013/1}
\textsl{Available online at \url{https://aops.com/community/p3181479}.}
\begin{mdframed}[style=mdpurplebox,frametitle={Problem statement}]
Let $ABC$ be a triangle and $D$, $E$, $F$ be the midpoints of arcs $BC$, $CA$, $AB$ on the circumcircle.
Line $\ell_a$ passes through the feet of the perpendiculars
from $A$ to $\ol{DB}$ and $\ol{DC}$.
Line $m_a$ passes through the feet of the perpendiculars from $D$ to $\ol{AB}$ and $\ol{AC}$.
Let $A_1$ denote the intersection of lines $\ell_a$ and $m_a$.
Define points $B_1$ and $C_1$ similarly.
Prove that triangles $DEF$ and $A_1B_1C_1$ are similar to each other.
\end{mdframed}
In fact, it is true for any points $D$, $E$, $F$ on the circumcircle.
More strongly we contend:
\begin{claim*}
  Point $A_1$ is the midpoint of $\ol{HD}$.
\end{claim*}
\begin{proof}
  Lines $m_a$ and $\ell_a$ are Simson lines,
  so they both pass through the point $(a+b+c+d)/2$
  in complex coordinates.
\end{proof}

\begin{center}
  \begin{asy}
    pair A = dir(130);
    pair B = dir(200);
    pair C = dir(-20);
    pair D = dir(-90);
    pair P1 = foot(A,B,D);
    pair P2 = foot(A,C,D);
    pair Q1 = foot(D,A,B);
    pair Q2 = foot(D,A,C);
    pair H = A+B+C;
    draw(H--D, deepgreen);
    draw(unitcircle);
    filldraw(A--B--C--cycle, lightcyan + opacity(0.1), blue);
    draw(B--D--C, blue);
    draw(P1--A--P2, dashed+red);
    draw(Q1--D--Q2, dashed+orange);
    draw(P1--P2, red);
    draw(Q1--Q2, orange);
    draw(P1--B, blue);
    draw(Q1--B, blue);
    dot(Q1, orange);
    dot(Q2, orange);
    dot(P1, red);
    dot(P2, red);
    dot("$A_1$", (A+B+C+D)/2, dir(-90));
    dot("$A$", A, A);
    dot("$B$", B, B);
    dot("$C$", C, C);
    dot("$D$", D, D);
    dot("$H$", H, dir(90));
  \end{asy}
\end{center}

Hence $A_1 B_1 C_1$ is similar to $DEF$ through a homothety at $H$ with
ratio $\half$.
\pagebreak

\subsection{TSTST 2013/2}
\textsl{Available online at \url{https://aops.com/community/p3181480}.}
\begin{mdframed}[style=mdpurplebox,frametitle={Problem statement}]
A finite sequence of integers $a_1, a_2, \dots, a_n$ is called
\emph{regular} if there exists a real number $x$ satisfying
\[ \left\lfloor kx \right\rfloor = a_k \quad \text{for } 1 \le k \le n. \]
Given a regular sequence $a_1, a_2, \dots, a_n$, for $1 \le k \le n$ we say that
the term $a_k$ is \emph{forced} if the following condition is satisfied:
the sequence \[ a_1, \; a_2, \; \dots, \; a_{k-1}, \; b \]
is regular if and only if $b = a_k$.

Find the maximum possible number of forced terms
in a regular sequence with $1000$ terms.
\end{mdframed}
The answer is $985$. WLOG, by shifting $a_1 = 0$ (clearly $a_1$ isn't forced).
Now, we construct regular sequences inductively using the following procedure.
Start with the inequality \[ \tfrac01 \le x < \tfrac11. \]
Then for each $k = 2, 3, \dots, 1000$ we perform the following procedure.
If there is no fraction of the form $F = \frac mk$ in the interval $A \le x < B$,
then $a_k$ is forced,
and the interval of possible $x$ values does not change.
Otherwise, $a_k$ is not forced,
and we pick a value of $a_k$ and update the interval accordingly.

The theory of \href{https://en.wikipedia.org/wiki/Farey_sequence}{Farey sequences}
tells us that when we have a stage $\tfrac ab \le x < \tfrac cd$
then the next time we will find a fraction
in that interval is exactly $\tfrac{a+c}{b+d}$ (at time $k=b+d$),
and it will be the only such fraction.

So essentially, starting with $\tfrac 01 \le x < \tfrac 11$
we repeatedly replace one of the endpoints of the intervals with the mediant,
until one of the denominators exceeds $1000$;
we are trying to minimize the number of non-forced terms,
which is the number of denominators that appear in this process.
It is not hard to see that this optimum occurs by always replacing the smaller of the denominators,
so that the sequence is
\begin{align*}
 \tfrac{0}{1} &\le x < \tfrac{1}{1} \\
 \tfrac{0}{1} &\le x < \tfrac{1}{2} \\
 \tfrac{1}{3} &\le x < \tfrac{1}{2} \\
 \tfrac{1}{3} &\le x < \tfrac{2}{5} \\
 \tfrac{3}{8} &\le x < \tfrac{2}{5} \\
 \tfrac{3}{8} &\le x < \tfrac{5}{13}
\end{align*}
and so on; we see that the non-forced terms in this optimal configuration
are exactly the Fibonacci numbers.
There are $15$ Fibonacci numbers less than $1000$,
hence the answer $1000 - 15 = 985$.
\pagebreak

\subsection{TSTST 2013/3}
\textsl{Available online at \url{https://aops.com/community/p3181481}.}
\begin{mdframed}[style=mdpurplebox,frametitle={Problem statement}]
Divide the plane into an infinite square grid by drawing
all the lines $x=m$ and $y=n$ for $m,n \in \ZZ$.
Next, if a square's upper-right corner has both coordinates even, color it black;
otherwise, color it white (in this way, exactly $1/4$ of the squares are black
and no two black squares are adjacent).
Let $r$ and $s$ be odd integers,
and let $(x,y)$ be a point in the interior of any white square
such that $rx-sy$ is irrational.
Shoot a laser out of this point with slope $r/s$;
lasers pass through white squares and reflect off black squares.
Prove that the path of this laser will from a closed loop.
\end{mdframed}
Here is Sammy Luo's solution.
Fix the speed of light at $\sqrt{r^2+s^2}$ units per second.
We prove periodicity every six seconds.

We re-color the white squares as red, blue, or green
according as to whether they have a black square directly
to the left/right, above/below, or neither, as shown below.
Finally, we fix time zero to be a moment just before the laser passes
a horizontal (WLOG) lattice line (not necessarily a wall).
Shown below is an example for $(r,s) = (3,5)$.

\begin{center}
\begin{asy}
size(11cm);
int Mx = 3;
int My = 6;
for (int i=0; i<=Mx; ++i) {
for (int j=0; j<=My; ++j) {
fill(shift(2*i, 2*j) * unitsquare, opacity(0.1)+lightgreen);
if (i != Mx) { fill(shift(2*i+1, 2*j) * unitsquare, opacity(0.1)+lightred); }
if (j != My) { fill(shift(2*i, 2*j+1) * unitsquare, opacity(0.1)+lightblue); }
if ((i != Mx) && (j != My)) { fill(shift(2*i+1, 2*j+1) * unitsquare, black); }
}
}


pair ne_until_y(pair O, real y) {
  return extension(O, O+(3,5), (0,y), (1,y) );
}
pair ne_until_x(pair O, real x) {
  return extension(O, O+(3,5), (x,0), (x,1) );
}
pair se_until_y(pair O, real y) {
  return extension(O, O+(-3,5), (0,y), (1,y) );
}
pair se_until_x(pair O, real x) {
  return extension(O, O+(-3,5), (x,0), (x,1) );
}


pair[] P = new pair[10];
P[0] = (2.0, 0.84);
P[1] = ne_until_y(P[0], 3);
P[2] = se_until_y(P[1], 2);
P[3] = ne_until_x(P[2], 5);
P[4] = se_until_x(P[3], 4);
P[5] = ne_until_x(P[4], 5);
P[6] = se_until_y(P[5], 9);
P[7] = ne_until_y(P[6], 8);
P[8] = se_until_y(P[7], 11);
P[9] = ne_until_x(P[8], 1);

for (int i=0; i<=8; ++i) { draw(P[i]--P[i+1], orange); }

transform t = yscale(-1)*rotate(-90);
pair[] Q = new pair[8];
Q[0] = t*P[0];
Q[1] = t*ne_until_y(P[0], 1);
Q[2] = t*ne_until_y(P[0], 2);
Q[3] = t*ne_until_x(P[0], 3);
Q[4] = t*ne_until_y(P[0], 3); // = P[1]
Q[5] = t*se_until_y(P[1], 2);
Q[6] = t*ne_until_x(P[2], 4);
Q[7] = t*ne_until_y(P[2], 3);

// for (int i=0; i<=7; ++i) { dot(Q[i], lightblue); }

picture rotated = t*currentpicture;
currentpicture.erase();
add(rotated);

dotfactor *= 0.5;
defaultpen(fontsize(6pt));
dot("$h$", Q[0], dir(-90), red);
dot("$v$", Q[1], dir(90), orange);
dot("$v$", Q[2], dir(-45), orange);
dot("$h$", Q[3], dir(-45), orange);
dot("$v$", Q[4], dir(180)*2, orange);
dot("$v$", Q[5], dir(105)*2, orange);
dot("$h$", Q[6], dir(90), orange);
dot("$v$", Q[7], dir(90), orange);

dot(t*P[3], red);
dot(t*se_until_x(P[5], 4), red);
dot(t*P[9], red);
\end{asy}
\end{center}

The main idea is to keep track of every time the laser passes
a lattice line (again, not necessarily a wall).
There are four possible types of events:
\begin{itemize}
  \ii A horizontal $h$ event where the laser switches from red to green
  (or vice-versa);
  \ii A horizontal $h$ event where the laser rebounds off a wall,
  remaining in a blue square,
  but flips the $x$-component of its velocity;
  \ii A vertical $v$ event where the laser switches from blue to green
  (or vice-versa)
  \ii A vertical $v$ event where the laser rebounds off a wall,
  remaining in a red square,
  but flips the $y$-component of its velocity.
\end{itemize}

The first key observation is that:
\begin{claim*}
  In the first second, the laser will encounter exactly
  $r$ horizontal events and $s$ vertical events.
  In every second after that,
  the same sequence of $r+s$ events occurs.
\end{claim*}
\begin{proof}
  Bouncing off a wall doesn't change this
  as opposed to if the laser had passed through the wall.
\end{proof}
We let the \emph{key-word} be the sequence $w$ of $r+s$ letters
corresponding to the sequence.
For example, the picture above denotes an example
with keyword $w = hvvhvvhv$;
so no matter what, every second,
the laser will encounter eight lattice lines,
which are horizontal and vertical in that order.

\begin{claim*}
  Color is periodic every $3$ seconds.
\end{claim*}
\begin{proof}
  The free group generated by $h$ and $v$
  acts on the set $\{R,G,B\}$ of colors in an obvious way;
  consider this right action.
  First we consider the color of the square after each second.
  Note that with respect to color,
  each letter is an involution;
  so as far as color changes are concerned,
  it's enough to work with the reduced word $w'$
  obtained by modding out by $h^2 = 1$ and $v^2 = 1$.
  (For example, $w' = hv$ in our example.)
  In general, $w' = (hv)^k$ or $w' = (vh)^k$,
  for some odd integer $k$ (since $k \equiv r \equiv s \equiv 1 \pmod 2$).
  Now we see that the action of $hv$
  on the set of colors is
  $\text{red} \mapsto \text{blue} \mapsto \text{green} \mapsto \text{red}$,
  and similarly for $vh$ (being the inverse).
  This implies that the color is periodic every three seconds.
\end{proof}

Now in a 3-second period,
consider the $3r$ horizontal events
and $3s$ vertical events (both are odd).
In order for the color to remain the same
(as the only color changes are $R \leftrightarrow G$ for $h$
and $B \leftrightarrow G$ for $v$)
there must have been an even number of color swaps for each orientation.
Therefore there was an odd number of wall collisions of each orientation.
So, the laser is pointing in the opposite direction at the end of 3 seconds.

Finally, let $x_t$ be the fractional part of the $x$
coordinate after $t$ seconds
(the $y$-coordinate is always zero by our setup at these moments).
Note that
\[ x_{t+1} =
  \begin{cases}
    x_t & \text{even number of vertical wall collisions} \\
    1 - x_t & \text{odd numbers of vertical wall collisions}
  \end{cases}
\]
Since over the there seconds there were
an odd number of vertical collisions;
it follows $x_3 = 1 - x_0$.
Thus at the end of three seconds,
the laser is in a symmetric position
from the start;
and in $6$ seconds it will form a closed loop.
\pagebreak

\section{Solutions to Day 2}
\subsection{TSTST 2013/4}
\textsl{Available online at \url{https://aops.com/community/p3181482}.}
\begin{mdframed}[style=mdpurplebox,frametitle={Problem statement}]
Circle $\omega$, centered at $X$, is internally tangent to circle $\Omega$,
centered at $Y$, at $T$.
Let $P$ and $S$ be variable points on $\Omega$ and $\omega$,
respectively, such that line $PS$ is tangent to $\omega$ (at $S$).
Determine the locus of $O$ -- the circumcenter of triangle $PST$.
\end{mdframed}
The answer is a circle centered at $Y$
with radius $\sqrt{YX \cdot YT}$,
minus the two points on line $XY$ itself.

We let $PS$ meet $\Omega$ again at $P'$,
and let $O'$ be the circumcenter of $\triangle TPS'$.
Note that $O'$, $X$, $O$ are collinear on the perpendicular
bisector of line $\ol{TS}$
Finally, we let $M$ denote the arc midpoint of $PP'$
which lies on line $TS$ (by homothety).

\begin{center}
\begin{asy}
import graph; size(9cm);
pen qqffff = rgb(0.,1.,1.); pen qqwuqq = rgb(0.,0.39215,0.); pen ffdxqq = rgb(1.,0.84313,0.); pen yqqqqq = rgb(0.50196,0.,0.);
pair X = (-2.,0.), Y = (2.,0.), T = (-9.50185,0.), P = (7.49229,10.10580), O = (5.46686,-5.82994), M = (-0.78960,11.15843);
pair S = (-3.81946,7.27786);
pair Pp = (-7.60181,6.33228);
pair Op = (-4.49671,1.94937);
fill(T--S--P--cycle, palegreen);
fill(T--Op--Y--cycle, palegreen);
fill(T--S--Pp--cycle, palecyan);
fill(T--O--Y--cycle, palecyan);
draw(circle(X, 7.50185), linewidth(0.6) + red);
draw(circle(Y, 11.50185), linewidth(0.6) + red);
draw(circle(Y, 6.78287), linewidth(0.6) + qqwuqq);
draw(T--Op--Y--O--T--Y, grey);
draw(T--Pp--M--P--T--M, black);
draw(Pp--P, black);
draw(Op--O, grey);
dot("$X$", X, dir((8.005, 21.422)));
dot("$Y$", Y, dir((8.687, 21.422)));
dot("$T$", T, dir(T-Y));
dot("$S$", S, dir(135));
dot("$P$", P, dir(P-Y));
dot("$O$", O, dir(-45));
dot("$P'$", Pp, dir(Pp-Y));
dot("$O'$", Op, dir(40));
dot("$M$", M, dir((8.912, 22.013)));
\end{asy}
\end{center}
By three applications of Salmon theorem, we have the following spiral
similarities all centered at $T$:
\begin{align*}
  \triangle TSP &\overset{+}{\sim} \triangle TO'Y \\
  \triangle TP'S &\overset{+}{\sim} \triangle TYO \\
  \triangle TP'P &\overset{+}{\sim} \triangle TO'O.
\end{align*}
However, the shooting lemma also gives us two similarities:
\begin{align*}
  \triangle TP'M &\overset{+}{\sim} \triangle TSP \\
  \triangle TMP &\overset{+}{\sim} \triangle TP'S.
\end{align*}
Putting everything together, we find that
\[ TP'MP \overset{+}{\sim} TO'YO. \]
Then by shooting lemma, $YO'^2 = YX \cdot YT$, so $O$
indeed lies on the claimed circle.

As the line $\ol{O'O}$ may be any line through $X$ other than line $XY$
(one takes $S$ to be the reflection of $T$ across this line)
one concludes the only two non-achievable points are
the diametrically opposite ones on line $XY$ of this circle
(because this leads to the only degenerate situation where $S=T$).
\pagebreak

\subsection{TSTST 2013/5}
\textsl{Available online at \url{https://aops.com/community/p3181483}.}
\begin{mdframed}[style=mdpurplebox,frametitle={Problem statement}]
Let $p$ be a prime.
Prove that in a complete graph with $1000p$ vertices
whose edges are labelled with integers,
one can find a cycle whose sum of labels is divisible by $p$.
\end{mdframed}
Select $p-1$ disjoint triangles arbitrarily.  If any of these triangles have $0$ sum modulo $p$ we are done.  Otherwise, we may label the vertices $u_i$, $x_i$, and $v_i$ (where $1 \le i \le p-1$) in such a way that $u_ix_i + x_iv_i \neq u_iv_i$.

Let $A_i = \left\{ u_ix_i+x_iv_i, u_iv_i \right\}$.  We can show that $\left\lvert A_1 + A_2 + \dots + A_t \right\rvert \ge \min \left\{ p,t+1 \right\}$ for each $1 \le t \le p-1$, by using induction on $t$ alongside Cauchy-Davenport.  So, $A_1 + A_2 + \dots + A_{p-1}$ spans all of $\ZZ_p$.  All that's left to do is join the triangles together to form a cycle, and then delete either $u_ix_i,x_iv_i$ or $u_iv_i$ from each triangle in such a way that the final sum is $0$ mod $p$.
\pagebreak

\subsection{TSTST 2013/6}
\textsl{Available online at \url{https://aops.com/community/p3181484}.}
\begin{mdframed}[style=mdpurplebox,frametitle={Problem statement}]
Let $\NN$ be the set of positive integers.
Find all functions $f \colon \NN \to \NN$ that satisfy the equation
\[ f^{abc-a}(abc) + f^{abc-b}(abc) + f^{abc-c}(abc) = a + b + c \]
for all $a,b,c \ge 2$. (Here $f^k$ means $f$ applied $k$ times.)
\end{mdframed}
The answer is $f(n) = n-1$ for $n \ge 3$ with $f(1)$ and $f(2)$ arbitrary;
check these work.

\begin{lemma*}
 We have $f^{t^2-t}(t^2) = t$ for all $t$.
\end{lemma*}
\begin{proof}
 We say $1 \le k \le 8$ is good if $f^{t^9-t^k}(t^9) = t^k$ for all $t$.
 First, we observe that
 \[ f^{t^9-t^3}(t^9) = t^3 \quad\text{and}\quad f^{t^3-t}(t^3) = t
   \implies f^{t^9-t}(t^9) = t.  \]
 so $k=1$ and $k=3$ are good.
 Then taking $(a,b,c) = (t,t^4,t^4)$,
 $(a,b,c) = (t^2, t^3, t^4)$
 gives that $k = 4$ and $k = 2$ are good, respectively.
 % $(a,b,c) = (t,t^3,t^5)$,  gives $5$ good
% Similarly, $(a,b,c) = (t,t^2,t^6)$ and $(a,b,c) = (t,t,t^7)$
% gives $6$ and $7$ are good.
 The lemma follows from this $k=1$ and $k=2$ being good.
\end{proof}

Now, letting $t = abc$ we combine
\begin{align*}
 f^{t-a}(a) + f^{t-b}(b) + f^{t-c}(c) &= a+b+c \\
 f^{t^2-ab} (t^2) + f^{t^2-t} (t^2) + f^{t^2-c} (t^2) &= ab + t + c \\
 \implies \left[ f^{t-a}(t) - a \right]
 + \left[ f^{t-b}(t) - b \right]
 &=
 \left[ f^{t-ab}(t) - ab \right]
\end{align*}
by subtracting and applying the lemma repeatedly.
In other words, we have proven the second lemma:
\begin{lemma*}
  Let $t$ be fixed, and define $g_t(n) = f^{t-n}(t) - n$ for $n < t$.
  If $a,b \ge 2$ and $ab \mid t$, $ab < t$,  then $g_t(a) + g_t(b) = g_t(ab)$.
\end{lemma*}

Now let $a,b \ge 2$ be arbitrary, and let $p > q > \max\{a,b\}$ be primes.
Suppose $s = a^p b^q$ and $t = s^2$; then
\[ p g_t(a) + q g_t(b)
  = g_t\left( a^p b^q \right) = g_t(s)
  = f^{s^2-s}(s) - s = 0.  \]
Now \[ q \mid g_t(a) > -a
  \quad\text{and}\quad p \mid g_t(b) > -b
  \implies g_t(a) = g_t(b) = 0.  \]
and so we conclude $f^{t-a}(t) = a$
and $f^{t-b}(t) = b$ for $a,b \ge 2$.

In particular, if $a = n$ and $b = n+1$
then we deduce $f(n+1) = n$ for all $n \ge 2$, as desired.

\begin{remark*}
  If you let $c = (ab)^2$ after the first lemma,
  you recover the $2$-variable version!
\end{remark*}
\pagebreak

\section{Solutions to Day 3}
\subsection{TSTST 2013/7}
\textsl{Available online at \url{https://aops.com/community/p3181485}.}
\begin{mdframed}[style=mdpurplebox,frametitle={Problem statement}]
A country has $n$ cities, labelled $1,2,3,\dots,n$.
It wants to build exactly $n-1$ roads between certain pairs of cities
so that every city is reachable from every other city via some sequence of roads.
However, it is not permitted to put roads between pairs of cities
that have labels differing by exactly $1$,
and it is also not permitted to put a road between cities $1$ and $n$.
Let $T_n$ be the total number of possible ways to build these roads.
\begin{enumerate}[(a)]
  \ii For all odd $n$, prove that $T_n$ is divisible by $n$.
  \ii For all even $n$, prove that $T_n$ is divisible by $n/2$.
\end{enumerate}
\end{mdframed}
You can just spin the tree!

Fixing $n$, the group $G = \ZZ/n\ZZ$ acts on the set of trees
by rotation (where we imagine placing $1,2,\dots,n$ along a circle).

\begin{claim*}
  For odd $n$, all trees have trivial stabilizer.
\end{claim*}
\begin{proof}
  One way to see this is to look at the degree sequence.
  Suppose $g^e$ fixes a tree $T$.
  Then so does $g^k$, for $k = \gcd(e,n)$.
  Then it follows that $n/k$ divides $\sum_v \deg v = 2n-2$.
  Since $\gcd(2n-2, n) = 1$ we must then have $k = n$.
\end{proof}

The proof for even $n$ is identical except that $\gcd(2n-2, n) = 2$
and hence each tree either has stabilizer with size $\le 2$.

There is also a proof using linear algebra,
using Kirchoff's tree formula.  (Overkill.)
\pagebreak

\subsection{TSTST 2013/8}
\textsl{Available online at \url{https://aops.com/community/p3181486}.}
\begin{mdframed}[style=mdpurplebox,frametitle={Problem statement}]
Define a function $f \colon \NN \to \NN$ by $f(1) = 1$,
$f(n+1) = f(n) + 2^{f(n)}$ for every positive integer $n$.
Prove that $f(1)$, $f(2)$, \dots, $f(3^{2013})$
leave distinct remainders when divided by $3^{2013}$.
\end{mdframed}
I'll prove by induction on $k \ge 1$ that any $3^k$
consecutive values of $f$ produce distinct residues modulo $3^k$.
The base case $k=1$ is easily checked
($f$ is always odd, hence $f$ cycles $1$, $0$, $2$ mod $3$).

For the inductive step, assume it's true up to $k$.
Since $2^\bullet \pmod{3^{k+1}}$ cycles every $2 \cdot 3^k$,
and $f$ is always odd, it follows that
\begin{align*}
  f(n+3^k) - f(n)
  &= 2^{f(n)} + 2^{f(n+1)} + \dots + 2^{f(n+3^k-1)} \pmod{3^{k+1}} \\
  &\equiv 2^1 + 2^3 + \dots + 2^{2 \cdot 3^k-1} \pmod{3^{k+1}} \\
  &= 2 \cdot \frac{4^{3^k}-1}{4-1}.
\end{align*}
Hence
\[
  f(n+3^k)-f(n) \equiv C \pmod{3^{k+1}}
  \qquad\text{where} \qquad C = 2 \cdot \frac{4^{3^k}-1}{4-1}
\]
noting that $C$ does not depend on $n$. Exponent lifting gives $\nu_3(C) = k$
hence $f(n)$, $f(n+3^k)$, $f(n+2\cdot3^k)$ differ mod $3^{k+1}$ for all $n$,
and the inductive hypothesis now solves the problem.
\pagebreak

\subsection{TSTST 2013/9}
\textsl{Available online at \url{https://aops.com/community/p3181487}.}
\begin{mdframed}[style=mdpurplebox,frametitle={Problem statement}]
Let $r$ be a rational number in the interval $[-1,1]$
and let $\theta = \cos^{-1} r$.
Call a subset $S$ of the plane good if $S$ is unchanged
upon rotation by $\theta$ around any point of $S$
(in both clockwise and counterclockwise directions).
Determine all values of $r$ satisfying the following property:
The midpoint of any two points in a good set also lies in the set.
\end{mdframed}
The answer is that $r$ has this property
if and only if $r = \frac{4n-1}{4n}$ for some integer $n$.

Throughout the solution, we will let $r = \frac ab$ with $b > 0$ and
$\gcd(a,b) = 1$. We also let
\[ \omega = e^{i\theta} = \frac ab \pm \frac{\sqrt{b^2-a^2}}{b} i. \]
This means we may work with complex multiplication in the usual way;
the rotation of $z$ through center $c$ is given by $z \mapsto \omega(z-c)+c$.

For most of our proof, we start by constructing a good set as follows.
\begin{itemize}
  \ii Start by letting $S_0 = \{0, 1\}$.
  \ii Let $S_i$ consist of $S_{i-1}$ plus
  all points that can be obtained by rotating a point of $S_{i-1}$
  through a different point of $S_{i-1}$ (with scale factor $\omega$).
  \ii Let $S_\infty = \bigcup_{i \ge 0} S_i$.
\end{itemize}
The set $S_\infty$ is the (minimal, by inclusion) good set containing
$0$ and $1$.
We are going to show that for most values of $r$,
we have $\half \notin S_\infty$.

\begin{claim*}
  If $b$ is odd, then $\half \notin S_\infty$.
\end{claim*}
\begin{proof}
  Idea: denominators that appear are always odd.

  Consider the ring
  \[ A = \ZZ_{\{b\}} = \left\{ \frac{s}{t} \mid s,t \in \ZZ, t \mid b^\infty \right\} \]
  which consists of all rational numbers
  whose denominators divide $b^\infty$.
  Then, $0, 1, \omega \in A[\sqrt{b^2-a^2}]$
  and hence $S_\infty \subseteq A[\sqrt{b^2-a^2}]$ too.
  (This works even if $\sqrt{b^2-a^2} \in \ZZ$,
  in which case $S_\infty \subseteq A = A[\sqrt{b^2-a^2}]$.)

  But $\half \notin A[\sqrt{b^2-a^2}]$.
\end{proof}

\begin{claim*}
  If $b$ is even and $|b-a| \neq 1$, then $\half \notin S_\infty$.
\end{claim*}
\begin{proof}
  Idea: take modulo a prime dividing $b-a$.

  Let $D = b^2-a^2 \equiv 3 \pmod 4$.
  Let $p$ be a prime divisor of $b-a$ with odd multiplicity.
  Because $\gcd(a,b) = 1$, we have $p \neq 2$ and $p \nmid b$.

  Consider the ring
  \[ A = \ZZ_{(p)} = \left\{ \frac{s}{t} \mid s,t \in \ZZ, p \perp t \right\} \]
  which consists of all rational numbers
  whose denominators are coprime to $p$.
  Then, $0, 1, \omega \in A[\sqrt{-D}]$
  and hence $S_\infty \subseteq A[\sqrt{-D}]$ too.

  Now, there is a well-defined ``mod-$p$'' ring homomorphism
  \[ \Psi \colon A[\sqrt{-D}] \to \FF_p \quad\text{by}\quad
    x+y\sqrt{-D} \mapsto x \bmod p \]
  which commutes with addition and multiplication (as $p \mid D$).
  Under this map,
  \[ \omega \mapsto \frac ab \bmod p = 1. \]
  Consequently, the rotation $z \mapsto \omega(z-c)+c$
  is just the identity map modulo $p$.
  In other words, the pre-image of any point in $S_\infty$
  under $\Psi$ must be either $\Psi(0) = 0$ or $\Psi(1) = 1$.

  However, $\Psi(1/2) = 1/2$ is neither of these.
  So this point cannot be achieved.
\end{proof}

\begin{claim*}
  Suppose $a = 2n-1$ and $b=2n$ for $n$ an odd integer.
  Then $\half \notin S_\infty$
\end{claim*}
\begin{proof}
  Idea: $\omega$ is ``algebraic integer'' sans odd denominators.

  This time, we define the ring
  \[ B = \ZZ_{(2)} = \left\{ \frac{s}{t} \mid s,t \in \ZZ, t \text{ odd} \right\} \]
  of rational numbers with odd denominator.
  We carefully consider the ring $B[\omega]$ where
  \[ \omega = \frac{2n-1 \pm \sqrt{1-4n}}{2n}. \]
  So $S_\infty \subseteq B[\omega]$ as $0, 1, \omega \in B[\omega]$.

  I claim that $B[\omega]$ is an integral extension of $B$;
  equivalently that $\omega$ is integral over $B$.
  Indeed, $\omega$ is the root of the monic polynomial
  \[  (T-1)^2 + \frac1n (T-1) - \frac 1n = 0 \]
  where $\frac1n \in B$ makes sense as $n$ is odd.

  On the other hand, $\half$ is not integral over $B$
  so it is not an element of $B[\omega]$.
\end{proof}

It remains to show that if $r = \frac{4n-1}{4n}$, then goods sets
satisfy the midpoint property.
Again starting from the points $z_0 = 0$, $z_1 = 1$ construct the sequence
\begin{align*}
  z_2 &= \omega(z_1-z_0) + z_0 \\
  z_3 &= \omega\inv(z_0-z_2) + z_2 \\
  z_4 &= \omega\inv(z_2-z_3) + z_3 \\
  z_5 &= \omega(z_3-z_4) + z_4
\end{align*}
as shown in the diagram below.
\begin{center}
\begin{asy}
  size(8cm);
  pair z0 = (0,0);
  pair z1 = (1,0);
  real theta = 28.96;
  pair z2 = rotate(theta, z0) * z1;
  pair z3 = rotate(-theta, z2) * z0;
  pair z4 = rotate(-theta, z3) * z2;
  pair z5 = rotate(theta, z4) * z3;
  draw(arc(z0, z1, z2), lightgreen);
  draw(arc(z3, z4, z2), lightgreen);
  draw(arc(z2, z3, z0), lightgreen);
  draw(arc(z4, z3, z5), lightgreen);
  draw(z1--z0--z2--z3--z4--z5, deepcyan);
  dot("$z_0=0$", z0, dir(-45), red);
  dot("$z_1=1$", z1, dir(-45), red);
  dot("$z_2$", z2, dir(90));
  dot("$z_3$", z3, dir(90));
  dot("$z_4=2r-1$", z4, dir(-90));
  dot("$z_5=2r-2$", z5, dir(-90));
\end{asy}
\end{center}
This construction shows that if we have the length-one segment
$\{0,1\}$ then we can construct the length-one segment $\{2r-2, 2r-1\}$.
In other words, we can shift the segment to the left by
\[ 1-(2r-1) = 2(1-r) = \frac{1}{2n}. \]
Repeating this construction $n$ times gives the desired midpoint $\half$.
\pagebreak


\end{document}
