% © Evan Chen
% Downloaded from https://web.evanchen.cc/

\documentclass[11pt]{scrartcl}
\usepackage[sexy]{evan}
\ihead{\footnotesize\textbf{\thetitle}}
\ohead{\footnotesize\href{http://web.evanchen.cc}{\ttfamily web.evanchen.cc},
    updated \today}
\title{JMO 2016 Solution Notes}
\date{\today}

\begin{document}

\maketitle

\begin{abstract}
This is a compilation of solutions
for the 2016 JMO.
Some of the solutions are my own work,
but many are from the official solutions provided by the organizers
(for which they hold any copyrights),
and others were found by users on the Art of Problem Solving forums.

These notes will tend to be a bit more advanced and terse than the ``official''
solutions from the organizers.
In particular, if a theorem or technique is not known to beginners
but is still considered ``standard'', then I often prefer to
use this theory anyways, rather than try to work around or conceal it.
For example, in geometry problems I typically use directed angles
without further comment, rather than awkwardly work around configuration issues.
Similarly, sentences like ``let $\mathbb{R}$ denote the set of real numbers''
are typically omitted entirely.

Corrections and comments are welcome!
\end{abstract}

\tableofcontents
\newpage

\addtocounter{section}{-1}
\section{Problems}
\begin{enumerate}[\bfseries 1.]
\item %% Problem 1
The isosceles triangle $\triangle ABC$, with $AB=AC$,
is inscribed in the circle $\omega$.
Let $P$ be a variable point on the arc $BC$
that does not contain $A$,
and let $I_B$ and $I_C$ denote the incenters of
triangles $\triangle ABP$ and $\triangle ACP$, respectively.
Prove that as $P$ varies, the circumcircle
of triangle $\triangle PI_{B}I_{C}$ passes through a fixed point.

\item %% Problem 2
Prove that there exists a positive integer $n < 10^6$
such that $5^n$ has six consecutive zeros in its decimal representation.

\item %% Problem 3
Let $X_1, X_2, \dots, X_{100}$ be a sequence of mutually distinct nonempty subsets of a set $S$.
Any two sets $X_i$ and $X_{i+1}$ are disjoint and their
union is not the whole set $S$,
that is, $X_i\cap X_{i+1}=\emptyset$ and $X_i\cup X_{i+1}\neq S$,
for all $i\in\{1, \dots, 99\}$.
Find the smallest possible number of elements in $S$.

\item %% Problem 4
Find, with proof, the least integer $N$ such that
if any $2016$ elements are removed
from the set $\{1, 2, \dots, N\}$,
one can still find $2016$ distinct numbers
among the remaining elements with sum $N$.

\item %% Problem 5
Let $\triangle ABC$ be an acute triangle, with $O$ as its circumcenter.
Point $H$ is the foot of the perpendicular from $A$ to line $BC$,
and points $P$ and $Q$ are the feet of the perpendiculars
from $H$ to the lines $AB$ and $AC$, respectively.

Given that \[ AH^2 = 2AO^2, \]
prove that the points $O$, $P$, and $Q$ are collinear.

\item %% Problem 6
Find all functions $f \colon \RR \to \RR$ such that
for all real numbers $x$ and $y$,
\[ (f(x)+xy) \cdot f(x-3y)
 + (f(y)+xy) \cdot f(3x-y)
 = (f(x+y))^2. \]

\end{enumerate}
\pagebreak

\section{Solutions to Day 1}
\subsection{JMO 2016/1, proposed by Ivan Borsenco, Zuming Feng}
\textsl{Available online at \url{https://aops.com/community/p6213607}.}
\begin{mdframed}[style=mdpurplebox,frametitle={Problem statement}]
The isosceles triangle $\triangle ABC$, with $AB=AC$,
is inscribed in the circle $\omega$.
Let $P$ be a variable point on the arc $BC$
that does not contain $A$,
and let $I_B$ and $I_C$ denote the incenters of
triangles $\triangle ABP$ and $\triangle ACP$, respectively.
Prove that as $P$ varies, the circumcircle
of triangle $\triangle PI_{B}I_{C}$ passes through a fixed point.
\end{mdframed}
Let $M$ be the midpoint of arc $BC$ not containing $A$.
We claim $M$ is the desired fixed point.

\begin{center}
\begin{asy}
pair A = dir(90);
pair M_B = dir(152);
pair B = M_B*M_B/A;
pair M_C = A*A/M_B;
pair C = A*A/B;
pair P = dir(250);

draw(unitcircle, blue);
filldraw(A--B--P--cycle, opacity(0.1)+lightblue, blue);
filldraw(A--C--P--cycle, opacity(0.1)+lightblue, blue);
draw(B--C, blue);

pair M = -A;
pair I_B = incenter(P, A, B);
pair I_C = incenter(P, A, C);

draw(I_B--P--I_C, blue);
draw(A--P, blue);

filldraw(incircle(A, B, P), opacity(0.05)+heavycyan, heavycyan+dotted);
filldraw(incircle(A, C, P), opacity(0.05)+heavycyan, heavycyan+dotted);


filldraw(M_B--I_B--M--cycle, opacity(0.2)+yellow, orange+1);
filldraw(M_C--I_C--M--cycle, opacity(0.2)+yellow, orange+1);

draw(circumcircle(P, I_B, I_C), red+dashed);

dot("$A$", A, dir(A));
dot("$M_B$", M_B, dir(M_B));
dot("$B$", B, dir(B));
dot("$M_C$", M_C, dir(M_C));
dot("$C$", C, dir(C));
dot("$P$", P, dir(P));
dot("$M$", M, dir(M));
dot("$I_B$", I_B, dir(160));
dot("$I_C$", I_C, dir(170));

/* TSQ Source:

A = dir 90
M_B = dir 152
B = M_B*M_B/A
M_C = A*A/M_B
C = A*A/B
P = dir 250

unitcircle blue
A--B--P--cycle 0.1 lightblue / blue
A--C--P--cycle 0.1 lightblue / blue
B--C blue

M = -A
I_B = incenter P A B R160
I_C = incenter P A C R170

I_B--P--I_C blue
A--P blue

incircle A B P 0.05 heavycyan / heavycyan dotted
incircle A C P 0.05 heavycyan / heavycyan dotted


M_B--I_B--M--cycle 0.2 yellow / orange+1
M_C--I_C--M--cycle 0.2 yellow / orange+1

circumcircle P I_B I_C red dashed

*/
\end{asy}
\end{center}

Since $\angle MPA = 90^\circ$ and ray $PA$ bisects $\angle I_{B}PI_{C}$,
it suffices to show that $MI_B = MI_C$.
Let $M_B$, $M_C$ be the second intersections of $PI_B$ and $PI_C$ with circumcircle.
Now $M_{B}I_{B} = M_{B}B = M_{C}C = M_{C}I_{C}$, and moreover $MM_B = MM_C$,
and $\angle I_{B}M_{B}M = \frac{1}{2} \widehat{PM} = \angle I_{C}M_{C}M$,
so triangles $\triangle I_{B}M_{B}M \cong \triangle I_{C}M_{C}M$.



\begin{remark}
Complex in the obvious way DOES NOT WORK,
because the usual claim (``the fixed point is arc midpoint'') is FALSE
if the hypothesis that $P$ lies in the interior of the arc is dropped.
See figure below.
\begin{center}
\begin{asy}
pair A = dir(90);
pair M_B = dir(152);
pair B = M_B*M_B/A;
pair M_C = A*A/M_B;
pair C = A*A/B;
pair P = dir(140);

draw(unitcircle, blue);
filldraw(A--B--P--cycle, opacity(0.1)+lightblue, blue);
filldraw(A--C--P--cycle, opacity(0.1)+lightblue, blue);
draw(B--C, blue);

pair M = -A;
pair I_B = incenter(P, A, B);
pair I_C = incenter(P, A, C);

draw(A--P, blue);

filldraw(incircle(A, B, P), opacity(0.05)+heavycyan, heavycyan+dotted);
filldraw(incircle(A, C, P), opacity(0.05)+heavycyan, heavycyan+dotted);

draw(circumcircle(P, I_B, I_C), red+dashed);

dot("$A$", A, dir(A));
dot("$B$", B, dir(B));
dot("$C$", C, dir(C));
dot("$P$", P, dir(P));
dot("$M$", M, dir(M));
dot("$I_B$", I_B, dir(100));
dot("$I_C$", I_C, dir(340));

/* TSQ Source:

A = dir 90
M_B := dir 152
B = M_B*M_B/A
M_C := A*A/M_B
C = A*A/B
P = dir 140

unitcircle blue
A--B--P--cycle 0.1 lightblue / blue
A--C--P--cycle 0.1 lightblue / blue
B--C blue

M = -A
I_B = incenter P A B R100
I_C = incenter P A C R340

A--P blue

incircle A B P 0.05 heavycyan / heavycyan dotted
incircle A C P 0.05 heavycyan / heavycyan dotted

circumcircle P I_B I_C red dashed

*/
\end{asy}
\end{center}
Fun story, I pointed this out to Zuming during grading;
I was the only one that realized the subtlety.
\end{remark}
\pagebreak

\subsection{JMO 2016/2, proposed by Evan Chen}
\textsl{Available online at \url{https://aops.com/community/p6213569}.}
\begin{mdframed}[style=mdpurplebox,frametitle={Problem statement}]
Prove that there exists a positive integer $n < 10^6$
such that $5^n$ has six consecutive zeros in its decimal representation.
\end{mdframed}
We will prove that $\boxed{n = 20 + 2^{19} = 524308}$ fits the bill.

First, we claim that
\[ 5^n \equiv 5^{20} \pmod{5^{20}} \qquad\text{and}\qquad
  5^n \equiv 5^{20} \pmod{2^{20}}. \]
Indeed, the first equality holds since both sides are $0 \pmod{5^{20}}$,
and the second by $\varphi(2^{20}) = 2^{19}$ and Euler's theorem.
Hence \[ 5^n \equiv 5^{20} \pmod{10^{20}}. \]

In other words, the last $20$ digits of $5^n$
will match the decimal representation of $5^{20}$, with leading zeros.
However, we have
\[ 5^{20} = \frac{1}{2^{20}} \cdot 10^{20}
  < \frac{1}{1000^2} \cdot 10^{20} = 10^{-6} \cdot 10^{20} \]
and hence those first six of those $20$ digits will all be zero.
This completes the proof!
(To be concrete, it turns out that $5^{20} = 95367431640625$
and so the last $20$ digits of $5^n$ will be $00000095367431640625$.)

\begin{remark*}
  Many of the first posts in the JMO 2016 discussion thread
  (see \url{https://aops.com/community/c5h1230514})
  claimed that the problem was ``super easy''.
  In fact, the problem was solved by only about 10\% of contestants.
\end{remark*}

\paragraph{Authorship comments.}
This problem was inspired by the observation $5^8 \equiv 5^4 \pmod{10^4}$,
i.e.\ that $5^8$ ended with $0625$.

I noticed this one day back in November, when I was lying on my bed after
a long afternoon and was mindlessly computing powers of $5$ in my head
because I was too tired to do much else.
When I reached $5^8$ I noticed for the first time that the ending
$0625$ was actually induced by $5^4$.
(Given how much MathCounts I did, I really should have known this earlier!)

Thinking about this for a few more seconds,
I realized one could obtain arbitrarily long strings of $0$'s
by using a similar trick modulo larger powers of $10$.
This surprised me, because I would have thought that if this was true,
then I would have learned about it back in my contest days.
However, I could not find any references,
and I thought the result was quite nice,
so I submitted it as a proposal for the JMO,
where I thought it might be appreciated.

The joke about six consecutive zeros is due to Zuming Feng.
\pagebreak

\subsection{JMO 2016/3, proposed by Iurie Boreico}
\textsl{Available online at \url{https://aops.com/community/p6213589}.}
\begin{mdframed}[style=mdpurplebox,frametitle={Problem statement}]
Let $X_1, X_2, \dots, X_{100}$ be a sequence of mutually distinct nonempty subsets of a set $S$.
Any two sets $X_i$ and $X_{i+1}$ are disjoint and their
union is not the whole set $S$,
that is, $X_i\cap X_{i+1}=\emptyset$ and $X_i\cup X_{i+1}\neq S$,
for all $i\in\{1, \dots, 99\}$.
Find the smallest possible number of elements in $S$.
\end{mdframed}
Solution with Danielle Wang: the answer is that $|S| \ge 8$.

\textbf{Proof of sufficiency}
Since we must have $2^{|S|} \geq 100$, we must have $|S| \geq 7$.

To see that $|S| = 8$ is the minimum possible size,
consider a chain on the set $S = \{1, 2, \dots, 7\}$
satisfying $X_i \cap X_{i+1} = \emptyset$ and $X_i \cup X_{i+1} \neq S$.
Because of these requirements any subset of size $4$ or more
can only be neighbored by sets of size $2$ or less,
of which there are $\binom 71 + \binom 72 = 28$ available.
Thus, the chain can contain no more than $29$ sets of size $4$ or more
and no more than $28$ sets of size $2$ or less.
Finally, since there are only $\binom 73 = 35$ sets of size $3$ available,
the total number of sets in such a chain
can be at most $29 + 28 + 35 = 92 < 100$.

\textbf{Construction}
We will provide an inductive construction for a
chain of subsets $X_1, X_2, \dots, X_{2^{n-1} + 1}$
of  $S = \left\{ 1, \dots, n \right\}$
satisfying $X_i \cap X_{i+1} = \varnothing$
and $X_i \cup X_{i+1} \neq S$ for each $n \geq 4$.

For $S = \{1, 2, 3, 4\}$,
the following chain of length $2^3 + 1 = 9$ will work:
\[
\begin{array}{ccccccccc}
  34 & 1 & 23 & 4 & 12 & 3 & 14 & 2 & 13
\end{array}.
\]
Now, given a chain of subsets of $\{1, 2, \dots, n\}$
the following procedure produces a
chain of subsets of $\{1, 2, \dots, n+1\}$:
\begin{enumerate}
\item take the original chain, delete any element,
  and make two copies of this chain, which now has even length;
\item glue the two copies together,
  joined by $\varnothing$ in between; and then
\item insert the element $n+1$ into the sets in alternating positions
  of the chain starting with the first.
\end{enumerate}
For example, the first iteration of this construction gives:
\[
 \begin{array}{ccccccccc}
 345 & 1 & 235 & 4 & 125 & 3 & 145 & 2 & 5 \\
 34 & 15 & 23 & 45 & 12 & 35 & 14 & 25 &
 \end{array}
\]
It can be easily checked that if the original chain satisfies the requirements,
then so does the new chain, and if the original chain has length $2^{n-1}+1$,
then the new chain has length $2^{n}+1$, as desired.
This construction yields a chain of length $129$
when $S = \{1, 2, \dots, 8\}$.

\begin{remark*}
  Here is the construction for $n=8$ in its full glory.
\[
 \begin{array}{ccccccccc}
 345678 & 1 & 235678 & 4 & 125678 & 3 & 145678 & 2 & 5678 \\
 34 & 15678 & 23 & 45678 & 12 & 35678 & 14 & 678 & \\
 345 & 1678 & 235 & 4678 & 125 & 3678 & 145 & 2678 & 5 \\
 34678 & 15 & 23678 & 45 & 12678 & 35 & 78 & & \\\hline
 3456 & 178 & 2356 & 478 & 1256 & 378 & 1456 & 278 & 56 \\
 3478 & 156 & 2378 & 456 & 1278 & 356 & 1478 & 6 & \\
 34578 & 16 & 23578 & 46 & 12578 & 36 & 14578 & 26 & 578 \\
 346 & 1578 & 236 & 4578 & 126 & 8 & & & \\ \hline\hline
 34567 & 18 & 23567 & 48 & 12567 & 38 & 14567 & 28 & 567 \\
 348 & 1567 & 238 & 4567 & 128 & 3567 & 148 & 67 & \\
 3458 & 167 & 2358 & 467 & 1258 & 367 & 1458 & 267 & 58 \\
 3467 & 158 & 2367 & 458 & 1267 & 358 & 7 & & \\\hline
 34568 & 17 & 23568 & 47 & 12568 & 37 & 14568 & 27 & 568 \\
 347 & 1568 & 237 & 4568 & 127 & 3568 & 147 & 68 & \\
 3457 & 168 & 2357 & 468 & 1257 & 368 & 1457 & 268 & 57 \\
 3468 & 157 & 2368 & 457 & 1268 & & & & \\
 \end{array}
\]
\end{remark*}
\pagebreak

\section{Solutions to Day 2}
\subsection{JMO 2016/4, proposed by Gregory Galperin}
\textsl{Available online at \url{https://aops.com/community/p6220314}.}
\begin{mdframed}[style=mdpurplebox,frametitle={Problem statement}]
Find, with proof, the least integer $N$ such that
if any $2016$ elements are removed
from the set $\{1, 2, \dots, N\}$,
one can still find $2016$ distinct numbers
among the remaining elements with sum $N$.
\end{mdframed}
The answer is
\[ N = 2017 + 2018 + \dots + 4032
  = 1008 \cdot 6049 = 6097392. \]

To see that $N$ must be at least this large,
consider the situation
when $1$, $2$, \dots, $2016$ are removed.
Among the remaining elements,
any sum of $2016$ elements is certainly
at least $2017 + 2018 + \dots + 4032$.

Now we show this value of $N$ works.
Consider the $3024$ pairs of numbers
$(1, 6048)$, $(2, 6047)$, \dots, $(3024, 3025)$.
Regardless of which $2016$ elements of $\{1, 2, \dots, N\}$ are deleted,
at least $3024 - 2016 = 1008$ of these pairs have both elements remaining.
Since each pair has sum $6049$, we can take these pairs to be the desired numbers.
\pagebreak

\subsection{JMO 2016/5, proposed by Zuming Feng, Jacek Fabrykowski}
\textsl{Available online at \url{https://aops.com/community/p6220305}.}
\begin{mdframed}[style=mdpurplebox,frametitle={Problem statement}]
Let $\triangle ABC$ be an acute triangle, with $O$ as its circumcenter.
Point $H$ is the foot of the perpendicular from $A$ to line $BC$,
and points $P$ and $Q$ are the feet of the perpendiculars
from $H$ to the lines $AB$ and $AC$, respectively.

Given that \[ AH^2 = 2AO^2, \]
prove that the points $O$, $P$, and $Q$ are collinear.
\end{mdframed}
We present two approaches.

\paragraph{First approach (synthetic).}
First, since $AP \cdot AB = AH^2 = AQ \cdot AC$, it follows that $PQCB$ is cyclic.
Consequently, we have $AO \perp PQ$.

\begin{center}
\begin{asy}
size(8cm);
pair A = dir(110);
pair H = A+1.414*dir(-90);
pair U = H+6*dir(0);
pair V = H-6*dir(0);
pair B = OP(U--V, unitcircle);
pair C = IP(U--V, unitcircle);
pair P = foot(H, A, B);
pair Q = foot(H, A, C);

pair D = -A;
draw(A--B--C--cycle, blue);
filldraw(unitcircle, opacity(0.1)+lightcyan, blue);

draw(A--H);
draw(P--H--Q);

pair K = circumcenter(A, B, C);
draw(P--Q, blue);

draw(A--D--C, red);
filldraw(circumcircle(P, Q, B), opacity(0.1)+yellow, dashed+orange);
filldraw(circumcircle(Q, C, D), opacity(0.1)+yellow, dashed+orange);

dot("$A$", A, dir(A));
dot("$H$", H, dir(H));
dot("$B$", B, dir(B));
dot("$C$", C, dir(C));
dot("$P$", P, dir(P));
dot("$Q$", Q, dir(70));
dot("$D$", D, dir(D));
dot("$K$", K, dir(60));
/* Source generated by TSQ */
\end{asy}
\end{center}

Let $K$ be the foot of $A$ onto $PQ$,
and let $D$ be the point diametrically opposite $A$.
Thus $A$, $K$, $O$, $D$ are collinear.

Since quadrilateral $KQCD$ is cyclic ($\angle QKD = \angle QCD = 90^\circ$),
we have
\[ AK \cdot AD = AQ \cdot AC = AH^2
  \implies AK = \frac{AH^2}{AD} = \frac{AH^2}{2AO} = AO \]
so $K = O$.

\paragraph{Second approach (coordinates), with Joshua Hsieh.}
We impose coordinates with $H$ at the origin
and $A = (0,a)$, $B = (-b,0)$, $C = (c,0)$, for $a,b,c > 0$.

\begin{claim*}
  The circumcenter has coordinates $(\frac{c-b}{2}, \frac a2 - \frac{bc}{2a})$.
\end{claim*}
\begin{proof}
  This is a known lemma but but we reproduce its proof for completeness.
  It uses the following steps:
  \begin{itemize}
    \ii By power of a point, the second intersection
    of line $AH$ with the circumcircle is $(0, -\frac{bc}{a})$.
    \ii Since the orthocenter is the reflection of this point across line $BC$,
    the orthocenter is given exactly by $(0, \frac{bc}{a})$.
    \ii The centroid is is $\frac{\vec A + \vec B + \vec C}{3} = (\frac{c-b}{3}, \frac a3)$.
    \ii Since $\vec H - \vec O = 3( \vec G - \vec O)$
    according to the Euler line, we have $\vec O = \frac 32 \vec G - \frac 12 \vec H$.
    This gives the desired formula. \qedhere
  \end{itemize}
\end{proof}
Note that $HQ = \frac{HA \cdot HC}{AC} = \frac{ac}{\sqrt{a^2+c^2}}$.
If we let $T$ be the foot from $Q$ to $BC$,
then $\triangle HQT \overset{\sim}{+} \triangle AHC$
and so the $x$-coordinate of $Q$ is given by
$HQ \cdot \frac{AH}{AC} = \frac{a^2c}{a^2+c^2}$.
Repeating the analogous calculation for $Q$ and $P$ gives
\begin{align*}
  Q &= \left( \frac{a^2 c}{a^2+c^2}, \frac{ac^2}{a^2+c^2} \right) \\
  P &= \left( -\frac{a^2 b}{a^2+b^2}, \frac{ab^2}{a^2+b^2} \right).
\end{align*}
Then, $O$, $P$, $Q$ are collinear if and only if the
following shoelace determinant vanishes
(with denominators cleared out):
\begin{align*}
  0
  &= \det
  \begin{bmatrix}
    -a^2b & ab^2 & a^2+b^2 \\
    a^2c & ac^2 & a^2+c^2 \\
    a(c-b) & a^2-bc & 2a
  \end{bmatrix}
  = a \det
  \begin{bmatrix}
    -ab & ab^2 & a^2+b^2 \\
    ac & ac^2 & a^2+c^2 \\
    c-b & a^2-bc & 2a
  \end{bmatrix} \\
  &= a \det
  \begin{bmatrix}
    -a(b+c) & a(b^2-c^2) & b^2-c^2 \\
    ac & ac^2 & a^2+c^2 \\
    c-b & a^2-bc & 2a
  \end{bmatrix}
  = a(b+c) \det
  \begin{bmatrix}
    -a & a(b-c) & b-c \\
    ac & ac^2 & a^2+c^2 \\
    c-b & a^2-bc & 2a
  \end{bmatrix} \\
  &= a(b+c) \cdot \big[ -a(a^2c^2-a^4+bc(a^2+c^2))
  + ac(b-c) \left( -a^2-bc \right)
- (b-c)^2 \cdot a^3 \big] \\
  &= a^2(b+c)(a^4-a^2b^2-b^2c^2-c^2a^2).
\end{align*}
On the other hand,
\begin{align*}
  AH^2 &= a^2 \\
  2AO^2 &= 2 \left[ \left( \frac{c-b}{2} \right)^2
      + \left( - \frac{a}{2} - \frac{bc}{2a} \right)^2 \right]
  = \frac{a^2+b^2+c^2 + \frac{b^2c^2}{a^2}}{2} \\
  \implies AH^2 - 2AO^2 &= \half \left( a^2-b^2-c^2-\frac{b^2c^2}{a^2} \right).
\end{align*}
So the conditions are equivalent.
\pagebreak

\subsection{JMO 2016/6, proposed by Titu Andreescu}
\textsl{Available online at \url{https://aops.com/community/p6220308}.}
\begin{mdframed}[style=mdpurplebox,frametitle={Problem statement}]
Find all functions $f \colon \RR \to \RR$ such that
for all real numbers $x$ and $y$,
\[ (f(x)+xy) \cdot f(x-3y)
 + (f(y)+xy) \cdot f(3x-y)
 = (f(x+y))^2. \]
\end{mdframed}
We claim that the only two functions satisfying
the requirements are $f(x) \equiv 0$ and $f(x) \equiv x^2$.
These work.

First, taking $x=y=0$ in the given yields $f(0) = 0$,
and then taking $x=0$ gives $f(y)f(-y) = f(y)^2$.
So also $f(-y)^2 = f(y)f(-y)$, from which we conclude $f$ is even.
Then taking $x = -y$ gives
\[ \forall x \in \RR : \qquad
f(x) = x^2 \qquad\text{or}\qquad f(4x) = 0 \qquad(\bigstar) \]
for all $x$.

\begin{remark*}
  Note that an example of a function satisfying $(\bigstar)$ is
  \[
    f(x)
    =
    \begin{cases}
      x^2 & \text{if } |x| < 1 \\
      \log(x^{42} + 2016^{\cos(x)}) & \text{if } 1 \le |x| < 4 \\
      0 & \text{if } |x| \ge 4.
    \end{cases}
  \]
  So, yes, we are currently in a world of trouble, still.
\end{remark*}

Now we claim
\begin{claim*}
  $f(z) = 0 \iff f(2z) = 0 \qquad (\spadesuit)$.
\end{claim*}
\begin{proof}
  Let $(x,y)=(3t,t)$ in the given to get
  \[ \left( f(t)+3t^2 \right)f(8t) = f(4t)^2. \]
  Now if $f(4t) \neq 0$ (in particular, $t \neq 0$),
  then $f(8t) \neq 0$.
  Thus we have $(\spadesuit)$ in the reverse direction.

  Then $f(4t) \neq 0 \overset{(\bigstar)}{\implies} f(t) = t^2 \neq 0
  \overset{(\spadesuit)}{\implies} f(2t) \neq 0$
  implies the forwards direction,
  the last step being the reverse direction $(\spadesuit)$.
\end{proof}
%so by $(\bigstar)$ we have $f(t) = t^2$, and moreover $f(8t) \neq 0$.
%But we also have
%\[ \left( f(t/4)+\frac{3}{16}t^2 \right)f(2t) = f(t)^2 = t^4 \neq 0 \]
%and hence $f(2t) \neq 0$ as well.
%Thus we have seen that $f(4t) \neq 0 \implies f(8t), f(2t) \neq 0$
%which is logically equivalent to $(\spadesuit)$.

By putting together $(\bigstar)$ and $(\spadesuit)$ we finally get
\[ \forall x \in \RR : \qquad
f(x) = x^2 \qquad\text{or}\qquad f(x) = 0 \qquad(\heartsuit) \]

We are now ready to approach the main problem.
Assume there's an $a \neq 0$ for which $f(a) = 0$;
we show that $f \equiv 0$.

Let $b \in \RR$ be given.
Since $f$ is even, we can assume without loss of generality that $a, b > 0$.
Also, note that $f(x) \ge 0$ for all $x$ by $(\heartsuit)$.
By using $(\spadesuit)$ we can generate $c > b$
such that $f(c) = 0$ by taking $c = 2^n a$ for a large enough integer $n$.
Now, select $x, y > 0$ such that
$x-3y=b$ and $x+y=c$. That is,
\[
  (x,y) = \left( \frac{3c+b}{4}, \frac{c-b}{4} \right).
\]
Substitution into the original equation gives
\[
  0 = \left( f(x) + xy \right) f(b)
  + \left( f(y) + xy \right) f(3x-y)
  \ge \left( f(x) + xy \right) f(b).
\]
But since $f(b) \ge 0$, it follows $f(b) = 0$, as desired.
\pagebreak


\end{document}
