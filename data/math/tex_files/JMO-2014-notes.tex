% © Evan Chen
% Downloaded from https://web.evanchen.cc/

\documentclass[11pt]{scrartcl}
\usepackage[sexy]{evan}
\ihead{\footnotesize\textbf{\thetitle}}
\ohead{\footnotesize\href{http://web.evanchen.cc}{\ttfamily web.evanchen.cc},
    updated \today}
\title{JMO 2014 Solution Notes}
\date{\today}

\begin{document}

\maketitle

\begin{abstract}
This is a compilation of solutions
for the 2014 JMO.
Some of the solutions are my own work,
but many are from the official solutions provided by the organizers
(for which they hold any copyrights),
and others were found by users on the Art of Problem Solving forums.

These notes will tend to be a bit more advanced and terse than the ``official''
solutions from the organizers.
In particular, if a theorem or technique is not known to beginners
but is still considered ``standard'', then I often prefer to
use this theory anyways, rather than try to work around or conceal it.
For example, in geometry problems I typically use directed angles
without further comment, rather than awkwardly work around configuration issues.
Similarly, sentences like ``let $\mathbb{R}$ denote the set of real numbers''
are typically omitted entirely.

Corrections and comments are welcome!
\end{abstract}

\tableofcontents
\newpage

\addtocounter{section}{-1}
\section{Problems}
\begin{enumerate}[\bfseries 1.]
\item %% Problem 1
Let $a$, $b$, $c$ be real numbers greater than or equal to $1$.
Prove that
\[ \min \left( \frac{10a^2-5a+1}{b^2-5b+10},
  \frac{10b^2-5b+1}{c^2-5c+10},
  \frac{10c^2-5c+1}{a^2-5a+10} \right) \le abc. \]

\item %% Problem 2
Let $\triangle ABC $ be a non-equilateral,
acute triangle with $\angle A = 60\dg$,
and let $O$ and $H$ denote the circumcenter and orthocenter
of $\triangle{ABC}$, respectively.
\begin{enumerate}[(a)]
  \ii Prove that line $OH$ intersects both segments $AB$ and $AC$
  at two points $P$ and $Q$, respectively.
  \ii Denote by $s$ and $t$ the respective areas of triangle $APQ$
  and quadrilateral $BPQC$.
  Determine the range of possible values for $s/t$.
\end{enumerate}

\item %% Problem 3
Find all $f :\ZZ \to \ZZ$ such that
\[
  xf\left( 2f(y)-x \right) + y^2f\left( 2x-f(y) \right)
  = \frac{f(x)^2}{x} + f\left( yf(y) \right)
\]
for all $x,y \in \ZZ$ such that $x \neq 0$.

\item %% Problem 4
Let $b \ge 2$ be a fixed integer,
and let $s_b(n)$ denote the sum of the base-$b$ digits of $n$.
Show that there are infinitely many positive
integers that cannot be represented in the from $n + s_b(n)$
where $n$ is a positive integer.

\item %% Problem 5
Let $k$ be a positive integer.
Two players $A$ and $B$ play a game on an infinite grid of regular hexagons.
Initially all the grid cells are empty.
Then the players alternately take turns with $A$ moving first.
In her move, $A$ may choose two adjacent hexagons in the grid
which are empty and place a counter in both of them.
In his move, $B$ may choose any counter on the board and remove it.
If at any time there are $k$ consecutive grid cells
in a line all of which contain a counter, $A$ wins.
Find the minimum value of $k$ for which $A$ cannot
win in a finite number of moves, or prove that no such minimum value exists.

\item %% Problem 6
Let $ABC$ be a triangle with incenter $I$,
incircle $\gamma$ and circumcircle $\Gamma$.
Let $M$, $N$, $P$ be the midpoints of $\ol{BC}$, $\ol{CA}$, $\ol{AB}$
and let $E$, $F$ be the tangency points of $\gamma$
with $\ol{CA}$ and $\ol{AB}$, respectively.
Let $U$, $V$ be the intersections of line $EF$
with line $MN$ and line $MP$, respectively,
and let $X$ be the midpoint of arc $BAC$ of $\Gamma$.
\begin{enumerate}[(a)]
  \ii Prove that $I$ lies on ray $CV$.
  \ii Prove that line $XI$ bisects $\ol{UV}$.
\end{enumerate}

\end{enumerate}
\pagebreak

\section{Solutions to Day 1}
\subsection{JMO 2014/1, proposed by Titu Andreescu}
\textsl{Available online at \url{https://aops.com/community/p3477681}.}
\begin{mdframed}[style=mdpurplebox,frametitle={Problem statement}]
Let $a$, $b$, $c$ be real numbers greater than or equal to $1$.
Prove that
\[ \min \left( \frac{10a^2-5a+1}{b^2-5b+10},
  \frac{10b^2-5b+1}{c^2-5c+10},
  \frac{10c^2-5c+1}{a^2-5a+10} \right) \le abc. \]
\end{mdframed}
Notice that
\[ \frac{10a^2-5a+1}{a^2-5a+10} \le a^3 \]
since it rearranges to $(a-1)^5 \ge 0$.
Cyclically multiply to get
\[
  \prod_{\text{cyc}} \left( \frac{10a^2-5a+1}{b^2-5b+10} \right)
  \le (abc)^3
\]
and the minimum is at most the geometric mean.
\pagebreak

\subsection{JMO 2014/2, proposed by Zuming Feng}
\textsl{Available online at \url{https://aops.com/community/p3477702}.}
\begin{mdframed}[style=mdpurplebox,frametitle={Problem statement}]
Let $\triangle ABC $ be a non-equilateral,
acute triangle with $\angle A = 60\dg$,
and let $O$ and $H$ denote the circumcenter and orthocenter
of $\triangle{ABC}$, respectively.
\begin{enumerate}[(a)]
  \ii Prove that line $OH$ intersects both segments $AB$ and $AC$
  at two points $P$ and $Q$, respectively.
  \ii Denote by $s$ and $t$ the respective areas of triangle $APQ$
  and quadrilateral $BPQC$.
  Determine the range of possible values for $s/t$.
\end{enumerate}
\end{mdframed}
We begin with some synthetic work.
Let $I$ denote the incenter, and recall (``fact 5'')
that the arc midpoint $M$ is the center of $(BIC)$,
which we denote by $\gamma$.

Now we have that
\[ \angle BOC = \angle BIC = \angle BHC = 120\dg. \]
Since all three centers lie inside $ABC$ (as it was acute),
and hence on the opposite side of $\ol{BC}$ as $M$,
it follows that $O$, $I$, $H$ lie on minor arc $BC$ of $\gamma$.

We note this implies (a) already,
as line $OH$ meets line $BC$ outside of segment $BC$.

\begin{center}
\begin{asy}
pair A = dir(50);
pair B = dir(210);
pair C = dir(330);

pair M = dir(270);
pair I = incenter(A, B, C);
pair O = circumcenter(A, B, C);
pair H = orthocenter(A, B, C);
pair P = extension(O, H, A, B);
pair Q = extension(O, H, A, C);

filldraw(A--P--Q--cycle, opacity(0.2)+lightgreen, deepgreen);
filldraw(B--P--Q--C--cycle, opacity(0.2)+lightcyan, deepgreen);
filldraw(circumcircle(A, B, C), opacity(0.1)+yellow, red);
filldraw(CP(M, I), opacity(0.1)+lightblue, dotted+blue);
draw(A--M, red);

filldraw(A--O--M--H--cycle, opacity(0.1)+orange, lightred);

dot("$A$", A, dir(A));
dot("$B$", B, dir(B));
dot("$C$", C, dir(C));
dot("$M$", M, dir(M));
dot("$I$", I, dir(20));
dot("$O$", O, dir(90));
dot("$H$", H, dir(50));
dot("$P$", P, dir(P));
dot("$Q$", Q, dir(Q));

/* TSQ Source:

A = dir 50
B = dir 210
C = dir 330

M = dir 270
I = incenter A B C R20
O = circumcenter A B C R90
H = orthocenter A B C R50
P = extension O H A B
Q = extension O H A C

A--P--Q--cycle 0.2 lightgreen / deepgreen
B--P--Q--C--cycle 0.2 lightcyan / deepgreen
circumcircle A B C 0.1 yellow / red
CP M I 0.1 lightblue / dotted blue
A--M red

A--O--M--H--cycle 0.1 orange / lightred

*/
\end{asy}
\end{center}

\begin{claim*}
  Triangle $APQ$ is equilateral with side length $\frac{b+c}{3}$.
\end{claim*}
\begin{proof}
  Let $R$ be the circumradius.
  We have $R = OM = OA = MH$, and even $AH = 2R \cos A = R$,
  so $AOMH$ is a rhombus.
  Thus $\ol{OH} \perp \ol{AM}$ and in this way
  we derive that $\triangle APQ$ is isosceles, hence equilateral.

  Finally, since $\angle PBH = 30\dg$, and $\angle BPH = 120\dg$,
  it follows that $\triangle BPH$ is isosceles and $BP = PH$.
  Similarly, $CQ = QH$.
  So $b+c = AP + BP + AQ + QC = AP + AQ + PQ$ as needed.
\end{proof}

Finally, we turn to the boring task of
extracting the numerical answer.
We have
\[ \frac{s}{s+t} = \frac{[APQ]}{[ABC]}
  = \frac{\frac{\sqrt3}4 \left( \frac{b+c}{3} \right)^2}%
  {\frac{\sqrt3}4 bc}
  = \frac{b^2+2bc+c^2}{9bc}
  = \frac19 \left( 2 + \frac bc + \frac cb \right). \]

So the problem is reduced to analyzing the behavior of $b/c$.
For this, we imagine fixing $\Gamma$ the circumcircle of $ABC$,
as well as the points $B$ and $C$.
Then as we vary $A$ along the ``topmost'' arc of measure $120\dg$,
we find $b/c$ is monotonic with values $1/2$ and $2$ at endpoints,
and by continuity all values $b/c \in (1/2,2)$ can be achieved.

So \[ \half < \frac bc < 2
  \implies 4/9 < \frac{s}{s+t} < 1/2
  \implies 4/5 < \frac st < 1 \]
as needed.
\pagebreak

\subsection{JMO 2014/3, proposed by Titu Andreescu}
\textsl{Available online at \url{https://aops.com/community/p3477690}.}
\begin{mdframed}[style=mdpurplebox,frametitle={Problem statement}]
Find all $f :\ZZ \to \ZZ$ such that
\[
  xf\left( 2f(y)-x \right) + y^2f\left( 2x-f(y) \right)
  = \frac{f(x)^2}{x} + f\left( yf(y) \right)
\]
for all $x,y \in \ZZ$ such that $x \neq 0$.
\end{mdframed}
The answer is $f(x) \equiv 0$ and $f(x) \equiv x^2$. Check that these work.

Now let's prove these are the only solutions.
Put $y=0$ to obtain
\[ x f\left( 2f(0)-x \right) = \frac{f(x)^2}{x} + f(0).  \]

The nicest part of the problem is the following step:
\begin{claim*}
  We have $f(0)=0$
\end{claim*}
\begin{proof}
  If not, select a prime $p \nmid f(0)$ and put $x=p \neq 0$.
  In the above, we find that $p \mid f(p)^2$,
  so $p \mid f(p)$ and hence $p \mid \tfrac{f(p)^2}{p}$.
  From here we derive $p \mid f(0)$, contradiction.
  Hence \[ f(0) = 0. \]
\end{proof}

\begin{claim*}
  We have $f(x) \in \{0,x^2\}$ for each individual $x$.
\end{claim*}
\begin{proof}
  The above then implies that
  \[ x^2f(-x) = f(x)^2 \]
  holds for all nonzero $x$, but also for $x=0$.
  Let us now check that $f$ is an even function.
  In the above, we may also derive $f(-x)^2 = x^2f(x)$.
  If $f(x) \neq f(-x)$ (and hence $x \neq 0$),
  then subtracting the above and factoring implies that
  $f(x) + f(-x) = -x^2$;
  we can then obtain by substituting the relation
  \[ \left[ f(x) + \frac 12x^2 \right]^2 = -\frac 34 x^4 < 0 \]
  which is impossible.
  This means $f(x)^2 = x^2f(x)$, thus
  \[ f(x) \in \{0, x^2\} \qquad \forall x. \]
\end{proof}

Now suppose there exists a nonzero integer $t$ with $f(t) = 0$.
We will prove that $f(x) \equiv 0$.
Put $y=t$ in the given to obtain that
\[ t^2 f(2x) = 0 \]
for any integer $x \neq 0$, and hence conclude that $f(2 \ZZ) \equiv 0$.
Then selecting $x = 2k \neq 0$ in the given implies that
\[ y^2 f(4k-f(y)) = f(yf(y)). \]

Assume for contradiction that $f(m) = m^2$ now for some odd $m \neq 0$.
Evidently \[ m^2 f(4k-m^2) = f(m^3). \]
If $f(m^3) \neq 0$ this forces $f(4k-m^2) \neq 0$,
and hence $m^2(4k-m^2)^2 = m^6$ for arbitrary $k \neq 0$, which is clearly absurd.
That means \[ f(4k-m^2) = f(m^2-4k) = f(m^3) = 0 \] for each $k \neq 0$.
Since $m$ is odd, $m^2 \equiv 1 \pmod 4$,
and so $f(n) = 0$ for all $n$ other than $\pm m^2$
(since we cannot select $k=0$).

Now $f(m) = m^2$ means that $m = \pm 1$.
Hence either $f(x) \equiv 0$ or
\[ f(x) = \begin{cases} 1 & x = \pm 1 \\ 0 & \text{otherwise}. \end{cases} \]
To show that the latter fails,
we simply take $x=5$ and $y=1$ in the given.

Hence, the only solutions are $f(x) \equiv 0$ and $f(x) \equiv x^2$.
\pagebreak

\section{Solutions to Day 2}
\subsection{JMO 2014/4, proposed by Palmer Mebane}
\textsl{Available online at \url{https://aops.com/community/p3478579}.}
\begin{mdframed}[style=mdpurplebox,frametitle={Problem statement}]
Let $b \ge 2$ be a fixed integer,
and let $s_b(n)$ denote the sum of the base-$b$ digits of $n$.
Show that there are infinitely many positive
integers that cannot be represented in the from $n + s_b(n)$
where $n$ is a positive integer.
\end{mdframed}
For brevity let $f(n) = n + s_b(n)$.
Select any integer $M$.
Observe that $f(x) \ge b^{2M}$ for any $x \ge b^{2M}$,
but also $f(b^{2M}-k) \ge b^{2M}$ for $k = 1, 2, \dots, M$,
since the base-$b$ expansion of $b^{2M}-k$ will start out with
at least $M$ digits $b-1$.

Thus $f$ omits at least $M$ values in $[1, b^{2M}]$ for any $M$.
\pagebreak

\subsection{JMO 2014/5, proposed by Palmer Mebane}
\textsl{Available online at \url{https://aops.com/community/p3478584}.}
\begin{mdframed}[style=mdpurplebox,frametitle={Problem statement}]
Let $k$ be a positive integer.
Two players $A$ and $B$ play a game on an infinite grid of regular hexagons.
Initially all the grid cells are empty.
Then the players alternately take turns with $A$ moving first.
In her move, $A$ may choose two adjacent hexagons in the grid
which are empty and place a counter in both of them.
In his move, $B$ may choose any counter on the board and remove it.
If at any time there are $k$ consecutive grid cells
in a line all of which contain a counter, $A$ wins.
Find the minimum value of $k$ for which $A$ cannot
win in a finite number of moves, or prove that no such minimum value exists.
\end{mdframed}
The answer is $k = 6$.

\medskip

\textbf{Proof that $A$ cannot win if $k=6$}.
We give a strategy for $B$ to prevent $A$'s victory.
Shade in every third cell, as shown in the figure below.
Then $A$ can never cover two shaded cells simultaneously on her turn.
Now suppose $B$ always removes a counter on a shaded cell
(and otherwise does whatever he wants).
Then he can prevent $A$ from ever getting six consecutive counters,
because any six consecutive cells contain two shaded cells.

\begin{center}
  \begin{asy}
    path hexagon = scale(0.5)*(dir(30)--dir(90)--dir(150)--dir(210)--dir(270)--dir(330)--cycle);
    unitsize(0.6cm);
    void mark(int x, int y) {
      pair P = dir(0)*x + dir(60)*y;
      filldraw(shift(P)*hexagon, purple, black+1);
    }
    void spot(int x, int y) {
      pair P = dir(0)*x + dir(60)*y;
      draw(shift(P)*hexagon);
    }
    int N = 4;
    for (int x=-N; x<=N; ++x) {
      for (int y=-N; y<=N; ++y) {
        if ( (abs(x+y)) > N ) continue;
        if (
          (x-y) % 3 == 0
        ) mark(x,y);
        else spot(x,y);
      }
    }
  \end{asy}
\end{center}

\medskip

\textbf{Example of a strategy for $A$ when $k=5$}.
We describe a winning strategy for $A$ explicitly.
Note that after $B$'s first turn there is one counter,
so then $A$ may create an equilateral triangle,
and hence after $B$'s second turn there are two consecutive counters.
Then, on her third turn,
$A$ places a pair of counters two spaces away on the same line.
Label the two inner cells $x$ and $y$ as shown below.
\begin{center}
  \begin{asy}
    path hexagon = scale(0.5)*(dir(30)--dir(90)--dir(150)--dir(210)--dir(270)--dir(330)--cycle);
    unitsize(0.6cm);
    void mark(int x, int y) {
      pair P = dir(0)*x + dir(60)*y;
      filldraw(shift(P)*hexagon, palecyan, black);
      filldraw(CR(P, 0.3), paleblue, blue);
    }
    void spot(int x, int y) {
      pair P = dir(0)*x + dir(60)*y;
      draw(shift(P)*hexagon);
    }
    for (int x=-3; x<=4; ++x) {
      for (int y=-2; y<=2; ++y) {
        if ( x+y < -3 ) continue;
        if ( x+y > 4 ) continue;
        spot(x,y);
      }
    }
    mark(-2, 0);
    mark(-1, 0);
    mark( 2, 0);
    mark( 3, 0);
    label("$x$", (-1)*dir(0), deepgreen);
    label("$y$", (2)*dir(0), deepgreen);
  \end{asy}
\end{center}
Now it is $B$'s turn to move;
in order to avoid losing immediately,
he must remove either $x$ or $y$.
Then on any subsequent turn, $A$ can replace $x$ or $y$
(whichever was removed)
and add one more adjacent counter.
This continues until either $x$ or $y$ has all its neighbors filled
(we ask $A$ to do so in such a way that
she avoids filling in the two central cells between $x$ and $y$
as long as possible).

So, let's say without loss of generality (by symmetry) that $x$
is completely surrounded by tokens.
Again, $B$ must choose to remove $x$ (or $A$ wins on her next turn).
After $x$ is removed by $B$, consider the following figure.
\begin{center}
  \begin{asy}
    path hexagon = scale(0.5)*(dir(30)--dir(90)--dir(150)--dir(210)--dir(270)--dir(330)--cycle);
    unitsize(0.6cm);
    void mark(int x, int y) {
      pair P = dir(0)*x + dir(60)*y;
      filldraw(shift(P)*hexagon, palecyan, black);
      filldraw(CR(P, 0.3), paleblue, blue);
    }
    void spot(int x, int y) {
      pair P = dir(0)*x + dir(60)*y;
      draw(shift(P)*hexagon);
    }
    for (int x=-4; x<=5; ++x) {
      for (int y=-3; y<=3; ++y) {
        if ( x+y < -4 ) continue;
        if ( x+y > 5 ) continue;
        spot(x,y);
      }
    }
    mark(-2, 0);
    mark( 0, 0);
    mark( 0, -1);
    mark(-1, 1);
    mark(-1, -1);
    mark(-2, 1);

    mark( 2, 0);
    mark( 2, 1);
    mark( 3, 0);
    label("$x$", (-1)*dir(0), deepgreen);
    label("$y$", (2)*dir(0), deepgreen);
    void win(int x, int y, pen p1, pen p2) {
      pair P = dir(0)*x + dir(60)*y;
      filldraw(shift(P)*hexagon, p1, p2);
    }
    win(1,0, lightgreen, deepgreen);
    win(0,1, lightgreen, deepgreen);
    win(0,2, lightred, red);
    win(0,3, lightred, red);
    win(4,0, lightred, red);
    win(5,0, lightred, red);
  \end{asy}
\end{center}
We let $A$ play in the two marked green cells.
Then, regardless of what move $B$ plays,
one of the two choices of moves marked in red lets $A$ win.
Thus, we have described a winning strategy when $k=5$ for $A$.
\pagebreak

\subsection{JMO 2014/6, proposed by Titu Andreescu, Cosmin Pohoata}
\textsl{Available online at \url{https://aops.com/community/p3478583}.}
\begin{mdframed}[style=mdpurplebox,frametitle={Problem statement}]
Let $ABC$ be a triangle with incenter $I$,
incircle $\gamma$ and circumcircle $\Gamma$.
Let $M$, $N$, $P$ be the midpoints of $\ol{BC}$, $\ol{CA}$, $\ol{AB}$
and let $E$, $F$ be the tangency points of $\gamma$
with $\ol{CA}$ and $\ol{AB}$, respectively.
Let $U$, $V$ be the intersections of line $EF$
with line $MN$ and line $MP$, respectively,
and let $X$ be the midpoint of arc $BAC$ of $\Gamma$.
\begin{enumerate}[(a)]
  \ii Prove that $I$ lies on ray $CV$.
  \ii Prove that line $XI$ bisects $\ol{UV}$.
\end{enumerate}
\end{mdframed}
The fact that $I = \ol{BU} \cap \ol{CV}$ and is Lemma 1.45 from EGMO.

As for (b), we note:
\begin{claim*}
  Line $IX$ is a symmedian of $\triangle IBC$.
\end{claim*}
\begin{proof}
  Recall that $(BIC)$ has circumcenter
  coinciding with the antipode of $X$ (by ``Fact 5'').
  So this follows from the fact
  that $\ol{XB}$ and $\ol{XC}$ are tangent.
\end{proof}
Since $BVUC$ is cyclic with diagonals intersecting at $I$,
and $IX$ is symmedian of $\triangle IBC$,
it is median of $\triangle IUV$, as needed.

\begin{remark*}
  [Alternate solution to (b) by Gunmay Handa]
  It's well known that $X$ is the midpoint of $\ol{I_b I_c}$
  (by considering the nine-point circle of the excentral triangle).
  However, $\ol{UV} \parallel \ol{I_b I_c}$
  and $I = \ol{I_b U} \cap \ol{I_c V}$, implying the result.
\end{remark*}
\pagebreak


\end{document}
