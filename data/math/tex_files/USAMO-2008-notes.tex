% © Evan Chen
% Downloaded from https://web.evanchen.cc/

\documentclass[11pt]{scrartcl}
\usepackage[sexy]{evan}
\ihead{\footnotesize\textbf{\thetitle}}
\ohead{\footnotesize\href{http://web.evanchen.cc}{\ttfamily web.evanchen.cc},
    updated \today}
\title{USAMO 2008 Solution Notes}
\date{\today}

\begin{document}

\maketitle

\begin{abstract}
This is a compilation of solutions
for the 2008 USAMO.
Some of the solutions are my own work,
but many are from the official solutions provided by the organizers
(for which they hold any copyrights),
and others were found by users on the Art of Problem Solving forums.

These notes will tend to be a bit more advanced and terse than the ``official''
solutions from the organizers.
In particular, if a theorem or technique is not known to beginners
but is still considered ``standard'', then I often prefer to
use this theory anyways, rather than try to work around or conceal it.
For example, in geometry problems I typically use directed angles
without further comment, rather than awkwardly work around configuration issues.
Similarly, sentences like ``let $\mathbb{R}$ denote the set of real numbers''
are typically omitted entirely.

Corrections and comments are welcome!
\end{abstract}

\tableofcontents
\newpage

\addtocounter{section}{-1}
\section{Problems}
\begin{enumerate}[\bfseries 1.]
\item %% Problem 1
Prove that for each positive integer $n$,
there are pairwise relatively prime integers $k_0$, \dots, $k_n$,
all strictly greater than $1$, such that
$k_0k_1 \dots k_n - 1$ is the product of two consecutive integers.

\item %% Problem 2
Let $ABC$ be an acute, scalene triangle,
and let $M$, $N$, and $P$ be the midpoints of
$\ol{BC}$, $\ol{CA}$, and $\ol{AB}$, respectively.
Let the perpendicular bisectors of $\ol{AB}$ and $\ol{AC}$
intersect ray $AM$ in points $D$ and $E$ respectively,
and let lines $BD$ and $CE$ intersect in point $F$, inside triangle $ABC$.
Prove that points $A$, $N$, $F$, and $P$ all lie on one circle.

\item %% Problem 3
Let $n$ be a positive integer.  Denote by $S_n$ the set of points $(x, y)$
with integer coordinates such that
\[ \left\lvert x\right\rvert + \left\lvert y + \half \right\rvert < n. \]
A path is a sequence of distinct points
$(x_1 , y_1), (x_2, y_2), \dots, (x_\ell, y_\ell)$ in $S_n$
such that, for $i = 2, \dots, \ell$,
the distance between $(x_i , y_i)$ and $(x_{i-1} , y_{i-1} )$ is $1$.

Prove that the points in $S_n$ cannot be partitioned into fewer than $n$ paths.

\item %% Problem 4
For which integers $n \ge 3$ can one find a triangulation
of regular $n$-gon consisting only of isosceles triangles?

\item %% Problem 5
Three nonnegative real numbers $r_1$, $r_2$, $r_3$
are written on a blackboard.
These numbers have the property
that there exist integers $a_1$, $a_2$, $a_3$, not all zero,
satisfying $a_1r_1 + a_2r_2 + a_3r_3 = 0$.
We are permitted to perform the following operation:
find two numbers $x$, $y$ on the blackboard with $x \le y$,
then erase $y$ and write $y - x$ in its place.
Prove that after a finite number of such operations,
we can end up with at least one $0$ on the blackboard.

\item %% Problem 6
At a certain mathematical conference,
every pair of mathematicians are either friends or strangers.
At mealtime, every participant eats in one of two large dining rooms.
Each mathematician insists upon eating in a room which
contains an even number of his or her friends.
Prove that the number of ways that the mathematicians
may be split between the two rooms is a power of two
(i.e.\ is of the form $2^k$ for some positive integer $k$).

\end{enumerate}
\pagebreak

\section{Solutions to Day 1}
\subsection{USAMO 2008/1, proposed by Titu Andreescu}
\textsl{Available online at \url{https://aops.com/community/p1116186}.}
\begin{mdframed}[style=mdpurplebox,frametitle={Problem statement}]
Prove that for each positive integer $n$,
there are pairwise relatively prime integers $k_0$, \dots, $k_n$,
all strictly greater than $1$, such that
$k_0k_1 \dots k_n - 1$ is the product of two consecutive integers.
\end{mdframed}
In other words, if we let
\[ P(x) = x(x+1) + 1 \]
then we would like there to be infinitely many primes
dividing some $P(t)$ for some integer $t$.

In fact, this result is true in much greater generality.
We first state:
\begin{theorem}
  [Schur's theorem]
  If $P(x) \in \ZZ[x]$ is nonconstant and $P(0) = 1$,
  then there are infinitely many primes
  which divide $P(t)$ for some integer $t$.
\end{theorem}
\begin{proof}
  If $P(0) = 0$, this is clear.
  So assume $P(0) = c \neq 0$.

  Let $S$ be any finite set of prime numbers.
  Consider then the value
  \[ P\left(k \prod_{p \in S} p\right) \]
  for some integer $k$.
  It is $1 \pmod p$ for each prime $p$,
  and if $k$ is large enough it should not be equal to $1$
  (because $P$ is not constant).
  Therefore, it has a prime divisor not in $S$.
\end{proof}
\begin{remark*}
  In fact the result holds without the assumption $P(0) \neq 1$.
  The proof requires only small modifications,
  and a good exercise would be to write down a similar
  proof that works first for $P(0) = 20$,
  and then for any $P(0) \neq 0$.
  (The $P(0) = 0$ case is vacuous,
  since then $P(x)$ is divisible by $x$.)
\end{remark*}

To finish the proof, let $p_1$, \dots, $p_n$ be primes
and $x_i$ be integers such that
\begin{align*}
  P(x_1) &\equiv 0 \pmod{p_1} \\
  P(x_2) &\equiv 0 \pmod{p_2} \\
  &\vdotswithin\equiv \\
  P(x_n) &\equiv 0 \pmod{p_n}
\end{align*}
as promised by Schur's theorem.
Then, by Chinese remainder theorem,
we can find $x$ such that $x \equiv x_i \pmod{p_i}$
for each $i$, whence $P(x)$ has at least $n$ prime factor.
\pagebreak

\subsection{USAMO 2008/2, proposed by Zuming Feng}
\textsl{Available online at \url{https://aops.com/community/p1116181}.}
\begin{mdframed}[style=mdpurplebox,frametitle={Problem statement}]
Let $ABC$ be an acute, scalene triangle,
and let $M$, $N$, and $P$ be the midpoints of
$\ol{BC}$, $\ol{CA}$, and $\ol{AB}$, respectively.
Let the perpendicular bisectors of $\ol{AB}$ and $\ol{AC}$
intersect ray $AM$ in points $D$ and $E$ respectively,
and let lines $BD$ and $CE$ intersect in point $F$, inside triangle $ABC$.
Prove that points $A$, $N$, $F$, and $P$ all lie on one circle.
\end{mdframed}
We present four solutions.

\paragraph{Barycentric solution.}
First, we find the coordinates of $D$.
As $D$ lies on $\ol{AM}$, we know $D=(t:1:1)$ for some $t$.
Now by perpendicular bisector formula, we find
\[ 0 = b^2(t-1) + (a^2-c^2) \implies t = \frac{c^2+b^2-a^2}{b^2}. \]
Thus we obtain \[ D = \left( 2S_A : c^2 : c^2 \right). \]
Analogously $E = (2S_A : b^2 : b^2)$,
and it follows that \[ F = \left( 2S_A : b^2 : c^2 \right). \]

The sum of the coordinates of $F$ is
\[ (b^2+c^2-a^2)+b^2+c^2=  2b^2+2c^2-a^2. \]
Hence the reflection of $A$ over $F$ is simply
\[ 2F-A = \left( 2(b^2+c^2-a^2)-(2b^2+2c^2-a^2) : 2b^2 : 2c^2 \right)
  = \left(-a^2 : 2b^2 : 2c^2 \right). \]
It is evident that $F'$ lies on $(ABC) : -a^2yz - b^2zx - c^2xy = 0$,
and we are done.

\paragraph{Synthetic solution (harmonic).}
Here is a synthetic solution.
Let $X$ be the point so that $APXN$ is a cyclic harmonic quadrilateral.
We contend that $X = F$.
To see this it suffices to prove $B$, $X$, $D$ collinear
(and hence $C$, $X$, $E$ collinear by symmetry).

\begin{center}
\begin{asy}
pair A = dir(130);
pair P = dir(210);
pair N = dir(330);
pair B = 2*P-A;
pair C = 2*N-A;
pair O = -A;
pair T = midpoint(P--N);
pair D = extension(A, T, P, O);
pair E = extension(A, T, N, O);
pair X = extension(B, D, E, C);
pair M = extension(A, T, B, C);
draw(N--T, heavygreen);
draw(T--P, lightblue);
draw(X--P, heavygreen);

draw(B--C--N, lightblue);
draw(P--O--N, lightblue);
draw(N--E, lightblue);
filldraw(unitcircle, opacity(0.1)+lightcyan, blue);

draw(D--X--C, red+dashed);

filldraw(A--B--X--cycle, opacity(0.1)+lightgreen, heavygreen);
filldraw(A--M--N--cycle, opacity(0.1)+lightgreen, heavygreen);

dot("$A$", A, dir(A));
dot("$P$", P, dir(P));
dot("$N$", N, dir(10));
dot("$B$", B, dir(B));
dot("$C$", C, dir(C));
dot("$O$", O, dir(O));
dot("$T$", T, dir(45));
dot("$D$", D, dir(60));
dot("$E$", E, dir(240));
dot("$X$", X, dir(X));
dot("$M$", M, dir(M));

/* TSQ Source:

A = dir 130
P = dir 210
N = dir 330 R10
B = 2*P-A
C = 2*N-A
O = -A
T = midpoint P--N R45
D = extension A T P O R60
E = extension A T N O R240
X = extension B D E C
M = extension A T B C
N--T heavygreen
T--P lightblue
X--P heavygreen

B--C--N lightblue
P--O--N lightblue
N--E lightblue
unitcircle 0.1 lightcyan / blue

D--X--C red dashed

A--B--X--cycle 0.1 lightgreen / heavygreen
A--M--N--cycle 0.1 lightgreen / heavygreen

*/
\end{asy}
\end{center}

Let $T$ be the midpoint of $\ol{PN}$,
so $\triangle APX \sim \triangle ATN$.
So $\triangle ABX \sim \triangle AMN$, ergo
\[ \dang XBA = \dang NMA = \dang BAM = \dang BAD = \dang DBA \]
as desired.
\paragraph{Angle chasing solution (Mason Fang).}
Obviously $ANOP$ is concyclic.
\begin{claim*}
  Quadrilateral $BFOC$ is cyclic.
\end{claim*}
\begin{proof}
  Write
  \begin{align*}
  \dang BFC = \dang FBC+\dang BCF
  &=\dang FBA +\dang ABC+\dang BCA+\dang ACF\\
  &=\dang DBA +\dang ABC+\dang BCA+\dang ACE\\
  &=\dang BAD +\dang ABC+\dang BCA+\dang EAC\\
  &=2\angle BAC=\angle BOC. \qedhere
  \end{align*}
\end{proof}
Define $Q = \ol{AA} \cap \ol{BC}$.
\begin{claim*}
  Point $Q$ lies on $\ol{FO}$.
\end{claim*}
\begin{proof}
  Write
  \begin{align*}
     \dang BOQ = \dang BOA+\dang AOQ
     &=2\dang BCA+90^{\dg}+\dang AQO\\
     &=2\dang BCA+90^{\dg}+\dang AMO\\
     &=2\dang BCA+90^{\dg}+\dang AMC+90^{\dg}\\
     &=\dang BCA+\dang MAC =\dang BCA+\dang ACE \\
     &=\dang BCE =\dang BOF. \qedhere
  \end{align*}
\end{proof}
As $Q$ is the radical center of $(ANOP)$, $(ABC)$ and $(BFOC)$,
this implies the result.
\paragraph{Inversive solution (Kelin Zhu).}
Invert about $A$ with radius $\sqrt{bc}$
followed by a reflection over the angle bisector of $\angle A$,
and denote the image of a point $X$ by $X'$.
The inverted problem now states the following:

\begin{quote}
  In triangle $AP^{\ast}N^{\ast}$,
  let $B^{\ast}$, $C^{\ast}$ be the midpoints of $AP^{\ast}$, $AN^{\ast}$
  and $D^{\ast}$, $E^{\ast}$ be the intersection of the $A$ symmedian
  with $(AP^{\ast})$, $(AN^{\ast})$, respectively.
  $(AB^{\ast}D^{\ast})$, $(AC^{\ast}E^{\ast})$
  intersect at a point $F^{\ast}$; prove that it lies on $P^{\ast}N^{\ast}$.
\end{quote}

I claim that, in fact, the midpoint of $P^{\ast}N^{\ast}$ is the desired intersection.
Redefine that point as $F^{\ast}$ and I will prove that
$(AB^{\ast}D^{\ast})$, $(AC^{\ast}E^{\ast})$ pass through it.

Note that
\[ \angle AD^{\ast}B^{\ast}=\angle D^{\ast}AB^{\ast}=\angle F^{\ast}AN^{\ast}=\angle AF^{\ast}B^{\ast}, \]
where the first equality is due to $B^{\ast}$ being the circumcircle of $AD^{\ast}P^{\ast}$,
the second equality is due to the definition of the symmedian,
and the third equality is due to the parallelogram $AB^{\ast}F^{\ast}C^{\ast}$.
A symmetric argument for $C$ finishes.
\pagebreak

\subsection{USAMO 2008/3, proposed by Gabriel Carroll}
\textsl{Available online at \url{https://aops.com/community/p1116367}.}
\begin{mdframed}[style=mdpurplebox,frametitle={Problem statement}]
Let $n$ be a positive integer.  Denote by $S_n$ the set of points $(x, y)$
with integer coordinates such that
\[ \left\lvert x\right\rvert + \left\lvert y + \half \right\rvert < n. \]
A path is a sequence of distinct points
$(x_1 , y_1), (x_2, y_2), \dots, (x_\ell, y_\ell)$ in $S_n$
such that, for $i = 2, \dots, \ell$,
the distance between $(x_i , y_i)$ and $(x_{i-1} , y_{i-1} )$ is $1$.

Prove that the points in $S_n$ cannot be partitioned into fewer than $n$ paths.
\end{mdframed}
\paragraph{First solution (local).}
We proceed by induction on $n$.
The base case $n=1$ is clear, so suppose $n > 1$.
Let $S$ denote the set of points
\[ S = \left\{ (x,y) : x + \left\lvert y+\frac12 \right\rvert \ge n - 2  \right\}. \]
An example when $n=4$ is displayed below.
\begin{center}
\begin{asy}
int n = 4;
for (int a=-4; a<=4; ++a) {
for (int b=-4; b<=4; ++b) {
if (abs(a) + abs(b+0.5) < n) {
if (a + abs(b+0.5) > n-2) dot( (a,b), red+4 );
else dot( (a,b), black+3 );
}
}
}
draw((0,3)--(0,2)--(1,2)--(1,1)--(2,1)--(2,0)--(3,0)--(3,-1)--(2,-1)--(2,-2)--(1,-2)--(1,-3)--(0,-3)--(0,-4), red);
draw(CR((3,0), 0.3), deepgreen);
label("$a$", (3.3,0), dir(0), deepgreen);
\end{asy}
\end{center}
For any minimal partition $\mathcal P$ of $S_n$,
let $P$ denote the path passing through the point $a = (n-1,0)$.
Then the intersection of $P$ with $S$ consists of several disconnected paths;
let $N$ be the number of nodes in the component containing $a$, and
pick $\mathcal P$ such that $N$ is maximal.
We claim that in this case $P = S$.

Assume not.
First, note $a = (n-1,0)$ must be connected to $b = (n-1, -1)$
(otherwise join them to decrease the number of paths).

Now, starting from $a = (n-1,0)$ walk along $P$ away from $b$
until one of the following three conditions is met:
\begin{itemize}
  \ii We reach a point $v$ not in $S$.
  Let $w$ be the point before $v$,
  and $x$ the point in $S$ adjacent to $w$.
  Then delete $vw$ and add $wx$.
  This increases $N$ while leaving the number of edges unchanged:
  so this case can't happen.

  \ii We reach an endpoint $v$ of $P$ (which may be $a$),
  lying inside the set $S$, which is not the topmost point $(0,n-1)$.
  Let $w$ be the next point of $S$.
  Delete any edge touching $w$ and add edge $vw$.
  This increases $N$ while leaving the number of edges unchanged:
  so this case can't happen.

  \ii We reach the topmost point $(0,n-1)$.
\end{itemize}
Thus we see that $P$ must follow $S$ until reaching the topmost point $(0,n-1)$.
Similarly it must reach the bottom-most point $(0,-n)$.
Hence $P=S$.

The remainder of $S_n$ is just $S_{n-1}$,
and hence this requires at least $n-1$ paths to
cover by the inductive hypothesis.
So $S_n$ requires at least $n$ paths, as desired.

%A bit easy for a \#3, seeing that I got this during while trying to sleep on a train ride to visit my grandmother (and I normally suck at combo). This is more or less equivalent to MellowMelon's solution, but here's how I phrased it in my head.

\begin{remark*}
  [Motivational comments from Evan]
  Basically the idea is that I wanted to peel away the
  right path $S$ highlighted in red in the figure, so that one could induct.
  But the problem is that the red path might not actually exist,
  e.g.\ the set of paths might contain the mirror of $S$ instead.

  Nonetheless, in those equality cases I found I could perturb some edges
  (e.g.\ change from $(-1,n-2)$--$(0,n-2)$ to $(0,n-2)$--$(1,n-2)$).
  So the idea then was to do little changes and try to convert
  the given partition into one where the red path $S$ exists,
  (and then peel it away for induction)
  without decreasing the total number of paths.

  To make this work, you actually want the incisions to begin
  ear the points $a$ and $b$,
  because that's the point of $S$ that is most constrained
  (e.g.\ you get $a$--$b$ right away for free),
  and assemble the path from there.
  (If you try to do it from the top, it's much less clear what's happening.)
  That's why the algorithm starts the mutations from around a.
\end{remark*}

\paragraph{Second solution (global).}
Here is a much shorter official solution,
which is much trickier to find, and ``global'' in nature.

Color the upper half of the diagram with a blue/red checkerboard pattern
such that the uppermost point $(n-1,0)$ is blue.
Reflect it over to the bottom, as shown.
\begin{center}
\begin{asy}
bool even(int x) {
  return (x % 2 ==  0
  );
}
int n = 4;
for (int a=-4; a<=4; ++a) {
for (int b=-4; b<=4; ++b) {
if (abs(a) + abs(b+0.5) < n) {
if ( (b >= 0) && even(a+b+n+1)) dot( (a,b), blue+4 );
else if ( (b < 0) && even(a+b+n)) dot( (a,b), blue+4 );
else dot( (a,b), red+3 );
}
}
}
\end{asy}
\end{center}

Assume there are $m$ paths.
Cut in two any paths with two adjacent blue points;
this occurs only along the horizontal symmetry axis.
Thus:
\begin{itemize}
  \ii After cutting there are at most $m+n$ paths,
  since at most $n$ cuts occur.
  \ii On the other hand,
  there are $2n$ more blue points than red points.
  Hence after cutting there must be at least $2n$ paths
  (since each path alternates colors, except possibly for double-red pairs).
\end{itemize}
So $m+n \ge 2n$, hence $m \ge n$.


\begin{remark*}
  This problem turned out to be known already.
  It appears in this reference:
  \begin{quote}
  Nikolai Beluhov, Nyakolko Zadachi po Shahmatna
  Kombinatorika, \emph{Matematika Plyus},
  2006, issue 4, pages 61--64.
  \end{quote}
  Section 1 of 2 was reprinted with revisions as
  Nikolai Beluhov, Dolgii Put Korolya, \emph{Kvant},
  2010, issue 4, pages 39--41.
  The reprint is available at
  \url{http://kvant.ras.ru/pdf/2010/2010-04.pdf}.
\end{remark*}

\begin{remark*}
  [Nikolai Beluhov]
  As pointed out in the reference above,
  this problem arises naturally when we try to
  estimate the greatest possible length of a closed king tour
  on the chessboard of size $n \times n$ with $n$ even,
  a question posed by Igor Akulich in Progulki
  Korolya, \emph{Kvant}, 2000, issue 3, pages 47--48.
  Each one of the two references above contains a proof
  that the answer is $n + \sqrt{2}(n^2 - n)$.
\end{remark*}
\pagebreak

\section{Solutions to Day 2}
\subsection{USAMO 2008/4, proposed by Gregory Galperin}
\textsl{Available online at \url{https://aops.com/community/p1116177}.}
\begin{mdframed}[style=mdpurplebox,frametitle={Problem statement}]
For which integers $n \ge 3$ can one find a triangulation
of regular $n$-gon consisting only of isosceles triangles?
\end{mdframed}
The answer is $n$ of the form $2^a(2^b+1)$
where $a$ and $b$ are nonnegative integers not both zero.

Call the polygon $A_1 \dots A_n$
with indices taken modulo $n$.
We refer to segments $A_1 A_2$, $A_2 A_3$, \dots, $A_n A_1$
as \emph{short sides}.
Each of these must be in the triangulation.
Note that
\begin{itemize}
  \ii when $n$ is even,
  the isosceles triangles triangle using a short side $A_1 A_2$
  are $\triangle A_n A_1 A_2$ and $\triangle A_1 A_2 A_3$ only,
  which we call \emph{small}.
  \ii when $n$ is odd, in addition to the small triangles,
  we have $\triangle A_{\half(n+3)} A_1 A_2$,
  which we call \emph{big}.
\end{itemize}

This leads to the following two claims.
\begin{claim*}
  If $n > 4$ is even, then $n$ works iff $n/2$ does.
\end{claim*}
\begin{proof}
  All short sides must be part of a small triangle;
  after drawing these in, we obtain an $n/2$-gon.
  \begin{center}
  \begin{asy}
    size(3cm);
    draw(unitcircle, dotted);
    picture p;
    dot(p, dir(0));
    dot(p, dir(30));
    dot(p, dir(60));
    fill(p, dir(0)--dir(30)--dir(60)--cycle, opacity(0.1)+lightred);
    draw(p, dir(0)--dir(30)--dir(60), blue);
    draw(p, dir(0)--dir(60), red);
    for (int i=0; i<6; ++i) {
      add(rotate(60*i)*p);
    }
  \end{asy}
  \end{center}
  Thus the sides of $\mathcal P$ must pair off,
  and when we finish drawing we have an $n/2$-gon.
\end{proof}
Since $n = 4$ works, this implies all powers of $2$ work
and it remains to study the case when $n$ is odd.

\begin{claim*}
  If $n > 1$ is odd, then $n$ works if and only if $n = 2^b+1$
  for some positive integer $b$.
\end{claim*}
\begin{proof}
  We cannot have all short sides part of small triangles
  for parity reasons, so some side, must be part of a big triangle.
  Since big triangles contain the center $O$,
  there can be at most one big triangle too.

  Then we get $\half(n-1)$ small triangles,
  pairing up the remaining sides.
  Now repeating the argument with the triangles on each half
  shows that the number $n-1$ must be a power of $2$, as needed.
\end{proof}
\pagebreak

\subsection{USAMO 2008/5, proposed by Kiran Kedlaya}
\textsl{Available online at \url{https://aops.com/community/p1116189}.}
\begin{mdframed}[style=mdpurplebox,frametitle={Problem statement}]
Three nonnegative real numbers $r_1$, $r_2$, $r_3$
are written on a blackboard.
These numbers have the property
that there exist integers $a_1$, $a_2$, $a_3$, not all zero,
satisfying $a_1r_1 + a_2r_2 + a_3r_3 = 0$.
We are permitted to perform the following operation:
find two numbers $x$, $y$ on the blackboard with $x \le y$,
then erase $y$ and write $y - x$ in its place.
Prove that after a finite number of such operations,
we can end up with at least one $0$ on the blackboard.
\end{mdframed}
We first show we can decrease the quantity
$\left\lvert a_1 \right\rvert + \left\lvert a_2 \right\rvert + \left\lvert a_3 \right\rvert$
as long as $0 \notin \left\{ a_1,a_2,a_3 \right\}$.
Assume $a_1 > 0$ and $r_1 > r_2 > r_3$
without loss of generality and consider two cases.

\begin{itemize}
  \ii $r_2 > 0$ or $r_3 > 0$; these cases are identical.
  If $r_2 > 0$ then $r_3 < 0$ and get
  \[ 0 = a_1r_1 + a_2r_2 + a_3r_3 > a_1r_3 + a_3r_3 \implies a_1 + a_3 < 0 \]
  so $\left\lvert a_1 + a_3 \right\rvert < \left\lvert a_3 \right\rvert$,
  and hence we perform $(r_1,r_2,r_3) \mapsto (r_1-r_3,r_2,r_3)$.

  \ii Both $r_2$ and $r_3$ are less than zero.
  Assume for contradiction that
  $\left\lvert a_1+a_2 \right\rvert \ge -a_2$
  and $\left\lvert a_1+a_3 \right\rvert \ge -a_3$ both hold
  (if either fails then we use $(r_1,r_2,r_3) \mapsto (r_1-r_2,r_2,r_3)$
  and $(r_1,r_2,r_3) \mapsto (r_1-r_3,r_2,r_3)$, respectively).
  Clearly $a_1+a_2$ and $a_1+a_3$ are both positive in this case,
  so we get $a_1 + 2a_2$ and $a_1 + 2a_3 \ge 0$;
  adding gives $a_1+a_2+a_3 \ge 0$.
  But
  \begin{align*}
    0 &= a_1r_1 + a_2r_2 + a_3r_3 \\
    &> a_1r_2 + a_2r_2 + a_3r_2 \\
    &= r_2(a_1+a_2+a_3)  \\
    \implies 0 &< a_1+a_2+a_3.
  \end{align*}
\end{itemize}

Since this covers all cases,
we see that we can always decrease
$\left\lvert a_1 \right\rvert + \left\lvert a_2 \right\rvert + \left\lvert a_3 \right\rvert$
whenever $0 \notin \left\{ a_1,a_2,a_3 \right\}$.
Because the $a_i$ are integers this cannot occur indefinitely,
so eventually one of the $a_i$'s is zero.
At this point we can just apply the Euclidean Algorithm, so we're done.
\pagebreak

\subsection{USAMO 2008/6, proposed by Sam Vandervelde}
\textsl{Available online at \url{https://aops.com/community/p1116182}.}
\begin{mdframed}[style=mdpurplebox,frametitle={Problem statement}]
At a certain mathematical conference,
every pair of mathematicians are either friends or strangers.
At mealtime, every participant eats in one of two large dining rooms.
Each mathematician insists upon eating in a room which
contains an even number of his or her friends.
Prove that the number of ways that the mathematicians
may be split between the two rooms is a power of two
(i.e.\ is of the form $2^k$ for some positive integer $k$).
\end{mdframed}
Take the obvious graph interpretation
where we are trying to $2$-color a graph.
Let $A$ be the adjacency matrix of the graph over $\FF_2$,
except the diagonal of $A$ has $\deg v \pmod 2$ instead of zero.
Then let $\vec d$ be the main diagonal.
Splittings then correspond to $A \vec v = \vec d$.
It's then immediate that the number of ways is either zero
or a power of two, since if it is nonempty it is a coset of $\ker A$.

Thus we only need to show that:
\begin{claim*}
  At least one coloring exists.
\end{claim*}
\begin{proof}
If not, consider a minimal counterexample $G$.
Clearly there is at least one odd vertex $v$.
Consider the graph with vertex set $G-v$,
where all pairs of neighbors of $v$ have their edges complemented.
By minimality, we have a good coloring here.
One can check that this extends to a good coloring on $G$
by simply coloring $v$ with the color matching an even number of its neighbors.
This breaks minimality of $G$, and hence all graphs $G$ have a coloring.
\end{proof}

It's also possible to use linear algebra.
We prove the following lemma:
\begin{lemma*}[grobber]
Let $V$ be a finite dimensional vector space,
$T \colon V \to V$ and $w \in V$.
Then $w$ is in the image of $T$ if and only if
there are no $\xi \in V^\vee$ for which $\xi(w) \neq 0$
and yet $\xi \circ T = 0$.
\end{lemma*}

\begin{proof}
Clearly if $T(v) = w$, then no $\xi$ exists.
Conversely, assume $w$ is not in the image of $T$.
Then the image of $T$ is linearly independent from $w$.
Take a basis $e_1$, \dots, $e_m$ for the image of $T$, add $w$,
and then extend it to a basis for all of $V$.
Then have $\xi$ kill all $e_i$ but not $w$.
\end{proof}

\begin{corollary*}
  In a symmetric matrix $A$ mod $2$,
  there exists a vector $v$ such that $Av$ is a copy of the diagonal of $A$.
\end{corollary*}
\begin{proof}
  Let $\xi$ be such that $\xi \circ T = 0$.
  Look at $\xi$ as a column vector $w^\top$, and let $d$ be the diagonal.
  Then \[ 0 = w^\top \cdot T \cdot w = \xi(d) \]
  because this extracts the sum of coefficients submatrix of $T$,
  and all the symmetric entries cancel off.
  Thus no $\xi$ as in the previous lemma exists.
\end{proof}
This corollary gives the desired proof.
\pagebreak


\end{document}
