% © Evan Chen
% Downloaded from https://web.evanchen.cc/

\documentclass[11pt]{scrartcl}
\usepackage[sexy]{evan}
\ihead{\footnotesize\textbf{\thetitle}}
\ohead{\footnotesize\href{http://web.evanchen.cc}{\ttfamily web.evanchen.cc},
    updated \today}
\title{IMO 2008 Solution Notes}
\date{\today}

\begin{document}

\maketitle

\begin{abstract}
This is a compilation of solutions
for the 2008 IMO.
Some of the solutions are my own work,
but many are from the official solutions provided by the organizers
(for which they hold any copyrights),
and others were found by users on the Art of Problem Solving forums.

These notes will tend to be a bit more advanced and terse than the ``official''
solutions from the organizers.
In particular, if a theorem or technique is not known to beginners
but is still considered ``standard'', then I often prefer to
use this theory anyways, rather than try to work around or conceal it.
For example, in geometry problems I typically use directed angles
without further comment, rather than awkwardly work around configuration issues.
Similarly, sentences like ``let $\mathbb{R}$ denote the set of real numbers''
are typically omitted entirely.

Corrections and comments are welcome!
\end{abstract}

\tableofcontents
\newpage

\addtocounter{section}{-1}
\section{Problems}
\begin{enumerate}[\bfseries 1.]
\item %% Problem 1
Let $H$ be the orthocenter of an acute-angled triangle $ABC$.
The circle $\Gamma_{A}$ centered at the midpoint of $\ol{BC}$ and passing
through $H$ intersects the sideline $BC$ at points  $A_1$ and $A_2$.
Similarly, define the points $B_1$, $B_2$, $C_1$, and $C_2$.
Prove that six points $A_1$, $A_2$, $B_1$, $B_2$, $C_1$, $C_2$ are concyclic.

\item %% Problem 2
Let $x$, $y$, $z$ be real numbers with $xyz = 1$,
all different from $1$.
Prove that
\[ \frac{x^2}{(x-1)^2} + \frac{y^2}{(y-1)^2}
+ \frac{z^2}{(z-1)^2} \ge 1 \]
and show that equality holds for infinitely many choices
of rational numbers $x$, $y$, $z$.

\item %% Problem 3
Prove that there are infinitely many positive integers $n$
such that $n^2+1$ has a prime factor greater than $2n + \sqrt{2n}$.

\item %% Problem 4
Find all functions $f$ from the positive reals to the positive reals such that
\[ \frac{f(w)^2 + f(x)^2}{f(y^2)+f(z^2)} = \frac{w^2+x^2}{y^2+z^2} \]
for all positive real numbers $w$, $x$, $y$, $z$ satisfying $wx=yz$.

\item %% Problem 5
Let $n$ and $k$ be positive integers
with $k \geq n$ and $k - n$ an even number.
There are $2n$ lamps labelled $1$, $2$, \dots, $2n$
each of which can be either on or off.
Initially all the lamps are off.
We consider sequences of steps:
at each step one of the lamps is switched
(from on to off or from off to on).
Let $N$ be the number of such sequences consisting of $k$ steps
and resulting in the state where lamps $1$ through $n$ are all on,
and lamps $n + 1$ through $2n$ are all off.
Let $M$ be number of such sequences consisting of $k$ steps,
resulting in the state where lamps $1$ through $n$ are all on,
and lamps $n + 1$ through $2n$ are all off,
but where none of the lamps $n + 1$ through $2n$ is ever switched on.
Determine $\frac{N}{M}$.

\item %% Problem 6
Let $ABCD$ be a convex quadrilateral with $BA \neq BC$.
Denote the incircles of triangles $ABC$ and $ADC$
by $\omega_1$ and $\omega_2$ respectively.
Suppose that there exists a circle $\omega$ tangent
to ray $BA$ beyond $A$ and to the ray $BC$ beyond $C$,
which is also tangent to the lines $AD$ and $CD$.
Prove that the common external tangents to
$\omega_1$ and $\omega_2$ intersect on $\omega$.

\end{enumerate}
\pagebreak

\section{Solutions to Day 1}
\subsection{IMO 2008/1}
\textsl{Available online at \url{https://aops.com/community/p1190553}.}
\begin{mdframed}[style=mdpurplebox,frametitle={Problem statement}]
Let $H$ be the orthocenter of an acute-angled triangle $ABC$.
The circle $\Gamma_{A}$ centered at the midpoint of $\ol{BC}$ and passing
through $H$ intersects the sideline $BC$ at points  $A_1$ and $A_2$.
Similarly, define the points $B_1$, $B_2$, $C_1$, and $C_2$.
Prove that six points $A_1$, $A_2$, $B_1$, $B_2$, $C_1$, $C_2$ are concyclic.
\end{mdframed}
Let $D$, $E$, $F$ be the centers of $\Gamma_A$, $\Gamma_B$, $\Gamma_C$
(in other words, the midpoints of the sides).

We first show that $B_1$, $B_2$, $C_1$, $C_2$ are concyclic.
It suffices to prove that $A$
lies on the radical axis of the circles $\Gamma_B$ and $\Gamma_C$.

\begin{center}
\begin{asy}
pair A = dir(110);
pair B = dir(210);
pair C = dir(330);
draw(A--B--C--cycle);

pair D = midpoint(B--C);
pair E = midpoint(C--A);
pair F = midpoint(A--B);
pair H = orthocenter(A, B, C);
draw(E--F);

pair X = -H+2*foot(H, E, F);
path w1 = Drawing(CP(E,H));
path w2 = Drawing(CP(F,H));
pair B_1 = IP(w1, A--C);
pair B_2 = OP(w1, A--C);
pair C_1 = IP(w2, A--B);
pair C_2 = OP(w2, A--B);

pair T = foot(A, B, C);
draw(A--T, dotted);

dot("$A$", A, dir(A));
dot("$B$", B, dir(B));
dot("$C$", C, dir(C));
dot("$D$", D, dir(D));
dot("$E$", E, dir(E));
dot("$F$", F, dir(F));
dot("$H$", H, 1.4*dir(-10));
dot("$X$", X, 1.4*dir(10));
dot("$B_1$", B_1, 1.5*dir(90));
dot("$B_2$", B_2, 1.2*dir(-10));
dot("$C_1$", C_1, 1.2*dir(120));
dot("$C_2$", C_2, dir(C_2));

/* TSQ Source:

A = dir 110
B = dir 210
C = dir 330
A--B--C--cycle

D = midpoint B--C
E = midpoint C--A
F = midpoint A--B
H = orthocenter A B C 1.4R-10
E--F

X = -H+2*foot H E F 1.4R10
! path w1 = Drawing(CP(E,H));
! path w2 = Drawing(CP(F,H));
B_1 = IP w1 A--C 1.5R90
B_2 = OP w1 A--C 1.2R-10
C_1 = IP w2 A--B 1.2R120
C_2 = OP w2 A--B

T := foot A B C
A--T dotted

*/
\end{asy}
\end{center}

Let $X$ be the second intersection of $\Gamma_B$ and $\Gamma_C$.
Clearly $\ol{XH}$ is perpendicular to the line
joining the centers of the circles, namely $\ol{EF}$.
But $\ol{EF} \parallel \ol{BC}$, so $\ol{XH} \perp \ol{BC}$.
Since $\ol{AH} \perp \ol{BC}$ as well,
we find that $A$, $X$, $H$ are collinear, as needed.

Thus, $B_1$, $B_2$, $C_1$, $C_2$ are concyclic.
Similarly, $C_1$, $C_2$, $A_1$, $A_2$ are concyclic,
as are $A_1$, $A_2$, $B_1$, $B_2$.
Now if any two of these three circles coincide, we are done;
else the pairwise radical axii are not concurrent, contradiction.
(Alternatively, one can argue directly that $O$ is the center of all
three circles, by taking the perpendicular bisectors.)
\pagebreak

\subsection{IMO 2008/2}
\textsl{Available online at \url{https://aops.com/community/p1190551}.}
\begin{mdframed}[style=mdpurplebox,frametitle={Problem statement}]
Let $x$, $y$, $z$ be real numbers with $xyz = 1$,
all different from $1$.
Prove that
\[ \frac{x^2}{(x-1)^2} + \frac{y^2}{(y-1)^2}
+ \frac{z^2}{(z-1)^2} \ge 1 \]
and show that equality holds for infinitely many choices
of rational numbers $x$, $y$, $z$.
\end{mdframed}
Let $x=a/b$, $y=b/c$, $z=c/a$, so we want to show
\[ \left(\frac{a}{a-b}\right)^2+\left(\frac{b}{b-c}\right)^2
  +\left(\frac{c}{c-a}\right)^2\ge 1.\]
A very boring computation shows this is equivalent to
\[ \frac{(a^2b+b^2c+c^2a-3abc)^2}%
  {(a-b)^2(b-c)^2(c-a)^2}\ge 0\]
which proves the inequality
(and it is unsurprising we are in such a situation,
given that there is an infinite curve of rationals).

For equality, it suffices to show there are infinitely
many integer solutions to
\[ a^2b+b^2c+c^2a=3abc
  \iff \frac ac + \frac ba + \frac ca = 3
\]
or equivalently that
there are infinitely many rational solutions to
\[ u + v + \frac{1}{uv} = 3. \]
For any $0 \neq u \in \QQ$ the real solution for $u$ is
\[ v = \frac{-u + (u-1)\sqrt{1-4/u} + 3}{2} \]
and there are certainly infinitely many rational numbers $u$
for which $1-4/u$ is a rational square
(say, $u = \frac{-4}{q^2-1}$ for $q \neq \pm 1$ a rational number).
\pagebreak

\subsection{IMO 2008/3}
\textsl{Available online at \url{https://aops.com/community/p1190546}.}
\begin{mdframed}[style=mdpurplebox,frametitle={Problem statement}]
Prove that there are infinitely many positive integers $n$
such that $n^2+1$ has a prime factor greater than $2n + \sqrt{2n}$.
\end{mdframed}
The idea is to pick the prime $p$ first!

Select any large prime $p \ge 2013$,
and let $h = \left\lceil \sqrt p \right\rceil$.
We will try to find an $n$ such that
\[ n \le \frac 12 (p-h) \quad \text{and} \quad p \mid n^2+1. \]
This implies $p \ge 2n+\sqrt{p}$
which is enough to ensure $p \ge 2n + \sqrt{2n}$.

Assume $p \equiv 1 \pmod 8$ henceforth.
Then there exists some $\frac 12 p < x < p$
such that $x^2 \equiv -1 \pmod p$,
and we set \[ x = \frac{p+1}{2} + t. \]
\begin{claim*}
  We have $t \ge \frac{h-1}{2}$ and hence may take $n = p-x$.
\end{claim*}
\begin{proof}
Assume for contradiction this is false; then
\begin{align*}
  0 &\equiv  4(x^2+1) \pmod{p} \\
  &= \left( p+1+2t \right)^2 + 4 \\
  &\equiv (2t+1)^2 + 4 \pmod{p} \\
  &< h^2+4
\end{align*}
So we have that $(2t+1)^2+4$ is positive and divisible by $p$,
yet at most $\left\lceil \sqrt{p} \right\rceil^2 + 4 < 2p$.
So it must be the case that $(2t+1)^2+4 = p$,
but this has no solutions modulo $8$.
\end{proof}
\pagebreak

\section{Solutions to Day 2}
\subsection{IMO 2008/4}
\textsl{Available online at \url{https://aops.com/community/p1191683}.}
\begin{mdframed}[style=mdpurplebox,frametitle={Problem statement}]
Find all functions $f$ from the positive reals to the positive reals such that
\[ \frac{f(w)^2 + f(x)^2}{f(y^2)+f(z^2)} = \frac{w^2+x^2}{y^2+z^2} \]
for all positive real numbers $w$, $x$, $y$, $z$ satisfying $wx=yz$.
\end{mdframed}
The answers are $f(x) \equiv x$ and $f(x) \equiv 1/x$.
These work, so we show they are the only ones.

First, setting $(t,t,t,t)$ gives $f(t^2) = f(t)^2$.
In particular, $f(1) = 1$.
Next, setting $(t, 1, \sqrt t, \sqrt t)$ gives
\[ \frac{f(t)^2 + 1}{2f(t)} = \frac{t^2 + 1}{2t} \]
which as a quadratic implies $f(t) \in \{t, 1/t\}$.

Now assume $f(a) = a$ and $f(b) = 1/b$.
Setting $(\sqrt a, \sqrt b, 1, \sqrt{ab})$ gives
\[ \frac{a + 1/b}{f(ab) + 1} = \frac{a+b}{ab+1}. \]
One can check the two cases on $f(ab)$ each imply
$a=1$ and $b=1$ respectively.
Hence the only answers are those claimed.
\pagebreak

\subsection{IMO 2008/5}
\textsl{Available online at \url{https://aops.com/community/p1191679}.}
\begin{mdframed}[style=mdpurplebox,frametitle={Problem statement}]
Let $n$ and $k$ be positive integers
with $k \geq n$ and $k - n$ an even number.
There are $2n$ lamps labelled $1$, $2$, \dots, $2n$
each of which can be either on or off.
Initially all the lamps are off.
We consider sequences of steps:
at each step one of the lamps is switched
(from on to off or from off to on).
Let $N$ be the number of such sequences consisting of $k$ steps
and resulting in the state where lamps $1$ through $n$ are all on,
and lamps $n + 1$ through $2n$ are all off.
Let $M$ be number of such sequences consisting of $k$ steps,
resulting in the state where lamps $1$ through $n$ are all on,
and lamps $n + 1$ through $2n$ are all off,
but where none of the lamps $n + 1$ through $2n$ is ever switched on.
Determine $\frac{N}{M}$.
\end{mdframed}
The answer is $2^{k-n}$.

Consider the following map $\Psi$ from $N$-sequences to $M$-sequences:
\begin{itemize}
  \ii change every instance of $n+1$ to $1$;
  \ii change every instance of $n+2$ to $2$;
  \ii[$\vdots$]
  \ii change every instance of $2n$ to $n$.
\end{itemize}
(For example, suppose $k=9$, $n=3$;
then $144225253 \mapsto 111222223$.)

Clearly this is map is well-defined and surjective.
So all that remains is:
\begin{claim*}
  Every $M$-sequence has exactly $2^{k-n}$ pre-images under $\Psi$.
\end{claim*}
\begin{proof}
  Indeed, suppose that there are $c_1$ instances of lamp $1$.
  Then we want to pick an odd subset of the $1$'s to change to $n+1$'s,
  so $2^{c_1 - 1}$ ways to do this.
  And so on.
  Hence the number of pre-images is
  \[ \prod_i 2^{c_i - 1} = 2^{k-n}. \qedhere \]
\end{proof}
\pagebreak

\subsection{IMO 2008/6}
\textsl{Available online at \url{https://aops.com/community/p1191671}.}
\begin{mdframed}[style=mdpurplebox,frametitle={Problem statement}]
Let $ABCD$ be a convex quadrilateral with $BA \neq BC$.
Denote the incircles of triangles $ABC$ and $ADC$
by $\omega_1$ and $\omega_2$ respectively.
Suppose that there exists a circle $\omega$ tangent
to ray $BA$ beyond $A$ and to the ray $BC$ beyond $C$,
which is also tangent to the lines $AD$ and $CD$.
Prove that the common external tangents to
$\omega_1$ and $\omega_2$ intersect on $\omega$.
\end{mdframed}
By the external version of Pitot theorem, the existence
of $\omega$ implies that
\[ BA + AD = CB + CD. \]
Let $\ol{PQ}$ and $\ol{ST}$ be diameters of $\omega_1$ and $\omega_2$
with $P, T \in \ol{AC}$.
Then the length relation on $ABCD$ implies that $P$ and $T$
are reflections about the midpoint of $\ol{AC}$.

Now orient $AC$ horizontally and let $K$ be the ``uppermost'' point of $\omega$, as shown.

\begin{center}
\begin{asy}
size(12cm);
pair W = dir(48.4);
pair X = dir(68.4);
pair Y = dir(138.4);
pair Z = dir(173.4);

pair A = 2*X*Z/(X+Z);
pair B = 2*W*Z/(W+Z);
pair C = 2*W*Y/(W+Y);
pair D = 2*X*Y/(X+Y);

draw(arc(origin, 1, 40, 180), dashed+orange);

filldraw(incircle(A, B, C), opacity(0.1)+lightred, red);
filldraw(incircle(A, D, C), opacity(0.1)+lightred, red);
filldraw(A--B--C--D--cycle, opacity(0.1)+yellow, orange);
draw(A--C, red);

pair I_B = incenter(A, B, C);
pair I_D = incenter(A, D, C);
pair P = foot(I_B, A, C);
pair Q = 2*I_B-P;
pair T = foot(I_D, A, C);
pair S = 2*I_D-T;

draw(A--Z, orange);
draw(C--W, orange);
pair E = 2*Y*Z/(Y+Z);
pair F = 2*X*W/(X+W);
draw(E--D--F, orange);

pair K = extension(B, Q, D, P);
draw(B--K--P, dashed+red);
draw(P--Q, red);
draw(S--T, red);

dot("$W$", W, dir(W));
dot("$X$", X, dir(95));
dot("$Y$", Y, dir(Y));
dot("$Z$", Z, dir(Z));
dot("$A$", A, dir(135));
dot("$B$", B, dir(B));
dot("$C$", C, dir(45));
dot("$D$", D, dir(-D));
dot("$P$", P, dir(270));
dot("$Q$", Q, 0.8*dir(95));
dot("$T$", T, dir(45));
dot("$S$", S, dir(225));
dot("$K$", K, dir(315));

/* TSQ Source:

!size(12cm);
W = dir 48.4
X = dir 68.4 R95
Y = dir 138.4
Z = dir 173.4

A = 2*X*Z/(X+Z) R135
B = 2*W*Z/(W+Z)
C = 2*W*Y/(W+Y) R45
D = 2*X*Y/(X+Y) R-D

!draw(arc(origin, 1, 40, 180), dashed+orange);

incircle A B C 0.1 lightred / red
incircle A D C 0.1 lightred / red
A--B--C--D--cycle 0.1 yellow / orange
A--C red

I_B := incenter A B C
I_D := incenter A D C
P = foot I_B A C R270
Q = 2*I_B-P 0.8R95
T = foot I_D A C R45
S = 2*I_D-T R225

A--Z orange
C--W orange
E := 2*Y*Z/(Y+Z)
F := 2*X*W/(X+W)
E--D--F orange

K = extension B Q D P R315
B--K--P dashed red
P--Q red
S--T red

*/
\end{asy}
\end{center}

Consequently, a homothety at $B$ maps $Q$, $T$, $K$ to each other
(since $T$ is the uppermost of the excircle, $Q$ of the incircle).
Similarly, a homothety at $D$ maps $P$, $S$, $K$ to each other.
As $\ol{PQ}$ and $\ol{ST}$ are parallel diameters
it then follows $K$ is the exsimilicenter of $\omega_1$ and $\omega_2$.
\pagebreak


\end{document}
