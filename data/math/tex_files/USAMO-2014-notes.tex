% © Evan Chen
% Downloaded from https://web.evanchen.cc/

\documentclass[11pt]{scrartcl}
\usepackage[sexy]{evan}
\ihead{\footnotesize\textbf{\thetitle}}
\ohead{\footnotesize\href{http://web.evanchen.cc}{\ttfamily web.evanchen.cc},
    updated \today}
\title{USAMO 2014 Solution Notes}
\date{\today}

\begin{document}

\maketitle

\begin{abstract}
This is a compilation of solutions
for the 2014 USAMO.
Some of the solutions are my own work,
but many are from the official solutions provided by the organizers
(for which they hold any copyrights),
and others were found by users on the Art of Problem Solving forums.

These notes will tend to be a bit more advanced and terse than the ``official''
solutions from the organizers.
In particular, if a theorem or technique is not known to beginners
but is still considered ``standard'', then I often prefer to
use this theory anyways, rather than try to work around or conceal it.
For example, in geometry problems I typically use directed angles
without further comment, rather than awkwardly work around configuration issues.
Similarly, sentences like ``let $\mathbb{R}$ denote the set of real numbers''
are typically omitted entirely.

Corrections and comments are welcome!
\end{abstract}

\tableofcontents
\newpage

\addtocounter{section}{-1}
\section{Problems}
\begin{enumerate}[\bfseries 1.]
\item %% Problem 1
Let $a$, $b$, $c$, $d$ be real numbers
such that $b-d \ge 5$ and all zeros
$x_1$, $x_2$, $x_3$, and $x_4$ of the polynomial
$P(x)=x^4+ax^3+bx^2+cx+d$ are real.
Find the smallest value the product
$(x_1^2+1)(x_2^2+1)(x_3^2+1)(x_4^2+1)$ can take.

\item %% Problem 2
Find all $f :\ZZ \to \ZZ$ such that
\[
  xf\left( 2f(y)-x \right) + y^2f\left( 2x-f(y) \right)
  = \frac{f(x)^2}{x} + f\left( yf(y) \right)
\]
for all $x,y \in \ZZ$ such that $x \neq 0$.

\item %% Problem 3
Prove that there exists an infinite set of points
\[ \dots, \; P_{-3}, \; P_{-2},\; P_{-1},\; P_0,\; P_1,\; P_2,\; P_3,\; \dots \]
in the plane with the following property:
For any three distinct integers $a$, $b$, and $c$,
points $P_a$, $P_b$, and $P_c$ are collinear if and only if $a+b+c=2014$.

\item %% Problem 4
Let $k$ be a positive integer.
Two players $A$ and $B$ play a game on an infinite grid of regular hexagons.
Initially all the grid cells are empty.
Then the players alternately take turns with $A$ moving first.
In her move, $A$ may choose two adjacent hexagons in the grid
which are empty and place a counter in both of them.
In his move, $B$ may choose any counter on the board and remove it.
If at any time there are $k$ consecutive grid cells
in a line all of which contain a counter, $A$ wins.
Find the minimum value of $k$ for which $A$ cannot
win in a finite number of moves, or prove that no such minimum value exists.

\item %% Problem 5
Let $ABC$ be a triangle with orthocenter $H$ and
let $P$ be the second intersection of the circumcircle of
triangle $AHC$ with the internal bisector of $\angle BAC$.
Let $X$ be the circumcenter of triangle $APB$
and let $Y$ be the orthocenter of triangle $APC$.
Prove that the length of segment $XY$ is equal to
the circumradius of triangle $ABC$.

\item %% Problem 6
Prove that there is a constant $c>0$ with the following property:
If $a$, $b$, $n$ are positive integers such that $\gcd(a+i, b+j)>1$
for all $i, j \in \{0, 1, \dots, n\}$, then
\[ \min\{a, b\}> (cn)^{n/2}. \]

\end{enumerate}
\pagebreak

\section{Solutions to Day 1}
\subsection{USAMO 2014/1, proposed by Titu Andreescu}
\textsl{Available online at \url{https://aops.com/community/p3477753}.}
\begin{mdframed}[style=mdpurplebox,frametitle={Problem statement}]
Let $a$, $b$, $c$, $d$ be real numbers
such that $b-d \ge 5$ and all zeros
$x_1$, $x_2$, $x_3$, and $x_4$ of the polynomial
$P(x)=x^4+ax^3+bx^2+cx+d$ are real.
Find the smallest value the product
$(x_1^2+1)(x_2^2+1)(x_3^2+1)(x_4^2+1)$ can take.
\end{mdframed}
The answer is $\boxed{16}$.
This can be achieved by taking $x_1 = x_2 = x_3 = x_4 = 1$,
whence the product is $2^4 = 16$, and $b-d = 5$.

Now, we prove this is a lower bound.
Let $i = \sqrt{-1}$.
The key observation is that
\[ \prod_{j=1}^4 \left( x_j^2 + 1 \right)
  = \prod_{j=1}^4 (x_j - i)(x_j + i)
  = P(i)P(-i). \]
Consequently, we have
\begin{align*}
  \left( x_1^2 + 1 \right)
  \left( x_2^2 + 1 \right)
  \left( x_3^2 + 1 \right)
  \left( x_1^2 + 1 \right)
  &= (b-d-1)^2 + (a-c)^2 \\
  &\ge (5-1)^2 + 0^2 = 16.
\end{align*}
This proves the lower bound.
\pagebreak

\subsection{USAMO 2014/2, proposed by Titu Andreescu}
\textsl{Available online at \url{https://aops.com/community/p3477690}.}
\begin{mdframed}[style=mdpurplebox,frametitle={Problem statement}]
Find all $f :\ZZ \to \ZZ$ such that
\[
  xf\left( 2f(y)-x \right) + y^2f\left( 2x-f(y) \right)
  = \frac{f(x)^2}{x} + f\left( yf(y) \right)
\]
for all $x,y \in \ZZ$ such that $x \neq 0$.
\end{mdframed}
The answer is $f(x) \equiv 0$ and $f(x) \equiv x^2$. Check that these work.

Now let's prove these are the only solutions.
Put $y=0$ to obtain
\[ x f\left( 2f(0)-x \right) = \frac{f(x)^2}{x} + f(0).  \]

The nicest part of the problem is the following step:
\begin{claim*}
  We have $f(0)=0$
\end{claim*}
\begin{proof}
  If not, select a prime $p \nmid f(0)$ and put $x=p \neq 0$.
  In the above, we find that $p \mid f(p)^2$,
  so $p \mid f(p)$ and hence $p \mid \tfrac{f(p)^2}{p}$.
  From here we derive $p \mid f(0)$, contradiction.
  Hence \[ f(0) = 0. \]
\end{proof}

\begin{claim*}
  We have $f(x) \in \{0,x^2\}$ for each individual $x$.
\end{claim*}
\begin{proof}
  The above then implies that
  \[ x^2f(-x) = f(x)^2 \]
  holds for all nonzero $x$, but also for $x=0$.
  Let us now check that $f$ is an even function.
  In the above, we may also derive $f(-x)^2 = x^2f(x)$.
  If $f(x) \neq f(-x)$ (and hence $x \neq 0$),
  then subtracting the above and factoring implies that
  $f(x) + f(-x) = -x^2$;
  we can then obtain by substituting the relation
  \[ \left[ f(x) + \frac 12x^2 \right]^2 = -\frac 34 x^4 < 0 \]
  which is impossible.
  This means $f(x)^2 = x^2f(x)$, thus
  \[ f(x) \in \{0, x^2\} \qquad \forall x. \]
\end{proof}

Now suppose there exists a nonzero integer $t$ with $f(t) = 0$.
We will prove that $f(x) \equiv 0$.
Put $y=t$ in the given to obtain that
\[ t^2 f(2x) = 0 \]
for any integer $x \neq 0$, and hence conclude that $f(2 \ZZ) \equiv 0$.
Then selecting $x = 2k \neq 0$ in the given implies that
\[ y^2 f(4k-f(y)) = f(yf(y)). \]

Assume for contradiction that $f(m) = m^2$ now for some odd $m \neq 0$.
Evidently \[ m^2 f(4k-m^2) = f(m^3). \]
If $f(m^3) \neq 0$ this forces $f(4k-m^2) \neq 0$,
and hence $m^2(4k-m^2)^2 = m^6$ for arbitrary $k \neq 0$, which is clearly absurd.
That means \[ f(4k-m^2) = f(m^2-4k) = f(m^3) = 0 \] for each $k \neq 0$.
Since $m$ is odd, $m^2 \equiv 1 \pmod 4$,
and so $f(n) = 0$ for all $n$ other than $\pm m^2$
(since we cannot select $k=0$).

Now $f(m) = m^2$ means that $m = \pm 1$.
Hence either $f(x) \equiv 0$ or
\[ f(x) = \begin{cases} 1 & x = \pm 1 \\ 0 & \text{otherwise}. \end{cases} \]
To show that the latter fails,
we simply take $x=5$ and $y=1$ in the given.

Hence, the only solutions are $f(x) \equiv 0$ and $f(x) \equiv x^2$.
\pagebreak

\subsection{USAMO 2014/3, proposed by Sam Vandervelde}
\textsl{Available online at \url{https://aops.com/community/p3477763}.}
\begin{mdframed}[style=mdpurplebox,frametitle={Problem statement}]
Prove that there exists an infinite set of points
\[ \dots, \; P_{-3}, \; P_{-2},\; P_{-1},\; P_0,\; P_1,\; P_2,\; P_3,\; \dots \]
in the plane with the following property:
For any three distinct integers $a$, $b$, and $c$,
points $P_a$, $P_b$, and $P_c$ are collinear if and only if $a+b+c=2014$.
\end{mdframed}
The construction
\[ P_n = \left( n-\frac{2014}{3},
  \left( n-\frac{2014}{3} \right)^3 \right) \]
works fine, and follows from the following claim:
\begin{claim*}
  If $x$, $y$, $z$ are distinct real numbers
  then the points $(x,x^3)$, $(y,y^3)$, $(z,z^3)$
  are collinear if and only if $x+y+z=0$.
\end{claim*}
\begin{proof}
  Note that by the ``shoelace formula'',
  the collinearity is equivalent to
  \[
    0 =
    \det \begin{bmatrix}
      x & x^3 & 1 \\
      y & y^3 & 1 \\
      z & z^3 & 1 \\
    \end{bmatrix}
  \]
  But the determinant equals
  \[ \sum_{\text{cyc}} x(y^3-z^3)
    = (x-y)(y-z)(z-x)(x+y+z). \qedhere  \]
\end{proof}
\pagebreak

\section{Solutions to Day 2}
\subsection{USAMO 2014/4, proposed by Palmer Mebane}
\textsl{Available online at \url{https://aops.com/community/p3478584}.}
\begin{mdframed}[style=mdpurplebox,frametitle={Problem statement}]
Let $k$ be a positive integer.
Two players $A$ and $B$ play a game on an infinite grid of regular hexagons.
Initially all the grid cells are empty.
Then the players alternately take turns with $A$ moving first.
In her move, $A$ may choose two adjacent hexagons in the grid
which are empty and place a counter in both of them.
In his move, $B$ may choose any counter on the board and remove it.
If at any time there are $k$ consecutive grid cells
in a line all of which contain a counter, $A$ wins.
Find the minimum value of $k$ for which $A$ cannot
win in a finite number of moves, or prove that no such minimum value exists.
\end{mdframed}
The answer is $k = 6$.

\medskip

\textbf{Proof that $A$ cannot win if $k=6$}.
We give a strategy for $B$ to prevent $A$'s victory.
Shade in every third cell, as shown in the figure below.
Then $A$ can never cover two shaded cells simultaneously on her turn.
Now suppose $B$ always removes a counter on a shaded cell
(and otherwise does whatever he wants).
Then he can prevent $A$ from ever getting six consecutive counters,
because any six consecutive cells contain two shaded cells.

\begin{center}
  \begin{asy}
    path hexagon = scale(0.5)*(dir(30)--dir(90)--dir(150)--dir(210)--dir(270)--dir(330)--cycle);
    unitsize(0.6cm);
    void mark(int x, int y) {
      pair P = dir(0)*x + dir(60)*y;
      filldraw(shift(P)*hexagon, purple, black+1);
    }
    void spot(int x, int y) {
      pair P = dir(0)*x + dir(60)*y;
      draw(shift(P)*hexagon);
    }
    int N = 4;
    for (int x=-N; x<=N; ++x) {
      for (int y=-N; y<=N; ++y) {
        if ( (abs(x+y)) > N ) continue;
        if (
          (x-y) % 3 == 0
        ) mark(x,y);
        else spot(x,y);
      }
    }
  \end{asy}
\end{center}

\medskip

\textbf{Example of a strategy for $A$ when $k=5$}.
We describe a winning strategy for $A$ explicitly.
Note that after $B$'s first turn there is one counter,
so then $A$ may create an equilateral triangle,
and hence after $B$'s second turn there are two consecutive counters.
Then, on her third turn,
$A$ places a pair of counters two spaces away on the same line.
Label the two inner cells $x$ and $y$ as shown below.
\begin{center}
  \begin{asy}
    path hexagon = scale(0.5)*(dir(30)--dir(90)--dir(150)--dir(210)--dir(270)--dir(330)--cycle);
    unitsize(0.6cm);
    void mark(int x, int y) {
      pair P = dir(0)*x + dir(60)*y;
      filldraw(shift(P)*hexagon, palecyan, black);
      filldraw(CR(P, 0.3), paleblue, blue);
    }
    void spot(int x, int y) {
      pair P = dir(0)*x + dir(60)*y;
      draw(shift(P)*hexagon);
    }
    for (int x=-3; x<=4; ++x) {
      for (int y=-2; y<=2; ++y) {
        if ( x+y < -3 ) continue;
        if ( x+y > 4 ) continue;
        spot(x,y);
      }
    }
    mark(-2, 0);
    mark(-1, 0);
    mark( 2, 0);
    mark( 3, 0);
    label("$x$", (-1)*dir(0), deepgreen);
    label("$y$", (2)*dir(0), deepgreen);
  \end{asy}
\end{center}
Now it is $B$'s turn to move;
in order to avoid losing immediately,
he must remove either $x$ or $y$.
Then on any subsequent turn, $A$ can replace $x$ or $y$
(whichever was removed)
and add one more adjacent counter.
This continues until either $x$ or $y$ has all its neighbors filled
(we ask $A$ to do so in such a way that
she avoids filling in the two central cells between $x$ and $y$
as long as possible).

So, let's say without loss of generality (by symmetry) that $x$
is completely surrounded by tokens.
Again, $B$ must choose to remove $x$ (or $A$ wins on her next turn).
After $x$ is removed by $B$, consider the following figure.
\begin{center}
  \begin{asy}
    path hexagon = scale(0.5)*(dir(30)--dir(90)--dir(150)--dir(210)--dir(270)--dir(330)--cycle);
    unitsize(0.6cm);
    void mark(int x, int y) {
      pair P = dir(0)*x + dir(60)*y;
      filldraw(shift(P)*hexagon, palecyan, black);
      filldraw(CR(P, 0.3), paleblue, blue);
    }
    void spot(int x, int y) {
      pair P = dir(0)*x + dir(60)*y;
      draw(shift(P)*hexagon);
    }
    for (int x=-4; x<=5; ++x) {
      for (int y=-3; y<=3; ++y) {
        if ( x+y < -4 ) continue;
        if ( x+y > 5 ) continue;
        spot(x,y);
      }
    }
    mark(-2, 0);
    mark( 0, 0);
    mark( 0, -1);
    mark(-1, 1);
    mark(-1, -1);
    mark(-2, 1);

    mark( 2, 0);
    mark( 2, 1);
    mark( 3, 0);
    label("$x$", (-1)*dir(0), deepgreen);
    label("$y$", (2)*dir(0), deepgreen);
    void win(int x, int y, pen p1, pen p2) {
      pair P = dir(0)*x + dir(60)*y;
      filldraw(shift(P)*hexagon, p1, p2);
    }
    win(1,0, lightgreen, deepgreen);
    win(0,1, lightgreen, deepgreen);
    win(0,2, lightred, red);
    win(0,3, lightred, red);
    win(4,0, lightred, red);
    win(5,0, lightred, red);
  \end{asy}
\end{center}
We let $A$ play in the two marked green cells.
Then, regardless of what move $B$ plays,
one of the two choices of moves marked in red lets $A$ win.
Thus, we have described a winning strategy when $k=5$ for $A$.
\pagebreak

\subsection{USAMO 2014/5, proposed by Titu Andreescu, Cosmin Pohoata}
\textsl{Available online at \url{https://aops.com/community/p3478581}.}
\begin{mdframed}[style=mdpurplebox,frametitle={Problem statement}]
Let $ABC$ be a triangle with orthocenter $H$ and
let $P$ be the second intersection of the circumcircle of
triangle $AHC$ with the internal bisector of $\angle BAC$.
Let $X$ be the circumcenter of triangle $APB$
and let $Y$ be the orthocenter of triangle $APC$.
Prove that the length of segment $XY$ is equal to
the circumradius of triangle $ABC$.
\end{mdframed}
\begin{center}
\begin{asy}
  size(8cm);
  pair A = Drawing("A", dir(110), dir(110));
  pair B = Drawing("B", dir(210), dir(210));
  pair C = Drawing("C", dir(330), dir(330));
  draw(A--B--C--cycle);
  draw(unitcircle);
  pair Q = dir(30); // 330 + 1/2 * (330-210) (mod 360)
  Drawing("Q", Q, Q);
  pair P = A+C-A*C*conj(Q);
  P = Drawing("P", P, dir(Q-P));
  pair X = Drawing("X", circumcenter(A,P,B), dir(225));
  pair Y = Drawing("Y", orthocenter(A,P,C), dir(225));
  pair X1 = Drawing("X'", A+C-A*C*conj(X), dir(45));
  pair Y1 = Drawing("Y'", A+C-A*C*conj(Y), dir(45));
  Drawing("H", A+B+C, dir(170));
  draw(circumcircle(A,P,C));
  pair B1 = Drawing("B'", A+C-A*C*conj(B), dir(45));
  draw(A--Q--C--cycle);
  draw(X--Y);
  draw(X1--Y1);
  Drawing("O", origin, dir(45));
\end{asy}
\end{center}

We eliminate the floating orthocenter
by reflecting $P$ across $\ol{AC}$ to $Q$.
Then $Q$ lies on $(ABC)$ and moreover $\angle QAC = \tfrac 12 \angle BAC$.
This motivates us to reflect $B$, $X$, $Y$
to $B'$, $X'$, $Y'$ and complex bash with respect to $\triangle AQC$.
Obviously \[ y' = a + q + c. \]
Now we need to compute $x'$.
You can get this using the formula
\[ x' = a + \frac{(b'-a)(q-a)\left( \ol{q-a}-\ol{b'-a} \right)}{(b'-a)\ol{(q-a)}
  - \ol{(b'-a)}(q-a)}. \]
Using the angle condition we know
$b = \frac{c^3}{q^2}$, and then that
\[ b' = a+c - ac\ol b = a+c-\frac{aq^2}{c^2}. \]
Therefore
\begin{align*}
  x' &= a + \frac{\left( c-\frac{aq^2}{c^2} \right)\left( q-a \right)\left( \frac 1q - \frac 1a - \frac 1c + \frac{c^2}{aq^2}  \right)}{\left( c-\frac{aq^2}{c^2} \right)\left( \frac 1q - \frac 1a \right) - \left( \frac 1c - \frac{c^2}{aq^2} \right)\left( q-a \right)} \\
  &= a + \frac{\frac{c^3-aq^2}{c^2} \left( q-a \right)\left( \frac 1q - \frac 1a - \frac 1c + \frac{c^2}{aq^2} \right)}{-\frac{c^3-aq^2}{c^2}\frac{q-a}{qa} + \frac{c^3-aq^2}{aq^2c}(q-a)} \\
  &= a + \frac{\frac 1q - \frac 1a - \frac 1c + \frac{c^2}{aq^2}}{-\frac{1}{qa} + \frac{c}{aq^2}} \\
  &= a + \frac{c^2-q^2 + aq-\frac{aq^2}{c}}{c-q} \\
  &= a + c + q + \frac{aq}{c}
\end{align*}
whence
\[ \left\lvert x'-y' \right\rvert
  = \left\lvert \frac{aq}{c} \right\rvert
  = 1. \]
\pagebreak

\subsection{USAMO 2014/6, proposed by Gabriel Dospinescu}
\textsl{Available online at \url{https://aops.com/community/p3478578}.}
\begin{mdframed}[style=mdpurplebox,frametitle={Problem statement}]
Prove that there is a constant $c>0$ with the following property:
If $a$, $b$, $n$ are positive integers such that $\gcd(a+i, b+j)>1$
for all $i, j \in \{0, 1, \dots, n\}$, then
\[ \min\{a, b\}> (cn)^{n/2}. \]
\end{mdframed}
Let $N = n+1$ and assume $N$ is (very) large.
We construct an $N \times N$ with cells $(i,j)$
where $0 \le i, j \le n$ and in each cell
place a prime $p$ dividing $\gcd (a+i, b+j)$.

The central claim is at least $50\%$ of the primes
in this table exceed $0.001n^2$.
We count the maximum number of squares they could occupy:
\[
  \sum_p \left\lceil \frac{N}{p} \right\rceil^2
  \le \sum_p \left( \frac Np + 1 \right)^2
  = N^2 \sum_p \frac{1}{p^2} + 2N \sum_p \frac1p + \sum_p 1. \]
Here the summation runs over primes $p \le 0.001n^2$.

Let $r = \pi(0.001n^2)$ denote the number of such primes.
Now we consider the following three estimates.
First, \[ \sum_p \frac{1}{p^2} < \frac 12 \]
which follows by adding all the primes directly with some computation.
Moreover,
\[ \sum_p \frac 1p < \sum_{k=1}^r \frac 1k = O(\log r) < o(N) \]
using the harmonic series bound,
and \[ \sum_p 1 < r \sim O \left( \frac{N^2}{\ln N} \right) < o(N^2) \]
via Prime Number Theorem.
Hence the sum in question is certainly less than $\half N^2$ for $N$ large
enough, establishing the central claim.

Hence some column $a+i$ has at least one
half of its primes greater than $0.001n^2$.
Because this is greater than $n$ for large $n$,
these primes must all be distinct,
so $a+i$ exceeds their product,
which is larger than
\[ \left( 0.001n^2 \right)^{N/2} > c^n \cdot n^n \]
where $c$ is some constant (better than the requested bound).
\pagebreak


\end{document}
