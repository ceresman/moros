% © Evan Chen
% Downloaded from https://web.evanchen.cc/

\documentclass[11pt]{scrartcl}
\usepackage[sexy]{evan}
\ihead{\footnotesize\textbf{\thetitle}}
\ohead{\footnotesize\href{http://web.evanchen.cc}{\ttfamily web.evanchen.cc},
    updated \today}
\title{JMO 2022 Solution Notes}
\date{\today}

\begin{document}

\maketitle

\begin{abstract}
This is a compilation of solutions
for the 2022 JMO.
Some of the solutions are my own work,
but many are from the official solutions provided by the organizers
(for which they hold any copyrights),
and others were found by users on the Art of Problem Solving forums.

These notes will tend to be a bit more advanced and terse than the ``official''
solutions from the organizers.
In particular, if a theorem or technique is not known to beginners
but is still considered ``standard'', then I often prefer to
use this theory anyways, rather than try to work around or conceal it.
For example, in geometry problems I typically use directed angles
without further comment, rather than awkwardly work around configuration issues.
Similarly, sentences like ``let $\mathbb{R}$ denote the set of real numbers''
are typically omitted entirely.

Corrections and comments are welcome!
\end{abstract}

\tableofcontents
\newpage

\addtocounter{section}{-1}
\section{Problems}
\begin{enumerate}[\bfseries 1.]
\item %% Problem 1
For which positive integers $m$ does there exist an infinite sequence in $\ZZ/m\ZZ$
which is both an arithmetic progression and a geometric progression,
but is nonconstant?

\item %% Problem 2
Let $a$ and $b$ be positive integers.
Every cell of an $(a+b+1)\times (a+b+1)$ grid is colored either amber or bronze
such that there are at least $a^2+ab-b$ amber cells
and at least $b^2+ab-a$ bronze cells.
Prove that it is possible to choose $a$ amber cells and $b$ bronze cells
such that no two of the $a+b$ chosen cells lie in the same row or column.

\item %% Problem 3
Let $b\geq 2$ and $w\geq 2$ be fixed integers, and $n=b+w$.
Given are $2b$ identical black rods and $2w$ identical white rods,
each of side length $1$.

We assemble a regular $2n$-gon using these rods
so that parallel sides are the same color.
Then, a convex $2b$-gon $B$ is formed by translating the black rods,
and a convex $2w$-gon $W$ is formed by translating the white rods.
An example of one way of doing the assembly when $b=3$ and $w=2$ is shown below,
as well as the resulting polygons $B$ and $W$.
\begin{center}
\begin{asy}
size(10cm);
real w = 2*Sin(18);
real h = 0.10 * w;
real d = 0.33 * h;
picture wht;
picture blk;

draw(wht, (0,0)--(w,0)--(w+d,h)--(-d,h)--cycle);
fill(blk, (0,0)--(w,0)--(w+d,h)--(-d,h)--cycle, black);

// draw(unitcircle, blue+dotted);

// Original polygon
add(shift(dir(108))*blk);
add(shift(dir(72))*rotate(324)*blk);
add(shift(dir(36))*rotate(288)*wht);
add(shift(dir(0))*rotate(252)*blk);
add(shift(dir(324))*rotate(216)*wht);

add(shift(dir(288))*rotate(180)*blk);
add(shift(dir(252))*rotate(144)*blk);
add(shift(dir(216))*rotate(108)*wht);
add(shift(dir(180))*rotate(72)*blk);
add(shift(dir(144))*rotate(36)*wht);

// White shifted
real Wk = 1.2;
pair W1 = (1.8,0.1);
pair W2 = W1 + w*dir(36);
pair W3 = W2 + w*dir(108);
pair W4 = W3 + w*dir(216);
path Wgon = W1--W2--W3--W4--cycle;
draw(Wgon);
pair WO = (W1+W3)/2;
transform Wt = shift(WO)*scale(Wk)*shift(-WO);
draw(Wt * Wgon);
label("$W$", WO);
/*
draw(W1--Wt*W1);
draw(W2--Wt*W2);
draw(W3--Wt*W3);
draw(W4--Wt*W4);
*/

// Black shifted
real Bk = 1.10;
pair B1 = (1.5,-0.1);
pair B2 = B1 + w*dir(0);
pair B3 = B2 + w*dir(324);
pair B4 = B3 + w*dir(252);
pair B5 = B4 + w*dir(180);
pair B6 = B5 + w*dir(144);
path Bgon = B1--B2--B3--B4--B5--B6--cycle;
pair BO = (B1+B4)/2;
transform Bt = shift(BO)*scale(Bk)*shift(-BO);
fill(Bt * Bgon, black);
fill(Bgon, white);
label("$B$", BO);
\end{asy}
\end{center}
Prove that the difference of the areas of $B$ and $W$
depends only on the numbers $b$ and $w$,
and not on how the $2n$-gon was assembled.

\item %% Problem 4
Let $ABCD$ be a rhombus, and let $K$ and $L$ be points
such that $K$ lies inside the rhombus, $L$ lies outside the rhombus,
and $KA = KB = LC = LD$.
Prove that there exist points $X$ and $Y$ on lines $AC$ and $BD$
such that $KXLY$ is also a rhombus.

\item %% Problem 5
Find all pairs of primes $(p, q)$ for which $p-q$ and $pq-q$ are both perfect squares.

\item %% Problem 6
Let $a_0$, $b_0$, $c_0$ be complex numbers, and define
\begin{align*}
  a_{n+1} &= a_n^2 + 2b_nc_n \\
  b_{n+1} &= b_n^2 + 2c_na_n \\
  c_{n+1} &= c_n^2 + 2a_nb_n
\end{align*}
for all nonnegative integers $n$.
Suppose that $\max{\{|a_n|, |b_n|, |c_n|\}} \leq 2022$ for all $n \ge 0$.
Prove that \[ |a_0|^2 + |b_0|^2 + |c_0|^2 \leq 1. \]

\end{enumerate}
\pagebreak

\section{Solutions to Day 1}
\subsection{JMO 2022/1, proposed by Holden Mui}
\textsl{Available online at \url{https://aops.com/community/p24774800}.}
\begin{mdframed}[style=mdpurplebox,frametitle={Problem statement}]
For which positive integers $m$ does there exist an infinite sequence in $\ZZ/m\ZZ$
which is both an arithmetic progression and a geometric progression,
but is nonconstant?
\end{mdframed}
Answer: $m$ must \emph{not} be squarefree.

The problem is essentially asking when there exists
a nonconstant arithmetic progression in $\ZZ/m\ZZ$
which is also a geometric progression.
Now,
\begin{itemize}
  \ii If $m$ is squarefree, then consider three $(s-d, d, s+d)$ in arithmetic progression.
  It's geometric if and only if $d^2 = (s-d)(s+d) \pmod m$, meaning $d^2 \equiv 0 \pmod m$.
  Then $d \equiv 0 \pmod m$.
  So any arithmetic progression which is also geometric is constant in this case.

  \ii Conversely if $p^2 \mid m$ for some prime $p$,
  then any arithmetic progression with common difference $m/p$ is geometric
  by the same calculation.
\end{itemize}
\pagebreak

\subsection{JMO 2022/2, proposed by Ankan Bhattacharya}
\textsl{Available online at \url{https://aops.com/community/p24774812}.}
\begin{mdframed}[style=mdpurplebox,frametitle={Problem statement}]
Let $a$ and $b$ be positive integers.
Every cell of an $(a+b+1)\times (a+b+1)$ grid is colored either amber or bronze
such that there are at least $a^2+ab-b$ amber cells
and at least $b^2+ab-a$ bronze cells.
Prove that it is possible to choose $a$ amber cells and $b$ bronze cells
such that no two of the $a+b$ chosen cells lie in the same row or column.
\end{mdframed}
\begin{claim*}
  There exists a transversal $T_a$ with at least $a$ amber cells.
  Analogously, there exists a transversal $T_b$ with at least $b$ bronze cells.
\end{claim*}
\begin{proof}
  If one picks a random transversal, the expected value
  of the number of amber cells is at least
  \[ \frac{a^2+ab-b^2}{a+b+1} = (a-1) + \frac{1}{a+b+1} > a-1. \qedhere \]
\end{proof}

Now imagine we transform $T_a$ to $T_b$ in some number of steps,
by repeatedly choosing cells $c$ and $c'$ and swapping them
with the two other corners of the rectangle formed by their row/column,
as shown in the figure.
\begin{center}
\begin{asy}
  filldraw(shift(0,0)*unitsquare, lightcyan, blue);
  filldraw(shift(3,2)*unitsquare, lightcyan, blue);
  draw(shift(3,0)*unitsquare, dotted+blue);
  draw(shift(0,2)*unitsquare, dotted+blue);
  label("$c$", (0.5,0.5));
  label("$c'$", (3.5,2.5));

  label(scale(2.5)*"$\implies$", (5.5, 1.5));
  filldraw(shift(7,2)*unitsquare, lightcyan, blue);
  filldraw(shift(10,0)*unitsquare, lightcyan, blue);
  draw(shift(7,0)*unitsquare, dotted+blue);
  draw(shift(10,2)*unitsquare, dotted+blue);
\end{asy}
\end{center}
By ``discrete intermediate value theorem'',
the number of amber cells will be either $a$ or $a+1$
at some point during this transformation.
This completes the proof.
\pagebreak

\subsection{JMO 2022/3, proposed by Ankan Bhattacharya}
\textsl{Available online at \url{https://aops.com/community/p24775345}.}
\begin{mdframed}[style=mdpurplebox,frametitle={Problem statement}]
Let $b\geq 2$ and $w\geq 2$ be fixed integers, and $n=b+w$.
Given are $2b$ identical black rods and $2w$ identical white rods,
each of side length $1$.

We assemble a regular $2n$-gon using these rods
so that parallel sides are the same color.
Then, a convex $2b$-gon $B$ is formed by translating the black rods,
and a convex $2w$-gon $W$ is formed by translating the white rods.
An example of one way of doing the assembly when $b=3$ and $w=2$ is shown below,
as well as the resulting polygons $B$ and $W$.
\begin{center}
\begin{asy}
size(10cm);
real w = 2*Sin(18);
real h = 0.10 * w;
real d = 0.33 * h;
picture wht;
picture blk;

draw(wht, (0,0)--(w,0)--(w+d,h)--(-d,h)--cycle);
fill(blk, (0,0)--(w,0)--(w+d,h)--(-d,h)--cycle, black);

// draw(unitcircle, blue+dotted);

// Original polygon
add(shift(dir(108))*blk);
add(shift(dir(72))*rotate(324)*blk);
add(shift(dir(36))*rotate(288)*wht);
add(shift(dir(0))*rotate(252)*blk);
add(shift(dir(324))*rotate(216)*wht);

add(shift(dir(288))*rotate(180)*blk);
add(shift(dir(252))*rotate(144)*blk);
add(shift(dir(216))*rotate(108)*wht);
add(shift(dir(180))*rotate(72)*blk);
add(shift(dir(144))*rotate(36)*wht);

// White shifted
real Wk = 1.2;
pair W1 = (1.8,0.1);
pair W2 = W1 + w*dir(36);
pair W3 = W2 + w*dir(108);
pair W4 = W3 + w*dir(216);
path Wgon = W1--W2--W3--W4--cycle;
draw(Wgon);
pair WO = (W1+W3)/2;
transform Wt = shift(WO)*scale(Wk)*shift(-WO);
draw(Wt * Wgon);
label("$W$", WO);
/*
draw(W1--Wt*W1);
draw(W2--Wt*W2);
draw(W3--Wt*W3);
draw(W4--Wt*W4);
*/

// Black shifted
real Bk = 1.10;
pair B1 = (1.5,-0.1);
pair B2 = B1 + w*dir(0);
pair B3 = B2 + w*dir(324);
pair B4 = B3 + w*dir(252);
pair B5 = B4 + w*dir(180);
pair B6 = B5 + w*dir(144);
path Bgon = B1--B2--B3--B4--B5--B6--cycle;
pair BO = (B1+B4)/2;
transform Bt = shift(BO)*scale(Bk)*shift(-BO);
fill(Bt * Bgon, black);
fill(Bgon, white);
label("$B$", BO);
\end{asy}
\end{center}
Prove that the difference of the areas of $B$ and $W$
depends only on the numbers $b$ and $w$,
and not on how the $2n$-gon was assembled.
\end{mdframed}
We are going to prove that one may swap a black rod with an adjacent white rod
(as well as the rods parallel to them)
without affecting the difference in the areas of $B-W$.
Let $\vec u$ and $\vec v$ denote the originally black and white vectors
that were adjacent on the $2n$-gon and are now going to be swapped.
Let $\vec x$ denote the sum of all the other black vectors between
$\vec u$ and $-\vec u$, and define $\vec y$ similarly.
See the diagram below,
where $B_0$ and $W_0$ are the polygons before the swap,
and $B_1$ and $W_1$ are the resulting changed polygons.

\begin{center}
\begin{asy}
size(12cm);
picture B0;
picture W0;
picture B1;
picture W1;

pen br = black + 2;
pen bp = black + dashed;
path barc = (0,0)..(1,1)..(6,1)..(7,0);

draw(B0, (3,-1)--(4,1), br);
draw(B0, (-4,-1)--(-3,1), br);
draw(B0, shift(-3,1)*barc, bp);
draw(B0, shift(3,-1)*rotate(180)*barc, bp);
label(B0, "$B_0$", (0,0));

draw(B0, (-3,1)--(4,1), red, EndArrow(TeXHead), Margins);
draw(B0, (3,-1)--(-4,-1), red, EndArrow(TeXHead), Margins);
label(B0, "$\vec x$", (-3,1)--(4,1), dir(90), red);
label(B0, "$-\vec x$", (3,-1)--(-4,-1), dir(-90), red);
label(B0, "$\vec u$", (-3.5,0), dir(160));
draw(B0, (-4,-1)--(-3,1), black+1, EndArrow(TeXHead));

draw(B1, (4,-1)--(3,1), br);
draw(B1, (-3,-1)--(-4,1), br);
label(B1, "$B_1$", (0,0));
draw(B1, shift(-4,1)*barc, bp);
draw(B1, shift(4,-1)*rotate(180)*barc, bp);
draw(B1, (-4,1)--(3,1), red, EndArrow(TeXHead), Margins);
draw(B1, (4,-1)--(-3,-1), red, EndArrow(TeXHead), Margins);
label(B1, "$\vec x$", (-4,1)--(3,1), dir(90), red);
label(B1, "$-\vec x$", (4,-1)--(-3,-1), dir(180), red);
label(B1, "$\vec v$", (-3.5,0), dir(210));
draw(B1, (-3,-1)--(-4,1), black+1, EndArrow(TeXHead));

path warc = (0,0)..(4,1)..(9,-2);
draw(W0, shift(-5,1)*warc, bp);
draw(W0, shift(5,-3)*rotate(180)*warc, bp);
draw(W0, (5,-3)--(4,-1), br);
draw(W0, (-4,-1)--(-5,1), br);
draw(W0, (-5,1)--(4,-1), red, EndArrow(TeXHead), Margins);
draw(W0, (5,-3)--(-4,-1), red, EndArrow(TeXHead), Margins);
draw(W0, (5,-3)--(4,-1), white+1.2);
draw(W0, (-4,-1)--(-5,1), white+1.2);
label(W0, "$W_0$", (0,-1));
label(W0, "$\vec v$", (-4.5,0), dir(200));
label(W0, "$\vec y$", (-5,1)--(4,-1), dir(70), red);
label(W0, "$-\vec y$", (5,-3)--(-4,-1), dir(250), red);

draw(W1, shift(-4,1)*warc, bp);
draw(W1, shift(4,-3)*rotate(180)*warc, bp);
draw(W1, (4,-3)--(5,-1), br);
draw(W1, (-5,-1)--(-4,1), br);
draw(W1, (4,-3)--(5,-1), white+1.2);
draw(W1, (-5,-1)--(-4,1), white+1.2);
draw(W1, (-4,1)--(5,-1), red, EndArrow(TeXHead), Margins);
draw(W1, (4,-3)--(-5,-1), red, EndArrow(TeXHead), Margins);
label(W1, "$W_1$", (0,-1));
label(W1, "$\vec u$", (-4.5,0), dir(140));
label(W1, "$\vec y$", (0.5,0), dir(70), red);
label(W1, "$-\vec y$", (-0.5,-2), dir(250), red);

add(B0);
add(shift(12,0)*B1);
add(shift(0,-6)*W0);
add(shift(12,-6)*W1);
\end{asy}
\end{center}

Observe that the only change in $B$ and $W$
is in the parallelograms shown above in each diagram.
Letting $\wedge$ denote the wedge product,
we need to show that
\[ \vec u \wedge \vec x - \vec v \wedge \vec y
  = \vec v \wedge \vec x - \vec u \wedge \vec y \]
which can be rewritten as
\[ (\vec u - \vec v) \wedge (\vec x + \vec y) = 0. \]
In other words, it would suffice to show
$\vec u - \vec v$ and $\vec x + \vec y$ are parallel.
(Students not familiar with wedge products can replace
every $\wedge$ with the cross product $\times$ instead.)
\begin{claim*}
  Both $\vec u - \vec v$ and $\vec x + \vec y$
  are perpendicular to vector $\vec u + \vec v$.
\end{claim*}
\begin{proof}
  We have $(\vec u - \vec v) \perp (\vec u + \vec v)$
  because $\vec u$ and $\vec v$ are the same length.

  For the other perpendicularity, note that
  $\vec u + \vec v + \vec x + \vec y$
  traces out a diameter of the circumcircle of the original $2n$-gon;
  call this diameter $AB$, so
  \[ A + \vec u + \vec v + \vec x + \vec y = B. \]
  Now point $A + \vec u + \vec v$ is a point on this semicircle,
  which means (by the inscribed angle theorem)
  the angle between $\vec u + \vec v$ and $\vec x + \vec y$ is $90^\circ$.
\end{proof}
\pagebreak

\section{Solutions to Day 2}
\subsection{JMO 2022/4, proposed by Ankan Bhattacharya}
\textsl{Available online at \url{https://aops.com/community/p24774800}.}
\begin{mdframed}[style=mdpurplebox,frametitle={Problem statement}]
Let $ABCD$ be a rhombus, and let $K$ and $L$ be points
such that $K$ lies inside the rhombus, $L$ lies outside the rhombus,
and $KA = KB = LC = LD$.
Prove that there exist points $X$ and $Y$ on lines $AC$ and $BD$
such that $KXLY$ is also a rhombus.
\end{mdframed}
To start, notice that $\triangle AKB \cong \triangle DLC$ by SSS.
Then by the condition $K$ lies inside the rhombus while $L$ lies outside it,
we find that the two congruent triangles are just translations of each other
(i.e.\ they have the same orientation).

\paragraph{First solution.}
Let $M$ be the midpoint of $\ol{KL}$ and is $O$ the center of the rhombus.
\begin{claim*}
  $\ol{MO} \perp \ol{AB}$.
\end{claim*}
\begin{proof}
  Let $U$ and $V$ denote the midpoint of $\ol{AB}$ and $\ol{CD}$ respectively.
  Then $\ol{KU}$ and $\ol{LV}$ are obviously translates,
  and perpendicular to $\ol{AB} \parallel \ol{CD}$.
  Since $M$ is the midpoint of $\ol{KL}$ and $O$ is the midpoint of $\ol{UV}$,
  the result follows.
\end{proof}

We choose $X$ and $Y$ to be the intersections of
the perpendicular bisector of $\ol{KL}$ with $\ol{AC}$ and $\ol{BD}$.

\begin{center}
\begin{asy}
  pair A = (-4,0);
  pair B = (0,-3);
  pair C = -A;
  pair D = -B;
  pair K = midpoint(A--B) + dir(90)*(B-A)*0.2;
  pair L = D+K-A;
  pair X = -A+2*foot(K,A,C);
  pair Y = -D+2*foot(L,B,D);
  pair O = origin;
  pair M = midpoint(K--L);
  pair U = midpoint(A--B);
  pair V = midpoint(C--D);

  filldraw(A--B--C--D--cycle, opacity(0.1)+lightred, red);
  draw(A--C, red);
  draw(B--D, red);
  filldraw(A--K--B--cycle, opacity(0.1)+cyan, blue);
  filldraw(C--L--D--cycle, opacity(0.1)+cyan, blue);
  draw(X--Y, deepgreen);
  draw(L--K, deepgreen);
  draw(K--U, black+1.5);
  draw(L--V, black+1.5);
  draw(M--O, black+1.5);

  dot("$A$", A, dir(210));
  dot("$B$", B, dir(340));
  dot("$C$", C, dir(-30));
  dot("$D$", D, dir(160));
  dot("$K$", K, dir(100));
  dot("$L$", L, dir(90));
  dot("$X$", X, dir(305));
  dot("$Y$", Y, dir(200));
  dot("$M$", M, dir(-5));
  dot("$O$", O, dir(225));
  dot("$U$", U, dir(250));
  dot("$V$", V, dir(250));
\end{asy}
\end{center}

\begin{claim*}
  The midpoint of $\ol{XY}$ coincides with the midpoint of $\ol{KL}$.
\end{claim*}
\begin{proof}
  Because
  \begin{align*}
    \ol{XY} &\perp \ol{KL} \parallel \ol{BC} \\
    \ol{MO} &\perp \ol{AB} \\
    \ol{BD} &\perp \ol{AC}
  \end{align*}
  it follows that $\triangle MOY$,
  which was determined by the three lines $\ol{XY}$, $\ol{MO}$, $\ol{BD}$,
  is similar to $\triangle ABC$.
  In particular, it is isosceles with $MY = MO$.
  Analogously, $MX = MO$.
\end{proof}

\begin{remark*}
  It is also possible to simply use coordinates to prove both claims.
\end{remark*}

\paragraph{Second solution (author's solution).}
In this solution, we instead define
$X$ and $Y$ as the intersections of the circles
centered at $K$ and $L$ of equal radii $KA$,
which will be denoted $\omega_K$ and $\omega_L$.
It is clear that $KXLY$ is a rhombus under this construction,
so it suffices to show that $X$ and $Y$ lie on $AC$ and $BD$ (in some order).

\begin{center}
\begin{asy}
  pair A = (-4,0);
  pair B = (0,-3);
  pair C = -A;
  pair D = -B;
  pair K = midpoint(A--B) + dir(90)*(B-A)*0.2;
  pair L = D+K-A;
  pair X = -A+2*foot(K,A,C);
  pair Y = -D+2*foot(L,B,D);

  filldraw(A--B--C--D--cycle, opacity(0.1)+lightred, red);
  draw(A--C, red);
  draw(B--D, red);
  filldraw(A--K--B--cycle, opacity(0.1)+cyan, blue);
  filldraw(C--L--D--cycle, opacity(0.1)+cyan, blue);
  draw(X--L--Y--K--cycle, deepgreen);
  draw(circle(K,abs(K-A)), grey);
  draw(circle(L,abs(L-C)), grey);

  dot("$A$", A, dir(210));
  dot("$B$", B, dir(340));
  dot("$C$", C, dir(-30));
  dot("$D$", D, dir(160));
  dot("$K$", K, dir(250));
  dot("$L$", L, dir(90));
  dot("$X$", X, dir(305));
  dot("$Y$", Y, dir(200));
  label("$\omega_L$", L + abs(L-C)*dir(40), dir(40));
  label("$\omega_K$", K + abs(K-B)*dir(220), dir(220));
\end{asy}
\end{center}

To see this, let $\ol{AC}$ meet $\omega_K$ again at $X'$.
We have
\[ \dang CX'D = \dang BX'C = \dang BX'A = \half \opname{m} \arc{AB} = \opname{m} \arc{CD} \]
where the arcs are directed modulo $360\dg$;
here $\arc{AB}$ is the arc of $\omega_K$ cut out by $\dang AXB$,
and $\arc{DC}$ is the analogous arc of $\omega_L$.
This implies $X'$ lies on $\omega_L$ by the inscribed angle theorem.
Hence $X = X'$, and it follows $X$ lies on $\ol{AC}$.

Analogously $Y$ lies on $BD$.
\begin{remark*}
  The angle calculation above can also be replaced with a length calculation, as follows.

  Let $M$ and $N$ be the projections of $K$ and $L$ onto $\ol{AC}$, respectively.
  Then $X'$ is the reflection of $A$ across $M$;
  analogously, the second intersection $X''$ with $\ol{AC}$
  should be the reflection of $C$ across $N$.
  So to get $X = X' = X''$, we would need to show $AC = 2MN$.

  However, note that $AKLD$ is a parallelogram.
  As $MN$ was the projection of $\ol{KL}$ onto $\ol{AC}$,
  its length should be the same as the projection of $\ol{AD}$ onto $\ol{AC}$,
  which is obviously $\half AC$ because the projection of $D$ onto $\ol{AC}$ is
  exactly the midpoint of $\ol{AC}$ (i.e.\ the center of the rhombus).
\end{remark*}
\pagebreak

\subsection{JMO 2022/5, proposed by Holden Mui}
\textsl{Available online at \url{https://aops.com/community/p24774670}.}
\begin{mdframed}[style=mdpurplebox,frametitle={Problem statement}]
Find all pairs of primes $(p, q)$ for which $p-q$ and $pq-q$ are both perfect squares.
\end{mdframed}
The answer is $(3,2)$ only.
This obviously works so we focus on showing it is the only one.

\paragraph{Approach using difference of squares (from author).}
Set
\begin{align*}
  a^2 &= p-q \\
  b^2 &= pq-q.
\end{align*}
Note that $0 < a < p$, and $0 < b < p$ (because $q \le p$).
Now subtracting gives
\[ \underbrace{(b-a)}_{<p} \underbrace{(b+a)}_{<2p} = b^2-a^2 = p(q-1) \]
The inequalities above now force $b+a = p$.
Hence $q-1 = b-a$.

This means $p$ and $q-1$ have the same parity,
which can only occur if $q = 2$.
Finally, taking mod $3$ shows $p \equiv 0 \pmod 3$.
So $(3,2)$ is the only possibility (and it does work).

\paragraph{Divisibility approach (Aharshi Roy).}
Since $pq-q = q(p-1)$ is a square, it follows that $q$ divides $p-1$
and that $\frac{p-1}{q}$ is a perfect square too.
Hence the number
\[ s^2 \coloneqq (p-q) \cdot \frac{p-1}{q} = \frac{p^2-qp-p+q}{q} \]
is also a perfect square.
Rewriting this equation gives
\[ q = \frac{p^2-p}{s^2 + (p-1)}. \]
In particular, $s^2 + (p-1)$ divides $p^2-p$, and in particular $s \leq p$.
We consider two cases:
\begin{itemize}
  \ii If $s^2+(p-1)$ is not divisible by $p$, then it must divide $p-1$,
  which can only happen if $s^2 = 0$, or $p = q$.
  However, it's easy to check there are no solutions in this case.

  \ii Otherwise, we should have $s^2 \equiv 1 \pmod p$,
  so either $s = 1$ or $s = p-1$.
  If $s = p-1$ we get $q=1$ which is absurd.
  On the other hand, if $s=1$ we conclude $q = p-1$ and hence $q=2$, $p=3$.
\end{itemize}
\pagebreak

\subsection{JMO 2022/6, proposed by Ankan Bhattacharya}
\textsl{Available online at \url{https://aops.com/community/p24775314}.}
\begin{mdframed}[style=mdpurplebox,frametitle={Problem statement}]
Let $a_0$, $b_0$, $c_0$ be complex numbers, and define
\begin{align*}
  a_{n+1} &= a_n^2 + 2b_nc_n \\
  b_{n+1} &= b_n^2 + 2c_na_n \\
  c_{n+1} &= c_n^2 + 2a_nb_n
\end{align*}
for all nonnegative integers $n$.
Suppose that $\max{\{|a_n|, |b_n|, |c_n|\}} \leq 2022$ for all $n \ge 0$.
Prove that \[ |a_0|^2 + |b_0|^2 + |c_0|^2 \leq 1. \]
\end{mdframed}
For brevity, set $s_n \coloneqq |a_n|^2 + |b_n|^2 + |c_n|^2$.
Note that the $s_n$ are real numbers.
\begin{claim*}
  [Key miraculous identity]
  We have
  \[ s_{n+1} - s_n^2 = 2 |a_n \ol{b_n} + b_n \ol{c_n} + c_n \ol{a_n}|^2. \]
\end{claim*}
\begin{proof}
  We prove this by mechanical calculation. First,
  \begin{align*}
    s_{n+1} &= \left\lvert a_n^2+2b_nc_n \right\rvert^2 +
    \left\lvert b_n^2+2c_na_n \right\rvert^2 +
    \left\lvert c_n^2+2a_nb_n \right\rvert^2 \\
    &= \sum_{\text{cyc}} \left\lvert a_n^2+2b_nc_n \right\rvert^2 \\
    &= \sum_{\text{cyc}} (a_n^2+2b_nc_n)(\ol{a_n}^2 + 2\ol{b_n}\ol{c_n}) \\
    &= \sum_{\text{cyc}} \left( |a_n|^4 + 2 \ol{a_n}^2 b_nc_n
      + 2a_n^2 \ol{b_n} \ol{c_n}  + 4 |b_n|^2 |c_n|^2 \right) \\
      &= s_n^2 + 2 \sum_{\text{cyc}}
      \left( \ol{a_n}^2 b_nc_n
      + a_n^2 \ol{b_n} \ol{c_n} + |b_n|^2 |c_n|^2 \right).
  \end{align*}
  Meanwhile,
	\begin{align*}
		|a_n \ol{b_n} + b_n \ol{c_n} + c_n \ol{a_n}|^2
		&= \left( a_n \ol{b_n} + b_n \ol{c_n} + c_n \ol{a_n} \right)
		\left( \ol{a_n} b_n + \ol{b_n} c_n + \ol{c_n} a_n \right) \\
		&= |a_n|^2 |b_n^2| + a_n \ol{b_n}^2 c_n + a_n^2 \ol{b_n} \ol{c_n} \\
		&\qquad + \ol{a_n} b_n^2 \ol{c_n} + |b_n|^2 |c_n|^2 + a_n b_n \ol{c_n}^2 \\
		&\qquad + \ol{a_n}^2 b_n c_n + \ol{a_n} \ol{b_n} c_n^2 + |a_n|^2 |c_n|^2
	\end{align*}
	which exactly matches the earlier sum, term for term.
\end{proof}

In particular, $s_{n+1} \ge s_n^2$, so applying repeatedly,
\[ s_n \ge s_0^{2^n}. \]
Hence if $s_0 > 1$, it follows $s_n$ is unbounded, contradicting
$\max{\{|a_n|, |b_n|, |c_n|\}} \leq 2022$.

\begin{remark*}
The originally intended solution was to capture all three
recursions in the following way.
First, change the recursion to
\begin{align*}
  a_{n+1} &= a_n^2 + 2b_nc_n \\
  c_{n+1} &= b_n^2 + 2c_na_n \\
  b_{n+1} &= c_n^2 + 2a_nb_n
\end{align*}
which is OK because we are just rearranging the terms in each triple.
Then if $\omega$ is any complex number with $\omega^3 = 1$, and we define
\[ z_n \coloneqq a_n + b_n \omega + c_n \omega^2, \]
the recursion amounts to saying that $z_{n+1} = z_n^2$.
This allows us to analyze $|z_n|$ in a similar way as above,
as now $|z_n| = |z_0|^{2^n}$.
\end{remark*}
\pagebreak


\end{document}
