% © Evan Chen
% Downloaded from https://web.evanchen.cc/

\documentclass[11pt]{scrartcl}
\usepackage[sexy]{evan}
\ihead{\footnotesize\textbf{\thetitle}}
\ohead{\footnotesize\href{http://web.evanchen.cc}{\ttfamily web.evanchen.cc},
    updated \today}
\title{USAMO 2009 Solution Notes}
\date{\today}

\begin{document}

\maketitle

\begin{abstract}
This is a compilation of solutions
for the 2009 USAMO.
Some of the solutions are my own work,
but many are from the official solutions provided by the organizers
(for which they hold any copyrights),
and others were found by users on the Art of Problem Solving forums.

These notes will tend to be a bit more advanced and terse than the ``official''
solutions from the organizers.
In particular, if a theorem or technique is not known to beginners
but is still considered ``standard'', then I often prefer to
use this theory anyways, rather than try to work around or conceal it.
For example, in geometry problems I typically use directed angles
without further comment, rather than awkwardly work around configuration issues.
Similarly, sentences like ``let $\mathbb{R}$ denote the set of real numbers''
are typically omitted entirely.

Corrections and comments are welcome!
\end{abstract}

\tableofcontents
\newpage

\addtocounter{section}{-1}
\section{Problems}
\begin{enumerate}[\bfseries 1.]
\item %% Problem 1
Given circles $\omega_1$ and $\omega_2$ intersecting at points $X$ and $Y$,
let $\ell_1$ be a line through the center of $\omega_1$
intersecting $\omega_2$ at points $P$ and $Q$
and let $\ell_2$ be a line through the center of $\omega_2$
intersecting $\omega_1$ at points $R$ and $S$.
Prove that if $P$, $Q$, $R$, and $S$ lie on a circle
then the center of this circle lies on line $XY$.

\item %% Problem 2
Let $n$ be a positive integer.
Determine the size of the largest subset of $\{ -n, -n+1, \dots, n-1, n\}$
which does not contain three elements $a$, $b$, $c$ (not necessarily distinct)
satisfying $a+b+c=0$.

\item %% Problem 3
We define a \emph{chessboard polygon} to be a simple polygon whose sides are
situated along lines of the form $x=a$ or $y=b$, where $a$ and $b$ are integers.
These lines divide the interior into unit squares, which are shaded alternately
grey and white so that adjacent squares have different colors.
To tile a chessboard polygon by dominoes is to
exactly cover the polygon by non-overlapping $ 1 \times 2$ rectangles.
Finally, a \emph{tasteful tiling} is one which avoids the
two configurations of dominoes and colors shown on the left below.
Two tilings of a $3 \times 4$ rectangle are shown;
the first one is tasteful, while the second is not,
due to the vertical dominoes in the upper right corner.

\begin{center}
\begin{asy}
size(300);
pen rt = linewidth(2.5) + rgb(0.6,0.2,0.2);
void chessboard(int a, int b, int eps, pair P) {
  for(int i = 0; i < a; ++i) {
  for(int j = 0; j < b; ++j) {
    if((i+j+eps)#2 == (i+j+eps)/2)
    fill(shift(P.x+i,P.y+j)*unitsquare,rgb(0.6,0.6,0.6));
  }
  }
  draw(P--P+(a,0)--P+(a,b)--P+(0,b)--cycle, rt);
}

chessboard(2,2,0,(0,0));
chessboard(2,2,1,(2.5,0));
chessboard(4,3,1,(6,0));
chessboard(4,3,1,(11,0));
label("Distasteful tilings",(2.25,2.5),fontsize(10));

/* draw lines */
draw((1,0)--(1,2), rt);
draw((2.5,1)--(4.5,1), rt);
draw((7,0)--(7,2)--(6,2)--(10,2)--(9,2)--(9,0)--(9,1)--(7,1), rt);
draw((8,2)--(8,3), rt);
draw((12,0)--(12,2)--(11,2)--(13,2), rt);
draw((13,1)--(15,1)--(14,1)--(14,3), rt);
draw((13,0)--(13,3), rt);
\end{asy}
\end{center}

Prove that (a) if a chessboard polygon can be tiled by dominoes,
then it can be done so tastefully, and (b) such a tasteful tiling is unique.

\item %% Problem 4
For $n \ge 2$, let $a_1$, $a_2$, \dots, $a_n$ be positive real numbers such that
\[
  \left( a_1 + a_2 + \dots + a_n \right)
  \left( \frac{1}{a_1} + \frac{1}{a_2} + \dots + \frac{1}{a_n} \right)
  \le \left( n + \half \right)^2.
\]
Prove that $\max\left( a_1, \dots, a_n \right) \le 4 \min\left( a_1, \dots, a_n \right)$.

\item %% Problem 5
Trapezoid $ABCD$, with $\ol{AB} \parallel \ol{CD}$,
is inscribed in circle $\omega$ and point $G$ lies inside triangle $BCD$.
Rays $AG$ and $BG$ meet $\omega$ again at points $P$ and $Q$, respectively.
Let the line through $G$ parallel to $\ol{AB}$ intersect
$\ol{BD}$ and $\ol{BC}$ at points $R$ and $S$, respectively.
Prove that quadrilateral $PQRS$ is cyclic
if and only if $\ol{BG}$ bisects $\angle CBD$.

\item %% Problem 6
Let $s_1, s_2, s_3, \dots$ be an infinite,
nonconstant sequence of rational numbers,
meaning it is not the case that $s_1 = s_2 = s_3 = \dots$.
Suppose that $t_1, t_2, t_3, \dots$ is also an infinite,
nonconstant sequence of rational numbers
with the property that $(s_i - s_j)(t_i - t_j)$
is an integer for all $i$ and $j$.
Prove that there exists a rational number $r$
such that $(s_i - s_j)r$ and $(t_i - t_j)/r$
are integers for all $i$ and $j$.

\end{enumerate}
\pagebreak

\section{Solutions to Day 1}
\subsection{USAMO 2009/1, proposed by Ian Le}
\textsl{Available online at \url{https://aops.com/community/p1485133}.}
\begin{mdframed}[style=mdpurplebox,frametitle={Problem statement}]
Given circles $\omega_1$ and $\omega_2$ intersecting at points $X$ and $Y$,
let $\ell_1$ be a line through the center of $\omega_1$
intersecting $\omega_2$ at points $P$ and $Q$
and let $\ell_2$ be a line through the center of $\omega_2$
intersecting $\omega_1$ at points $R$ and $S$.
Prove that if $P$, $Q$, $R$, and $S$ lie on a circle
then the center of this circle lies on line $XY$.
\end{mdframed}
Let $r_1$, $r_2$, $r_3$ denote the
circumradii of $\omega_1$, $\omega_2$, and $\omega_3$, respectively.

\begin{center}
\begin{asy}
pair O_1 = dir(220);
pair O_2 = dir(320);
pair O = dir(110);
pair T = foot(O, O_1, O_2);
pair X = midpoint(O--T);
pair Y = 2*T-X;

filldraw(CP(O_1, X), opacity(0.1)+lightcyan, lightblue);
filldraw(CP(O_2, X), opacity(0.1)+lightcyan, lightblue);

draw(O--Y, red+dashed);

pair K = foot(O_1, O, O_2);
pair P = IP(O_1--K, CP(O_2, X));
pair Q = 2*K-P;
draw(O_1--Q, red);
draw(O--O_2, heavycyan);

pair L = foot(O_2, O, O_1);
pair R = IP(O_2--L, CP(O_1, X));
pair S = 2*L-R;
draw(O_2--S, red);
draw(O--O_1, heavycyan);

draw(arc(O, abs(P-O), 180, 360), heavygreen);

dot("$O_1$", O_1, dir(O_1));
dot("$O_2$", O_2, dir(O_2));
dot("$O$", O, dir(O));
dot("$X$", X, dir(100));
dot("$Y$", Y, dir(Y));
dot("$P$", P, dir(120));
dot("$Q$", Q, dir(Q));
dot("$R$", R, dir(40));
dot("$S$", S, dir(S));

/* TSQ Source:

O_1 = dir 220
O_2 = dir 320
O = dir 110
T := foot O O_1 O_2
X = midpoint O--T R100
Y = 2*T-X

CP O_1 X 0.1 lightcyan / lightblue
CP O_2 X 0.1 lightcyan / lightblue

O--Y red dashed

K := foot O_1 O O_2
P = IP O_1--K CP O_2 X R120
Q = 2*K-P
O_1--Q red
O--O_2 heavycyan

L := foot O_2 O O_1
R = IP O_2--L CP O_1 X R40
S = 2*L-R
O_2--S red
O--O_1 heavycyan

!draw(arc(O, abs(P-O), 180, 360), heavygreen);

*/
\end{asy}
\end{center}

We wish to show that $O_3$ lies on the radical axis of $\omega_1$ and $\omega_2$.
Let us encode the conditions using power of a point.
Because $O_1$ is on the radical axis of $\omega_2$ and $\omega_3$,
\begin{align*}
  \opname{Pow}_{\omega_2}(O_1) &= \opname{Pow}_{\omega_3}(O_1) \\
  \implies O_1O_2^2 - r_2^2 &= O_1O_3^2 - r_3^2.
  \intertext{Similarly, because $O_2$ is on the radical axis of $\omega_1$ and $\omega_3$, we have}
  \opname{Pow}_{\omega_1}(O_2) &= \opname{Pow}_{\omega_3}(O_2) \\
  \implies O_1O_2^2 - r_1^2 &= O_2O_3^2 - r_3^2. \\
  \intertext{Subtracting the two gives}
  (O_1O_2^2-r_2^2)-(O_1O_2^2-r_1^2) &= (O_1O_3^2-r_3^2)-(O_2O_3^2-r_3^2) \\
  \implies r_1^2-r_2^2 &= O_1O_3^2-O_2O_3^2 \\
  \implies O_2O_3^2 - r_2^2 &= O_1O_3^2 - r_1^2 \\
  \implies \opname{Pow}_{\omega_2}(O_3) &= \opname{Pow}_{\omega_1} (O_3)
\end{align*}
as desired.
\pagebreak

\subsection{USAMO 2009/2, proposed by Kiran Kedlaya, Tewordos Amdeberhan}
\textsl{Available online at \url{https://aops.com/community/p1485139}.}
\begin{mdframed}[style=mdpurplebox,frametitle={Problem statement}]
Let $n$ be a positive integer.
Determine the size of the largest subset of $\{ -n, -n+1, \dots, n-1, n\}$
which does not contain three elements $a$, $b$, $c$ (not necessarily distinct)
satisfying $a+b+c=0$.
\end{mdframed}
The answer is $n$ with $n$ even and $n+1$ with $n$ odd;
the construction is to take all odd numbers.

To prove this is maximal, it suffices to show it for $n$ even;
we do so by induction on even $n \ge 2$ with the base case being trivial.
Letting $A$ be the subset, we consider three cases:
\begin{enumerate}[(i)]
  \ii If $|A \cap \{-n,-n+1,n-1,n\}| \le 2$,
  then by the hypothesis for $n-2$ we are done.
  \ii If both $n \in A$ and $-n \in A$,
  then there can be at most $n-2$ elements in $A \setminus \{\pm n\}$,
  one from each of the pairs $(1,n-1)$, $(2,n-2)$, $\dots$
  and their negations.
  \ii If $n, n-1, -n+1 \in A$ and $-n \notin A$,
  and at most $n-3$ more can be added,
  one from each of $(1, n-2)$, $(2, n-3)$, $\dots$
  and $(-2, -n+2)$, $(-3, -n+3)$, $\dots$.
  (In particular $-1 \notin A$.
  Analogous case for $-A$ if $n \notin A$.)
\end{enumerate}
Thus in all cases, $|A| \le n$ as needed.

\begin{remark*}
  A few examples of equality cases:
  \begin{itemize}
    \ii All odd numbers
    \ii All numbers with absolute value at least $n/2$
    \ii For $n$ even, the set $\{1, 2, \dots, n\}$
  \end{itemize}
  This list isn't exhaustive
  e.g.\ for $n=4$, the set $\{-3,2,3,4\}$ achieves the optimum.
\end{remark*}
\pagebreak

\subsection{USAMO 2009/3, proposed by Sam Vandervelde}
\textsl{Available online at \url{https://aops.com/community/p1485321}.}
\begin{mdframed}[style=mdpurplebox,frametitle={Problem statement}]
We define a \emph{chessboard polygon} to be a simple polygon whose sides are
situated along lines of the form $x=a$ or $y=b$, where $a$ and $b$ are integers.
These lines divide the interior into unit squares, which are shaded alternately
grey and white so that adjacent squares have different colors.
To tile a chessboard polygon by dominoes is to
exactly cover the polygon by non-overlapping $ 1 \times 2$ rectangles.
Finally, a \emph{tasteful tiling} is one which avoids the
two configurations of dominoes and colors shown on the left below.
Two tilings of a $3 \times 4$ rectangle are shown;
the first one is tasteful, while the second is not,
due to the vertical dominoes in the upper right corner.

\begin{center}
\begin{asy}
size(300);
pen rt = linewidth(2.5) + rgb(0.6,0.2,0.2);
void chessboard(int a, int b, int eps, pair P) {
  for(int i = 0; i < a; ++i) {
  for(int j = 0; j < b; ++j) {
    if((i+j+eps)#2 == (i+j+eps)/2)
    fill(shift(P.x+i,P.y+j)*unitsquare,rgb(0.6,0.6,0.6));
  }
  }
  draw(P--P+(a,0)--P+(a,b)--P+(0,b)--cycle, rt);
}

chessboard(2,2,0,(0,0));
chessboard(2,2,1,(2.5,0));
chessboard(4,3,1,(6,0));
chessboard(4,3,1,(11,0));
label("Distasteful tilings",(2.25,2.5),fontsize(10));

/* draw lines */
draw((1,0)--(1,2), rt);
draw((2.5,1)--(4.5,1), rt);
draw((7,0)--(7,2)--(6,2)--(10,2)--(9,2)--(9,0)--(9,1)--(7,1), rt);
draw((8,2)--(8,3), rt);
draw((12,0)--(12,2)--(11,2)--(13,2), rt);
draw((13,1)--(15,1)--(14,1)--(14,3), rt);
draw((13,0)--(13,3), rt);
\end{asy}
\end{center}

Prove that (a) if a chessboard polygon can be tiled by dominoes,
then it can be done so tastefully, and (b) such a tasteful tiling is unique.
\end{mdframed}
\paragraph{Proof of (a).} This is easier, and by induction.
Let $\mathcal P$ denote the chessboard polygon which can be tiled by dominoes.

Consider a \emph{lower-left} square $s$ of the polygon,
and WLOG is it white (other case similar).
Then we have two cases:
\begin{itemize}
  \ii If there exists a domino tiling of $\mathcal P$
  where $s$ is covered by a vertical domino,
  then delete this domino and apply induction on the rest of $\mathcal P$.
  This additional domino will not cause any distasteful tilings.

  \ii Otherwise, assume $s$ is covered by a horizontal domino
  in \emph{every} tiling.
  Again delete this domino and apply induction on the rest of $\mathcal P$.
  The resulting tasteful tiling should not have another horizontal domino
  adjacent to the one covering $s$, because otherwise
  we could have replaced that $2 \times 2$ square
  with two vertical dominoes to arrive in the first case.
  So this additional domino will not cause any distasteful tilings.
\end{itemize}

\begin{remark*}
  The second case can actually arise, for example in the following picture.
  \begin{center}
  \begin{asy}
    unitsize(0.6cm);
    fill(shift(1,0)*unitsquare,rgb(0.6,0.6,0.6));
    fill(shift(0,1)*unitsquare,rgb(0.6,0.6,0.6));
    fill(shift(2,1)*unitsquare,rgb(0.6,0.6,0.6));
    draw( (0,0)--(0,3)--(1,3)--(1,2)--(3,2)--(3,1)--(2,1)--(2,0)--cycle,
      linewidth(2.5) + rgb(0.6,0.2,0.2));
  \end{asy}
  \end{center}
  Thus one cannot just try to cover $s$ with a vertical
  domino and claim the rest of $\mathcal P$ is tile-able.
  So the induction is not as easy as one might hope.

  One can phrase the solution algorithmically too,
  in the following way: any time we see a distasteful tiling,
  we rotate it to avoid the bad pattern.
  The bottom-left corner eventually becomes stable,
  and an induction shows the termination of the algorithm.
\end{remark*}

\paragraph{Proof of (b).} We now turn to proving uniqueness.
Suppose for contradiction there are two distinct tasteful tilings.
Overlaying the two tilings on top of each other induces
several \emph{cycles} of overlapping dominoes at positions where the tilings differ.

Henceforth, it will be convenient to work with the lattice $\ZZ^2$,
treating the squares as black/white points, and we do so.
Let $\gamma$ be any such cycle and let $s$ denote a
lower left point, and again WLOG it is black.
Orient $\gamma$ counterclockwise henceforth.
Restrict attention to the lattice polygon $\mathcal Q$ enclosed by $\gamma$
(we consider points of $\gamma$ as part of $\mathcal Q$).

In one of the two tilings of (lattice points of) $\mathcal Q$,
the point $s$ will be covered by a horizontal domino;
in the other tiling $s$ will be covered by a vertical domino.
From now on we will focus only on the latter one.
Observe that we now have a set of dominoes along $\gamma$,
such that $\gamma$ points from the white point to the black point
within each domino.

Now impose coordinates so that $s = (0,0)$.
Consider the stair-case sequence of points
$p_0 = s = (0,0)$, $p_1 = (1,0)$, $p_2 = (1,1)$, $p_3 = (2,1)$, and so on.
By hypothesis, $p_0$ is covered by a vertical domino.
Then $p_1$ must be covered by a horizontal domino, to avoid a distasteful tiling.
Then if $p_2$ is in $\mathcal Q$, then it must be covered by a vertical domino
to avoid a distasteful tiling, and so on.
We may repeat this argument as long the points $p_i$ lie inside $\mathcal Q$.
(See figure below; the staircase sequence is highlighted by red halos.)
\begin{center}
\begin{asy}
  unitsize(1.5cm);

  margin m = Margin(3,3);
  arrowbar ar = EndArrow(TeXHead);
  pen gamma = blue + 1;
  draw( (0,0)--(1,0), gamma, ar, m);
  draw( (1,0)--(2,0), gamma, ar, m);
  draw( (2,0)--(3,0), gamma, ar, m);
  draw( (3,0)--(3,-1), gamma, ar, m);
  draw( (3,-1)--(4,-1), gamma, ar, m);
  draw( (4,-1)--(4,0), gamma, ar, m);
  draw( (4,0)--(4,1), gamma, ar, m);
  draw( (4,1)--(4,2), gamma, ar, m);
  draw( (4,2)--(3,2), gamma, ar, m);
  draw( (3,2)--(3,3), gamma, ar, m);
  draw( (3,3)--(4,3), gamma, ar, m);
  draw( (4,3)--(4,4), gamma, ar, m);
  draw( (4,4)--(3,4), gamma, ar, m);
  draw( (3,4)--(2,4), gamma, ar, m);
  draw( (2,4)--(1,4), gamma, ar, m);
  draw( (1,4)--(1,3), gamma, ar, m);
  draw( (1,3)--(0,3), gamma, ar, m);
  draw( (0,3)--(0,2), gamma, ar, m);
  draw( (0,2)--(0,1), gamma, ar, m);
  draw( (0,1)--(0,0), gamma, ar, m);

  fill( (0,0)--(3,0)--(3,-1)--(4,-1)--(4,2)--(3,2)
    --(3,3)--(4,3)--(4,4)--(1,4)--(1,3)--(0,3)--cycle,
    opacity(0.1) + lightcyan);
  label("$s$", (0,0), 1.5*dir(225));
  label("$a$", (3,2), 2*dir(45), red);
  label("$b$", (1,0), 2*dir(225), red);
  draw( (3,2)--(1,0), lightred);

  real r = 0.07;
  filldraw(CR( (3,-1), r), grey, black);
  filldraw(CR( (0,0), r), grey, black);
  filldraw(CR( (2,0), r), grey, black);
  filldraw(CR( (4,0), r), grey, black);
  filldraw(CR( (1,1), r), grey, black);
  filldraw(CR( (3,1), r), grey, black);
  filldraw(CR( (0,2), r), grey, black);
  filldraw(CR( (2,2), r), grey, black);
  filldraw(CR( (4,2), r), grey, black);
  filldraw(CR( (1,3), r), grey, black);
  filldraw(CR( (3,3), r), grey, black);
  filldraw(CR( (2,4), r), grey, black);
  filldraw(CR( (4,4), r), grey, black);

  filldraw(CR( (4,-1), r), white, black);
  filldraw(CR( (1,0), r), white, black);
  filldraw(CR( (3,0), r), white, black);
  filldraw(CR( (0,1), r), white, black);
  filldraw(CR( (2,1), r), white, black);
  filldraw(CR( (4,1), r), white, black);
  filldraw(CR( (1,2), r), white, black);
  filldraw(CR( (3,2), r), white, black);
  filldraw(CR( (0,3), r), white, black);
  filldraw(CR( (2,3), r), white, black);
  filldraw(CR( (4,3), r), white, black);
  filldraw(CR( (1,4), r), white, black);
  filldraw(CR( (3,4), r), white, black);

  real s = 3*r;
  fill(CR( (0,0), s), opacity(0.2)+lightred );
  fill(CR( (1,0), s), opacity(0.2)+lightred );
  fill(CR( (1,1), s), opacity(0.2)+lightred );
  fill(CR( (2,1), s), opacity(0.2)+lightred );
  fill(CR( (2,2), s), opacity(0.2)+lightred );
  fill(CR( (3,2), s), opacity(0.2)+lightred );

  pen domino = heavygreen+1.2;
  path vd = box( (-0.4, -0.4), (0.4, 1.4) );
  path hd = rotate(-90)*vd;
  draw(shift(0,0)*vd, domino);
  draw(shift(1,1)*vd, domino);
  draw(shift(2,2)*vd, domino);
  draw(shift(1,0)*hd, domino);
  draw(shift(2,1)*hd, domino);
  draw(shift(3,2)*hd, domino);
\end{asy}
\end{center}

The curve $\gamma$ by definition should cross $y=x-1$ at the point $b = (1,0)$.
Let $a$ denote the first point of this sequence
after $p_1$ for which $\gamma$ \emph{crosses} $y=x-1$ again.

Now $a$ is tiled by a vertical domino whose black point is to the right of $\ell$.
But the line segment $\ell$ cuts $\mathcal Q$ into two parts,
and the orientation of $\gamma$ has this path also entering from the right.
This contradicts the fact that the orientation of $\gamma$
points only from white to black within dominoes.
This contradiction completes the proof.

\begin{remark*}
  Note the problem is false if you allow holes
  (consider a $3 \times 3$ with the middle square deleted).
\end{remark*}
\pagebreak

\section{Solutions to Day 2}
\subsection{USAMO 2009/4, proposed by Titu Andreescu}
\textsl{Available online at \url{https://aops.com/community/p1485147}.}
\begin{mdframed}[style=mdpurplebox,frametitle={Problem statement}]
For $n \ge 2$, let $a_1$, $a_2$, \dots, $a_n$ be positive real numbers such that
\[
  \left( a_1 + a_2 + \dots + a_n \right)
  \left( \frac{1}{a_1} + \frac{1}{a_2} + \dots + \frac{1}{a_n} \right)
  \le \left( n + \half \right)^2.
\]
Prove that $\max\left( a_1, \dots, a_n \right) \le 4 \min\left( a_1, \dots, a_n \right)$.
\end{mdframed}
Assume $a_1$ is the largest and $a_2$ is the smallest.
Let $M = a_1/a_2$.
We wish to show $M \le 4$.

In left-hand side of given, write as $a_2+a_1 + \dots + a_n$.
By Cauchy Schwarz, one obtains
\begin{align*}
  \left( n+\half \right)^2 &\ge
    \left( a_2 + a_1 + a_3 + \dots + a_n  \right)
    \left( \frac{1}{a_1} + \frac{1}{a_2} + \frac{1}{a_3}
      + \dots + \frac{1}{a_n} \right) \\
  &\ge \left(\sqrt{\frac{a_2}{a_1}} + \sqrt{\frac{a_1}{a_2}}
    + 1 + \dots + 1 \right)^2 \\
  &\ge \left(\sqrt{1/M} + \sqrt{M} + (n-2) \right)^2.
\end{align*}
Expanding and solving for $M$ gives $1/4 \le M \le 4$ as needed.
\pagebreak

\subsection{USAMO 2009/5, proposed by Zuming Feng}
\textsl{Available online at \url{https://aops.com/community/p1485200}.}
\begin{mdframed}[style=mdpurplebox,frametitle={Problem statement}]
Trapezoid $ABCD$, with $\ol{AB} \parallel \ol{CD}$,
is inscribed in circle $\omega$ and point $G$ lies inside triangle $BCD$.
Rays $AG$ and $BG$ meet $\omega$ again at points $P$ and $Q$, respectively.
Let the line through $G$ parallel to $\ol{AB}$ intersect
$\ol{BD}$ and $\ol{BC}$ at points $R$ and $S$, respectively.
Prove that quadrilateral $PQRS$ is cyclic
if and only if $\ol{BG}$ bisects $\angle CBD$.
\end{mdframed}
Perform an inversion around $B$ with arbitrary radius,
and denote the inverse of a point $Z$ with $Z^\ast$.

\begin{center}
  \begin{asy}
    unitsize(2cm);
    pair A = Drawing("A", dir(130), dir(130));
    pair B = Drawing("B", dir(50), dir(50));
    pair C = Drawing("C", dir(-20), dir(-20));
    pair D = Drawing("D", dir(200), dir(200));
    pair Q = Drawing("Q", dir(270), dir(270));
    pair G = Drawing("G", 0.3*Q+0.7*B, 1.414*dir(290));
    pair R = Drawing("R", extension(B,D,G,G-(1,0)), dir(135));
    pair S = Drawing("S", extension(B,C,G,G+(1,0)), dir(45));
    pair P = Drawing("P", 2*foot(origin,A,G)-A, dir(G-A));
    draw(unitcircle);
    draw(D--A--B--C--D--B--Q--P--A);
    draw(R--S);
    draw(circumcircle(P,Q,R), dashed);
  \end{asy}
  \quad
  \begin{asy}
    unitsize(2cm);
    pair B = Drawing("B", dir(130), dir(110));
    pair R = Drawing("R^\ast", dir(210), dir(210));
    pair S = Drawing("S^\ast", dir(330), dir(330));
    pair G = Drawing("G^\ast", dir(270), dir(290));
    pair D = Drawing("D^\ast", .6*R+.4*B, 1.414*dir(230));
    pair C = Drawing("C^\ast", .6*S+.4*B, 1.414*dir(-90));
    pair Q = Drawing("Q^\ast", extension(B,G,C,D), dir(45));
    pair A = Drawing("A^\ast", extension(B,2/(B+conj(B)),C,D), dir(180));
    pair P = Drawing("P^\ast", 2*foot(circumcenter(A,B,G),C,D)-A,dir(-25));
    draw(unitcircle);
    draw(circumcircle(B,C,D));
    draw(A--P);
    draw(R--B--S--cycle);
    draw(G--B);
    draw(circumcircle(A,B,G));
    draw(A--B);
    /*
    pair X = Drawing("X", 2*foot(origin,P,G)-G, dir(P-G));
    draw(G--X, dashed);
    */
  \end{asy}
\end{center}

After inversion, we obtain a cyclic quadrilateral
$BS^\ast G^\ast R^\ast$ and points $C^\ast$, $D^\ast$
on $\ol{BS^\ast}$, $\ol{BR^\ast}$,
such that $(BC^\ast D^\ast)$ is tangent to $(BS^\ast G^\ast R^\ast)$ ---
in other words, so that $\ol{C^\ast D^\ast}$
is parallel to $\ol{S^\ast R^\ast}$.
Point $A^\ast$ lies on line $\ol{C^\ast D^\ast}$
so that $\ol{A^\ast B}$ is tangent to $(BS^\ast G^\ast R^\ast)$.
Points $P^\ast$ and $Q^\ast$ are the
intersections of $(A^\ast BG^\ast)$ and $\ol{BG^\ast}$
with line $C^\ast D^\ast$.

Observe that $P^\ast Q^\ast R^\ast S^\ast$ is a trapezoid,
so it is cyclic if and only if it isosceles.

Let $X$ be the second intersection of line $G^\ast P^\ast$
with $(B S^\ast R^\ast)$.  Because
\[ \dang Q^\ast P^\ast G^\ast
  = \dang A^\ast B G^\ast = \dang BXG^\ast \]
we find that $BXS^\ast R^\ast$ is an isosceles trapezoid.

If $G^\ast$ is indeed the midpoint of the arc
then everything is clear by symmetry now.
Conversely, if $P^\ast R^\ast = Q^\ast S^\ast$ then
that means $P^\ast Q^\ast R^\ast S^\ast$ is a cyclic trapezoid,
and hence that the perpendicular bisectors of
$\ol{P^\ast Q^\ast}$ and $\ol{R^\ast S^\ast}$ are the same.
Hence $B$, $X$, $P^\ast$, $Q^\ast$ are symmetric around this line.
This forces $G^\ast$ to be the midpoint of arc $R^\ast S^\ast$ as desired.
\pagebreak

\subsection{USAMO 2009/6, proposed by Gabriel Carroll}
\textsl{Available online at \url{https://aops.com/community/p1485204}.}
\begin{mdframed}[style=mdpurplebox,frametitle={Problem statement}]
Let $s_1, s_2, s_3, \dots$ be an infinite,
nonconstant sequence of rational numbers,
meaning it is not the case that $s_1 = s_2 = s_3 = \dots$.
Suppose that $t_1, t_2, t_3, \dots$ is also an infinite,
nonconstant sequence of rational numbers
with the property that $(s_i - s_j)(t_i - t_j)$
is an integer for all $i$ and $j$.
Prove that there exists a rational number $r$
such that $(s_i - s_j)r$ and $(t_i - t_j)/r$
are integers for all $i$ and $j$.
\end{mdframed}
First we eliminate the silly edge case:
\begin{claim*}
  For some $i$ and $j$, we have
  $(s_i - s_j)(t_i - t_j) \neq 0$.
\end{claim*}
\begin{proof}
  Assume not.
  WLOG $s_1 \neq s_2$, then $t_1 = t_2$.
  Now select $i$ such that $t_i \neq t_1 = t_2$.
  Then either $t_i - s_1 \neq 0$
  or $t_i - s_2 \neq 0$, contradiction.
\end{proof}

So, WLOG (by permutation) that $n = (s_1-s_2)(t_1-t_2) \neq 0$.
By shifting and scaling appropriately,
we may assume
\[ s_1 = t_1 = 0, \quad s_2 = 1, \quad t_2 = n. \]
Thus we deduce \[ s_i t_i \in \ZZ, \quad s_i t_j + s_j t_i \in \ZZ
  \qquad \forall i,j. \]

\begin{claim*}
  For any index $i$,
  $t_i \in \ZZ$, $s_i \in \frac 1n \ZZ$.
\end{claim*}
\begin{proof}
  We have $s_i t_i \in \ZZ$
  and $t_i + n s_i \in \ZZ$ by problem condition.
  By looking at $\nu_p$ of this,
  we conclude $ns_i, t_i \in \ZZ$
  (since if either as negative $p$-adic value,
  so does the other, and then $s_i t_i \notin \ZZ$).
\end{proof}

Last claim:
\begin{claim*}
  If $d = \gcd t_\bullet$, then $ds_i \in \ZZ$ for all $i$.
\end{claim*}
\begin{proof}
  Consider a prime $p \mid n$, and let $e = \nu_p(t_j)$.
  We will show $\nu_p(s_i) \ge -e$ for any $i$.

  This is already true for $i = j$,
  so assume $i \neq j$.
  Assume for contradiction $\nu_p(s_i) < -e$.
  Then $\nu_p(t_i) > e = \nu_p(t_k)$.
  Since $\nu_p(s_k) \ge -e$ we deduce
  $\nu_p(s_i t_k) < \nu_p(s_k t_i)$;
  so $\nu_p(s_i t_k) \ge 0$ and $\nu_p(s_i) \ge -e$ as desired.
\end{proof}
\pagebreak


\end{document}
