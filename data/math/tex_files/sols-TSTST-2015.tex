\documentclass[11pt]{scrartcl}
\usepackage[sexy]{evan}
\ihead{\footnotesize\textbf{\thetitle}}
\ohead{\footnotesize\theauthor}
\begin{document}

\title{TSTST 2015 Solution Notes}
\subtitle{Pittsburgh, PA}
\date{\today}

\maketitle
\begin{abstract}
This is a compilation of solutions
for the 2015 TSTST.
Some of the solutions are my own work,
but many are from the official solutions provided by the organizers
(for which they hold any copyrights),
and others were found by users on the Art of Problem Solving forums.

These notes will tend to be a bit more advanced and terse than the ``official''
solutions from the organizers.
In particular, if a theorem or technique is not known to beginners
but is still considered ``standard'', then I often prefer to
use this theory anyways, rather than try to work around or conceal it.
For example, in geometry problems I typically use directed angles
without further comment, rather than awkwardly work around configuration issues.
Similarly, sentences like ``let $\mathbb{R}$ denote the set of real numbers''
are typically omitted entirely.

Corrections and comments are welcome!
\end{abstract}
\tableofcontents
\newpage

\addtocounter{section}{-1}
\section{Problems}
\begin{enumerate}[\bfseries 1.]
\item %% Problem 1
Let $a_1$, $a_2$, \dots, $a_n$ be a sequence of real numbers,
and let $m$ be a fixed positive integer less than $n$.
We say an index $k$ with $1 \le k \le n$ is \textit{good} if there exists
some $\ell$ with $1 \le \ell \le m$ such that
\[ a_k + a_{k + 1} + \dotsb + a_{k + \ell - 1} \ge 0, \]
where the indices are taken modulo $n$.
Let $T$ be the set of all good indices. Prove that
\[ \sum_{k \in T} a_k \ge 0. \]

\item %% Problem 2
Let $ABC$ be a scalene triangle. Let $K_a$, $L_a$, and
$M_a$ be the respective intersections with $BC$ of the internal
angle bisector, external angle bisector, and the median from
$A$. The circumcircle of $AK_aL_a$ intersects $AM_a$ a second time
at a point $X_a$ different from $A$. Define $X_b$ and $X_c$
analogously. Prove that the circumcenter of $X_aX_bX_c$ lies on
the Euler line of $ABC$.

\item %% Problem 3
Let $P$ be the set of all primes, and let $M$ be a
non-empty subset of $P$. Suppose that for any non-empty subset
$\{p_1, p_2, \dots, p_k\}$ of $M$, all prime factors of
$p_1 p_2 \dots p_k + 1$ are also in $M$. Prove that $M = P$.

\item %% Problem 4
Let $x,y,z$ be real numbers (not necessarily positive)
such that $x^4 + y^4 + z^4 + xyz = 4$.
Prove that $x \le 2$ and \[ \sqrt{2-x} \ge \frac{y+z}{2}. \]

\item %% Problem 5
Let $\varphi(n)$ denote the number of positive integers
less than $n$ that are relatively prime to $n$.
Prove that there exists a positive integer $m$ for which the equation
$\varphi(n) = m$ has at least $2015$ solutions in $n$.

\item %% Problem 6
A \emph{Nim-style game} is defined as follows. Two
positive integers $k$ and $n$ are specified, along with a finite
set $S$ of $k$-tuples of integers (not necessarily positive). At
the start of the game, the $k$-tuple $(n,0,0,\dots,0)$ is written
on the blackboard.

A legal move consists of erasing the tuple $(a_1,a_2,\dots,a_k)$
which is written on the blackboard and replacing it with
$(a_1+b_1,a_2+b_2,\dots,a_k+b_k)$, where $(b_1,b_2,\dots,b_k)$
is an element of the set $S$. Two players take turns making legal
moves, and the first to write a negative integer loses. In the
event that neither player is ever forced to write a negative
integer, the game is a draw.

Prove that there is a choice of $k$ and $S$ with the following
property: the first player has a winning strategy if $n$ is a
power of $2$, and otherwise the second player has a winning
strategy.

\end{enumerate}
\pagebreak

\section{Solutions to Day 1}
\subsection{TSTST 2015/1, proposed by Mark Sellke}
\textsl{Available online at \url{https://aops.com/community/p5017901}.}
\begin{mdframed}[style=mdpurplebox,frametitle={Problem statement}]
Let $a_1$, $a_2$, \dots, $a_n$ be a sequence of real numbers,
and let $m$ be a fixed positive integer less than $n$.
We say an index $k$ with $1 \le k \le n$ is \textit{good} if there exists
some $\ell$ with $1 \le \ell \le m$ such that
\[ a_k + a_{k + 1} + \dotsb + a_{k + \ell - 1} \ge 0, \]
where the indices are taken modulo $n$.
Let $T$ be the set of all good indices. Prove that
\[ \sum_{k \in T} a_k \ge 0. \]
\end{mdframed}
First we prove the result if the indices are not taken modulo $n$.
Call a number $\ell$-good if $\ell$ is the \emph{smallest} number
such that $a_k + a_{k+1} + \dots + a_{k+\ell-1} \ge 0$, and $\ell \le m$.
Then if $a_k$ is $\ell$-good,
the numbers $a_{k+1}, \dots, a_{k+\ell-1}$ are good as well.

Then by greedy from left to right,
we can group all the good numbers into blocks with nonnegative sums.
Repeatedly take the first good number,
if $\ell$-good, group it with the next $\ell$ numbers.
An example for $m = 3$:
\[ \langle 4 \rangle
  \quad \langle -1 \quad -2 \quad 6 \rangle
  \quad -9
  \quad -7
  \quad \langle 3 \rangle
  \quad \langle -2 \quad 4 \rangle
  \quad \langle -1. \]

We can now return to the original problem.
Let $N$ be a large integer; applying the algorithm to
$N$ copies of the sequence, we deduce that
\[ N \sum_{k \in T} a_k + c_N \ge 0 \]
where $c_N$ represents some ``error'' from left-over terms.
As $|c_N| \le \sum |a_i|$, by taking $N$ large enough we deduce the problem.

\begin{remark*}
  This solution was motivated by looking at the case $m = 1$,
  realizing how dumb it was, then looking at $m = 2$,
  and realizing it was equally dumb.
\end{remark*}
\pagebreak

\subsection{TSTST 2015/2, proposed by Ivan Borsenco}
\textsl{Available online at \url{https://aops.com/community/p5017915}.}
\begin{mdframed}[style=mdpurplebox,frametitle={Problem statement}]
Let $ABC$ be a scalene triangle. Let $K_a$, $L_a$, and
$M_a$ be the respective intersections with $BC$ of the internal
angle bisector, external angle bisector, and the median from
$A$. The circumcircle of $AK_aL_a$ intersects $AM_a$ a second time
at a point $X_a$ different from $A$. Define $X_b$ and $X_c$
analogously. Prove that the circumcenter of $X_aX_bX_c$ lies on
the Euler line of $ABC$.
\end{mdframed}
The main content of the problem:
\begin{claim*}
  $\angle H X_a G = 90\dg$.
\end{claim*}
This implies the result,
since then the desired circumcenter is the midpoint of $\ol{GH}$.
(This is the main difficulty; the Euler line is a red herring.)

In what follows, we abbreviate $K_a$ $L_a$, $M_a$, $X_a$
to $K$, $L$, $M$, $X$.

\begin{proof}
  [First proof by Brokard]
  To do this, it suffices to show that $M$ has the
  same power with respect to the circle
  with diameter $\ol{AH}$ and the circle with diameter $\ol{K L}$.
  In fact I claim both circles are orthogonal
  to the circle with diameter $\ol{BC}$!
  The former follows from Brokard's theorem, noting that $A$
  is on the polar of $H$,
  and the latter follows from the harmonic bundle.

  \begin{center}
  \begin{asy}
  size(11cm);
  pair A = dir(150);
  pair B = dir(200);
  pair C = dir(340);
  filldraw(A--B--C--cycle, opacity(0.1)+lightred, red);

  pair M = midpoint(B--C);

  pair H = orthocenter(A, B, C);
  pair G = centroid(A, B, C);
  pair X = foot(H, M, A);

  pair K = extension(A, incenter(A, B, C), B, C);
  pair L = extension(A, A+dir(90)*(K-A), B, C);

  draw(K--A--L--B, red);
  draw(circumcircle(A, K, L), orange+dashed);
  draw(A--M, red);
  draw(X--H, orange);

  dot("$A$", A, dir(A));
  dot("$B$", B, dir(270));
  dot("$C$", C, dir(270));
  dot("$M$", M, dir(M));
  dot("$H$", H, dir(270));
  dot("$G$", G, dir(10));
  dot("$X$", X, dir(10));
  dot("$K$", K, dir(K));
  dot("$L$", L, dir(L));

  /* TSQ Source:

  !size(11cm);
  A = dir 150
  B = dir 200 R270
  C = dir 340 R270
  A--B--C--cycle 0.1 lightred / red

  M = midpoint B--C

  H = orthocenter A B C R270
  G = centroid A B C R10
  X = foot H M A R10

  K = extension A incenter A B C B C
  L = extension A A+dir(90)*(K-A) B C

  K--A--L--B red
  circumcircle A K L orange dashed
  A--M red
  X--H orange

  */
  \end{asy}
  \end{center}
  Then $\ol{AM}$ is the radical axis, so $X$ lies on both circles.
\end{proof}

\begin{proof}
  [Second proof by orthocenter reflection, Bendit Chan]
  As before, we know $MX \cdot MA = MK \cdot ML = MB \cdot MC$,
  but $X$ lies inside segment $AM$.
  Construct parallelogram $ABA'C$.
  Then $MX \cdot MA' = MB \cdot MC$,
  so $XBA'C$ is concyclic.

  However, it is well-known the circumcircle of $\triangle BA'C$
  (which is the reflection of $(ABC)$ across $\ol{BC}$)
  passes through $H$ and in fact has diameter $\ol{A'H}$.
  So this gives $\angle HXA' = 90\dg$ as needed.
\end{proof}

\begin{proof}
  [Third proof by barycentric coordinates]
  Alternatively we may just compute $X = (a^2:2S_A:2S_A)$.
  Let $F = (0 : S_C : S_B)$ be the foot from $H$.
  Then we check that $X H F M$ is cyclic,
  which is power of a point from $A$.
\end{proof}
\pagebreak

\subsection{TSTST 2015/3, proposed by Alex Zhai}
\textsl{Available online at \url{https://aops.com/community/p5017928}.}
\begin{mdframed}[style=mdpurplebox,frametitle={Problem statement}]
Let $P$ be the set of all primes, and let $M$ be a
non-empty subset of $P$. Suppose that for any non-empty subset
$\{p_1, p_2, \dots, p_k\}$ of $M$, all prime factors of
$p_1 p_2 \dots p_k + 1$ are also in $M$. Prove that $M = P$.
\end{mdframed}
The following solution was found by user \texttt{Aiscrim} on AOPS.

Obviously $|M| = \infty$.
Assume for contradiction $p \notin M$.
We say a prime $q \in M$ is \emph{sparse}
if there are only finitely many elements of $M$
which are $q \pmod p$
(in particular there are finitely many sparse primes).

Now let $C$ be the product of all sparse primes (note $p \nmid C$).
First, set $a_0 = 1$.
For $k \ge 0$, consider then the prime factorization of the number
\[ Ca_k + 1. \]
No prime in its factorization is sparse,
so consider the number $a_{k+1}$ obtained by
\textbf{replacing each prime in its factorization
with some arbitrary representative of that prime's residue class}.
In this way we select a number $a_{k+1}$ such that
\begin{itemize}
  \ii $a_{k+1} \equiv Ca_k + 1 \pmod p$, and
  \ii $a_{k+1}$ is a product of distinct primes in $M$.
\end{itemize}
In particular, $a_k \equiv C^k + C^{k-1} + \dots + 1 \pmod p$

But since $C \not\equiv 0 \pmod p$,
we can find a $k$ such that $a_k \equiv 0 \pmod p$
(namely, $k=p-1$ if $C \equiv 1$ and $k=p-2$ else)
which is clearly impossible since $a_k$ is a product of primes in $M$!
\pagebreak

\section{Solutions to Day 2}
\subsection{TSTST 2015/4, proposed by Alyazeed Basyoni}
\textsl{Available online at \url{https://aops.com/community/p5017801}.}
\begin{mdframed}[style=mdpurplebox,frametitle={Problem statement}]
Let $x,y,z$ be real numbers (not necessarily positive)
such that $x^4 + y^4 + z^4 + xyz = 4$.
Prove that $x \le 2$ and \[ \sqrt{2-x} \ge \frac{y+z}{2}. \]
\end{mdframed}
We prove that the condition $x^4+y^4+z^4+xyz = 4$ implies
\[ \sqrt{2-x} \geq \frac{y+z}{2}. \]

We first prove the  easy part.
\begin{claim*}
  We have $x \le 2$.
\end{claim*}
\begin{proof}
  Indeed, AM-GM gives that
  \begin{align*}
    5 =x^4+y^4+(z^4+1)+xyz
      &= \frac{3x^4}{4}+\left(\frac{x^4}{4}+y^4\right)+(z^4+1)+xyz \\
      &\geq \frac{3x^4}{4}+x^2y^2+2z^2+xyz.
  \end{align*}
  We evidently have that $x^2y^2+2z^2+xyz \geq 0$ because the
  quadratic form $a^2+ab+2b^2$ is positive definite, so $x^4 \leq
  \frac{20}{3} \implies x \leq 2$.
\end{proof}
Now, the desired statement is
implied by its square, so it suffices to show that
\[ 2-x \geq \left(\frac{y+z}{2}\right)^2 \]

We are going to proceed by contradiction
(it seems that many solutions do this)
and assume that
\[ 2-x < \left( \frac{y+z}{2} \right)^2
  \iff 4x + y^2 + 2yz + z^2 > 8. \]
By AM-GM,
\begin{align*}
  x^4 + 3 & \ge 4x \\
  \tfrac{y^4+1}{2} &\ge y^2 \\
  \tfrac{z^4+1}{2} &\ge z^2
\end{align*}
which yields that
\[ x^4+\frac{y^4+z^4}{2}+2yz+4 > 8. \]
If we replace $x^4 = 4 - (y^4+z^4+xyz)$ now,
this gives
\[ -\frac{y^4+z^4}{2}+(2-x)yz > 0 \implies (2-x)yz > \frac{y^4+z^4}{2}. \]
Since $2-x$ and the right-hand side are positive,
we have $yz \geq 0$.
Now
\[ \frac{y^4+z^4}{2yz} < 2-x < \left(\frac{y+z}{2}\right)^2
  \implies 2y^4+2z^4 < yz(y+z)^2 = y^3z+2y^2z^2+yz^3. \]
This is clearly false by AM-GM, so we have a contradiction.
\pagebreak

\subsection{TSTST 2015/5}
\textsl{Available online at \url{https://aops.com/community/p5017821}.}
\begin{mdframed}[style=mdpurplebox,frametitle={Problem statement}]
Let $\varphi(n)$ denote the number of positive integers
less than $n$ that are relatively prime to $n$.
Prove that there exists a positive integer $m$ for which the equation
$\varphi(n) = m$ has at least $2015$ solutions in $n$.
\end{mdframed}
Here are two explicit solutions.

\paragraph{First solution with ad-hoc subsets, by Evan Chen.}
I consider the following eleven prime numbers:
\[ S = \left\{ 11, 13, 17, 19, 29, 31, 37, 41, 43, 61, 71 \right\}. \]
This has the property that for any $p \in S$,
all prime factors of $p-1$ are one digit.

Let $N = (210)^{\text{billion}}$,
and consider $M = \varphi\left( N \right)$.
For any subset $T \subset S$, we have
\[ M = \varphi\left(
  \frac{N}{\prod_{p\in T} (p-1)} \prod_{p \in T} p \right). \]
Since $2^{\left\lvert S \right\rvert} > 2015$ we're done.
\begin{remark*}
This solution is motivated by the deep fact that
$\varphi(11 \cdot 1000) = \varphi(10 \cdot 1000)$,
for example.
\end{remark*}

\paragraph{Second solution with smallest primes, by Yang Liu.}
Let $2 = p_1 < p_2 < \dots < p_{2015}$
be the \emph{smallest} $2015$ primes.
Then the $2015$ numbers
\begin{align*}
  n_1 &= (p_1-1) p_2 \dots p_{2015} \\
  n_2 &= p_1 (p_2-1) \dots p_{2015} \\
  &\vdotswithin= \\
  n_{2015} &= p_1 p_2 \dots \left( p_{2015}-1 \right)
\end{align*}
all have the same phi value, namely
\[ \varphi(p_1 p_2 \dots p_{2015}) = \prod_{i=1}^{2015} (p_i-1). \]
\pagebreak

\subsection{TSTST 2015/6, proposed by Linus Hamilton}
\textsl{Available online at \url{https://aops.com/community/p5017871}.}
\begin{mdframed}[style=mdpurplebox,frametitle={Problem statement}]
A \emph{Nim-style game} is defined as follows. Two
positive integers $k$ and $n$ are specified, along with a finite
set $S$ of $k$-tuples of integers (not necessarily positive). At
the start of the game, the $k$-tuple $(n,0,0,\dots,0)$ is written
on the blackboard.

A legal move consists of erasing the tuple $(a_1,a_2,\dots,a_k)$
which is written on the blackboard and replacing it with
$(a_1+b_1,a_2+b_2,\dots,a_k+b_k)$, where $(b_1,b_2,\dots,b_k)$
is an element of the set $S$. Two players take turns making legal
moves, and the first to write a negative integer loses. In the
event that neither player is ever forced to write a negative
integer, the game is a draw.

Prove that there is a choice of $k$ and $S$ with the following
property: the first player has a winning strategy if $n$ is a
power of $2$, and otherwise the second player has a winning
strategy.
\end{mdframed}
Here we present a solution with $14$ registers and $22$ moves.
Initially $X=n$ and all other variables are zero.

\begin{center}
\scriptsize
\setlength{\tabcolsep}{3pt}
\begin{tabular}{l|rr|r|rrr|rrr|r|rr|rr}
&$X$&$Y$&Go&$S_X^0$&$S_X$&$S_X'$&$S_Y^0$&$S_Y$&$S_Y'$&Cl&$A$&$B$&Die&Die'\\\hline
Init&-1&&1&&&&&&&&&1&1&1\\
Begin&1&&-1&1&&&&&&&-1&1&&\\\hline
Sleep&&&&&&&&&&&1&-1&&\\\hline
StartX&&&&-1&1&&&&&&-1&1&&\\
WorkX&-1&&&&-1&1&&&&&-1&1&&\\
WorkX'&-1&1&&&1&-1&&&&&-1&1&&\\
DoneX&&&&&-1&&1&&&&-1&1&&\\
WrongX&-1&&&&&&-1&&&&&-1&&\\\hline
StartY&&&&&&&-1&1&&&-1&1&&\\
WorkY&&-1&&&&&&-1&1&&-1&1&&\\
WorkY'&1&-1&&&&&&1&-1&&-1&1&&\\
DoneY&&&&1&&&&-1&&&-1&1&&\\
WrongY&&-1&&-1&&&&&&&&-1&&\\\hline
ClaimX&-1&&&-1&&&&&&1&-1&1&&\\
ClaimY&&-1&&&&&-1&&&1&-1&1&&\\
FakeX&-1&&&&&&&&&-1&&-1&&\\
FakeY&&-1&&&&&&&&-1&&-1&&\\
Win&&&&&&&&&&-1&-1&&&\\\hline
PunA&&&&&&&&&&&&-2&&\\
PunB&&&&&&&&&&&-1&-1&&\\
Kill&&&&&&&&&&&&-1&-2&1\\
Kill'&&&&&&&&&&&&-1&1&-2
\end{tabular}
\end{center}

Now, the ``game'' is played as follows.
The mechanics are controlled by the \emph{turn counters} $A$ and $B$.

Observe the game starts with Alice playing Init.
Thereafter, we say that the game is
\begin{itemize}
  \ii In the \emph{main part} if $A+B = 1$,
  and no one has played Init a second time.
  \ii In the \emph{death part} otherwise.
\end{itemize}
Observe that in the main state,
on Alice's turn we always have $(A,B) = (1,0)$
and on Bob's turn we always have $(A,B) = (0,1)$.

\begin{claim*}
  A player who plays Init a second time must lose.
  In particular, a player who makes a move when $A=B=0$ must lose.
\end{claim*}
\begin{proof}
  Situations with $A+B \ge 2$ cannot occur during main part,
  so there are only a few possibilities.
  \begin{itemize}
  \ii Suppose the offending player is in a situation where $A=B=0$.
  Then he/she must play Init.
  At this point, the opposing player can respond by playing Kill.
  Then the offending player must play Init again.
  The opposing player now responds with Kill'.
  This iteration continues until $X$ reaches a negative number
  and the offending player loses.
  \ii Suppose Alice has $(A,B) = (1,0)$ but plays Init again anyways.
  Then Bob responds with PunB to punish her; he then wins as in the first case.
  \ii Suppose Bob has $(A,B) = (0,1)$ but plays Init again anyways.
  Alice responds with PunA in the same way. \qedhere
  \end{itemize}
\end{proof}

Thus we may assume that players avoid the death part at all costs.
Hence the second moves consist of Bob playing Sleep,
and then Alice playing Begin
(thus restoring the value of $n$ in $X$), then Bob playing Sleep.

Now we return to analysis of the main part.
We say the game is in \emph{state} $S$ for
$S \in \{S_X^0, S_X, S_X', S_Y^0, S_Y, S_Y', \text{Cl} \}$
if $S = 1$ and all other variables are zero.
By construction, this is always the case.
From then on the main part is divided into several phases:
\begin{itemize}
  \ii An \emph{$X$-phase}: this begins with Alice at $S_X^0$, and
  ends when the game is in a state other than $S_X$ and $S_X'$.
  (She can never return to $S_X^0$ during an $X$-phase.)
  \ii A \emph{$Y$-phase}: this begins with Alice at $S_Y^0$, and
  ends when the game is in a state other than $S_Y$ and $S_Y'$.
  (She can never return to $S_Y^0$ during a $Y$-phase.)
\end{itemize}

\begin{claim*}
Consider an $X$-phase in which $(X,Y) = (x,0)$, $x > 1$.
Then Alice can complete the phase without losing if and only if $x$ is even;
if so she begins a $Y$-phase with $(X,Y) = (0,x/2)$.
\end{claim*}
\begin{proof}
As $x >1$, Alice cannot play ClaimX since Bob will respond with FakeX and win.
Now by alternating between WorkX and WorkX',
Alice can repeatedly deduct $2$ from $X$ and add $1$ to $Y$,
leading to $(x-2,y+1)$, $(x-4,y+2)$, and so on.
(During this time, Bob can only play Sleep.)
Eventually, she must stop this process by playing DoneX, which begins a $Y$-phase.

Now note that unless $X=0$, Bob now has a winning move WrongX.
Conversely he may only play Sleep if $X=0$.
\end{proof}

We have an analogous claim for $Y$-phases.
Thus if $n$ is not a power of $2$, we see that Alice eventually loses.

Now suppose $n = 2^k$;
then Alice reaches $(X,Y) = (0,2^{k-1}), \; (2^{k-2}, 0), \; \dots$
until either reaching $(1,0)$ or $(0,1)$.
At this point she can play ClaimX or ClaimY, respectively;
the game is now in state Cl.
Bob cannot play either FakeX or FakeY, so he must play Sleep,
and then Alice wins by playing Win.
Thus Alice has a winning strategy when $n = 2^k$.
\pagebreak


\end{document}
