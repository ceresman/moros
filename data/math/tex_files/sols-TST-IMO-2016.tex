\documentclass[11pt]{scrartcl}
\usepackage[sexy]{evan}
\ihead{\footnotesize\textbf{\thetitle}}
\ohead{\footnotesize\theauthor}
\begin{document}

\title{USA IMO TST 2016 Solutions}
\subtitle{United States of America --- IMO Team Selection Tests}
\date{60\ts{th} IMO 2016 Hong Kong}

\maketitle

\tableofcontents
\newpage

\addtocounter{section}{-1}
\section{Problems}
\begin{enumerate}[\bfseries 1.]
\item %% Problem 1
Let $S = \{1, \dots, n\}$.
Given a bijection $f \colon S \to S$ an \emph{orbit} of $f$ is a
set of the form $\{x, f(x), f(f(x)), \dots \}$ for some $x \in S$.
We denote by $c(f)$ the number of distinct orbits of $f$.
For example, if $n=3$ and $f(1)=2$, $f(2)=1$, $f(3)=3$,
the two orbits are $\{1,2\}$ and $\{3\}$, hence $c(f)=2$.

Given $k$ bijections $f_1$, \dots, $f_k$ from $S$ to itself, prove that
\[ c(f_1) + \dots + c(f_k) \le n(k-1) + c(f) \]
where $f \colon S \to S$ is the composed function $f_1 \circ \dots \circ f_k$.

\item %% Problem 2
Let $ABC$ be a scalene triangle with circumcircle $\Omega$,
and suppose the incircle of $ABC$ touches $BC$ at $D$.
The angle bisector of $\angle A$ meets $BC$ and $\Omega$ at $K$ and $M$.
The circumcircle of $\triangle DKM$ intersects the $A$-excircle
at $S_1$, $S_2$, and $\Omega$ at $T \neq M$.
Prove that line $AT$ passes through either $S_1$ or $S_2$.

\item %% Problem 3
Let $p$ be a prime number. Let $\FF_p$ denote the integers modulo $p$,
and let $\FF_p[x]$ be the set of polynomials with coefficients in $\FF_p$.
Define $\Psi \colon \FF_p[x] \to \FF_p[x]$ by
\[ \Psi\left( \sum_{i=0}^n a_i x^i \right) = \sum_{i=0}^n a_i x^{p^i}. \]
Prove that for nonzero polynomials $F,G \in \FF_p[x]$,
\[ \Psi(\gcd(F,G)) = \gcd(\Psi(F), \Psi(G)). \]
%Here, a polynomial $Q$ divides $P$ if there exists $R \in \FF_p[x]$
%such that $P(x) - Q(x) R(x)$ is the polynomial with all coefficients $0$
%(with all addition and multiplication in the coefficients taken modulo $p$),
%and the gcd of two polynomials is the highest degree polynomial
%with leading coefficient $1$ which divides both of them.
%A non-zero polynomial is a polynomial with not all coefficients $0$.
%As an example of multiplication, $(x+1)(x+2)(x+3) = x^3+x^2+x+1$
%in $\FF_5[x]$.

\item %% Problem 4
Let $\sqrt{3}=1.b_1b_2b_3\dots_{(2)}$ be the binary representation
of $\sqrt 3$. Prove that for any positive integer $n$, at least one of
the digits $b_n, b_{n+1}, \dots, b_{2n}$ equals 1.

\item %% Problem 5
Let $n \ge 4$ be an integer.
Find all functions $W \colon \{1, \dots, n\}^2 \to \RR$ such that
for every partition $[n] = A \cup B \cup C$ into disjoint sets,
\[ \sum_{a \in A} \sum_{b \in B} \sum_{c \in C} W(a,b) W(b,c)
= |A| |B| |C|. \]

\item %% Problem 6
Let $ABC$ be an acute scalene triangle
and let $P$ be a point in its interior.
Let $A_1$, $B_1$, $C_1$ be projections of $P$ onto
triangle sides $BC$, $CA$, $AB$, respectively.
Find the locus of points $P$ such that
$AA_1$, $BB_1$, $CC_1$ are concurrent
and $\angle PAB + \angle PBC + \angle PCA = 90\dg$.

\end{enumerate}
\pagebreak

\section{Solutions to Day 1}
\subsection{USA TST 2016/1, proposed by Maria Monks}
\textsl{Available online at \url{https://aops.com/community/p5679356}.}
\begin{mdframed}[style=mdpurplebox,frametitle={Problem statement}]
Let $S = \{1, \dots, n\}$.
Given a bijection $f \colon S \to S$ an \emph{orbit} of $f$ is a
set of the form $\{x, f(x), f(f(x)), \dots \}$ for some $x \in S$.
We denote by $c(f)$ the number of distinct orbits of $f$.
For example, if $n=3$ and $f(1)=2$, $f(2)=1$, $f(3)=3$,
the two orbits are $\{1,2\}$ and $\{3\}$, hence $c(f)=2$.

Given $k$ bijections $f_1$, \dots, $f_k$ from $S$ to itself, prove that
\[ c(f_1) + \dots + c(f_k) \le n(k-1) + c(f) \]
where $f \colon S \to S$ is the composed function $f_1 \circ \dots \circ f_k$.
\end{mdframed}
Most motivated solution is to consider $n - c(f)$ and show this is the transposition distance.
Dumb graph theory works as well.
\pagebreak

\subsection{USA TST 2016/2, proposed by Evan Chen}
\textsl{Available online at \url{https://aops.com/community/p5679361}.}
\begin{mdframed}[style=mdpurplebox,frametitle={Problem statement}]
Let $ABC$ be a scalene triangle with circumcircle $\Omega$,
and suppose the incircle of $ABC$ touches $BC$ at $D$.
The angle bisector of $\angle A$ meets $BC$ and $\Omega$ at $K$ and $M$.
The circumcircle of $\triangle DKM$ intersects the $A$-excircle
at $S_1$, $S_2$, and $\Omega$ at $T \neq M$.
Prove that line $AT$ passes through either $S_1$ or $S_2$.
\end{mdframed}
We present an angle-chasing solution,
and then a more advanced alternative finish.

\paragraph{First solution (angle chasing).}
Assume for simplicity $AB < AC$.
Let $E$ be the contact point of the $A$-excircle on $BC$;
also let ray $TD$ meet $\Omega$ again at $L$.
From the fact that $\angle MTL = \angle MTD = 180^{\circ} - \angle MKD$,
we can deduce that $\angle MTL = \angle ACM$,
meaning that $L$ is the reflection of $A$ across the
perpendicular bisector $\ell$ of $BC$.
If we reflect $T$, $D$, $L$ over $\ell$, we deduce $A$, $E$ and the reflection of $T$ across
$\ell$ are collinear, which implies that $\angle BAT = \angle CAE$.

Now, consider the reflection point $E$ across line $AI$, say $S$.
Since ray $AI$ passes through the $A$-excenter, $S$ lies on the $A$-excircle.
Since $\angle BAT = \angle CAE$, $S$ also lies on ray $AT$.
But the circumcircles of triangles $DKM$ and $KME$ are congruent (from $DM = EM$),
so $S$ lies on the circumcircle of $\triangle DKM$ too.
Hence $S$ is the desired intersection point.

\begin{center}
\begin{asy}
size(12cm);
pair A = dir(40);
pair B = dir(210);
pair C = dir(330);
pair I = incenter(A, B, C);
pair D = foot(I, B, C);
pair K = extension(A, I, B, C);
pair M = dir(-90);
pair E = B+C-D;
pair I_A = 2*M-I;

pair S = -E+2*foot(E, K, M);

pair L = dir(90);

filldraw(A--B--C--cycle, opacity(0.2)+lightblue, blue);
filldraw(unitcircle, opacity(0.1)+lightblue, blue);

filldraw(circumcircle(D, K, M), opacity(0.2)+lightgreen, green);
filldraw(circumcircle(E, K, M), opacity(0.2)+lightred, red);

draw(A--I_A, blue);

draw(arc(I_A, abs(E-I_A), -20, 120), heavycyan);
draw(E--S, orange);

pair T = extension(I, L, A, S);
draw(E--A--S, orange+dashed);

dot("$A$", A, dir(A));
dot("$B$", B, dir(B));
dot("$C$", C, dir(C));
dot("$D$", D, dir(90));
dot("$K$", K, dir(100));
dot("$M$", M, dir(M));
dot("$E$", E, dir(E));
dot("$I_A$", I_A, dir(I_A));
dot("$S$", S, dir(10));
dot("$T$", T, dir(T));

/* Source generated by TSQ

!size(12cm);
A = dir 40
B = dir 210
C = dir 330
I := incenter A B C
D = foot I B C R90
K = extension A I B C R100
M = dir -90
E = B+C-D
I_A = 2*M-I

S = -E+2*foot E K M R10
L := dir 90

A--B--C--cycle 0.2 lightblue / blue
unitcircle 0.1 lightblue / blue

circumcircle D K M 0.2 lightgreen / green
circumcircle E K M 0.2 lightred / red

A--I_A blue

!draw(arc(I_A, abs(E-I_A), -20, 120), heavycyan);
E--S orange

T = extension I L A S
E--A--S orange dashed
*/
\end{asy}
\end{center}

\paragraph{Second solution (advanced).}
It's known that $T$ is the touch-point of the $A$-mixtilinear incircle.
Let $E$ be contact point of $A$-excircle on $BC$.
Now the circumcircles of $\triangle DKM$ and $\triangle KME$ are congruent,
since $DM = ME$ and the angles at $K$ are supplementary.
Let $S$ be the reflection of $E$ across line $KM$, which by
the above the above comment lies on the circumcircle of $\triangle DKM$.
Since $KM$ passes through the $A$-excenter, $S$ also lies on the $A$-excircle.
But $S$ also lies on line $AT$, since lines $AT$ and $AE$ are isogonal
(the mixtilinear cevian is isogonal to the Nagel line).
Thus $S$ is the desired intersection point.

\paragraph{Authorship comments.}
This problem comes from an observation of mine:
let $ABC$ be a triangle,
let the $\angle A$ bisector meet $\ol{BC}$ and $(ABC)$ at $E$ and $M$.
Let $W$ be the tangency point of the $A$-mixtilinear excircle
with the circumcircle of $ABC$.
Then $A$-Nagel line passed through a common intersection
of the circumcircle of $\triangle MEW$
and the $A$-mixtilinear incircle.

This problem is the inverted version of this observation.
\pagebreak

\subsection{USA TST 2016/3, proposed by Mark Sellke}
\textsl{Available online at \url{https://aops.com/community/p5679392}.}
\begin{mdframed}[style=mdpurplebox,frametitle={Problem statement}]
Let $p$ be a prime number. Let $\FF_p$ denote the integers modulo $p$,
and let $\FF_p[x]$ be the set of polynomials with coefficients in $\FF_p$.
Define $\Psi \colon \FF_p[x] \to \FF_p[x]$ by
\[ \Psi\left( \sum_{i=0}^n a_i x^i \right) = \sum_{i=0}^n a_i x^{p^i}. \]
Prove that for nonzero polynomials $F,G \in \FF_p[x]$,
\[ \Psi(\gcd(F,G)) = \gcd(\Psi(F), \Psi(G)). \]
%Here, a polynomial $Q$ divides $P$ if there exists $R \in \FF_p[x]$
%such that $P(x) - Q(x) R(x)$ is the polynomial with all coefficients $0$
%(with all addition and multiplication in the coefficients taken modulo $p$),
%and the gcd of two polynomials is the highest degree polynomial
%with leading coefficient $1$ which divides both of them.
%A non-zero polynomial is a polynomial with not all coefficients $0$.
%As an example of multiplication, $(x+1)(x+2)(x+3) = x^3+x^2+x+1$
%in $\FF_5[x]$.
\end{mdframed}
Observe that $\Psi$ is also a linear map of $\FF_p$ vector spaces,
and that $\Psi(xP) = \Psi(P)^p$ for any $P \in \FF_p[x]$.
(In particular, $\Psi(1) = x$, not $1$, take caution!)

\paragraph{First solution (Ankan Bhattacharya).}
We start with:
\begin{claim*}
  If $P \mid Q$ then $\Psi(P) \mid \Psi(Q)$.
\end{claim*}
\begin{proof}
  Set $Q = PR$, where $R = \sum_{i=0}^k r_i x^i$.
  Then
  \[ \Psi(Q) = \Psi\left( P\sum_{i=0}^k r_i x^i \right) \\
    = \sum_{i=0}^k \Psi\left( P \cdot r_i x^i  \right) \\
    = \sum_{i=0}^k r_i \Psi(P)^{p^i} \]
  which is divisible by $\Psi(P)$.
\end{proof}

This already implies
\[ \Psi(\gcd(F,G)) \mid \gcd(\Psi(F), \Psi(G)). \]
For the converse, by Bezout there exists $A,B \in \FF_p[x]$
such that $AF + BG = \gcd(F,G)$, so taking $\Psi$ of both sides gives
\[ \Psi(AF) + \Psi(BG) = \Psi\left( \gcd(F,G) \right). \]
The left-hand side is divisible by $\gcd(\Psi(F), \Psi(G))$
since the first term is divisible by $\Psi(F)$
and the second term is divisible by $\Psi(G)$.
So $\gcd(\Psi(F), \Psi(G)) \mid \Psi(\gcd(F,G))$
and noting both sides are monic we are done.


\paragraph{Second solution.}
Here is an alternative (longer but more conceptual)
way to finish without Bezout lemma.
Let $\beth \subseteq \FF_p[x]$ denote the set of polynomials
in the image of $\Psi$,
thus $\Psi \colon \FF_p[x] \to \beth$ is a bijection on the level of sets.

\begin{claim*}
  If $A,B \in \beth$ then $\gcd(A,B) \in \beth$.
\end{claim*}
\begin{proof}
  It suffices to show that if $A$ and $B$ are monic,
  and $\deg A > \deg B$,
  then the remainder when $A$ is divided by $B$ is in $\beth$.
  Suppose $\deg A = p^k$ and $B = x^{p^{k-1}} - c_2x^{p^{k-2}} - \dots - c_k$.
  Then
  \begin{align*}
    x^{p^k} &\equiv \left( c_2x^{p^{k-2}} + c_3x^{p^{k-3}}
    + \dots + c_k  \right)^p \pmod B  \\
    &\equiv c_2x^{p^{k-1}} + c_3x^{p^{k-2}} \dots + c_k \pmod B
  \end{align*}
  since exponentiation by $p$ commutes with addition in $\FF_p$.
  This is enough to imply the conclusion.
  The proof if $\deg B$ is smaller less than $p^{k-1}$ is similar.
\end{proof}

Thus, if we view $\FF_p[x]$ and $\beth$ as partially ordered sets
under polynomial division, then $\gcd$ is the
``greatest lower bound'' or ``meet'' in both partially ordered sets.
We will now prove that $\Psi$ is an \emph{isomorphism} of the posets.
We have already seen that $P \mid Q \implies \Psi(P) \mid \Psi(Q)$
from the first solution. For the converse:

\begin{claim*}
  If $\Psi(P) \mid \Psi(Q)$ then $P \mid Q$.
\end{claim*}
\begin{proof}
  Suppose $\Psi(P) \mid \Psi(Q)$, but $Q=PA+B$ where $\deg B < \deg P$.
  Thus $\Psi(P) \mid \Psi(PA) + \Psi(B)$, hence $\Psi(P) \mid \Psi(B)$,
  but $\deg \Psi(P) > \deg \Psi(B)$ hence $\Psi(B) = 0 \implies B = 0$.
\end{proof}

This completes the proof.

\begin{remark*}
   In fact $\Psi \colon \FF_p[x] \to \beth$
   is a ring isomorphism
   if we equip $\beth$ with function composition
   as the ring multiplication.
   Indeed in the proof of the first claim
   (that $P \mid Q \implies \Psi(P) \mid \Psi(Q)$) we saw that
   \[ \Psi(RP) = \sum_{i=0}^k r_i \Psi(P)^{p^i} = \Psi(R) \circ \Psi(P). \]
\end{remark*}
\pagebreak

\section{Solutions to Day 2}
\subsection{USA TST 2016/4, proposed by Iurie Boreico}
\textsl{Available online at \url{https://aops.com/community/p6368201}.}
\begin{mdframed}[style=mdpurplebox,frametitle={Problem statement}]
Let $\sqrt{3}=1.b_1b_2b_3\dots_{(2)}$ be the binary representation
of $\sqrt 3$. Prove that for any positive integer $n$, at least one of
the digits $b_n, b_{n+1}, \dots, b_{2n}$ equals 1.
\end{mdframed}
Assume the contrary, so that for some integer $k$ we have
\[k < 2^{n-1} \sqrt 3 < k + \frac{1}{2^{n+1}}. \]
Squaring gives
\begin{align*}
k^2 < 3 \cdot 2^{2n-2} &< k^2 + \frac{k}{2^n} + \frac{1}{2^{2n+2}} \\
&\le k^2 + \frac{2^{n-1} \sqrt 3}{2^n} + \frac{1}{2^{2n+2}} \\
&= k^2 + \frac{\sqrt3}{2} + \frac{1}{2^{2n+2}} \\
&\le k^2 + \frac{\sqrt3}{2} + \frac{1}{16} \\
&< k^2+1
\end{align*}
and this is a contradiction.
\pagebreak

\subsection{USA TST 2016/5, proposed by Zilin Jiang}
\textsl{Available online at \url{https://aops.com/community/p6368185}.}
\begin{mdframed}[style=mdpurplebox,frametitle={Problem statement}]
Let $n \ge 4$ be an integer.
Find all functions $W \colon \{1, \dots, n\}^2 \to \RR$ such that
for every partition $[n] = A \cup B \cup C$ into disjoint sets,
\[ \sum_{a \in A} \sum_{b \in B} \sum_{c \in C} W(a,b) W(b,c)
= |A| |B| |C|. \]
\end{mdframed}
Of course, $W(k,k)$ is arbitrary for $k \in [n]$.
We claim that $W(a,b) = \pm 1$ for any $a \neq b$, with the sign fixed.
(These evidently work.)

First, let $X_{abc} = W(a,b)W(b,c)$ for all distinct $a$, $b$, $c$,
so the given condition is
\[ \sum_{a,b,c \in A \times B \times C} X_{abc} = |A| |B| |C|. \]
Consider the given equation with the particular choices
\begin{itemize}
  \ii $A = \{1\}$, $B = \{2\}$, $C = \{3,4,\dots,n\}$.
  \ii $A = \{1\}$, $B = \{3\}$, $C = \{2,4,\dots,n\}$.
  \ii $A = \{1\}$, $B = \{2,3\}$, $C = \{4,\dots,n\}$.
\end{itemize}
This gives
\begin{align*}
  X_{123} + X_{124} + \dots + X_{12n} &= n-2 \\
  X_{132} + X_{134} + \dots + X_{13n} &= n-2 \\
  (X_{124} + \dots + X_{12n})
  + (X_{134} + \dots + X_{13n}) &= 2(n-3).
\end{align*}
Adding the first two and
subtracting the last one gives $X_{123} + X_{132} = 2$.
Similarly, $X_{123} + X_{321} = 2$,
and in this way we have $X_{321} = X_{132}$.
Thus $W(3,2)W(2,1) = W(1,3)W(3,2)$,
and since $W(3,2) \neq 0$ (clearly) we get $W(2,1) = W(3,2)$.

Analogously, for any distinct $a$, $b$, $c$ we have $W(a,b) = W(b,c)$.
For $n \ge 4$ this is enough to imply $W(a,b) = \pm 1$ for $a \neq b$
where the choice of sign is the same for all $a$ and $b$.

\begin{remark*}
Surprisingly, the $n = 3$ case has ``extra'' solutions for
$W(1,2) = W(2,3) = W(3,1) = \pm1$,
$W(2,1) = W(3,2) = W(1,3) = \mp1$.
\end{remark*}
\begin{remark*}
[Intuition]
It should still be possible to solve the problem
with $X_{abc}$ in place of $W(a,b) W(b,c)$,
because we have about far more equations than variables $X_{a,b,c}$
so linear algebra assures us we almost certainly have a unique solution.
\end{remark*}
\pagebreak

\subsection{USA TST 2016/6, proposed by Ivan Borsenco}
\textsl{Available online at \url{https://aops.com/community/p6368189}.}
\begin{mdframed}[style=mdpurplebox,frametitle={Problem statement}]
Let $ABC$ be an acute scalene triangle
and let $P$ be a point in its interior.
Let $A_1$, $B_1$, $C_1$ be projections of $P$ onto
triangle sides $BC$, $CA$, $AB$, respectively.
Find the locus of points $P$ such that
$AA_1$, $BB_1$, $CC_1$ are concurrent
and $\angle PAB + \angle PBC + \angle PCA = 90\dg$.
\end{mdframed}
In complex numbers with $ABC$ the unit circle,
it is equivalent to solving the following two cubic equations
in $p$ and $q = \ol p$:
\begin{align*}
 (p-a)(p-b)(p-c) &= (abc)^2 (q -1/a)(q - 1/b)(q - 1/c) \\
 0 &= \prod_{\text{cyc}} (p+c-b-bcq) + \prod_{\text{cyc}} (p+b-c-bcq).
\end{align*}
Viewing this as two cubic curves in $(p,q) \in \CC^2$,
by \emph{B\'ezout's Theorem} it follows there are at most nine solutions
(unless both curves are not irreducible,
but it's easy to check the first one cannot be factored).
Moreover it is easy to name nine solutions (for $ABC$ scalene):
the three vertices, the three excenters, and $I$, $O$, $H$.
Hence the answer is just those three triangle centers $I$, $O$ and $H$.

\begin{remark*}
  On the other hand it is not easy to solve the cubics by hand;
  I tried for an hour without success.
  So I think this solution is only feasible
  with knowledge of algebraic geometry.
\end{remark*}
\begin{remark*}
  These two cubics have names:
  \begin{itemize}
    \ii The locus of $\angle PAB + \angle PBC + \angle PCA = 90\dg$
    is the \textbf{McCay cubic},
    which is the locus of points $P$
    for which $P$, $P^\ast$, $O$ are collinear.
    \ii The locus of the pedal condition
    is the \textbf{Darboux cubic},
    which is the locus of points $P$
    for which $P$, $P^\ast$, $L$ are collinear,
    $L$ denoting the de Longchamps point.
  \end{itemize}
  Assuming $P \neq P^\ast$,
  this implies $P$ and $P^\ast$
  both lie on the Euler line of $\triangle ABC$,
  which is possible only if $P=O$ or $P=H$.

  Some other synthetic solutions are posted at
  \url{https://aops.com/community/c6h1243902p6368189}.
\end{remark*}
\pagebreak


\end{document}
