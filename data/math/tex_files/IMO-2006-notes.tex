% © Evan Chen
% Downloaded from https://web.evanchen.cc/

\documentclass[11pt]{scrartcl}
\usepackage[sexy]{evan}
\ihead{\footnotesize\textbf{\thetitle}}
\ohead{\footnotesize\href{http://web.evanchen.cc}{\ttfamily web.evanchen.cc},
    updated \today}
\title{IMO 2006 Solution Notes}
\date{\today}

\begin{document}

\maketitle

\begin{abstract}
This is a compilation of solutions
for the 2006 IMO.
Some of the solutions are my own work,
but many are from the official solutions provided by the organizers
(for which they hold any copyrights),
and others were found by users on the Art of Problem Solving forums.

These notes will tend to be a bit more advanced and terse than the ``official''
solutions from the organizers.
In particular, if a theorem or technique is not known to beginners
but is still considered ``standard'', then I often prefer to
use this theory anyways, rather than try to work around or conceal it.
For example, in geometry problems I typically use directed angles
without further comment, rather than awkwardly work around configuration issues.
Similarly, sentences like ``let $\mathbb{R}$ denote the set of real numbers''
are typically omitted entirely.

Corrections and comments are welcome!
\end{abstract}

\tableofcontents
\newpage

\addtocounter{section}{-1}
\section{Problems}
\begin{enumerate}[\bfseries 1.]
\item %% Problem 1
Let $ABC$ be a triangle with incenter $I$.
A point $P$ in the interior of the triangle satisfies
\[ \angle PBA + \angle PCA = \angle PBC + \angle PCB. \]
Show that $AP \ge AI$ and that equality holds if and only if $P=I$.

\item %% Problem 2
Let $P$ be a regular $2006$-gon.
A diagonal is called \emph{good} if its endpoints
divide the boundary of $P$ into two parts,
each composed of an odd number of sides of $P$.
The sides of $P$ are also called \emph{good}.
Suppose $P$ has been dissected into triangles by $2003$ diagonals,
no two of which have a common point in the interior of $P$.
Find the maximum number of isosceles triangles having two good
sides that could appear in such a configuration.

\item %% Problem 3
Determine the least real number $M$ such that the inequality
\[ \left\lvert ab(a^2-b^2)+bc(b^2-c^2)+ca(c^2-a^2) \right\rvert
  \leq M\left( a^2+b^2+c^2 \right)^2 \]
holds for all real numbers $a$, $b$ and $c$.

\item %% Problem 4
Determine all pairs $(x,y)$ of integers such that
\[ 1 + 2^x + 2^{2x+1} = y^2. \]

\item %% Problem 5
Let $P(x)$ be a polynomial of degree $n > 1$
with integer coefficients and let $k$ be a positive integer.
Consider the polynomial
\[ Q(x) = P(P(\dots P(P(x)) \dots )) \] where $P$ occurs $k$ times.
Prove that there are at most $n$ integers $t$ such that $Q(t) = t$.

\item %% Problem 6
Assign to each side $b$ of a convex polygon $P$
the maximum area of a triangle that has $b$
as a side and is contained in $P$.
Show that the sum of the areas assigned
to the sides of $P$ is at least twice the area of $P$.

\end{enumerate}
\pagebreak

\section{Solutions to Day 1}
\subsection{IMO 2006/1}
\textsl{Available online at \url{https://aops.com/community/p571966}.}
\begin{mdframed}[style=mdpurplebox,frametitle={Problem statement}]
Let $ABC$ be a triangle with incenter $I$.
A point $P$ in the interior of the triangle satisfies
\[ \angle PBA + \angle PCA = \angle PBC + \angle PCB. \]
Show that $AP \ge AI$ and that equality holds if and only if $P=I$.
\end{mdframed}
The condition rewrites as
\[
  \angle PBC + \angle PCB
  = (\angle B - \angle PBC)
  + (\angle C - \angle PCB)
  \implies
  \angle PBC + \angle PCB = \frac{\angle B + \angle C}{2}
\]
which means that
\[ \angle BPC = 180\dg - \frac{\angle B + \angle C}{2}
  = 90\dg + \frac{\angle A}{2}
  = \angle BIC.
\]
Since $P$ and $I$ are both inside $\triangle ABC$
that implies $P$ lies on the circumcircle of $\triangle BIC$.

It's well-known (by ``Fact 5'') that the circumcenter
of $\triangle BIC$ is the arc midpoint $M$ of $\widehat{BC}$.
Therefore
\[ AI + IM = AM \le AP + PM \implies AI \le AP \]
with equality holding iff $A$, $P$, $M$ are collinear, or $P=I$.
\pagebreak

\subsection{IMO 2006/2}
\textsl{Available online at \url{https://aops.com/community/p571973}.}
\begin{mdframed}[style=mdpurplebox,frametitle={Problem statement}]
Let $P$ be a regular $2006$-gon.
A diagonal is called \emph{good} if its endpoints
divide the boundary of $P$ into two parts,
each composed of an odd number of sides of $P$.
The sides of $P$ are also called \emph{good}.
Suppose $P$ has been dissected into triangles by $2003$ diagonals,
no two of which have a common point in the interior of $P$.
Find the maximum number of isosceles triangles having two good
sides that could appear in such a configuration.
\end{mdframed}
Call a triangle with the desired property \emph{special}.
We prove the maximum number of special triangles is $1003$,
achieved by paring up the sides of the polygon.

We present two solutions for the upper bound.
Both of them rely first on two geometric notes:
\begin{itemize}
  \ii In a special triangle, the good sides are congruent
  (and not congruent to the third side).
  \ii No two isosceles triangles share a good side.
\end{itemize}

\textbf{Solution using bijections}:
Call a good diagonal \textbf{special} if it's part of a special triangle;
special diagonals come in pairs.
Consider the minor arc cut out by a special diagonal $d$,
which has an odd number of sides.
Since special diagonals come in pairs,
one can associate to $d$ a side of the polygon
not covered by any special diagonals from $d$.
Hence there are at most $2006$ special diagonals,
so at most $1003$ special triangles.

\textbf{Solution using graph theory}:
Consider the tree $T$ formed by the $2004$
triangles in the dissection, with obvious adjacency.
Let $F$ be the forest obtained by deleting
any edge corresponding to a good diagonal.
Then the resulting graph $F$ has only degrees $1$ and $3$,
with special triangles only occurring at degree $1$ vertices.

If there are $k$ good diagonals drawn,
then this forest consists of $k+1$ trees.
A tree with $n_i$ vertices ($0 \le i \le k$)
consequently has $\frac{n_i+2}{2}$ leaves.
However by the earlier remark at least $k$ leaves
don't give special triangles
(one on each side of a special diagonal);
so the number of leaves that do give good triangles is at most
\[ -k + \sum_i \frac{n_i+2}{2}
  = -k + \frac{2004 + 2(k+1)}{2} = 1003. \]
\pagebreak

\subsection{IMO 2006/3}
\textsl{Available online at \url{https://aops.com/community/p571945}.}
\begin{mdframed}[style=mdpurplebox,frametitle={Problem statement}]
Determine the least real number $M$ such that the inequality
\[ \left\lvert ab(a^2-b^2)+bc(b^2-c^2)+ca(c^2-a^2) \right\rvert
  \leq M\left( a^2+b^2+c^2 \right)^2 \]
holds for all real numbers $a$, $b$ and $c$.
\end{mdframed}
It's the same as
\[ \left\lvert (a-b)(b-c)(c-a)(a+b+c) \right\rvert
\le M \left( a^2+b^2+c^2 \right)^2. \]
Let $x=a-b$, $y=b-c$, $z=c-a$, $s=a+b+c$.
Then we want to have
\[
  \left\lvert xyzs \right\rvert
  \le \frac{M}{9} (x^2+y^2+z^2+s^2)^2.
\]
Here $x+y+z=0$.

Now if $x$ and $y$ have the same sign,
we can replace them with the average
(this increases the LHS and decreases RHS).
So we can have $x=y$, $z=-2x$.
Now WLOG $x > 0$ to get
\[ 2x^3 \cdot s \le \frac{M}{9} \left( 6x^2+s^2 \right)^2. \]
After this routine calculation gives
$M = \frac{9}{32}\sqrt2$ works and is optimal
(by $6x^2+s^2 = 2x^2 + 2x^2 + 2x^2 + s^2$ and AM-GM).
\pagebreak

\section{Solutions to Day 2}
\subsection{IMO 2006/4, proposed by Zuming Feng (USA)}
\textsl{Available online at \url{https://aops.com/community/p572815}.}
\begin{mdframed}[style=mdpurplebox,frametitle={Problem statement}]
Determine all pairs $(x,y)$ of integers such that
\[ 1 + 2^x + 2^{2x+1} = y^2. \]
\end{mdframed}
Answers: $(0, \pm 2)$, $(4, \pm 23)$, which work.

Assume $x \ge 4$.
\[ 2^x \left( 1 + 2^{x+1} \right)
  = 2^x + 2^{2x+1} = y^2 - 1 = (y-1)(y+1). \]
So either:
\begin{itemize}
  \ii $y = 2^{x-1} m + 1$ for some odd $m$, so
  \[ 1 + 2^{x+1} = m\left( 2^{x-2}m+1 \right)
    \implies 2^x = \frac{4(1-m)}{m^2-8}. \]
  \ii $y = 2^{x-1} m - 1$ for some odd $m$, so
  \[ 1 + 2^{x+1} = m\left( 2^{x-2}m-1 \right)
    \implies 2^x = \frac{4(1+m)}{m^2-8}. \]
\end{itemize}
In particular we need $4|1 \pm m| \ge 2^4 |m^2-8|$,
which is enough to imply $m < 5$.
From here easily recover $x = 4$, $m = 3$ as the last solution
(in the second case).
\pagebreak

\subsection{IMO 2006/5}
\textsl{Available online at \url{https://aops.com/community/p572821}.}
\begin{mdframed}[style=mdpurplebox,frametitle={Problem statement}]
Let $P(x)$ be a polynomial of degree $n > 1$
with integer coefficients and let $k$ be a positive integer.
Consider the polynomial
\[ Q(x) = P(P(\dots P(P(x)) \dots )) \] where $P$ occurs $k$ times.
Prove that there are at most $n$ integers $t$ such that $Q(t) = t$.
\end{mdframed}
First, we prove that:
\begin{claim*}
  [Putnam 2000 et al]
  If a number is periodic under $P$
  then in fact it's fixed by $P \circ P$.
\end{claim*}
\begin{proof}
  Let $x_1$, $x_2$, \dots, $x_n$ be a minimal orbit.
  Then
  \[ x_i - x_{i+1} \mid P(x_i) - P(x_{i+1})
    = x_{i+1} - x_{i+2} \]
  and so on cyclically.

  If any of the quantities are zero we are done.
  Else, we must eventually have $x_i - x_{i+1} = -(x_{i+1} - x_{i+2})$,
  so $x_i = x_{i+2}$ and we get $2$-periodicity.
\end{proof}

The tricky part is to study the $2$-orbits.
Suppose there exists a fixed pair $u \neq v$
with $P(u) = v$, $P(v) = u$.
(If no such pair exists, we are already done.)
Let $(a,b)$ be any other pair with $P(a) = b$, $P(b) = a$,
possibly even $a = b$, but $\{a,b\} \cap \{u,v\} = \varnothing$.
Then we should have
\[ u-a \mid P(u)-P(a) = v-b
  \mid P(v) - P(b) = u-a \]
and so $u-a$ and $v-b$ divide each other (and are nonzero).
Similarly, $u-b$ and $v-a$ divide each other.

Hence $u-a = \pm (v-b)$ and $u-b = \pm (v-a)$.
We consider all four cases:
\begin{itemize}
  \ii If $u-a = v-b$ and $u-b = v-a$
  then $u-v = b-a = a-b$, contradiction.
  \ii If $u-a = -(v-b)$ and $u-b = -(v-a)$
  then $u+v = u-v = a+b$.
  \ii If $u-a = -(v-b)$ and $u-b = v-a$,
  we get $a+b = u+v$ from the first one
  (discarding the second).
  \ii If $u-a = v-b$ and $u-b = -(v-a)$,
  we get $a+b = u+v$ from the second one
  (discarding the first one).
\end{itemize}
Thus in all possible situations we have
\[ a+b = c \coloneqq u+v \]
a fixed constant.

Therefore, any pair $(a,b)$ with $P(a) = b$
and $P(b) = a$ actually satisfies $P(a) = c-a$.
And since $\deg P > 1$,
this means there are at most $n$ roots to $a+P(a)=c$, as needed.
\pagebreak

\subsection{IMO 2006/6}
\textsl{Available online at \url{https://aops.com/community/p572824}.}
\begin{mdframed}[style=mdpurplebox,frametitle={Problem statement}]
Assign to each side $b$ of a convex polygon $P$
the maximum area of a triangle that has $b$
as a side and is contained in $P$.
Show that the sum of the areas assigned
to the sides of $P$ is at least twice the area of $P$.
\end{mdframed}
We say a polygon in \emph{almost convex}
if all its angles are at most $180\dg$.

Note that given any convex or almost convex polygon,
we can take any side $b$ and add another vertex on it,
and the sum of the labels doesn't change
(since the label of a side is the length of the side
times the distance of the farthest point).

\begin{lemma*}
  Let $N$ be an even integer.
  Then any almost convex $N$-gon with area $S$
  should have an inscribed triangle with area at least $2S/N$.
\end{lemma*}
The main work is the proof of the lemma.

Label the polygon $P_0 P_1 \dots P_{N-1}$.
Consider the $N/2$ major diagonals of the almost convex $N$-gon,
$P_0 P_{N/2}$, $P_1 P_{N/2+1}$, et cetera.
A \emph{butterfly} refers to a self-intersecting quadrilateral
$P_i P_{i+1} P_{i+1+N/2} P_{i+N/2}$.
An example of a butterfly is shown below for $N=8$.
\begin{center}
\begin{asy}
pair P_0 = (-1,-3);
pair P_1 = ( 0,-3);
pair P_2 = (2.5,-3);
pair P_3 = (3,0);
pair P_4 = (1.8,1.2);
pair P_5 = (1.1,1.9);
pair P_6 = (0.3,2.7);
pair P_7 = (-2,0.4);
filldraw(P_0--P_2--P_3--P_6--P_7--cycle, opacity(0.1)+lightcyan, blue);

filldraw(P_0--P_4--P_5--P_1--cycle, opacity(0.2)+lightred, red);
draw(P_1--P_5, red);
draw(P_2--P_6, red);
draw(P_3--P_7, red);

dot("$P_0$", P_0, dir(P_0));
dot("$P_1$", P_1, dir(P_1));
dot("$P_2$", P_2, dir(P_2));
dot("$P_3$", P_3, dir(P_3));
dot("$P_4$", P_4, dir(P_4));
dot("$P_5$", P_5, dir(P_5));
dot("$P_6$", P_6, dir(P_6));
dot("$P_7$", P_7, dir(P_7));

/* TSQ Source:

P_0 = (-1,-3)
P_1 = ( 0,-3)
P_2 = (2.5,-3)
P_3 = (3,0)
P_4 = (1.8,1.2)
P_5 = (1.1,1.9)
P_6 = (0.3,2.7)
P_7 = (-2,0.4)
P_0--P_2--P_3--P_6--P_7--cycle 0.1 lightcyan / blue

P_0--P_4--P_5--P_1--cycle 0.2 lightred / red
P_1--P_5 red
P_2--P_6 red
P_3--P_7 red

*/
\end{asy}
\end{center}

\begin{claim*}
Every point $X$ in the polygon
is contained in the wingspan of some butterfly.
\end{claim*}
\begin{proof}
Consider a windmill-like process which
\begin{itemize}
  \ii starts from some oriented red line $P_0 P_{N/2}$,
  oriented to face $P_0 P_{N/2}$
  \ii rotates through $P_0 P_{N/2} \cap P_1 P_{N/2+1}$
  to get line $P_1 P_{N/2+1}$,
  \ii rotates through $P_1 P_{N/2+1} \cap P_2 P_{N/2+2}$
  to get line $P_2 P_{N/2+2}$,
  \ii \dots et cetera, until returning to line $P_{N/2} P_0$,
  but in the reverse orientation.
\end{itemize}
At the end of the process, every point in the plane has switched
sides with our moving line.
The moment that $X$ crosses the moving red line,
we get it contained in a butterfly, as needed.
\end{proof}

\begin{claim*}
  If $ABDC = P_i P_{i+1} P_{i+1+N/2} P_{i+N/2}$ is a butterfly,
  one of the triangles $ABC$, $BCD$, $CDA$, $DAB$
  has area at least that of the butterfly.
\end{claim*}
\begin{proof}
  Let the diagonals of the butterfly meet at $O$,
  and let $a = AO$, $b = BO$, $c = CO$, $d = DO$.
  If we assume WLOG $d = \min(a,b,c,d)$
  then it follows $[ABC] = [AOB] + [BOC] \ge [AOB] + [COD]$, as needed.
\end{proof}

Now, since the $N/2$ butterflies cover an area of $S$,
it follows that one of the butterflies
has area at least $S / (N/2) = 2S/N$,
and so that butterfly gives a triangle with area at least $2S/N$,
completing the proof of the lemma.

\bigskip

\textbf{Main proof}:
Let $a_1$, \dots, $a_n$ be the numbers assigned to the sides.
Assume for contradiction $a_1 + \dots + a_n < 2S$.
We pick even integers $m_1$, $m_2$, \dots, $m_n$ such that
\begin{align*}
  \frac{a_1}{S} &< \frac{2m_1}{m_1 + \dots + m_n} \\
  \frac{a_2}{S} &< \frac{2m_2}{m_1 + \dots + m_n} \\
  &\vdotswithin\le \\
  \frac{a_n}{S} &< \frac{2m_n}{m_1 + \dots + m_n}.
\end{align*}
which is possible by rational approximation,
since the right-hand sides sum to $2$
and the left-hand sides sum to strictly less than $2$.

Now we break every side of $P$ into $m_i$ equal parts
to get an almost convex $N$-gon, where
$N = m_1 + \dots + m_n$.

The main lemma then gives us a triangle $\Delta$
of the almost convex $N$-gon
which has area at least $\frac{2S}{N}$.
If $\Delta$ used the $i$th side
then it then follows the label $a_i$ on that side should be
at least $m_i \cdot \frac{2S}{N}$, contradiction.
\pagebreak


\end{document}
