% © Evan Chen
% Downloaded from https://web.evanchen.cc/

\documentclass[11pt]{scrartcl}
\usepackage[sexy]{evan}
\ihead{\footnotesize\textbf{\thetitle}}
\ohead{\footnotesize\href{http://web.evanchen.cc}{\ttfamily web.evanchen.cc},
    updated \today}
\title{USAMO 1996 Solution Notes}
\date{\today}

\begin{document}

\maketitle

\begin{abstract}
This is a compilation of solutions
for the 1996 USAMO.
Some of the solutions are my own work,
but many are from the official solutions provided by the organizers
(for which they hold any copyrights),
and others were found by users on the Art of Problem Solving forums.

These notes will tend to be a bit more advanced and terse than the ``official''
solutions from the organizers.
In particular, if a theorem or technique is not known to beginners
but is still considered ``standard'', then I often prefer to
use this theory anyways, rather than try to work around or conceal it.
For example, in geometry problems I typically use directed angles
without further comment, rather than awkwardly work around configuration issues.
Similarly, sentences like ``let $\mathbb{R}$ denote the set of real numbers''
are typically omitted entirely.

Corrections and comments are welcome!
\end{abstract}

\tableofcontents
\newpage

\addtocounter{section}{-1}
\section{Problems}
\begin{enumerate}[\bfseries 1.]
\item %% Problem 1
Prove that the average of the numbers $n \sin n^{\circ}$ for
$n = 2,4,6,\dots,180$ is $\cot 1^{\circ}$.

\item %% Problem 2
For any nonempty set $S$ of real numbers,
let $\sigma(S)$ denote the sum of the elements of $S$.
Given a set $A$ of $n$ positive integers,
consider the collection of all distinct sums $\sigma(S)$ as $S$ ranges
over the nonempty subsets of $A$.
Prove that this collection of sums can be
partitioned into $n$ classes so that in each class,
the ratio of the largest sum to the smallest sum does not exceed $2$.

\item %% Problem 3
Let $ABC$ be a triangle. Prove that there is a line $\ell$ (in the plane
of triangle $ABC$) such that the intersection of the interior of
triangle $ABC$ and the interior of its reflection $A'B'C'$ in $\ell$ has
area more than $\frac23$ the area of triangle $ABC$.

\item %% Problem 4
An $n$-term sequence $(x_1, x_2, \dots, x_n)$
in which each term is either $0$ or $1$ is called a binary sequence of length $n$.
Let $a_n$ be the number of binary sequences of length $n$ containing
no three consecutive terms equal to $0$, $1$, $0$ in that order.
Let $b_n$ be the number of binary sequences of length $n$ that
contain no four consecutive terms equal to 0, 0, 1, 1 or 1, 1, 0, 0 in that order.
Prove that $b_{n+1} = 2a_n$ for all positive integers $n$.

\item %% Problem 5
Let $ABC$ be a triangle,
and $M$ an interior point such that
$\angle MAB=10^\circ$, $\angle MBA=20^\circ$,
$\angle MAC=40^\circ$ and $\angle MCA=30^\circ$.
Prove that the triangle is isosceles.

\item %% Problem 6
Determine with proof whether there is a subset $X \subseteq \ZZ$
with the following property: for any $n \in \ZZ$,
there is exactly one solution to $a+2b = n$, with $a,b \in X$.

\end{enumerate}
\pagebreak

\section{Solutions to Day 1}
\subsection{USAMO 1996/1}
\textsl{Available online at \url{https://aops.com/community/p353049}.}
\begin{mdframed}[style=mdpurplebox,frametitle={Problem statement}]
Prove that the average of the numbers $n \sin n^{\circ}$ for
$n = 2,4,6,\dots,180$ is $\cot 1^{\circ}$.
\end{mdframed}
Because
\[ n \sin n\dg + (180-n) \sin(180\dg-n\dg) = 180 \sin n\dg \]
So enough to show that
\[ \sum_{n=0}^{89} \sin (2n)\dg = \cot1\dg \]
Let $\zeta = \cos2\dg + i \sin 2 \dg$ be a primitive root.
Then
\begin{align*}
  \sum_{n=0}^{89} \frac{\zeta^n - \zeta^{-n}}{2i}
  &= \frac{1}{2i}
  \left[ \frac{\zeta^{90}-1}{\zeta-1} - \frac{\zeta^{-90} - 1}{\zeta^{-1}-1} \right] \\
  &= \frac{1}{2i}
  \left[ \frac{-2}{\zeta-1} - \frac{-2}{\zeta^{-1}-1} \right] \\
  &= \frac{1}{-i} \frac{\zeta\inv-\zeta}{(\zeta-1)(\zeta\inv-1)}
  = i \cdot \frac{\zeta+1}{\zeta-1}.
\end{align*}
Also,
\begin{align*}
  \cot 1\dg &= \frac{\cos 1\dg}{\sin 1\dg}
  = \frac{(\cos1\dg)^2}{\cos1\dg\sin1\dg} \\
  &= \frac{\frac{\cos2\dg+1}{2}}{\frac{\sin2\dg}{2}}
  = \frac{\half(\zeta+\zeta\inv)+1}{\frac{1}{2i}(\zeta-\zeta\inv)} \\
  &= i \cdot \frac{(\zeta+1)^2}{\zeta^2-1} = i \cdot \frac{\zeta+1}{\zeta-1}.
\end{align*}
So we're done.
\pagebreak

\subsection{USAMO 1996/2}
\textsl{Available online at \url{https://aops.com/community/p353051}.}
\begin{mdframed}[style=mdpurplebox,frametitle={Problem statement}]
For any nonempty set $S$ of real numbers,
let $\sigma(S)$ denote the sum of the elements of $S$.
Given a set $A$ of $n$ positive integers,
consider the collection of all distinct sums $\sigma(S)$ as $S$ ranges
over the nonempty subsets of $A$.
Prove that this collection of sums can be
partitioned into $n$ classes so that in each class,
the ratio of the largest sum to the smallest sum does not exceed $2$.
\end{mdframed}
By induction on $n$ with $n=1$ being easy.

For the inductive step, assume
\[ A = \left\{ a_1 > a_2 > \dots > a_n \right\}. \]
Fix any index $k$ with the property that
\[ a_k > \frac{\sigma(A)}{2^k} \]
(which must exist since $\half+\frac14+\dots+\frac{1}{2^k} < 1$).
Then
\begin{itemize}
  \ii We make $k$ classes for the sums between $\frac{\sigma(A)}{2^k}$
  and $\sigma(A)$; this handles every set
  which has any element in $\{a_1, \dots, a_k\}$.
  \ii We make $n-k$ classes via induction hypothesis
  on $\{a_{k+1}, \dots, a_n\}$.
\end{itemize}
This solves the problem.
\pagebreak

\subsection{USAMO 1996/3}
\textsl{Available online at \url{https://aops.com/community/p353052}.}
\begin{mdframed}[style=mdpurplebox,frametitle={Problem statement}]
Let $ABC$ be a triangle. Prove that there is a line $\ell$ (in the plane
of triangle $ABC$) such that the intersection of the interior of
triangle $ABC$ and the interior of its reflection $A'B'C'$ in $\ell$ has
area more than $\frac23$ the area of triangle $ABC$.
\end{mdframed}
All that's needed is:
\begin{claim*}
  If $ABC$ is a triangle where $\half < \frac{AB}{AC} < 1$, then
  the $\angle A$ bisector works.
\end{claim*}
\begin{proof}
  Let the $\angle A$-bisector meet $BC$ at $D$.
  The overlapped area is $2[ABD]$ and
  \[ \frac{[ABD]}{[ABC]} = \frac{BD}{BC} = \frac{AB}{AB+AC} \]
  by angle bisector theorem.
\end{proof}

In general, suppose $x < y < z$ are sides of a triangle.
Then $\half < \frac yz < 1$ by triangle inequality as needed.
\pagebreak

\section{Solutions to Day 2}
\subsection{USAMO 1996/4}
\textsl{Available online at \url{https://aops.com/community/p353058}.}
\begin{mdframed}[style=mdpurplebox,frametitle={Problem statement}]
An $n$-term sequence $(x_1, x_2, \dots, x_n)$
in which each term is either $0$ or $1$ is called a binary sequence of length $n$.
Let $a_n$ be the number of binary sequences of length $n$ containing
no three consecutive terms equal to $0$, $1$, $0$ in that order.
Let $b_n$ be the number of binary sequences of length $n$ that
contain no four consecutive terms equal to 0, 0, 1, 1 or 1, 1, 0, 0 in that order.
Prove that $b_{n+1} = 2a_n$ for all positive integers $n$.
\end{mdframed}
Consider the map from
sequences of the latter form to sequences of the first form by
\[ (y_1, \dots, y_{n+1}) \mapsto (y_1 + y_2, y_2 + y_3, \dots, y_n + y_{n+1}). \]
It is $2$-to-$1$. The end.
\pagebreak

\subsection{USAMO 1996/5}
\textsl{Available online at \url{https://aops.com/community/p2}.}
\begin{mdframed}[style=mdpurplebox,frametitle={Problem statement}]
Let $ABC$ be a triangle,
and $M$ an interior point such that
$\angle MAB=10^\circ$, $\angle MBA=20^\circ$,
$\angle MAC=40^\circ$ and $\angle MCA=30^\circ$.
Prove that the triangle is isosceles.
\end{mdframed}
Let $\theta = \angle MBC < 80\dg$.
By trig Ceva, we get
\[
  \frac{\sin 10\dg}{\sin 40\dg}
  \cdot
  \frac{\sin \theta}{\sin 20\dg}
  \cdot
  \frac{\sin 30\dg}{\sin (80\dg-\theta)}
  =
  1.
\]
This simplifies to
\[ \sin \theta = 4 \sin(80\dg- \theta) \sin 40\dg \cos 10\dg.\]
\begin{claim*}
  We have $\theta=60\dg$.
\end{claim*}
\begin{proof}
  The left-hand side is increasing in $\theta$ and the right-hand
  side is decreasing in $\theta$,
  so at most one value of $\theta$ works.
  But we also have
  \begin{align*}
    4\sin20\dg \sin40\dg \cos10\dg
    &= 2\left( \cos20\dg-\cos60\dg \right) \cos 10\dg \\
    &= 2\cos20\dg \cos 10\dg + \sin 80\dg \\
    &= (\cos 30\dg - \cos 10\dg) + \sin 80\dg = \cos 30\dg
  \end{align*}
  as desired.
\end{proof}
\pagebreak

\subsection{USAMO 1996/6}
\textsl{Available online at \url{https://aops.com/community/p353061}.}
\begin{mdframed}[style=mdpurplebox,frametitle={Problem statement}]
Determine with proof whether there is a subset $X \subseteq \ZZ$
with the following property: for any $n \in \ZZ$,
there is exactly one solution to $a+2b = n$, with $a,b \in X$.
\end{mdframed}
The idea is generating functions,
but extra care is required since exponents will be in $\ZZ$
rather than in $\ZZ_{\ge 0}$.

However, consider formally the limit
\[ f(x) = \prod_{k \ge 0} \left( 1 + x^{(-4)^k} \right). \]
For size reasons, this indeed converges formally to a power series,
in the sense that the coefficient of any $x^k$
is eventually zero or one for all partial sums.

We claim $X = \{ n : [x^n] f(x) = 1 \}$ works.

For a given $n$, we can truncate the sum at some large $N$
again for size reasons.
For convenience assume $N$ is even.
Now set
\[ f_N(x) = \prod_{k = 0}^N \left( 1 + x^{(-4)^k}  \right). \]
Next, we compute
\[
  f_N(x) f_N(x^2)
  = \frac{(1+x)(1+x^2) \dots ( 1+x^{2^{2N+1}} )}%
  {x^{2+8+\dots+2^{2N-1}}}
  = \dots + x^{-2} + x^{-1} + 1 + x + x^2 + \dots
\]
as desired.
\pagebreak


\end{document}
