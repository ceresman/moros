\documentclass[11pt]{scrartcl}
\usepackage[sexy]{evan}
\ihead{\footnotesize\textbf{\thetitle}}
\ohead{\footnotesize\theauthor}
\begin{document}

\title{USA TSTST 2016 Solutions}
\subtitle{United States of America --- TST Selection Test}
\date{58\ts{th} IMO 2017 Brazil and 6\ts{th} EGMO 2017 Switzerland}

\maketitle

\tableofcontents
\newpage

\addtocounter{section}{-1}
\section{Problems}
\begin{enumerate}[\bfseries 1.]
\item %% Problem 1
Let $A = A(x,y)$ and $B = B(x,y)$ be
two-variable polynomials with real coefficients.
Suppose that $A(x,y)/B(x,y)$ is a polynomial in $x$
for infinitely many values of $y$,
and a polynomial in $y$ for infinitely many values of $x$.
Prove that $B$ divides $A$, meaning there exists a third polynomial $C$
with real coefficients such that $A = B \cdot C$.

\item %% Problem 2
Let $ABC$ be a scalene triangle with orthocenter $H$ and circumcenter $O$
and denote by $M$, $N$ the midpoints of $\ol{AH}$, $\ol{BC}$.
Suppose the circle $\gamma$ with diameter $\ol{AH}$ meets
the circumcircle of $ABC$ at $G \neq A$,
and meets line $\ol{AN}$ at $Q \neq A$.
The tangent to $\gamma$ at $G$ meets line $OM$ at $P$.
Show that the circumcircles of $\triangle GNQ$
and $\triangle MBC$ intersect on $\ol{PN}$.

\item %% Problem 3
Decide whether or not there exists a nonconstant polynomial $Q(x)$
with integer coefficients with the following property:
for every positive integer $n > 2$, the numbers
\[ Q(0), \; Q(1), Q(2),  \; \dots, \; Q(n-1) \]
produce at most $0.499n$ distinct residues when taken modulo $n$.

\item %% Problem 4
Prove that if $n$ and $k$ are positive integers
satisfying $\varphi^k(n) = 1$, then $n \le 3^k$.
(Here $\varphi^k$ denotes $k$ applications of the Euler phi function.)

\item %% Problem 5
In the coordinate plane are finitely many \emph{walls},
which are disjoint line segments, none of which are parallel to either axis.
A bulldozer starts at an arbitrary point and moves in the $+x$ direction.
Every time it hits a wall, it turns at a right angle to its path,
away from the wall, and continues moving.
(Thus the bulldozer always moves parallel to the axes.)

Prove that it is impossible for the bulldozer
to hit both sides of every wall.

\item %% Problem 6
Let $ABC$ be a triangle with incenter $I$,
and whose incircle is tangent to $\ol{BC}$, $\ol{CA}$, $\ol{AB}$
at $D$, $E$, $F$, respectively.
Let $K$ be the foot of the altitude from $D$ to $\ol{EF}$.
Suppose that the circumcircle of $\triangle AIB$
meets the incircle at two distinct points $C_1$ and $C_2$,
while the circumcircle of $\triangle AIC$ meets the incircle
at two distinct points $B_1$ and $B_2$.
Prove that the radical axis of the circumcircles of
$\triangle BB_1B_2$ and $\triangle CC_1C_2$ passes through
the midpoint $M$ of $\ol{DK}$.

\end{enumerate}
\pagebreak

\section{Solutions to Day 1}
\subsection{TSTST 2016/1, proposed by Victor Wang}
\textsl{Available online at \url{https://aops.com/community/p6575197}.}
\begin{mdframed}[style=mdpurplebox,frametitle={Problem statement}]
Let $A = A(x,y)$ and $B = B(x,y)$ be
two-variable polynomials with real coefficients.
Suppose that $A(x,y)/B(x,y)$ is a polynomial in $x$
for infinitely many values of $y$,
and a polynomial in $y$ for infinitely many values of $x$.
Prove that $B$ divides $A$, meaning there exists a third polynomial $C$
with real coefficients such that $A = B \cdot C$.
\end{mdframed}
This is essentially an application of the division algorithm,
but the details require significant care.

First, we claim that $A/B$ can be written as a polynomial in $x$
whose coefficients are rational functions in $y$.
To see this, use the division algorithm to get
\[ A = Q \cdot B + R \qquad Q,R \in (\RR(y))[x] \]
where $Q$ and $R$ are polynomials in $x$
whose coefficients are rational functions in $y$,
and moreover $\deg_x B > \deg_x R$.

Now, we claim that $R \equiv 0$.
Indeed, we have by hypothesis that for infinitely many values of $y_0$
that $B(x,y_0)$ divides $A(x, y_0)$,
which means $B(x,y_0) \mid R(x,y_0)$ as polynomials in $\RR[x]$.
Now, we have $\deg_x B(x,y_0) > \deg_x R(x,y_0)$
outside of finitely many values of $y_0$ (but not all of them!);
this means for infinitely many $y_0$ we have $R(x,y_0) \equiv 0$.
So each coefficient of $x^i$ (in $\RR(y)$)
has infinitely many roots, hence is a zero polynomial.

Consequently, we are able to write $A / B = F(x,y) / M(y)$
where $F \in \RR[x,y]$ and $M \in \RR[y]$ are each polynomials.
Repeating the same argument now gives
\[ \frac AB = \frac{F(x,y)}{M(y)} = \frac{G(x,y)}{N(x)}. \]
Now, by unique factorization of polynomials in $\RR[x,y]$,
we can discuss GCD's.
So, we tacitly assume $\gcd(F,M) = \gcd(G,N) = (1)$.
Also, we obviously have $\gcd(M,N) = (1)$.
But $F \cdot N = G \cdot M$, so $M \mid F \cdot N$,
thus we conclude $M$ is the constant polynomial.
This implies the result.

\begin{remark*} This fact does not generalize to arbitrary functions that are separately polynomial:
see e.g. \url{http://aops.com/community/c6h523650p2978180}. \end{remark*}
\pagebreak

\subsection{TSTST 2016/2, proposed by Evan Chen}
\textsl{Available online at \url{https://aops.com/community/p6575204}.}
\begin{mdframed}[style=mdpurplebox,frametitle={Problem statement}]
Let $ABC$ be a scalene triangle with orthocenter $H$ and circumcenter $O$
and denote by $M$, $N$ the midpoints of $\ol{AH}$, $\ol{BC}$.
Suppose the circle $\gamma$ with diameter $\ol{AH}$ meets
the circumcircle of $ABC$ at $G \neq A$,
and meets line $\ol{AN}$ at $Q \neq A$.
The tangent to $\gamma$ at $G$ meets line $OM$ at $P$.
Show that the circumcircles of $\triangle GNQ$
and $\triangle MBC$ intersect on $\ol{PN}$.
\end{mdframed}
We present two solutions,
one using essentially only power of a point,
and the other more involved.

\paragraph{First solution (found by contestants).}
Denote by $\triangle DEF$ the orthic triangle.
Observe $\ol{PA}$ and $\ol{PG}$ are tangents to $\gamma$,
since $\ol{OM}$ is the perpendicular bisector of $\ol{AG}$.
Also note that $\ol{AG}$, $\ol{EF}$, $\ol{BC}$ are concurrent at some point $R$
by radical axis on $(ABC)$, $\gamma$, $(BC)$.

Now, consider circles $(PAGM)$, $(MFDNE)$, and $(MBC)$.
They intersect at $M$ but have radical center $R$, so are coaxial;
assume they meet again at $T \in \ol{RM}$, say.
Then $\angle PTM$ and $\angle MTN$ are both right angles, hence $T$ lies on $\ol{PN}$.

Finally $H$ is the orthocenter of $\triangle ARN$,
and thus the circle with diameter $\ol{RN}$ passes through $G$, $Q$, $N$.

\begin{center}
\begin{asy}
size(11cm);
pair A = dir(120);
pair B = dir(210);
pair C = dir(330);
pair N = midpoint(B--C);
pair H = orthocenter(A, B, C);
pair M = midpoint(A--H);
pair O = circumcenter(A, B, C);
pair P = extension(O, M, A, B-C+A);
pair T = foot(M, P, N);
pair D = foot(A, B, C);
pair E = foot(B, C, A);
pair F = foot(C, A, B);

filldraw(A--B--C--cycle, opacity(0.1)+lightblue, blue);
draw(unitcircle, blue);

draw(A--D, heavygreen);
draw(B--E, heavygreen);
draw(C--F, heavygreen);
draw(E--F, heavygreen);
draw(B--M--C, lightred);
// filldraw(circumcircle(M, B, C), opacity(0.05)+lightred, red);

filldraw(circumcircle(D, E, F), opacity(0.05)+lightred, red);
draw(O--P--N, heavycyan);
draw(A--P, heavycyan);

filldraw(circumcircle(A, E, F), opacity(0.1)+lightgreen, heavygreen);
pair Q = foot(H, A, N);
pair G = -A+2*foot(A, M, P);

pair R = extension(E, F, B, C);
draw(R--F, heavygreen);
draw(R--B, blue);
draw(R--Q, heavygreen);
draw(A--R, heavygreen);
draw(A--N, blue);
draw(P--G, heavycyan);
draw(G--N, blue);
draw(M--R, heavygreen);
filldraw(circumcircle(R, G, T), opacity(0.05)+lightred, red);

dot("$A$", A, dir(A));
dot("$B$", B, dir(B));
dot("$C$", C, dir(C));
dot("$N$", N, dir(N));
dot("$H$", H, dir(H));
dot("$M$", M, dir(M));
dot("$O$", O, dir(225));
dot("$P$", P, dir(P));
dot("$T$", T, dir(T));
dot("$D$", D, dir(D));
dot("$E$", E, dir(E));
dot("$F$", F, dir(F));
dot("$Q$", Q, dir(Q));
dot("$G$", G, dir(G));
dot("$R$", R, dir(R));

/* TSQ Source:

!size(11cm);
A = dir 120
B = dir 210
C = dir 330
N = midpoint B--C
H = orthocenter A B C
M = midpoint A--H
O = circumcenter A B C R225
P = extension O M A B-C+A
T = foot M P N
D = foot A B C
E = foot B C A
F = foot C A B

A--B--C--cycle 0.1 lightblue / blue
unitcircle blue

A--D heavygreen
B--E heavygreen
C--F heavygreen
E--F heavygreen
B--M--C lightred
// circumcircle M B C 0.05 lightred / red

circumcircle D E F 0.1 lightred / red
O--P--N heavycyan
A--P heavycyan

circumcircle A E F 0.1 lightgreen / heavygreen
Q = foot H A N
G = -A+2*foot A M P

R = extension E F B C
R--F heavygreen
R--B blue
R--Q heavygreen
A--R heavygreen
A--N blue
P--G heavycyan
G--N blue
M--R heavygreen
circumcircle R G T 0.1 lightred / red

*/
\end{asy}
\end{center}

\paragraph{Alternate solution (by proposer).}
Let $L$ be diametrically opposite $A$ on the circumcircle.
Denote by $\triangle DEF$ the orthic triangle.
Let $X = \ol{AH} \cap \ol{EF}$.
Finally, let $T$ be the second intersection of $(MFDNE)$ and $(MBC)$.

\begin{center}
\begin{asy}
size(11cm);
pair A = dir(120);
pair B = dir(210);
pair C = dir(330);
pair N = midpoint(B--C);
pair H = orthocenter(A, B, C);
pair M = midpoint(A--H);
pair O = circumcenter(A, B, C);
pair P = extension(O, M, A, B-C+A);
pair X = extension(P, N, A, H);
pair T = foot(M, P, N);
pair D = foot(A, B, C);
pair E = foot(B, C, A);
pair F = foot(C, A, B);
pair L = -A;
pair K = extension(H, N, A, P);

filldraw(A--B--C--cycle, opacity(0.1)+lightblue, blue);
filldraw(B--L--C--cycle, opacity(0.1)+lightblue, blue);
draw(unitcircle, blue);
draw(A--L, blue);

draw(A--D, heavygreen);
draw(B--E, heavygreen);
draw(C--F, heavygreen);
draw(E--F, heavygreen);
draw(B--M--C, lightred);
filldraw(circumcircle(M, B, C), opacity(0.05)+lightred, red);

filldraw(circumcircle(D, E, F), opacity(0.05)+lightred, red);
draw(A--K--L, heavycyan);
draw(O--P--N, heavycyan);

filldraw(circumcircle(A, E, F), opacity(0.1)+lightgreen, heavygreen);
pair Q = foot(H, A, N);
pair G = -A+2*foot(A, M, P);

pair R = extension(E, F, B, C);
draw(R--F, heavygreen);
draw(R--B, blue);
draw(R--Q, heavygreen);
draw(A--R, heavygreen);
draw(A--N, blue);
draw(P--G, heavycyan);
draw(M--R, heavygreen);
filldraw(circumcircle(R, G, T), opacity(0.05)+lightred, red);

dot("$A$", A, dir(A));
dot("$B$", B, dir(B));
dot("$C$", C, dir(C));
dot("$N$", N, dir(N));
dot("$H$", H, dir(H));
dot("$M$", M, dir(M));
dot("$O$", O, dir(225));
dot("$P$", P, dir(P));
dot("$X$", X, dir(X));
dot("$T$", T, dir(T));
dot("$D$", D, dir(D));
dot("$E$", E, dir(E));
dot("$F$", F, dir(F));
dot("$L$", L, dir(L));
dot("$K$", K, dir(K));
dot("$Q$", Q, dir(Q));
dot("$G$", G, dir(G));
dot("$R$", R, dir(R));

/* TSQ Source:

!size(11cm);
A = dir 120
B = dir 210
C = dir 330
N = midpoint B--C
H = orthocenter A B C
M = midpoint A--H
O = circumcenter A B C R225
P = extension O M A B-C+A
X = extension P N A H
T = foot M P N
D = foot A B C
E = foot B C A
F = foot C A B
L = -A
K = extension H N A P

A--B--C--cycle 0.1 lightblue / blue
B--L--C--cycle 0.1 lightblue / blue
unitcircle blue
A--L blue

A--D heavygreen
B--E heavygreen
C--F heavygreen
E--F heavygreen
B--M--C lightred
circumcircle M B C 0.05 lightred / red

circumcircle D E F 0.05 lightred / red
A--K--L heavycyan
O--P--N heavycyan

circumcircle A E F 0.1 lightgreen / heavygreen
Q = foot H A N
G = -A+2*foot A M P

R = extension E F B C
R--F heavygreen
R--B blue
R--Q heavygreen
A--R heavygreen
A--N blue
P--G heavycyan
M--R heavygreen
circumcircle R G T 0.05 lightred / red

*/
\end{asy}
\end{center}

We begin with a few easy observations.
First, points $H$, $G$, $N$, $L$ are collinear and $\angle AGL = 90\dg$.
Also, $Q$ is the foot from $H$ to $\ol{AN}$.
Consequently, lines $AG$, $EF$, $HQ$, $BC$, $TM$
concur at a point $R$ (radical axis).
Moreover, we already know $\angle MTN = 90\dg$.
This implies $T$ lies on the circle with diameter $\ol{RN}$,
which is exactly the circumcircle of $\triangle GQN$.

Note by Brokard's Theorem on $AFHE$, the point $X$ is the orthocenter of $\triangle MBC$.
But $\angle MTN = 90\dg$ already, and $N$ is the midpoint of $\ol{BC}$.
Consequently, points $T$, $X$, $N$ are collinear.

Finally, we claim $P$, $X$, $N$ are collinear, which solves the problem.
Note $P = \ol{GG} \cap \ol{AA}$.
Set $K = \ol{HNL} \cap \ol{AP}$.
Then by noting \[ -1 = (D,X;A,H) \overset{N}{=} (\infty, \ol{NX} \cap \ol{AK}; A, K) \]
we see that $\ol{NX}$ bisects segment $\ol{AK}$, as desired.
(A more projective finish is to show that $\ol{PXN}$ is the polar of $R$ to $\gamma$).

\begin{remark*}
The original problem proposal reads as follows:
\begin{quote}
Let $ABC$ be a triangle with orthocenter $H$ and circumcenter $O$
and denote by $M$, $N$ the midpoints of $\ol{AH}$, $\ol{BC}$.
Suppose ray $OM$ meets the line parallel to $\ol{BC}$ through $A$ at $P$.
Prove that the line through the circumcenter of $\triangle MBC$
and the midpoint of $\ol{OH}$ is parallel to $\ol{NP}$.
\end{quote}
The points $G$ and $Q$ were added to the picture later
to prevent the problem from being immediate by coordinates.
\end{remark*}
\pagebreak

\subsection{TSTST 2016/3, proposed by Yang Liu}
\textsl{Available online at \url{https://aops.com/community/p6575217}.}
\begin{mdframed}[style=mdpurplebox,frametitle={Problem statement}]
Decide whether or not there exists a nonconstant polynomial $Q(x)$
with integer coefficients with the following property:
for every positive integer $n > 2$, the numbers
\[ Q(0), \; Q(1), Q(2),  \; \dots, \; Q(n-1) \]
produce at most $0.499n$ distinct residues when taken modulo $n$.
\end{mdframed}
We claim that
\[ Q(x) = 420(x^2-1)^2 \]
works.
Clearly, it suffices to prove the result when $n=4$ and when $n$ is an odd prime $p$.
The case $n=4$ is trivial, so assume now $n=p$ is an odd prime.

First, we prove the following easy claim.
\begin{claim*}
  For any odd prime $p$, there are at least $\frac12(p-3)$
  values of $a$ for which $\left( \frac{1-a^2}{p} \right) = +1$.
\end{claim*}
\begin{proof}
  Note that if $k \neq 0$, $k \neq \pm 1$, $k^2 \neq -1$, then $a = 2(k+k\inv)$ works.
  Also $a=0$ works.
\end{proof}

Let $F(x) = (x^2-1)^2$.
The range of $F$ modulo $p$ is contained within the $\half(p+1)$ quadratic residues modulo $p$.
On the other hand, if for some $t$ neither of $1 \pm t$ is a quadratic residue,
then $t^2$ is omitted from the range of $F$ as well.
Call such a value of $t$ \emph{useful}, and let $N$ be the number of useful residues.
We aim to show $N \ge \frac14 p - 2$.

We compute a lower bound on the number $N$ of useful $t$ by writing
\begin{align*}
  N &= \frac{1}{4} \left( \sum_t \left[ \left(1 - \left(\frac{1-t}{p} \right) \right)
    \left(1 - \left(\frac{1+t}{p} \right) \right) \right]
    - \left( 1- \left( \frac2p \right) \right)
    - \left( 1- \left( \frac{-2}p \right) \right)
  \right) \\
  &\ge \frac{1}{4} \sum_t \left[ \left(1 - \left(\frac{1-t}{p} \right) \right)
    \left(1 - \left(\frac{1+t}{p} \right) \right) \right] -1 \\
  &= \frac{1}{4} \left(p + \sum_t \left(\frac{1-t^2}{p} \right) \right) -1 \\
  &\ge \frac14 \left( p + (+1) \cdot \tfrac12(p-3) + 0 \cdot 2
    + (-1) \cdot ( (p-2) - \tfrac12(p-3)) \right) - 1 \\
  &\ge \frac14 \left( p - 5 \right).
\end{align*}

Thus, the range of $F$ has size at most
\[ \half(p+1) - \half N \le \frac38(p+3). \]
This is less than $0.499p$ for any $p \ge 11$.

\begin{remark*}
  In fact, the computation above is essentially an equality.
  There are only two points where terms are dropped:
  one, when $p \equiv 3 \pmod 4$ there are no $k^2 = -1$ in the lemma,
  and secondly, the terms $1-\left( 2/p \right)$ and $1-\left( -2/p \right)$
  are dropped in the initial estimate for $N$.
  With suitable modifications, one can show that in fact,
  the range of $F$ is exactly equal to
  \[
    \half(p+1) - \half N =
    \begin{cases}
      \frac18(3p+5) & p \equiv 1 \pmod 8 \\
      \frac18(3p+7) & p \equiv 3 \pmod 8 \\
      \frac18(3p+9) & p \equiv 5 \pmod 8 \\
      \frac18(3p+3) & p \equiv 7 \pmod 8.
    \end{cases}
  \]
\end{remark*}
\pagebreak

\section{Solutions to Day 2}
\subsection{TSTST 2016/4, proposed by Linus Hamilton}
\textsl{Available online at \url{https://aops.com/community/p6580534}.}
\begin{mdframed}[style=mdpurplebox,frametitle={Problem statement}]
Prove that if $n$ and $k$ are positive integers
satisfying $\varphi^k(n) = 1$, then $n \le 3^k$.
(Here $\varphi^k$ denotes $k$ applications of the Euler phi function.)
\end{mdframed}
The main observation is that the exponent of $2$ decreases
by at most $1$ with each application of $\varphi$.
This will give us the desired estimate.

Define the \emph{weight} function $w$ on positive integers as follows:
it satisfies
\begin{align*}
  w(ab) &= w(a)+w(b); \\
  w(2) &= 1; \quad\text{and} \\
  w(p) &= w(p-1) \quad \text{for any prime $p > 2$}.
\end{align*}
By induction, we see that $w(n)$ counts the powers of $2$
that are produced as $\varphi$ is repeatedly applied to $n$.
In particular, $k \ge w(n)$.

From $w(2) = 1$,
it suffices to prove that $w(p) \ge \log_3 p$ for every $p > 2$.
We use strong induction and note that
\[
 w(p) = w(2) + w\left( \frac{p-1}{2} \right)
 \ge 1 + \log_3(p-1) - \log_3 2 \ge \log_3 p
\]
for any $p > 2$.
This solves the problem.

\begin{remark*}
One can motivate this solution through small cases $2^x 3^y$
like $2^x 17^w$, $2^x 3^y 7^z$, $2^x 11^t$.

Moreover, the stronger bound \[ n \le 2 \cdot 3^{k-1} \]
is true and best possible.
\end{remark*}
\pagebreak

\subsection{TSTST 2016/5, proposed by Linus Hamilton, Cynthia Stoner}
\textsl{Available online at \url{https://aops.com/community/p6580545}.}
\begin{mdframed}[style=mdpurplebox,frametitle={Problem statement}]
In the coordinate plane are finitely many \emph{walls},
which are disjoint line segments, none of which are parallel to either axis.
A bulldozer starts at an arbitrary point and moves in the $+x$ direction.
Every time it hits a wall, it turns at a right angle to its path,
away from the wall, and continues moving.
(Thus the bulldozer always moves parallel to the axes.)

Prove that it is impossible for the bulldozer
to hit both sides of every wall.
\end{mdframed}
We say a wall $v$ is \emph{above} another wall $w$ if some point on
$v$ is directly above a point on $w$.
(This relation is anti-symmetric, as walls do not intersect).

The critical claim is as follows:
\begin{claim*}
  There exists a lowest wall,
  i.e.\ a wall not above any other walls.
\end{claim*}
\begin{proof}
  Assume not.
  Then we get a directed cycle of some length $n \ge 3$:
  it's possible to construct a series of points $P_i$, $Q_i$,
  for $i = 1, \dots, n$ (indices modulo $n$), such that
  the point $Q_i$ is directly above $P_{i+1}$ for each $i$,
  the segment $\ol{Q_i P_{i+1}}$ does not intersect any wall in its interior,
  and finally each segment $\ol{P_i Q_i}$ is contained inside a wall.
  This gives us a broken line on $2n$ vertices which is not self-intersecting.

  Now consider the leftmost vertical segment $\ol{Q_i P_{i+1}}$
  and the rightmost vertical segment $\ol{Q_j P_{j+1}}$.
  The broken line gives a path from $P_{i+1}$ to $Q_j$,
  as well as a path from $P_{j+1}$ to $Q_i$.
  These clearly must intersect, contradiction.
\end{proof}
\begin{remark*}
  This claim is Iran TST 2010.
\end{remark*}

Thus if the bulldozer eventually moves upwards indefinitely,
it may never hit the bottom side of the lowest wall.
Similarly, if the bulldozer eventually moves downwards indefinitely,
it may never hit the upper side of the highest wall.
\pagebreak

\subsection{TSTST 2016/6, proposed by Danielle Wang}
\textsl{Available online at \url{https://aops.com/community/p6580553}.}
\begin{mdframed}[style=mdpurplebox,frametitle={Problem statement}]
Let $ABC$ be a triangle with incenter $I$,
and whose incircle is tangent to $\ol{BC}$, $\ol{CA}$, $\ol{AB}$
at $D$, $E$, $F$, respectively.
Let $K$ be the foot of the altitude from $D$ to $\ol{EF}$.
Suppose that the circumcircle of $\triangle AIB$
meets the incircle at two distinct points $C_1$ and $C_2$,
while the circumcircle of $\triangle AIC$ meets the incircle
at two distinct points $B_1$ and $B_2$.
Prove that the radical axis of the circumcircles of
$\triangle BB_1B_2$ and $\triangle CC_1C_2$ passes through
the midpoint $M$ of $\ol{DK}$.
\end{mdframed}
\paragraph{First solution (Allen Liu).}
Let $X$, $Y$, $Z$ be midpoints of $EF$, $FD$, $DE$,
and let $G$ be the Gergonne point.
By radical axis on $(AEIF)$, $(DEF)$, $(AIC)$
we see that $B_1$, $X$, $B_2$ are collinear.
Likewise, $B_1$, $Z$, $B_2$ are collinear,
so lines $B_1B_2$ and $XZ$ coincide.
Similarly, lines $C_1C_2$ and $XY$ coincide.
In particular lines $B_1B_2$ and $C_1C_2$ meet at $X$.

\begin{center}
\begin{asy}
size(12cm);

pair A = dir(70);
pair B = dir(240);
pair C = dir(300);
pair I = incenter(A, B, C);
pair D = foot(I, B, C);
pair E = foot(I, C, A);
pair F = foot(I, A, B);
pair G = extension(A, D, B, E);
filldraw(A--B--C--cycle, opacity(0.1)+lightcyan, blue);
draw(D--E--F--cycle, blue);
draw(incircle(A, B, C), blue);
draw(A--D, blue);
draw(B--E, blue);
draw(C--F, blue);

pair Q = extension(G, G+F-D, E, F);
pair S = extension(G, G+F-D, E, D);
pair R = extension(G, G+D-E, F, D);
pair U = extension(G, G+D-E, F, E);

pair X = midpoint(E--F);
pair Y = midpoint(F--D);
pair Z = midpoint(D--E);

pair Qp = 2*Q-G;
pair Rp = 2*R-G;
pair Sp = 2*S-G;
pair Up = 2*U-G;

pair B_1 = OP(circumcircle(B, Qp, Sp), incircle(A, B, C));
pair B_2 = IP(circumcircle(B, Qp, Sp), incircle(A, B, C));
pair C_1 = OP(circumcircle(C, Rp, Up), incircle(A, B, C));
pair C_2 = IP(circumcircle(C, Rp, Up), incircle(A, B, C));


draw(circumcircle(B, B_1, B_2), red);
draw(circumcircle(C, C_1, C_2), red);

pair V = extension(X, Y, A, B);
pair W = extension(X, Z, A, C);

pair T = extension(B, W, C, V);
draw(X--T, yellow);
draw(unitcircle, blue);
draw(B--W, heavygreen);
draw(C--V, heavygreen);
draw(B--V, blue);
draw(C--W, blue);

draw(C_2--V, heavygreen);
draw(B_2--W, heavygreen);

dot("$A$", A, dir(A));
dot("$B$", B, dir(B));
dot("$C$", C, dir(C));
dot("$D$", D, dir(D));
dot("$E$", E, dir(E));
dot("$F$", F, dir(170));
dot("$G$", G, dir(G));
dot("$X$", X, dir(110));
dot("$Y$", Y, dir(350));
dot("$Z$", Z, dir(Z));
dot("$B_1$", B_1, dir(B_1));
dot("$B_2$", B_2, dir(100));
dot("$C_1$", C_1, dir(C_1));
dot("$C_2$", C_2, dir(40));
dot("$V$", V, dir(V));
dot("$W$", W, dir(W));
dot("$T$", T, dir(T));

/* Source generated by TSQ */
\end{asy}
\end{center}

Note $G$ is the symmedian point of $DEF$, so
it is well-known that $XG$ passes through the midpoint of $DK$.
So we just have to prove $G$ lies on the radical axis.


First, note that $\triangle DEF$ is the cevian triangle
of the Gergonne point $G$.
Set $V = \ol{XY} \cap \ol{AB}$, $W = \ol{XZ} \cap \ol{AC}$,
and $T = \ol{BW} \cap \ol{CV}$.

We begin with the following completely projective claim.
\begin{claim*}
  The points $X$, $G$, $T$ are collinear.
\end{claim*}
\begin{proof}
  It suffices to view $\triangle XYZ$ as any
  cevian triangle of $\triangle DEF$
  (which is likewise any cevian triangle of $\triangle ABC$).
  Then
  \begin{itemize}
    \ii By Cevian Nest on $\triangle ABC$,
    it follows that $\ol{AX}$, $\ol{BY}$, $\ol{CZ}$ are concurrent.
    \ii Hence $\triangle BYV$ and $\triangle CZW$ are perspective.
    \ii Hence $\triangle BZW$ and $\triangle CYV$ are perspective too.
    \ii Hence we deduce by Desargues theorem that $T$, $X$,
    and $\ol{BZ} \cap \ol{CY}$ are collinear.
    \ii Finally, the Cevian Nest theorem applied on $\triangle GBC$
    (which has cevian triangles $\triangle DFE$, $\triangle XZY$)
    we deduce $G$, $X$, and $\ol{BZ} \cap \ol{CY}$, proving the claim.
  \end{itemize}
  One could also proceed by using
  barycentric coordinates on $\triangle DEF$.
\end{proof}
\begin{remark*}
  [Eric Shen]
  The first four bullets can be replaced
  by non-projective means:
  one can check that $\ol{BZ} \cap \ol{CY}$ is the radical center
  of $(BIC)$, $(BB_1B_2)$, $(CC_1C_2)$
  and therefore it lies on line $\ol{XT}$.
\end{remark*}


Now, we contend point $V$ is the radical center $(CC_1C_2)$, $(ABC)$ and $(DEF)$.
To see this, let $V' = \ol{ED} \cap \ol{AB}$;
then $(FV';AB)$ is harmonic, and $V$ is the midpoint of $\ol{FV'}$,
and thus $VA \cdot VB = VF^2 = VC_1 \cdot VC_2$.

So in fact $\ol{CV}$ is the radical axis of $(ABC)$ and $(CC_1C_2)$.

Similarly, $\ol{BW}$ is the radical axis of $(ABC)$ and $(BB_1B_2)$.
Thus $T$ is the radical center of
$(ABC)$, $(BB_1B_2)$, $(CC_1C_2)$.

This completes the proof, as now $\ol{XT}$ is the desired radical axis.

\paragraph{Second solution (Evan Chen).}
Let $X$, $Y$, $Z$ be midpoints of $EF$, $FD$, $DE$,
and let $G$ be the Gergonne point.
By radical axis on $(AEIF)$, $(DEF)$, $(AIC)$
we see that $B_1$, $X$, $B_2$ are collinear.
Likewise, $B_1$, $Z$, $B_2$ are collinear,
so lines $B_1B_2$ and $XZ$ coincide.
Similarly, lines $C_1C_2$ and $XY$ coincide.
In particular lines $B_1B_2$ and $C_1C_2$ meet at $X$.

Note $G$ is the symmedian point of $DEF$, so
it is well-known that $XG$ passes through the midpoint of $DK$.
So we just have to prove $G$ lies on the radical axis.

\begin{center}
 \begin{asy}
  size(14cm);

  pair A = dir(50);
  pair B = dir(220);
  pair C = dir(320);
  pair I = incenter(A, B, C);
  pair D = foot(I, B, C);
  pair E = foot(I, C, A);
  pair F = foot(I, A, B);
  pair G = extension(A, D, B, E);
  filldraw(A--B--C--cycle, opacity(0.1)+lightcyan, blue);
  draw(D--E--F--cycle, blue);
  draw(incircle(A, B, C), blue);
  draw(A--D, blue);
  draw(B--E, blue);
  draw(C--F, blue);

  pair P = extension(G, G+E-F, D, F);
  pair T = extension(G, G+E-F, D, E);
  pair Q = extension(G, G+F-D, E, F);
  pair S = extension(G, G+F-D, E, D);
  pair R = extension(G, G+D-E, F, D);
  pair U = extension(G, G+D-E, F, E);

  pair X = midpoint(E--F);
  pair Y = midpoint(F--D);
  pair Z = midpoint(D--E);

  pair Pp = 2*P-G;
  pair Qp = 2*Q-G;
  pair Rp = 2*R-G;
  pair Sp = 2*S-G;
  pair Tp = 2*T-G;
  pair Up = 2*U-G;

  draw(Pp--Tp, heavygreen);
  draw(Rp--Up, heavygreen);
  draw(Qp--Sp, heavygreen);

  pair V = extension(B, X, Sp, Z);
  pair W = extension(B, R, Qp, X);

  pair B_1 = OP(circumcircle(B, Qp, Sp), incircle(A, B, C));
  pair B_2 = IP(circumcircle(B, Qp, Sp), incircle(A, B, C));
  pair C_1 = OP(circumcircle(C, Rp, Up), incircle(A, B, C));
  pair C_2 = IP(circumcircle(C, Rp, Up), incircle(A, B, C));

  pair K = foot(D, E, F);
  pair M = midpoint(D--K);
  draw(D--K, blue);

  draw(C_1--C_2, red);
  draw(B_1--B_2, red);
  filldraw(circumcircle(P, Q, R), opacity(0.1)+lightgreen, green+1.4);
  filldraw(circumcircle(Pp, Qp, Rp), opacity(0.1)+lightgreen, green);
  draw(circumcircle(Sp, Q, P), orange);
  draw(circumcircle(B, B_1, B_2), red);
  draw(circumcircle(C, C_1, C_2), red);

  draw(X--M, yellow);
  draw(B--V, orange);
  draw(B--W, orange);

  dot("$A$", A, dir(A));
  dot("$B$", B, dir(B));
  dot("$C$", C, dir(C));
  dot("$I$", I, dir(I));
  dot("$D$", D, dir(D));
  dot("$E$", E, dir(E));
  dot("$F$", F, dir(F));
  dot("$G$", G, dir(G));
  dot("$P$", P, dir(200));
  dot("$T$", T, dir(100));
  dot("$Q$", Q, dir(70));
  dot("$S$", S, dir(260));
  dot("$R$", R, dir(250));
  dot("$U$", U, dir(U));
  dot("$X$", X, dir(X));
  dot("$Y$", Y, dir(350));
  dot("$Z$", Z, dir(Z));
  dot("$P'$", Pp, dir(Pp));
  dot("$Q'$", Qp, dir(Qp));
  dot("$R'$", Rp, dir(240));
  dot("$S'$", Sp, dir(Sp));
  dot("$T'$", Tp, dir(Tp));
  dot("$U'$", Up, dir(60));
  dot("$V$", V, dir(V));
  dot("$W$", W, dir(W));
  dot("$B_1$", B_1, dir(B_1));
  dot("$B_2$", B_2, dir(B_2));
  dot("$C_1$", C_1, dir(C_1));
  dot("$C_2$", C_2, dir(C_2));
  dot("$K$", K, dir(200));
  dot("$M$", M, dir(M));
 \end{asy}
\end{center}

Construct parallelograms $GPFQ$, $GRDS$, $GTUE$
such that $P,R \in DF$, $S,T \in DE$, $Q,U \in EF$.
As $FG$ bisects $PQ$ and is isogonal to $FZ$,
we find $PQED$, hence $PQRU$, is cyclic.
Repeating the same logic and noticing $PR$, $ST$, $QU$ not concurrent,
all six points $PQRSTU$ are cyclic.
Moreover, since $PQ$ bisects $GF$, we see that
a dilation with factor $2$ at $G$ sends $PQ$ to $P', Q' \in AB$, say,
with $F$ the midpoint of $P'Q'$.
Define $R', S' \in BC$ similarly now and $T', U' \in CA$.

Note that $EQPDS'$ is in cyclic too,
as $\dang DS'Q = \dang DRS = \dang DEF$.
By homothety through $B$, points $B$, $P$, $X$ are collinear;
assume they meet $(EQPDS')$ again at $V$.
Thus $EVQPDS'$ is cyclic, and now
\[ \dang BVS' = \dang PVS' = \dang PQS = \dang PTS = \dang FED = \dang XEZ = \dang XVZ \]
hence $V$ lies on $(BQ'S')$.

Since $FB \parallel QP$, we get $EVFB$ is cyclic too,
so $XV \cdot XB = XE \cdot XF$ now;
thus $X$ lies on the radical axis of $(BS'Q')$ and $(DEF)$.
By the same argument with $W \in BZ$, we get $Z$ lies on the radical axis too.
Thus the radical axis of $(BS'Q')$ and $(DEF)$ must be line $XZ$,
which coincides with $B_1B_2$; so $(BB_1B_2) = (BS'Q')$.

Analogously, $(CC_1C_2) = (CR'U')$.
Since $G = Q'S' \cap R'U'$, we need only prove that $Q'R'S'U'$ is cyclic.
But $QRSU$ is cyclic, so we are done.

The circle $(PQRSTU)$ is called the \emph{Lemoine circle} of $ABC$.
\pagebreak


\end{document}
