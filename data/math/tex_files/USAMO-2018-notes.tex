% © Evan Chen
% Downloaded from https://web.evanchen.cc/

\documentclass[11pt]{scrartcl}
\usepackage[sexy]{evan}
\ihead{\footnotesize\textbf{\thetitle}}
\ohead{\footnotesize\href{http://web.evanchen.cc}{\ttfamily web.evanchen.cc},
    updated \today}
\title{USAMO 2018 Solution Notes}
\date{\today}

\begin{document}

\maketitle

\begin{abstract}
This is a compilation of solutions
for the 2018 USAMO.
Some of the solutions are my own work,
but many are from the official solutions provided by the organizers
(for which they hold any copyrights),
and others were found by users on the Art of Problem Solving forums.

These notes will tend to be a bit more advanced and terse than the ``official''
solutions from the organizers.
In particular, if a theorem or technique is not known to beginners
but is still considered ``standard'', then I often prefer to
use this theory anyways, rather than try to work around or conceal it.
For example, in geometry problems I typically use directed angles
without further comment, rather than awkwardly work around configuration issues.
Similarly, sentences like ``let $\mathbb{R}$ denote the set of real numbers''
are typically omitted entirely.

Corrections and comments are welcome!
\end{abstract}

\tableofcontents
\newpage

\addtocounter{section}{-1}
\section{Problems}
\begin{enumerate}[\bfseries 1.]
\item %% Problem 1
Let $a$, $b$, $c$ be positive real numbers such that $a+b+c = 4\sqrt[3]{abc}$.
Prove that
\[ 2(ab+bc+ca) + 4 \min (a^2, b^2, c^2) \ge a^2 + b^2 + c^2. \]

\item %% Problem 2
Find all functions $f \colon (0,\infty) \to (0,\infty)$ such that
\[
  f\left( x+\frac 1y \right)
  + f\left( y+\frac 1z \right)
  + f\left( z+\frac 1x \right)
  = 1
\]
for all $x,y,z > 0$ with $xyz = 1$.

\item %% Problem 3
Let $n \ge 2$ be an integer, and let $\{a_1, \dots, a_m\}$ denote
the $m = \varphi(n)$ integers less than $n$ and relatively prime to $n$.
Assume that every prime divisor of $m$ also divides $n$.
Prove that $m$ divides $a_1^k + \dots + a_m^k$ for every positive
integer $k$.

\item %% Problem 4
Let $p$ be a prime, and let $a_1$, \dots, $a_p$ be integers.
Show that there exists an integer $k$ such that the numbers
\[ a_1 + k, \; a_2 + 2k, \; \dots, \; a_p + pk \]
produce at least $\half p$ distinct remainders upon division by $p$.

\item %% Problem 5
Let $ABCD$ be a convex cyclic quadrilateral with
$E = \ol{AC} \cap \ol{BD}$, $F = \ol{AB} \cap \ol{CD}$,
$G = \ol{DA} \cap \ol{BC}$.
The circumcircle of $\triangle ABE$
intersects line $CB$ at $B$ and $P$,
and the circumcircle of $\triangle ADE$
intersects line $CD$ at $D$ and $Q$.
Assume $C$, $B$, $P$, $G$
and $C$, $Q$, $D$, $F$ are collinear in that order.
Let $M = \ol{FP} \cap \ol{GQ}$.
Prove that $\angle MAC = 90\dg$.

\item %% Problem 6
Let $a_n$ be the number of permutations $(x_1, \dots, x_n)$ of $(1, \dots, n)$
such that the ratios $x_k / k$ are all distinct.
Prove that $a_n$ is odd for all $n \ge 1$.

\end{enumerate}
\pagebreak

\section{Solutions to Day 1}
\subsection{USAMO 2018/1, proposed by Titu Andreescu}
\textsl{Available online at \url{https://aops.com/community/p10226140}.}
\begin{mdframed}[style=mdpurplebox,frametitle={Problem statement}]
Let $a$, $b$, $c$ be positive real numbers such that $a+b+c = 4\sqrt[3]{abc}$.
Prove that
\[ 2(ab+bc+ca) + 4 \min (a^2, b^2, c^2) \ge a^2 + b^2 + c^2. \]
\end{mdframed}
WLOG let $c = \min(a,b,c) = 1$ by scaling.
The given inequality becomes equivalent to
\[ 4ab + 2a + 2b + 3 \ge (a+b)^2 \qquad \forall a+b = 4(ab)^{1/3}-1. \]
Now, let $t = (ab)^{1/3}$ and eliminate $a+b$ using the condition, to get
\[ 4t^3 + 2(4t-1) + 3 \ge (4t-1)^2
  \iff 0 \le 4t^3 - 16t^2 + 16t = 4t(t-2)^2 \]
which solves the problem.

Equality occurs only if $t=2$,
meaning $ab = 8$ and $a+b=7$, which gives
\[ \{a,b\} = \left\{ \frac{7 \pm \sqrt{17}}{2} \right\} \]
with the assumption $c = 1$.
Scaling gives the curve of equality cases.
\pagebreak

\subsection{USAMO 2018/2, proposed by Titu Andreescu, Nikolai Nikolov}
\textsl{Available online at \url{https://aops.com/community/p10226145}.}
\begin{mdframed}[style=mdpurplebox,frametitle={Problem statement}]
Find all functions $f \colon (0,\infty) \to (0,\infty)$ such that
\[
  f\left( x+\frac 1y \right)
  + f\left( y+\frac 1z \right)
  + f\left( z+\frac 1x \right)
  = 1
\]
for all $x,y,z > 0$ with $xyz = 1$.
\end{mdframed}
The main part of the problem is to show all solutions are linear.
As always, let $x=b/c$, $y=c/a$, $z=a/b$
(classical inequality trick).
Then the problem becomes
\[ \sum_{\text{cyc}} f\left( \frac{b+c}{a} \right) = 1. \]
Let $f(t) = g(\frac{1}{t+1})$,
equivalently $g(s) = f(1/s-1)$.
Thus $g \colon (0,1) \to (0,1)$
which satisfies $\sum_{\text{cyc}} g\left( \frac{a}{a+b+c} \right) = 1$,
or equivalently
\[ \boxed{g(a) + g(b) + g(c) = 1} \qquad \forall a+b+c=1. \]

The rest of the solution is dedicated to solving this
equivalent functional equation in $g$.
It is a lot of technical details and I will only outline them
(with apologies to the contestants who didn't have that luxury).

\begin{claim*}
  The function $g$ is linear.
\end{claim*}
\begin{proof}
  This takes several steps, all of which are technical.
  We begin by proving $g$ is linear over $[1/8, 3/8]$.
  \begin{itemize}
    \ii First, whenever $a+b \le 1$ we have
    \[ 1 - g(1-(a+b)) = g(a) + g(b) = 2 g\left( \frac{a+b}{2} \right). \]
    Hence $g$ obeys Jensen's functional equation over $(0,1/2)$.
    \ii Define $h \colon [0,1] \to \RR$
    by $h(t) = g(\frac{2t+1}{8}) - (1-t) \cdot g(1/8) - t \cdot g(3/8)$,
    then $h$ satisfies Jensen's functional equation too over $[0,1]$.
    We have also arranged that $h(0) = h(1) = 0$, hence $h(1/2) = 0$ as well.
    \ii Since
    \[ h(t) = h(t) + h(1/2) = 2h(t/2+1/4)
      = h(t+1/2) + h(0) = h(t+1/2) \]
    for any $t < 1/2$, we find $h$ is periodic modulo $1/2$.
    It follows one can extend $\wt h$ by
    \[ \wt h \colon \RR \to \RR \qquad\text{by}\qquad
      \wt h(t) = h(t - \left\lfloor t \right\rfloor) \]
    and still satisfy Jensen's functional equation.
    Because $\wt h(0) = 0$, it's well-known this implies $\wt h$ is additive
    (because $\wt h(x+y) = 2\wt h\left( (x+y)/2 \right) = \wt h(x) + \wt h(y)$
    for any real numbers $x$ to $y$).
  \end{itemize}
  But $\wt h$ is bounded below on $[0,1]$ since $g \ge 0$,
  and since $\wt h$ is also additive,
  it follows (well-known) that $\wt h$ is linear.
  Thus $h$ is the zero function.
  So, the function $g$ is linear over $[1/8,3/8]$;
  thus we may write $g(x) = kx + \ell$, valid for $1/8 \le x \le 3/8$.

  Since $3g(1/3) = 1$, it follows $k + 3\ell = 1$.

  For $0 < x < 1/8$ we have
  $g(x) = 2g(0.15) - g(0.3-x) = 2(0.15k+\ell) - (k(0.3-x)+\ell) = kx+\ell$,
  so $g$ is linear over $(0,3/8)$ as well.
  Finally, for $3/8 < x < 1$, we use the given equation
  \[
    1 = g\left( \frac{1-x}{2} \right)
    + g\left( \frac{1-x}{2} \right)
    + g(x)
    \implies g(x) = 1 - 2\left( k \cdot \frac{1-x}{2} + \ell \right)
    = kx+\ell
  \]
  since $\frac{1-x}{2} < \frac{5}{16} < \frac{3}{8}$.
  Thus $g$ is linear over all.
\end{proof}

Putting this back in,
we deduce that $g(x) = kx + \frac{1-k}{3}$ for some $k \in [-1/2,1]$,
and so \[ f(x) = \frac{k}{x+1} + \frac{1-k}{3} \]
for some $k \in [-1/2,1]$.
All such functions work.
\pagebreak

\subsection{USAMO 2018/3, proposed by Ivan Borsenco}
\textsl{Available online at \url{https://aops.com/community/p10226222}.}
\begin{mdframed}[style=mdpurplebox,frametitle={Problem statement}]
Let $n \ge 2$ be an integer, and let $\{a_1, \dots, a_m\}$ denote
the $m = \varphi(n)$ integers less than $n$ and relatively prime to $n$.
Assume that every prime divisor of $m$ also divides $n$.
Prove that $m$ divides $a_1^k + \dots + a_m^k$ for every positive
integer $k$.
\end{mdframed}
For brevity, given any $n$,
we let $A(n) = \left\{ 1 \le x \le n, \gcd(x,n) = 1 \right\}$
(thus $|A(n)| = \varphi(n)$).
Also, let $S(n,k) = \sum_{a \in A(n)} a^k$.

We will prove the stronger statement
(which eliminates the hypothesis on $n$).
\begin{claim*}
  Let $n \ge 2$ be arbitrary (and $k \ge 0$).
  If $p \mid n$, then \[ \nu_p (\varphi(n)) \le \nu_p(S(n,k)). \]
\end{claim*}

We start with the special case where $n$ is a prime power.
\begin{lemma*}
  Let $p$ be prime, $e \ge 1$, $k \ge 0$.
  We always have
  \[ S(p^e, k) = \sum_{x \in A(p^e)} x^k \equiv 0 \pmod{p^{e-1}}.  \]
\end{lemma*}
\begin{proof}
  For $p$ odd, this follows by taking a primitive root $g$ modulo $p^{e}$.
  We will have
  \[ S(p^e, k) \equiv 1 + g^k + g^{2k} + \dots + g^{(\varphi(p^e)-1)k}
    \equiv \frac{g^{\varphi(p^e) k} - 1}{g^k - 1}. \]
  If $p-1 \nmid k$, then the denominator is not divisible by $p$
  and hence the entire expression is $0 \pmod{p^e}$.
  In the other case where $p-1 \mid k$,
  since $\nu_p(\varphi(p^e)) = e-1$,
  the exponent lifting lemma implies
  \[ \nu_p\left( (g^k)^{\varphi(p^e)} - 1 \right)
    = \nu_p(g^k-1) + (e-1) \]
  and so the conclusion is true here too.

  In the annoying case $p = 2$, the proof is broken into two cases:
  for $k$ odd it follows by pairing $x$ with $2^e-x$
  and when $k$ is even one can take $5$ as a generator
  of all the quadratic residues as in the $p > 2$ case.
\end{proof}
\begin{corollary*}
  We have $\nu_p(1^k + \dots + t^k) \ge \nu_p(t) - 1$ for any $k$,
  $t$, $p$.
\end{corollary*}
\begin{proof}
  Assume $p \mid t$.
  Handle the terms in that sum divisible by $p$ (by induction)
  and apply the lemma a bunch of times.
\end{proof}

Now the idea is to add primes $q$ one at a time to $n$,
starting from the base case $n = p^e$.
So, formally we proceed by induction on the number
of prime divisors of $n$.
We'll also assume $k \ge 1$ in what follows
since the base case $k=0$ is easy.

\begin{itemize}
\ii First, suppose we want to go from $n$ to $nq$
where $q \nmid n$.
In that case $\varphi(nq)$ gained $\nu_p(q-1)$ factors of $p$
and then we need to show $\nu_p(S(nq,k)) \ge \nu_p(\varphi(n)) + \nu_p(q-1)$.
The trick is to write
\[ A(nq) = \left\{ a + nh \mid a \in A(n)
    \text{ and } h = 0,\dots,q-1 \right\}
    \setminus qA(n)  \]
and then expand using binomial theorem:
\begin{align*}
  S(nq, k) &=
  \sum_{a \in A(n)} \sum_{h=0}^{q-1} (a+nh)^k
  - \sum_{a \in A(n)} (qa)^k \\
  &= -q^k S(n,k) + \sum_{a \in A(n)} \sum_{h=0}^{q-1} \sum_{j=0}^k
    \left[ \binom kj a^{k-j} n^j h^j \right] \\
  &= -q^k S(n,k) +
    \sum_{j=0}^k \left[ \binom kj n^j \left( \sum_{a \in A(n)} a^{k-j} \right)
    \left( \sum_{h=0}^{q-1} h^j \right) \right] \\
  &= -q^k S(n,k) + \sum_{j=0}^k \left[ \binom kj n^j S(n,k-j)
    \left( \sum_{h=1}^{q-1} h^j \right) \right] \\
  &= (q-q^k) S(n,k)
    + \sum_{j=1}^k \left[ \binom kj n^j S(n,k-j)
    \left( \sum_{h=1}^{q-1} h^j \right) \right].
\end{align*}
We claim every term here has enough powers of $p$.
For the first term, $S(n,k)$
has at least $\nu_p(\varphi(n))$ factors of $p$;
and we have the $q-q^k$ multiplier out there.
For the other terms, we apply induction to $S(n,k-j)$;
moreover $\sum_{h=1}^{q-1} h^j$
has at least $\nu_p(q-1)-1$ factors of $p$ by corollary,
and we get one more factor of $p$ (at least) from $n^j$.

\ii On the other hand, if $q$ already divides $n$,
then this time
\[ A(nq) = \left\{ a + nh \mid a \in A(n)
    \text{ and } h = 0,\dots,q-1 \right\}.  \]
and we have no additional burden of $p$ to deal with;
the same calculation gives
\[ S(nq,k) = qS(n,k)
  + \sum_{j=1}^k \left[ \binom kj n^j S(n,k-j)
  \left( \sum_{h=1}^{q-1} h^j \right) \right].
\]
which certainly has enough factors of $p$ already.
\end{itemize}

\begin{remark*}
  A curious bit about the problem is
  that $\nu_p(\varphi(n))$ can exceed $\nu_p(n)$,
  and so it is not true that the residues
  of $A(n)$ are well-behaved modulo $\varphi(n)$.

  As an example, let $n = 2 \cdot 3 \cdot 7 \cdot 13 = 546$,
  so $m = \varphi(n) = 1 \cdot 2 \cdot 6 \cdot 12 = 144$.
  Then $A(n)$ contains $26$ elements which are $1 \bmod 9$
  and $23$ elements which are $4 \bmod 9$.
\end{remark*}

\begin{remark*}
  The converse of the problem is true too
  (but asking both parts would make this too long for exam).
\end{remark*}
\pagebreak

\section{Solutions to Day 2}
\subsection{USAMO 2018/4, proposed by Ankan Bhattacharya}
\textsl{Available online at \url{https://aops.com/community/p10232389}.}
\begin{mdframed}[style=mdpurplebox,frametitle={Problem statement}]
Let $p$ be a prime, and let $a_1$, \dots, $a_p$ be integers.
Show that there exists an integer $k$ such that the numbers
\[ a_1 + k, \; a_2 + 2k, \; \dots, \; a_p + pk \]
produce at least $\half p$ distinct remainders upon division by $p$.
\end{mdframed}
For each $k = 0, \dots, p-1$ let $G_k$ be the graph
on $\{1, \dots, p\}$ where we join $\{i,j\}$ if and only if
\[ a_i + ik \equiv a_j + jk \pmod p
  \iff k \equiv - \frac{a_i - a_j}{i-j} \pmod p. \]
So we want a graph $G_k$ with at least $\half p$ connected components.

However, each $\{i,j\}$ appears in exactly one graph $G_k$,
so some graph has at most $\frac 1p \binom p2 = \half(p-1)$ edges
(by ``pigeonhole'').
This graph has at least $\half(p+1)$ connected components, as desired.

\begin{remark*}
  Here is an example for $p=5$ showing equality can occur:
  \[
    \begin{bmatrix}
      0 & 0 & 3 & 4 & 3 \\
      0 & 1 & 0 & 2 & 2 \\
      0 & 2 & 2 & 0 & 1 \\
      0 & 3 & 4 & 3 & 0 \\
      0 & 4 & 1 & 1 & 4
    \end{bmatrix}.
  \]
  Ankan Bhattacharya points out more generally
  that $a_i = i^2$ is sharp in general.
\end{remark*}
\pagebreak

\subsection{USAMO 2018/5, proposed by Kada Williams}
\textsl{Available online at \url{https://aops.com/community/p10232392}.}
\begin{mdframed}[style=mdpurplebox,frametitle={Problem statement}]
Let $ABCD$ be a convex cyclic quadrilateral with
$E = \ol{AC} \cap \ol{BD}$, $F = \ol{AB} \cap \ol{CD}$,
$G = \ol{DA} \cap \ol{BC}$.
The circumcircle of $\triangle ABE$
intersects line $CB$ at $B$ and $P$,
and the circumcircle of $\triangle ADE$
intersects line $CD$ at $D$ and $Q$.
Assume $C$, $B$, $P$, $G$
and $C$, $Q$, $D$, $F$ are collinear in that order.
Let $M = \ol{FP} \cap \ol{GQ}$.
Prove that $\angle MAC = 90\dg$.
\end{mdframed}
We present three general routes.
(The second route, using the fact that $\ol{AC}$
is an angle bisector, has many possible variations.)

\paragraph{First solution (Miquel points).}
This is indeed a Miquel point problem,
but the main idea is to focus on the self-intersecting
cyclic quadrilateral $PBQD$ as the key player,
rather than on the given $ABCD$.

Indeed, we will prove that $A$ is its Miquel point;
this follows from the following two claims.
\begin{claim*}
  The self-intersecting quadrilateral $PQDB$ is cyclic.
\end{claim*}
\begin{proof}
  By power of a point from $C$:
  $CQ \cdot CD = CA \cdot CE = CB \cdot CP$.
\end{proof}
\begin{claim*}
  Point $E$ lies on line $PQ$.
\end{claim*}
\begin{proof}
  $\dang AEP = \dang ABP = \dang ABC = \dang ADC
  = \dang ADQ = \dang AEQ$.
\end{proof}

\begin{center}
\begin{asy}
pair P = dir(240);
pair B = dir(300);
pair Q = dir(20);
pair D = dir(50);
pair E = extension(Q, P, B, D);
pair C = extension(D, Q, B, P);
pair H = extension(D, P, B, Q);
pair A = foot(H, E, C);
filldraw(unitcircle, opacity(0.1)+lightcyan, lightblue);

pair G = extension(D, A, B, C);
pair F = extension(B, A, C, D);

filldraw(circumcircle(P, A, E), opacity(0.1)+green, heavygreen);
filldraw(circumcircle(Q, A, E), opacity(0.1)+green, heavygreen);
filldraw(circumcircle(A, B, C), opacity(0.05)+yellow, heavygreen+dotted);
draw(B--F--C, grey);
draw(D--G--C, grey);
draw(P--H--B, blue);
draw(P--B--Q--D--cycle, blue);
draw(A--C, grey);
draw(P--Q, blue);
draw(B--D, blue);

pair M = extension(P, F, G, Q);
draw(F--P, heavycyan);
draw(G--Q, heavycyan);

dot("$P$", P, dir(P));
dot("$B$", B, dir(B));
dot("$Q$", Q, dir(Q));
dot("$D$", D, dir(D));
dot("$E$", E, dir(E));
dot("$C$", C, dir(C));
dot("$H$", H, dir(H));
dot("$A$", A, dir(E-C));
dot("$G$", G, dir(G));
dot("$F$", F, dir(F));
dot("$M$", M, dir(B-M));

/* TSQ Source:

P = dir 240
B = dir 300
Q = dir 20
D = dir 50
E = extension Q P B D
C = extension D Q B P
H = extension D P B Q
A = foot H E C RE-C
unitcircle 0.1 lightcyan / lightblue

G = extension D A B C
F = extension B A C D

circumcircle P A E 0.1 green / heavygreen
circumcircle Q A E 0.1 green / heavygreen
circumcircle A B C 0.05 yellow / heavygreen dotted
B--F--C grey
D--G--C grey
P--H--B blue
P--B--Q--D--cycle blue
A--C grey
P--Q blue
B--D blue

M = extension P F G Q
F--P heavycyan
G--Q heavycyan

*/
\end{asy}
\end{center}

To finish, let $H = \ol{PD} \cap \ol{BQ}$.
By properties of the Miquel point, we have $A$ is the
foot from $H$ to $\ol{CE}$.
But also, points $M$, $A$, $H$ are collinear
by Pappus theorem on $\ol{BPG}$ and $\ol{DQF}$, as desired.

\paragraph{Second solution (projective).}
We start with a synthetic observation.
\begin{claim*}
  The line $\ol{AC}$ bisects $\angle PAD$ and $\angle BAQ$.
\end{claim*}
\begin{proof}
  Angle chase:
  $\dang PAC = \dang PAE = \dang PBE = \dang CBD = \dang CAD$.
\end{proof}

There are three ways to finish from here:
\begin{itemize}
  \ii (Michael Kural) Suppose the external bisector
  of $\angle PAD$ and $\angle BAQ$
  meet lines $BC$ and $DC$ at $X$ and $Y$.
  Then \[ -1 = (GP;XC) = (FD;YC) \]
  which is enough to imply that $\ol{XY}$, $\ol{GQ}$, $\ol{PF}$
  are concurrent (by so-called prism lemma).

  \ii (Daniel Liu) Alternatively,
  apply the dual Desargues involution theorem
  to complete quadrilateral $GQFPCM$, through the point $A$.
  This gives that an involutive pairing of
  \[ (AC,AM) \; (AP,AQ) \; (AG, AF). \]
  This is easier to see if we project it onto the
  line $\ell$ through $C$ perpendicular to $\ol{AC}$;
  if we let $P'$, $Q'$, $G'$, $F'$ be the images of the last four lines,
  we find the involution coincides with negative inversion through $C$
  with power $\sqrt{CP' \cdot CQ'}$
  which implies that $\ol{AM} \cap \ell$ is an infinity point, as desired.

  \ii (Kada Williams) The official solution instead shows the external angle
  bisector by a long trig calculation.
\end{itemize}

\paragraph{Third solution (inversion, Andrew Wu).}
Noting that $CE \cdot CA = CP \cdot CB = CQ \cdot CD$,
we perform an inversion at $C$ swapping these pairs of points.
The point $G$ is mapped to a point $G^\ast$ ray $CB$
for which $QEG^\ast C$ is cyclic, but then
\[ \dang CG^\ast E = \dang CQE = \dang CQP = \dang DBC = \dang CBE \]
and so we conclude $EB = EG^\ast$.
Similarly, $ED = EF^\ast$.

Finally, $M^\ast = (CG^\ast D) \cap (CF^\ast B) \neq C$,
and we wish to show that $\angle EM^\ast C = 90\dg$.

\begin{center}
\begin{asy}
pair D = dir(100);
pair B = dir(220);
pair C = dir(320);
pair E = 0.3*D+0.7*B;
pair K = foot(E, B, C);
pair L = foot(E, D, C);
pair Gs = 2*K-B;
pair Fs = 2*L-D;
pair Ms = (D*B-Fs*Gs)/(D+B-Fs-Gs);

filldraw(circumcircle(Gs, D, C), opacity(0.05)+yellow, orange);
filldraw(circumcircle(Fs, B, C), opacity(0.05)+yellow, orange);

fill(B--C--D--cycle, opacity(0.1)+yellow);
draw(B--C--D, orange);
draw(B--D, red);
draw(Gs--E--K, red);
draw(Fs--E--L, red);
filldraw(circumcircle(E, L, K), opacity(0.1)+red, lightred+dashed);

dot("$D$", D, dir(D));
dot("$B$", B, dir(B));
dot("$C$", C, dir(C));
dot("$E$", E, dir(E));
dot("$K$", K, dir(K));
dot("$L$", L, dir(L));
dot("$G^\ast$", Gs, dir(Gs));
dot("$F^\ast$", Fs, dir(30));
dot("$M^\ast$", Ms, dir(Ms));

/* TSQ Source:

D = dir 100
B = dir 220
C = dir 320
E = 0.3*D+0.7*B
K = foot E B C
L = foot E D C
G* = 2*K-B
F* = 2*L-D R30
M* = (D*B-Fs*Gs)/(D+B-Fs-Gs)

circumcircle Gs D C 0.05 yellow / orange
circumcircle Fs B C 0.05 yellow / orange

!fill(B--C--D--cycle, opacity(0.1)+yellow);
B--C--D orange
B--D red
Gs--E--K red
Fs--E--L red
circumcircle E L K 0.1 red / lightred dashed

*/
\end{asy}
\end{center}

Note that $M^\ast$ is the center of the
spiral similarity sending $\ol{BG^\ast}$ to $\ol{F^\ast D}$.
Hence it also maps the midpoint $K$ of $BG^\ast$
to the midpoint $L$ of $\ol{F^\ast E}$.
Consequently, $M^\ast$ lies on the circumcircle $KLC$ as well.
In other words, $ELCKM^\ast$ is a cyclic pentagon
with circumdiameter $\ol{CE}$, as desired.
\pagebreak

\subsection{USAMO 2018/6, proposed by Richard Stong}
\textsl{Available online at \url{https://aops.com/community/p10232388}.}
\begin{mdframed}[style=mdpurplebox,frametitle={Problem statement}]
Let $a_n$ be the number of permutations $(x_1, \dots, x_n)$ of $(1, \dots, n)$
such that the ratios $x_k / k$ are all distinct.
Prove that $a_n$ is odd for all $n \ge 1$.
\end{mdframed}
This is the official solution; the proof has two main insights.

The first idea:
\begin{lemma*}
  If a permutation $x$ works, so does the inverse permutation.
\end{lemma*}

Thus it suffices to consider permutations $x$
in which all cycles have length at most $2$.
Of course, there can be at most one fixed point
(since that gives the ratio $1$),
and hence exactly one if $n$ is odd, none if $n$ is even.

We consider the graph $K_n$ such that the edge
$\{i,j\}$ is labeled with $i / j$ (for $i < j$).
The permutations we're considering are then equivalent to maximal
matchings of this $K_n$.
We call such a matching \emph{fantastic} if it has an
all of distinct edge labels.

\bigskip

Now the second insight is that if edges $ab$ and $cd$
have the same label for $a<b$ and $c<d$,
then so do edges $ac$ and $bd$.
Thus:
\begin{definition*}
  Given a matching $\mathcal M$ as above we say the \emph{neighbors}
  of $\mathcal M$ are those other matchings obtained as follows:
  for each label $\ell$,
  we take some disjoint pairs of edges (possibly none)
  with label $\ell$ and apply the above switching operation
  (in which we replace $ab$ and $cd$ with $ac$ and $bd$).
\end{definition*}
This neighborship relation is reflexive,
and most importantly it is \emph{symmetric}
(because one can simply reverse the moves).
But it is not transitive.

The second observation is that:
\begin{claim*}
  The matching $\mathcal M$ has an odd number of neighbors
  (including itself) if and only if it is fantastic.
\end{claim*}
\begin{proof}
  Consider the label $\ell$,
  and assume it appears $n_\ell \ge 1$ times.

  If we pick $k$ disjoint pairs and swap them,
  the number of ways to do this is
  $\binom{n_\ell}{2k} (2k-1)!!$, and so the total
  number of ways to perform operations on the edges
  labeled $\ell$ is
  \[ \sum_k \binom{n_\ell}{2k} (2k-1)!!
    \equiv \sum_k \binom{n_\ell}{2k}
    = 2^{n_\ell-1} \pmod 2.
  \]
  This is even if and only if $n_\ell > 1$.

  Finally, note that the number of neighbors
  of $\mathcal M$ is the product across all $\ell$ of the above.
  So it is odd if and only if each factor is odd,
  if and only if $n_\ell = 1$ for every $\ell$.
\end{proof}

To finish, consider a huge simple graph $\Gamma$ on all the
maximal matchings, with edge relations given by neighbor relation
(we don't consider vertices to be connected to themselves).
Observe that:
\begin{itemize}
  \ii Fantastic matchings correspond to isolated vertices
  (of degree zero, with no other neighbors) of $\Gamma$.
  \ii The rest of the vertices of $\Gamma$ have odd degrees
  (one less than the neighbor count)
  \ii The graph $\Gamma$ has an even number of vertices of odd degree
  (this is true for any simple graph, see ``handshake lemma'').
  \ii The number of vertices of $\Gamma$ is odd,
  namely $\left( 2\left\lceil n/2 \right\rceil - 1 \right)!!$.
\end{itemize}
This concludes the proof.
\pagebreak


\end{document}
