% © Evan Chen
% Downloaded from https://web.evanchen.cc/

\documentclass[11pt]{scrartcl}
\usepackage[sexy]{evan}
\ihead{\footnotesize\textbf{\thetitle}}
\ohead{\footnotesize\href{http://web.evanchen.cc}{\ttfamily web.evanchen.cc},
    updated \today}
\title{USAMO 2002 Solution Notes}
\date{\today}

\begin{document}

\maketitle

\begin{abstract}
This is a compilation of solutions
for the 2002 USAMO.
Some of the solutions are my own work,
but many are from the official solutions provided by the organizers
(for which they hold any copyrights),
and others were found by users on the Art of Problem Solving forums.

These notes will tend to be a bit more advanced and terse than the ``official''
solutions from the organizers.
In particular, if a theorem or technique is not known to beginners
but is still considered ``standard'', then I often prefer to
use this theory anyways, rather than try to work around or conceal it.
For example, in geometry problems I typically use directed angles
without further comment, rather than awkwardly work around configuration issues.
Similarly, sentences like ``let $\mathbb{R}$ denote the set of real numbers''
are typically omitted entirely.

Corrections and comments are welcome!
\end{abstract}

\tableofcontents
\newpage

\addtocounter{section}{-1}
\section{Problems}
\begin{enumerate}[\bfseries 1.]
\item %% Problem 1
Let $S$ be a set with $2002$ elements,
and let $N$ be an integer with $0 \leq N \leq 2^{2002}$.
Prove that it is possible to color every subset of $S$
either black or white so that the following conditions hold:
\begin{enumerate}
  \ii[(a)] the union of any two white subsets is white;
  \ii[(b)] the union of any two black subsets is black;
  \ii[(c)] there are exactly $N$ white subsets.
\end{enumerate}

\item %% Problem 2
Let $ABC$ be a triangle such that
\[
  \left( \cot \frac{A}{2} \right)^2
  + \left( 2\cot \frac{B}{2} \right)^2
  + \left( 3\cot \frac{C}{2} \right)^2
  = \left( \frac{6s}{7r} \right)^2,
\]
where $s$ and $r$ denote its semiperimeter and its inradius, respectively.
Prove that triangle $ABC$ is similar to a triangle $T$ whose
side lengths are all positive integers
with no common divisors and determine these integers.

\item %% Problem 3
Prove that any monic polynomial (a polynomial with leading coefficient $1$)
of degree $n$ with real coefficients is the average
of two monic polynomials of degree $n$ with $n$ real roots.

\item %% Problem 4
Determine all functions $f \colon \RR \to \RR$ such that
\[ f(x^2 - y^2) = x f(x) - y f(y) \]
for all pairs of real numbers $x$ and $y$.

\item %% Problem 5
Let $a$, $b$ be integers greater than $2$.
Prove that there exists a positive integer $k$
and a finite sequence $n_1, n_2, \dots, n_k$ of
positive integers such that $n_1 = a$, $n_k = b$,
and $n_i n_{i+1}$ is divisible by $n_i + n_{i+1}$
for each $i$ ($1 \leq i < k$).

\item %% Problem 6
I have an $n \times n$ sheet of stamps, from which I've been asked
to tear out blocks of three adjacent stamps in a single row or column.
(I can only tear along the perforations separating adjacent stamps,
and each block must come out of the sheet in one piece.)
Let $b(n)$ be the smallest number of blocks I can tear out
and make it impossible to tear out any more blocks.
Prove that there are real constants $c$ and $d$ such that
\[ \frac{1}{7} n^2 - cn \leq b(n) \leq \frac{1}{5} n^2 + dn  \]
for all $n > 0$.

\end{enumerate}
\pagebreak

\section{Solutions to Day 1}
\subsection{USAMO 2002/1, proposed by Gabriel Carroll}
\textsl{Available online at \url{https://aops.com/community/p337845}.}
\begin{mdframed}[style=mdpurplebox,frametitle={Problem statement}]
Let $S$ be a set with $2002$ elements,
and let $N$ be an integer with $0 \leq N \leq 2^{2002}$.
Prove that it is possible to color every subset of $S$
either black or white so that the following conditions hold:
\begin{enumerate}
  \ii[(a)] the union of any two white subsets is white;
  \ii[(b)] the union of any two black subsets is black;
  \ii[(c)] there are exactly $N$ white subsets.
\end{enumerate}
\end{mdframed}
We will solve the problem with $2002$
replaced by an arbitrary integer $n \ge 0$.
In other words, we prove:
\begin{claim*}
  For any nonnegative integers $n$ and $N$ with $0 \le N \le 2^n$,
  it is possible to color the $2^n$ subsets
  of $\{1, \dots, n\}$ black and white
  satisfying the conditions of the problem.
\end{claim*}
The proof is by induction on $n$.
When $n = 1$ the problem is easy to do by hand,
so this gives us a base case.

For the inductive step, we divide into two cases:
\begin{itemize}
  \ii If $N \le 2^{n-1}$,
  then we take a coloring of subsets of $\{1, \dots, n-1\}$
  with $N$ white sets;
  then we color the remaining $2^{n-1}$ sets
  (which contain $n$) black.

  \ii If $N > 2^{n-1}$,
  then we take a coloring of subsets of $\{1, \dots, n-1\}$
  with $N - 2^{n-1}$ white sets;
  then we color the remaining $2^{n-1}$ sets
  (which contain $n$) white.
\end{itemize}
\pagebreak

\subsection{USAMO 2002/2}
\textsl{Available online at \url{https://aops.com/community/p337847}.}
\begin{mdframed}[style=mdpurplebox,frametitle={Problem statement}]
Let $ABC$ be a triangle such that
\[
  \left( \cot \frac{A}{2} \right)^2
  + \left( 2\cot \frac{B}{2} \right)^2
  + \left( 3\cot \frac{C}{2} \right)^2
  = \left( \frac{6s}{7r} \right)^2,
\]
where $s$ and $r$ denote its semiperimeter and its inradius, respectively.
Prove that triangle $ABC$ is similar to a triangle $T$ whose
side lengths are all positive integers
with no common divisors and determine these integers.
\end{mdframed}
Let $x = s-a$, $y = s-b$, $z = s-c$ in the usual fashion, then the equation reads
\[ x^2 + 4y^2 + 9z^2 = \left( \frac67(x+y+z) \right)^2. \]
However, by Cauchy-Schwarz, we have
\[ \left( 1 + \tfrac14 + \tfrac19 \right)\left( x^2 + 4y^2 + 9z^2 \right) \ge \left( x+y+z \right)^2  \]
with equality if and only if $1 : \tfrac12 : \tfrac13 = x : 2y : 3z$,
id est $x : y : z = 1 : \tfrac14 : \tfrac19 = 36 : 9 : 4$.
This is equivalent to $y+z : z+x : x+y = 13 : 40 : 45$.

%Wow.
%I was laughing at how un-geometric this problem looked,
%as it was the only geometry problem on the USAMO 2002.
%Then I actually tried it and realized it was not geometry.

\begin{remark*}
You can tell this is not a geometry problem
because you eliminate the cotangents right away to get an algebra problem\dots
and then you realize the problem claims that one equation
can determine three variables up to scaling,
at which point you realize it has to be an inequality
(otherwise degrees of freedom don't work).
So of course, Cauchy-Schwarz\dots
\end{remark*}
\pagebreak

\subsection{USAMO 2002/3}
\textsl{Available online at \url{https://aops.com/community/p337849}.}
\begin{mdframed}[style=mdpurplebox,frametitle={Problem statement}]
Prove that any monic polynomial (a polynomial with leading coefficient $1$)
of degree $n$ with real coefficients is the average
of two monic polynomials of degree $n$ with $n$ real roots.
\end{mdframed}
First,
\begin{lemma*}
  If $p$ is a monic polynomial of degree $n$,
  and $p(1)p(2) < 0$, $p(2)p(3) < 0$, \dots, $p(n-1)p(n) < 0$
  then $p$ has $n$ real roots.
\end{lemma*}
\begin{proof}
  The intermediate value theorem already guarantees
  the existence of $n-1$ real roots.

  The last root is obtained by considering cases on $n \pmod 2$.
  \begin{itemize}
  \ii If $n$ is even, then $p(1)$ and $p(n)$ have opposite sign,
  while we must have either
  \[ \lim_{x \to -\infty} p(x) = \lim_{x \to \infty} p(x) = \pm \infty \]
  so we get one more root.
  \ii The $n$ odd case is similar,
  with $p(1)$ and $p(n)$ now having the same sign,
  but $\lim_{x \to -\infty} p(x) = -\lim_{x \to \infty} p(x)$ instead. \qedhere
 \end{itemize}
\end{proof}

Let $f(n)$ be the monic polynomial and let
$M > 1000\max_{t=1, \dots, n} |f(t)|+1000$.
Then we may select reals $a_1, \dots, a_n$ and $b_1, \dots, b_n$
such that for each $k = 1, \dots, n$, we have
\begin{align*}
  a_k + b_k &= 2f(k) \\
  (-1)^k a_k & > M \\
  (-1)^{k+1} b_k & > M.
\end{align*}
We may interpolate monic polynomials $g$ and $h$ through the $a_k$ and $b_k$
(if the $a_k$, $b_k$ are selected ``generically'' from each other).
Then one can easily check $f = \tfrac12(g+h)$ works.
\pagebreak

\section{Solutions to Day 2}
\subsection{USAMO 2002/4}
\textsl{Available online at \url{https://aops.com/community/p337857}.}
\begin{mdframed}[style=mdpurplebox,frametitle={Problem statement}]
Determine all functions $f \colon \RR \to \RR$ such that
\[ f(x^2 - y^2) = x f(x) - y f(y) \]
for all pairs of real numbers $x$ and $y$.
\end{mdframed}
The answer is $f(x) = cx$, $c \in \RR$ (these obviously work).

First, by putting $x=0$ and $y=0$ respectively
we have \[ f(x^2) = xf(x) \quad\text{and}\quad f(-y^2) = -yf(y). \]
From this we deduce that $f$ is odd, in particular $f(0) = 0$.
Then, we can rewrite the given as $f(x^2-y^2) + f(y^2) = f(x^2)$.
Combined with the fact that $f$ is odd, we deduce that $f$ is additive
(i.e. $f(a+b)=f(a)+f(b)$).

\begin{remark*}
  [Philosophy]
  At this point we have $f(x^2) \equiv xf(x)$ and $f$ additive,
  and everything we have including the given equation
  is a direct corollary of these two.
  So it makes sense to only focus on these two conditions.
\end{remark*}

Then
\begin{align*}
  f( (x+1)^2 ) &= (x+1)f(x+1) \\
  \implies f(x^2) + 2f(x) + f(1)
  &= (x+1)f(x) + (x+1)f(1)
\end{align*}
which readily gives $f(x) = f(1)x$.
\pagebreak

\subsection{USAMO 2002/5, proposed by Gabriel Carroll}
\textsl{Available online at \url{https://aops.com/community/p337862}.}
\begin{mdframed}[style=mdpurplebox,frametitle={Problem statement}]
Let $a$, $b$ be integers greater than $2$.
Prove that there exists a positive integer $k$
and a finite sequence $n_1, n_2, \dots, n_k$ of
positive integers such that $n_1 = a$, $n_k = b$,
and $n_i n_{i+1}$ is divisible by $n_i + n_{i+1}$
for each $i$ ($1 \leq i < k$).
\end{mdframed}
Consider a graph $G$ on the vertex set $\{3, 4, \dots\}$
and with edges between $v$, $w$ if $v + w \mid vw$;
the problem is equivalent to showing that $G$ is connected.

First, note that $n$ is connected to $n(n-1)$, $n(n-1)(n-2)$, etc.\
up to $n!$.
But for $n > 2$, $n!$ is connected to $(n+1)!$ too:
\begin{itemize}
  \ii $n! \to (n+1)!$ if $n$ is even
  \ii $n! \to 2n! \to (n+1)!$ if $n$ is odd.
\end{itemize}
This concludes the problem.
\pagebreak

\subsection{USAMO 2002/6}
\textsl{Available online at \url{https://aops.com/community/p337852}.}
\begin{mdframed}[style=mdpurplebox,frametitle={Problem statement}]
I have an $n \times n$ sheet of stamps, from which I've been asked
to tear out blocks of three adjacent stamps in a single row or column.
(I can only tear along the perforations separating adjacent stamps,
and each block must come out of the sheet in one piece.)
Let $b(n)$ be the smallest number of blocks I can tear out
and make it impossible to tear out any more blocks.
Prove that there are real constants $c$ and $d$ such that
\[ \frac{1}{7} n^2 - cn \leq b(n) \leq \frac{1}{5} n^2 + dn  \]
for all $n > 0$.
\end{mdframed}
For the lower bound: there are $2n(n-2)$ places one could put a block.
Note that each block eliminates at most $14$ such places.

For the upper bound, the construction of $\frac15$ is easy to build.
Here is an illustration of one possible construction
for $n=9$ which generalizes readily,
using only vertical blocks.
\[
\begin{bmatrix}
  A &   & E &   & I & L &   & P &   \\
  A &   & E & G &   & L &   & P & R \\
  A & C &   & G &   & L & N &   & R \\
    & C &   & G & J &   & N &   & R \\
    & C & F &   & J &   & N & Q &   \\
  B &   & F &   & J & M &   & Q &   \\
  B &   & F & H &   & M &   & Q & S \\
  B & D &   & H &   & M & O &   & S \\
    & D &   & H & K &   & O &   & S \\
\end{bmatrix}
\]

Actually, for the lower bound,
one may improve $1/7$ to $1/6$.
Count the number $A$ of pairs of adjacent squares
one of which is torn out and the other which is not:
\begin{itemize}
  \ii For every deleted block, there are eight neighboring squares,
  at least two on each long edge which have been deleted too.
  Hence $N \le 6b(n)$.
  \ii For every block still alive and not on the border, there are
  four neighboring squares, and clearly at least two are deleted.
  Hence $N \ge 2\left( (n-2)^2 - 3b(n) \right)$.
\end{itemize}
Collating these solves the problem.
\pagebreak


\end{document}
