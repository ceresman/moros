% © Evan Chen
% Downloaded from https://web.evanchen.cc/

\documentclass[11pt]{scrartcl}
\usepackage[sexy]{evan}
\ihead{\footnotesize\textbf{\thetitle}}
\ohead{\footnotesize\href{http://web.evanchen.cc}{\ttfamily web.evanchen.cc},
    updated \today}
\title{JMO 2011 Solution Notes}
\date{\today}

\begin{document}

\maketitle

\begin{abstract}
This is a compilation of solutions
for the 2011 JMO.
Some of the solutions are my own work,
but many are from the official solutions provided by the organizers
(for which they hold any copyrights),
and others were found by users on the Art of Problem Solving forums.

These notes will tend to be a bit more advanced and terse than the ``official''
solutions from the organizers.
In particular, if a theorem or technique is not known to beginners
but is still considered ``standard'', then I often prefer to
use this theory anyways, rather than try to work around or conceal it.
For example, in geometry problems I typically use directed angles
without further comment, rather than awkwardly work around configuration issues.
Similarly, sentences like ``let $\mathbb{R}$ denote the set of real numbers''
are typically omitted entirely.

Corrections and comments are welcome!
\end{abstract}

\tableofcontents
\newpage

\addtocounter{section}{-1}
\section{Problems}
\begin{enumerate}[\bfseries 1.]
\item %% Problem 1
Find all positive integers $n$ such that $2^n+12^n+2011^n$ is a perfect square.

\item %% Problem 2
Let $a$, $b$, $c$ be positive real numbers
such that $a^2+b^2+c^2+(a+b+c)^2 \le 4$. Prove that
\[ \frac{ab+1}{(a+b)^2}
  + \frac{bc+1}{(b+c)^2}
  + \frac{ca+1}{(c+a)^2} \ge 3. \]

\item %% Problem 3
For a point $P = (a,a^2)$ in the coordinate plane,
let $\ell(P)$ denote the line passing through $P$ with slope $2a$.
Consider the set of triangles with vertices of the form
$P_1 = (a_1, a_1^2)$, $P_2 = (a_2, a_2^2)$, $P_3 = (a_3, a_3^2)$,
such that the intersection of the lines $\ell(P_1), \ell(P_2), \ell(P_3)$
form an equilateral triangle $\Delta$.
Find the locus of the center of $\Delta$ as $P_1P_2P_3$ ranges over all such triangles.

\item %% Problem 4
A \emph{word} is defined as any finite string of letters.
A word is a \emph{palindrome} if it reads the same backwards and forwards.
Let a sequence of words $W_0, W_1, W_2, \dots$ be defined as follows:
$W_0 = a, W_1 = b$, and for $n \ge 2$,
$W_n$ is the word formed by writing $W_{n-2}$ followed by $W_{n-1}$.
Prove that for any $n \ge 1$, the word formed by writing
$W_1, W_2, W_3, \dots, W_n$ in succession is a palindrome.

\item %% Problem 5
Points $A,B,C,D,E$ lie on a circle $\omega$
and point $P$ lies outside the circle.
The given points are such that
(i) lines $PB$ and $PD$ are tangent to $\omega$,
(ii) $P, A, C$ are collinear,
and (iii) $\ol{DE} \parallel \ol{AC}$.
Prove that $\ol{BE}$ bisects $\ol{AC}$.

\item %% Problem 6
Consider the assertion that for each positive integer $n\geq2$,
the remainder upon dividing $2^{2^n}$ by $2^n-1$ is a power of $4$.
Either prove the assertion or find (with proof) a counterexample.

\end{enumerate}
\pagebreak

\section{Solutions to Day 1}
\subsection{JMO 2011/1, proposed by Titu Andreescu}
\textsl{Available online at \url{https://aops.com/community/p2254778}.}
\begin{mdframed}[style=mdpurplebox,frametitle={Problem statement}]
Find all positive integers $n$ such that $2^n+12^n+2011^n$ is a perfect square.
\end{mdframed}
The answer $n=1$ works, because $2^1+12^1+2011^1=45^2$.
We prove it's the only one.
\begin{itemize}
  \ii If $n \ge 2$ is even, then modulo $3$ we have
  $2^n+12^n+2011^n \equiv 1+0+1 \equiv 2 \pmod 3$
  so it is not a square.

  \ii If $n \ge 3$ is odd, then modulo $4$ we have
  $2^n+12^n+2011^n \equiv 0+0+3 \equiv 3 \pmod 4$
  so it is not a square.
\end{itemize}
This completes the proof.
\pagebreak

\subsection{JMO 2011/2, proposed by Titu Andreescu}
\textsl{Available online at \url{https://aops.com/community/p2254758}.}
\begin{mdframed}[style=mdpurplebox,frametitle={Problem statement}]
Let $a$, $b$, $c$ be positive real numbers
such that $a^2+b^2+c^2+(a+b+c)^2 \le 4$. Prove that
\[ \frac{ab+1}{(a+b)^2}
  + \frac{bc+1}{(b+c)^2}
  + \frac{ca+1}{(c+a)^2} \ge 3. \]
\end{mdframed}
The condition becomes $2 \ge a^2+b^2+c^2 + ab+bc+ca$.
Therefore,
\begin{align*}
  \sum_{\text{cyc}} \frac{2ab+2}{(a+b)^2}
  &\ge \sum_{\text{cyc}} \frac{2ab+(a^2+b^2+c^2+ab+bc+ca)}{(a+b)^2} \\
  &= \sum_{\text{cyc}} \frac{(a+b)^2+(c+a)(c+b)}{(a+b)^2} \\
  &= 3 + \sum_{\text{cyc}} \frac{(c+a)(c+b)}{(a+b)^2} \\
  &\ge 3 + 3\sqrt[3]{\prod_{\text{cyc}} \frac{(c+a)(c+b)}{(a+b)^2}}
  = 3 + 3 = 6
\end{align*}
with the last line by AM-GM.
This completes the proof.
\pagebreak

\subsection{JMO 2011/3, proposed by Zuming Feng}
\textsl{Available online at \url{https://aops.com/community/p2254823}.}
\begin{mdframed}[style=mdpurplebox,frametitle={Problem statement}]
For a point $P = (a,a^2)$ in the coordinate plane,
let $\ell(P)$ denote the line passing through $P$ with slope $2a$.
Consider the set of triangles with vertices of the form
$P_1 = (a_1, a_1^2)$, $P_2 = (a_2, a_2^2)$, $P_3 = (a_3, a_3^2)$,
such that the intersection of the lines $\ell(P_1), \ell(P_2), \ell(P_3)$
form an equilateral triangle $\Delta$.
Find the locus of the center of $\Delta$ as $P_1P_2P_3$ ranges over all such triangles.
\end{mdframed}
The answer is the line $y = -1/4$.
I did not find this problem inspiring,
so I will not write out most of the boring calculations
since most solutions are just going to be
``use Cartesian coordinates and grind all the way through''.

The ``nice'' form of the main claim is as follows
(which is certainly overkill for the present task,
but is too good to resist including):
\begin{claim*}
  [Naoki Sato]
  In general, the orthocenter of $\Delta$ lies on
  the directrix $y = -1/4$ of the parabola
  (even if the triangle $\Delta$ is not equilateral).
\end{claim*}
\begin{proof}
  By writing out the equation $y = 2a_i x - a_i^2$ for $\ell(P_i)$,
  we find the vertices of the triangle are located at
  \[
    \left( \frac{a_1+a_2}{2}, a_1a_2 \right); \quad
    \left( \frac{a_2+a_3}{2}, a_2a_3 \right); \quad
    \left( \frac{a_3+a_1}{2}, a_3a_1 \right).
  \]
  The coordinates of the orthocenter can be checked explicitly to be
  \[ H = \left( \frac{a_1 + a_2 + a_3 + 4 a_1 a_2 a_3}{2},
      - \frac 14 \right). \]
  An advanced synthetic proof of this fact is given at
  \url{https://aops.com/community/p2255814}.
\end{proof}

This claim already shows that every point lies on $y = -1/4$.
We now turn to showing that, even when restricted to equilateral triangles,
we can achieve every point on $y = -1/4$.
In what follows $a = a_1$, $b = a_2$, $c = a_3$ for legibility.

\begin{claim*}
  Lines $\ell(a)$, $\ell(b)$, $\ell(c)$
  form an equilateral triangle if and only if
  \begin{align*}
    a+b+c &= -12abc \\
    ab+bc+ca &= -\frac34.
  \end{align*}
  Moreover, the $x$-coordinate of the equilateral triangle is $\frac13(a+b+c)$.
\end{claim*}
\begin{proof}
  The triangle is equilateral if and only if the centroid and orthocenter coincide, i.e.
  \[ \left( \frac{a+b+c}{3}, \frac{ab+bc+ca}{3} \right) = G = H =
    \left( \frac{a+b+c+4abc}{2}, -\frac14 \right). \]
  Setting the $x$ and $y$ coordinates equal,
  we derive the claimed equations.
\end{proof}

Let $\lambda$ be any real number.
We are tasked to show that
\[ P(X) = X^3 - 3\lambda \cdot X^2 - \frac34 X + \frac{\lambda}{4} \]
has three real roots (with multiplicity);
then taking those roots as $(a,b,c)$
yields a valid equilateral-triangle triple
whose $x$-coordinate is exactly $\lambda$, be the previous claim.

To prove that, pick the values
\begin{align*}
  P(-\sqrt{3}/2) &= -2\lambda \\
  P(0) &= \tfrac14 \lambda \\
  P(\sqrt{3}/2) &= -2\lambda.
\end{align*}
The intermediate value theorem
(at least for $\lambda \neq 0$) implies that $P$
should have at least two real roots now,
and since $P$ has degree $3$, it has all real roots.
That's all.
\pagebreak

\section{Solutions to Day 2}
\subsection{JMO 2011/4, proposed by Gabriel Carroll}
\textsl{Available online at \url{https://aops.com/community/p2254808}.}
\begin{mdframed}[style=mdpurplebox,frametitle={Problem statement}]
A \emph{word} is defined as any finite string of letters.
A word is a \emph{palindrome} if it reads the same backwards and forwards.
Let a sequence of words $W_0, W_1, W_2, \dots$ be defined as follows:
$W_0 = a, W_1 = b$, and for $n \ge 2$,
$W_n$ is the word formed by writing $W_{n-2}$ followed by $W_{n-1}$.
Prove that for any $n \ge 1$, the word formed by writing
$W_1, W_2, W_3, \dots, W_n$ in succession is a palindrome.
\end{mdframed}
To aid in following the solution, here are the first several words:
\begin{align*}
  W_0 &= a \\
  W_1 &= b \\
  W_2 &= ab \\
  W_3 &= bab \\
  W_4 &= abbab \\
  W_5 &= bababbab \\
  W_6 &= abbabbababbab \\
  W_7 &= bababbababbabbababbab
\end{align*}
We prove that $W_1 W_2 \dots W_n$ is a palindrome by induction on $n$.
The base cases $n=1,2,3,4$ can be verified by hand.

For the inductive step, we let $\ol{X}$ denote the word $X$ written
backwards. Then
\begin{align*}
  W_1 W_2 \dots W_{n-3} W_{n-2} W_{n-1} W_n
  &\overset{\text{IH}}{=} (\ol{W_{n-1}} \ol{W_{n-2}} \ol{W_{n-3}}
    \dots \ol{W_2} \ol{W_1}) W_n \\
  &= (\ol{W_{n-1}} \ol{W_{n-2}} \ol{W_{n-3}}
    \dots \ol{W_2} \ol{W_1}) W_{n-2} W_{n-1} \\
  &= \ol{W_{n-1}} \ol{W_{n-2}} (\ol{W_{n-3}}
    \dots \ol{W_2} \ol{W_1}) W_{n-2} W_{n-1}
\end{align*}
with the first equality being by the induction hypothesis.
By induction hypothesis again the inner parenthesized term
is also a palindrome, and so this completes the proof.
\pagebreak

\subsection{JMO 2011/5, proposed by Zuming Feng}
\textsl{Available online at \url{https://aops.com/community/p2254813}.}
\begin{mdframed}[style=mdpurplebox,frametitle={Problem statement}]
Points $A,B,C,D,E$ lie on a circle $\omega$
and point $P$ lies outside the circle.
The given points are such that
(i) lines $PB$ and $PD$ are tangent to $\omega$,
(ii) $P, A, C$ are collinear,
and (iii) $\ol{DE} \parallel \ol{AC}$.
Prove that $\ol{BE}$ bisects $\ol{AC}$.
\end{mdframed}
We present two solutions.

\paragraph{First solution using harmonic bundles.}
Let $M = \ol{BE} \cap \ol{AC}$
and let $\infty$ be the point at infinity
along $\ol{DE} \parallel \ol{AC}$.
\begin{center}
\begin{asy}
size(8cm);
pair B = dir(100);
pair D = dir(210);
pair E = dir(330);
pair P = 2*B*D/(B+D);
pair A = OP(P--(P+8*(E-D)), unitcircle);
pair M = extension(B, E, A, P);
pair C = 2*M-A;

filldraw(unitcircle, opacity(0.1)+lightblue, lightblue);
draw(P--A, lightblue);
draw(B--P--D, lightblue);
draw(D--E, heavygreen);
draw(B--E, heavygreen);
draw(A--E--C, heavygreen);

draw(B--A--D--C--cycle, heavycyan);

dot("$B$", B, dir(B));
dot("$D$", D, dir(D));
dot("$E$", E, dir(E));
dot("$P$", P, dir(P));
dot("$A$", A, dir(A));
dot("$M$", M, dir(45));
dot("$C$", C, dir(100));

/* TSQ Source:

B = dir 100
D = dir 210
E = dir 330
P = 2*B*D/(B+D)
A = IP P--(P+8*(E-D)) unitcircle
M = extension B E A P R45
C = 2*M-A R100

unitcircle 0.1 lightblue / lightblue
P--A lightblue
B--P--D lightblue
D--E heavygreen
B--E heavygreen
A--E--C heavygreen

B--A--D--C--cycle heavycyan

*/
\end{asy}
\end{center}
Note that $ABCD$ is harmonic, so
\[ -1 = (AC;BD) \overset{E}{=} (AC;M\infty) \]
implying $M$ is the midpoint of $\ol{AC}$.


\paragraph{Second solution using complex numbers (Cynthia Du).}
Suppose we let $b$, $d$, $e$ be free on unit circle,
so $p = \frac{2bd}{b+d}$.
Then $d/c = a/e$, and $a + c = p + ac \ol p$.
Consequently,
\begin{align*}
  ac &= de \\
  \half(a + c) &= \frac{bd}{b+d} + de \cdot \frac{1}{b+d}
    = \frac{d(b+e)}{b+d}. \\
  \frac{a+c}{2ac} &= \frac{(b+e)}{e(b+d)}.
\end{align*}
From here it's easy to see
\[ \frac{a+c}{2} + \frac{a+c}{2ac} \cdot be = b + e \]
which is what we wanted to prove.
\pagebreak

\subsection{JMO 2011/6, proposed by Sam Vandervelde}
\textsl{Available online at \url{https://aops.com/community/p2254810}.}
\begin{mdframed}[style=mdpurplebox,frametitle={Problem statement}]
Consider the assertion that for each positive integer $n\geq2$,
the remainder upon dividing $2^{2^n}$ by $2^n-1$ is a power of $4$.
Either prove the assertion or find (with proof) a counterexample.
\end{mdframed}
We claim $n = 25$ is a counterexample.
Since $2^{25} \equiv 2^0 \pmod{2^{25}-1}$, we have
\[ 2^{2^{25}} \equiv 2^{2^{25} \bmod{25}}
  \equiv 2^7 \bmod{2^{25}-1} \]
and the right-hand side is actually the remainder,
since $0 < 2^7 < 2^{25}$.
But $2^7$ is not a power of $4$.

\begin{remark*}
  Really, the problem is just equivalent
  for asking $2^n$ to have odd remainder when divided by $n$.
\end{remark*}
\pagebreak


\end{document}
