\documentclass[11pt]{scrartcl}
\usepackage[sexy]{evan}
\ihead{\footnotesize\textbf{\thetitle}}
\ohead{\footnotesize\theauthor}
\begin{document}

\title{USA IMO TST 2017 Solutions}
\subtitle{United States of America --- IMO Team Selection Tests}
\date{58\ts{th} IMO 2017 Brazil}

Note that in this solutions file,
we present slightly stronger versions of
problems 4 and 6 on the January TST
than actually appeared on the exams.

\maketitle

\tableofcontents
\newpage

\addtocounter{section}{-1}
\section{Problems}
\begin{enumerate}[\bfseries 1.]
\item %% Problem 1
In a sports league, each team uses a set of at most $t$ signature colors.
A set $S$ of teams is \emph{color-identifiable} if one can assign
each team in $S$ one of their signature colors,
such that no team in $S$ is assigned
\emph{any} signature color of a different team in $S$.
For all positive integers $n$ and $t$,
determine the maximum integer $g(n,t)$ such that:
In any sports league with exactly $n$ distinct colors
present over all teams, one can always
find a color-identifiable set of size at least $g(n,t)$.

\item %% Problem 2
Let $ABC$ be an acute scalene triangle with circumcenter $O$,
and let $T$ be on line $BC$ such that $\angle TAO = 90\dg$.
The circle with diameter $\ol{AT}$
intersects the circumcircle of $\triangle BOC$ at two points
$A_1$ and $A_2$, where $OA_1 < OA_2$.
Points $B_1$, $B_2$, $C_1$, $C_2$ are defined analogously.
\begin{enumerate}
  \ii[(a)]
  Prove that $\ol{AA_1}$, $\ol{BB_1}$, $\ol{CC_1}$ are concurrent.
  \ii[(b)]
  Prove that $\ol{AA_2}$, $\ol{BB_2}$, $\ol{CC_2}$ are concurrent
  on the Euler line of triangle $ABC$.
\end{enumerate}

\item %% Problem 3
Let $P, Q \in \RR[x]$ be relatively prime nonconstant polynomials.
Show that there can be at most three real numbers $\lambda$
such that $P + \lambda Q$ is the square of a polynomial.

\item %% Problem 4
You are cheating at a trivia contest.
For each question, you can peek at each of the
$n > 1$ other contestant's guesses before writing your own.
For each question, after all guesses are submitted, the emcee announces the correct answer.
A correct guess is worth $0$ points.
An incorrect guess is worth $-2$ points for other contestants,
but only $-1$ point for you, because you hacked the scoring system.
After announcing the correct answer, the emcee proceeds to read out the next question.
Show that if you are leading by $2^{n-1}$ points at any time,
then you can surely win first place.

\item %% Problem 5
Let $ABC$ be a triangle with altitude $\ol{AE}$.
The $A$-excircle touches $\ol{BC}$ at $D$,
and intersects the circumcircle at two points $F$ and $G$.
Prove that one can select points $V$ and $N$
on lines $DG$ and $DF$ such that quadrilateral $EVAN$ is a rhombus.

\item %% Problem 6
Prove that there are infinitely many triples $(a,b,p)$ of integers,
with $p$ prime and $0 < a \le b < p$,
for which $p^5$ divides $(a+b)^p - a^p - b^p$.

\end{enumerate}
\pagebreak

\section{Solutions to Day 1}
\subsection{USA TST 2017/1, proposed by Po-Shen Loh}
\textsl{Available online at \url{https://aops.com/community/p7389115}.}
\begin{mdframed}[style=mdpurplebox,frametitle={Problem statement}]
In a sports league, each team uses a set of at most $t$ signature colors.
A set $S$ of teams is \emph{color-identifiable} if one can assign
each team in $S$ one of their signature colors,
such that no team in $S$ is assigned
\emph{any} signature color of a different team in $S$.
For all positive integers $n$ and $t$,
determine the maximum integer $g(n,t)$ such that:
In any sports league with exactly $n$ distinct colors
present over all teams, one can always
find a color-identifiable set of size at least $g(n,t)$.
\end{mdframed}
Answer: $\left\lceil n/t \right\rceil$.

To see this is an upper bound, note that one can easily construct
a sports league with that many teams anyways.

A quick warning:
\begin{remark*}
  [Misreading the problem]
  It is common to misread the problem by ignoring the word ``any''.
  Here is an illustration.

  Suppose we have two teams, MIT and Harvard;
  the colors of MIT are red/grey/black, and
  the colors of Harvard are red/white.
  (Thus $n=4$ and $t=3$.)
  The assignment of MIT to grey and Harvard to red
  is \emph{not} acceptable because red is a
  signature color of MIT, even though not the one assigned.
\end{remark*}

We present two proofs of the lower bound.

\paragraph{Approach by deleting teams (Gopal Goel).}
Initially, place all teams in a set $S$.
Then we repeat the following algorithm:
\begin{quote}
If there is a team all of whose signature colors
are shared by some other team in $S$ already,
then we delete that team.
\end{quote}
(If there is more than one such team, we pick arbitrarily.)

At the end of the process,
all $n$ colors are still present at least once,
so at least $\left\lceil n/t \right\rceil$ teams remain.
Moreover, since the algorithm is no longer possible,
the remaining set $S$ is already color-identifiable.

\begin{remark*}
  [Gopal Goel]
  It might seem counter-intuitive
  that we are \emph{deleting} teams from the full set
  when the original problem is trying to get a large set $S$.

  This is less strange when one thinks
  of it instead as ``safely deleting useless teams''.
  Basically, if one deletes such a team,
  the problem statement implies that the task must still be possible,
  since $g(n,t)$ does not depend on the number of teams:
  $n$ is the number of colors present,
  and deleting a useless team does not change this.
  It turns out that this optimization is already enough to solve the problem.
\end{remark*}

\paragraph{Approach by adding colors.}
For a constructive algorithmic approach,
the idea is to greedy pick by color (rather than by team),
taking at each step the least used color.
Select the color $C_1$ with the \emph{fewest} teams using it,
and a team $T_1$ using it.
Then delete all colors $T_1$ uses, and all teams which use $C_1$.
Note that
\begin{itemize}
  \ii By problem condition,
  this deletes at most $t$ teams total.
  \ii Any remaining color $C$ still has at least one user.
  Indeed, if not, then $C$ had the same set of teams
  as $C_1$ did (by minimality of $C$),
  but then it should have deleted as a color of $T_1$.
\end{itemize}

Now repeat this algorithm with $C_2$ and $T_2$, and so on.
This operations uses at most $t$ colors each time,
so we select at least $\left\lceil n/t \right\rceil$ colors.

\begin{remark*}
  A greedy approach by team \emph{does not work}.
  For example, suppose we try to ``grab teams until no more can be added''.

  As before, assume our league has teams, MIT and Harvard;
  the colors of MIT are red/grey/black, and
  the colors of Harvard are red/white.
  (Thus $n=4$ and $t=3$.)
  If we start by selecting MIT and red, then
  it is impossible to select any more teams; but $g(n,t) = 2$.
\end{remark*}
\pagebreak

\subsection{USA TST 2017/2, proposed by Evan Chen}
\textsl{Available online at \url{https://aops.com/community/p7389108}.}
\begin{mdframed}[style=mdpurplebox,frametitle={Problem statement}]
Let $ABC$ be an acute scalene triangle with circumcenter $O$,
and let $T$ be on line $BC$ such that $\angle TAO = 90\dg$.
The circle with diameter $\ol{AT}$
intersects the circumcircle of $\triangle BOC$ at two points
$A_1$ and $A_2$, where $OA_1 < OA_2$.
Points $B_1$, $B_2$, $C_1$, $C_2$ are defined analogously.
\begin{enumerate}
  \ii[(a)]
  Prove that $\ol{AA_1}$, $\ol{BB_1}$, $\ol{CC_1}$ are concurrent.
  \ii[(b)]
  Prove that $\ol{AA_2}$, $\ol{BB_2}$, $\ol{CC_2}$ are concurrent
  on the Euler line of triangle $ABC$.
\end{enumerate}
\end{mdframed}
Let triangle $ABC$ have circumcircle $\Gamma$.
Let $\triangle XYZ$ be the tangential triangle of $\triangle ABC$
(hence $\Gamma$ is the incircle of $\triangle XYZ$),
and denote by $\Omega$ its circumcircle.
Suppose the symmedian $\ol{AX}$ meets $\Gamma$ again at $D$,
and let $M$ be the midpoint of $\ol{AD}$.
Finally, let $K$ be the Miquel point of quadrilateral $ZBCY$,
meaning it is the intersection of $\Omega$
and the circumcircle of $\triangle BOC$ (other than $X$).

\begin{center}
\begin{asy}
size(12cm);
pair A = dir(115);
pair B = dir(200);
pair C = dir(340);
draw(A--B--C--cycle, blue);
filldraw(circumcircle(A, B, C), opacity(0.1)+lightblue, blue);
pair O = circumcenter(A, B, C);

pair X = 2*B*C/(B+C);
pair Y = 2*C*A/(C+A);
pair Z = 2*A*B/(A+B);

filldraw(circumcircle(X, Y, Z), opacity(0.1)+lightcyan, cyan);
draw(X--Y--Z--cycle, cyan);

filldraw(circumcircle(B, O, C), opacity(0.1)+heavycyan, heavycyan);
pair T = extension(Y, Z, B, C);
draw(Z--T, lightcyan);
draw(T--B, blue+dashed);

pair D = -A+2*foot(O, A, X);
pair U = foot(A, B, C);
filldraw(circumcircle(T, A, U), opacity(0.1)+lightgreen, heavygreen);

pair V = circumcenter(X, Y, Z);
pair M = midpoint(A--D);
pair L = IP(V--(Y+Z-V), circumcircle(X, Y, Z));
pair K = extension(L, A, O, U);
draw(A--X, heavycyan);
draw(K--L, heavygreen+dashed);

dot("$A$", A, dir(A));
dot("$B$", B, dir(B));
dot("$C$", C, dir(C));
dot("$O$", O, dir(O));
dot("$X$", X, dir(X));
dot("$Y$", Y, dir(Y));
dot("$Z$", Z, dir(Z));
dot("$T$", T, dir(T));
dot("$D$", D, dir(D));
dot("$M$", M, dir(M));
dot("$L$", L, dir(L));
dot("$K$", K, dir(K));

/* TSQ Source:

!size(12cm);
A = dir 115
B = dir 200
C = dir 340
A--B--C--cycle blue
circumcircle A B C 0.1 lightblue / blue
O = circumcenter A B C

X = 2*B*C/(B+C)
Y = 2*C*A/(C+A)
Z = 2*A*B/(A+B)

circumcircle X Y Z 0.1 lightcyan / cyan
X--Y--Z--cycle cyan

circumcircle B O C 0.1 heavycyan / heavycyan
T = extension Y Z B C
Z--T lightcyan
T--B blue dashed

D = -A+2*foot O A X
U := foot A B C
circumcircle T A U 0.1 lightgreen / heavygreen

V := circumcenter X Y Z
M = midpoint A--D
L = IP V--(Y+Z-V) circumcircle X Y Z
K = extension L A O U
A--X heavycyan
K--L heavygreen dashed

*/
\end{asy}
\end{center}

We first claim that $M$ and $K$ are $A_1$ and $A_2$.
In that case $OM < OA < OK$, so $M = A_1$, $K = A_2$.

To see that $M = A_1$, note that $\angle OMX = 90\dg$,
and moreover that $\ol{TA}$, $\ol{TD}$ are tangents to $\Gamma$,
whence we also have $M = \ol{TO} \cap \ol{AD}$.
Thus $M$ lies on both $(BOC)$ and $(AT)$.
This solves part (a) of the problem:
the concurrency point is the symmedian point of $\triangle ABC$.

Now, note that since $K$ is the Miquel point,
\[ \frac{ZK}{YK} = \frac{ZB}{YC} = \frac{ZA}{YA} \]
and hence $\ol{KA}$ is an angle bisector of $\angle ZKY$.
Thus from $(TA;YZ)=-1$ we obtain $\angle TKA = 90\dg$.

It remains to show $\ol{AK}$ passes through a fixed point on the Euler line.
We claim it is the exsimilicenter of $\Gamma$ and $\Omega$.
Let $L$ be the midpoint of the arc $YZ$ of $\triangle XYZ$ not containing $X$.
Then we know that $K$, $A$, $L$ are collinear.
Now the positive homothety sending $\Gamma$ to $\Omega$ maps $A$ to $L$;
this proves the claim.
Finally, it is well-known that the line through $O$
and the circumcenter of $\triangle XYZ$
coincides with the Euler line of $\triangle ABC$;
hence done.

A second approach to (b) presented by many contestants
is to take an inversion around the circumcircle of $ABC$.
In that situation, the part reduces to the following
known lemma: if $\ol{AH_a}$, $\ol{BH_b}$, $\ol{CH_c}$
are the altitudes of a triangle,
then the circumcircles of triangles $OAH_a$, $BOH_b$, $COH_c$
are coaxial, and the radical axis coincides with the Euler line.
Indeed one simply observes that the orthocenter
has equal power to all three circles.

\paragraph{Authorship comments.}
This problem was inspired by the fact that $K$, $A$, $L$
are collinear in the figure,
which was produced by one of my students (Ryan Kim)
in a solution to a homework problem.
I realized for example that this implied that line $AK$
passed through the $X_{56}$ point of $\triangle XYZ$
(which lies on the Euler line of $\triangle ABC$).

This problem was the result of playing around with
the resulting very nice picture:
all the power comes from the ``magic'' point $K$.
\pagebreak

\subsection{USA TST 2017/3, proposed by Alison Miller}
\textsl{Available online at \url{https://aops.com/community/p7389123}.}
\begin{mdframed}[style=mdpurplebox,frametitle={Problem statement}]
Let $P, Q \in \RR[x]$ be relatively prime nonconstant polynomials.
Show that there can be at most three real numbers $\lambda$
such that $P + \lambda Q$ is the square of a polynomial.
\end{mdframed}
This is true even with $\RR$ replaced by $\CC$,
and it will be necessary to work in this generality.

\paragraph{First solution using transformations.}
We will prove the claim in the following form:

\begin{claim*}
Assume $P, Q \in \CC[x]$ are relatively prime.
If $\alpha P + \beta Q$ is a square for four different
choices of the ratio $[\alpha : \beta]$
then $P$ and $Q$ must be constant.
\end{claim*}

Call pairs $(P,Q)$ as in the claim \emph{bad};
so we wish to show the only bad pairs are pairs of constant polynomials.
Assume not, and take a bad pair with $\deg P + \deg Q$ minimal.

By a suitable M\"obius transformation,
we may transform $(P,Q)$ so that the four ratios are $[1:0]$,
$[0:1]$, $[1:-1]$ and $[1:-k]$,
so we find there are polynomials $A$ and $B$ such that
\begin{align*}
  A^2 - B^2 &= C^2 \\
  A^2 - k B^2 &= D^2
\end{align*}
where $A^2 = P+\lambda_1 Q$, $B^2 = P+\lambda_2 Q$, say.
Of course $\gcd(A,B) = 1$.

Consequently, we have $C^2 = (A+B)(A-B)$
and $D^2 = (A+\mu B)(A-\mu B)$ where $\mu^2 = k$.
Now $\gcd(A,B) = 1$, so $A+B$, $A-B$, $A+ \mu B$ and $A - \mu B$
are squares; id est $(A,B)$ is bad.
This is a contradiction, since $\deg A + \deg B < \deg P + \deg Q$.

\paragraph{Second solution using derivatives (by Zack Chroman).}
We will assume without loss of generality that $\deg P \neq \deg Q$;
if not, then one can replace $(P,Q)$ with $(P+cQ,Q)$
for a suitable constant $c$.

Then, there exist $\lambda_i \in \CC$ and polynomials $R_i$
for $i=1,2,3,4$ such that
\begin{align*}
  P + \lambda_i Q &= R_i^2 \\
  \implies P' + \lambda_i Q' &= 2 R_i R_i' \\
  \implies R_i &\mid Q'(P+\lambda_i Q) - Q(P' + \lambda_i Q')
  = Q'P - QP'.
\end{align*}
On the other hand by Euclidean algorithm it follows that
$R_i$ are relatively prime to each other.
Therefore
\[ R_1 R_2 R_3 R_4 \mid Q' P - Q P'. \]
However, we have
\[ \sum_1^4 \deg R_i \ge
  \frac{3\max(\deg P, \deg Q) + \min(\deg P, \deg Q)}{2}
  \ge \deg P + \deg Q > \deg(Q'P - QP').  \]
This can only occur if $Q'P - QP' = 0$ or $(P/Q)' = 0$
by the quotient rule!
But $P/Q$ can't be constant, the end.

\begin{remark*}
  The result is previously known; see e.g.\
  Lemma 1.6 of \url{http://math.mit.edu/~ebelmont/ec-notes.pdf}
  or Exercise 6.5.L(a) of Vakil's notes.
\end{remark*}
\pagebreak

\section{Solutions to Day 2}
\subsection{USA TST 2017/4, proposed by Linus Hamilton}
\textsl{Available online at \url{https://aops.com/community/p7732191}.}
\begin{mdframed}[style=mdpurplebox,frametitle={Problem statement}]
You are cheating at a trivia contest.
For each question, you can peek at each of the
$n > 1$ other contestant's guesses before writing your own.
For each question, after all guesses are submitted, the emcee announces the correct answer.
A correct guess is worth $0$ points.
An incorrect guess is worth $-2$ points for other contestants,
but only $-1$ point for you, because you hacked the scoring system.
After announcing the correct answer, the emcee proceeds to read out the next question.
Show that if you are leading by $2^{n-1}$ points at any time,
then you can surely win first place.
\end{mdframed}
We will prove the result with $2^{n-1}$ replaced
even by $2^{n-2}+1$.

We first make the following reductions.
First, change the weights to be $+1$, $-1$, $0$ respectively
(rather than $0$, $-2$, $-1$); this clearly has no effect.
Also, WLOG that all contestants except you initially have score zero
(and that your score exceeds $2^{n-2}$).
WLOG ignore rounds in which all answers are the same.
Finally, ignore rounds in which you get the correct answer,
since that leaves you at least as well off as before --- in other words,
we'll assume your score is always fixed,
but you can pick any group of people with the same answers
and ensure they lose 1 point,
while some other group gains $1$ point.

The key observation is the following.
Consider two rounds $R_1$ and $R_2$ such that:
\begin{itemize}
  \ii In round $R_1$, some set $S$ of contestants gains a point.
  \ii In round $R_2$, the set $S$ of contestants all have the same answer.
\end{itemize}
Then, if we copy the answers of contestants in $S$ during $R_2$,
then the sum of the scorings in $R_1$ and $R_2$ cancel each other out.
In other words we can then ignore $R_1$ and $R_2$ forever.

We thus consider the following strategy.
We keep a list $\mathcal L$ of subsets of $\{1, \dots, n\}$, initially empty.
Now do the following strategy:
\begin{itemize}
  \ii On a round, suppose there exists a set $S$ of people
  with the same answer such that $S \in \mathcal L$.
  $($If multiple exist, choose one arbitrarily.)
  Then, copy the answer of $S$, causing them to lose a point.
  Delete $S$ from $\mathcal L$.
  (Importantly, we do not add any new sets to $\mathcal L$.)
  \ii Otherwise, copy any set $T$ of contestants,
  selecting $|T| \ge n/2$ if possible.
  Let $S$ be the set of contestants who answer correctly (if any),
  and add $S$ to the list $\mathcal L$.
  Note that $|S| \le n/2$, since $S$ is disjoint from $T$.
\end{itemize}
By construction, $\mathcal L$ has no duplicate sets.
So the score of any contestant $c$ is bounded above by the number of times
that $c$ appears among sets in $\mathcal L$.
The number of such sets is clearly at most $\half \cdot 2^{n-1}$.
So, if you lead by $2^{n-2}+1$ then you ensure victory.
This completes the proof!

\begin{remark*}
Several remarks are in order.
First, we comment on the bound $2^{n-2} + 1$ itself.
The most natural solution using only the
list idea gives an upper bound of $(2^n-2)+1$,
which is the number of nonempty proper subsets of $\{1, \dots, n\}$.
Then, there are two optimizations one can observe:
\begin{itemize}
  \ii In fact we can improve to the number of times
  any particular contestant $c$ appears in some set,
  rather than the total number of sets.
  \ii When adding new sets $S$ to $\mathcal L$, one can ensure $|S| \le n/2$.
\end{itemize}
Either observation alone improves the bound from $2^n-1$ to $2^{n-1}$,
but both together give the bound $2^{n-2} + 1$.
Additionally, when $n$ is odd the calculation of subsets
actually gives $2^{n-2} - \half \binom{n-1}{\frac{n-1}{2}} + 1$.
This gives the best possible value at both $n=2$ and $n=3$.
It seems likely some further improvements are possible,
and the true bound is suspected to be polynomial in $n$.

Secondly, the solution is highly motivated by considering a true/false contest
in which only two distinct answers are given per question.
However, a natural mistake (which graders assessed as a two-point deduction)
is to try and prove that in fact
one can ``WLOG'' we are in the two-question case.
The proof of this requires substantially more care than expected.
For instance, set $n = 3$.
If $\mathcal L = \left\{ \{1\}, \{2\}, \{3\}  \right\}$
then it becomes impossible to prevent a duplicate set from appearing
in $\mathcal L$ if all contestants give distinct answers.
One might attempt to fix this by instead adding to $\mathcal L$
the \emph{complement} of the set $T$ described above.
The example $\mathcal L = \left\{ \{1,2\}, \{2,3\}, \{3,1\}  \right\}$
(followed again by a round with all distinct answers)
shows that this proposed fix does not work either.
This issue affects all variations of the above approach.

Because the USA TST did not have any solution-writing process at this time,
this issue was not noticed until January 15 (less than a week before the exam).
When it was brought up by email by Evan,
every organizer who had testsolved the problem had apparently made the same error.
\end{remark*}
\begin{remark*}
  Here are some motivations for the solution:
  \begin{enumerate}
    \ii The exponential bound $2^n$ suggests looking at subsets.
    \ii The $n = 2$ case suggests the idea of ``repeated rounds''.
    (I think this $n=2$ case is actually really good.)
    \ii The ``two distinct answers'' case suggests looking at rounds
    as partitions (even though the WLOG does not work,
    at least not without further thought).
    \ii There's something weird about this problem:
    it's a finite bound over unbounded time.
    This is a hint to \emph{not} worry excessively about
    the actual scores, which turn out to be almost irrelevant.
  \end{enumerate}
\end{remark*}
\pagebreak

\subsection{USA TST 2017/5, proposed by Danielle Wang, Evan Chen}
\textsl{Available online at \url{https://aops.com/community/p7732197}.}
\begin{mdframed}[style=mdpurplebox,frametitle={Problem statement}]
Let $ABC$ be a triangle with altitude $\ol{AE}$.
The $A$-excircle touches $\ol{BC}$ at $D$,
and intersects the circumcircle at two points $F$ and $G$.
Prove that one can select points $V$ and $N$
on lines $DG$ and $DF$ such that quadrilateral $EVAN$ is a rhombus.
\end{mdframed}
Let $I$ denote the incenter, $J$ the $A$-excenter,
and $L$ the midpoint of $\ol{AE}$.
Denote by $\ol{IY}$, $\ol{IZ}$ the tangents
from $I$ to the $A$-excircle.
Note that lines $\ol{BC}$, $\ol{GF}$, $\ol{YZ}$ then concur at $H$
(unless $AB=AC$, but this case is obvious),
as it's the radical center of cyclic hexagon $BICYJZ$,
the circumcircle and the $A$-excircle.

\begin{center}
\begin{asy}
size(12cm);
pair A = dir(110);
pair B = dir(210);
pair C = dir(330);
pair I = incenter(A, B, C);
pair M = dir(270);
pair J = 2*M-I;
pair D = foot(J, B, C);
pair E = foot(A, B, C);

pair F = IP(CP(J, D), unitcircle);
pair G = OP(CP(J, D), unitcircle);

filldraw(unitcircle, opacity(0.1)+lightblue, lightblue);
draw(arc(J,abs(D-J),-30,210), lightblue);

pair L = midpoint(A--E);
pair V = extension(G, D, L, L+B-C);
pair N = extension(F, D, L, L+B-C);
draw(A--B--C--cycle, lightblue);
draw(A--E, lightblue);
draw(G--V, lightblue);
draw(N--F, lightblue);
filldraw(E--V--A--N--cycle, opacity(0.1)+lightgreen, heavygreen);
draw(V--N, heavygreen);
pair H = extension(G, F, B, C);
draw(C--H--G, lightblue);

filldraw(circumcircle(B, I, C), opacity(0.1)+lightred, lightred);
pair Y = IP(circumcircle(B, I, C), circumcircle(D, F, G));
pair Z = OP(circumcircle(B, I, C), circumcircle(D, F, G));
pair T = -D+2*foot(J, D, I);
draw(T--H, blue);
draw(Z--H, blue);
draw(L--T, blue);
draw(Y--I--Z, blue);

dot("$A$", A, dir(A));
dot("$B$", B, dir(B));
dot("$C$", C, dir(C));
dot("$I$", I, dir(I));
dot("$M$", M, dir(M));
dot("$J$", J, dir(J));
dot("$D$", D, dir(D));
dot("$E$", E, dir(E));
dot("$F$", F, dir(270));
dot("$G$", G, dir(270));
dot("$L$", L, dir(L));
dot("$V$", V, dir(V));
dot("$N$", N, dir(N));
dot("$H$", H, dir(H));
dot("$Y$", Y, dir(Y));
dot("$Z$", Z, dir(Z));
dot("$T$", T, dir(T));

/* TSQ Source:

!size(12cm);
A = dir 110
B = dir 210
C = dir 330
I = incenter A B C
M = dir 270
J = 2*M-I
D = foot J B C
E = foot A B C

F = IP CP J D unitcircle R270
G = OP CP J D unitcircle R270

unitcircle 0.1 lightblue / lightblue
!draw(arc(J,abs(D-J),-30,210), lightblue);

L = midpoint A--E
V = extension G D L L+B-C
N = extension F D L L+B-C
A--B--C--cycle lightblue
A--E lightblue
G--V lightblue
N--F lightblue
E--V--A--N--cycle 0.1 lightgreen / heavygreen
V--N heavygreen
H = extension G F B C
C--H--G lightblue

circumcircle B I C 0.1 lightred / lightred
Y = IP circumcircle B I C circumcircle D F G
Z = OP circumcircle B I C circumcircle D F G
T = -D+2*foot J D I
T--H blue
Z--H blue
L--T blue
Y--I--Z blue

*/
\end{asy}
\end{center}

Now let $\ol{HD}$ and $\ol{HT}$ be the tangents
from $H$ to the $A$-excircle.
It follows that $\ol{DT}$ is the symmedian of $\triangle DZY$,
hence passes through $I = \ol{YY} \cap \ol{ZZ}$.
Moreover, it's well known that $\ol{DI}$ passes through $L$,
the midpoint of the $A$-altitude
(for example by homothety).
Finally, $(DT;FG) = -1$,
hence project through $D$ onto the line through $L$
parallel to $\ol{BC}$ to obtain $(\infty L; VN) = -1$ as desired.

\paragraph{Authorship comments.}
This is a joint proposal with Danielle Wang (mostly by her).
The formulation given was that the tangents to the $A$-excircle
at $F$ and $G$ was on line $\ol{DI}$;
I solved this formulation using the radical axis argument above.
I then got the idea to involve the point $L$,
already knowing it was on $\ol{DI}$.
Observing the harmonic quadrilateral,
I took perspectivity through $M$ onto the line through $L$ parallel to $\ol{BC}$
(before this I had tried to use the $A$-altitude with little luck).
This yields the rhombus in the problem.
\pagebreak

\subsection{USA TST 2017/6, proposed by Noam Elkies}
\textsl{Available online at \url{https://aops.com/community/p7732203}.}
\begin{mdframed}[style=mdpurplebox,frametitle={Problem statement}]
Prove that there are infinitely many triples $(a,b,p)$ of integers,
with $p$ prime and $0 < a \le b < p$,
for which $p^5$ divides $(a+b)^p - a^p - b^p$.
\end{mdframed}
The key claim is that if $p \equiv 1 \pmod 3$,
then
\[ p(x^2+xy+y^2)^2 \text{ divides } (x+y)^p - x^p - y^p \]
as polynomials in $x$ and $y$.
Since it's known that one can select $a$ and $b$ such that
$p^2 \mid a^2 + ab + b^2$, the conclusion follows.
(The theory of quadratic forms tells us we can do it with $p^2 = a^2+ab+b^2$;
Thue's lemma lets us do it by solving $x^2+x+1 \equiv 0 \pmod{p^2}$.)

To prove this, it is the same to show that
\[ (x^2+x+1)^2 \text{ divides } F(x) \coloneqq (x+1)^p - x^p - 1. \]
since the binomial coefficients $\binom pk$ are clearly divisible by $p$.
Let $\zeta$ be a third root of unity.
Then $F(\zeta) = (1+\zeta)^p - \zeta^p - 1 = -\zeta^2 - \zeta - 1 = 0$.
Moreover, $F'(x) = p(x+1)^{p-1} -  px^{p-1}$,
so $F'(\zeta) = p - p = 0$.
Hence $\zeta$ is a double root of $F$ as needed.

(Incidentally, $p = 2017$ works!)

\begin{remark*}
One possible motivation for this solution is the case $p = 7$.
It is nontrivial even to prove that $p^2$ can divide the expression
if we exclude the situation $a+b=p$ (which provably never achieves $p^3$).
As $p = 3, 5$ fails considering the $p = 7$ polynomial gives
\[ (x+1)^7 - x^7 - 1 = 7x(x+1) \left( x^4 + 2x^3 + 3x^2 + 2x + 1 \right). \]
The key is now to notice that the last factor is $(x^2+x+1)^2$,
which suggests the entire solution.

In fact, even if $p \equiv 2 \pmod 3$,
the polynomial $x^2+x+1$ still divides $(x+1)^p-x^p-1$.
So even the $p = 5$ case can motivate the main idea.
\end{remark*}
\pagebreak


\end{document}
