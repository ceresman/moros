% © Evan Chen
% Downloaded from https://web.evanchen.cc/

\documentclass[11pt]{scrartcl}
\usepackage[sexy]{evan}
\ihead{\footnotesize\textbf{\thetitle}}
\ohead{\footnotesize\href{http://web.evanchen.cc}{\ttfamily web.evanchen.cc},
    updated \today}
\title{USAMO 2004 Solution Notes}
\date{\today}

\begin{document}

\maketitle

\begin{abstract}
This is a compilation of solutions
for the 2004 USAMO.
Some of the solutions are my own work,
but many are from the official solutions provided by the organizers
(for which they hold any copyrights),
and others were found by users on the Art of Problem Solving forums.

These notes will tend to be a bit more advanced and terse than the ``official''
solutions from the organizers.
In particular, if a theorem or technique is not known to beginners
but is still considered ``standard'', then I often prefer to
use this theory anyways, rather than try to work around or conceal it.
For example, in geometry problems I typically use directed angles
without further comment, rather than awkwardly work around configuration issues.
Similarly, sentences like ``let $\mathbb{R}$ denote the set of real numbers''
are typically omitted entirely.

Corrections and comments are welcome!
\end{abstract}

\tableofcontents
\newpage

\addtocounter{section}{-1}
\section{Problems}
\begin{enumerate}[\bfseries 1.]
\item %% Problem 1
Let $ABCD$ be a quadrilateral circumscribed about a circle,
whose interior and exterior angles are at least $60$ degrees.
Prove that
\[ \frac{1}{3}|AB^3 - AD^3|
  \le |BC^3 - CD^3| \le 3|AB^3 - AD^3|.  \]
When does equality hold?

\item %% Problem 2
Let $a_1$, $a_2$, \dots, $a_n$ be integers whose greatest common
divisor is $1$.
Let $S$ be a set of integers with the following properties:
\begin{enumerate}
  \ii[(a)] $a_i \in S$ for $i = 1, \dots, n$.
  \ii[(b)] $a_i - a_j \in S$ for $i, j = 1, \dots, n$,
  not necessarily distinct.
  \ii[(c)] If $x, y \in S$ and $x+y \in S$,
  then $x-y \in S$ too.
\end{enumerate}
Prove that $S = \ZZ$.

\item %% Problem 3
For what real values of $k > 0$ is it possible to
dissect a $1 \times k$ rectangle
into two similar but noncongruent polygons?

\item %% Problem 4
Alice and Bob play a game on a $6$ by $6$ grid.
On his turn, a player chooses a rational number not yet appearing in the grid
and writes it in an empty square of the grid.
Alice goes first and then the players alternate.
When all squares have numbers written in them, in each row,
the square with the greatest number in that row is colored black.
Alice wins if he can then draw a line from the top of the grid to the bottom of the grid
that stays in black squares, and Bob wins if he can't.
(If two squares share a vertex,
Alice can draw a line from one to the other that stays in those two squares.)
Find, with proof, a winning strategy for one of the players.

\item %% Problem 5
Let $a$, $b$, $c$ be positive reals.
Prove that
\[ ( a^5-a^2+3 )( b^5-b^2+3 )( c^5-c^2+3 )
  \ge \left( a+b+c \right)^3. \]

\item %% Problem 6
A circle $\omega$ is inscribed in a quadrilateral $ABCD$.
Let $I$ be the center of $\omega$.
Suppose that \[ (AI + DI)^2 + (BI + CI)^2 = (AB + CD)^2.  \]
Prove that $ABCD$ is an isosceles trapezoid.

\end{enumerate}
\pagebreak

\section{Solutions to Day 1}
\subsection{USAMO 2004/1, proposed by Titu Andreescu}
\textsl{Available online at \url{https://aops.com/community/p17439}.}
\begin{mdframed}[style=mdpurplebox,frametitle={Problem statement}]
Let $ABCD$ be a quadrilateral circumscribed about a circle,
whose interior and exterior angles are at least $60$ degrees.
Prove that
\[ \frac{1}{3}|AB^3 - AD^3|
  \le |BC^3 - CD^3| \le 3|AB^3 - AD^3|.  \]
When does equality hold?
\end{mdframed}
Clearly it suffices to show the left inequality.
Since $AB+CD = BC+AD \implies |AB-AD| = |BC-CD|$, it suffices to prove
\[ \frac13(AB^2 + AB \cdot AD + AD^2)
\le BC^2 + BC \cdot CD + CD^2. \]
This follows by noting that
\begin{align*}
  BC^2 + BC \cdot CD + CD^2
  &\ge BC^2 + CD^2 - 2(BC)(CD)\cos(\angle BCD) \\
  &= BD^2 \\
  &= AB^2 + AD^2 - 2(AB)(AD)\cos(\angle BAD) \\
  &\ge AB^2 + AD^2 - AB \cdot AD \\
  &\ge \tfrac13(AB^2 + AD^2 + AB \cdot AD)
\end{align*}
the last line following by AM-GM.

The equality holds iff $ABCD$ is a kite with $AB=AD$, $CB=CD$.
\pagebreak

\subsection{USAMO 2004/2, proposed by Kiran Kedlaya}
\textsl{Available online at \url{https://aops.com/community/p17440}.}
\begin{mdframed}[style=mdpurplebox,frametitle={Problem statement}]
Let $a_1$, $a_2$, \dots, $a_n$ be integers whose greatest common
divisor is $1$.
Let $S$ be a set of integers with the following properties:
\begin{enumerate}
  \ii[(a)] $a_i \in S$ for $i = 1, \dots, n$.
  \ii[(b)] $a_i - a_j \in S$ for $i, j = 1, \dots, n$,
  not necessarily distinct.
  \ii[(c)] If $x, y \in S$ and $x+y \in S$,
  then $x-y \in S$ too.
\end{enumerate}
Prove that $S = \ZZ$.
\end{mdframed}
The idea is to show any linear combination of the $a_i$ are in $S$,
which implies (by Bezout) that $S = \ZZ$.
This is pretty intuitive, but the details require some care
(in particular there is a parity obstruction at the second lemma).

First, we make the following simple observations:
\begin{itemize}
  \ii $0 \in S$, by putting $i=j=1$ in (b).
  \ii $s \in S \iff -s \in S$, by putting $x=0$ in (c).
\end{itemize}

Now, we prove that:
\begin{lemma*}
  For any integers $c$, $d$, and indices $i$, $j$,
  we have $ca_i + da_j \in S$.
\end{lemma*}
\begin{proof}
  We will assume $c,d > 0$ since the other cases are analogous.
  In that case it follows by induction on $c+d$
  for example $ca_i + (d-1)a_j$, $a_j$, $ca_i + da_j$ in $S$
  implies $ca_i + (d+1)a_j \in S$.
\end{proof}

\begin{lemma*}
  For any nonzero integers $c_1$, $c_2$, \dots, $c_m$,
  and any distinct indices $\{i_1 , i_2 , \dots , i_m\}$, we have
  \[ \sum_k c_k a_{i_k} \in S. \]
\end{lemma*}
\begin{proof}
  By induction on $m$, with base case $m \le 2$ already done.

  For the inductive step,
  we will assume that $i_1 = 1$, $i_2 = 2$, et cetera,
  for notational convenience.
  The proof is then split into two cases.


  \emph{First Case}: some $c_i$ is even.
  WLOG $c_1 \neq 0$ is even and note that
  \begin{align*}
    x &\coloneqq \half c_1 a_1 + \sum_{k \ge 3} c_k a_k \in S \\
    y &\coloneqq -\half c_1 a_1 - c_2 a_2 \in S \\
    x+y &= -c_2 a_2 + \sum_{k \ge 3} c_k a_k \in S \\
    \implies x-y &= \sum_{k \ge 1} c_k a_k \in S.
  \end{align*}


  \emph{Second Case}: all $c_i$ are odd.
  We reduce this to the first case as follows.
  Let $u = \frac{a_1}{\gcd(a_1, a_2)}$ and $v = \frac{a_2}{\gcd(a_1, a_2)}$.
  Then $\gcd(u,v) = 1$ and so WLOG $u$ is odd.
  Then
  \[
    c_1 a_1 + c_2 a_2
    = (c_1 + v)a_1 + (c_2 - u) a_2
  \]
  and so we can replace our given combination
  by $(c_1+v)a_1 + (c_2-u)a_2 + c_3a_3 + \dots$
  which now has an even coefficient for $a_2$.
\end{proof}

We then apply the lemma at $m = n$;
this implies the result since Bezout's lemma
implies that $\sum c_i a_i = 1$ for some choice of $c_i \in \ZZ$.
\pagebreak

\subsection{USAMO 2004/3, proposed by Ricky Liu}
\textsl{Available online at \url{https://aops.com/community/p17441}.}
\begin{mdframed}[style=mdpurplebox,frametitle={Problem statement}]
For what real values of $k > 0$ is it possible to
dissect a $1 \times k$ rectangle
into two similar but noncongruent polygons?
\end{mdframed}
Answer: the dissection is possible for every $k > 0$ except for $k = 1$.
% Here a ``dissection'' does not need to be a single cut.

\textbf{Construction}.
By symmetry it suffices to give a construction for $k > 1$
(since otherwise we replace $k$ by $k\inv$).
For every integer $n \ge 2$ and real number $r > 1$,
we define a shape $\mathcal R(n,r)$ as follows.
\begin{itemize}
  \ii We start with a rectangle of width $1$ and height $r$.
  To its left, we glue a rectangle of height $r$ and width $r^2$ to its left.
  \ii Then, we glue a rectangle of width $1+r^2$ and height $r^3$ below our figure,
  followed by a rectangle of height $r+r^3$ and width $r^4$ to the left of our figure.
  \ii Next, we glue a rectangle of width $1+r^2+r^4$ and height $r^5$ below our figure,
  followed by a rectangle of height $r+r^3+r^5$ and width $r^6$ to the left of our figure.
\end{itemize}
\dots and so on, up until we have put $2n$ rectangles together.
The picture $\mathcal R(3,r)$ is shown below as an example.
\begin{center}
\begin{asy}
size(8cm);
defaultpen(fontsize(10pt));

real r = 1.6;

pair A = (0,r^3+r);
pair B = (r^4,r^3+r);
pair C = (r^4+r^2,r^3+r);
pair D = (r^4+r^2+1,r^3+r);
pair E = (r^4,r^3);
pair F = (r^4+r^2, r^3);
pair G = (r^4+r^2+1, r^3);
pair O = (0,0);
pair H = (r^4,0);
pair I = (r^4+r^2+1,0);
pair W = (-r^6, A.y);
pair X = (W.x,-r^5);
pair Y = (0,-r^5);
pair Z = (I.x,-r^5);

pen p = black+1.3;
pen q = red;
fill(O--A--B--H--cycle, palecyan);
fill(B--C--F--E--cycle, palecyan);
fill(W--A--Y--X--cycle, palecyan);
fill(C--D--G--F--cycle, palegreen);
fill(E--G--I--H--cycle, palegreen);
fill(O--Y--Z--I--cycle, palegreen);

draw(O--A, q);
draw(B--E, q);
draw(F--G, q);
draw(H--I, q);
draw(W--D--Z--X--cycle, p);
draw(C--F--E--H--O--Y, p);

label("$1$", (C+D)/2, dir(90));
label("$r^2$", (B+C)/2, dir(90));
label("$r^4$", (A+B)/2, dir(90));
label("$r^6$", (W+A)/2, dir(90));

label("$r^1$", (D+G)/2, dir(0));
label("$r^3$", (G+I)/2, dir(0));
label("$r^5$", (I+Z)/2, dir(0));
\end{asy}
\end{center}
Observe that by construction,
the entire shape $\mathcal R(n,r)$ is a rectangle
which consists of two similar ``staircase'' polygons
(which are not congruent, since $r>1$).
Note that $\mathcal R(n,r)$ is similar to a $1 \times f_n(r)$ rectangle
where $f_n(r)$ is the aspect ratio of $\mathcal R(n,r)$, defined by
\[ f_n(r) = \frac{1+r^2+\dots+r^{2n}}{r+r^3+\dots+r^{2n-1}}
  = r + \frac{1}{r+r^3+\dots+r^{2n-1}}.  \]

We claim that this is enough.
Indeed for each fixed $n$, note that
\[ \lim_{r \to 1^+} f_n(r) = 1 + \frac1n
  \; \text{ and } \;
  \lim_{r \to \infty} f_n(r) = \infty. \]
Since $f_n$ is continuous, it achieves all values greater than $1+\frac1n$.
Thus by taking sufficiently large $n$ (such that $k > 1+\frac1n$),
we obtain a valid construction for any $k > 1$.

\bigskip

\textbf{Proof of impossibility for a square}.
Now we show that $k = 1$ is impossible (the tricky part!).
Suppose we have a squared dissected into
two similar polygons $\mathcal P \sim \mathcal Q$.
Let $\Gamma$ be their common boundary.
By counting the number of sides of $\mathcal P$ and $\mathcal Q$
we see $\Gamma$ must run from one side of the square to an opposite side
(possibly ending at a corner of the square).
We orient the figure so $\Gamma$ runs from north to south,
with $\mathcal P$ to the west and $\mathcal Q$ to the east.
\begin{center}
\begin{asy}
size(4cm);
pair A = (0,0);
pair B = (1,0);
pair C = (1,1);
pair D = (0,1);
pair W = (0.4,0);
pair X = (0.3,0.7);
pair Y = (0.7,0.3);
pair Z = (0.6,1);
filldraw(A--W--X--Y--Z--D--cycle, palecyan, black);
filldraw(B--W--X--Y--Z--C--cycle, palegreen, black);
draw( W--X--Y--Z, black+1 );
label("$\Gamma$", Y, dir(15));
label("$\mathcal P$", (0.1,0.5), blue);
label("$\mathcal Q$", (0.9,0.5), deepgreen);
\end{asy}
\end{center}

Let $s$ be the longest length of a segment in $\Gamma$.
\begin{claim*}
  The longest side length of $\mathcal P$ is $\max(s,1)$.
  Similarly, the longest side length of $\mathcal Q$ is $\max(s,1)$ as well.
\end{claim*}
\begin{proof}
  The only edges of $\mathcal P$ not in $\Gamma$
  are the west edge of our original square, which has length $1$,
  and the north/south edges of $\mathcal P$ (if any),
  which have length at most $1$.
  An identical argument works for $\mathcal Q$.
\end{proof}
It follows the longest sides of $\mathcal P$ and $\mathcal Q$ have the same length!
Hence the two polygons are in fact congruent, ending the proof.
\pagebreak

\section{Solutions to Day 2}
\subsection{USAMO 2004/4, proposed by Melanie Wood}
\textsl{Available online at \url{https://aops.com/community/p17438}.}
\begin{mdframed}[style=mdpurplebox,frametitle={Problem statement}]
Alice and Bob play a game on a $6$ by $6$ grid.
On his turn, a player chooses a rational number not yet appearing in the grid
and writes it in an empty square of the grid.
Alice goes first and then the players alternate.
When all squares have numbers written in them, in each row,
the square with the greatest number in that row is colored black.
Alice wins if he can then draw a line from the top of the grid to the bottom of the grid
that stays in black squares, and Bob wins if he can't.
(If two squares share a vertex,
Alice can draw a line from one to the other that stays in those two squares.)
Find, with proof, a winning strategy for one of the players.
\end{mdframed}
Bob can win.
Label the first two rows as follows:
\[
  \begin{bmatrix}
    a & b & c & d & e & f \\
    d' & e' & f' & a' & b' & c'
  \end{bmatrix}
\]
These twelve boxes thus come in six \emph{pairs}, $(a, a')$, $(b, b')$ and so on.

\begin{claim*}
  Bob can ensure that the order relation
  of the labels is the same between the two rows,
  meaning that $a < b$ if and only if $a' < b'$, and so on.
\end{claim*}
\begin{proof}
  If Alice plays $q$ in some box in the first two rows, then Bob can plays $q + \eps$
  in the corresponding box in the same pair, for some sufficiently small $\eps$
  (in terms of the existing numbers).

  When Alice writes a number in any other row, Bob writes anywhere in rows $3$ to $6$.
\end{proof}

Under this strategy the black squares in the first two rows will be a pair
and therefore will not touch, so Bob wins.
\pagebreak

\subsection{USAMO 2004/5, proposed by Titu Andreescu}
\textsl{Available online at \url{https://aops.com/community/p17397}.}
\begin{mdframed}[style=mdpurplebox,frametitle={Problem statement}]
Let $a$, $b$, $c$ be positive reals.
Prove that
\[ ( a^5-a^2+3 )( b^5-b^2+3 )( c^5-c^2+3 )
  \ge \left( a+b+c \right)^3. \]
\end{mdframed}
Observe that for all real numbers $a$, the inequality
\[ a^5-a^2+3 \ge a^3+2 \]
holds.
Then the problem follows by H\"{o}lder in the form
\[ (a^3+1+1)(1+b^3+1)(1+1+c^3) \ge (a+b+c)^3. \]
\pagebreak

\subsection{USAMO 2004/6, proposed by Zuming Feng}
\textsl{Available online at \url{https://aops.com/community/p17437}.}
\begin{mdframed}[style=mdpurplebox,frametitle={Problem statement}]
A circle $\omega$ is inscribed in a quadrilateral $ABCD$.
Let $I$ be the center of $\omega$.
Suppose that \[ (AI + DI)^2 + (BI + CI)^2 = (AB + CD)^2.  \]
Prove that $ABCD$ is an isosceles trapezoid.
\end{mdframed}
Here's a completely algebraic solution.
WLOG $\omega$ has radius $1$, and let $a$, $b$, $c$, $d$ be the
lengths of the tangents from $A$, $B$, $C$, $D$ to $\omega$.
It is known that
\[ a+b+c+d = abc+bcd+cda+dab \qquad (\star) \]
which can be proved by, say $\tan$-addition formula.
Then, the content of the problem is to show that
\[
  (\sqrt{a^2+1}+\sqrt{d^2+1})^2
  + (\sqrt{b^2+1}+\sqrt{c^2+1})^2
  \le (a+b+c+d)^2
\]
subject to $(\star)$,
with equality only when $a=d=\tfrac1b=\tfrac1c$.

Let $S = ab+bc+cd+da+ac+bd$. Then the inequality is
\[ \sqrt{(a^2+1)(d^2+1)} + \sqrt{(b^2+1)(c^2+1)} \le S - 2. \]
Now, by \textbf{USAMO 2014 Problem 1} and the condition $(\star)$,
we have that $(a^2+1)(b^2+1)(c^2+1)(d^2+1) = (S - abcd - 1)^2$.
So squaring both sides, the inequality becomes
\[ (ad)^2+(bc)^2 + a^2+b^2+c^2+d^2 \le S^2 - 6S + 2abcd + 4. \]
To simplify this, we use the identities
\begin{align*}
  S^2 &= 6abcd + \sum_{\text{sym}} a^2bc
    + \frac14\sum_{\text{sym}} a^2b^2 \\
  (a+b+c+d)^2 &= (abc+bcd+cda+dab)(a+b+c+d) \\
  &= 4abcd + \frac12\sum_{\text{sym}} a^2bc
\end{align*}
So $S^2+2abcd = \frac14\textstyle\sum_{\text{sym}}
  a^2b^2 + 2(a^2+b^2+c^2+d^2) + 4S$
and the inequality we want to prove reduces to
\[ 2S \le (ab)^2+(ac)^2+(bd)^2+(cd)^2 + 4 + a^2 + b^2 + c^2 + d^2. \]
This follows by AM-GM since
\begin{align*}
  (ab)^2 + 1 &\ge 2ab \\
  (ac)^2 + 1 &\ge 2ac \\
  (bd)^2 + 1 &\ge 2bd \\
  (cd)^2 + 1 &\ge 2cd \\
  a^2 + d^2 &\ge 2ad \\
  b^2 + c^2 &\ge 2bc.
\end{align*}
The equality case is when $ab=ac=bd=cd=1$, $a=d$, $b=c$,
as needed to imply an isosceles trapezoid.

\begin{remark*}
Note that a priori one expects an inequality.
Indeed,
\begin{itemize}
  \ii Quadrilaterals with incircles have four degrees of freedom.
  \ii There is one condition imposed.
  \ii Isosceles trapezoid with incircles have two degrees of freedom.
\end{itemize}
\end{remark*}
\pagebreak


\end{document}
