% © Evan Chen
% Downloaded from https://web.evanchen.cc/

\documentclass[11pt]{scrartcl}
\usepackage[sexy]{evan}
\ihead{\footnotesize\textbf{\thetitle}}
\ohead{\footnotesize\href{http://web.evanchen.cc}{\ttfamily web.evanchen.cc},
    updated \today}
\title{JMO 2020 Solution Notes}
\date{\today}

\begin{document}

\maketitle

\begin{abstract}
This is a compilation of solutions
for the 2020 JMO.
Some of the solutions are my own work,
but many are from the official solutions provided by the organizers
(for which they hold any copyrights),
and others were found by users on the Art of Problem Solving forums.

These notes will tend to be a bit more advanced and terse than the ``official''
solutions from the organizers.
In particular, if a theorem or technique is not known to beginners
but is still considered ``standard'', then I often prefer to
use this theory anyways, rather than try to work around or conceal it.
For example, in geometry problems I typically use directed angles
without further comment, rather than awkwardly work around configuration issues.
Similarly, sentences like ``let $\mathbb{R}$ denote the set of real numbers''
are typically omitted entirely.

Corrections and comments are welcome!
\end{abstract}

\tableofcontents
\newpage

\addtocounter{section}{-1}
\section{Problems}
\begin{enumerate}[\bfseries 1.]
\item %% Problem 1
Let $n \ge 2$ be an integer.
Carl has $n$ books arranged on a bookshelf.
Each book has a height and a width.
No two books have the same height,
and no two books have the same width.

Initially, the books are arranged in
increasing order of height from left to right.
In a \emph{move}, Carl picks any two adjacent books
where the left book is wider and shorter than the right book,
and swaps their locations.
Carl does this repeatedly until no further moves are possible.

Prove that regardless of how Carl makes his moves,
he must stop after a finite number of moves, and when he does stop,
the books are sorted in increasing order of width from left to right.

\item %% Problem 2
Let $\omega$ be the incircle of a
fixed equilateral triangle $ABC$.
Let $\ell$ be a variable line that is tangent to $\omega$
and meets the interior of segments $BC$ and $CA$
at points $P$ and $Q$, respectively.
A point $R$ is chosen such that $PR=PA$ and $QR=QB$.
Find all possible locations of the point $R$,
over all choices of $\ell$.

\item %% Problem 3
An empty $2020 \times 2020 \times 2020$ cube is given,
and a $2020 \times 2020$ grid of square unit cells is drawn on each of its six faces.
A \emph{beam} is a $1 \times 1 \times 2020$ rectangular prism.
Several beams are placed inside the cube subject to the following conditions:
\begin{itemize}
\item The two $1 \times 1$ faces of each beam coincide
  with unit cells lying on opposite faces of the cube.
  (Hence, there are $3 \cdot 2020^2$ possible positions for a beam.)
\item No two beams have intersecting interiors.
\item The interiors of each of the four $1 \times 2020$ faces of each beam touch
  either a face of the cube or the interior of the face of another beam.
\end{itemize}
What is the smallest positive number of beams that can be placed to satisfy these conditions?

\item %% Problem 4
Let $ABCD$ be a convex quadrilateral inscribed in a circle and satisfying
\[ DA < AB = BC < CD. \]
Points $E$ and $F$ are chosen on sides $CD$ and $AB$
such that $\ol{BE} \perp \ol{AC}$ and $\ol{EF} \parallel \ol{BC}$.
Prove that $FB=FD$.

\item %% Problem 5
Suppose that $(a_1, b_1)$, $(a_2, b_2)$, \dots, $(a_{100}, b_{100})$
are distinct ordered pairs of nonnegative integers.
Let $N$ denote the number of pairs of integers $(i,j)$ satisfying
$1 \le i < j \le 100$ and $\left\lvert a_ib_j - a_jb_i \right\rvert = 1$.
Determine the largest possible value of $N$
over all possible choices of the $100$ ordered pairs.

\item %% Problem 6
Let $n \ge 2$ be an integer.
Let $P(x_1, x_2, \dots, x_n)$ be a nonconstant
$n$-variable polynomial with real coefficients.
Assuming that $P$ vanishes whenever two of its arguments are equal,
prove that $P$ has at least $n!$ terms.

\end{enumerate}
\pagebreak

\section{Solutions to Day 1}
\subsection{JMO 2020/1, proposed by Milan Haiman}
\textsl{Available online at \url{https://aops.com/community/p15952780}.}
\begin{mdframed}[style=mdpurplebox,frametitle={Problem statement}]
Let $n \ge 2$ be an integer.
Carl has $n$ books arranged on a bookshelf.
Each book has a height and a width.
No two books have the same height,
and no two books have the same width.

Initially, the books are arranged in
increasing order of height from left to right.
In a \emph{move}, Carl picks any two adjacent books
where the left book is wider and shorter than the right book,
and swaps their locations.
Carl does this repeatedly until no further moves are possible.

Prove that regardless of how Carl makes his moves,
he must stop after a finite number of moves, and when he does stop,
the books are sorted in increasing order of width from left to right.
\end{mdframed}
We say that a pair of books $(A,B)$ is \emph{height-inverted}
if $A$ is to the left of $B$ and taller than $A$.
Similarly define \emph{width-inverted} pairs.

Note that every operation decreases the number of width-inverted pairs.
This proves the procedure terminates,
since the number of width-inverted pairs starts at $\binom n2$
and cannot increase indefinitely.

Now consider a situation where no more moves are possible.
Assume for contradiction two consecutive books $(A,B)$ are still width-inverted.
Since the operation isn't possible anymore, they are also height-inverted.
In particular, the operation could never have swapped $A$ and $B$.
But this contradicts the assumption there were no height-inverted pairs initially.
\pagebreak

\subsection{JMO 2020/2, proposed by Titu Andreescu, Waldemar Pompe}
\textsl{Available online at \url{https://aops.com/community/p15952801}.}
\begin{mdframed}[style=mdpurplebox,frametitle={Problem statement}]
Let $\omega$ be the incircle of a
fixed equilateral triangle $ABC$.
Let $\ell$ be a variable line that is tangent to $\omega$
and meets the interior of segments $BC$ and $CA$
at points $P$ and $Q$, respectively.
A point $R$ is chosen such that $PR=PA$ and $QR=QB$.
Find all possible locations of the point $R$,
over all choices of $\ell$.
\end{mdframed}
Let $r$ be the inradius.
Let $T$ be the tangency point of $\ol{PQ}$
on arc $\widehat{DE}$ of the incircle, which we consider varying.
We define $R_1$ and $R_2$ to be the two intersections
of the circle centered at $P$ with radius $PA$,
and the circle centered at $Q$ with radius $QB$.
We choose $R_1$ to lie on the opposite side of $C$ as line $PQ$.

\begin{center}
\begin{asy}
size(11cm);
pair A = dir(150);
pair B = dir(30);
pair C = dir(270);
pair D = midpoint(B--C);
pair E = midpoint(A--C);
pair I = incenter(A, B, C);
draw(incircle(A, B, C), blue);
pair R_1 = dir(70);
pair P = extension(midpoint(A--R_1), I, B, C);
pair Q = extension(midpoint(B--R_1), I, A, C);
pair T = extension(R_1, I, P, Q);
filldraw(A--B--C--cycle, opacity(0.1)+yellow, blue);
draw(P--Q, blue);

pair Ap = 4*D;
pair Bp = 4*E;
pair R_2 = 2*T-R_1;

draw(R_2--P--R_1, grey);
draw(P--A, grey);
draw(R_2--R_1, heavycyan);
draw(A--Ap, deepgreen);
draw(B--Bp, deepgreen);

draw(arc(I,B,A), red+1);
draw(arc(I,Bp,Ap), red+1);

dot("$A$", A, dir(A));
dot("$B$", B, dir(B));
dot("$C$", C, dir(C));
dot("$D$", D, dir(10));
dot("$E$", E, dir(170));
dot("$I$", I, dir(100));
dot("$R_1$", R_1, dir(R_1));
dot("$P$", P, dir(P));
dot("$Q$", Q, dir(Q));
dot("$T$", T, dir(290));
dot("$A'$", Ap, dir(Ap));
dot("$B'$", Bp, dir(Bp));
dot("$R_2$", R_2, dir(R_2));

/* TSQ Source:

!size(11cm);
A = dir 150
B = dir 30
C = dir 270
D = midpoint B--C R10
E = midpoint A--C R170
I = incenter A B C R100
incircle A B C blue
R_1 = dir 70
P = extension midpoint A--R_1 I B C
Q = extension midpoint B--R_1 I A C
T = extension R_1 I P Q R290
A--B--C--cycle 0.1 yellow / blue
P--Q blue

A' = 4*D
B' = 4*E
R_2 = 2*T-R_1

R_2--P--R_1 grey
P--A grey
R_2--R_1 heavycyan
A--Ap deepgreen
B--Bp deepgreen

!draw(arc(I,B,A), red+1);
!draw(arc(I,Bp,Ap), red+1);

*/
\end{asy}
\end{center}

\begin{claim*}
  The point $R_1$ is the unique point on ray $TI$ with $R_1I = 2r$.
\end{claim*}
\begin{proof}
  Define $S$ to be the point on ray $TI$ with $SI = 2r$.
  Note that there is a homothety at $I$ which maps $\triangle DTE$
  to $\triangle ASB$, for some point $S$.

  Note that since $TASD$ is an isosceles trapezoid,
  it follows $PA = PS$.
  Similarly, $QB = QS$.
  So it follows that $S = R_1$.
\end{proof}

Since $T$ can be any point on the open arc $\widehat{DE}$,
it follows that the locus of $R_1$
is exactly the open $120\dg$ arc of $\widehat{AB}$
of the circle centered at $I$ with radius $2r$
(i.e.\ the circumcircle of $ABC$).

It remains to characterize $R_2$.
Since $TI = r$, $IR_1 = 2r$, it follows $TR_2 = 3r$ and $IR_2 = 4r$.
Define $A'$ on ray $DI$ such that $A'I = 4r$,
and $B'$ on ray $IE$ such that $B'I = 4r$.
Then it follows, again by homothety,
that the locus of $R_2$ is the $120\dg$ arc $\widehat{A'B'}$
of the circle centered at $I$ with radius $4r$.

In conclusion, the locus of $R$ is the two open $120\dg$ arcs we identified.
\pagebreak

\subsection{JMO 2020/3, proposed by Alex Zhai}
\textsl{Available online at \url{https://aops.com/community/p15952773}.}
\begin{mdframed}[style=mdpurplebox,frametitle={Problem statement}]
An empty $2020 \times 2020 \times 2020$ cube is given,
and a $2020 \times 2020$ grid of square unit cells is drawn on each of its six faces.
A \emph{beam} is a $1 \times 1 \times 2020$ rectangular prism.
Several beams are placed inside the cube subject to the following conditions:
\begin{itemize}
\item The two $1 \times 1$ faces of each beam coincide
  with unit cells lying on opposite faces of the cube.
  (Hence, there are $3 \cdot 2020^2$ possible positions for a beam.)
\item No two beams have intersecting interiors.
\item The interiors of each of the four $1 \times 2020$ faces of each beam touch
  either a face of the cube or the interior of the face of another beam.
\end{itemize}
What is the smallest positive number of beams that can be placed to satisfy these conditions?
\end{mdframed}
Answer: $3030$ beams.

\medskip

\textbf{Construction}:
We first give a construction with $3n/2$ beams for any $n \times n \times n$ box,
where $n$ is an even integer.
Shown below is the construction for $n=6$, which generalizes.
(The left figure shows the cube in 3d;
the right figure shows a direct view of the three visible faces.)
\begin{center}
\begin{asy}
import three;
import bsp;

size(8cm);
settings.prc = false;
settings.render = 0;
settings.tex="pdflatex";
settings.outformat="pdf";

triple O = (9,7,5);
currentprojection = orthographic(O);
currentlight = light(gray(2),specular=gray(0.7), specularfactor=3, (7,7,5));

pen finalpen = yellow+2;
pen meshpenlight = rgb(0.95,0.95,0.99);

int M = 6;
int s = 0;
for (int i=0; i<=M; ++i) {
draw( (s,i,0)--(s,i,M), meshpenlight);
draw( (s,0,i)--(s,M,i), meshpenlight);
draw( (i,s,0)--(i,s,M), meshpenlight);
draw( (0,s,i)--(M,s,i), meshpenlight);
draw( (i,0,s)--(i,M,s), meshpenlight);
draw( (0,i,s)--(M,i,s), meshpenlight);
}

draw(box((0,0,0), (M,M,M)), finalpen);

pen penA = red;
pen penB = cyan;
pen penC = green;

draw(shift(0,0,0)*unitcube, penA);
draw(shift(0,0,1)*unitcube, penA);
draw(shift(0,0,2)*unitcube, penA);
draw(shift(0,0,3)*unitcube, penA);
draw(shift(0,0,4)*unitcube, penA);
draw(shift(0,0,5)*unitcube, penA);

draw(shift(1,0,0)*unitcube, penB);
draw(shift(1,1,0)*unitcube, penB);
draw(shift(1,2,0)*unitcube, penB);
draw(shift(1,3,0)*unitcube, penB);
draw(shift(1,4,0)*unitcube, penB);
draw(shift(1,5,0)*unitcube, penB);

draw(shift(0,1,1)*unitcube, penC);
draw(shift(1,1,1)*unitcube, penC);
draw(shift(2,1,1)*unitcube, penC);
draw(shift(3,1,1)*unitcube, penC);
draw(shift(4,1,1)*unitcube, penC);
draw(shift(5,1,1)*unitcube, penC);

draw(shift(2,2,0)*unitcube, penA);
draw(shift(2,2,1)*unitcube, penA);
draw(shift(2,2,2)*unitcube, penA);
draw(shift(2,2,3)*unitcube, penA);
draw(shift(2,2,4)*unitcube, penA);
draw(shift(2,2,5)*unitcube, penA);

draw(shift(3,0,2)*unitcube, penB);
draw(shift(3,1,2)*unitcube, penB);
draw(shift(3,2,2)*unitcube, penB);
draw(shift(3,3,2)*unitcube, penB);
draw(shift(3,4,2)*unitcube, penB);
draw(shift(3,5,2)*unitcube, penB);

draw(shift(0,3,3)*unitcube, penC);
draw(shift(1,3,3)*unitcube, penC);
draw(shift(2,3,3)*unitcube, penC);
draw(shift(3,3,3)*unitcube, penC);
draw(shift(4,3,3)*unitcube, penC);
draw(shift(5,3,3)*unitcube, penC);

draw(shift(4,4,0)*unitcube, penA);
draw(shift(4,4,1)*unitcube, penA);
draw(shift(4,4,2)*unitcube, penA);
draw(shift(4,4,3)*unitcube, penA);
draw(shift(4,4,4)*unitcube, penA);
draw(shift(4,4,5)*unitcube, penA);

draw(shift(5,0,4)*unitcube, penB);
draw(shift(5,1,4)*unitcube, penB);
draw(shift(5,2,4)*unitcube, penB);
draw(shift(5,3,4)*unitcube, penB);
draw(shift(5,4,4)*unitcube, penB);
draw(shift(5,5,4)*unitcube, penB);

draw(shift(0,5,5)*unitcube, penC);
draw(shift(1,5,5)*unitcube, penC);
draw(shift(2,5,5)*unitcube, penC);
draw(shift(3,5,5)*unitcube, penC);
draw(shift(4,5,5)*unitcube, penC);
draw(shift(5,5,5)*unitcube, penC);

pen meshpen = grey;
int s = M;
for (int i=1; i<=M-1; ++i) {
draw( (s,i,0)--(s,i,M), meshpen);
draw( (s,0,i)--(s,M,i), meshpen);
draw( (i,s,0)--(i,s,M), meshpen);
draw( (0,s,i)--(M,s,i), meshpen);
draw( (i,0,s)--(i,M,s), meshpen);
draw( (0,i,s)--(M,i,s), meshpen);
}

draw((M,M,M)--(0,M,M), finalpen);
draw((M,M,M)--(M,0,M), finalpen);
draw((M,M,M)--(M,M,0), finalpen);
\end{asy}
\quad
\begin{asy}
size(8cm);
picture redface, greenface, blueface;

void draw_grid(picture pic) {
  for (int i=1; i<6; ++i) {
    draw(pic, (i,0)--(i,6), grey);
    draw(pic, (0,i)--(6,i), grey);
  }
  draw(pic, scale(6)*unitsquare, yellow+2);
}

fill(redface, shift(0,5)*unitsquare, red);
fill(redface, shift(2,3)*unitsquare, red);
fill(redface, shift(4,1)*unitsquare, red);
draw_grid(redface);

fill(greenface, shift(5,5)*unitsquare, green);
fill(greenface, shift(3,3)*unitsquare, green);
fill(greenface, shift(1,1)*unitsquare, green);
draw_grid(greenface);

fill(blueface, shift(0,4)*unitsquare, cyan);
fill(blueface, shift(2,2)*unitsquare, cyan);
fill(blueface, shift(4,0)*unitsquare, cyan);
draw_grid(blueface);

add(shift(0,8.5)*shift(3,3)*rotate(-45)*shift(-3,-3)*redface);
add(shift(-4,0)*greenface);
add(shift( 4,0)*blueface);
label("Left face", (-1,0), dir(-90));
label("Right face", (7,0), dir(-90));
label("Top face", (3,16), dir(90));
\end{asy}
\end{center}
To be explicit,
impose coordinate axes such that one corner of the cube is the origin.
We specify a beam by two opposite corners.
The $3n/2$ beams come in three directions, $n/2$ in each direction:
\begin{itemize}
  \ii $(0,0,0) \to (1,1,n)$, $(2,2,0) \to (3,3,n)$, $(4,4,0) \to (5,5,n)$, and so on;
  \ii $(1,0,0) \to (2,n,1)$, $(3,0,2) \to (4,n,3)$, $(5,0,4) \to (6,n,5)$, and so on;
  \ii $(0,1,1) \to (n,2,2)$, $(0,3,3) \to (n,4,4)$, $(0,5,5) \to (n,6,6)$, and so on.
\end{itemize}
This gives the figure we drew earlier and shows $3030$ beams is possible.

\bigskip

\textbf{Necessity}:
We now show at least $3n/2$ beams are necessary.
Maintain coordinates,
and call the beams $x$-beams, $y$-beams, $z$-beams
according to which plane their long edges are perpendicular too.
Let $N_x$, $N_y$, $N_z$ be the number of these.

\begin{claim*}
  If $\min(N_x, N_y, N_z) = 0$, then at least $n^2$ beams are needed.
\end{claim*}
\begin{proof}
  Assume WLOG that $N_z = 0$.
  Orient the cube so the $z$-plane touches the ground.
  Then each of the $n$ layers of the cube (from top to bottom)
  must be completely filled, and so at least $n^2$ beams are necessary,
\end{proof}

We henceforth assume $\min(N_x, N_y, N_z) > 0$.

\begin{claim*}
  If $N_z > 0$, then we have $N_x + N_y \ge n$.
\end{claim*}
\begin{proof}
  Again orient the cube so the $z$-plane touches the ground.
  We see that for each of the $n$ layers of the cube (from top to bottom),
  there is at least one $x$-beam or $y$-beam.
  (Pictorially, some of the $x$ and $y$ beams form a ``staircase''.)
  This completes the proof.
\end{proof}
Proceeding in a similar fashion, we arrive at the three relations
\begin{align*}
  N_x + N_y &\ge n \\
  N_y + N_z &\ge n \\
  N_z + N_x &\ge n.
\end{align*}
Summing gives $N_x  + N_y + N_z \ge 3n/2$ too.

\begin{remark*}
  The problem condition has the following ``physics'' interpretation.
  Imagine the cube is a metal box which is sturdy enough that
  all beams must remain orthogonal to the faces of the box
  (i.e.\ the beams cannot spin).
  Then the condition of the problem is exactly what is needed so that,
  if the box is shaken or rotated, the beams will not move.
\end{remark*}

\begin{remark*}
  Walter Stromquist points out that the number of constructions
  with $3030$ beams is actually enormous:
  not dividing out by isometries,
  the number is $(2 \cdot 1010!)^3$.
\end{remark*}
\pagebreak

\section{Solutions to Day 2}
\subsection{JMO 2020/4, proposed by Milan Haiman}
\textsl{Available online at \url{https://aops.com/community/p15952890}.}
\begin{mdframed}[style=mdpurplebox,frametitle={Problem statement}]
Let $ABCD$ be a convex quadrilateral inscribed in a circle and satisfying
\[ DA < AB = BC < CD. \]
Points $E$ and $F$ are chosen on sides $CD$ and $AB$
such that $\ol{BE} \perp \ol{AC}$ and $\ol{EF} \parallel \ol{BC}$.
Prove that $FB=FD$.
\end{mdframed}
We present three approaches.
We note that in the second two approaches,
the result remains valid even if $AB \neq BC$,
as long $E$ is replaced by the point on $\ol{AC}$
satisfying $EA = EC$.
So the result is actually somewhat more general.

\paragraph{First solution by inscribed angle theorem.}
Since $\ol{EF} \parallel \ol{BC}$ we may
set $\theta = \angle FEB = \angle CBE = \angle EBF$.
This already implies $FE = FB$,
so we will in fact prove that $F$ is the circumcenter of $\triangle BED$.
\begin{center}
\begin{asy}
pair A = dir(175);
pair B = dir(270);
pair C = B*B/A;
pair D = dir(135);
pair E = extension(B, origin, C, D);
pair F = circumcenter(B, D, E);
filldraw(unitcircle, opacity(0.1)+lightcyan, blue);
draw(D--F--B, lightred);
draw(F--E, lightred);
draw(B--D--C, blue);
draw(F--A--C, blue);
draw(E--B--C, blue);
draw(A--D, blue);
draw(CP(F, E), lightred);

markangle(1,7.0,E,B,F,deepgreen);
markangle(1,9.0,C,B,E,deepgreen);
markangle(1,9.0,F,E,B,deepgreen);

markangle(2,9.0,B,A,C,deepcyan);
markangle(2,9.0,B,D,C,deepcyan);
markangle(2,9.0,A,C,B,deepcyan);

dot("$A$", A, dir(A));
dot("$B$", B, dir(B));
dot("$C$", C, dir(C));
dot("$D$", D, dir(D));
dot("$E$", E, dir(90));
dot("$F$", F, dir(250));

/* TSQ Source:

A = dir 175
B = dir 270
C = B*B/A
D = dir 135
E = extension B origin C D R90
F = circumcenter B D E R250
unitcircle 0.1 lightcyan / blue
D--F--B lightred
F--E lightred
B--D--C blue
F--A--C blue
E--B--C blue
A--D blue
CP F E lightred

!markangle(1,7.0,E,B,F,deepgreen);
!markangle(1,9.0,C,B,E,deepgreen);
!markangle(1,9.0,F,E,B,deepgreen);

!markangle(2,9.0,B,A,C,deepcyan);
!markangle(2,9.0,B,D,C,deepcyan);
!markangle(2,9.0,A,C,B,deepcyan);

*/
\end{asy}
\end{center}
Note that $\angle BDC = \angle BAC = 90\dg - \theta$.
However, $\angle BFE = 180\dg - 2 \theta$.
So by the inscribed angle theorem, $D$ lies on the circle
centered at $F$ with radius $FE = FB$, as desired.

\begin{remark*}
  Another approach to the given problem is to show that $B$
  is the $D$-excenter of $\triangle DAE$,
  and $F$ is the arc midpoint of $\widehat{DAE}$
  of the circumcircle of $\triangle DAE$.
  In my opinion, this approach is much clumsier.
\end{remark*}

\paragraph{Second general solution by angle chasing.}
By Reim's theorem, $AFED$ is cyclic.
\begin{center}
\begin{asy}
pair A = dir(175);
pair B = dir(255);
pair C = dir(5);
pair D = dir(135);
pair E = extension(B, origin, C, D);
pair F = extension(A, B, E, E+B-C);
filldraw(unitcircle, opacity(0.1)+lightcyan, blue);
draw(D--F, lightblue);
draw(A--E, lightblue);
draw(F--B--D--C, blue);
draw(F--A--C, blue);
draw(E--B--C, blue);
draw(A--D, blue);
draw(E--F, lightred);

draw(circumcircle(E, A, D), lightred);

markangle(2,13.0,C,A,E,deepgreen);
markangle(2,13.0,E,C,A,deepgreen);
markangle(2,13.0,D,B,F,deepgreen);
markangle(2,13.0,F,D,B,deepgreen);

dot("$A$", A, dir(A));
dot("$B$", B, dir(B));
dot("$C$", C, dir(C));
dot("$D$", D, dir(D));
dot("$E$", E, dir(70));
dot("$F$", F, dir(F));

/* TSQ Source:

A = dir 175
B = dir 255
C = dir 5
D = dir 135
E = extension B origin C D R70
F = extension A B E E+B-C
unitcircle 0.1 lightcyan / blue
D--F lightblue
A--E lightblue
F--B--D--C blue
F--A--C blue
E--B--C blue
A--D blue
E--F lightred

circumcircle E A D lightred

!markangle(2,13.0,C,A,E,deepgreen);
!markangle(2,13.0,E,C,A,deepgreen);
!markangle(2,13.0,D,B,F,deepgreen);
!markangle(2,13.0,F,D,B,deepgreen);

*/
\end{asy}
\end{center}
Hence
\begin{align*}
  \dang FDB &= \dang FDC - \dang BDC = \dang FAE - \dang FAC \\
  &= \dang CAE = \dang ECA = \dang DCA = \dang DBA = \dang DBF
\end{align*}
as desired.


\paragraph{Third general solution by Pascal.}
Extend rays $AE$ and $DF$ to meet the circumcircle again at $G$ and $H$.
By Pascal's theorem on $HDCBAG$,
it follows that $E$, $F$, and $GH \cap BC$ are collinear,
which means that $\ol{EF} \parallel \ol{GH} \parallel \ol{BC}$.
\begin{center}
\begin{asy}
pair A = dir(175);
pair B = dir(255);
pair C = dir(5);
pair D = dir(135);
filldraw(unitcircle, opacity(0.1)+lightcyan, blue);
pair H = D*B/A;
pair G = A*C/D;
pair E = extension(A, G, D, C);
pair F = extension(A, B, D, H);

draw(A--B--C--D--cycle, blue);
draw(A--G, deepgreen);
draw(D--H, deepgreen);
draw(E--F, red);
draw(G--H, red);

draw(arc(origin,C,G), orange+1.5);
draw(arc(origin,D,A), orange+1.5);
draw(arc(origin,H,B), orange+1.5);

dot("$A$", A, dir(A));
dot("$B$", B, dir(B));
dot("$C$", C, dir(C));
dot("$D$", D, dir(D));
dot("$H$", H, dir(H));
dot("$G$", G, dir(G));
dot("$E$", E, dir(E));
dot("$F$", F, dir(F));

/* TSQ Source:

A = dir 175
B = dir 255
C = dir 5
D = dir 135
unitcircle 0.1 lightcyan / blue
H = D*B/A
G = A*C/D
E = extension A G D C
F = extension A B D H

A--B--C--D--cycle blue
A--G deepgreen
D--H deepgreen
E--F red
G--H red

!draw(arc(origin,C,G), orange+1.5);
!draw(arc(origin,D,A), orange+1.5);
!draw(arc(origin,H,B), orange+1.5);

*/
\end{asy}
\end{center}
Since $EA=EC$, it follows $DAGC$ in isosceles trapezoid.
But also $GHBC$ is an isosceles trapezoid.
Thus $\mathrm{m}\widehat{DA} = \mathrm{m}\widehat{GC}
= \mathrm{m}\widehat{BH}$,
so $DAHB$ is an isosceles trapezoid.
Thus $FD = FB$.

\begin{remark*}
  Addicts of projective geometry can use Pascal
  on $DBCAHG$ to finish rather than noting the equal arcs.
\end{remark*}
\pagebreak

\subsection{JMO 2020/5, proposed by Ankan Bhattacharya}
\textsl{Available online at \url{https://aops.com/community/p15952792}.}
\begin{mdframed}[style=mdpurplebox,frametitle={Problem statement}]
Suppose that $(a_1, b_1)$, $(a_2, b_2)$, \dots, $(a_{100}, b_{100})$
are distinct ordered pairs of nonnegative integers.
Let $N$ denote the number of pairs of integers $(i,j)$ satisfying
$1 \le i < j \le 100$ and $\left\lvert a_ib_j - a_jb_i \right\rvert = 1$.
Determine the largest possible value of $N$
over all possible choices of the $100$ ordered pairs.
\end{mdframed}
The answer is $197$.
In general, if $100$ is replaced by $n \ge 2$ the answer is $2n-3$.

The idea is that if we let $P_i = (a_i, b_i)$ be a point
in the coordinate plane, and let $O = (0,0)$
then we wish to maximize the number of triangles
$\triangle O P_i P_j$ which have area $1/2$.
Call such a triangle \emph{good}.

\medskip

\textbf{Construction of $197$ points}:
It suffices to use the points
$(1,0)$, $(1,1)$, $(2,1)$, $(3,1)$, \dots, $(99,1)$ as shown.
Notice that:
\begin{itemize}
  \ii There are $98$ good triangles
  with vertices $(0,0)$, $(k,1)$ and $(k+1,1)$ for $k=1, \dots, 98$.

  \ii There are $99$ good triangles
  with vertices $(0,0)$, $(1,0)$ and $(k,1)$ for $k=1, \dots, 99$.
\end{itemize}
This is a total of $98 + 99 = 197$ triangles.
\begin{center}
\begin{asy}
  draw( (6,0)--(0,0)--(0,2), grey, Arrows );
  dot("$O$", (0,0), dir(225), red );
  dot("$(1,0)$", (1,0), dir(-90), blue);
  dot("$(1,1)$", (1,1), dir(90), blue);
  dot("$(2,1)$", (2,1), dir(90), blue);
  dot("$(3,1)$", (3,1), dir(90), blue);
  dot("$(4,1)$", (4,1), dir(90), blue);
  label("$\dotsb$", (5,1), blue);
\end{asy}
\end{center}

\medskip

\textbf{Proof that $197$ points is optimal}:
We proceed by induction on $n$ to show the bound of $2n-3$.
The base case $n=2$ is evident.

For the inductive step,
suppose (without loss of generality) that the point $P = P_n = (a,b)$
is the farthest away from the point $O$ among all points.
\begin{claim*}
  This farthest point $P = P_n$ is part of at most two good triangles.
\end{claim*}
\begin{proof}
  We must have $\gcd(a,b) = 1$ for $P$ to be in any good triangles at all,
  since otherwise any divisor of $\gcd(a,b)$ also divides $2[OPQ]$.
  Now, we consider the locus of all points $Q$ for which $[OPQ] = 1/2$.
  It consists of two parallel lines passing with slope $OP$, as shown.
  \begin{center}
  \begin{asy}
    size(10cm);
    int M = 6;
    pair P = (3,2);
    pair O = (0,0);
    draw(O--P, blue);
    draw(arc(O,abs(P),0,90), deepcyan);
    for (int i=-M#2; i<=M; ++i) {
      draw( (-M#2,i)--(M,i), grey);
      draw( (i,-M#2)--(i,M), grey);
    }
    draw( (-M#2,0)--(M,0), black+1.4, Arrows(TeXHead) );
    draw( (0,-M#2)--(0,M), black+1.4, Arrows(TeXHead) );

    pair X = (1,1);
    pair Y = O+P-X;
    dot("$(u,v)$", X, dir(135), red);
    dot("$(u',v')$", Y, dir(-45), red);

    pair X1 = X+P;
    pair X2 = X-P;
    draw(X1--X2, red+dashed, Arrows);

    pair Y1 = Y+P;
    pair Y2 = Y-P;
    draw(Y1--Y2, red+dashed, Arrows);

    dot("$O$", O, dir(225), blue);
    dot("$P=(a,b)$", P, dir(30), blue);
  \end{asy}
  \end{center}
  Since $\gcd(a,b)=1$, see that only two lattice points on this locus
  actually lie inside the quarter-circle centered at $O$ with radius $OP$.
  Indeed if one of the points is $(u,v)$
  then the others on the line are $(u \pm a, v \pm b)$ where the signs match.
  This proves the claim.
\end{proof}
This claim allows us to complete the induction by simply deleting $P_n$.
\pagebreak

\subsection{JMO 2020/6, proposed by Ankan Bhattacharya}
\textsl{Available online at \url{https://aops.com/community/p15952921}.}
\begin{mdframed}[style=mdpurplebox,frametitle={Problem statement}]
Let $n \ge 2$ be an integer.
Let $P(x_1, x_2, \dots, x_n)$ be a nonconstant
$n$-variable polynomial with real coefficients.
Assuming that $P$ vanishes whenever two of its arguments are equal,
prove that $P$ has at least $n!$ terms.
\end{mdframed}
We present two solutions.

\paragraph{First solution using induction (by Ankan).}
Begin with the following observation:
\begin{claim*}
  Let $1 \le i < j \le n$.
  There is no term of $P$ which omits both $x_i$ and $x_j$.
\end{claim*}
\begin{proof}
  Note that $P$ ought to become identically zero
  if we set $x_i = x_j = 0$,
  since it is zero for any choice of the remaining $n-2$ variables,
  and the base field $\RR$ is infinite.
\end{proof}
\begin{remark*}
  [Technical warning for experts]
  The fact we used is not true if $\RR$ is replaced by a field
  with finitely many elements, such as $\FF_p$,
  even with one variable.
  For example the one-variable polynomial $X^p-X$ vanishes
  on every element of $\FF_p$,
  by Fermat's little theorem.
\end{remark*}

We proceed by induction on $n \ge 2$ with the base case $n=2$ being clear.
Assume WLOG $P$ is not divisible by any of $x_1$, \dots, $x_n$,
since otherwise we may simply divide out this factor.
Now for the inductive step, note that
\begin{itemize}
  \ii The polynomial $P(0, x_2, x_3, \dots, x_n)$
  obviously satisfies the inductive hypothesis
  and is not identically zero since $x_1 \nmid P$,
  so it has at least $(n-1)!$ terms.
  \ii Similarly, $P(x_1, 0, x_3, \dots, x_n)$
  also has at least $(n-1)!$ terms.
  \ii Similarly, $P(x_1, x_2, 0, \dots, x_n)$
  also has at least $(n-1)!$ terms.
  \ii \dots and so on.
\end{itemize}
By the claim, all the terms obtained in this way
came from different terms of the original polynomial $P$.
Therefore, $P$ itself has at least $n \cdot (n-1)! = n!$ terms.

\begin{remark*}
  Equality is achieved by the
  Vandermonde polynomial
  $P = \prod_{1 \le i < j \le n} (x_i-x_j)$.
\end{remark*}

\paragraph{Second solution using Vandermonde polynomial (by Yang Liu).}
Since $x_i - x_j$ divides $P$ for any $i \neq j$, it follows that
$P$ should be divisible by the Vandermonde polynomial
\[ V = \prod_{i<j} (x_j-x_i)
  = \sum_{\sigma} \opname{sgn}(\sigma) x_1^{\sigma(0)}
  x_2^{\sigma(1)} \dots x_n^{\sigma(n-1)} \]
where the sum runs over all permutations $\sigma$ on $\{0, \dots, n-1\}$.

Consequently, we may write
\[ P = \sum_{\sigma} \opname{sgn}(\sigma)
  x_1^{\sigma(0)} x_2^{\sigma(1)} \dots x_n^{\sigma(n-1)} Q \]
The main idea is that each of the $n!$ terms of the above
sum has a monomial not appearing in any of the other terms.

As an example, consider $x_1^{n-1} x_2^{n-2} \dots x_{n-1}^1 x_n^0$.
Among all monomial in $Q$, consider the monomial
$x_1^{e_1} x_2^{e_2} \dots x_n^{e_n}$
with the largest $e_1$, then largest $e_2$, \dots.
(In other words, take the lexicographically largest
$(e_1, \dots, e_n)$.)
This term
\[ x_1^{e_1 + (n-1)} x_2^{e_2 + (n-2)} \dots x_n^{e_n} \]
can't appear anywhere else because it is strictly lexicographically larger
than any other term appearing in any other expansion.

Repeating this argument with every $\sigma$ gives the conclusion.
\pagebreak


\end{document}
