% © Evan Chen
% Downloaded from https://web.evanchen.cc/

\documentclass[11pt]{scrartcl}
\usepackage[sexy]{evan}
\ihead{\footnotesize\textbf{\thetitle}}
\ohead{\footnotesize\href{http://web.evanchen.cc}{\ttfamily web.evanchen.cc},
    updated \today}
\title{JMO 2010 Solution Notes}
\date{\today}

\begin{document}

\maketitle

\begin{abstract}
This is a compilation of solutions
for the 2010 JMO.
Some of the solutions are my own work,
but many are from the official solutions provided by the organizers
(for which they hold any copyrights),
and others were found by users on the Art of Problem Solving forums.

These notes will tend to be a bit more advanced and terse than the ``official''
solutions from the organizers.
In particular, if a theorem or technique is not known to beginners
but is still considered ``standard'', then I often prefer to
use this theory anyways, rather than try to work around or conceal it.
For example, in geometry problems I typically use directed angles
without further comment, rather than awkwardly work around configuration issues.
Similarly, sentences like ``let $\mathbb{R}$ denote the set of real numbers''
are typically omitted entirely.

Corrections and comments are welcome!
\end{abstract}

\tableofcontents
\newpage

\addtocounter{section}{-1}
\section{Problems}
\begin{enumerate}[\bfseries 1.]
\item %% Problem 1
Let $P(n)$ be the number of permutations $(a_1, \dots, a_n)$
of the numbers $(1, 2, \dots, n)$ for which $ka_k$ is a perfect square
for all $1 \leq k \leq n$.
Find with proof the smallest $n$ such that $P (n)$ is a multiple of $2010$.

\item %% Problem 2
Let $n > 1$ be an integer.  Find, with proof, all sequences
$x_1$, $x_2$, \dots, $x_{n-1}$ of positive integers
with the following three properties:
\begin{enumerate}[(a)]
 \ii $x_1 < x_2 < \dotsb < x_{n-1}$;
 \ii $x_i + x_{n-i} = 2n$ for all $i = 1, 2, \dots , n - 1$;
 \ii given any two indices $i$ and $j$ (not necessarily distinct)
 for which $x_i + x_j < 2n$,
 there is an index $k$ such that $x_i + x_j = x_k$.
\end{enumerate}

\item %% Problem 3
Let $AXYZB$ be a convex pentagon inscribed in a semicircle of diameter $AB$.
Denote by $P$, $Q$, $R$, $S$ the feet of the perpendiculars
from $Y$ onto lines $AX$, $BX$, $AZ$, $BZ$, respectively.
Prove that the acute angle formed by lines $PQ$ and $RS$
is half the size of $\angle XOZ$,
where $O$ is the midpoint of segment $AB$.

\item %% Problem 4
A triangle is called a \emph{parabolic} triangle
if its vertices lie on a parabola $y = x^2$.
Prove that for every nonnegative integer $n$,
there is an odd number $m$ and a parabolic triangle
with vertices at three distinct points
with integer coordinates with area $(2^nm)^2$.

\item %% Problem 5
Two permutations $a_1,a_2,\dots,a_{2010}$
and $b_1,b_2,\dots,b_{2010}$ of the numbers $1,2,\dots,2010$
are said to intersect if $a_k=b_k$ for some value of $k$ in the range $1\le k\le 2010$.
Show that there exist $1006$ permutations of the numbers $1,2,\dots,2010$
such that any other such permutation is guaranteed to
intersect at least one of these $1006$ permutations.

\item %% Problem 6
Let $ABC$ be a triangle with $\angle A = 90^{\circ}$. Points $D$ and $E$ lie on sides $AC$ and $AB$, respectively, such that $\angle ABD = \angle DBC$ and $\angle ACE = \angle ECB$. Segments $BD$ and $CE$ meet at $I$. Determine whether or not it is possible for segments $AB$, $AC$, $BI$, $ID$, $CI$, $IE$ to all have integer lengths.

\end{enumerate}
\pagebreak

\section{Solutions to Day 1}
\subsection{JMO 2010/1, proposed by Andy Niedermier}
\textsl{Available online at \url{https://aops.com/community/p1860909}.}
\begin{mdframed}[style=mdpurplebox,frametitle={Problem statement}]
Let $P(n)$ be the number of permutations $(a_1, \dots, a_n)$
of the numbers $(1, 2, \dots, n)$ for which $ka_k$ is a perfect square
for all $1 \leq k \leq n$.
Find with proof the smallest $n$ such that $P (n)$ is a multiple of $2010$.
\end{mdframed}
The answer is $n= 4489$.

We begin by giving a complete description of $P(n)$:
\begin{claim*}
  We have
  \[ P(n) = \prod_{c \text{ squarefree}} \left\lfloor \sqrt{\frac nc} \right\rfloor! \]
\end{claim*}
\begin{proof}
Every positive integer can be uniquely expressed in the form $c \cdot m^2$
where $c$ is a squarefree integer and $m$ is a perfect square.
So we may, for each squarefree positive integer $c$, define the set
\[ S_c = \left\{ c \cdot 1^2, c \cdot 2^2, c \cdot 3^2, \dots \right\}
  \cap \{1, 2, \dots, n\} \]
and each integer from $1$ through $n$ will be in exactly one $S_c$.
Note also that
\[ \left\lvert S_c \right\rvert = \left\lfloor
  \sqrt{\frac nc} \right\rfloor. \]

Then, the permutations in the problem are exactly
those which send elements of $S_c$ to elements of $S_c$.
In other words,
\[ P(n) = \prod_{c \text{ squarefree}} |S_c|!
  = \prod_{c \text{ squarefree}} \left\lfloor \sqrt{\frac nc} \right\rfloor!
  \qedhere \]
\end{proof}

We want the smallest $n$ such that $2010$ divides $P(n)$.
\begin{itemize}
  \ii Note that $P(67^2)$ contains $67!$ as a term,
  which is divisible by $2010$, so $67^2$ is a candidate.
  \ii On the other hand, if $n < 67^2$,
  then no term in the product for $P(n)$ is divisible by the prime $67$.
\end{itemize}
So $n = 67^2 = 4489$ is indeed the minimum.
\pagebreak

\subsection{JMO 2010/2, proposed by R\u{a}zvan Gelca}
\textsl{Available online at \url{https://aops.com/community/p1860914}.}
\begin{mdframed}[style=mdpurplebox,frametitle={Problem statement}]
Let $n > 1$ be an integer.  Find, with proof, all sequences
$x_1$, $x_2$, \dots, $x_{n-1}$ of positive integers
with the following three properties:
\begin{enumerate}[(a)]
 \ii $x_1 < x_2 < \dotsb < x_{n-1}$;
 \ii $x_i + x_{n-i} = 2n$ for all $i = 1, 2, \dots , n - 1$;
 \ii given any two indices $i$ and $j$ (not necessarily distinct)
 for which $x_i + x_j < 2n$,
 there is an index $k$ such that $x_i + x_j = x_k$.
\end{enumerate}
\end{mdframed}
The answer is $x_k = 2k$ only, which obviously work,
so we prove they are the only ones.

Let $x_1 < x_2 < \dots < x_n$ be
any sequence satisfying the conditions.
Consider:
\[ x_1 + x_1 < x_1 + x_2 < x_1 + x_3 < \dots
  < x_1 + x_{n-2}. \]
All these are results of condition (c),
since $x_1 + x_{n-2} < x_1 + x_{n-1} = 2n$.
So each of these must be a member of the sequence.

However, there are $n-2$ of these terms,
and there are exactly $n-2$ terms greater
than $x_1$ in our sequence.
Therefore, we get the one-to-one correspondence below:
\begin{align*}
  x_2 &= x_1 + x_1 \\
  x_3 &= x_1 + x_2 \\
  & \vdotswithin= \\
  x_{n-1} &= x_1 + x_{n-2}
\end{align*}
It follows that $x_2 = 2x_1$,
so that $x_3 = 3x_1$ and so on.
Therefore, $x_m = m x_1$.
We now solve for $x_1$ in condition (b)
to find that $x_1=2$ is the only solution,
and the desired conclusion follows.
\pagebreak

\subsection{JMO 2010/3, proposed by Titu Andreescu}
\textsl{Available online at \url{https://aops.com/community/p1860802}.}
\begin{mdframed}[style=mdpurplebox,frametitle={Problem statement}]
Let $AXYZB$ be a convex pentagon inscribed in a semicircle of diameter $AB$.
Denote by $P$, $Q$, $R$, $S$ the feet of the perpendiculars
from $Y$ onto lines $AX$, $BX$, $AZ$, $BZ$, respectively.
Prove that the acute angle formed by lines $PQ$ and $RS$
is half the size of $\angle XOZ$,
where $O$ is the midpoint of segment $AB$.
\end{mdframed}
Let $T$ be the foot from $Y$ to $\ol{AB}$.
Then the Simson line implies that lines $PQ$ and $RS$ meet at $T$.

\begin{center}
\begin{asy}
pair A = dir(180);
pair B = dir(0);
pair X = dir(140);
pair Y = dir(100);
pair Z = dir(60);

pair P = foot(Y, A, X);
pair Q = foot(Y, B, X);
pair R = foot(Y, A, Z);
pair S = foot(Y, B, Z);

pair T = extension(P, Q, R, S);

draw(A--X--Y--Z--B, orange);
filldraw(unitcircle, opacity(0.1)+orange, red);

draw(Y--P, red);
draw(Y--Q, red);
draw(Y--R, red);
draw(Y--S, red);
draw(P--X--B, brown);
draw(P--T--S, red);
draw(S--Z--A, brown);

dot("$A$", A, dir(A));
dot("$B$", B, dir(B));
dot("$X$", X, dir(X));
dot("$Y$", Y, dir(Y));
dot("$Z$", Z, dir(Z));
dot("$P$", P, dir(P));
dot("$Q$", Q, dir(Q));
dot("$R$", R, dir(R));
dot("$S$", S, dir(S));
dot("$T$", T, dir(T));

/* TSQ Source:

A = dir 180
B = dir 0
X = dir 140
Y = dir 100
Z = dir 60

P = foot Y A X
Q = foot Y B X
R = foot Y A Z
S = foot Y B Z

T = extension P Q R S

A--X--Y--Z--B orange
unitcircle 0.1 orange / red

Y--P red
Y--Q red
Y--R red
Y--S red
P--X--B brown
P--T--S red
S--Z--A brown

*/
\end{asy}
\end{center}

Now it's straightforward to see $APYRT$ is cyclic
(in the circle with diameter $\ol{AY}$),
and therefore
\[ \angle RTY = \angle RAY = \angle ZAY. \]
Similarly,
\[ \angle YTQ = \angle YBQ = \angle YBX. \]
Summing these gives $\angle RTQ$ is equal to
half the measure of arc $\widehat{XZ}$ as needed.

(Of course, one can also just angle chase;
the Simson line is not so necessary.)
\pagebreak

\section{Solutions to Day 2}
\subsection{JMO 2010/4, proposed by Zuming Feng}
\textsl{Available online at \url{https://aops.com/community/p1860772}.}
\begin{mdframed}[style=mdpurplebox,frametitle={Problem statement}]
A triangle is called a \emph{parabolic} triangle
if its vertices lie on a parabola $y = x^2$.
Prove that for every nonnegative integer $n$,
there is an odd number $m$ and a parabolic triangle
with vertices at three distinct points
with integer coordinates with area $(2^nm)^2$.
\end{mdframed}
For $n=0$, take instead $(a,b) = (1,0)$.

For $n > 0$, consider a triangle
with vertices at $(a,a^2)$, $(-a,a^2)$ and $(b, b^2)$.
Then the area of this triangle was equal to
\[ \half (2a) \left( b^2-a^2 \right) = a(b^2-a^2). \]

To make this equal $2^{2n} m^2$,
simply pick $a = 2^{2n}$,
and then pick $b$ such that $b^2-m^2 = 2^{4n}$,
for example $m = 2^{4n-2}-1$ and $b=2^{4n-2}+1$.
\pagebreak

\subsection{JMO 2010/5, proposed by Gregory Galperin}
\textsl{Available online at \url{https://aops.com/community/p1860912}.}
\begin{mdframed}[style=mdpurplebox,frametitle={Problem statement}]
Two permutations $a_1,a_2,\dots,a_{2010}$
and $b_1,b_2,\dots,b_{2010}$ of the numbers $1,2,\dots,2010$
are said to intersect if $a_k=b_k$ for some value of $k$ in the range $1\le k\le 2010$.
Show that there exist $1006$ permutations of the numbers $1,2,\dots,2010$
such that any other such permutation is guaranteed to
intersect at least one of these $1006$ permutations.
\end{mdframed}
A valid choice is the following $1006$ permutations:
\[
\begin{array}{cccc ccc | ccccc}
1 & 2 & 3 & \dotsb & 1004 & 1005 & 1006 & 1007 & 1008 & \dotsb & 2009 & 2010 \\
2 & 3 & 4 & \dotsb & 1005 & 1006 & 1 & 1007 & 1008 & \dotsb & 2009 & 2010 \\
3 & 4 & 5 & \dotsb & 1006 & 1 & 2 & 1007 & 1008 & \dotsb & 2009 & 2010 \\
\vdots & \vdots & \vdots & \ddots & \vdots & \vdots & \vdots &
\vdots& \vdots & \vdots & \vdots & \vdots \\
1004 & 1005 & 1006 & \dotsb & 1001 & 1002 & 1003 & 1007 & 1008 & \dotsb & 2009 & 2010 \\
1005 & 1006 & 1 & \dotsb & 1002 & 1003 & 1004 & 1007 & 1008 & \dotsb & 2009 & 2010 \\
1006 & 1 & 2 & \dotsb & 1003 & 1004 & 1005 & 1007 & 1008 & \dotsb & 2009 & 2010 \\
\end{array}
\]
This works. Indeed, any permutation should have one of $\{1, 2, \dots, 1006\}$
somewhere in the first $1006$ positions,
so one will get an intersection.

\begin{remark*}
  In fact, the last $1004$ entries do not matter with this
  construction, and we chose to leave them as $1007, 1008, \dots, 2010$
  only for concreteness.
\end{remark*}
\begin{remark*}
  Using Hall's marriage lemma one may prove that
  the result becomes false with $1006$ replaced by $1005$.
\end{remark*}
\pagebreak

\subsection{JMO 2010/6, proposed by Zuming Feng}
\textsl{Available online at \url{https://aops.com/community/p1860753}.}
\begin{mdframed}[style=mdpurplebox,frametitle={Problem statement}]
Let $ABC$ be a triangle with $\angle A = 90^{\circ}$. Points $D$ and $E$ lie on sides $AC$ and $AB$, respectively, such that $\angle ABD = \angle DBC$ and $\angle ACE = \angle ECB$. Segments $BD$ and $CE$ meet at $I$. Determine whether or not it is possible for segments $AB$, $AC$, $BI$, $ID$, $CI$, $IE$ to all have integer lengths.
\end{mdframed}
The answer is no.
We prove that it is not even possible that $AB$, $AC$, $CI$, $IB$ are all integers.

\begin{center}
  \begin{asy}
    size(5cm);
    pair B = dir(140);
    pair A = conj(B);
    pair C = -B;
    filldraw(A--B--C--cycle, opacity(0.1)+lightblue, blue);
    pair I = incenter(A, B, C);
    pair D = extension(B, I, A, C);
    pair E = extension(C, I, A, B);
    draw(B--D);
    draw(C--E);
    filldraw(incircle(A,B,C), opacity(0.1)+lightred, red+dotted);

    dot("$B$", B, dir(B));
    dot("$A$", A, dir(A));
    dot("$C$", C, dir(C));
    dot("$I$", I, dir(45));
    dot("$D$", D, dir(D));
    dot("$E$", E, dir(E));

    /* Source generated by TSQ */
  \end{asy}
\end{center}

First, we claim that $\angle BIC = 135\dg$.
To see why, note that
\[ \angle IBC + \angle ICB = \frac{\angle B}{2} + \frac{\angle C}{2}
  = \frac{90\dg}{2} = 45\dg. \]
So, $\angle BIC = 180\dg - (\angle IBC + \angle ICB) = 135\dg$, as desired.

We now proceed by contradiction.
The Pythagorean theorem implies
\[ BC^2 = AB^2 + AC^2 \]
and so $BC^2$ is an integer.
However, the law of cosines gives
\begin{align*}
  BC^2 &= BI^2 + CI^2 - 2 BI \cdot CI \cos \angle BIC \\
  &= BI^2 + CI^2 + BI \cdot CI \cdot \sqrt 2.
\end{align*}
which is irrational, and this produces the desired contradiction.
\pagebreak


\end{document}
