% © Evan Chen
% Downloaded from https://web.evanchen.cc/

\documentclass[11pt]{scrartcl}
\usepackage[sexy]{evan}
\ihead{\footnotesize\textbf{\thetitle}}
\ohead{\footnotesize\href{http://web.evanchen.cc}{\ttfamily web.evanchen.cc},
    updated \today}
\title{JMO 2019 Solution Notes}
\date{\today}

\begin{document}

\maketitle

\begin{abstract}
This is a compilation of solutions
for the 2019 JMO.
Some of the solutions are my own work,
but many are from the official solutions provided by the organizers
(for which they hold any copyrights),
and others were found by users on the Art of Problem Solving forums.

These notes will tend to be a bit more advanced and terse than the ``official''
solutions from the organizers.
In particular, if a theorem or technique is not known to beginners
but is still considered ``standard'', then I often prefer to
use this theory anyways, rather than try to work around or conceal it.
For example, in geometry problems I typically use directed angles
without further comment, rather than awkwardly work around configuration issues.
Similarly, sentences like ``let $\mathbb{R}$ denote the set of real numbers''
are typically omitted entirely.

Corrections and comments are welcome!
\end{abstract}

\tableofcontents
\newpage

\addtocounter{section}{-1}
\section{Problems}
\begin{enumerate}[\bfseries 1.]
\item %% Problem 1
There are $a+b$ bowls arranged in a row,
numbered $1$ through $a+b$,
where $a$ and $b$ are given positive integers.
Initially, each of the first $a$ bowls contains an apple,
and each of the last $b$ bowls contains a pear.
A legal move consists of moving an apple from bowl $i$ to bowl $i+1$
and a pear from bowl $j$ to bowl $j-1$,
provided that the difference $i-j$ is even.
We permit multiple fruits in the same bowl at the same time.
The goal is to end up with the first $b$ bowls each containing a pear
and the last $a$ bowls each containing an apple.
Show that this is possible if and only if the product $ab$ is even.

\item %% Problem 2
For which pairs of integers $(a, b)$ do there exist functions
$f\colon \ZZ \to \ZZ$ and $g \colon \ZZ \to \ZZ$ obeying
\[ f(g(x)) = x + a \quad \text{and} \quad g(f(x)) = x + b \]
for all integers $x$?

\item %% Problem 3
Let $ABCD$ be a cyclic quadrilateral
satisfying $AD^2 + BC^2 = AB^2$.
The diagonals of $ABCD$ intersect at $E$.
Let $P$ be a point on side $\ol{AB}$
satisfying $\angle APD = \angle BPC$.
Show that line $PE$ bisects $\ol{CD}$.

\item %% Problem 4
Let $ABC$ be a triangle with $\angle B > 90\dg$
and let $E$ and $F$ be the feet of the altitudes from $B$ and $C$.
Can line $EF$ be tangent to the $A$-excircle?

\item %% Problem 5
Let $n$ be a nonnegative integer.
Determine the number of ways to choose sets
$S_{ij} \subseteq \{1, 2, \dots, 2n\}$,
for all $0 \le i \le n$ and $0 \le j \le n$
(not necessarily distinct), such that
\begin{itemize}
  \ii $|S_{ij}| = i+j$, and
  \ii $S_{ij} \subseteq S_{kl}$ if $0 \le i \le k \le n$
  and $0 \le j \le l \le n$.
\end{itemize}

\item %% Problem 6
Let $m$ and $n$ be relatively prime positive integers.
The numbers $\frac mn$ and $\frac nm$ are written on a blackboard.
At any point, Evan may pick two of the numbers $x$ and $y$
written on the board and write either their arithmetic mean $\half(x+y)$
or their harmonic mean $\frac{2xy}{x+y}$.
For which $(m,n)$ can Evan write $1$ on the board in finitely many steps?

\end{enumerate}
\pagebreak

\section{Solutions to Day 1}
\subsection{JMO 2019/1, proposed by Jim Propp}
\textsl{Available online at \url{https://aops.com/community/p12189456}.}
\begin{mdframed}[style=mdpurplebox,frametitle={Problem statement}]
There are $a+b$ bowls arranged in a row,
numbered $1$ through $a+b$,
where $a$ and $b$ are given positive integers.
Initially, each of the first $a$ bowls contains an apple,
and each of the last $b$ bowls contains a pear.
A legal move consists of moving an apple from bowl $i$ to bowl $i+1$
and a pear from bowl $j$ to bowl $j-1$,
provided that the difference $i-j$ is even.
We permit multiple fruits in the same bowl at the same time.
The goal is to end up with the first $b$ bowls each containing a pear
and the last $a$ bowls each containing an apple.
Show that this is possible if and only if the product $ab$ is even.
\end{mdframed}
First we show that if $ab$ is even then the goal is possible.
We prove the result by induction on $a+b$.
\begin{itemize}
  \ii If $\min(a,b) = 0$ there is nothing to check.
  \ii If $\min(a,b) = 1$, say $a=1$, then $b$ is even,
  and we can swap the (only) leftmost apple
  with the rightmost pear by working only with those fruits.
  \ii Now assume $\min(a,b) \ge 2$ and $a+b$ is odd.
  Then we can swap the leftmost apple with rightmost pear
  by working only with those fruits,
  reducing to the situation of $(a-1, b-1)$
  which is possible by induction
  (at least one of them is even).
  \ii Finally assume $\min(a,b) \ge 2$ and $a+b$ is even
  (i.e.\ $a$ and $b$ are both even).
  Then we can swap the apple in position $1$
  with the pear in position $a+b-1$,
  and the apple in position $2$ with the pear in position $a+b$.
  This reduces to the situation of $(a-2, b-2)$
  which is also possible by induction.
\end{itemize}

Now we show that the result is impossible if $ab$ is odd.
Define
\begin{align*}
  X &= \text{number apples in odd-numbered bowls} \\
  Y &= \text{number pears in odd-numbered bowls}.
\end{align*}
Note that $X-Y$ does not change under this operation.
However, if $a$ and $b$ are odd,
then we initially have $X = \half(a+1)$ and $Y = \half(b-1)$,
while the target position has $X = \half(a-1)$ and $Y = \half(b+1)$.
So when $ab$ is odd this is not possible.

\begin{remark*}
Another proof that $ab$ must be even
is as follows.

First, note that apples only move right and pears only move left,
a successful operation must take exactly $ab$ moves.
So it is enough to prove that the \emph{number of moves}
made must be even.

However, the number of fruits in odd-numbered bowls
either increases by $+2$ or $-2$ in each move
(according to whether $i$ and $j$ are both even or both odd),
and since it ends up being the same at the end,
the number of moves must be even.

Alternatively, as pointed out in the official solutions,
one can consider the sums of squares of positions of fruits.
The quantity changes by
\[ \left[ (i+1)^2 + (j-1)^2 \right] - (i^2+j^2)
  = 2(i-j) + 2 \equiv 2 \pmod 4 \]
at each step,
and eventually the sums of squares returns to zero, as needed.
\end{remark*}
\pagebreak

\subsection{JMO 2019/2, proposed by Ankan Bhattacharya}
\textsl{Available online at \url{https://aops.com/community/p12189493}.}
\begin{mdframed}[style=mdpurplebox,frametitle={Problem statement}]
For which pairs of integers $(a, b)$ do there exist functions
$f\colon \ZZ \to \ZZ$ and $g \colon \ZZ \to \ZZ$ obeying
\[ f(g(x)) = x + a \quad \text{and} \quad g(f(x)) = x + b \]
for all integers $x$?
\end{mdframed}
The answer is if $a=b$ or $a=-b$.
In the former case, one can take $f(x) \equiv x+a$ and $g(x) \equiv x$.
In the latter case, one can take $f(x) \equiv -x+a$ and $g(x) = -x$.

Now we prove these are the only possibilities.
First:
\begin{claim*}
  The functions $f$ and $g$ are bijections.
\end{claim*}
\begin{proof}
  Surjectivity is obvious.
  To see injective, note that if $f(u) = f(v)$
  then $g(f(u)) = g(f(v)) \implies u+b = v+b \implies u=v$,
  and similarly for $g$.
\end{proof}

Note also that for any $x$, we have
\begin{align*}
  f(x+b) &= f(g(f(x))) = f(x) + a \\
  g(x+a) &= g(f(g(x))) = g(x) + b.
\end{align*}
If either $a$ is zero or $b$ is zero,
we immediately get the other is zero, and hence done.
So assume $ab \neq 0$.

If $|b| > |a|$, then two of
\[ \{ f(0), f(1), \dots, f(b-1) \} \pmod{|a|} \]
coincide, which together with repeatedly applying the first equation
above will then give a contradiction to injectivity of $f$.
Similarly, if $|a| > |b|$ swapping the roles of $f$ and $g$
(and $a$ and $b$) will give a contradiction to injectivity of $g$.
This completes the proof.

\begin{remark*}
  Here is a way to visualize the argument,
  so one can see pictorially what is going on.
  We draw two parallel number lines indexed by $\ZZ$.
  Starting from $0$, we draw red arrow from $0$ to $f(0)$,
  and then a blue arrow from $f(0)$ to $g(f(0)) = b$,
  and then a red arrow from $b$ to $g(b) = f(0)+a$, and so on.
  These arrows can be extended both directions,
  leading to an infinite ``squaretooth'' wave.
  The following is a picture of an example with $a,b > 0$.
  \begin{center}
  \begin{asy}
  size(11cm);
  usepackage("amsmath");
  usepackage("amssymb");
  draw( (-3,2)--(13,2), Arrows(TeXHead) );
  draw( (-3,-2)--(13,-2), Arrows(TeXHead) );
  label("$\mathbb Z$", (13, 2), dir(0));
  label("$\mathbb Z$", (13, -2), dir(0));

  pair bm1 = (-5,2);
  pair b0 = (0,2);
  pair b1 = (5,2);
  pair b2 = (10,2);
  pair b3 = (15,2);
  pair am1 = (-1,-2);
  pair a0 = (2.5,-2);
  pair a1 = (6,-2);
  pair a2 = (9.5,-2);
  pair a3 = (13,-2);

  dot("$0$", b0, dir(90));
  dot("$b$", b1, dir(90));
  dot("$2b$", b2, dir(90));
  dot("$f(0)-a$", am1, dir(-90));
  dot("$f(0)$", a0, dir(-90));
  dot("$f(0)+a$", a1, dir(-90));
  dot("$f(0)+2a$", a2, dir(-90));

  draw(midpoint(bm1--am1)--am1, red, EndArrow, Margins);
  label("$f$", midpoint(bm1--am1)--am1, dir(50), red);
  draw(b0--a0, red, EndArrow, Margins);
  label("$f$", b0--a0, dir(50), red);
  draw(b1--a1, red, EndArrow, Margins);
  label("$f$", b1--a1, dir(50), red);
  draw(b2--a2, red, EndArrow, Margins);
  label("$f$", b2--a2, dir(50), red);

  draw(am1--b0, blue, EndArrow, Margins);
  label("$g$", midpoint(am1--b0)--b0, dir(-50), blue);
  draw(a0--b1, blue, EndArrow, Margins);
  label("$g$", a0--b1, dir(-50), blue);
  draw(a1--b2, blue, EndArrow, Margins);
  label("$g$", a1--b2, dir(-50), blue);
  draw(a2--midpoint(a2--b3), blue, EndArrow, Margins);
  label("$g$", a2--midpoint(a2--b3), dir(-50), blue);
  \end{asy}
  \end{center}
  The problem is essentially trying to decompose
  our two copies of $\ZZ$ into multiple squaretooth waves.
  We expect for this to be possible, the ``width'' of the waves
  on the top and bottom must be the same --- i.e., that $|a| = |b|$.
\end{remark*}

\begin{remark*}
  This also suggests how to classify all functions $f$ and $g$
  satisfying the condition.
  If $a = b = 0$ then any pair of functions $f$ and $g$
  which are inverses to each other is okay.
  There are thus uncountably many pairs of functions $(f,g)$ here.

  If $a = b > 0$, then one sets $f(0)$, $f(1)$, \dots, $f(a-1)$
  as any values which are distinct modulo $b$,
  at which point $f$ and $g$ are uniquely determined.
  An example for $a = b = 3$ is
  \[ f(x) = \begin{cases}
      x + 42 & x \equiv 0 \pmod 3 \\
      x + 13 & x \equiv 1 \pmod 3 \\
      x - 37 & x \equiv 2 \pmod 3,
    \end{cases}
    \qquad
    g(x) = \begin{cases}
      x - 39 & x \equiv 0 \pmod 3 \\
      x + 40 & x \equiv 1 \pmod 3 \\
      x - 10 & x \equiv 2 \pmod 3.
    \end{cases} \]
  The analysis for $a = b < 0$ and $a = -b$ are similar,
  but we don't include the details here.
\end{remark*}
\pagebreak

\subsection{JMO 2019/3, proposed by Ankan Bhattacharya}
\textsl{Available online at \url{https://aops.com/community/p12189455}.}
\begin{mdframed}[style=mdpurplebox,frametitle={Problem statement}]
Let $ABCD$ be a cyclic quadrilateral
satisfying $AD^2 + BC^2 = AB^2$.
The diagonals of $ABCD$ intersect at $E$.
Let $P$ be a point on side $\ol{AB}$
satisfying $\angle APD = \angle BPC$.
Show that line $PE$ bisects $\ol{CD}$.
\end{mdframed}
Here are three solutions.
The first two are similar although the first one makes use of symmedians.
The last solution by inversion is more advanced.

\paragraph{First solution using symmedians.}
We define point $P$ to obey
\[ \frac{AP}{BP} = \frac{AD^2}{BC^2} = \frac{AE^2}{BE^2} \]
so that $\ol{PE}$ is the $E$-symmedian of $\triangle EAB$,
therefore the $E$-median of $\triangle ECD$.

Now, note that
\[ AD^2 = AP \cdot AB \quad\text{and}\quad BC^2 = BP \cdot BA. \]
This implies $\triangle APD \sim \triangle ADB$ and $\triangle BPC \sim \triangle BCA$.
Thus
\[ \dang DPA = \dang ADB = \dang ACB = \dang BCP \]
and so $P$ satisfies the condition as in the statement
(and is the unique point to do so), as needed.

\paragraph{Second solution using only angle chasing (by proposer).}
We again re-define $P$ to obey $AD^2 = AP \cdot AB$ and $BC^2 = BP \cdot BA$.
As before, this gives $\triangle APD \sim \triangle ABD$
and $\triangle BPC \sim \triangle BDP$ and so we let
\[ \theta \coloneqq \dang DPA = \dang ADB = \dang ACB = \dang BCP. \]
Our goal is to now show $\ol{PE}$ bisects $\ol{CD}$.

Let $K = \ol{AC} \cap \ol{PD}$ and $L = \ol{AD} \cap \ol{PC}$.
Since $\dang KPA = \theta = \dang ACB$, quadrilateral $BPKC$ is cyclic.
Similarly, so is $APLD$.

\begin{center}
\begin{asy}
size(8cm);

pair A = dir(200);
pair B = -conj(A);
pair D = dir(140);
pair O = origin;
pair P = extension(D, foot(D, A, O), A, B);
pair Z = 100*foot(P,O,B)-99*P;
pair C = IP(P--Z, unitcircle);

markangle(13.0, A, D, B, red);
markangle(13.0, A, C, B, red);
markangle(13.0, D, P, A, red);
markangle(13.0, B, P, C, red);

filldraw(unitcircle, opacity(0.1)+palecyan, deepcyan);
draw(A--B--C--D--cycle, deepcyan);
draw(A--C, deepgreen);
draw(B--D, deepgreen);
pair K = extension(D, P, A, C);
pair L = extension(C, P, D, B);
draw(D--P--C, blue);
pair E = extension(A, C, B, D);

dot("$A$", A, dir(A));
dot("$B$", B, dir(B));
dot("$D$", D, dir(D));
dot("$P$", P, dir(270));
dot("$C$", C, dir(C));
dot("$K$", K, dir(K));
dot("$L$", L, dir(20));
dot("$E$", E, dir(90));

/* TSQ Source:

!size(8cm);

A = dir 200
B = -conj(A)
D = dir 140
O := origin
P 270 = extension D foot D A O A B
Z := 100*foot(P,O,B)-99*P
C = IP P--Z unitcircle

!markangle(13.0, A, D, B, red);
!markangle(13.0, A, C, B, red);
!markangle(13.0, D, P, A, red);
!markangle(13.0, B, P, C, red);

unitcircle 0.1 palecyan / deepcyan
A--B--C--D--cycle deepcyan
A--C deepgreen
B--D deepgreen
K = extension D P A C
L 20 = extension C P D B
D--P--C blue
E = extension A C B D R90

*/
\end{asy}
\end{center}

Finally $AKLB$ is cyclic since
\[ \dang BKA = \dang BKC = \dang BPC = \theta = \dang DPA = \dang DLA = \dang BLA. \]
This implies $\dang CKL = \dang LBA = \dang DCK$, so $\ol{KL} \parallel \ol{BC}$.
Then $PE$ bisects $\ol{BC}$ by Ceva's theorem on $\triangle PCD$.

\paragraph{Third solution (using inversion).}
By hypothesis, the circle $\omega_a$ centered at $A$ with radius $AD$
is orthogonal to the circle $\omega_b$ centered at $B$ with radius $BC$.
For brevity, we let $\mathbf{I}_a$ and $\mathbf{I}_b$
denote inversion with respect to $\omega_a$ and $\omega_b$.

We let $P$ denote the intersection of $\ol{AB}$ with the radical axis of $\omega_a$ and $\omega_b$;
hence $P = \mathbf{I}_a(B) = \mathbf{I}_b(A)$.
This already implies that
\[ \dang DPA \overset{\mathbf{I}_a}{=} \dang ADB = \dang ACB \overset{\mathbf{I}_b}{=} \dang BPC \]
so $P$ satisfies the angle condition.

\begin{center}
\begin{asy}
size(10cm);

pair A = dir(200);
pair B = -conj(A);
pair D = dir(140);
pair O = origin;
pair P = extension(D, foot(D, A, O), A, B);
pair Z = 100*foot(P,O,B)-99*P;
pair C = IP(P--Z, unitcircle);
filldraw(unitcircle, opacity(0.1)+palecyan, deepcyan);
draw(A--B--C--D--cycle, deepcyan);
filldraw(CP(A, D), opacity(0.1)+lightred, red);
filldraw(CP(B, C), opacity(0.1)+orange, orange);
pair X = IP(CP(A, D), CP(B, C));
pair Y = OP(CP(A, D), CP(B, C));
draw(X--Y, yellow);
draw(A--C, deepcyan+dotted);
draw(B--D, deepcyan+dotted);
pair K = extension(D, P, A, C);
pair L = extension(C, P, D, B);
draw(D--P, red+dashed);
draw(C--P, orange+dashed);

dot("$A$", A, dir(A));
dot("$B$", B, dir(B));
dot("$D$", D, dir(D));
dot("$P$", P, dir(270));
dot("$C$", C, dir(C));
dot(X);
dot(Y);
dot("$K$", K, dir(K));
dot("$L$", L, dir(20));

/* TSQ Source:

!size(12cm);

A = dir 200
B = -conj(A)
D = dir 140
O := origin
P 270 = extension D foot D A O A B
Z := 100*foot(P,O,B)-99*P
C = IP P--Z unitcircle
unitcircle 0.1 palecyan / deepcyan
A--B--C--D--cycle deepcyan
CP A D 0.1 lightred / red
CP B C 0.1 orange / orange
X .= IP CP A D CP B C
Y .= OP CP A D CP B C
X--Y yellow
A--C deepcyan dotted
B--D deepcyan dotted
K = extension D P A C
L 20 = extension C P D B
D--P red dashed
C--P orange dashed

*/
\end{asy}
\end{center}

\begin{claim*}
  The point $K = \mathbf{I}_a(C)$ lies on $\omega_b$ and $\ol{DP}$.
  Similarly $L = \mathbf{I}_b(D)$ lies on  $\omega_a$ and $\ol{CP}$.
\end{claim*}
\begin{proof}
  The first assertion follows from the fact that $\omega_b$ is orthogonal to $\omega_a$.
  For the other, since $(BCD)$ passes through $A$, it follows $P = \mathbf{I}_a(B)$, $K = \mathbf{I}_a(C)$,
  and $D = \mathbf{I}_a(D)$ are collinear.
\end{proof}

Finally, since $C$, $L$, $P$ are collinear,
we get $A$ is concyclic with $K = \mathbf{I}_a(C)$,
$L = \mathbf{I}_a(L)$, $B = \mathbf{I}_a(B)$, i.e.\ that $AKLB$ is cyclic.
So $\ol{KL} \parallel \ol{CD}$ by Reim's theorem,
and hence $\ol{PE}$ bisects $\ol{CD}$ by Ceva's theorem.
\pagebreak

\section{Solutions to Day 2}
\subsection{JMO 2019/4, proposed by Ankan Bhattacharya, Zack Chroman, Anant Mudgal}
\textsl{Available online at \url{https://aops.com/community/p12195848}.}
\begin{mdframed}[style=mdpurplebox,frametitle={Problem statement}]
Let $ABC$ be a triangle with $\angle B > 90\dg$
and let $E$ and $F$ be the feet of the altitudes from $B$ and $C$.
Can line $EF$ be tangent to the $A$-excircle?
\end{mdframed}
We show it is not possible, by contradiction
(assuming $EF$ is indeed tangent).
Thus $BECF$ is a convex cyclic quadrilateral
inscribed in a circle with diameter $\ol{BC}$.
Note also that the $A$-excircle lies on the opposite side
from $A$ as line $EF$, since $A$, $E$, $C$ are collinear
in that order.

\paragraph{First solution by similarity.}
Note that $\triangle AEF$ is similar to $\triangle ABC$
(and oppositely oriented).
However, since they have the same $A$-exradius,
it follows they are congruent.

\begin{center}
\begin{asy}
size(6cm);
pair B = dir(213);
pair C = dir(47);
pair E = dir(115);
pair F = dir(280);

pair V = extension(F, E, B, C);
pair W = incenter(F, V, C);
pair J = 3.6*W-2.6*V;

pair T = foot(J, B, C);
pair A = extension(B, F, E, C);

filldraw(CP(J, T), opacity(0.1)+deepgreen, deepgreen);

pair U = OP(CP(J, T), CP(midpoint(B--J), J));
pair V = IP(CP(J, T), CP(midpoint(E--J), J));
fill(A--B--C--cycle, rgb(0.9,1,1));

draw(B--E, lightblue);
draw(C--F, lightblue);
draw(U--B--A--V, blue);
draw(B--C, blue);
draw(F--E, red);
draw(unitcircle, lightblue+dotted);

dot("$B$", B, dir(B));
dot("$C$", C, dir(C));
dot("$E$", E, dir(E));
dot("$F$", F, dir(F));
dot("$A$", A, dir(A));

/* TSQ Source:

!size(6cm);
B = dir 213
C = dir 47
E = dir 115
F = dir 280

V := extension F E B C
W := incenter F V C
J := 3.6*W-2.6*V

T := foot J B C
A = extension B F E C

CP J T 0.1 deepgreen / deepgreen

U := OP CP J T CP midpoint B--J J
V := IP CP J T CP midpoint E--J J
!fill(A--B--C--cycle, rgb(0.9,1,1));

B--E lightblue
C--F lightblue
U--B--A--V blue
B--C blue
F--E red
unitcircle lightblue dotted

*/
\end{asy}
\end{center}

Consequently we get $EF = BC$.
But this implies $BFCE$ is a rectangle, contradiction.

\paragraph{Second length solution by tangent lengths.}
By $t(\bullet)$ we mean the length of the tangent from $P$
to the $A$-excircle.
It is a classical fact for example that $t(A) = s$.
The main idea is to use the fact that
\[ a \cos A = EF = t(E) + t(F). \]
Here $EF = a \cos A$ follows from the extended law of sines
applied to the circle with diameter $\ol{BC}$,
since there we have $EF = BC \sin \angle ECF = a \sin \angle ACF = a \cos A$.
We may now compute
\begin{align*}
  t(E) &= t(A) - AE = s - c \cos A \\
  t(F) &= t(A) - AF = s - b \cos A.
\end{align*}
Therefore,
\begin{align*}
  a \cos A = 2s - (b+c) \cos A \implies (a+b+c) \cos A &= 2s \\
  \implies \cos A &= 1.
\end{align*}
This is an obvious contradiction.

\begin{remark*}
  On the other hand, there really is an \emph{equality case}
  with $A$ being some point at infinity (meaning $\cos A = 1$).
  So, this problem is ``sharper'' than one might expect;
  the answer is not ``obviously no''.
\end{remark*}

\paragraph{Third solution by Pitot and trigonometry.}
In fact, the $t(\bullet)$ notation from the previous
solution gives us a classical theorem
once we note the $A$-excircle is tangent to all four lines
$EF$, $BC$, $BF$ and $CE$:
\begin{claim*}
  [Pitot theorem]
  We have $BF + EF = BC + CE$.
\end{claim*}
\begin{proof}
  Here is a proof for completeness.
  By $t(B)$ we mean the length of the tangent from $B$ to
  the $A$-excircle, and define $t(C)$, $t(E)$, $t(F)$ similarly.
  Then
  \begin{align*}
    BF &= t(B) - t(F) & EF &= t(E) + t(F) \\
    BC &= t(B) + t(C) & CE &= t(E) - t(C)
  \end{align*}
  and summing gives the result.
\end{proof}
\begin{center}
\begin{asy}
pair B = dir(180);
pair C = -B;
pair E = dir(40);
pair F = -E;

pair X = B+2*dir(F-B);
pair Y = C+2*dir(C-E);
pair J = extension(C, X, B, Y);

filldraw(CP(J, foot(J, B, C)), opacity(0.1)+deepgreen, deepgreen);
pair K1 = B+dir(E-C)*1.7;
pair K2 = C+dir(E-C)*1.7;

pair U = foot(J, B, F);
pair V = foot(J, C, E);
label("$A$", midpoint(K1--K2), dir(C-E));

filldraw(K1--B--C--K2--cycle, opacity(0.1)+lightcyan, invisible);
draw(K1--B--C--K2, blue);
draw(F--E, red);
draw(B--U, blue);
draw(C--V, blue);
draw(B--E, blue);
draw(C--F, blue);

dot("$B$", B, dir(B));
dot("$C$", C, dir(C));
dot("$E$", E, dir(E));
dot("$F$", F, dir(F));
dot("$J$", J, dir(-90));

/* TSQ Source:

B = dir 180
C = -B
E = dir 40
F = -E

X := B+2*dir(F-B)
Y := C+2*dir(C-E)
J = extension C X B Y R-90

CP J foot J B C 0.1 deepgreen / deepgreen
K1 := B+dir(E-C)*1.7
K2 := C+dir(E-C)*1.7

U := foot J B F
V := foot J C E
!label("$A$", midpoint(K1--K2), dir(C-E));

K1--B--C--K2--cycle 0.1 lightcyan / invisible
K1--B--C--K2 blue
F--E red
B--U blue
C--V blue
B--E blue
C--F blue

*/
\end{asy}
\end{center}
We now calculate all the lengths using trigonometry:
\begin{align*}
  BC &= a \\
  BF &= a \cos(180\dg-B) = a \cos (A+C) \\
  CE &= a \cos C \\
  EF &= BC \sin \angle ECF = a \sin \angle ACF = a \cos A.
\end{align*}
Thus, we apparently have
\[ \cos(A+C) + \cos A = 1 + \cos C \]
but this is impossible since $\cos(A+C) < \cos C$
(since $A+C = 180-B < 90\dg$) and $\cos A < 1$.

\paragraph{Fourth solution by Pitot and Ptolemy (Evan Chen).}
We give a trig-free way to finish from Pitot's theorem
\[ BF+EF = BC+CE. \]
Assume that $x = BF$, $y = CE$, and $BC = 1$;
then the above relation becomes
\[ 1 + y - x = BC + CE - BF = EF = EF \cdot 1 = xy + \sqrt{(1-x^2)(1-y^2)} \]
with the last step by Ptolemy's theorem.
This rearranges to give
\[ (1+y)(1-x) = \sqrt{(1-x^2)(1-y^2)}
  \implies \frac{1+y}{1-y} = \frac{1+x}{1-x}
  \implies x = y \]
but that means $BECF$ is a rectangle:
contradicting the fact that lines $BE$ and $CF$ meet at a point $A$.

\paragraph{Fifth solution, by angle chasing only!.}
Let $J$ denote the $A$-excenter.
Then $J$ should be the intersection
of the internal bisectors of $\angle FEC$ and $\angle FBC$,
so it is the midpoint of arc $\widehat{FC}$
on the circle with diameter $\ol{BC}$.

\begin{center}
\begin{asy}
pair B = dir(180);
pair C = -B;
pair E = dir(40);
pair F = -E;

pair X = B+2*dir(F-B);
pair Y = C+2*dir(C-E);
pair J = extension(C, X, B, Y);

filldraw(CP(J, foot(J, B, C)), opacity(0.1)+deepgreen, deepgreen);
pair K1 = B+dir(E-C)*1.7;
pair K2 = C+dir(E-C)*1.7;

pair U = foot(J, B, F);
pair V = foot(J, C, E);
label("$A$", midpoint(K1--K2), dir(C-E));

filldraw(K1--B--C--K2--cycle, opacity(0.1)+lightcyan, invisible);
draw(K1--B--C--K2, blue);
draw(F--E, red);
draw(B--U, blue);
draw(C--V, blue);
draw(B--E, blue);
draw(C--F, blue);

filldraw(circumcircle(B, C, J), opacity(0.1)+yellow, orange+dashed);
draw(B--J--E, orange);

dot("$B$", B, dir(B));
dot("$C$", C, dir(C));
dot("$E$", E, dir(E));
dot("$F$", F, dir(F));
dot("$J$", J, dir(-90));

/* TSQ Source:

B = dir 180
C = -B
E = dir 40
F = -E

X := B+2*dir(F-B)
Y := C+2*dir(C-E)
J = extension C X B Y R-90

CP J foot J B C 0.1 deepgreen / deepgreen
K1 := B+dir(E-C)*1.7
K2 := C+dir(E-C)*1.7

U := foot J B F
V := foot J C E
!label("$A$", midpoint(K1--K2), dir(C-E));

K1--B--C--K2--cycle 0.1 lightcyan / invisible
K1--B--C--K2 blue
F--E red
B--U blue
C--V blue
B--E blue
C--F blue

circumcircle B C J 0.1 yellow / orange dashed
B--J--E orange

*/
\end{asy}
\end{center}

But now we get $\angle BJC = 90\dg$ from $J$ lying on this circle.
Yet $\angle BJC = 90\dg - \half \angle A$ in general,
so $\angle A = 0\dg$ which is impossible.

\paragraph{Sixth solution (Zuming Feng).}
This is similar to the preceding solution,
but phrased using contradiction and inequalities.
We let $X$ and $Y$ denote the tangency points of the $A$-excircle
on lines $AB$ and $AC$.
Moreover, let $J$ denote the $A$-excenter.

\begin{center}
\begin{asy}
size(8cm);
pair B = dir(213);
pair C = dir(47);
pair E = dir(115);
pair F = dir(280);

pair V = extension(F, E, B, C);
pair W = incenter(F, V, C);
pair J = 3.6*W-2.6*V;

pair T = foot(J, B, C);
pair A = extension(B, F, E, C);

filldraw(CP(J, T), opacity(0.1)+deepgreen, deepgreen);

pair X = OP(CP(J, T), CP(midpoint(B--J), J));
pair Y = IP(CP(J, T), CP(midpoint(E--J), J));
fill(A--B--C--cycle, rgb(0.9,1,1));

draw(B--E, lightblue);
draw(C--F, lightblue);
draw(X--B--A--Y, blue);
draw(B--C, blue);
draw(F--E, red);
draw(unitcircle, lightblue+dotted);

dot("$B$", B, dir(B));
dot("$C$", C, dir(C));
dot("$E$", E, dir(E));
dot("$F$", F, dir(F));
dot("$J$", J, dir(J));
dot("$A$", A, dir(A));
dot("$X$", X, dir(X));
dot("$Y$", Y, dir(Y));

/* TSQ Source:

!size(6cm);
B = dir 213
C = dir 47
E = dir 115
F = dir 280

V := extension F E B C
W := incenter F V C
J = 3.6*W-2.6*V

T := foot J B C
A = extension B F E C

CP J T 0.1 deepgreen / deepgreen

X = OP CP J T CP midpoint B--J J
Y = IP CP J T CP midpoint E--J J
!fill(A--B--C--cycle, rgb(0.9,1,1));

B--E lightblue
C--F lightblue
X--B--A--Y blue
B--C blue
F--E red
unitcircle lightblue dotted

*/
\end{asy}
\end{center}

Note that $AB > AE$ and $AX = AY$, therefore $BX < EY$.
By considering the right triangles $XBJ$ and $YEJ$
(which both have $JX = JY$),
we conclude $\tan \angle XBJ > \tan \angle YEJ$, thus
\[ \angle XBJ > \angle YEJ. \]
However, if line $EF$ was actually tangent to the $A$-excircle,
we would have
\[ 2\angle XBJ = \angle XBC = \angle FBC = \angle FEC
  = \angle FEY = 2 \angle JEY \]
which is a contradiction.

\paragraph{Seventh solution, by complex numbers, for comedic effect (Evan Chen).}
Let us denote the tangency points of the $A$-excircle
with sides $BC$, $CA$, $AB$ as $x$, $y$, $z$.
Assume moreover that line $EF$ is tangent to the $A$-excircle
at a point $P$.

Also, for brevity let $s = xy+yz+zx$.
Then, we have
\begin{align*}
  E = \frac{2py}{p+y} &= \half(b+y+y-y^2 \ol b)
  = \frac{zx}{z+x} + y - \frac{y^2}{z+x} \\
  \implies \frac{2}{\frac1p+\frac1y} &= \frac{xy+xz+zx-y^2}{z+x}
  \implies \frac{\frac1p+\frac1y}{2} = \frac{x+z}{s-y^2}. \\
  \intertext{Similarly by considering the point $F$,}
  \frac{\frac1p+\frac1z}{2} &= \frac{x+y}{s-z^2}. \\
  \intertext{Thus we can eliminate $P$ and obtain}
  \implies \frac{\frac1y-\frac1z}{2} &= \frac{x+z}{s-y^2} - \frac{x+y}{s-z^2}
  = \frac{-s(y-z) + x(y^2-z^2) + (y^3-z^3)}{(s-y^2)(s-z^2)} \\
  \iff \frac{1}{2yz} &= \frac{s - x(y+z) - (y^2+yz+z^2)}{(s-y^2)(s-z^2)}
    = \frac{-(y^2+z^2)}{(s-y^2)(s-z^2)} \\
  \iff 0 &= (s-y^2)(s-z^2) + 2yz(y^2+z^2) \\
  &= \left[ x(y+z) + y(z-y) \right]
    \left[ x(y+z) + z(y-z) \right] + 2yz(y^2+z^2) \\
  &= x^2(y+z)^2 - (y-z)^2 \cdot x(y+z) + yz(2y^2+2z^2 - (y-z)^2) \\
  &= x^2(y+z)^2 - (y-z)^2 \cdot x(y+z) + yz(y+z)^2 \\
  &= xyz(y+z) \left[ \frac xy + \frac xz
    - \frac yz - \frac zy + 2
    + \frac yx + \frac zx \right].
\end{align*}
However, $\triangle XYZ$ is obtuse with $\angle X > 90\dg$,
we have $y+z \neq 0$.
Note that
\begin{align*}
  \tfrac xy + \tfrac yx &= 2\opname{Re} \tfrac xy = 2 \cos(2 \angle XZY) \\
  \tfrac xz + \tfrac zx &= 2\opname{Re} \tfrac xz = 2 \cos(2 \angle XYZ) \\
  \tfrac yz + \tfrac zy &= 2\opname{Re} \tfrac yz < 2
\end{align*}
and since $\cos(2\angle XZY) + \cos(2\angle XYZ) > 0$
(say by sum-to-product), we are done.
\pagebreak

\subsection{JMO 2019/5, proposed by Ricky Liu}
\textsl{Available online at \url{https://aops.com/community/p12195861}.}
\begin{mdframed}[style=mdpurplebox,frametitle={Problem statement}]
Let $n$ be a nonnegative integer.
Determine the number of ways to choose sets
$S_{ij} \subseteq \{1, 2, \dots, 2n\}$,
for all $0 \le i \le n$ and $0 \le j \le n$
(not necessarily distinct), such that
\begin{itemize}
  \ii $|S_{ij}| = i+j$, and
  \ii $S_{ij} \subseteq S_{kl}$ if $0 \le i \le k \le n$
  and $0 \le j \le l \le n$.
\end{itemize}
\end{mdframed}
The answer is $(2n)! \cdot 2^{n^2}$.
First, we note that
$\varnothing = S_{00} \subsetneq S_{01} \subsetneq \dots \subsetneq S_{nn}
= \left\{ 1, \dots, 2n \right\}$
and thus multiplying by $(2n)!$
we may as well assume $S_{0i} = \left\{ 1, \dots, i \right\}$
and $S_{in} = \left\{ 1, \dots, n+i \right\}$.
We illustrate this situation by placing the sets in a grid,
as below for $n = 4$;
our goal is to fill in the rest of the grid.
\[
  \begin{bmatrix}
    1234 & 12345 & 123456 & 1234567 & 12345678\\
    123 \\
    12 \\
    1 \\
    \varnothing
  \end{bmatrix}
\]
We claim the number of ways to do so is $2^{n^2}$.
In fact, more strongly even the partial fillings
are given exactly by powers of $2$.
\begin{claim*}
  Fix a choice $T$ of cells we wish to fill in,
  such that whenever a cell is in $T$,
  so are all the cells above and left of it.
  (In other words, $T$ is a Young tableau.)
  The number of ways to fill in these cells with sets
  satisfying the inclusion conditions is $2^{|T|}$.
\end{claim*}
An example is shown below, with an indeterminate set marked in red
(and the rest of $T$ marked in blue).
\[
  \begin{bmatrix}
    1234 & 12345 & 123456 & 1234567 & 12345678\\
    123 & {\color{blue}1234} & {\color{blue}12346} & {\color{blue}123467} \\
    12 & {\color{blue}124} & {\color{red}1234 \text{ or } 1246} \\
    1 & {\color{blue}12} \\
    \varnothing & {\color{blue}2}
  \end{bmatrix}
\]
\begin{proof}
  The proof is by induction on $|T|$,
  with $|T| = 0$ being vacuous.

  Now suppose we have a corner $\begin{bmatrix}
  B & C \\ A & {\color{red}S} \end{bmatrix}$
  where $A$, $B$, $C$ are fixed and $S$ is to be chosen.
  Then we may write $B = A \cup \{x\}$ and $C = A \cup \{x,y\}$
  for $x,y \notin A$.
  Then the two choices of $S$ are $A \cup \{x\}$ (i.e.\ $B$)
  and $A \cup \{y\}$, and both of them are seen to be valid.

  In this way, we gain a factor of $2$
  any time we add one cell as above to $T$.
  Since we can achieve any Young tableau in this way,
  the induction is complete.
\end{proof}
\pagebreak

\subsection{JMO 2019/6, proposed by Yannick Yao}
\textsl{Available online at \url{https://aops.com/community/p12195834}.}
\begin{mdframed}[style=mdpurplebox,frametitle={Problem statement}]
Let $m$ and $n$ be relatively prime positive integers.
The numbers $\frac mn$ and $\frac nm$ are written on a blackboard.
At any point, Evan may pick two of the numbers $x$ and $y$
written on the board and write either their arithmetic mean $\half(x+y)$
or their harmonic mean $\frac{2xy}{x+y}$.
For which $(m,n)$ can Evan write $1$ on the board in finitely many steps?
\end{mdframed}
We claim this is possible if and only $m+n$ is a power of $2$.
Let $q = m/n$, so the numbers on the board
are $q$ and $1/q$.

\textbf{Impossibility}:
The main idea is the following.
\begin{claim*}
  Suppose $p$ is an odd prime.
  Then if the initial numbers on the board are $-1 \pmod p$,
  then all numbers on the board are $-1 \pmod p$.
\end{claim*}
\begin{proof}
  Let $a \equiv b \equiv -1 \pmod p$.
  Note that $2 \not\equiv 0 \pmod p$
  and $a+b \equiv -2 \not\equiv 0 \pmod p$.
  Thus $\frac{a+b}{2}$ and $\frac{2ab}{a+b}$
  both make sense modulo $p$ and are equal to $-1 \pmod p$.
\end{proof}

Thus if there exists \emph{any} odd prime divisor
$p$ of $m+n$
(implying $p \nmid mn$), then
\[ q \equiv \frac1q \equiv -1 \pmod p. \]
and hence all numbers will be $-1 \pmod p$ forever.
This implies that it's impossible to write $1$,
whenever $m+n$ is divisible by some odd prime.

\bigskip

\textbf{Construction}:
Conversely, suppose $m+n$ is a power of $2$.
We will actually construct $1$ without even using the harmonic mean.
\begin{center}
\begin{asy}
  unitsize(1.2cm);
  draw( (-4,0)--(4,0) );
  dot("$q$", (-4,0), dir(-90), blue);
  dot("$q^{-1}$", (4,0), dir(-90), blue);
  dot("$\frac{q+q^{-1}}{2}$", (0,0), dir(-90), blue);
  dot("$\frac{3q+q^{-1}}{4}$", (-2,0), dir(-90), red);
  dot("$\frac{q+3q^{-1}}{4}$", (2,0), dir(-90), red);
  dot("$\frac{7q+q^{-1}}{8}$", (-3,0), dir(90), deepgreen);
  dot("$\frac{5q+3q^{-1}}{8}$", (-1,0), dir(90), deepgreen);
  dot("$\frac{3q+5q^{-1}}{8}$", ( 1,0), dir(90), deepgreen);
  dot("$\frac{q+7q^{-1}}{8}$", ( 3,0), dir(90), deepgreen);
\end{asy}
\end{center}
Note that
\[
  \frac{n}{m+n} \cdot q
  + \frac{m}{m+n} \cdot \frac{1}{q}
  = 1
\]
and obviously by taking appropriate midpoints
(in a binary fashion) we can achieve this using
arithmetic mean alone.
\pagebreak


\end{document}
