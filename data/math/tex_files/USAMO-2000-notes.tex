% © Evan Chen
% Downloaded from https://web.evanchen.cc/

\documentclass[11pt]{scrartcl}
\usepackage[sexy]{evan}
\ihead{\footnotesize\textbf{\thetitle}}
\ohead{\footnotesize\href{http://web.evanchen.cc}{\ttfamily web.evanchen.cc},
    updated \today}
\title{USAMO 2000 Solution Notes}
\date{\today}

\begin{document}

\maketitle

\begin{abstract}
This is a compilation of solutions
for the 2000 USAMO.
Some of the solutions are my own work,
but many are from the official solutions provided by the organizers
(for which they hold any copyrights),
and others were found by users on the Art of Problem Solving forums.

These notes will tend to be a bit more advanced and terse than the ``official''
solutions from the organizers.
In particular, if a theorem or technique is not known to beginners
but is still considered ``standard'', then I often prefer to
use this theory anyways, rather than try to work around or conceal it.
For example, in geometry problems I typically use directed angles
without further comment, rather than awkwardly work around configuration issues.
Similarly, sentences like ``let $\mathbb{R}$ denote the set of real numbers''
are typically omitted entirely.

Corrections and comments are welcome!
\end{abstract}

\tableofcontents
\newpage

\addtocounter{section}{-1}
\section{Problems}
\begin{enumerate}[\bfseries 1.]
\item %% Problem 1
Call a real-valued function $f$ \emph{very convex} if
\[ \frac{f(x)+f(y)}{2}
  \ge f\left( \frac{x+y}{2} \right) + \left\lvert x-y \right\rvert  \]
holds for all real numbers $x$ and $y$.
Prove that no very convex function exists.

\item %% Problem 2
Let $S$ be the set of all triangles $ABC$ for which
\[ 5 \left( \frac{1}{AP} + \frac{1}{BQ} + \frac{1}{CR} \right)
  - \frac{3}{\min\{ AP, BQ, CR \}} = \frac{6}{r}, \]
where $r$ is the inradius and $P$, $Q$, $R$ are the points of tangency
of the incircle with sides $AB$, $BC$, $CA$ respectively.
Prove that all triangles in $S$ are isosceles and similar to one another.

\item %% Problem 3
A game of solitaire is played with $R$ red cards,
$W$ white cards, and $B$ blue cards.
A player plays all the cards one at a time.
With each play he accumulates a penalty.
If he plays a blue card, then he is charged a penalty
which is the number of white cards still in his hand.
If he plays a white card, then he is charged a penalty
which is twice the number of red cards still in his hand.
If he plays a red card, then he is charged a penalty
which is three times the number of blue cards still in his hand.

Find, as a function of $R$, $W$, and $B$, the minimal total penalty
a player can amass and the number of ways in which this minimum can be achieved.

\item %% Problem 4
Find the smallest positive integer $n$ such
that if $n$ squares of a $1000 \times 1000$ chessboard are colored,
then there will exist three colored squares
whose centers form a right triangle with sides
parallel to the edges of the board.

\item %% Problem 5
Let $A_1 A_2 A_3$ be a triangle,
and let $\omega_1$ be a circle in its plane
passing through $A_1$ and $A_2$.
Suppose there exists circles $\omega_2$, $\omega_3$, \dots, $\omega_7$
such that for $k = 2, 3, \dots, 7,$
circle $\omega_k$ is externally tangent to $\omega_{k-1}$ and passes
through $A_k$ and $A_{k+1}$ (indices mod $3$).
Prove that $\omega_7 = \omega_1$.

\item %% Problem 6
Let $a_1, b_1, a_2, b_2, \dots , a_n, b_n$ be nonnegative real numbers.
Prove that
\[ \sum_{i, j = 1}^{n} \min\{a_i a_j, b_i b_j\}
  \le \sum_{i, j = 1}^{n} \min\{a_i b_j, a_j b_i\}.  \]

\end{enumerate}
\pagebreak

\section{Solutions to Day 1}
\subsection{USAMO 2000/1}
\textsl{Available online at \url{https://aops.com/community/p299244}.}
\begin{mdframed}[style=mdpurplebox,frametitle={Problem statement}]
Call a real-valued function $f$ \emph{very convex} if
\[ \frac{f(x)+f(y)}{2}
  \ge f\left( \frac{x+y}{2} \right) + \left\lvert x-y \right\rvert  \]
holds for all real numbers $x$ and $y$.
Prove that no very convex function exists.
\end{mdframed}
For $C \ge 0$, we say a function $f$ is \emph{$C$-convex}
\[ \frac{f(x)+f(y)}{2}
  \ge f\left( \frac{x+y}{2} \right) + C\left\lvert x-y \right\rvert. \]

Suppose $f$ is $C$-convex.
Let $a < b < c < d < e$ be any arithmetic progression,
such that $t = |e-a|$.
Observe that
\begin{align*}
  f(a) + f(c) &\ge 2f(b) + C \cdot \half t \\
  f(c) + f(e) &\ge 2f(d) + C \cdot \half t \\
  f(b) + f(d) &\ge 2f(c) + C \cdot \half t
\end{align*}
Adding the first two to twice the third gives
\[ f(a) + f(e) \ge 2f(c) + 2C \cdot t. \]
So we conclude $C$-convex function is also $2C$-convex.
This is clearly not okay for $C > 0$.
\pagebreak

\subsection{USAMO 2000/2}
\textsl{Available online at \url{https://aops.com/community/p338078}.}
\begin{mdframed}[style=mdpurplebox,frametitle={Problem statement}]
Let $S$ be the set of all triangles $ABC$ for which
\[ 5 \left( \frac{1}{AP} + \frac{1}{BQ} + \frac{1}{CR} \right)
  - \frac{3}{\min\{ AP, BQ, CR \}} = \frac{6}{r}, \]
where $r$ is the inradius and $P$, $Q$, $R$ are the points of tangency
of the incircle with sides $AB$, $BC$, $CA$ respectively.
Prove that all triangles in $S$ are isosceles and similar to one another.
\end{mdframed}
We will prove the inequality
\[ \frac{2}{AP} + \frac{5}{BQ} + \frac{5}{CR} \ge \frac 6r \]
with equality when $AP : BQ : CR = 1 : 4 : 4$.
This implies the problem statement.

Letting $x= AP$, $y = BQ$, $z = CR$, the inequality becomes
\[ \frac2x + \frac5y + \frac5z \ge 6\sqrt{\frac{x+y+z}{xyz}}. \]
Squaring both sides and collecting terms gives
\[
  \frac{4}{x^2} + \frac{25}{y^2} + \frac{25}{z^2} + \frac{14}{yz}
  \ge \frac{16}{xy} + \frac{16}{xz}.
\]
If we replace $x=1/a$, $y=4/b$, $z=4/c$,
then it remains to prove the inequality
\[ 64a^2 + 25(b+c)^2 \ge 64a(b+c) + 36bc \]
where equality holds when $a=b=c$.
This follows by two applications of AM-GM:
\begin{align*}
  16 \left( 4a^2 + (b+c)^2 \right) &\ge 64a(b+c) \\
  9(b+c)^2 &\ge 36bc.
\end{align*}

Again one can tell this is an inequality by counting
degrees of freedom.
\pagebreak

\subsection{USAMO 2000/3}
\textsl{Available online at \url{https://aops.com/community/p338081}.}
\begin{mdframed}[style=mdpurplebox,frametitle={Problem statement}]
A game of solitaire is played with $R$ red cards,
$W$ white cards, and $B$ blue cards.
A player plays all the cards one at a time.
With each play he accumulates a penalty.
If he plays a blue card, then he is charged a penalty
which is the number of white cards still in his hand.
If he plays a white card, then he is charged a penalty
which is twice the number of red cards still in his hand.
If he plays a red card, then he is charged a penalty
which is three times the number of blue cards still in his hand.

Find, as a function of $R$, $W$, and $B$, the minimal total penalty
a player can amass and the number of ways in which this minimum can be achieved.
\end{mdframed}
The minimum penalty is
\[ f(B,W,R) = \min (BW, 2WR, 3RB) \]
or equivalently, the natural guess of
``discard all cards of one color first''
is actually optimal (though not necessarily unique).

This can be proven directly by induction.
Indeed the base case $BWR = 0$
(in which case zero penalty is clearly achievable).
The inductive step follows from
\[ f(B,W,R) = \min
  \begin{cases}
    f(B-1,W,R) + W \\
    f(B,W-1,R) + 2R \\
    f(B,W,R-1) + 3B.
  \end{cases}
\]

It remains to characterize the strategies.
This is an annoying calculation, so we just state the result.
\begin{itemize}
  \ii If any of the three quantities $BW$, $2WR$, $3RB$
  is strictly smaller than the other three,
  there is one optimal strategy.
  \ii If $BW = 2WR < 3RB$,
  there are $W+1$ optimal strategies,
  namely discarding from $0$ to $W$ white cards,
  then discarding all blue cards.
  (Each white card discarded still preserves $BW = 2WR$.)
  \ii If $2WR = 3RB < BW$,
  there are $R+1$ optimal strategies,
  namely discarding from $0$ to $R$ red cards,
  and then discarding all white cards.
  \ii If $3WR = RB < 2WR$,
  there are $B+1$ optimal strategies,
  namely discarding from $0$ to $B$ blue cards,
  and then discarding all red cards.
  \ii Now suppose $BW = 2WR = 3RB$.
  Discarding a card of one color
  ends up in exactly one of the previous three cases.
  This gives an answer of $R+W+B$ strategies.
\end{itemize}
\pagebreak

\section{Solutions to Day 2}
\subsection{USAMO 2000/4}
\textsl{Available online at \url{https://aops.com/community/p338084}.}
\begin{mdframed}[style=mdpurplebox,frametitle={Problem statement}]
Find the smallest positive integer $n$ such
that if $n$ squares of a $1000 \times 1000$ chessboard are colored,
then there will exist three colored squares
whose centers form a right triangle with sides
parallel to the edges of the board.
\end{mdframed}
The answer is $n = 1999$.

For a construction with $n = 1998$,
take a punctured L as illustrated below (with $1000$ replaced by $4$):
\[ \begin{bmatrix} 1 \\ 1 \\ 1 \\ & 1 & 1 & 1 \end{bmatrix}. \]

We now show that if there is no right triangle,
there are at most $1998$ tokens (colored squares).
In every column with more than two tokens,
we have token emit a bidirectional horizontal
death ray (laser) covering its entire row:
the hypothesis is that the death ray won't hit any other tokens.
\begin{center}
\begin{asy}
size(6cm);
for (int i=1; i<=8; ++i) {
  if ((i-1)*(i-3)*(i-5) != 0) draw( (0.5,i)--(8.5,i), red+1, Arrows(TeXHead) );
  for (int j=1; j<=8; ++j) {
    if ( (i-j)/2 == (i-j)#2 )
      filldraw(shift(i-0.5,j-0.5)*unitsquare, opacity(0.2)+grey, grey);
    else
      draw(shift(i-0.5,j-0.5)*unitsquare, grey);
  }
}

draw( (3,2)--(3,7), blue);
draw( (5,4)--(5,8), blue);
void token(pair P) { fill(circle(P, 0.2), black); }
token( (1,1) );
token( (2,1) );
token( (3,2) );
token( (3,6) );
token( (3,7) );
token( (4,1) );
token( (5,4) );
token( (5,8) );
token( (6,5) );
token( (7,3) );
token( (8,1) );
\end{asy}
\end{center}

Assume there are $n$ tokens and that $n > 1000$.
Then obviously some column has more than two tokens,
so at most $999$ tokens don't emit a death ray
(namely, any token in its own column).
Thus there are at least $n-999$ death rays.
On the other hand, we can have at most $999$ death rays total
(since it would not be okay for the whole board to have death rays,
as some row should have more than two tokens).
Therefore, $n \le 999 + 999 = 1998$ as desired.
\pagebreak

\subsection{USAMO 2000/5}
\textsl{Available online at \url{https://aops.com/community/p338089}.}
\begin{mdframed}[style=mdpurplebox,frametitle={Problem statement}]
Let $A_1 A_2 A_3$ be a triangle,
and let $\omega_1$ be a circle in its plane
passing through $A_1$ and $A_2$.
Suppose there exists circles $\omega_2$, $\omega_3$, \dots, $\omega_7$
such that for $k = 2, 3, \dots, 7,$
circle $\omega_k$ is externally tangent to $\omega_{k-1}$ and passes
through $A_k$ and $A_{k+1}$ (indices mod $3$).
Prove that $\omega_7 = \omega_1$.
\end{mdframed}
The idea is to keep track of the subtended arc
$\widehat{A_i A_{i+1}}$ of $\omega_i$ for each $i$.
To this end, let $\beta = \dang A_1 A_2 A_3$,
$\gamma = \dang A_2 A_3 A_1$
and $\alpha = \dang A_1 A_2 A_3$.

\begin{center}
\begin{asy}
size(8cm);
defaultpen(fontsize(9pt));
pair A2 = (0,0);
pair O1 = (-8,0);
pair O2 = (5,0);
pair A1 = abs(A2-O1)*dir(285)+O1;
pair A3 = abs(A2-O2)*dir(245)+O2;
pair O3 = extension(O2, A3, midpoint(A1--A3), midpoint(A1--A3)+dir(A1-A3)*dir(90));

filldraw(circle(O1,abs(A1-O1)), opacity(0.1)+lightgreen, deepgreen);
filldraw(circle(O2,abs(A2-O2)), opacity(0.1)+lightgreen, deepgreen);
filldraw(circle(O3,abs(A3-O3)), opacity(0.1)+lightgreen, deepgreen);
filldraw(A1--A2--A3--cycle, opacity(0.1)+orange, red);

pair J = (0,-3.2);
draw(O1--O2--O3, deepcyan+dotted);

dot("$O_1$", O1, dir(90), deepcyan);
dot("$O_2$", O2, dir(90), deepcyan);
dot("$O_3$", O3, dir(-90), deepcyan);
dot("$A_1$", A1, dir(A1-J), red);
dot("$A_2$", A2, dir(A2-J), red);
dot("$A_3$", A3, dir(A3-J), red);

label("$\alpha$", A1, 2.4*dir(J-A1), red);
label("$\beta$",  A2, 2*dir(J-A2), red);
label("$\gamma$", A3, 2*dir(J-A3), red);
\end{asy}
\end{center}

Initially, we set $\theta = \dang O_1 A_2 A_1$.
Then we compute
\begin{align*}
  \dang O_1 A_2 A_1 &= \theta \\
  \dang O_2 A_3 A_2 &= -\beta-\theta \\
  \dang O_3 A_1 A_3 &= \beta-\gamma+\theta \\
  \dang O_4 A_2 A_1 &= (\gamma-\beta-\alpha)-\theta \\
\end{align*}
and repeating the same calculation another round gives
\[ \dang O_7 A_2 A_1 = k-(k-\theta) = \theta \]
with $k = \gamma-\beta-\alpha$.
This implies $O_7 = O_1$, so $\omega_7 = \omega_1$.
\pagebreak

\subsection{USAMO 2000/6, proposed by Gheorghita Zbaganu}
\textsl{Available online at \url{https://aops.com/community/p108437}.}
\begin{mdframed}[style=mdpurplebox,frametitle={Problem statement}]
Let $a_1, b_1, a_2, b_2, \dots , a_n, b_n$ be nonnegative real numbers.
Prove that
\[ \sum_{i, j = 1}^{n} \min\{a_i a_j, b_i b_j\}
  \le \sum_{i, j = 1}^{n} \min\{a_i b_j, a_j b_i\}.  \]
\end{mdframed}
We present two solutions.

\paragraph{First solution by creating a single min (Vincent Huang and Ravi Boppana).}
Let $b_i = r_i a_i$ for each $i$,
and rewrite the inequality as
\[ \sum_{i,j} a_i a_j \left[
  \min (r_i, r_j) - \min(1, r_i r_j) \right] \ge 0. \]
We now do the key manipulation to convert the double min
into a separate single min.
Let $\eps_i = +1$ if $r_i \ge 1$, and $\eps_i = -1$ otherwise,
and let $s_i = |r_i-1|$.
Then we pass to absolute values:
\begin{align*}
  2\min(r_i,r_j) - 2\min(1,r_i r_j)
  &= |r_i r_j-1| - |r_i-r_j|  - (r_i-1)(r_j-1) \\
  &= |r_i r_j-1| - |r_i-r_j| - \eps_i \eps_j s_i s_j \\
  &= \eps_i \eps_j \min \left( |1-r_i r_j \pm (r_i-r_j)| \right)
  - \eps_i \eps_j s_i s_j \\
  &= \eps_i \eps_j \min \left( s_i(r_j+1), s_j(r_i+1) \right)
  - \eps_i \eps_j s_i s_j \\
  &= (\eps_i s_i)(\eps_j s_j)
  \min \left( \frac{r_j+1}{s_j}-1, \frac{r_i+1}{s_i}-1  \right).
\end{align*}
So let us denote $x_i = a_i \eps_i s_i \in \RR$,
and $t_i = \frac{r_i+1}{s_i}-1 \in \RR_{\ge 0}$.
Thus it suffices to prove that:
\begin{claim*}
  We have
  \[ \sum_{i,j} x_i x_j \min(t_i, t_j) \ge 0 \]
  for arbitrary $x_i \in \RR$, $t_i \in \RR_{\ge 0}$.
\end{claim*}
\begin{proof}
  One can just check this ``by hand''
  by assuming $t_1 \le t_2 \le \dots \le t_n$;
  then the left-hand side becomes
  \[ \sum_i t_i x_i^2 + 2 \sum_{i < j} 2 t_i x_i x_j
    = \sum_i (t_i - t_{i-1}) (x_i + x_{i+1} + \dots + x_n)^2 \ge 0.  \]
  There is also a nice proof using the integral identity
  \[ \min(t_i, t_j) = \int_0^\infty \mathbf1(u \le t_i) \mathbf1(u \le t_j) \; du \]
  where the $\mathbf1$ are indicator functions.
  Indeed,
  \begin{align*}
    \sum_{i,j} x_i x_j \min(t_i, t_j)
    &= \sum_{i,j} x_i x_j \int_0^\infty \mathbf1(u \le t_i) \mathbf1(u \le t_j) \; du \\
    &= \int_0^\infty \sum_i x_i \mathbf1(u \le t_i)
    \sum_j x_j \mathbf1(u \le t_j) \; du \\
    &= \int_0^\infty \left( \sum_i x_i \mathbf1(u \le t_i) \right)^2 \; du \\
    &\ge 0. \qedhere
  \end{align*}
\end{proof}


\paragraph{Second solution by smoothing (Alex Zhai).}
The case $n=1$ is immediate, so we'll proceed by induction on $n \ge 2$.

Again, let $b_i = r_i a_i$ for each $i$, and
write the inequality as
\[ L_n(a_1, \dots, a_n, r_1, \dots, r_n)
  \coloneqq \sum_{i,j} a_i a_j \left[
  \min (r_i, r_j) - \min(1, r_i r_j) \right] \ge 0. \]
First note that if $r_1 = r_2$ then
\[ L_n(a_1, a_2, a_3, \dots, r_1, r_1, r_3 \dots)
  = L_{n-1}(a_1+a_2, a_3, \dots, r_1, r_3, \dots) \]
and so our goal is to smooth to a situation
where two of the $r_i$'s are equal,
so that we may apply induction.

On the other hand, $L_n$ is a \emph{piecewise linear} function in $r_1 \ge 0$.
Let us smooth $r_1$ then.
Note that if the minimum is attained at $r_1 = 0$,
we can ignore $a_1$ and reduce to the $(n-1)$-variable case.
On the other hand, the minimum must be achieved at a cusp
which opens upward, which can only happen if $r_i r_j = 1$ for some $j$.
(The $r_i = r_j$ cusps open downward, sadly.)

In this way, whenever some $r_i$ is not equal to the reciprocal
of any other $r_\bullet$, we can smooth it.
This terminates; so we may smooth until we reach a situation for which
\[ \{ r_1, \dots, r_n \} = \{ 1/r_1, \dots, 1/r_n \}. \]
Now, assume WLOG that $r_1 = \max_i r_i$ and $r_2 = \min_i r_i$,
hence $r_1 r_2 = 1$ and $r_1 \ge 1 \ge r_2$.
We isolate the contributions from $a_1$, $a_2$, $r_1$ and $r_2$.
\begin{align*}
  L_n(\dots) &= a_1^2 \left[ r_1 - 1  \right] + a_2^2 \left[ r_2 - r_2^2 \right]
    + 2a_1a_2 \left[ r_2 - 1 \right] \\
  &+ 2a_1 \left[ (a_3r_3+\dots+a_n r_n) - (a_3+\dots+a_n) \right] \\
  &+ 2a_2r_2 \left[ (a_3+\dots+a_n) - (a_3r_3+\dots+a_n r_n) \right] \\
  &+ \sum_{i=3}^n \sum_{j=3}^n a_i a_j \left[
    \min (r_i, r_j) - \min(1, r_i r_j) \right].
\end{align*}
The idea now is to smooth via
\[ (a_1, a_2, r_1, r_2)
  \longrightarrow
  \left( a_1, \frac 1t a_2, \frac 1t r_1, tr_2 \right) \]
where $t \ge 1$ is such that
$\frac 1t r_1 \ge \max(1, r_3, \dots, r_n)$ holds.
(This choice is such that $a_1$ and $a_2 r_2$ are unchanged,
because we don't know the sign of $\sum_{i \ge 3} (1-r_i) a_i$
and so the post-smoothing value is still at least the max.)
Then,
\begin{align*}
  &\phantom{ {} = {} } L_n(a_1, a_2, \dots, r_1, r_2, \dots)
  - L_n\left(a_1, \frac 1t a_2, \dots, \frac1t r_1, tr_2\right) \\
  &= a_1^2 \left( r_1 - \frac 1t r_1 \right)
  + a_2^2 \left( r_2 - \frac 1t r_2 \right)
  + 2a_1a_2 \left( \frac 1t - 1 \right) \\
  &= \left( 1 - \frac 1t \right) \left( r_1 a_1^2 + r_2 a_2^2 - 2a_1a_2 \right)
  \ge 0
\end{align*}
the last line by AM-GM.
Now pick $t = \frac{r_1}{\max(1, r_3, \dots, r_n)}$,
and at last we can induct down.
\pagebreak


\end{document}
