\documentclass[11pt]{scrartcl}
\usepackage[sexy]{evan}
\ihead{\footnotesize\textbf{\thetitle}}
\ohead{\footnotesize\theauthor}
\begin{document}

\title{USA TSTST 2021 Solutions}
\subtitle{United States of America --- TST Selection Test}
\author{Andrew Gu and Evan Chen}
\date{63\ts{rd} IMO 2022 Norway and 11\ts{th} EGMO 2022 Hungary}

\maketitle

\tableofcontents
\newpage

\addtocounter{section}{-1}
\section{Problems}
\begin{enumerate}[\bfseries 1.]
\item %% Problem 1
Let $ABCD$ be a quadrilateral inscribed in a circle with center $O$.
Points $X$ and $Y$ lie on sides $AB$ and $CD$, respectively.
Suppose the circumcircles of $ADX$ and $BCY$
meet line $XY$ again at $P$ and $Q$, respectively.
Show that $OP=OQ$.

\item %% Problem 2
Let $a_1 < a_2 < a_3 < a_4 < \dotsb$ be an
infinite sequence of real numbers in the interval $(0,1)$.
Show that there exists a number that occurs
exactly once in the sequence
\[ \frac{a_1}{1}, \; \frac{a_2}{2}, \;
  \frac{a_3}{3}, \; \frac{a_4}{4}, \; \dots. \]

\item %% Problem 3
Find all positive integers $k > 1$ for which there exists a positive integer
$n$ such that $\binom{n}{k}$ is divisible by $n$, and $\binom{n}{m}$ is not
divisible by $n$ for $2\leq m < k$.

\item %% Problem 4
Let $a$ and $b$ be positive integers.
Suppose that there are infinitely many pairs of positive integers $(m, n)$
for which $m^2+an+b$ and $n^2+am+b$ are both perfect squares.
Prove that $a$ divides $2b$.

\item %% Problem 5
Let $T$ be a tree on $n$ vertices with exactly $k$ leaves.
Suppose that there exists a subset of
at least $\frac{n+k-1}{2}$ vertices of $T$,
no two of which are adjacent.
Show that the longest path in $T$ contains
an even number of edges.

\item %% Problem 6
Triangles $ABC$ and $DEF$ share circumcircle $\Omega$ and incircle $\omega$
so that points $A$, $F$, $B$, $D$, $C$, and $E$ occur in this order along $\Omega$.
Let $\Delta_A$ be the triangle formed by lines $AB$, $AC$, and $EF$,
and define triangles $\Delta_B$, $\Delta_C, \dots, \Delta_F$ similarly.
Furthermore, let $\Omega_A$ and $\omega_A$ be the circumcircle and incircle
of triangle $\Delta_A$, respectively, and define circles
$\Omega_B$, $\omega_B, \dots, \Omega_F$, $\omega_F$ similarly.
\begin{enumerate}[(a)]
  \item Prove that the two common external tangents to circles $\Omega_A$ and $\Omega_D$
    and the two common external tangents to circles $\omega_A$ and $\omega_D$
    are either concurrent or pairwise parallel.

  \item Suppose that these four lines meet at point $T_A$,
    and define points $T_B$ and $T_C$ similarly.
    Prove that points $T_A$, $T_B$, and $T_C$ are collinear.
\end{enumerate}

\item %% Problem 7
Let $M$ be a finite set of lattice points
and $n$ be a positive integer.
A \emph{mine-avoiding path} is a path of lattice points with length $n$,
beginning at $(0,0)$ and ending at a point on the line $x+y=n$,
that does not contain any point in $M$.
Prove that if there exists a mine-avoiding path,
then there exist at least $2^{n-|M|}$ mine-avoiding paths.

\item %% Problem 8
Let $ABC$ be a scalene triangle.
Points $A_1$, $B_1$ and $C_1$ are chosen
on segments $BC$, $CA$, and $AB$, respectively,
such that $\triangle A_1B_1C_1$ and $\triangle ABC$
are similar.
Let $A_2$ be the unique point on line $B_1C_1$
such that $AA_2 = A_1A_2$.
Points $B_2$ and $C_2$ are defined similarly.
Prove that $\triangle A_2B_2C_2$ and $\triangle ABC$ are similar.

\item %% Problem 9
Let $q=p^r$ for a prime number $p$ and positive integer $r$.
Let $\zeta = e^{\frac{2\pi i}{q}}$.
Find the least positive integer $n$ such that
\[
  \sum_{\substack{1 \le k \le q \\ \gcd(k,p) = 1}}
  \frac{1}{(1 - \zeta^k)^n}
\]
is not an integer.
(The sum is over all $1\leq k\leq q$ with $p$ not dividing $k$.)

\end{enumerate}
\pagebreak

\section{Solutions to Day 1}
\subsection{TSTST 2021/1, proposed by Holden Mui}
\textsl{Available online at \url{https://aops.com/community/p23586650}.}
\begin{mdframed}[style=mdpurplebox,frametitle={Problem statement}]
Let $ABCD$ be a quadrilateral inscribed in a circle with center $O$.
Points $X$ and $Y$ lie on sides $AB$ and $CD$, respectively.
Suppose the circumcircles of $ADX$ and $BCY$
meet line $XY$ again at $P$ and $Q$, respectively.
Show that $OP=OQ$.
\end{mdframed}
We present many solutions.

\paragraph{First solution, angle chasing only (Ankit Bisain).}
Let lines $BQ$ and $DP$ meet $(ABCD)$ again at $D'$ and $B'$, respectively.
\begin{center}
\begin{asy}[width = 0.5\textwidth]
path omega = circle((0, 0), 1);
pair A = dir(70);
pair B = dir(180);
pair C = dir(200);
pair D = dir(340);
pair O = (0, 0);

real x = 0.2;
real y = 0.3;
pair X = (1-x) * A + x * B;
pair Y = (1-y) * C + y * D;

pair P = 2 * foot(circumcenter(A, D, X), X, Y) - X;
pair Q = 2 * foot(circumcenter(B, C, Y), X, Y) - Y;

dot("$A$", A, A);
dot("$B$", B, B);
dot("$C$", C, C);
dot("$D$", D, D);
dot("$O$", O, -dir(A+B+C+D));
dot("$X$", X, dir(A+B));
dot("$Y$", Y, dir(C+D));
dot("$P$", P, dir(30));
dot("$Q$", Q, dir(-60));

draw(omega);
draw(A -- B -- C -- D -- cycle);
draw(circumcircle(A, D, X));
draw(circumcircle(B, C, Y));
draw(X -- Y);
draw(P -- O -- Q, dotted);

pair Bp = 2 * foot(B, O, foot(O, X, Y)) - B;
dot("$B'$", Bp, dir(Bp));
pair Dp = 2 * foot(D, O, foot(O, X, Y)) - D;
dot("$D'$", Dp, dir(300));

draw(B -- Dp -- D -- Bp -- cycle, dashed);
\end{asy}
\end{center}
Then $BB' \parallel PX$ and $DD' \parallel QY$ by Reim's theorem. Segments
$BB', DD'$, and $PQ$ share a perpendicular bisector which passes through $O$, so
$OP=OQ$.

\paragraph{Second solution via isosceles triangles (from contestants).}
Let $T = \ol{BQ} \cap \ol{DP}$.
\begin{center}
\begin{asy}
  size(200);
  path omega = circle((0, 0), 1);
  pair A = dir(70);
  pair B = dir(180);
  pair C = dir(200);
  pair D = dir(340);
  pair O = (0, 0);

  real x = 0.2;
  real y = 0.3;
  pair X = (1-x) * A + x * B;
  pair Y = (1-y) * C + y * D;

  pair P = 2 * foot(circumcenter(A, D, X), X, Y) - X;
  pair Q = 2 * foot(circumcenter(B, C, Y), X, Y) - Y;

  pair T = extension(B, Q, D, P);

  dot("$A$", A, A);
  dot("$B$", B, dir(135));
  dot("$C$", C, C);
  dot("$D$", D, 2*dir(5));
  dot("$O$", O, dir(60));
  dot("$X$", X, dir(A+B));
  dot("$Y$", Y, dir(C+D));
  dot("$P$", P, dir(30));
  dot("$Q$", Q, dir(-60));
  dot("$T$", T, dir(240));

  draw(omega);
  draw(A -- B -- C -- D -- cycle);
  draw(circumcircle(A, D, X));
  draw(circumcircle(B, C, Y));
  draw(X -- Y);
  draw(P -- O -- Q, dotted);
  draw(B -- T -- P, dashed);

  pair O1 = circumcenter(B, O, D);
  real r = abs(O1 - O);
  draw(arc(O1, r, 45, 120), dashed);
\end{asy}
\end{center}
Note that $PQT$ is isosceles because
\[ \dang PQT = \dang YQB = \dang BCD = \dang BAD = \dang XPD = \dang TPQ.  \]
Then $(BODT)$ is cyclic because
\[\dang BOD = 2 \dang BCD = \dang PQT + \dang TPQ = \dang BTD.\]
Since $BO=OD$, $\ol{TO}$ is an angle bisector of $\dang BTD$. Since $\triangle PQT$ is isosceles, $\ol{TO} \perp \ol{PQ}$, so $OP = OQ$.

\paragraph{Third solution using a parallelogram (from contestants).}
Let $(BCY)$ meet $\ol{AB}$ again at $W$ and let $(ADX)$ meet $\ol{CD}$ again at $Z$. Additionally, let $O_1$ be the center of $(ADX)$ and $O_2$ be the center of $(BCY)$.
\begin{center}
\begin{asy}
size(200);
path omega = circle((0, 0), 1);
pair A = dir(70);
pair B = dir(180);
pair C = dir(200);
pair D = dir(340);
pair O = (0, 0);

real x = 0.2;
real y = 0.3;
pair X = (1-x) * A + x * B;
pair Y = (1-y) * C + y * D;

pair P = 2 * foot(circumcenter(A, D, X), X, Y) - X;
pair Q = 2 * foot(circumcenter(B, C, Y), X, Y) - Y;

pair T = extension(B, Q, D, P);

draw(omega);

draw(A -- B ^^ C -- D);
draw(circumcircle(A, D, X));
draw(circumcircle(B, C, Y));
draw(X -- Y);

pair W = 2 * foot(circumcenter(B, C, Y), A, B) - B;
pair Z = 2 * foot(circumcenter(A, D, X), C, D) - D;

pair O1 = circumcenter(A, D, X);
pair O2 = circumcenter(B, C, Y);
pair Op = circumcenter(X, Y, Z);

draw(W -- Z);
draw(circumcircle(X, Y, Z), dashed);

dot("$A$", A, A);
dot("$B$", B, B);
dot("$C$", C, C);
dot("$D$", D, D);
dot("$X$", X, dir(A+B));
dot("$Y$", Y, dir(C+D));
dot("$P$", P, dir(-45));
dot("$Q$", Q, dir(0));
dot("$W$", W, dir(A+B));
dot("$Z$", Z, dir(C+D));

dot("$O$", O, dir(300));
dot("$O_1$", O1, dir(O1));
dot("$O_2$", O2, dir(O2));
dot("$O'$", Op, dir(300));

draw(O -- O1 -- Op -- O2 -- cycle, dotted);
\end{asy}
\end{center}
Note that $(WXYZ)$ is cyclic since
\[\dang XWY + \dang YZX = \dang YWB + \dang XZD = \dang YCB + \dang XAD = 0^\circ,\]
so let $O'$ be the center of $(WXYZ)$. Since $\ol{AD} \parallel \ol{WY}$ and $\ol{BC} \parallel \ol{XZ}$ by Reim's theorem, $OO_1O'O_2$ is a parallelogram.

To finish the problem, note that projecting $O_1$, $O_2$, and $O'$ onto
$\ol{XY}$ gives the midpoints of $\ol{PX}$, $\ol{QY}$, and
$\ol{XY}$. Since $OO_1O'O_2$ is a parallelogram, projecting $O$ onto
$\ol{XY}$ must give the midpoint of $\ol{PQ}$, so $OP=OQ$.

\paragraph{Fourth solution using congruent circles (from contestants).}
Let the angle bisector of $\dang BOD$ meet $\ol{XY}$ at $K$.
\begin{center}
\begin{asy}
  size(200);
  path omega = circle((0, 0), 1);
  pair A = dir(70);
  pair B = dir(180);
  pair C = dir(200);
  pair D = dir(340);
  pair O = (0, 0);

  real x = 0.2;
  real y = 0.3;
  pair X = (1-x) * A + x * B;
  pair Y = (1-y) * C + y * D;

  pair P = 2 * foot(circumcenter(A, D, X), X, Y) - X;
  pair Q = 2 * foot(circumcenter(B, C, Y), X, Y) - Y;

  pair T = extension(B, Q, D, P);
  dot("$A$", A, A);
  dot("$B$", B, B);
  dot("$C$", C, C);
  dot("$D$", D, D);
  dot("$O$", O, 2*dir(15));
  dot("$X$", X, 1/3*dir(A+B));
  dot("$Y$", Y, dir(C+D));
  dot("$P$", P, dir(0));
  dot("$Q$", Q, dir(-30));


  pair K = extension(O, foot(O, B, D), X, Y);
  dot("$K$", K, 2*dir(K));
  draw(X -- K -- O);
  draw(B -- O -- D);

  draw(omega);
  draw(A -- B -- C -- D -- cycle);
  draw(circumcircle(A, D, X));
  draw(circumcircle(B, C, Y));
  draw(X -- Y);

  draw(circumcircle(O, P, D), dotted);
  draw(circumcircle(O, Q, B), dotted);
\end{asy}
\end{center}
Then $(BQOK)$ is cyclic because $\dang KOD = \dang BAD = \dang KPD$, and $(DOPK)$ is cyclic similarly.
By symmetry over $KO$, these circles have the same radius $r$, so
\[ OP = 2r \sin \angle OKP = 2r \sin \angle OKQ = OQ \]
by the Law of Sines.

\paragraph{Fifth solution by ratio calculation (from contestants).}
Let $\ol{XY}$ meet $(ABCD)$ at $X'$ and $Y'$.
\begin{center}
\begin{asy}
size(200);
path omega = circle((0, 0), 1);
pair A = dir(70);
pair B = dir(180);
pair C = dir(200);
pair D = dir(340);
pair O = (0, 0);

real x = 0.2;
real y = 0.3;
pair X = (1-x) * A + x * B;
pair Y = (1-y) * C + y * D;

pair P = 2 * foot(circumcenter(A, D, X), X, Y) - X;
pair Q = 2 * foot(circumcenter(B, C, Y), X, Y) - Y;

pair T = extension(B, Q, D, P);

pair Xp = intersectionpoints(circumcircle(A,B,C), Y + (Y-X)*10 -- X + (X-Y)*10)[0];
pair Yp = intersectionpoints(circumcircle(A,B,C), Y + (Y-X)*10 -- X + (X-Y)*10)[1];

fill(D -- P -- Xp -- cycle, lightblue+opacity(0.5));
fill(B -- Yp -- D -- cycle, lightblue+opacity(0.5));

// fill(B -- D -- X -- cycle, lightgreen+opacity(0.5));
//fill(B -- Xp -- D -- cycle, lightgreen+opacity(0.5));

dot("$A$", A, A);
dot("$B$", B, B);
dot("$C$", C, C);
dot("$D$", D, D);
dot("$O$", O, -dir(A+B+C+D));
dot("$X$", X, dir(A+B));
dot("$Y$", Y, dir(-45));
dot("$P$", P, dir(0));
dot("$Q$", Q, dir(-30));

draw(omega);
draw(A -- B -- C -- D -- cycle);
draw(circumcircle(A, D, X));
draw(circumcircle(B, C, Y));

draw(Xp -- Yp);

dot("$X'$", Xp, dir(Xp));
dot("$Y'$", Yp, dir(Yp));
\end{asy}
\end{center}
Since $\dang Y'BD = \dang PX'D$ and $\dang BY'D = \dang BAD = \dang X'PD$,
\[ \triangle BY'D \sim \triangle XP'D \implies PX' = BY' \cdot \frac{DX'}{BD}.\]
Similarly,
\[ \triangle BX'D \sim \triangle BQY' \implies QY' = DX' \cdot \frac{BY'}{BD}.\]
Thus $PX' = QY'$, which gives $OP=OQ$.

\paragraph{Sixth solution using radical axis (from author).}
Without loss of generality, assume $\ol{AD} \nparallel \ol{BC}$, as this case holds by continuity. Let $(BCY)$ meet $\ol{AB}$ again at $W$, let $(ADX)$ meet $\ol{CD}$ again at $Z$, and let $\ol{WZ}$ meet $(ADX)$ and $(BCY)$ again at $R$ and $S$.
\begin{center}
\begin{asy}
size(200);
path omega = circle((0, 0), 1);
pair A = dir(70);
pair B = dir(180);
pair C = dir(200);
pair D = dir(340);
pair O = (0, 0);

real x = 0.2;
real y = 0.3;
pair X = (1-x) * A + x * B;
pair Y = (1-y) * C + y * D;

pair P = 2 * foot(circumcenter(A, D, X), X, Y) - X;
pair Q = 2 * foot(circumcenter(B, C, Y), X, Y) - Y;

pair T = extension(B, Q, D, P);

draw(omega);

draw(A -- B ^^ C -- D);
draw(circumcircle(A, D, X));
draw(circumcircle(B, C, Y));
draw(X -- Y);

pair W = 2 * foot(circumcenter(B, C, Y), A, B) - B;
pair Z = 2 * foot(circumcenter(A, D, X), C, D) - D;
pair R = 2 * foot(circumcenter(A, D, X), W, Z) - Z;
pair S = 2 * foot(circumcenter(B, C, Y), W, Z) - W;

draw(W -- Z);
draw(circumcircle(P, Q, R), dashed);
draw(circumcircle(X, Y, Z), dashed);

dot("$A$", A, A);
dot("$B$", B, B);
dot("$C$", C, C);
dot("$D$", D, D);
dot("$X$", X, dir(A+B));
dot("$Y$", Y, dir(C+D));
dot("$P$", P, dir(-45));
dot("$Q$", Q, dir(0));
dot("$W$", W, dir(A+B));
dot("$Z$", Z, dir(C+D));
dot("$R$", R, dir(90));
dot("$S$", S, dir(210));
\end{asy}
\end{center}
Note that $(WXYZ)$ is cyclic since
\[\dang XWY + \dang YZX = \dang YWB + \dang XZD = \dang YCB + \dang XAD = 0^\circ\]
and $(PQRS)$ is cyclic since
\[\dang PQS = \dang YQS = \dang YWS = \dang PXZ = \dang PRZ = \dang SRP.\]
Additionally, $\ol{AD} \parallel \ol{PR}$ since
\[\dang DAX + \dang AXP + \dang XPR = \dang YWX + \dang WXY + \dang XYW = 0^\circ,\]
and $\ol{BC} \parallel \ol{SQ}$ similarly.

Lastly, $(ABCD)$ and $(PQRS)$ are concentric; if not, using the radical axis theorem twice shows that their radical axis must be parallel to both $\ol{AD}$ and $\ol{BC}$, contradiction.

\paragraph{Seventh solution using Cayley-Bacharach (author).}
Define points $W, Z, R, S$ as in the previous solution.
\begin{center}
\begin{asy}[width = 0.5\textwidth]
path omega = circle((0, 0), 1);
pair A = dir(70);
pair B = dir(180);
pair C = dir(200);
pair D = dir(340);
pair O = (0, 0);

real x = 0.2;
real y = 0.3;
pair X = (1-x) * A + x * B;
pair Y = (1-y) * C + y * D;

pair P = 2 * foot(circumcenter(A, D, X), X, Y) - X;
pair Q = 2 * foot(circumcenter(B, C, Y), X, Y) - Y;
draw(omega, green);

draw(A -- B ^^ C -- D, red);
draw(circumcircle(A, D, X), blue);
draw(circumcircle(B, C, Y), blue);
draw(X -- Y, green);

pair W = 2 * foot(circumcenter(B, C, Y), A, B) - B;
pair Z = 2 * foot(circumcenter(A, D, X), C, D) - D;
pair R = 2 * foot(circumcenter(A, D, X), W, Z) - Z;
pair S = 2 * foot(circumcenter(B, C, Y), W, Z) - W;

draw(W -- Z, green);
draw(circumcircle(P, Q, R), red+dashed);

dot("$A$", A, A);
dot("$B$", B, B);
dot("$C$", C, C);
dot("$D$", D, D);
dot("$X$", X, dir(A+B));
dot("$Y$", Y, dir(C+D));
dot("$P$", P, dir(-45));
dot("$Q$", Q, dir(0));
dot("$W$", W, dir(A+B));
dot("$Z$", Z, dir(C+D));
dot("$R$", R, dir(90));
dot("$S$", S, dir(210));
\end{asy}
\end{center}
The quartics $(ADXZ) \cup (BCWY)$ and $\ol{XY} \cup \ol{WZ} \cup (ABCD)$ meet at the 16 points
\[A, B, C, D, W, X, Y, Z, P, Q, R, S, I, I, J, J,\]
where $I$ and $J$ are the circular points at infinity. Since $\ol{AB} \cup \ol{CD} \cup (PQR)$ contains the 13 points
\[A,B,C,D,P,Q,R,W,X,Y,Z,I,J,\]
it must contain $S$, $I$, and $J$ as well, by quartic Cayley-Bacharach.
Thus, $(PQRS)$ is cyclic and intersects $(ABCD)$ at $I$, $I$, $J$, and $J$, implying that the two circles are concentric, as desired.

\begin{remark*}
  [Author comments]
  Holden says he came up with this problem via the Cayley-Bacharach solution,
  by trying to get two quartics to intersect.
\end{remark*}
\pagebreak

\subsection{TSTST 2021/2, proposed by Merlijn Staps}
\textsl{Available online at \url{https://aops.com/community/p23586635}.}
\begin{mdframed}[style=mdpurplebox,frametitle={Problem statement}]
Let $a_1 < a_2 < a_3 < a_4 < \dotsb$ be an
infinite sequence of real numbers in the interval $(0,1)$.
Show that there exists a number that occurs
exactly once in the sequence
\[ \frac{a_1}{1}, \; \frac{a_2}{2}, \;
  \frac{a_3}{3}, \; \frac{a_4}{4}, \; \dots. \]
\end{mdframed}
We present three solutions.
\paragraph{Solution 1 (Merlijn Staps).}
We argue by contradiction, so suppose that for each $\lambda$ for which the set
$S_\lambda = \{k : a_k/k = \lambda\}$ is  non-empty, it contains at least two
elements.  Note that $S_\lambda$ is always a finite set because  $a_k =
k\lambda$ implies $k < 1/\lambda$.

Write $m_\lambda$ and $M_\lambda$ for the smallest and largest element of
$S_\lambda$,  respectively, and define $T_\lambda = \{m_\lambda,
m_\lambda+1,\dots,M_\lambda\}$ as the smallest set of consecutive positive
integers  that contains $S_\lambda$.  Then all $T_\lambda$ are sets of at least
two consecutive positive integers, and moreover the $T_\lambda$ cover
$\NN$. Additionally, each positive integer is covered finitely many times
because there are only finitely many possible values of $m_{\lambda}$ smaller
than any fixed integer.

Recall that if three intervals have a point in common then one
of them is contained in the union of the other two. Thus, if any positive
integer is covered more than twice by the sets $T_{\lambda}$, we may throw out
one set while maintaining the property that the $T_{\lambda}$
cover $\NN$. By using the fact that each positive integer is covered
finitely many times, we can apply this process so that each positive integer is
eventually covered at most twice.
%By repeatedly throwing away some of the $T_\lambda$ we may assume
%that each positive integer is contained in at most two sets $T_\lambda$, while
%still maintaining the property that the $T_\lambda$ cover $\NN$; this
%follows from the fact that  if three intervals have a point in common then one
%of them is contained in the union of the other two.

Let $\Lambda$ denote the set
of the $\lambda$-values for which $T_\lambda$ remains in our collection of sets;
then $\bigcup_{\lambda \in \Lambda} T_\lambda = \NN$ and each positive
integer is contained in at most two sets $T_\lambda$.

We now obtain
\[
\sum_{\lambda \in \Lambda} \sum_{k \in T_\lambda} (a_{k+1}-a_k) \le 2 \sum_{k \ge 1} (a_{k+1} - a_k) \le 2.
\]
On the other hand,  because $a_{m_\lambda} = \lambda m_\lambda$ and $a_{M_\lambda} = \lambda M_\lambda$, we have
\begin{align*}
  2\sum_{k \in T_\lambda} (a_{k+1} - a_k) &\ge2 \sum_{m_\lambda \le k < M_\lambda}
  (a_{k+1} - a_k) = 2(a_{M_\lambda}-a_{m_\lambda}) = 2(M_\lambda-m_\lambda)\lambda
                                        \\
                                          & = 2(M_\lambda - m_\lambda) \cdot
                                        \frac{a_{m_\lambda}}{m_\lambda} \ge
                                        (M_\lambda - m_\lambda + 1) \cdot
                                        \frac{a_1}{m_\lambda} \ge a_1 \cdot
                                        \sum_{k \in T_\lambda} \frac1k.
\end{align*}
Combining this with our first estimate, and using the fact that the $T_\lambda$ cover $\NN$,  we obtain
\[
4 \ge 2 \sum_{\lambda \in \Lambda} \sum_{k \in T_\lambda} (a_{k+1}-a_k) \ge a_1 \sum_{\lambda \in \Lambda} \sum_{k \in T_\lambda} \frac1k \ge a_1 \sum_{k \ge 1} \frac1k,
\]
contradicting the fact that the harmonic series diverges.

\paragraph{Solution 2 (Sanjana Das).}

Assume for the sake of contradiction that no number appears exactly once in the
sequence. For every $i < j$ with $a_i/i = a_j/j$, draw an edge
between $i$ and $j$, so every $i$ has an edge (and being connected by an edge is
a transitive property). Call $i$ \emph{good} if it has an edge with some $j > i$.

First, each $i$ has finite degree -- otherwise \[\frac{a_{x_1}}{x_1} = \frac{a_{x_2}}{x_2} = \dotsb\] for an infinite increasing sequence of positive integers $x_i$, but then the $a_{x_i}$ are unbounded.

Now we use the following process to build a sequence of indices whose $a_i$ we can lower-bound:
\begin{itemize}
    \item Start at $x_1 = 1$, which is good.
    \item If we're currently at good index $x_i$, then let $s_i$ be the largest positive integer such that $x_i$ has an edge to $x_i + s_i$. (This exists because the degrees are finite.)
    \item Let $t_i$ be the smallest positive integer for which $x_i + s_i + t_i$ is good, and let this be $x_{i + 1}$. This exists because if all numbers $k \leq x \leq 2k$ are bad, they must each connect to some number less than $k$ (if two connect to each other, the smaller one is good), but then two connect to the same number, and therefore to each other -- this is the idea we will use later to bound the $t_i$ as well.
\end{itemize}

Then $x_i = 1 + s_1 + t_1 + \dotsb + s_{i - 1} + t_{i - 1}$, and we have
\[a_{x_{i + 1}} > a_{x_i + s_i} = \frac{x_i + s_i}{x_i} a_{x_i} = \frac{1 + (s_1
  + \dotsb + s_{i - 1} + s_i) + (t_1 + \dotsb + t_{i - 1})}{1 + (s_1 + \dotsb +
s_{i - 1}) + (t_1 + \dotsb + t_{i - 1})}a_{x_i}.\] This means
\[c_n := \frac{a_{x_n}}{a_1} > \prod_{i = 1}^{n-1} \frac{1 + (s_1 + \dotsb +
s_{i-1} + s_{i}) + (t_1 + \dotsb + t_{i-1})}{1 + (s_1 + \dotsb + s_{i-1}) + (t_1
+ \dotsb + t_{i-1})}.\]

\begin{lemma*}
  $t_1 + \dotsb + t_n \leq s_1 + \dotsb + s_n$ for each $n$.
\end{lemma*}

\begin{proof}
  Consider $1 \leq i \leq n$. Note that for every $i$, the $t_i - 1$ integers strictly between $x_i + s_i$ and $x_i + s_i + t_i$ are all bad, so each such index $x$ must have an edge to some $y < x$.

  First we claim that if $x \in (x_i + s_i, x_i + s_i + t_i)$, then $x$ cannot
  have an edge to $x_j$ for any $j \leq i$. This is because $x > x_i + s_i
  \geq x_j + s_j$, contradicting the fact that $x_j + s_j$ is the largest
  neighbor of $x_j$.

  This also means $x$ doesn't have an edge to $x_j + s_j$ for any $j \leq i$, since if it did, it would have an edge to $x_j$.

  Second, no two bad values of $x$ can have an edge, since then the smaller
  one is good. This also means no two bad $x$ can have an edge to the same $y$.

  Then each of the $\sum (t_i - 1)$ values in the intervals $(x_i+s_i,
  x_i+s_i+t_i)$ for $1 \leq i \leq n$ must have an edge to an unique $y$ in one
  of the intervals $(x_i, x_i + s_i)$ (not necessarily with the same $i$). Therefore
  \[\sum (t_i - 1) \leq \sum (s_i - 1)\implies \sum t_i \leq \sum s_i.\qedhere\]
\end{proof}

Now note that if $a > b$, then $\frac{a + x}{b + x} = 1 + \frac{a - b}{b + x}$
is decreasing in $x$. This means
\[c_n > \prod_{i = 1}^{n-1} \frac{1 + 2s_1 + \dotsb + 2s_{i-1} + s_{i}}{1 + 2s_1
    + \dotsb + 2s_{i-1}} > \prod_{i = 1}^{n-1} \frac{1 + 2s_1 + \dotsb +
    2s_{i-1} + 2s_{i}}{1 + 2s_1 + \dotsb + 2s_{i-1} + s_{i}},\]
By multiplying both products, we have a telescoping product, which results in
\[c_n^2 \geq 1 + 2s_1 + \dotsb + 2s_n + 2s_{n + 1}.\]
The right hand side is unbounded since the $s_i$ are positive integers, while
$c_n = a_{x_n}/a_1 < 1/a_1$ is bounded, contradiction.

\paragraph{Solution 3 (Gopal Goel).}

Suppose for sake of contradiction that the problem is false. Call an index $i$ a
\emph{pin} if
\[\frac{a_j}{j} = \frac{a_i}{i} \implies j\ge i.\]
\begin{lemma*}
There exists $k$ such that if we have $\tfrac{a_i}{i}=\tfrac{a_j}{j}$ with
$j > i \ge k$, then $j\le 1.1i$.
\end{lemma*}
\begin{proof}
  Note that for any $i$, there are only finitely many $j$ with
  $\frac{a_j}{j}=\frac{a_i}{i}$, otherwise $a_j=\frac{ja_i}{i}$ is unbounded.
  Thus it suffices to find $k$ for which $j\leq 1.1i$ when $j > i\geq k$.

  Suppose no such $k$ exists. Then, take a pair $j_1>i_1$ such that
  $\tfrac{a_{j_1}}{j_1} = \tfrac{a_{i_1}}{i_1}$ and $j_1>1.1i_1$, or
  $a_{j_1}>1.1a_{i_1}$. Now, since $k=j_1$ can't work, there exists a pair
  $j_2>i_2\ge i_1$ such that $\tfrac{a_{j_2}}{j_2} = \tfrac{a_{i_2}}{i_2}$ and
  $j_2>1.1i_2$, or $a_{j_2}>1.1a_{i_2}$. Continuing in this fashion, we see that
  \[a_{j_\ell}>1.1a_{i_\ell}> 1.1a_{j_{\ell-1}},\]
  so we have that $a_{j_\ell}>1.1^\ell a_{i_1}$. Taking $\ell>\log_{1.1}(1/a_1)$ gives the desired contradiction.
\end{proof}

\begin{lemma*}
For $N>k^2$, there are at most $0.8N$ pins in $[\sqrt{N},N)$.
\end{lemma*}
\begin{proof}
By the first lemma, we see that the number of pins in $[\sqrt{N},\tfrac{N}{1.1})$ is at most the number of non-pins in $[\sqrt{N},N)$. Therefore, if the number of pins in $[\sqrt{N},N)$ is $p$, then we have
\[p-N\left(1-\frac{1}{1.1}\right)\le N-p,\]
so $p\le 0.8N$, as desired.
\end{proof}
We say that $i$ is the pin of $j$ if it is the smallest index such that
$\tfrac{a_i}{i}=\tfrac{a_j}{j}$. The pin of $j$ is always a pin.

Given an index $i$, let $f(i)$ denote the largest index less than $i$ that is
not a pin (we leave the function undefined when no such index exists, as we are
only interested in the behavior for large $i$). Then $f$ is weakly increasing
and unbounded by the first lemma. Let $N_0$ be a positive integer such that
$f(\sqrt{N_0}) > k$.

Take any $N>N_0$ such that $N$ is not a pin. Let $b_0=N$, and $b_1$ be the pin
of $b_0$. Recursively define $b_{2i}=f(b_{2i-1})$, and $b_{2i+1}$ to be the pin of $b_{2i}$.

Let $\ell$ be the largest odd index such that $b_\ell\ge\sqrt{N}$. We first show
that $b_\ell\le 100\sqrt{N}$. Since $N > N_0$, we have $b_{\ell+1} > k$. By the
choice of $\ell$ we have $b_{\ell+2}<\sqrt{N}$, so \[b_{\ell+1}<1.1b_{\ell+2}<
1.1\sqrt{N}\] by the first lemma. We see
that all the indices from $b_{\ell+1}+1$ to $b_\ell$ must be pins, so we have
at least $b_\ell-1.1\sqrt{N}$ pins in $[\sqrt{N},b_\ell)$. Combined with the
second lemma, this shows that $b_\ell\le 100\sqrt{N}$.

Now, we have that $a_{b_{2i}}=\tfrac{b_{2i}}{b_{2i+1}}a_{b_{2i+1}}$ and
$a_{b_{2i+1}}> a_{b_{2i+2}}$, so combining gives us
\[\frac{a_{b_0}}{a_{b_\ell}}> \frac{b_0}{b_1}\frac{b_2}{b_3}\dotsm\frac{b_{\ell-1}}{b_\ell}.\]
Note that there are at least
\[(b_1-b_2)+(b_3-b_4)+\dotsb+(b_{\ell-2}-b_{\ell-1})\]
pins in $[\sqrt{N},N)$, so by the second lemma, that sum is at most $0.8N$. Thus,
\begin{align*}
  (b_0-b_1)+(b_2-b_3)+\dotsb+(b_{\ell-1}-b_\ell)
  &=b_0-[(b_1-b_2)+\dotsb + (b_{\ell-2}-b_{\ell-1})]-b_\ell
  \\ &\ge 0.2N-100\sqrt{N}.
\end{align*}
Then
\begin{align*}
\frac{b_0}{b_1}\frac{b_2}{b_3}\dotsm
\frac{b_{\ell-1}}{b_\ell} &\ge 1+\frac{b_0-b_1}{b_1}+\dotsb+\frac{b_{\ell-1}-b_\ell}{b_\ell} \\
&> 1 +\frac{b_0-b_1}{b_0} + \dotsb +
\frac{b_{\ell-1}-b_{\ell}}{b_0} \\
&\ge 1+\frac{0.2N-100\sqrt{N}}{N},
\end{align*}
which is at least $1.01$ if $N_0$ is large enough. Thus, we see that
\[a_N>1.01 a_{b_{\ell}} \geq 1.01 a_{\lfloor\sqrt{N}\rfloor}\]
if $N>N_0$ is not a pin. Since there are arbitrarily large non-pins, this implies that the sequence $(a_n)$ is unbounded, which is the desired contradiction.
\pagebreak

\subsection{TSTST 2021/3, proposed by Merlijn Staps}
\textsl{Available online at \url{https://aops.com/community/p23586679}.}
\begin{mdframed}[style=mdpurplebox,frametitle={Problem statement}]
Find all positive integers $k > 1$ for which there exists a positive integer
$n$ such that $\binom{n}{k}$ is divisible by $n$, and $\binom{n}{m}$ is not
divisible by $n$ for $2\leq m < k$.
\end{mdframed}
Such an $n$ exists for any $k$.

First,  suppose $k$ is prime.  We choose $n=(k-1)!$. For
$m<k$,  it follows from $m! \mid n$ that
\begin{align*}
  (n-1)(n-2) \dotsm (n-m+1) &\equiv (-1)(-2) \dotsm (-m+1) \\
  &\equiv (-1)^{m-1} (m-1)! \\
  &\not\equiv 0 \mod m!.
\end{align*}
We see that $\binom{n}{m}$ is not a multiple of $m$.  For $m=k$, note that
$\binom{n}{k} = \frac{n}{k} \binom{n-1}{k-1}$.  Because $k \nmid n$, we must
have $k \mid \binom{n-1}{k-1}$,  and it follows that $n \mid \binom{n}{k}$.

Now suppose $k$ is composite.
We will choose $n$ to satisfy a number of congruence relations.  For each prime $p \le k$,  let
\[t_p = \nu_p(\opname{lcm}(1,2,\dots,k-1)) = \max(\nu_p(1), \nu_p(2),\dots,\nu_p(k-1))\]
and choose $k_p \in \{1,2,\dots,k-1\}$ as large as possible such that $\nu_p(k_p)=t_p$.  We now require
\begin{align}
n \equiv 0 \mod p^{t_p+1} \qquad \mbox{if $p \nmid k$}; \label{eq:cong1}\\
\nu_p(n - k_p) = t_p+\nu_p(k) \qquad \mbox{if $p \mid k$}. \label{eq:cong2}
\end{align}
for all $p \le k$.  From the Chinese Remainder Theorem,  we know that an $n$ exists that satisfies \eqref{eq:cong1} and \eqref{eq:cong2} (indeed, a sufficient condition for \eqref{eq:cong2} is the congruence $n \equiv k_p + p^{t_p+\nu_p(k)} \mod p^{t_p+\nu_p(k)+1}$).  We show that this $n$ has the required property.

We first will compute $\nu_p(n-i)$ for primes $p<k$ and $1 \le i < k$.
\begin{itemize}
\item If $p \nmid k$, then we have $\nu_p(i), \nu_p(n-i) \le t_p$ and $\nu_p(n) > t_p$, so $\nu_p(n-i) = \nu_p(i)$;
\item If $p \mid k$ and $i \neq k_p$, then we have $\nu_p(i), \nu_p(n-i) \le t_p$ and $\nu_p(n) \ge t_p$, so again $\nu_p(n-i) = \nu_p(i)$;
\item If $p \mid k$ and $i = k_p$, then we have $\nu_p(n-i) = \nu_p(i) + \nu_p(k)$ by \eqref{eq:cong2}.
\end{itemize}
We conclude that $\nu_p(n-i) = \nu_p(i)$ always holds, except when $i = k_p$, when we have $\nu_p(n-i) = \nu_p(i) + \nu_p(k)$ (this formula holds irrespective of whether $p \mid k$ or $p \nmid k$).

We can now show that $\binom{n}{k}$ is divisible by $n$,  which amounts to
showing that $k!$ divides $(n-1)(n-2) \dotsm (n-k+1)$.  Indeed,  for each prime $p \le k$ we have
\begin{align*}
\nu_p\left( (n-1)(n-2) \dots (n-k+1) \right) &=  \nu_p(n-k_p) + \sum_{i<k,  i \neq k_p} \nu_p(n-i) \\
&= \nu_p(k_p) + \nu_p(k) + \sum_{i<k, i \neq k_p} \nu_p(i) \\
&= \sum_{i=1}^k \nu_p(i) = \nu_p(k!),
\end{align*}
so it follows that $(n-1)(n-2) \dotsm (n-k+1)$ is a multiple of $k!$.

Finally,  let $1<m<k$.  We will show that $n$ does \emph{not} divide
$\binom{n}{m}$,  which amounts to showing that $m!$ does not divide $(n-1)(n-2)
\dotsm (n-m+1)$.  First,  suppose that $m$ has a prime divisor $q$ that does not divide $k$.  Then we have
\begin{align*}
  \nu_q\left( (n-1)(n-2) \dots (n-m+1) \right) &= \sum_{i=1}^{m-1} \nu_q(n-i) \\
  &= \sum_{i=1}^{m-1} \nu_q(i) \\
  &= \nu_q((m-1)!) < \nu_q(m!),
\end{align*}
as desired. Therefore,  suppose that $m$ is only divisible by primes that divide $k$.  If there is such a prime $p$ with $\nu_p(m) > \nu_p(k)$,  then it follows that
\begin{align*}
  \nu_p\left( (n-1)(n-2) \dots (n-m+1) \right)
  &= \nu_p(k) + \sum_{i=1}^{m-1} \nu_p(i) \\
  &< \nu_p(m) + \sum_{i=1}^{m-1}
  \nu_p(i) \\
  &= \nu_p(m!),
\end{align*}
so $m!$ cannot divide $(n-1)(n-2) \dots (n-m+1)$.  On the other hand,  suppose
that $\nu_p(m) \le \nu_p(k)$ for all $p \mid k$, which would mean that $m \mid
k$ and hence $m \le \frac{k}{2}$.  Consider a prime $p$ dividing $m$.  We have
$k_p \ge \frac{k}{2}$,  because otherwise $2k_p$ could have been used instead of
$k_p$. It follows that $m \le \frac{k}{2} \le k_p$. Therefore, we obtain
\begin{align*}
  \nu_p\left( (n-1)(n-2) \dots (n-m+1) \right) &= \sum_{i=1}^{m-1} \nu_p(n-i) \\
  &= \sum_{i=1}^{m-1} \nu_p(i)  \\
  &= \nu_p((m-1)!) < \nu_p(m!),
\end{align*}
showing that $(n-1)(n-2) \dotsm (n-m+1)$ is not divisible by $m!$.
This shows that $\binom{n}{m}$ is not divisible by $n$ for $m<k$,
and hence $n$ does have the required property.
\pagebreak

\section{Solutions to Day 2}
\subsection{TSTST 2021/4, proposed by Holden Mui}
\textsl{Available online at \url{https://aops.com/community/p23864177}.}
\begin{mdframed}[style=mdpurplebox,frametitle={Problem statement}]
Let $a$ and $b$ be positive integers.
Suppose that there are infinitely many pairs of positive integers $(m, n)$
for which $m^2+an+b$ and $n^2+am+b$ are both perfect squares.
Prove that $a$ divides $2b$.
\end{mdframed}
Treating $a$ and $b$ as fixed,
we are given that there are infinitely many quadrpules $(m,n,r,s)$
which satisfy the system
\begin{gather*}
  m^2+an+b=(m+r)^2 \\
  n^2+am+b=(n+s)^2
\end{gather*}
We say that $(r,s)$ is \emph{exceptional}
if there exists infinitely many $(m,n)$ that satisfy.

\begin{claim*}
  If $(r,s)$ is exceptional, then either
  \begin{itemize}
    \ii $0 < r < a/2$, and $0 < s < \frac14 a^2$; or
    \ii $0 < s < a/2$, and $0 < r < \frac14 a^2$; or
    \ii $r^2 + s^2 \le 2b$.
  \end{itemize}
  In particular, finitely many pairs $(r,s)$ can be exceptional.
\end{claim*}

\begin{proof}
  Sum the two equations to get:
  \[ r^2+s^2-2b = (a-2r)m + (a-2s)n. \qquad (\dagger) \]
  If $0 < r < a/2$, then the idea is to use the bound
  $an+b \ge 2m+1$ to get $m \le \frac{an+b-1}{2}$.
  Consequently,
  \[ (n+s)^2 = n^2+am+b \le n^2 + a \cdot \frac{an+b-1}{2} + b \]
  For this to hold for infinitely many integers $n$,
  we need $2s \le \frac{a^2}{2}$, by comparing coefficients.

  A similar case occurs when $0 < s < a/2$.

  If $\min(r,s) > a/2$, then $(\dagger)$ forces $r^2+s^2 \le 2b$,
  giving the last case.
\end{proof}

Hence, there exists some particular pair $(r,s)$
for which there are infinitely many solutions
$(m,n)$. Simplifying the system gives
\begin{gather*}
  an = 2rm + r^2-b \\
  2sn = am + b-s^2
\end{gather*}
Since the system is linear,
for there to be infinitely many solutions $(m, n)$
the system must be dependent.
This gives \[\frac{a}{2s}=\frac{2r}{a}=\frac{r^2-b}{b-s^2}\]
so $a = 2\sqrt{rs}$ and $b = \frac{s^2\sqrt{r}+r^2\sqrt{s}}{\sqrt{r}+\sqrt{s}}$.
Since $rs$ must be square, we can reparametrize as $r=kx^2$, $s=ky^2$, and $\gcd(x, y)=1$.
This gives
\begin{align*}
    a &= 2kxy \\
    b &= k^2xy(x^2-xy+y^2).
\end{align*}
Thus, $a \mid 2b$, as desired.
\pagebreak

\subsection{TSTST 2021/5, proposed by Vincent Huang}
\textsl{Available online at \url{https://aops.com/community/p23864182}.}
\begin{mdframed}[style=mdpurplebox,frametitle={Problem statement}]
Let $T$ be a tree on $n$ vertices with exactly $k$ leaves.
Suppose that there exists a subset of
at least $\frac{n+k-1}{2}$ vertices of $T$,
no two of which are adjacent.
Show that the longest path in $T$ contains
an even number of edges.
\end{mdframed}
The longest path in $T$ must go between two leaves. The solutions presented here
will solve the problem by showing that in the unique $2$-coloring of $T$, all
leaves are the same color.

\paragraph{Solution 1 (Ankan Bhattacharya, Jeffery Li).}
\begin{lemma*}
  If $S$ is an independent set of $T$, then
  \[\sum_{v\in S}\deg(v)\leq n-1.\]
  Equality holds if and only if $S$ is one of the two components of the unique
  $2$-coloring of $T$.
\end{lemma*}

\begin{proof}
  Each edge of $T$ is incident to at most one vertex of $S$, so we obtain the
  inequality by counting how many vertices of $S$ each edge is incident to. For
  equality to hold, each edge is incident to exactly one vertex of $S$, which
  implies the $2$-coloring.
\end{proof}

We are given that there exists an independent set of at least $\frac{n+k-1}{2}$
vertices. By greedily choosing vertices of smallest degree, the sum of the
degrees of these vertices is at least
\[k+2\cdot\frac{n-k-1}{2}=n-1.\]
Thus equality holds everywhere, which implies that the independent set contains
every leaf and is one of the components of the $2$-coloring.

\paragraph{Solution 2 (Andrew Gu).}

\begin{lemma*}
  The vertices of $T$ can be partitioned into $k-1$ paths (i.e. the induced
  subgraph on each set of vertices is a path) such that all edges of $T$
  which are not part of a path are incident to an endpoint of a path.
\end{lemma*}

\begin{proof}
  Repeatedly trim the tree by taking a leaf and removing the longest path
  containing that leaf such that the remaining graph is still a tree.
\end{proof}

Now given a path of $a$ vertices, at most $\frac{a+1}{2}$ of those vertices can
be in an independent set of $T$. By the lemma, $T$ can be partitioned into $k-1$
paths of $a_1, \dots, a_{k-1}$ vertices, so the maximum size of an independent
set of $T$ is
\[\sum \frac{a_i+1}{2}=\frac{n+k-1}{2}.\]
For equality to hold, each path in the partition must have an odd number of
vertices, and has a unique $2$-coloring in red and blue where the endpoints are
red. The unique independent set of $T$ of size $\frac{n+k-1}{2}$ is then the set of red
vertices. By the lemma, the edges of $T$ which are not part of a path connect an
endpoint of a path (which is colored red) to another vertex (which must be blue,
because the red vertices are independent). Thus the coloring of the paths
extends to the unique $2$-coloring of $T$. The leaves of $T$ are endpoints of
paths, so they are all red.
\pagebreak

\subsection{TSTST 2021/6, proposed by Nikolai Beluhov}
\textsl{Available online at \url{https://aops.com/community/p23864189}.}
\begin{mdframed}[style=mdpurplebox,frametitle={Problem statement}]
Triangles $ABC$ and $DEF$ share circumcircle $\Omega$ and incircle $\omega$
so that points $A$, $F$, $B$, $D$, $C$, and $E$ occur in this order along $\Omega$.
Let $\Delta_A$ be the triangle formed by lines $AB$, $AC$, and $EF$,
and define triangles $\Delta_B$, $\Delta_C, \dots, \Delta_F$ similarly.
Furthermore, let $\Omega_A$ and $\omega_A$ be the circumcircle and incircle
of triangle $\Delta_A$, respectively, and define circles
$\Omega_B$, $\omega_B, \dots, \Omega_F$, $\omega_F$ similarly.
\begin{enumerate}[(a)]
  \item Prove that the two common external tangents to circles $\Omega_A$ and $\Omega_D$
    and the two common external tangents to circles $\omega_A$ and $\omega_D$
    are either concurrent or pairwise parallel.

  \item Suppose that these four lines meet at point $T_A$,
    and define points $T_B$ and $T_C$ similarly.
    Prove that points $T_A$, $T_B$, and $T_C$ are collinear.
\end{enumerate}
\end{mdframed}
\begin{center}
\begin{asy}
import geometry;

linemargin = 0;

real R = 120;
pair O = (0, 0);

pair A = rotate(35, O)*(R, 0);
pair B = rotate(210, O)*(R, 0);
pair C = rotate(330, O)*(R, 0);

circle omega = incircle(A, B, C);
circle k = circle((point) O, R);

pair D = rotate(255, O)*(R, 0);
line[] ts = tangents(omega, (point) D);
pair E = 2*(projection(ts[0])*O) - D;
pair F = 2*(projection(ts[1])*O) - D;

pair K = extension(A, B, E, F);
pair L = extension(A, B, F, D);
pair M = extension(B, C, F, D);
pair N = extension(B, C, D, E);
pair P = extension(C, A, D, E);
pair Q = extension(C, A, E, F);

triangle dA = triangle(Q, A, K);
triangle dB = triangle(L, B, M);
triangle dC = triangle(N, C, P);
triangle dD = triangle(M, D, N);
triangle dE = triangle(P, E, Q);
triangle dF = triangle(K, F, L);

circle oA = incircle(dA);
circle oB = incircle(dB);
circle oC = incircle(dC);
circle oD = incircle(dD);
circle oE = incircle(dE);
circle oF = incircle(dF);

circle kA = circumcircle(dA);
circle kB = circumcircle(dB);
circle kC = circumcircle(dC);
circle kD = circumcircle(dD);
circle kE = circumcircle(dE);
circle kF = circumcircle(dF);

point similitude(circle a, circle b) {
  return extension(a.C, b.C, a.C + (0, a.r), b.C + (0, b.r));
}

point TA = similitude(oA, oD);
point TB = similitude(oB, oE);
point TC = similitude(oC, oF);



draw(A--B--C..cycle);
draw(D--E--F..cycle);
draw(omega);
draw(k);

pen otp = red+opacity(0.5);

draw(TA--projection(tangents(oD, TA)[0])*oD.C, otp);
draw(TA--projection(tangents(oD, TA)[1])*oD.C, otp);
draw(TB--projection(tangents(oB, TB)[0])*oB.C, otp);
draw(TB--projection(tangents(oB, TB)[1])*oB.C, otp);
draw(TC--projection(tangents(oF, TC)[0])*oF.C, otp);
draw(TC--projection(tangents(oF, TC)[1])*oF.C, otp);
//draw(tangents(oA, TA), otp);
//draw(tangents(oB, TB), otp);
//draw(tangents(oC, TC), otp);

pen ktp = lightolive+opacity(0.5);

draw(TA--projection(tangents(kD, TA)[0])*kD.C, ktp);
draw(TA--projection(tangents(kD, TA)[1])*kD.C, ktp);
draw(TB--projection(tangents(kB, TB)[0])*kB.C, ktp);
draw(TB--projection(tangents(kB, TB)[1])*kB.C, ktp);
draw(TC--projection(tangents(kF, TC)[0])*kF.C, ktp);
draw(TC--projection(tangents(kF, TC)[1])*kF.C, ktp);
//draw(tangents(kA, TA), ktp);
//draw(tangents(kB, TB), ktp);
//draw(tangents(kC, TC), ktp);

pen op = blue;

draw(oA, op);
draw(oB, op);
draw(oC, op);
draw(oD, op);
draw(oE, op);
draw(oF, op);

pen kp = heavygreen;

draw(kA, kp);
draw(kB, kp);
draw(kC, kp);
draw(kD, kp);
draw(kE, kp);
draw(kF, kp);

draw(line(TA, TB), 1.0 + heavymagenta);



dot(A);
dot(B);
dot(C);
dot(D);
dot(E);
dot(F);
dot(TA);
dot(TB);
dot(TC);



label("$A$", A, dir(A));
label("$B$", B, dir(B));
label("$C$", C, dir(C));
label("$D$", D, dir(D));
label("$E$", E, dir(E));
label("$F$", F, dir(F));
label("$T_A$", TA, dir((1, 0)));
label("$T_B$", TB, dir((1, 0)));
label("$T_C$", TC, dir((1, 0)));

\end{asy}
\end{center}

Let $I$ and $r$ be the center and radius of $\omega$, and let $O$ and $R$ be the
center and radius of $\Omega$. Let $O_A$ and $I_A$ be the circumcenter and
incenter of triangle $\Delta_A$, and define $O_B$, $I_B, \dots, I_F$
similarly. Let $\omega$ touch $EF$ at $A_1$, and define $B_1$, $C_1, \dots,
F_1$ similarly.

\paragraph{Part (a).} All solutions to part (a) will prove the stronger claim
that
\[(\Omega_A\cup \omega_A)\sim (\Omega_D\cup \omega_D).\]
The four lines will concur at the homothetic center of these figures (possibly
at infinity).

\subparagraph{Solution 1 (author)} Let the second tangent to $\omega$ parallel to $EF$ meet lines $AB$ and $AC$ at $P$ and $Q$, respectively, and let the second tangent to $\omega$ parallel to $BC$ meet lines $DE$ and $DF$ at $R$ and $S$, respectively. Furthermore, let $\omega$ touch $PQ$ and $RS$ at $U$ and $V$, respectively.

Let $h$ be inversion with respect to $\omega$. Then $h$ maps $A$, $B$, and $C$ onto the midpoints of the sides of triangle $D_1E_1F_1$. So $h$ maps $k$ onto the Euler circle of triangle $D_1E_1F_1$.

Similarly, $h$ maps $k$ onto the Euler circle of triangle $A_1B_1C_1$.
Therefore, triangles $A_1B_1C_1$ and $D_1E_1F_1$ share a common nine-point
circle $\gamma$. Let $K$ be its center; its radius equals $\frac{1}{2}r$.

Let $H$ be the reflection of $I$ in $K$. Then $H$ is the common orthocenter of triangles $A_1B_1C_1$ and $D_1E_1F_1$.

Let $\gamma_U$ of center $K_U$ and radius $\frac{1}{2}r$ be the Euler circle of
triangle $UE_1F_1$, and let $\gamma_V$ of center $K_V$ and radius $\frac{1}{2}r$ be
the Euler circle of triangle $VB_1C_1$.

Let $H_U$ be the orthocenter of triangle $UE_1F_1$. Since quadrilateral
$D_1E_1F_1U$ is cyclic, vectors $\overrightarrow{HH_U}$ and
$\overrightarrow{D_1U}$ are equal. Consequently,
$\overrightarrow{KK_U}=\frac{1}{2}\overrightarrow{D_1U}$. Similarly,
$\overrightarrow{KK_V}=\frac{1}{2}\overrightarrow{A_1V}$.

Since both of $A_1U$ and $D_1V$ are diameters in $\omega$, vectors
$\overrightarrow{D_1U}$ and $\overrightarrow{A_1V}$ are equal. Therefore, $K_U$
and $K_V$ coincide, and so do $\gamma_U$ and $\gamma_V$.

As above, $h$ maps $\gamma_U$ onto the circumcircle of triangle $APQ$ and
$\gamma_V$ onto the circumcircle of triangle $DRS$. Therefore, triangles $APQ$
and $DRS$ are inscribed inside the same circle $\Omega_{AD}$.

Since $EF$ and $PQ$ are parallel, triangles $\Delta_A$ and $APQ$ are homothetic,
and so are figures $\Omega_A \cup \omega_A$ and $\Omega_{AD}\cup \omega$.
Consequently, we have
\[(\Omega_A\cup \omega_A)\sim (\Omega_{AD}\cup \omega)\sim (\Omega_D\cup
\omega_D),\]
which solves part (a).

%If figures $\omega_A \cup k_A$ and $\omega_D \cup k_D$ are noncongruent, then the two common external tangents to $\omega_A$ and $\omega_D$ and the two common external tangents to $k_A$ and $k_D$ meet at their homothetic center. Otherwise, the two common external tangents to $\omega_A$ and $\omega_D$ and the two common external tangents to $k_A$ and $k_D$ are pairwise parallel.

\subparagraph{Solution 2 (Michael Ren)} As in the previous solution, let the
second tangent to $\omega$ parallel to $EF$ meet $AB$ and $AC$ at $P$ and $Q$,
respectively. Let $(APQ)$ meet $\Omega$ again at $D'$, so that $D'$ is the
Miquel point of $\{AB, AC, BC, PQ\}$. Since the quadrilateral formed by these
lines has incircle $\omega$, it is classical that $D'I$ bisects $\angle PD'C$
and $BD'Q$ (e.g. by DDIT).

Let $\ell$ be the tangent to $\Omega$ at $D'$ and $D'I$ meet $\Omega$ again at
$M$. We have
\[
  \measuredangle(\ell, D'B) = \measuredangle D'CB = \measuredangle D'QP =
  \measuredangle(D'Q, EF).
\]
Therefore $D'I$ also bisects the angle between $\ell$ and the line parallel to
$EF$ through $D'$. This means that $M$ is one of the arc midpoints of $EF$.
Additionally, $D'$ lies on arc $BC$ not containing $A$, so $D'=D$.

Similarly, letting the second tangent to $\omega$ parallel to $BC$ meet $DE$ and
$DF$ again at $R$ and $S$, we have $ADRS$ cyclic.

\begin{lemma*}
  There exists a circle $\Omega_{AD}$ tangent to $\Omega_A$ and $\Omega_D$ at
  $A$ and $D$, respectively.
\end{lemma*}
\begin{proof}
  (This step is due to Ankan Bhattacharya.) It is equivalent to have
  $\measuredangle OAO_A = \measuredangle O_DDO$. Taking isogonals with respect
  to the shared angle of $\triangle ABC$ and $\Delta_A$, we see that
  \[\measuredangle OAO_A = \measuredangle(\perp EF, \perp BC) = \measuredangle
  (EF, BC).\]
  (Here, $\perp EF$ means the direction perpendicular to $EF$.) By symmetry,
  this is equal to $\measuredangle O_DDO$.
\end{proof}

The circle $(ADPQ)$ passes through $A$ and $D$, and is tangent to $\Omega_A$ by
homothety. Therefore it coincides with $\Omega_{AD}$, as does $(ADRS)$. Like the
previous solution, we finish with
\[(\Omega_A\cup \omega_A)\sim (\Omega_{AD}\cup \omega)\sim (\Omega_D\cup
\omega_D).\]

\subparagraph{Solution 3 (Andrew Gu)}
Construct triangles homothetic to $\Delta_A$ and $\Delta_D$ (with positive ratio) which
have the same circumcircle; it suffices to show that these copies have the same
incircle as well. Let $O$ be the center of this common circumcircle, which we
take to be the origin, and $M_{XY}$ denote the point on the circle such that the
tangent at that point is parallel to line $XY$ (from the two possible choices,
we make the choice that corresponds to the arc midpoint on $\Omega$, e.g. $M_{AB}$
should correspond to the arc midpoint on the internal angle bisector of $ACB$).
The left diagram below shows the original triangles $ABC$ and $DEF$, while the
right diagram shows the triangles homothetic to $\Delta_A$ and $\Delta_D$.

\begin{center}
  \begin{asy}
    size(6cm);
    pair O = origin;
    pair Mbc = dir(-90);
    pair Mca = dir(40);
    pair Mab = dir(160);
    pair A = -Mca*Mab/Mbc;
    pair B = -Mab*Mbc/Mca;
    pair C = -Mbc*Mca/Mab;
    pair I = Mbc+Mca+Mab;
    pair Mef = dir(75);
    pair D = 2*foot(O, I, Mef)-Mef;
    pair X = midpoint(D--I);
    pair Y = X+5*dir(90)*(X-D);
    pair Mfd = intersectionpoints(X--Y, unitcircle)[0];
    pair Mde = 2*foot(O, X, Mfd)-Mfd;
    pair E = -Mde*Mef/Mfd;
    pair F = -Mef*Mfd/Mde;

    draw(unitcircle);
    draw(A--B--C--cycle, red);
    draw(D--E--F--cycle, blue);
    draw(incircle(A, B, C));

    string[] names = {"$A$", "$B$", "$C$", "$M_{BC}$", "$M_{CA}$", "$M_{AB}$", "$I$", "$D$", "$E$", "$F$", "$M_{FD}$", "$M_{DE}$", "$M_{EF}$"};
    pair[] pts = {A, B, C, Mbc, Mca, Mab, I, D, E, F, Mfd, Mde, Mef};
    pair[] labels = {A, B, C, Mbc, Mca, Mab, I, D, E, F, Mfd, Mde, Mef};
    for(int i=0; i<names.length; ++i){
      dot(names[i], pts[i], dir(labels[i]));
    }
  \end{asy}
  \begin{asy}
    size(6cm);
    pair O = origin;
    pair Mbc = dir(-90);
    pair Mca = dir(40);
    pair Mab = dir(160);
    pair Mef = dir(75);
    pair oldI = Mbc+Mca+Mab;
    pair I = -Mef+Mca+Mab;
    pair oldD = 2*foot(O, oldI, Mef)-Mef;
    pair X = midpoint(oldD--oldI);
    pair Y = X+5*dir(90)*(X-oldD);
    pair Mfd = intersectionpoints(X--Y, unitcircle)[0];
    pair Mde = 2*foot(O, X, Mfd)-Mfd;
    pair A = Mca*Mab/Mef;
    pair B = Mab*Mef/Mca;
    pair C = Mef*Mca/Mab;
    pair D = Mfd*Mde/Mbc;
    pair E = Mde*Mbc/Mfd;
    pair F = Mbc*Mfd/Mde;

    draw(unitcircle);
    draw(C--A--B, red);
    draw(B--C, blue);
    draw(F--D--E, blue);
    draw(E--F, red);
    draw(incircle(A, B, C));

    string[] names = {" ", " ", " ", "$M_{BC}$", "$M_{CA}$", "$M_{AB}$", "$I$", " ", " ", " ", "$M_{FD}$", "$M_{DE}$", "$M_{EF}$"};
    pair[] pts = {A, B, C, Mbc, Mca, Mab, I, D, E, F, Mfd, Mde, Mef};
    pair[] labels = {A, B, C, Mbc, Mca, Mab, I, D, E, F, Mfd, Mde, Mef};
    for(int i=0; i<names.length; ++i){
      dot(names[i], pts[i], dir(labels[i]));
    }
  \end{asy}
\end{center}

Using the fact that the incenter is the orthocenter of the arc midpoints, the
incenter of $\Delta_A$ in this reference frame is $M_{AB}+M_{AC}-M_{EF}$ and the
incenter of $\Delta_D$ in this reference frame is $M_{DE}+M_{DF}-M_{BC}$. Since
$ABC$ and $DEF$ share a common incenter, we have
\[M_{AB}+M_{BC}+M_{CA}=M_{DE}+M_{EF}+M_{FD}.\]
Thus the copies of $\Delta_A$ and $\Delta_D$ have the same incenter, and
therefore the same incircle as well (Euler's theorem determines the inradius).

\paragraph{Part (b).}
We present several solutions for this part of the problem. Solutions 3 and 4
require solving part (a) first, while the others do not. Solutions 1, 4, and 5
define $T_A$ solely as the exsimilicenter of $\omega_A$ and $\omega_D$, whereas
solutions 2 and 3 define $T_A$ solely as the exsimilicenter of $\Omega_A$ and
$\Omega_D$.

\subparagraph{Solution 1 (author)}

By Monge's theorem applied to $\omega$, $\omega_A$, and $\omega_D$,
points $A$, $D$, and $T_A$ are collinear. Therefore, $T_A= AD\cap I_AI_D$.

Let $p$ be pole-and-polar correspondence with respect to $\omega$. Then $p$ maps
$A$ onto line $E_1F_1$ and $D$ onto line $B_1C_1$. Consequently, $p$ maps line
$AD$ onto $X_A=B_1C_1\cap E_1F_1$.

Furthermore, $p$ maps the line that bisects the angle formed by lines $AB$ and
$EF$ and does not contain $I$ onto the midpoint of segment $A_1F_1$. Similarly,
$p$ maps the line that bisects the angle formed by lines $AC$ and $EF$ and does
not contain $I$ onto the midpoint of segment $A_1E_1$. So $p$ maps $I_A$ onto
the midline of triangle $A_1E_1F_1$ opposite $A_1$. Similarly, $p$ maps $I_D$
onto the midline of triangle $D_1B_1C_1$ opposite $D_1$. Consequently, $p$ maps
line $I_AI_D$ onto the intersection point $Y_A$ of this pair of midlines, and
$p$ maps $T_A$ onto line $X_AY_A$.

As in the solution to part (a), let $H$ be the common orthocenter of triangles
$A_1B_1C_1$ and $D_1E_1F_1$. Let $H_A$ be the foot of the altitude from $A_1$ in
triangle $A_1B_1C_1$ and let $H_D$ be the foot of the altitude from $D_1$ in
triangle $D_1E_1F_1$. Furthermore, let $L_A=HA_1\cap E_1F_1$ and $L_D=HD_1\cap
B_1C_1$.

Since the reflection of $H$ in line $B_1C_1$ lies on $\omega$, $A_1H \cdot HH_A$ equals half the power of $H$ with respect to $\omega$. Similarly, $D_1H \cdot HH_D$ equals half the power of $H$ with respect to $\omega$.

Then $A_1H \cdot HH_A = D_1H \cdot HH_D$ and $A_1HH_D\sim D_1HH_A$. Since
$\angle HH_DL_A = 90^\circ = \angle HH_AL_D$, figures $A_1HH_DL_A$ and
$D_1HH_AL_D$ are similar as well. Consequently,
\[
  \frac{HL_A}{L_AA_1}=\frac{HL_D}{L_DD_1}=s
\]
as a signed ratio.
%$L_A$ divides segment $HA_1$ in
%the same signed ratio $s$ as $L_D$ divides segment $HD_1$.

Let the line through $A_1$ parallel to $E_1F_1$ and the line through $D_1$
parallel to $B_1C_1$ meet at $Z_A$. Then $HX_A/X_AZ_A=s$
%$X_A$ divides segment $HZ_A$ in signed ratio $s$
and $Y_A$ is the midpoint of segment $X_AZ_A$. Therefore, $H$ lies on
line $X_AY_A$. This means that $T_A$ lies on the polar of $H$ with respect to
$\omega$, and by symmetry so do $T_B$ and $T_C$.


\subparagraph{Solution 2 (author)}

As in the first solution to part (a), let
$h$ be inversion with respect to $\omega$, let $\gamma$ of center $K$ be the common
Euler circle of triangles $A_1B_1C_1$ and $D_1E_1F_1$, and let $H$ be their
common orthocenter.

Again as in the solution to part (a), $h$ maps $\Omega_A$ onto the nine-point
circle $\gamma_A$ of triangle $A_1E_1F_1$ and $\Omega_D$ onto the nine-point
circle $\gamma_D$ of triangle $D_1B_1C_1$.

Let $K_A$ and $K_D$ be the centers of $\gamma_A$ and $\gamma_D$, respectively,
and let $\ell_A$ be the perpendicular bisector of segment $K_AK_D$. Since
$\gamma_A$ and $\gamma_D$ are congruent (both of them are of radius
$\frac{1}{2}r$), they are reflections of each other across $\ell_A$.

Inversion $h$ maps the two common external tangents of $\Omega_A$ and $\Omega_D$
onto the two circles $\alpha$ and $\beta$ through $I$ that are tangent to both
of $\gamma_A$ and $\gamma_D$ in the same way. (That is, either internally to both or
externally to both.) Consequently, $\alpha$ and $\beta$ are reflections of each
other in $\ell_A$ and so their second point of intersection $S_A$, which $h$
maps $T_A$ onto, is the reflection of $I$ in $\ell_A$.

Define $\ell_B$, $\ell_C$, $S_B$, and $S_C$ similarly. Then $S_B$ is the reflection of $I$ in $\ell_B$ and $S_C$ is the reflection of $I$ in $\ell_C$.

As in the solution to part (a),
$\overrightarrow{KK_A}=\frac{1}{2}\overrightarrow{D_1A_1}$ and
$\overrightarrow{KK_D}=\frac{1}{2}\overrightarrow{A_1D_1}$. Consequently, $K$ is
the midpoint of segment $K_AK_D$ and so $K$ lies on $\ell_A$. Similarly, $K$
lies on $\ell_B$ and $\ell_C$.

Therefore, all four points $I$, $S_A$, $S_B$, and $S_C$ lie on the circle of center $K$ that contains $I$. (This is also the circle on diameter $IH$.) Since points $S_A$, $S_B$, and $S_C$ are concyclic with $I$, their images $T_A$, $T_B$, and $T_C$ under $h$ are collinear, and the solution is complete.

\subparagraph{Solution 3 (Ankan Bhattacharya)}

From either of the first two solutions to part (a), we know that there is a
circle $\Omega_{AD}$ passing through $A$ and $D$ which is (internally) tangent
to $\Omega_A$ and $\Omega_D$. By Monge's theorem applied to $\Omega_A,
\Omega_D$, and $\Omega_{AD}$, it follows that $A, D$, and $T_A$ are collinear.

The inversion at $T_A$ swapping $\Omega_A$ with $\Omega_D$ also swaps $A$ with
$D$ because $T_A$ lies on $AD$ and $A$ is not homologous to $D$. Let $\Omega_A$
and $\Omega_D$ meet $\Omega$ again at $L_A$ and $L_D$. Since $ADL_AL_D$ is
cyclic, the same inversion at $T_A$ also swaps $L_A$ and $L_D$, so $T_A=AD\cap
L_AL_D$.

By \href{https://aops.com/community/c6h598547p3551881}{Taiwan TST
2014}, $L_A$ and $L_D$ are the tangency points of the $A$-mixtilinear and
$D$-mixtilinear incircles, respectively, with $\Omega$. Therefore $AL_A\cap
DL_D$ is the exsimilicenter $X$ of $\Omega$ and $\omega$. Then $T_A, T_B,$ and
$T_C$ all lie on the polar of $X$ with respect to $\Omega$.

\subparagraph{Solution 4 (Andrew Gu)}

We show that $T_A$ lies on the radical axis of the point circle at $I$ and
$\Omega$, which will solve the problem.  Let $I_A$ and $I_D$ be the centers of
$\omega_A$ and $\omega_D$ respectively. By the Monge's theorem applied to
$\omega$, $\omega_A$, and $\omega_D$, points $A$, $D$, and $T_A$ are collinear.
Additionally, these other triples are collinear: $(A, I_A, I), (D, I_D, I),
(I_A, I_D, T)$.  By Menelaus's theorem, we have

\[\frac{T_AD}{T_AA}=\frac{I_AI}{I_AA}\cdot\frac{I_DD}{I_DI}.\]
If $s$ is the length of the side opposite $A$ in $\Delta_A$, then we compute
\begin{align*}
  \frac{I_AI}{I_AA} &=\frac{s/\cos(A/2)}{r_A/\sin(A/2)} \\
                    &=\frac{2R_A\sin(A)\sin(A/2)}{\cos(A/2)} \\
                    &=\frac{4R_A\sin^2(A/2)}{r_A} \\
                    &=\frac{4R_Ar^2}{r_AAI^2}.
\end{align*}
From part (a), we know that $\frac{R_A}{r_A}=\frac{R_D}{r_D}$.
Therefore, doing a similar calculation for $\frac{I_DD}{I_DI}$, we get
\begin{align*}
  \frac{T_AD}{T_AA} &=\frac{I_AI}{I_AA}\cdot\frac{I_DD}{I_DI} \\
                    &=\frac{4R_Ar^2}{r_AAI^2}\cdot \frac{r_DDI^2}{4R_Dr^2} \\
                    &=\frac{DI^2}{AI^2}.
\end{align*}
Thus $T_A$ is the point where the tangent to $(AID)$ at $I$ meets $AD$ and
$T_AI^2=T_AA\cdot T_AD$. This shows what we claimed at the start.

\subparagraph{Solution 5 (Ankit Bisain)}

As in the previous solution, it suffices to show that $\frac{I_AI}{AI_A}\cdot
\frac{DI_D}{I_DI} = \frac{DI^2}{AI^2}$. Let $AI$ and $DI$ meet $\Omega$ again at
$M$ and $N$, respectively. Let $\ell$ be the line parallel to $BC$ and tangent
to $\omega$ but different from $BC$. Then
\[
  \frac{DI_D}{I_DI}=\frac{d(D, BC)}{d(BC, \ell)} = \frac{DB\cdot DC/2R}{2r} =
  \frac{MI^2-MD^2}{4Rr}.
\]
%(One way to prove $DB\cdot DC=MI^2-MD^2$ is to let $CD$ meet $(BIC)$ again at
%$B'$, show that $DB=DB'$, and apply power of a point.)
Since $IDM\sim IAN$, we have
\[\frac{DI_D}{I_DI}\cdot \frac{I_AI}{AI_A} =
\frac{MI^2-MD^2}{NI^2-NA^2}=\frac{DI^2}{AI^2},\]
as desired.

%\begin{remark}
%The proof of part (a) also shows that lines $I_AO_A$ and $I_DO_D$ are parallel, and similarly for lines $I_BO_B$ and $I_EO_E$ and lines $I_CO_C$ and $I_FO_F$.
%\end{remark}
%
%\bigskip
%
%\begin{remark}
%All proofs of part (b) also show that lines $IO$ and $T_AT_BT_C$ are perpendicular. (Since $h$ maps $k$ onto $e$, the center $K$ of $e$ lies on line $IO$. Therefore, the reflection $H$ of $I$ in $K$ lies on line $IO$ as well.)
%\end{remark}

\begin{remark*}[Author comments on generalization of part (b) with a
  circumscribed hexagram]
Let triangles $ABC$ and $DEF$ be circumscribed about the same circle $\omega$ so that they form a hexagram. However, we do not require anymore that they are inscribed in the same circle.

Define circles $\Omega_A$, $\omega_A, \dots, \omega_F$ as in the problem.
Let $T^\text{Circ}_A$ be the intersection point of the two common external
tangents to circles $\Omega_A$ and $\Omega_D$, and define points $T^\text{Circ}_B$
and $T^\text{Circ}_C$ similarly. Also let $T^\text{In}_A$ be the intersection
point of the two common external tangents to circles $\omega_A$ and $\omega_D$, and define
points $T^\text{In}_B$ and $T^\text{In}_C$ similarly.

Then points $T^\text{Circ}_A$, $T^\text{Circ}_B$, and $T^\text{Circ}_C$ are collinear
and points $T^\text{In}_A$, $T^\text{In}_B$, and $T^\text{In}_C$ are also
collinear.

The second solution to part (b) of the problem works also for the circumcircles
part of the generalisation. To see that segments $K_AK_D$, $K_BK_E$, and
$K_CK_F$ still have a common midpoint, let $M$ be the centroid of points $A$,
$B$, $C$, $D$, $E$, and $F$. Then the midpoint of segment $K_AK_D$ divides
segment $OM$ externally in ratio $3 : 1$, and so do the other two midpoints as
well.

For the incircles part of the generalisation, we start out as in the first
solution to part (b) of the problem, and eventually we reduce everything to the
following:

\emph{Let points $A_1$, $B_1$, $C_1$, $D_1$, $E_1$, and $F_1$ lie on circle $\omega$. Let lines $B_1C_1$ and $E_1F_1$ meet at point $X_A$, let the line through $A_1$ parallel to $B_1C_1$ and the line through $D_1$ parallel to $E_1F_1$ meet at point $Z_A$, and define points $X_B$, $Z_B$, $X_C$, and $Z_C$ similarly. Then lines $X_AZ_A$, $X_BZ_B$, and $X_CZ_C$ are concurrent.}

Take $\omega$ as the unit circle and assign complex numbers $u$, $v$, $w$, $x$, $y$, and $z$ to points $A_1$, $F_1$, $B_1$, $D_1$, $C_1$, and $E_1$, respectively, so that when we permute $u$, $v$, $w$, $x$, $y$, and $z$ cyclically the configuration remains unchanged. Then by standard complex bash formulas we obtain that each two out of our three lines meet at $\varphi/\psi$, where \[\varphi = \sum_\text{Cyc} u^2vw(wx - wy + xy)(y - z)\] and \[\psi = {} - u^2w^2y^2 - v^2x^2z^2 - 4uvwxyz + \sum_\text{Cyc} u^2(vwxy - vwxz + vwyz - vxyz + wxyz).\]

(But the calculations were too difficult for me to do by hand, so I used SymPy.)
\end{remark*}

\bigskip

\begin{remark*}[Author comments on generalization of part (b) with an
  inscribed hexagram]
Let triangles $ABC$ and $DEF$ be inscribed inside the same circle $\Omega$ so that they form a hexagram. However, we do not require anymore that they are circumscribed about the same circle.

Define points $T^\text{Circ}_A$, $T^\text{Circ}_B$, \dots, $T^\text{In}_C$ as
in the previous remark. It looks like once again points $T^\text{Circ}_A$,
$T^\text{Circ}_B$, and $T^\text{Circ}_C$ are collinear and points $T^\text{In}_A$,
$T^\text{In}_B$, and $T^\text{In}_C$ are also collinear. However, I do not
have proofs of these claims.
\end{remark*}

\begin{remark*}[Further generalization from Andrew Gu]
  Let $ABC$ and $DEF$ be triangles which share an
  inconic, or equivalently share a circumconic.  Define points $T^\text{Circ}_A$,
  $T^\text{Circ}_B, \dots, T^\text{In}_C$ as in the previous remarks. Then
  it is conjectured that points $T^\text{Circ}_A$, $T^\text{Circ}_B$, and
  $T^\text{Circ}_C$ are collinear and points $T^\text{In}_A$, $T^\text{In}_B$,
  and $T^\text{In}_C$ are also collinear. (Note that extraversion may be
  required depending on the configuration of points, e.g. excircles instead of
  incircles.) Additionally, it appears that the insimilicenters of the
  circumcircles lie on a line perpendicular to the line through
  $T^\text{Circ}_A$, $T^\text{Circ}_B$, and $T^\text{Circ}_C$.
\end{remark*}
\pagebreak

\section{Solutions to Day 3}
\subsection{TSTST 2021/7, proposed by Ankit Bisain, Holden Mui}
\textsl{Available online at \url{https://aops.com/community/p24130213}.}
\begin{mdframed}[style=mdpurplebox,frametitle={Problem statement}]
Let $M$ be a finite set of lattice points
and $n$ be a positive integer.
A \emph{mine-avoiding path} is a path of lattice points with length $n$,
beginning at $(0,0)$ and ending at a point on the line $x+y=n$,
that does not contain any point in $M$.
Prove that if there exists a mine-avoiding path,
then there exist at least $2^{n-|M|}$ mine-avoiding paths.
\end{mdframed}
We present two approaches.

\paragraph{Solution 1.}
We prove the statement by induction on $n$. We use $n=0$ as a base case, where
the statement follows from $1\geq 2^{-|M|}$. For the inductive step, let $n >
0$. There exists at least one mine-avoiding path, which must pass through either
$(0, 1)$ or $(1, 0)$. We consider two cases:

\textbf{Case 1: there exist mine-avoiding paths starting at both $(1, 0)$ and
$(0, 1)$.}

By the inductive hypothesis, there are at least $2^{n-1-|M|}$ mine-avoiding
paths starting from each of $(1, 0)$ and $(0, 1)$. Then the total number of
mine-avoiding paths is at least $2^{n-1-|M|}+2^{n-1-|M|}=2^{n-|M|}$.

\textbf{Case 2: only one of $(1, 0)$ and $(0, 1)$ is on a mine-avoiding path.}

Without loss of generality, suppose no mine-avoiding path starts at $(0, 1)$.
Then some element of $M$ must be of the form $(0, k)$ for $1\leq k\leq n$ in
order to block the path along the $y$-axis. This element can be ignored for any
mine-avoiding path starting at $(1, 0)$. By the inductive hypothesis, there are
at least $2^{n-1-(|M|-1)}=2^{n-|M|}$ mine-avoiding paths.

This completes the induction step, which solves the problem.

\paragraph{Solution 2.}

\begin{lemma*}
  If $|M|<n$, there is more than one mine-avoiding path.
\end{lemma*}

\begin{proof}
  Let $P_0,P_1,\dots,P_{n}$ be a mine-avoiding path. Set $P_i=(x_i,y_i)$. For $0 \leq i < n$, define a path $Q_i$ as follows:

  \begin{itemize}
    \item Make the first $i+1$ points $P_0,P_1,\dots,P_i$.
    \item If $P_i \to P_{i+1}$ is one unit up, go right until $(n-y_i,y_i)$.
    \item If $P_i \to P_{i+1}$ is one unit right, go up until $(x_i,n-x_i)$.
  \end{itemize}

  \begin{center}
    \begin{asy}
      unitsize(3);
      draw((0,0)--(10,0)--(20,0)--(20,10)--(30,10)--(30,20));
      draw((0,0)--(0,10)--(0,20)--(0,30)--(0,40)--(0,50), rgb(1,0,0));
      draw((10,0)--(10,40),rgb(1,0,0));
      draw((20,0)--(50,0),rgb(1,0,0));
      draw((20,10)--(20,30),rgb(1,0,0));
      draw((30,10)--(40,10),rgb(1,0,0));
      draw((0,50)--(50,0),rgb(0,0,1) + dashed);
    \end{asy}

  \end{center}

  The diagram above is an example for $n=5$ with the new segments formed by the $Q_i$ in red, and the line $x+y=n$ in blue.

  By definition, $M$ has less than $n$ points, none of which are in the original path. Since all $Q_i$ only intersect in the original path, each mine is in at most one of $Q_0,Q_1,\dots,Q_{n-1}.$ By the Pigeonhole Principle, one of the $Q_i$ is mine-avoiding.
\end{proof}

Now, consider the following process:
\begin{itemize}
  \item Start at $(0,0)$.
  \item If there is only one choice of next step that is part of a mine-avoiding path, make that choice.
  \item Repeat the above until at a point with two possible steps that are part of mine-avoiding paths.
  \item Add a mine to the choice of next step with more mine-avoiding paths through it. If both have the same number of mine-avoiding paths through them, add a mine arbitrarily.
\end{itemize}

\begin{center}
  \begin{asy}
    unitsize(3);
    draw((-5,-5)--(45,-5));
    draw((-5,5)--(45,5));
    draw((-5,15)--(35,15));
    draw((-5,25)--(25,25));
    draw((-5,35)--(15,35));
    draw((-5,45)--(5,45));
    draw((-5,-5)--(-5,45));
    draw((5,-5)--(5,45));
    draw((15,-5)--(15,35));
    draw((25,-5)--(25,25));
    draw((35,-5)--(35,15));
    draw((45,-5)--(45,5));
    fill((-5,5)--(-5,15)--(5,15)--(5,5)--cycle);
    fill((25,5)--(25,15)--(35,15)--(35,5)--cycle);
    label("$1$",(10,30));
    label("$1$",(20,20));
    label("$1$",(40,0));
    label("$1$",(30,0));
    label("$2$",(20,0));
    label("$2$",(10,20));
    label("$1$",(20,10));
    label("$3$",(10,10));
    label("$5$",(10,0));
    label("$5$",(0,0));
    label("$0$",(0,20));
    label("$0$",(0,30));
    label("$0$",(0,40));
  \end{asy}
\end{center}

For instance, consider the above diagram for $n=4$. Lattice points are replaced with squares. Mines are black squares and each non-mine is labelled with the number of mine-avoiding paths passing through it. This process would start at $(0,0)$, go to $(1,0)$, then place a mine at $(1,1)$.


This path increases the size of $M$ by one, and reduces the number of mine-avoiding paths to a nonzero number at most half of the original. Repeat this process until there is only one path left. By our lemma, the number of mines must be at least $n$ by the end of the process, so the process was iterated at least $n-|M|$ times. By the halving property, there must have been at least $2^{n-|M|}$ mine-avoiding paths before the process, as desired.
\pagebreak

\subsection{TSTST 2021/8, proposed by Fedir Yudin}
\textsl{Available online at \url{https://aops.com/community/p24130228}.}
\begin{mdframed}[style=mdpurplebox,frametitle={Problem statement}]
Let $ABC$ be a scalene triangle.
Points $A_1$, $B_1$ and $C_1$ are chosen
on segments $BC$, $CA$, and $AB$, respectively,
such that $\triangle A_1B_1C_1$ and $\triangle ABC$
are similar.
Let $A_2$ be the unique point on line $B_1C_1$
such that $AA_2 = A_1A_2$.
Points $B_2$ and $C_2$ are defined similarly.
Prove that $\triangle A_2B_2C_2$ and $\triangle ABC$ are similar.
\end{mdframed}
We give three solutions.
\paragraph{Solution 1 (author).}
We'll use the following lemma.

\begin{lemma*}
  Suppose that $PQRS$ is a convex quadrilateral with $\angle P = \angle R$. Then
  there is a point $T$ on $QS$ such that $\angle QPT = \angle SRP$, $\angle
  TRQ = \angle RPS$, and $PT=RT$.
\end{lemma*}

Before proving the lemma, we will show how it solves the problem. The lemma
applied for the quadrilateral $AB_1A_1C_1$ with $\angle A = \angle A_1$ shows
that $\angle B_1A_1A_2 = \angle C_1AA_1$. This implies that the point $A_2$ in
$\triangle A_1B_1C_1$ corresponds to the point $A_1$ in $\triangle ABC$. Then
$\triangle A_2B_2C_2 \sim \triangle A_1B_1C_1 \sim \triangle ABC$, as desired.

Additionally, $PT=RT$ is a corollary of the angle conditions because
\[\measuredangle PRT = \measuredangle SRQ - \measuredangle TRQ - \measuredangle SRP = \measuredangle QPS - \measuredangle RPS -
\measuredangle QPT = \measuredangle TPR.\]
Therefore we only need to prove the angle conditions.

\subparagraph{Proof 1 of lemma}
  Denote $X = PQ \cap RS$ and $Y = PS \cap RQ$. Note that $\angle XPY = \angle
  XRY$, so $PRXY$ is cyclic. Let $T$ be the point of intersection of tangents to this circle at $P$ and $R$. By Pascal's theorem for the degenerate hexagon $PPXRRY$, we have $T \in QS$ (alternatively, $Q$, $S$, and $T$ are collinear on the pole of $PR \cap XY$ with respect to the circle). Also, $\measuredangle QPT = \measuredangle XRP = \measuredangle SRP$ and similarly $\measuredangle TRQ = \measuredangle RPY = \measuredangle RPS$, so we're done.

\begin{center}
  \begin{asy}
    size(8cm);

    pair P, Q, Rp, R, S, X, Y, T;

    P = dir(50);
    Q = dir(200);
    S = dir(340);
    Rp = dir(40);
    R = 2*foot(Rp, Q, S) - Rp;
    X = extension(P, Q, R, S);
    Y = extension(P, S, R, Q);
    T = extension(Q, S, (P+R)/2, (P+R)/2+dir(90)*(P-R));

    draw(P--Q--R--S--cycle);
    draw(P--X--S--Y--R--cycle);
    draw(Q--S);
    draw(P--T--R, red);

    //draw(anglemark(Q, P, T, 10));
    //draw(anglemark(S, R, P, 10));
    //draw(anglemark(T, R, Q, 9, 11));
    //draw(anglemark(R, P, S, 9, 11));

    draw(circumcircle(P, R, X));

    string[] names={"$P$","$Q$","$R$","$S$","$X$","$Y$","$T$"};
    pair[] pts={P, Q, R, S, X, Y, T};
    pair[] labels={dir(110), Q, dir(220), S, dir(120), dir(240), T};
    for(int i=0;i<names.length;++i){
      dot(names[i], pts[i], dir(labels[i]));
    }
  \end{asy}
\end{center}


\subparagraph{Proof 2 of lemma}
Let $P'$ and $R'$ be the reflections of $P$ and $R$ in $QS$. Note that $PR'$ and
$RP'$ intersect at a point $X$ on $QS$. Let $T$ be the second intersection of
the circumcircle of $\triangle PRX$ with $QS$. Note that
\begin{align*}
  \measuredangle PXT &= \measuredangle R'PQ + \measuredangle PQS \\
                       &= \measuredangle R'SQ + \measuredangle PQS  \\
                       &= \measuredangle QSR + \measuredangle PQS \\
                       &= \measuredangle (PQ, SR) \\
                       &= \measuredangle QPR + \measuredangle PRS.
\end{align*}
This means that
\begin{align*}
  \measuredangle QPT &= \measuredangle QPR - \measuredangle TPR \\
                       &= \measuredangle QPR - \measuredangle TXR \\
                       &= \measuredangle QPR - \measuredangle PXT \\
                       &= \measuredangle QPR - \measuredangle QPR -
                       \measuredangle PRS \\
                       &= \measuredangle SRP.
\end{align*}
Similarly, $\measuredangle QRT = \measuredangle SPR$, so we're done.

\begin{center}
  \begin{asy}
    size(8cm);

    pair P, Q, R, S, Pp, Rp, X, T;

    P = dir(50);
    Q = dir(200);
    S = dir(340);
    Rp = dir(70);
    R = 2*foot(Rp, Q, S) - Rp;
    Pp = 2*foot(P, Q, S) - P;
    X = extension(P, Rp, R, Pp);
    T = 2*foot(circumcenter(P, R, X), Q, S) - X;


    draw(P--Q--R--S--cycle);
    draw(P--R--T--cycle);
    draw(Pp--Q--Rp--S--cycle, dashed);
    draw(Q--X);
    draw(R--X--Rp, red);

    draw(circumcircle(P, Q, S));
    draw(circumcircle(R, Q, S));
    draw(circumcircle(P, R, X), blue);

    string[] names={"$P$","$Q$","$R$","$S$","$P'$","$R'$","$X$","$T$"};
    pair[] pts={P, Q, R, S, Pp, Rp, X, T};
    pair[] labels={P, Q, R, dir(330), Pp, Rp, X, dir(220)};
    for(int i=0;i<names.length;++i){
      dot(names[i], pts[i], dir(labels[i]));
    }

  \end{asy}
\end{center}

\subparagraph{Proof 3 of lemma}
  Let $T$ be the point on $QS$ such that $\angle QPT = \angle SRP$. Then we have
  \[
    \frac{QT}{TS} = \frac{\sin QPT \cdot PT / \sin PQT}{\sin TPS \cdot PT / \sin TSP} = \frac{PQ / \sin PRQ}{PS / \sin SRP} = \frac{R(\triangle PQR)}{R(\triangle PRS)},
  \]
  which is symmetric in $P$ and $R$, so we're done.

\paragraph{Solution 2 (Ankan Bhattacharya).}
We prove the main claim $\tfrac{B_1A_2}{A_2C_1} = \tfrac{BA_1}{A_1C}$.

Let $\triangle A_0B_0C_0$ be the medial triangle of $\triangle ABC$.
In addition, let $A_1'$ be the reflection of $A_1$ over $\ol{B_1C_1}$,
and let $X$ be the point satisfying $\triangle XBC \stackrel{-}{\sim} \triangle AB_1C_1$,
so that we have a compound similarity
\[ \triangle ABC \sqcup X
\stackrel{-}{\sim} \triangle A_1'B_1C_1 \sqcup A. \]
Finally, let $O_A$ be the circumcenter of $\triangle A_1'B_1C_1$,
and let $A_2^*$ be the point on $\ol{B_1C_1}$
satisfying $\tfrac{B_1A_2^*}{A_2^*C_1} = \tfrac{BA_1}{A_1C}$.

Recall that $O$ is the Miquel point of $\triangle A_1B_1C_1$,
as well as its orthocenter.
\begin{claim*}
  $\ol{AA_1'} \parallel \ol{BC}$.
\end{claim*}
\begin{proof}
  We need to verify that the foot from $A_1$ to $\ol{B_1C_1}$
  lies on the $A$-midline.
  This follows from the fact that $O$ is both the Miquel point
  and the orthocenter.
\end{proof}

\begin{claim*}
  $\ol{AX} \parallel \ol{B_1C_1}$.
\end{claim*}
\begin{proof}
  From the compound similarity,
  \[ \dang (\ol{BC}, \ol{AX})
  = \dang (\ol{AA_1'}, \ol{B_1C_1}). \]
  As $\ol{AA_1'} \parallel \ol{BC}$,
  we obtain $\ol{AX} \parallel \ol{B_1C_1}$.
\end{proof}

\begin{claim*}
  $\ol{AX} \perp \ol{A_1O}$.
\end{claim*}
\begin{proof}
  Because $O$ is the orthocenter of $\triangle A_1B_1C_1$.
\end{proof}

\begin{claim*}
  $\ol{AA_1'} \perp \ol{A_2^*O_A}$.
\end{claim*}
\begin{proof}
  Follows from $\ol{AX} \perp \ol{A_1O}$
  after the similarity
  \[ \triangle ABC \sqcup X
  \stackrel{-}{\sim} \triangle A_1'B_1C_1 \sqcup A.\qedhere\]
\end{proof}

\begin{claim*}
  $AA_2^* = A_1'A_2$.
\end{claim*}
\begin{proof}
  Since $\dang C_1AB_1 = \dang C_1A_1'B_1$,
  $AO_A = A_1'O_A$,
  so $\ol{AA_1'} \perp \ol{A_2^*O_A}$
  implies $AA_2^* = A_1'A_2^*$.
\end{proof}

Finally, $A_1'A_2^* = A_1A_2^*$ by reflections,
so $AA_2^* = A_1A_2^*$, and $A_2^* = A_2$.
\pagebreak
\pagebreak

\subsection{TSTST 2021/9, proposed by Victor Wang}
\textsl{Available online at \url{https://aops.com/community/p24130243}.}
\begin{mdframed}[style=mdpurplebox,frametitle={Problem statement}]
Let $q=p^r$ for a prime number $p$ and positive integer $r$.
Let $\zeta = e^{\frac{2\pi i}{q}}$.
Find the least positive integer $n$ such that
\[
  \sum_{\substack{1 \le k \le q \\ \gcd(k,p) = 1}}
  \frac{1}{(1 - \zeta^k)^n}
\]
is not an integer.
(The sum is over all $1\leq k\leq q$ with $p$ not dividing $k$.)
\end{mdframed}
Let $S_q$ denote the set of primitive $q$th roots of unity (thus,
the sum in question is a sum over $S_q$).
\paragraph{Solution 1 (author).}
Let $\zeta_p=e^{2\pi i/p}$ be a fixed primitive $p$th root of unity. Observe
that the given sum is an integer for all $n\leq 0$ (e.g.  because the sum is an
integer symmetric polynomial in the primitive $q$th roots of unity). By
expanding polynomials in the basis $(1-x)^{k}$, it follows that if the sum in
the problem statement is an integer for all $n < n_0$, then
\[\sum_{\omega\in S_q} \frac{f(\omega)}{(1-\omega)^{n}}\in \ZZ\]
for all $n < n_0$ and $f\in \ZZ[x]$, whereas for $n=n_0$ there is some
$f\in \ZZ[x]$ for which the sum is not an integer (e.g. $f=1$).

Let $z_q=r\phi(q)-q/p=p^{r-1}[r(p-1)-1]$. We claim that the answer is $n =
z_q+1$.  We prove this by induction on $r$. First is the base case $r=1$.

\begin{lemma*}
  There exist polynomials $u,v\in \ZZ[x]$ such that
  $(1-\omega)^{p-1}/p = u(\omega)$ and $p/(1-\omega)^{p-1} = v(\omega)$ for all
  $\omega\in S_p$.

  (What we are saying is that $p$ is $(1-\omega)^{p-1}$ times a \emph{unit}
  (invertible algebraic integer), namely $v(\omega)$.)
\end{lemma*}

\begin{proof}
  Note that $p = (1-\omega)\dotsm(1-\omega^{p-1})$. Thus we can write
  \begin{align*}
    \frac{p}{(1-\omega)^{p-1}} &=
    \frac{1-\omega}{1-\omega} \cdot\frac{1-\omega^2}{1-\omega} \dotsm
    \frac{1-\omega^{p-1}}{1-\omega}
  \end{align*}
  and take
  \[v(x)=\prod_{k=1}^{p-1}\frac{1-x^{k}}{1-x}.\]
  Similarly, the polynomial $u$ is
  \[u(x) = \prod_{k=1}^{p-1}\frac{1-x^{k\ell_k}}{1-x^k}\]
  where $\ell_k$ is a multiplicative inverse of $k$ modulo $p$.
\end{proof}

Now, the main idea: given $g\in \ZZ[x]$, observe that \[S = \sum_{\omega
\in S_p} (1-\omega)g(\omega)\] is divisible by $1-\zeta_p^k$ (i.e. it is
$1-\zeta_p^k$ times an algebraic integer) for every $k$ coprime to $p$.  By
symmetric sums, $S$ is an integer; since $S^{p-1}$ is divisible by
$(1-\zeta_p)\dotsm(1-\zeta_p^{p-1}) = p$, the integer $S$ must itself be
divisible by $p$. (Alternatively, since $h(x) := (1-x)g(x)$ vanishes at
$x=1$, one can interpret $S$ using a roots of unity filter: $S = p\cdot h([x^0]
+ [x^p] + \dotsb) \equiv 0\pmod{p}$.) Now write
\[
\ZZ
\ni \frac{S}{p}
= \sum_{\omega \in S_p} \frac{(1-\omega)^{p-1}}{p} \frac{g(\omega)}{(1-\omega)^{p-2}}
= \sum_{\omega \in S_p} u(\omega)\frac{g(\omega)}{(1-\omega)^{p-2}}.
\]
Taking $g = v\cdot (1-x)^k$ for $k\geq 0$, we see that the sum in the
problem statement is an integer for any $n\leq p-2$.

Finally, we have
\[\sum_{\omega\in S_p}\frac{u(\omega)}{(1-\omega)^{p-1}}=\sum_{\omega\in
S_p}\frac{1}{p}=\frac{p-1}{p}\notin\ZZ,\]
so the sum is not an integer for $n=p-1$.

Now let $r\ge2$ and assume the induction hypothesis for $r-1$.

\begin{lemma*}
  There exist polynomials $U,V\in \ZZ[x]$ such that
  $(1-\omega)^p/(1-\omega^p) = U(\omega)$ and $(1-\omega^p)/(1-\omega)^p =
  V(\omega)$ for all $\omega\in S_q$. (Again, these are units.)
\end{lemma*}

\begin{proof}
  Similarly to the previous lemma, we write $1-\omega^p =
  (1-\omega\zeta_p^0)\dotsm(1-\omega\zeta_p^{p-1})$. The polynomials $U$ and $V$
  are
  \begin{align*}
    U(x) &= \prod_{k=1}^{p-1}\frac{1-x^{(kq/p+1)\ell_k}}{1-x^{kq/p+1}} \\
    V(x) &= \prod_{k=1}^{p-1}\frac{1-x^{kq/p+1}}{1-x}
  \end{align*}
  where $\ell_k$ is a multiplicative inverse of $kq/p+1$ modulo $q$.
\end{proof}

\begin{corollary*}
  If $\omega\in S_q$, then $(1-\omega)^{\phi(q)}/p$ is a unit.
\end{corollary*}

\begin{proof}
  Induct on $r$. For $r=1$, this is the first lemma. For the inductive step,
we are given that $(1-\omega^p)^{\phi(q/p)}/p$ is a unit. By the second lemma,
$(1-\omega)^{\phi(q)}/(1-\omega^p)^{\phi(q/p)}$ is also a unit. Multiplying
these together yields another unit.
\end{proof}

Thus we have polynomials $A, B\in \ZZ[x]$ such that
\begin{align*}
  A(\omega) &= \frac{p}{(1-\omega)^{\phi(q)}}V(\omega)^{z_{q/p}} \\
  B(\omega) &= \frac{(1-\omega)^{\phi(q)}}{p}U(\omega)^{z_{q/p}}
\end{align*}
for all $\omega\in S_q$.

Given $g\in \ZZ[x]$, consider the $p$th roots of unity filter \[S(x) :=
\sum_{k=0}^{p-1} g(\zeta_p^k x) = p\cdot h(x^p),\] where $h\in \ZZ[x]$.
Then \[ph(\eta) = S(\omega) = \sum_{\omega^p = \eta} g(\omega)\] for all $\eta\in
S_{q/p}$, so
\begin{align*}
\frac{h(\eta)}{(1-\eta)^{z_{q/p}}}
= \frac{S(\omega)}{p(1-\eta)^{z_{q/p}}}
&= \sum_{\omega^p = \eta} \frac{(1-\omega)^{pz_{q/p}}}{(1-\omega^p)^{z_{q/p}}} \frac{g(\omega)}{p(1-\omega)^{pz_{q/p}}} \\
&= \sum_{\omega^p = \eta}
U(\omega)^{z_{q/p}}\frac{(1-\omega)^{\phi(q)}}{p}\frac{g(\omega)}{(1-\omega)^{z_q}}.
\end{align*}
(Implicit in the last line is $z_q=\phi(q)+pz_{q/p}$.)
Since $U(\omega)$ and $(1-\omega)^{\phi(q)}/p$ are units, we can let $g=A\cdot
f$ for arbitrary $f\in \ZZ[x]$, so that the expression in the summation
simplifies to $f(\omega)/(1-\omega)^{z_q}$.
From this we conclude that for any $f\in \ZZ[x]$, there exists $h\in
\ZZ[x]$ such that
\begin{align*}
  \sum_{\omega\in S_q}\frac{f(\omega)}{(1-\omega)^{z_q}} &= \sum_{\eta\in
  S_{q/p}}\sum_{\omega^p=\eta}\frac{f(\omega)}{(1-\omega)^{z_q}} \\
                                                         &= \sum_{\eta\in
                                                         S_{q/p}}\frac{h(\eta)}{(1-\eta)^{z_{q/p}}}.
\end{align*}
By the inductive hypothesis, this is always an integer.

In the other direction, for $\eta\in S_{q/p}$ we have
\begin{align*}
  \sum_{\omega^p = \eta} \frac{B(\omega)}{(1-\omega)^{1+z_q}} &=
  \sum_{\omega^p=\eta}\frac{1}{p(1-\omega^p)^{z_{q/p}}(1-\omega)} \\
& = \frac{1}{p(1-\eta)^{z_{q/p}}}\sum_{\omega^p = \eta} \frac{1}{1-\omega} \\
& = \frac{1}{p(1-\eta)^{z_{q/p}}}\left[\frac{px^{p-1}}{x^p - \eta}\right]_{x=1}
\\
& = \frac{1}{(1-\eta)^{1 + z_{q/p}}}.
\end{align*}
Summing over all $\eta\in S_{q/p}$, we conclude by the inductive hypothesis that
\[\sum_{\omega\in S_q}\frac{B(\omega)}{(1-\omega)^{1+z_q}}=\sum_{\eta\in
S_{q/p}}\frac{1}{(1-\eta)^{1+z_{q/p}}}\]
is not an integer. This completes the solution.

\paragraph{Solution 2 (Nikolai Beluhov).} %Newton sums

Suppose that the complex numbers $\frac{1}{1-\omega}$ for $\omega \in S_q$ are
the roots of
\[P(x) = x^d - c_1x^{d - 1} + c_2x^{d - 2} - \dotsb \pm c_d,\]
so that $c_k$ is their $k$-th elementary symmetric polynomial and $d = \phi(q) =
(p - 1)p^{r - 1}$. Additionally denote
\[S_n = \sum_{\omega\in S_q} \frac{1}{(1 - \omega)^n}.\]
Then, by Newton's identities,
\begin{align*}
  S_1 &= c_1, \\
  S_2 &= c_1S_1 - 2c_2, \\
  S_3 &= c_1S_2 - c_2S_1 + 3c_3,
\end{align*}
and so on. The general pattern when $n\leq d$ is
\[S_n = \left[\sum_{j = 1}^{n - 1} (-1)^{j + 1}c_jS_{n - j}\right] + (-1)^{n +
1}nc_n.\]
After that, when $n > d$, the pattern changes to
\[S_n = \sum_{j = 1}^d (-1)^{j + 1}c_jS_{n - j}.\]

\begin{lemma*}
  All of the $c_i$ are integers except for $c_d$. Furthermore, $c_d$ is $1/p$
  times an integer.
\end{lemma*}

\begin{proof}
The $q$th cyclotomic polynomial is
\[\Phi_q(x)=1+x^{p^{r-1}}+x^{2p^{r-1}}+\dotsb+x^{(p-1)p^{r-1}}.\]
The polynomial
\[Q(x) = 1+(1+x)^{p^{r-1}}+(1+x)^{2p^{r-1}}+\dotsb+(1+x)^{(p-1)p^{r-1}}\]
has roots $\omega-1$ for $\omega \in S_q$, so it is equal to $p(-x)^dP(-1/x)$ by
comparing constant coefficients. Comparing the remaining coefficients, we find
that $c_n$ is $1/p$ times the $x^{n}$ coefficient of $Q$.

Since $(x + y)^p \equiv x^p + y^p \pmod{p}$, we conclude that, modulo $p$,
\begin{align*}
  Q(x) &\equiv 1 + \big(1 + x^{p^{r - 1}}\big) + \big(1 + x^{p^{r - 1}}\big)^2 + \dotsb + \big(1 + x^{p^{r -
1}}\big)^{p - 1} \\
       &\equiv \Big[\big(1 + x^{p^{r - 1}}\big)^p - 1\Big]/x^{p^{r - 1}}.
\end{align*}

Since $\binom{p}{j}$ is a multiple of $p$ when $0 < j < p$, it follows that all
coefficients of $Q(x)$ are multiples of $p$ save for the leading one. Therefore,
$c_n$ is an integer when $n < d$, while $c_d$ is $1/p$ times an integer.
\end{proof}

By the recurrences above, $S_n$ is an integer for $n < d$. When $r = 1$, we get
that $dc_d$ is not an integer, so $S_d$ is not an integer, either. Thus the
answer for $r=1$ is $n=p-1$.

Suppose now that $r \ge 2$. Then $dc_d$ does become an integer, so $S_d$ is an integer as well.

%Using our recurrence for the $S_n$ with $n > d$, we see that the smallest $n$
%with $S_n$ non-integer is $d$ larger than the smallest $n$ with $S_n$ not a multiple of $p$.

\begin{lemma*}
  For all $n$ with $1 \le n \le d$, we have $\nu_p(nc_n) \ge r - 2$.
  Furthermore, the smallest $n$ such that $\nu_p(nc_n) = r - 2$ is $d - p^{r - 1} +
  1$.
\end{lemma*}
\begin{proof}
  The value of $nc_n$ is $1/p$ times the coefficient of $x^{n-1}$ in the
  derivative $Q'(x)$. This derivative is
\[p^{r - 1}(1 + x)^{p^{r - 1} - 1}\left[\sum_{k = 1}^{p - 1} k(1 + x)^{(k - 1)p^{r -
1}}\right].\]

What we want to prove reduces to showing that all coefficients of the
polynomial in the square brackets are multiples of $p$ except for the leading one.

Using the same trick $(x + y)^p \equiv x^p + y^p \pmod{p}$ as before and also
writing $w$ for $x^{p^{r - 1}}$, modulo $p$ the polynomial in the square brackets
becomes
\[1 + 2(1 + w) + 3(1 + w)^2 + \dotsb + (p - 1)(1 + w)^{p - 2}.\]
This is the derivative of
\[1 + (1 + w) + (1 + w)^2 + \dotsb + (1 + w)^{p - 1} = [(1 + w)^p - 1]/w\]
and so, since $\binom{p}{j}$ is a multiple of $p$ when $0 < j < p$, we are
done.
\end{proof}

Finally, we finish the problem with the following claim.
\begin{claim*}
  Let $m=d-p^{r-1}$. Then for all $k\geq 0$ and $1\leq j\leq d$, we have
  \begin{align*}
    \nu_p(S_{kd+m+1}) &= r-2-k \\
    \nu_p(S_{kd+m+j}) &\geq r-2-k.
  \end{align*}
\end{claim*}

\begin{proof}
  First, $S_1, S_2, \dots, S_m$ are all divisible by $p^{r - 1}$ by Newton's
  identities and the second lemma. Then $\nu_p(S_{m + 1}) = r - 2$ because
  \[\nu_p((m+1)c_{m+1})=r-2,\]
  and all other terms in the recurrence relation are divisible by $p^{r-1}$. We
  can similarly check that $\nu_p(S_{n})\geq r-2$ for $m+1\leq n \leq d$.
  Newton's identities combined with the first lemma now imply
  the following for $n > d$:
  \begin{itemize}
    \item If $\nu_p(S_{n-j})\geq \ell$ for all $1\leq j\leq d$ and
      $\nu_p(S_{n-d})\geq \ell+1$, then $\nu_p(S_{n})\geq \ell$.
    \item If $\nu_p(S_{n-j})\geq \ell$ for all $1\leq j\leq d$ and
      $\nu_p(S_{n-d})=\ell$, then $\nu_p(S_{n})=\ell-1$.
  \end{itemize}
  Together, these prove the claim by induction.
\end{proof}
By the claim, the smallest $n$ for which $\nu_p(S_n) < 0$ (equivalent to $S_n$
not being an integer, by the recurrences) is
\[n=(r-1)d+m+1=((p-1)r-1)p^{r-1}+1.\]

\begin{remark*}
  The original proposal was the following more general version:
  \begin{quote}
    Let $n$ be an integer with prime power factorization $q_1\dotsm q_m$.
    Let $S_n$ denote the set of primitive $n$th roots of unity. Find all tuples of nonnegative integers $(z_1,\dots,z_m)$ such that
    \[
      \sum_{\omega\in S_n} \frac{f(\omega)}{(1-\omega^{n/q_1})^{z_1}\dotsm (1-\omega^{n/q_m})^{z_m}} \in \ZZ
    \]
    for all polynomials $f\in \ZZ[x]$.
  \end{quote}
  The maximal $z_i$ are exponents in the prime ideal factorization of the \href{https://en.wikipedia.org/wiki/Different_ideal}{different ideal} of the cyclotomic extension $\QQ(\zeta_n)/\QQ$.
\end{remark*}

\begin{remark*}
Let $F = (x^p - 1)/(x-1)$ be the minimal polynomial of $\zeta_p = e^{2\pi i/p}$ over $\QQ$.
A calculation of Euler shows that
\[
(\ZZ[\zeta_p])^*
:= \{\alpha = g(\zeta_p)\in \QQ[\zeta_p]: \sum_{\omega \in S_p}
f(\omega)g(\omega)\in \ZZ\; \forall f\in \ZZ[x]\}
= \frac{1}{F'(\zeta_p)}\cdot \ZZ[\zeta_p],
\]
where
\[
F'(\zeta_p) = \frac{p\zeta_p^{p-1} - [1+\zeta_p+\dotsb+\zeta_p^{p-1}]}{1-\zeta_p} = p(1-\zeta_p)^{-1}\zeta_p^{p-1}
\]
is $(1-\zeta_p)^{[p-1]-1} = (1-\zeta_p)^{p-2}$ times a unit of
$\ZZ[\zeta_p]$. Here, $(\ZZ[\zeta_p])^*$ is the dual lattice of
$\ZZ[\zeta_p]$.
\end{remark*}

%\begin{remark*}
%Let $\alpha$ be algebraic of degree $d$ over $\QQ$, and let $f\in \QQ[x]$ be the minimal polynomial of $\alpha$ (of degree $d$).
%Hidden in the previous remark is Euler's calculation: the trace of $\alpha^i/f'(\alpha)$ (sum over conjugates) is $0$ for $i=0,1,\dots,d-2$ and $1$ for $i=d-1$, proven either using partial fractions, Lagrange interpolation, or other symmetric sum techniques (e.g. Newton sums).
%All of this can be phrased more directly (e.g. in terms of symmetric sums of $S_n$), so one could likely use standard elementary methods to directly compute the sums in the problem statement.
%\end{remark*}

\begin{remark*}
Let $K = \QQ(\omega)$, so $(p)$ factors as $(1-\omega)^{p-1}$ in the ring of integers $\mathcal{O}_K$ (which, for cyclotomic fields, can be shown to be $\ZZ[\omega]$).
%total ramification
In particular, the \emph{ramification index} $e$ of $(1-\omega)$ over $p$ is the exponent, $p-1$.
Since $e = p-1$ is not divisible by $p$, we have so-called \emph{tame ramification}.
%So look at the exponent of $1-\omega$ in the inverse different, get $[p-1]-1 = p-2$.
Now by the
\href{https://en.wikipedia.org/wiki/Different_ideal\#Ramification}{ramification
theory} of Dedekind's different ideal, the exponent $z_1$ that works when $n=p$ is $e-1 = p-2$.

Higher prime powers are more interesting because of wild ramification: $p$ divides $\phi(p^r) = p^{r-1}(p-1)$ if and only if $r>1$.
(This is a similar phenomena to how Hensel's lemma for $x^2 - c$ is more interesting mod powers of 2 than mod odd prime powers.)
\end{remark*}

\begin{remark*}
Let $F = (x^q - 1)/(x^{q/p}-1)$ be the minimal polynomial of $\zeta_q = e^{2\pi i/q}$ over $\QQ$.
The aforementioned calculation of Euler shows that
\[
(\ZZ[\zeta_q])^*
:= \{\alpha = g(\zeta_q)\in \QQ[\zeta_q]: \sum_{\omega \in S_q}
f(\omega)g(\omega)\in \ZZ \; \forall f\in \ZZ[x]\}
= \frac{1}{F'(\zeta_q)}\cdot \ZZ[\zeta_q],
\]
where the chain rule implies (using the computation from the prime case)
\[
F'(\zeta_q) = [p(1-\zeta_p)^{-1}\zeta_p^{p-1}]\cdot \frac{q}{p}\zeta_q^{(q/p) - 1} = q(1-\zeta_p)^{-1}\zeta_q^{-1}.
\]
is $(1-\zeta_q)^{r\phi(q) - q/p} = (1-\zeta_q)^{z_q}$ times a unit of $\ZZ[\zeta_q]$.
\end{remark*}

\pagebreak
\pagebreak


\end{document}
