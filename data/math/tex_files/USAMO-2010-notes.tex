% © Evan Chen
% Downloaded from https://web.evanchen.cc/

\documentclass[11pt]{scrartcl}
\usepackage[sexy]{evan}
\ihead{\footnotesize\textbf{\thetitle}}
\ohead{\footnotesize\href{http://web.evanchen.cc}{\ttfamily web.evanchen.cc},
    updated \today}
\title{USAMO 2010 Solution Notes}
\date{\today}

\begin{document}

\maketitle

\begin{abstract}
This is a compilation of solutions
for the 2010 USAMO.
Some of the solutions are my own work,
but many are from the official solutions provided by the organizers
(for which they hold any copyrights),
and others were found by users on the Art of Problem Solving forums.

These notes will tend to be a bit more advanced and terse than the ``official''
solutions from the organizers.
In particular, if a theorem or technique is not known to beginners
but is still considered ``standard'', then I often prefer to
use this theory anyways, rather than try to work around or conceal it.
For example, in geometry problems I typically use directed angles
without further comment, rather than awkwardly work around configuration issues.
Similarly, sentences like ``let $\mathbb{R}$ denote the set of real numbers''
are typically omitted entirely.

Corrections and comments are welcome!
\end{abstract}

\tableofcontents
\newpage

\addtocounter{section}{-1}
\section{Problems}
\begin{enumerate}[\bfseries 1.]
\item %% Problem 1
Let $AXYZB$ be a convex pentagon inscribed in a semicircle of diameter $AB$.
Denote by $P$, $Q$, $R$, $S$ the feet of the perpendiculars
from $Y$ onto lines $AX$, $BX$, $AZ$, $BZ$, respectively.
Prove that the acute angle formed by lines $PQ$ and $RS$
is half the size of $\angle XOZ$,
where $O$ is the midpoint of segment $AB$.

\item %% Problem 2
There are $n$ students standing in a circle, one behind the other.
The students have heights $h_1<h_2<\dots <h_n$.
If a student with height $h_k$ is standing directly
behind a student with height $h_{k-2}$ or less,
the two students are permitted to switch places.
Prove that it is not possible to make more than
$\binom{n}{3}$ such switches
before reaching a position in which no
further switches are possible.

\item %% Problem 3
The $2010$ positive real numbers $a_1$, $a_2$, \dots , $a_{2010}$
satisfy the inequality $a_i a_j \le i+j$
for all $1 \le i < j \le 2010$.
Determine, with proof, the largest possible value
of the product $a_1a_2\dots a_{2010}$.

\item %% Problem 4
Let $ABC$ be a triangle with $\angle A = 90^{\circ}$. Points $D$ and $E$ lie on sides $AC$ and $AB$, respectively, such that $\angle ABD = \angle DBC$ and $\angle ACE = \angle ECB$. Segments $BD$ and $CE$ meet at $I$. Determine whether or not it is possible for segments $AB$, $AC$, $BI$, $ID$, $CI$, $IE$ to all have integer lengths.

\item %% Problem 5
Let $q = \frac{3p-5}{2}$ where $p$ is an odd prime, and let
\[ S_q = \frac{1}{2\cdot 3 \cdot 4} + \frac{1}{5\cdot 6 \cdot 7}
+ \dots + \frac{1}{q(q+1)(q+2)}. \]
Prove that if $\frac{1}{p}-2S_q = \frac{m}{n}$ for integers $m$ and $n$,
then $m - n$ is divisible by $p$.

\item %% Problem 6
There are $68$ ordered pairs (not necessarily distinct)
of nonzero integers on a blackboard.
It's known that for no integer $k$ does both $(k,k)$ and $(-k,-k)$ appear.
A student erases some of the $136$ integers such that
no two erased integers have sum zero, and scores one point
for each ordered pair with at least one erased integer.
What is the maximum possible score the student can guarantee?

\end{enumerate}
\pagebreak

\section{Solutions to Day 1}
\subsection{USAMO 2010/1, proposed by Titu Andreescu}
\textsl{Available online at \url{https://aops.com/community/p1860802}.}
\begin{mdframed}[style=mdpurplebox,frametitle={Problem statement}]
Let $AXYZB$ be a convex pentagon inscribed in a semicircle of diameter $AB$.
Denote by $P$, $Q$, $R$, $S$ the feet of the perpendiculars
from $Y$ onto lines $AX$, $BX$, $AZ$, $BZ$, respectively.
Prove that the acute angle formed by lines $PQ$ and $RS$
is half the size of $\angle XOZ$,
where $O$ is the midpoint of segment $AB$.
\end{mdframed}
Let $T$ be the foot from $Y$ to $\ol{AB}$.
Then the Simson line implies that lines $PQ$ and $RS$ meet at $T$.

\begin{center}
\begin{asy}
pair A = dir(180);
pair B = dir(0);
pair X = dir(140);
pair Y = dir(100);
pair Z = dir(60);

pair P = foot(Y, A, X);
pair Q = foot(Y, B, X);
pair R = foot(Y, A, Z);
pair S = foot(Y, B, Z);

pair T = extension(P, Q, R, S);

draw(A--X--Y--Z--B, orange);
filldraw(unitcircle, opacity(0.1)+orange, red);

draw(Y--P, red);
draw(Y--Q, red);
draw(Y--R, red);
draw(Y--S, red);
draw(P--X--B, brown);
draw(P--T--S, red);
draw(S--Z--A, brown);

dot("$A$", A, dir(A));
dot("$B$", B, dir(B));
dot("$X$", X, dir(X));
dot("$Y$", Y, dir(Y));
dot("$Z$", Z, dir(Z));
dot("$P$", P, dir(P));
dot("$Q$", Q, dir(Q));
dot("$R$", R, dir(R));
dot("$S$", S, dir(S));
dot("$T$", T, dir(T));

/* TSQ Source:

A = dir 180
B = dir 0
X = dir 140
Y = dir 100
Z = dir 60

P = foot Y A X
Q = foot Y B X
R = foot Y A Z
S = foot Y B Z

T = extension P Q R S

A--X--Y--Z--B orange
unitcircle 0.1 orange / red

Y--P red
Y--Q red
Y--R red
Y--S red
P--X--B brown
P--T--S red
S--Z--A brown

*/
\end{asy}
\end{center}

Now it's straightforward to see $APYRT$ is cyclic
(in the circle with diameter $\ol{AY}$),
and therefore
\[ \angle RTY = \angle RAY = \angle ZAY. \]
Similarly,
\[ \angle YTQ = \angle YBQ = \angle YBX. \]
Summing these gives $\angle RTQ$ is equal to
half the measure of arc $\widehat{XZ}$ as needed.

(Of course, one can also just angle chase;
the Simson line is not so necessary.)
\pagebreak

\subsection{USAMO 2010/2, proposed by David Speyer}
\textsl{Available online at \url{https://aops.com/community/p1860777}.}
\begin{mdframed}[style=mdpurplebox,frametitle={Problem statement}]
There are $n$ students standing in a circle, one behind the other.
The students have heights $h_1<h_2<\dots <h_n$.
If a student with height $h_k$ is standing directly
behind a student with height $h_{k-2}$ or less,
the two students are permitted to switch places.
Prove that it is not possible to make more than
$\binom{n}{3}$ such switches
before reaching a position in which no
further switches are possible.
\end{mdframed}
The main claim is the following observation,
which is most motivated in the situation $j-i = 2$.
\begin{claim*}
  The students with heights $h_i$ and $h_j$
  switch at most $|j-i|-1$ times.
\end{claim*}
\begin{proof}
  By induction on $d = |j-i|$, assuming $j > i$.
  For $d = 1$ there is nothing to prove.

  For $d \ge 2$, look at only students $h_j$, $h_{i+1}$
  and $h_{i}$ ignoring all other students.
  After $h_j$ and $h_i$ switch the first time,
  the relative ordering of the students must be
  $h_i \to h_j \to h_{i+1}$.
  Thereafter $h_j$ must always switch with $h_{i+1}$
  before switching with $h_i$,
  so the inductive hypothesis applies to give the bound
  $1 + j - (i+1) - 1 = j-i - 1$.
\end{proof}
Hence, the number of switches is at most
\[ \sum_{1 \le i < j \le n} \left( |j-i| - 1 \right) = \binom n3. \]
\pagebreak

\subsection{USAMO 2010/3, proposed by Gabriel Carroll}
\textsl{Available online at \url{https://aops.com/community/p1860806}.}
\begin{mdframed}[style=mdpurplebox,frametitle={Problem statement}]
The $2010$ positive real numbers $a_1$, $a_2$, \dots , $a_{2010}$
satisfy the inequality $a_i a_j \le i+j$
for all $1 \le i < j \le 2010$.
Determine, with proof, the largest possible value
of the product $a_1a_2\dots a_{2010}$.
\end{mdframed}
The answer is $3 \times 7 \times 11 \times \dots \times 4019$,
which is clearly an upper bound
(and it's not too hard to show this is the lowest
number we may obtain by multiplying $1005$ equalities together;
this is essentially the rearrangement inequality).
The tricky part is the construction.
Intuitively we want $a_i \approx \sqrt{2i}$,
but the details require significant care.

Note that if this is achievable,
we will require $a_n a_{n+1} = 2n+1$ for all odd $n$.
Here are two constructions:
\begin{itemize}
\ii One can take the sequence such that $a_{2008} a_{2010} = 4018$
  and $a_n a_{n+1} = 2n+1$ for all $n = 1, 2, \dots, 2009$.
  This can be shown to work by some calculation. As an illustrative example,
  \[ a_1 a_4 = \frac{a_1a_2 \cdot a_3a_4}{a_2a_3} = \frac{3 \cdot 7}{5} < 5. \]

\ii In fact one can also take $a_n = \sqrt{2n}$ for all even $n$
  (and hence $a_{n-1} = \sqrt{2n} - \frac{1}{\sqrt{2n}}$ for such even $n$).
\end{itemize}

\begin{remark*}
This is a chief example of an ``abstract'' restriction-based approach.
One can motivate it in three steps:
\begin{itemize}
\ii The bound $3 \cdot 7 \cdot \dots \cdot 4019$ is provably best possible
upper bound by pairing the inequalities;
also the situation with $2010$ replaced by $4$
is constructible with bound $21$.
\ii We have $a_n \approx \sqrt{2n}$ heuristically;
in fact $a_n = \sqrt{2n}$ satisfies inequalities by AM-GM.
\ii So we are most worried about $a_i a_j \le i+j$
when $|i-j|$ is small, like $|i-j| = 1$.
\end{itemize}
I then proceeded to spend five hours on various constructions,
but it turns out that the right thing to do was just require
$a_k a_{k+1} = 2k+1$, to make sure these pass:
and the problem almost solves itself.
\end{remark*}

\begin{remark*}
  When $2010$ is replaced by $4$
  it is not too hard to manually write an explicit example:
  say $a_1 = \frac{\sqrt3}{1.1}$, $a_2 = 1.1\sqrt3$,
  $a_3 = \frac{\sqrt7}{1.1}$ and $a_4 = 1.1\sqrt7$.
  So this is a reason one might guess that
  $3 \times 7 \times \dots \times 4019$
  can actually be achieved in the large case.
\end{remark*}

\begin{remark*}
  Victor Wang says:
  I believe we can actually prove that WLOG (!) assume $a_i a_{i+1} = 2i+1$
  for all $i$ (but there are other ways to motivate that as well,
  like linear programming after taking logs),
  which makes things a bit simpler to think about.
\end{remark*}
\pagebreak

\section{Solutions to Day 2}
\subsection{USAMO 2010/4, proposed by Zuming Feng}
\textsl{Available online at \url{https://aops.com/community/p1860753}.}
\begin{mdframed}[style=mdpurplebox,frametitle={Problem statement}]
Let $ABC$ be a triangle with $\angle A = 90^{\circ}$. Points $D$ and $E$ lie on sides $AC$ and $AB$, respectively, such that $\angle ABD = \angle DBC$ and $\angle ACE = \angle ECB$. Segments $BD$ and $CE$ meet at $I$. Determine whether or not it is possible for segments $AB$, $AC$, $BI$, $ID$, $CI$, $IE$ to all have integer lengths.
\end{mdframed}
The answer is no.
We prove that it is not even possible that $AB$, $AC$, $CI$, $IB$ are all integers.

\begin{center}
  \begin{asy}
    size(5cm);
    pair B = dir(140);
    pair A = conj(B);
    pair C = -B;
    filldraw(A--B--C--cycle, opacity(0.1)+lightblue, blue);
    pair I = incenter(A, B, C);
    pair D = extension(B, I, A, C);
    pair E = extension(C, I, A, B);
    draw(B--D);
    draw(C--E);
    filldraw(incircle(A,B,C), opacity(0.1)+lightred, red+dotted);

    dot("$B$", B, dir(B));
    dot("$A$", A, dir(A));
    dot("$C$", C, dir(C));
    dot("$I$", I, dir(45));
    dot("$D$", D, dir(D));
    dot("$E$", E, dir(E));

    /* Source generated by TSQ */
  \end{asy}
\end{center}

First, we claim that $\angle BIC = 135\dg$.
To see why, note that
\[ \angle IBC + \angle ICB = \frac{\angle B}{2} + \frac{\angle C}{2}
  = \frac{90\dg}{2} = 45\dg. \]
So, $\angle BIC = 180\dg - (\angle IBC + \angle ICB) = 135\dg$, as desired.

We now proceed by contradiction.
The Pythagorean theorem implies
\[ BC^2 = AB^2 + AC^2 \]
and so $BC^2$ is an integer.
However, the law of cosines gives
\begin{align*}
  BC^2 &= BI^2 + CI^2 - 2 BI \cdot CI \cos \angle BIC \\
  &= BI^2 + CI^2 + BI \cdot CI \cdot \sqrt 2.
\end{align*}
which is irrational, and this produces the desired contradiction.
\pagebreak

\subsection{USAMO 2010/5, proposed by Titu Andreescu}
\textsl{Available online at \url{https://aops.com/community/p1860791}.}
\begin{mdframed}[style=mdpurplebox,frametitle={Problem statement}]
Let $q = \frac{3p-5}{2}$ where $p$ is an odd prime, and let
\[ S_q = \frac{1}{2\cdot 3 \cdot 4} + \frac{1}{5\cdot 6 \cdot 7}
+ \dots + \frac{1}{q(q+1)(q+2)}. \]
Prove that if $\frac{1}{p}-2S_q = \frac{m}{n}$ for integers $m$ and $n$,
then $m - n$ is divisible by $p$.
\end{mdframed}
By partial fractions, we have
\[
  \frac{2}{(3k-1)(3k)(3k+1)}
  = \frac{1}{3k-1} - \frac{2}{3k} + \frac{1}{3k+1}.
\]
Thus
\begin{align*}
  2S_q
  &= \left( \frac12 - \frac23 + \frac14 \right)
  + \left( \frac 15 - \frac26 + \frac17 \right)
  + \dots
  + \left( \frac1q - \frac2{q+1} + \frac{1}{q+2} \right) \\
  &= \left( \half + \frac13 + \frac14 + \dots + \frac{1}{q+2} \right)
  - 3\left( \frac13 + \frac 16 + \dots + \frac{1}{q+1} \right) \\
  &= \left( \half + \frac13 + \frac14 + \dots + \frac{1}{q+2} \right)
  - \left( \frac11 + \frac 12 + \dots + \frac{1}{\frac{q+1}{3}} \right) \\
  \implies 2S_q - \frac1p + 1
  &=  \left( \frac11 + \frac12 + \dots + \frac{1}{p-1} \right)
  + \left( \frac{1}{p+1} + \frac{1}{p+2} \dots + \frac{1}{q+2} \right)
  - \left( \frac11 + \frac 12 + \dots + \frac{1}{\frac{q+1}{3}} \right)
\end{align*}

Now we are ready to take modulo $p$.
The given says that $q-p+2 = \frac{q+1}{3}$, so
\begin{align*}
  2S_q - \frac1p + 1
  &=  \left( \frac11 + \frac12 + \dots + \frac{1}{p-1} \right)
  + \left( \frac{1}{p+1} + \frac{1}{p+2} + \dots + \frac{1}{q+2} \right)
  - \left( \frac11 + \frac 12 + \dots + \frac{1}{\frac{q+1}{3}} \right) \\
  &\equiv  \left( \frac11 + \frac12 + \dots + \frac{1}{p-1} \right)
  + \left( \frac11 + \frac12 + \dots + \frac{1}{q-p+2} \right)
  - \left( \frac11 + \frac 12 + \dots + \frac{1}{\frac{q+1}{3}} \right) \\
  &= \frac11 + \frac12+ \dots + \frac{1}{p-1} \\
  &\equiv 0 \pmod p.
\end{align*}

So $\frac1p - 2S_q \equiv 1 \pmod p$ which is the desired.
\pagebreak

\subsection{USAMO 2010/6, proposed by Zuming Feng, Paul Zeitz}
\textsl{Available online at \url{https://aops.com/community/p1860794}.}
\begin{mdframed}[style=mdpurplebox,frametitle={Problem statement}]
There are $68$ ordered pairs (not necessarily distinct)
of nonzero integers on a blackboard.
It's known that for no integer $k$ does both $(k,k)$ and $(-k,-k)$ appear.
A student erases some of the $136$ integers such that
no two erased integers have sum zero, and scores one point
for each ordered pair with at least one erased integer.
What is the maximum possible score the student can guarantee?
\end{mdframed}
The answer is $43$.

The structure of this problem is better understood as follows:
we construct a multigraph whose vertices are the entries,
and the edges are the $68$ ordered pairs on the blackboard.
To be precise, construct a multigraph $G$ with vertices
$a_1$, $b_1$, \dots, $a_n$, $b_n$,
with $a_i = -b_i$ for each $i$.
The ordered pairs then correspond to $68$ edges in $G$,
with self-loops allowed (WLOG) only for vertices $a_i$.
The student may then choose one of $\{a_i, b_i\}$ for each $i$
and wishes to maximize the number of edges adjacent
to the set of chosen vertices.

\begin{center}
\begin{asy}
size(5cm);
pair a1 = (-1,6);
pair b1 = (1,6);
pair a2 = (-1,4);
pair b2 = (1,4);
pair a3 = (-1,2);
pair b3 = (1,2);
pair a4 = (-1,0);
pair b4 = (1,0);

draw(a1--a2, red);
draw(a1--b3, red);
draw(a3--b2, red);
draw(b1..(0.7,4)..b3, red);
draw(circle(a2+0.24*dir(0), 0.24), heavycyan);
draw(circle(a2+0.28*dir(0), 0.28), heavycyan);
draw(circle(a3+0.25*dir(0), 0.25), heavycyan);
draw(a4--b3, red);
draw(a4..(0,0.7)..b3, red);
draw(a4--b4, red);

dot("$3$", a1, dir(180));
dot("$7$", a2, dir(180));
dot("$8$", a3, dir(180));
dot("$2$", a4, dir(180));
dot("$-3$", b1, dir(0));
dot("$-7$", b2, dir(0));
dot("$-8$", b3, dir(0));
dot("$-2$", b4, dir(0));
\end{asy}
\end{center}

First we use the probabilistic method to show $N \ge 43$.
We select the real number $p = \frac{\sqrt5-1}{2} \approx 0.618$
satisfying $p = 1-p^2$.
For each $i$ we then select $a_i$ with probability $p$
and $b_i$ with probability $1-p$.
Then
\begin{itemize}
  \ii Every self-loop $(a_i, a_i)$ is chosen with probability $p$.
  \ii Any edge $(b_i, b_j)$ is chosen with probability $1-p^2$.
\end{itemize}
All other edges are selected with probability at least $p$,
so in expectation we have $68p \approx 42.024$ edges scored.
Hence $N \ge 43$.

For a construction showing $43$ is optimal, we let $n = 8$,
and put five self-loops on each $a_i$,
while taking a single $K_8$ on the $b_i$'s.
The score achieved for selecting $m$ of the $a_i$'s
and $8-m$ of the $b_i$'s is
\[ 5m + \left( \binom82-\binom m2 \right) \le 43 \]
with equality when either $m=5$ and $m=6$.

\begin{remark*}
  [Colin Tang]
  Here is one possible motivation for
  finding the construction.
  In equality case we probably want
  all the edges to either be $a_i$ loops or $b_i b_j$ edges.
  Now if $b_i$ and $b_j$ are not joined by an edge,
  one can ``merge them together'',
  also combining the corresponding $a_i$'s,
  to get another multigraph with $68$ edges
  whose optimal score is at most the original ones.
  So by using this smoothing algorithm,
  we can reduce to a situation where the $b_i$ and $b_j$
  are all connected to each other.

  It's not unnatural to assume it's a clique then,
  at which point fiddling with parameters gives the construction.
  Also, there is a construction for $\left\lceil 2/3 n \right\rceil$
  which is not too difficult to find,
  and applying this smoothing operation to this construction
  could suggest a clique of at least $8$ vertices too.
\end{remark*}

\begin{remark*}
  [David Lee]
  One could consider changing the probability $p(n)$
  to be a function of the number $n$ of non-loops
  (hence there are $68-n$ loops);
  we would then have
  \[ \mathbb E[\text{points}]
    \ge (68-n)p(n) + n(1-p(n)^2). \]
  The optimal value of $p(n)$ is then
  \[
    p(n) = \begin{cases}
      \frac{68-n}{2n} = \frac{34}{n} - \half & n \ge 23 \\
      1 & n < 22.
    \end{cases}
  \]
  For $n > 23$ we then have
  \begin{align*}
    E(n) = & (68-n)\left( \frac{34}{n}-\half \right)
  + n\left( 1-\left( \frac{34}{n}-\half \right)^2 \right) \\
    &= \frac{5n}{4} + \frac{34^2}{n} - 34
  \end{align*}
  which has its worst case at around $5n^2 = 68^2$,
  so at $n=30$ and $n=31$.
  Indeed, one can find
  \begin{align*}
    E(30) &= 42.033 \\
    E(31) &= 42.040.
  \end{align*}
  This gives another way to get the lower bound $43$,
  and gives a hint about approximately how many non-loops
  one would want in order to achieve such a bound.
\end{remark*}
\pagebreak


\end{document}
