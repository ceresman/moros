% © Evan Chen
% Downloaded from https://web.evanchen.cc/

\documentclass[11pt]{scrartcl}
\usepackage[sexy]{evan}
\ihead{\footnotesize\textbf{\thetitle}}
\ohead{\footnotesize\href{http://web.evanchen.cc}{\ttfamily web.evanchen.cc},
    updated \today}
\title{USAMO 1997 Solution Notes}
\date{\today}

\begin{document}

\maketitle

\begin{abstract}
This is a compilation of solutions
for the 1997 USAMO.
Some of the solutions are my own work,
but many are from the official solutions provided by the organizers
(for which they hold any copyrights),
and others were found by users on the Art of Problem Solving forums.

These notes will tend to be a bit more advanced and terse than the ``official''
solutions from the organizers.
In particular, if a theorem or technique is not known to beginners
but is still considered ``standard'', then I often prefer to
use this theory anyways, rather than try to work around or conceal it.
For example, in geometry problems I typically use directed angles
without further comment, rather than awkwardly work around configuration issues.
Similarly, sentences like ``let $\mathbb{R}$ denote the set of real numbers''
are typically omitted entirely.

Corrections and comments are welcome!
\end{abstract}

\tableofcontents
\newpage

\addtocounter{section}{-1}
\section{Problems}
\begin{enumerate}[\bfseries 1.]
\item %% Problem 1
Let $p_1, p_2, p_3, \dots$ be the prime numbers
listed in increasing order,
and let $0 < x_0 < 1$ be a real number between 0 and 1.
For each positive integer $k$, define
\[ x_k = \begin{cases} 0 & \mbox{if} \; x_{k-1} = 0, \\[.1in]
  {\displaystyle \left\{ \frac{p_k}{x_{k-1}} \right\}}
  & \mbox{if} \; x_{k-1} \neq 0,
  \end{cases}  \]
where $\{x\}$ denotes the fractional part of $x$.
Find, with proof, all $x_0$ satisfying $0 < x_0 < 1$
for which the sequence $x_0, x_1, x_2, \dots$ eventually becomes $0$.

\item %% Problem 2
Let $ABC$ be a triangle.
Take noncollinear points $D$, $E$, $F$ on the perpendicular bisectors
of $BC$, $CA$, $AB$ respectively.
Show that the lines through $A$, $B$, $C$
perpendicular to $EF$, $FD$, $DE$ respectively are concurrent.

\item %% Problem 3
Prove that for any integer $n$,
there exists a unique polynomial $Q$ with coefficients
in $\{0,1,\dots,9\}$ such that $Q(-2) = Q(-5) = n$.

\item %% Problem 4
To clip a convex $n$-gon means to choose a pair of
consecutive sides $AB$, $BC$ and to replace them
by the three segments $AM$, $MN$, and $NC$,
where $M$ is the midpoint of $AB$ and $N$ is the midpoint of $BC$.
In other words, one cuts off the triangle $MBN$ to obtain a convex $(n+1)$-gon.
A regular hexagon $\mathcal{P}_6$ of area 1 is clipped to obtain a
heptagon $\mathcal{P}_7$.
Then $\mathcal{P}_7$ is clipped (in one of the seven possible ways)
to obtain an octagon $\mathcal{P}_8$, and so on.
Prove that no matter how the clippings are done,
the area of $\mathcal{P}_n$ is greater than $\frac 13$, for all $n \geq 6$.

\item %% Problem 5
If $a,b,c > 0$ prove that
\[
  \frac{1}{a^3+b^3+abc}
  + \frac{1}{b^3+c^3+abc}
  + \frac{1}{c^3+a^3+abc}
  \le \frac{1}{abc}.
\]

\item %% Problem 6
Suppose the sequence of nonnegative integers
$a_1, a_2, \dots, a_{1997}$ satisfies
\[ a_i + a_j \leq a_{i+j} \leq a_i + a_j + 1  \]
for all $i,j \geq 1$ with $i + j \leq 1997$.
Show that there exists a real number $x$ such that
$a_n = \lfloor nx \rfloor$ for all $1 \leq n \leq 1997$.

\end{enumerate}
\pagebreak

\section{Solutions to Day 1}
\subsection{USAMO 1997/1}
\textsl{Available online at \url{https://aops.com/community/p343871}.}
\begin{mdframed}[style=mdpurplebox,frametitle={Problem statement}]
Let $p_1, p_2, p_3, \dots$ be the prime numbers
listed in increasing order,
and let $0 < x_0 < 1$ be a real number between 0 and 1.
For each positive integer $k$, define
\[ x_k = \begin{cases} 0 & \mbox{if} \; x_{k-1} = 0, \\[.1in]
  {\displaystyle \left\{ \frac{p_k}{x_{k-1}} \right\}}
  & \mbox{if} \; x_{k-1} \neq 0,
  \end{cases}  \]
where $\{x\}$ denotes the fractional part of $x$.
Find, with proof, all $x_0$ satisfying $0 < x_0 < 1$
for which the sequence $x_0, x_1, x_2, \dots$ eventually becomes $0$.
\end{mdframed}
The answer is $x_0$ rational.

If $x_0$ is irrational, then all $x_i$ are irrational by induction.
So the sequence cannot become zero.

If $x_0$ is rational, then all are.
Now one simply observes that the denominators of $x_n$
are strictly decreasing, until we reach $0 = \frac 01$.
This concludes the proof.

\begin{remark*}
  The sequence $p_k$ could have been any sequence of integers.
\end{remark*}
\pagebreak

\subsection{USAMO 1997/2}
\textsl{Available online at \url{https://aops.com/community/p210283}.}
\begin{mdframed}[style=mdpurplebox,frametitle={Problem statement}]
Let $ABC$ be a triangle.
Take noncollinear points $D$, $E$, $F$ on the perpendicular bisectors
of $BC$, $CA$, $AB$ respectively.
Show that the lines through $A$, $B$, $C$
perpendicular to $EF$, $FD$, $DE$ respectively are concurrent.
\end{mdframed}
The three lines are the radical axii of the three circles
centered at $D$, $E$, $F$, so they concur.
\pagebreak

\subsection{USAMO 1997/3}
\textsl{Available online at \url{https://aops.com/community/p343873}.}
\begin{mdframed}[style=mdpurplebox,frametitle={Problem statement}]
Prove that for any integer $n$,
there exists a unique polynomial $Q$ with coefficients
in $\{0,1,\dots,9\}$ such that $Q(-2) = Q(-5) = n$.
\end{mdframed}
If we let
\[ Q(x) = \sum_{k \ge 0} a_k x^k \]
then $a_k$ is uniquely determined by $n \pmod{2^k}$ and $n \pmod{5^k}$.
Indeed, we can extract the coefficients of $Q$
exactly by the following algorithm:
\begin{itemize}
  \ii Define $b_0 = c_0 = n$.
  \ii For $i \ge 0$, let $a_i$ be the unique digit
  satisfying $a_i \equiv b_i \pmod 2$, $a_i \equiv c_i \pmod 5$.
  Then, define
  \[ b_{i+1} = \frac{b_i - a_i}{-2}, \qquad
    c_{i+1} = \frac{c_i - a_i}{-5}. \]
\end{itemize}
The proof is automatic by Chinese remainder theorem,
so this shows uniqueness already.
The tricky part is to show that all $a_i$ are eventually zero
(i.e.\ the ``existence'' step is nontrivial
because a polynomial may only have finitely many nonzero terms).

In fact, we will prove the following claim:
\begin{claim*}
  Suppose $b_0$ and $c_0$ are any integers such that
  \[ b_0 \equiv c_0 \pmod 3.\]
  Then defining $b_i$ and $c_i$ as above,
  we have $b_i \equiv c_i \pmod 3$ for all $i$,
  and $b_N = c_N = 0$ for large enough $N$.
\end{claim*}
\begin{proof}
Dropping the subscripts for ease of notation,
we are looking at the map
\[ (b, c) \mapsto \left( \frac{b-a}{-2},
    \frac{c-a}{-5} \right) \]
for some $0 \le a \le 9$ (a function in $b$ and $c$).

The $b \equiv c \pmod 3$ is clearly preserved.
Also, examining the size,
\begin{itemize}
  \ii If $|c| > 2$,
  we have $\left\lvert \frac{c-a}{-5} \right\rvert
  \le \frac{|c|+9}{5} < |c|$.
  Thus, we eventually reach a pair with $|c| \le 2$.
  \ii Similarly, if $|b| > 9$,
  we have $\left\lvert \frac{b-a}{-2} \right\rvert
  \le \frac{|b|+9}{2} < |b|$,
  so we eventually reach a pair with $|b| \le 9$.
\end{itemize}
this leaves us with $5 \cdot 19 = 95$ ordered pairs to check
(though only about one third have $b \equiv c \pmod 3$).
This can be done by the following code:
\begin{lstlisting}[language=Python]
import functools
@functools.lru_cache()
def f(x0, y0):
  if x0 == 0 and y0 == 0:
    return 0
  if x0 % 2 == (y0 % 5) % 2:
    d = y0 % 5
  else:
    d = (y0 % 5) + 5

  x1 = (x0 - d) // (-2)
  y1 = (y0 - d) // (-5)

  return 1 + f(x1, y1)

for x in range(-9, 10):
for y in range(-2, 3):
  if (x % 3 == y % 3):
    print(f"({x:2d}, {y:2d}) finished in {f(x,y)} moves")
\end{lstlisting}
As this gives the output
\begin{lstlisting}
(-9,  0) finished in 5 moves
(-8, -2) finished in 5 moves
(-8,  1) finished in 5 moves
(-7, -1) finished in 5 moves
(-7,  2) finished in 5 moves
(-6,  0) finished in 3 moves
(-5, -2) finished in 3 moves
(-5,  1) finished in 3 moves
(-4, -1) finished in 3 moves
(-4,  2) finished in 3 moves
(-3,  0) finished in 3 moves
(-2, -2) finished in 3 moves
(-2,  1) finished in 3 moves
(-1, -1) finished in 3 moves
(-1,  2) finished in 3 moves
( 0,  0) finished in 0 moves
( 1, -2) finished in 2 moves
( 1,  1) finished in 1 moves
( 2, -1) finished in 2 moves
( 2,  2) finished in 1 moves
( 3,  0) finished in 2 moves
( 4, -2) finished in 2 moves
( 4,  1) finished in 2 moves
( 5, -1) finished in 2 moves
( 5,  2) finished in 2 moves
( 6,  0) finished in 4 moves
( 7, -2) finished in 4 moves
( 7,  1) finished in 4 moves
( 8, -1) finished in 4 moves
( 8,  2) finished in 4 moves
( 9,  0) finished in 4 moves
\end{lstlisting}
we are done.
\end{proof}
\pagebreak

\section{Solutions to Day 2}
\subsection{USAMO 1997/4}
\textsl{Available online at \url{https://aops.com/community/p343875}.}
\begin{mdframed}[style=mdpurplebox,frametitle={Problem statement}]
To clip a convex $n$-gon means to choose a pair of
consecutive sides $AB$, $BC$ and to replace them
by the three segments $AM$, $MN$, and $NC$,
where $M$ is the midpoint of $AB$ and $N$ is the midpoint of $BC$.
In other words, one cuts off the triangle $MBN$ to obtain a convex $(n+1)$-gon.
A regular hexagon $\mathcal{P}_6$ of area 1 is clipped to obtain a
heptagon $\mathcal{P}_7$.
Then $\mathcal{P}_7$ is clipped (in one of the seven possible ways)
to obtain an octagon $\mathcal{P}_8$, and so on.
Prove that no matter how the clippings are done,
the area of $\mathcal{P}_n$ is greater than $\frac 13$, for all $n \geq 6$.
\end{mdframed}
Call the original hexagon $ABCDEF$.
We show the area common to triangles $ACE$ and $BDF$ is in every
$\mathcal{P}_n$; this solves the problem since the area is $1/3$.

For every side of a clipped polygon,
we define its \emph{foundation} recursively as follows:
\begin{itemize}
  \ii $AB$, $BC$, $CD$, $DE$, $EF$, $FA$
  are each their own foundation
  (we also call these \emph{original sides}).
  \ii When a new clipped edge is added,
  its foundation is the union of the foundations
  of the two edges it touches.
\end{itemize}
Hence, any foundations are nonempty subsets
of original sides.

\begin{claim*}
  All foundations are in fact at most
  two-element sets of adjacent original sides.
\end{claim*}
\begin{proof}
  It's immediate by induction
  that any two adjacent sides have at most two elements
  in the union of their foundations, and if there are two,
  they are two adjacent original sides.
\end{proof}

Now, if a side has foundation contained in $\{AB,BC\}$, say,
then the side should be contained within triangle $ABC$.
Hence the side does not touch $AC$.
This proves the problem.
\pagebreak

\subsection{USAMO 1997/5}
\textsl{Available online at \url{https://aops.com/community/p2971}.}
\begin{mdframed}[style=mdpurplebox,frametitle={Problem statement}]
If $a,b,c > 0$ prove that
\[
  \frac{1}{a^3+b^3+abc}
  + \frac{1}{b^3+c^3+abc}
  + \frac{1}{c^3+a^3+abc}
  \le \frac{1}{abc}.
\]
\end{mdframed}
From $a^3 + b^3 \ge ab(a+b)$, the left-hand side becomes
\[
  \sum_{\text{cyc}} \frac{1}{a^3+b^3+abc}
  \le \sum_{\text{cyc}} \frac{1}{ab(a+b+c)}
  = \frac{1}{abc} \sum_{\text{cyc}} \frac{c}{a+b+c}
  = \frac{1}{abc}.
\]
\pagebreak

\subsection{USAMO 1997/6}
\textsl{Available online at \url{https://aops.com/community/p343876}.}
\begin{mdframed}[style=mdpurplebox,frametitle={Problem statement}]
Suppose the sequence of nonnegative integers
$a_1, a_2, \dots, a_{1997}$ satisfies
\[ a_i + a_j \leq a_{i+j} \leq a_i + a_j + 1  \]
for all $i,j \geq 1$ with $i + j \leq 1997$.
Show that there exists a real number $x$ such that
$a_n = \lfloor nx \rfloor$ for all $1 \leq n \leq 1997$.
\end{mdframed}
We are trying to show there exists an $x \in \RR$
such that
\[ \frac{a_n}{n} \le x < \frac{a_n+1}{n} \qquad \forall n. \]
This means we need to show
\[ \max_i \frac{a_i}{i} < \min_j \frac{a_j+1}{j}. \]
Replace 1997 by $N$.
We will prove this by induction,
but we will need some extra hypotheses on the indices $i,j$
which are used above.

\begin{claim*}
  Suppose that
  \begin{itemize}
    \ii Integers $a_1$, $a_2$, \dots, $a_N$ satisfy the given conditions.
    \ii Let $i = \opname{argmax}_n \frac{a_n}{n}$;
    if there are ties, pick the smallest $i$.
    \ii Let $j = \opname{argmin}_n \frac{a_n+1}{n}$;
    if there are ties, pick the smallest $j$.
  \end{itemize}
  Then \[ \frac{a_i}{i} < \frac{a_j+1}{j}. \]
  Moreover, these two fractions are in lowest terms,
  and are adjacent in the Farey sequence of order $\max(i,j)$.
\end{claim*}
\begin{proof}
  By induction on $N \ge 1$ with the base case clear.
  So suppose we have the induction hypothesis
  with numbers $a_1$, \dots, $a_{N-1}$,
  with $i$ and $j$ as promised.

  Now, consider the new number $a_N$.
  We have two cases:
  \begin{itemize}
    \ii Suppose $i+j > N$.
    Then, no fraction with denominator $N$
    can lie strictly inside the interval;
    so we may write for some integer $b$
    \[ \frac bN \le \frac{a_i}{i}
      < \frac{a_j+1}{j} \le \frac{b+1}{N}. \]
    For purely algebraic reasons we have
    \[ \frac{b-a_i}{N-i} \le \frac bN \le \frac{a_i}{i}
      < \frac{a_j+1}{j} \le \frac{b+1}{N}
      \le \frac{b-a_j}{N-j}. \]
    Now,
    \begin{align*}
      a_N &\ge a_i + a_{N-i}
        \ge a_i + (N-i) \cdot \frac{a_i}{i} \\
      &\ge a_i + (b-a_i) = b \\
      a_N &\le a_j + a_{N-j} + 1
        \le (a_j+1) + (N-j) \cdot \frac{a_j+1}{j} \\
      &= (a_j+1) + (b-a_j) = b+1.
    \end{align*}
    Thus $a_N \in \{b,b+1\}$.
    This proves that $\frac{a_N}{N} \le \frac{a_i}{i}$
    while $\frac{a_N+1}{N} \ge \frac{a_j+1}{j}$.
    Moreover, the pair $(i,j)$ does not change,
    so all inductive hypotheses carry over.

    \ii On the other hand, suppose $i+j = N$.
    Then we have
    \[ \frac{a_i}{i} < \frac{a_i + a_j + 1}{N} < \frac{a_j+1}{j}.  \]
    Now, we know $a_N$ could be either $a_i + a_j$ or $a_i + a_j + 1$.
    If it's the former, then $(i,j)$ becomes $(i,N)$.
    If it's the latter, then $(i,j)$ becomes $(N,j)$.
    The properties of Farey sequences ensure that
    the $\frac{a_i + a_j + 1}{N}$ is reduced, either way.
  \end{itemize}
\end{proof}
\pagebreak


\end{document}
