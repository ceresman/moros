% © Evan Chen
% Downloaded from https://web.evanchen.cc/

\documentclass[11pt]{scrartcl}
\usepackage[sexy]{evan}
\ihead{\footnotesize\textbf{\thetitle}}
\ohead{\footnotesize\href{http://web.evanchen.cc}{\ttfamily web.evanchen.cc},
    updated \today}
\title{JMO 2023 Solution Notes}
\date{\today}

\begin{document}

\maketitle

\begin{abstract}
This is a compilation of solutions
for the 2023 JMO.
Some of the solutions are my own work,
but many are from the official solutions provided by the organizers
(for which they hold any copyrights),
and others were found by users on the Art of Problem Solving forums.

These notes will tend to be a bit more advanced and terse than the ``official''
solutions from the organizers.
In particular, if a theorem or technique is not known to beginners
but is still considered ``standard'', then I often prefer to
use this theory anyways, rather than try to work around or conceal it.
For example, in geometry problems I typically use directed angles
without further comment, rather than awkwardly work around configuration issues.
Similarly, sentences like ``let $\mathbb{R}$ denote the set of real numbers''
are typically omitted entirely.

Corrections and comments are welcome!
\end{abstract}

\tableofcontents
\newpage

\addtocounter{section}{-1}
\section{Problems}
\begin{enumerate}[\bfseries 1.]
\item %% Problem 1
Find all triples of positive integers $(x,y,z)$ satisfying
\[ 2(x+y+z+2xyz)^2 = (2xy+2yz+2zx+1)^2 + 2023. \]

\item %% Problem 2
In an acute triangle $ABC$, let $M$ be the midpoint of $\ol{BC}$.
Let $P$ be the foot of the perpendicular from $C$ to $AM$.
Suppose that the circumcircle of triangle $ABP$
intersects line $BC$ at two distinct points $B$ and $Q$.
Let $N$ be the midpoint of $\ol{AQ}$.
Prove that $NB = NC$.

\item %% Problem 3
Consider an $n$-by-$n$ board of unit squares for some odd positive integer $n$.
We say that a collection $C$ of identical dominoes is a
maximal grid-aligned configuration on the board if $C$ consists of $(n^2-1)/2$
dominoes where each domino covers exactly two neighboring squares
and the dominoes don't overlap: $C$ then covers all but one square on the board.
We are allowed to slide (but not rotate) a domino on the board to
cover the uncovered square, resulting in a new maximal grid-aligned configuration
with another square uncovered. Let $k(C)$ be the number of distinct maximal
grid-aligned configurations obtainable from $C$ by repeatedly sliding dominoes.

Find the maximum possible value of $k(C)$ as a function of $n$.

\item %% Problem 4
Two players, Blake and Ruby, play the following game
on an infinite grid of unit squares, all initially colored white.
The players take turns starting with Blake.
On Blake's turn, Blake selects one white unit square and colors it blue.
On Ruby's turn, Ruby selects two white unit squares and colors them red.
The players alternate until Blake decides to end the game.
At this point, Blake gets a score, given by the number of unit squares in the
largest (in terms of area) simple polygon containing only blue unit squares.

What is the largest score Blake can guarantee?

\item %% Problem 5
Positive integers $a$ and $N$ are fixed,
and $N$ positive integers are written on a blackboard.
Alice and Bob play the following game.
On Alice's turn, she must replace some integer $n$ on the board with $n+a$,
and on Bob's turn he must replace some even integer $n$ on the board with $n/2$.
Alice goes first and they alternate turns.
If on his turn Bob has no valid moves, the game ends.

After analyzing the $N$ integers on the board, Bob realizes that,
regardless of what moves Alice makes,
he will be able to force the game to end eventually.
Show that, in fact, for this value of $a$ and these $N$ integers on the board,
the game is guaranteed to end regardless of Alice's or Bob's moves.

\item %% Problem 6
Isosceles triangle $ABC$, with $AB=AC$, is inscribed in circle $\omega$.
Let $D$ be an arbitrary point inside $BC$ such that $BD \neq DC$.
Ray $AD$ intersects $\omega$ again at $E$ (other than $A$).
Point $F$ (other than $E$) is chosen on $\omega$ such that $\angle DFE = 90\dg$.
Line $FE$ intersects rays $AB$ and $AC$ at points $X$ and $Y$, respectively.
Prove that $\angle XDE = \angle EDY$.

\end{enumerate}
\pagebreak

\section{Solutions to Day 1}
\subsection{JMO 2023/1, proposed by Titu Andreescu}
\textsl{Available online at \url{https://aops.com/community/p27349258}.}
\begin{mdframed}[style=mdpurplebox,frametitle={Problem statement}]
Find all triples of positive integers $(x,y,z)$ satisfying
\[ 2(x+y+z+2xyz)^2 = (2xy+2yz+2zx+1)^2 + 2023. \]
\end{mdframed}
Answer: $(3,3,2)$ and permutations.

The solution hinges upon the following claim:
\begin{claim*}
   The identity
   \[ 2(x+y+z+2xyz)^2 - (2xy+2yz+2zx+1)^2 = (2x^2-1)(2y^2-1)(2z^2-1) \]
   is true.
\end{claim*}
\begin{proof}
  This can be proved by manually expanding; we show where it ``came from''.
  In algebraic number theory, there is a \emph{norm function}
  $\opname{Norm} \colon \QQ(\sqrt2) \to \QQ$
  defined by
  \[ \opname{Norm}(a+b\sqrt2) = a^2-2b^2 \]
  which is \emph{multiplicative}, meaning
  \[ \opname{Norm}(u \cdot v) = \opname{Norm}(u) \cdot \opname{Norm}(v). \]

  This means that for any rational numbers $x$, $y$, $z$, we should have
  \begin{align*}
    &\phantom{=} \opname{Norm}
    \left( (1+\sqrt2x)(1+\sqrt2y)(1+\sqrt2z) \right) \\
    &= \opname{Norm}(1+\sqrt2x)
      \cdot \opname{Norm}(1+\sqrt2y)
      \cdot \opname{Norm}(1+\sqrt2z).
  \end{align*}
  But $(1+\sqrt2x)(1+\sqrt2y)(1+\sqrt2z) = (2xy+2yz+2zx+1) + (x+y+z+2xyz)\sqrt2$
  so the above equation is the negative of the desired identity.
\end{proof}

We are thus reduced to find positive integers $x$, $y$, $z$ satisfying
\[ (2x^2-1)(2y^2-1)(2z^2-1) = 2023 = 7 \cdot 17^2. \]
Each of the factors is a positive integer greater than $1$.
The only divisors of $2023$ of the form $2t^2-1$ are $1$, $7$, $17$.
This gives the answers claimed.
\pagebreak

\subsection{JMO 2023/2, proposed by Holden Mui}
\textsl{Available online at \url{https://aops.com/community/p27349297}.}
\begin{mdframed}[style=mdpurplebox,frametitle={Problem statement}]
In an acute triangle $ABC$, let $M$ be the midpoint of $\ol{BC}$.
Let $P$ be the foot of the perpendicular from $C$ to $AM$.
Suppose that the circumcircle of triangle $ABP$
intersects line $BC$ at two distinct points $B$ and $Q$.
Let $N$ be the midpoint of $\ol{AQ}$.
Prove that $NB = NC$.
\end{mdframed}
We show several different approaches.
In all solutions, let $D$ denote the foot of the altitude from $A$.

\begin{center}
\begin{asy}
size(10cm);

pair A = dir(115);
pair B = dir(210);
pair C = dir(330);
pair D = foot(A, B, C);
pair M = midpoint(B--C);
pair P = foot(C, A, M);
pair Q = 2*M-D;

filldraw(A--B--C--cycle, opacity(0.1)+cyan, blue);
filldraw(circumcircle(A, B, P), opacity(0.1)+yellow, orange+dashed);
draw(A--D, blue);
draw(A--P--C, deepgreen);

draw(A--Q, red+dashed);
pair N = midpoint(A--Q);
draw(B--N--C, red+dotted);

pair R = foot(B, A, M);
draw(B--R, dashed);

dot("$A$", A, dir(A));
dot("$B$", B, dir(B));
dot("$C$", C, dir(C));
dot("$D$", D, dir(D));
dot("$M$", M, dir(45));
dot("$P$", P, dir(P));
dot("$Q$", Q, dir(315));
dot("$N$", N, dir(45));
dot("$R$", R, dir(30));

/* -----------------------------------------------------------------+
|                 TSQX: by CJ Quines and Evan Chen                  |
| https://github.com/vEnhance/dotfiles/blob/main/py-scripts/tsqx.py |
+-------------------------------------------------------------------+
!size(10cm);
A = dir 115
B = dir 210
C = dir 330
D = foot A B C
M 45 = midpoint B--C
P = foot C A M
Q 315 = 2*M-D
A--B--C--cycle / 0.1 cyan / blue
circumcircle A B P / 0.1 yellow / orange dashed
A--D blue
A--P--C deepgreen
A--Q / red dashed
N 45 = midpoint A--Q
B--N--C / red dotted
R 30 = foot B A M
B--R / dashed
*/
\end{asy}
\end{center}

\paragraph{Most common synthetic approach.}
The solution hinges on the following claim:
\begin{claim*}
  $Q$ coincides with the reflection of $D$ across $M$.
\end{claim*}
\begin{proof}
  Note that $\dang ADC = \dang APC = 90\dg$, so $ADPC$ is cyclic.
  Then by power of a point (with the lengths directed),
  \[ MB \cdot MQ = MA \cdot MP = MC \cdot MD. \]
  Since $MB = MC$, the claim follows.
\end{proof}
It follows that $\ol{MN} \parallel \ol{AD}$,
as $M$ and $N$ are respectively the midpoints of $\ol{AQ}$ and $\ol{DQ}$.
Thus $\ol{MN} \perp \ol{BC}$,
and so $N$ lies on the perpendicular bisector of $\ol{BC}$, as needed.
\begin{remark*}
  [David Lin]
  One can prove the main claim without power of a point as well, as follows:
  Let $R$ be the foot from $B$ to $\ol{AM}$, so $BRCP$ is a parallelogram.
  Note that $ABDR$ is cyclic, and hence
  \[ \dang DRM = \dang DBA = QBA = \dang QPA = \dang QPM. \]
  Thus, $\ol{DR} \parallel \ol{PQ}$, so $DRQ$ is also a parallelogram.
\end{remark*}

\paragraph{Synthetic approach with no additional points at all.}
\begin{claim*}
  $\triangle BPC \sim \triangle ANM$ (oppositely oriented).
\end{claim*}
\begin{proof}
  We have $\triangle BMP \sim \triangle AMQ$ from the given concyclicity of $ABPQ$.
  Then
  \[ \frac{BM}{BP} = \frac{AM}{AQ} \implies
    \frac{2BM}{BP} = \frac{AM}{AQ/2} \implies
    \frac{BC}{BP} = \frac{AM}{AN} \]
  implying the similarity (since $\dang MAQ = \dang BPM$).
\end{proof}
This similarity gives us the equality of directed angles
\[ \dang \left( BC, MN \right) = -\dang \left( PC, AM  \right) = 90\dg \]
as desired.

\paragraph{Synthetic approach using only the point $R$.}
Again let $R$ be the foot from $B$ to $\ol{AM}$, so $BRCP$ is a parallelogram.
\begin{claim*}
  $ARQC$ is cyclic; equivalently, $\triangle MAQ \sim \triangle MCR$.
\end{claim*}
\begin{proof}
  $MR \cdot MA = MP \cdot MA = MB \cdot MQ = MC \cdot MQ$.
\end{proof}

Note that in $\triangle MCR$, the $M$-median is parallel to $\ol{CP}$
and hence perpendicular to $\ol{RM}$.
The same should be true in $\triangle MAQ$ by the similarity,
so $\ol{MN} \perp \ol{MQ}$ as needed.

\paragraph{Cartesian coordinates approach with power of a point.}
Suppose we set $B = (-1,0)$, $M = (0,0)$, $C = (1,0)$, and $A = (a,b)$.
One may compute:
\begin{align*}
  \overleftrightarrow{AM} : 0 &= bx - ay \iff y = \frac ba x \\
  \overleftrightarrow{CP} : 0 &= a(x-1) + by
    \iff y = -\frac ab (x-1) = -\frac ab x + \frac ab. \\
  P &= \left( \frac{a^2}{a^2+b^2}, \frac{ab}{a^2+b^2} \right) \\
\end{align*}
Now note that
\[ AM = \sqrt{a^2+b^2}, \qquad PM = \frac{a}{\sqrt{a^2+b^2}} \]
together with power of a point
\[ AM \cdot PM = BM \cdot QM \]
to immediately deduce that $Q = (a,0)$.
Hence $N = (0, b/2)$ and we're done.

\paragraph{Cartesian coordinates approach without power of a point (outline).}
After computing $A$ and $P$ as above, one could also directly calculate
\begin{align*}
  \text{Perpendicular bisector of $\ol{AB}$}:
    y &= -\frac{a+1}{b} x + \frac{a^2+b^2-1}{2b} \\
  \text{Perpendicular bisector of $\ol{PB}$}:
    y &= - \left( \frac{2a}{b} + \frac{b}{a} \right) x - \frac{b}{2a} \\
  \text{Perpendicular bisector of $\ol{PA}$}:
    y &= - \frac{a}{b} x + \frac{a+a^2+b^2}{2b}. \\
  \text{Circumcenter of $\triangle PAB$}
    &= \left( -\frac{a+1}{2}, \frac{2a^2+2a+b^2}{2b} \right).
\end{align*}
This is enough to extract the coordinates of $Q = (\bullet, 0)$,
because $B = (-1,0)$ is given, and the $x$-coordinate
of the circumcenter should be the average of the $x$-coordinates of $B$ and $Q$.
In other words, $Q = (-a,0)$.
Hence, $N = \left( 0, \frac b2 \right)$, as needed.

\paragraph{Ill-advised barycentric approach (outline).}
Use reference triangle $ABC$.
The $A$-median is parametrized by $(t:1:1)$ for $t \in \RR$.
So because of $\ol{CP} \perp \ol{AM}$, we are looking for $t$ such that
\[
  \left( \frac{t \vec A + \vec B + \vec C}{t+2} - \vec C \right)
  \perp \left( A - \frac{\vec B + \vec C}{2} \right).
\]
This is equivalent to
\[
  \left( t \vec A + \vec B - (t+1) \vec C \right)
  \perp \left( 2 \vec A - \vec B - \vec C \right).
\]
By the perpendicularity formula for barycentric coordinates (EGMO 7.16),
this is equivalent to
\begin{align*}
  0 &= a^2t - b^2 \cdot (3t+2) + c^2 \cdot (2-t) \\
  &= \left( a^2-3b^2-c^2 \right) t - 2(b^2-c^2) \\
  \implies t &= \frac{2(b^2-c^2)}{a^2-3b^2-c^2}.
\end{align*}
In other words,
\[ P = \left( 2(b^2-c^2) : a^2-3b^2-c^2 : a^2-3b^2-c^2 \right). \]
A long calculation gives
$a^2 y_P z_P + b^2 z_P x_P + c^2 x_P y_P
= (a^2-3b^2-c^2)(a^2-b^2+c^2)(a^2-2b^2-2c^2)$.
Together with $x_P+y_P+z_P=2a^2-4b^2-4c^2$,
this makes the equation of $(ABP)$ as
\[
  0=-a^2yz-b^2zx-c^2xy
  + \frac{a^2-b^2+c^2}{2} z(x+y+z).
\]
To solve for $Q$, set $x=0$ to get to get
\[ a^2yz = \frac{a^2-b^2+c^2}{2} z(y+z)
  \implies \frac yz = \frac{a^2-b^2+c^2}{a^2+b^2-c^2}. \]
In other words,
\[ Q = \left(  0 : a^2-b^2+c^2 : a^2+b^2-c^2 \right). \]
Taking the average with $A = (1,0,0)$ then gives
\[ N = \left(  2a^2 : a^2-b^2+c^2 : a^2+b^2-c^2 \right). \]
The equation for the perpendicular bisector of $\ol{BC}$
is given by (see EGMO 7.19)
\[ 0 = a^2(z-y)+x(c^2-b^2) \]
which contains $N$, as needed.

\paragraph{Extremely ill-advised complex numbers approaches (outline).}
Suppose we pick $a$, $b$, $c$ as the unit circle, and let $m = (b+c)/2$.
Using the fully general ``foot'' formula, one can get
\[
  p = \frac{(a-m) \ol c + (\ol a - \ol m)c
    + \ol a m - a \ol m}{2(\ol a - \ol m)}
    = \frac{a^2 b - a^2 c - a b^2 - 2 a b c - a c^2 + b^2 c + 3 b c^2}
    {4bc-2a(b+c)}
\]
Meanwhile, an extremely ugly calculation will eventually yield
\[ q = \frac{\frac{bc}{a}+b+c-a}{2} \]
so
\[ n = \frac{a+q}{2} = \frac{a+b+c+\frac{bc}{a}}{4}
  = \frac{(a+b)(a+c)}{2a}. \]
There are a few ways to then verify $NB = NC$.
The simplest seems to be to verify that
\[ \frac{n - \frac{b+c}{2}}{b-c} = \frac{a-b-c+\frac{bc}{a}}{4(b-c)}
  = \frac{(a-b)(a-c)}{2a(b-c)} \]
is pure imaginary, which is clear.
\pagebreak

\subsection{JMO 2023/3, proposed by Holden Mui}
\textsl{Available online at \url{https://aops.com/community/p27349423}.}
\begin{mdframed}[style=mdpurplebox,frametitle={Problem statement}]
Consider an $n$-by-$n$ board of unit squares for some odd positive integer $n$.
We say that a collection $C$ of identical dominoes is a
maximal grid-aligned configuration on the board if $C$ consists of $(n^2-1)/2$
dominoes where each domino covers exactly two neighboring squares
and the dominoes don't overlap: $C$ then covers all but one square on the board.
We are allowed to slide (but not rotate) a domino on the board to
cover the uncovered square, resulting in a new maximal grid-aligned configuration
with another square uncovered. Let $k(C)$ be the number of distinct maximal
grid-aligned configurations obtainable from $C$ by repeatedly sliding dominoes.

Find the maximum possible value of $k(C)$ as a function of $n$.
\end{mdframed}
The answer is that
\[ k(C) \le \left( \frac{n+1}{2} \right)^2. \]
\begin{remark*}
  [Comparison with USAMO version]
  In the USAMO version of the problem,
  students instead are asked to find all possible values of $k(C)$.
  The answer is
  $k(C) \in \left\{ 1, 2, \dots, \left( \frac{n-1}{2} \right)^2 \right\}
  \cup \left\{ \left( \frac{n+1}{2} \right)^2 \right\}$.
\end{remark*}

Index the squares by coordinates $(x,y) \in \{1,2,\dots,n\}^2$.
We say a square is \emph{special} if it is empty or
it has the same parity in both coordinates as the empty square.

Construct a directed graph $G = G(C)$ whose vertices are special squares as follows:
for each domino on a special square $s$, we draw a directed edge
from $s$ to the special square that domino points to, if any.
(If the special square has both odd coordinates,
all special squares have an outgoing edge except the empty cell.
In the even-even case, some arrows may point ``off the board''
and not be drawn.)

\begin{center}
\begin{asy}
size(6cm);

pen border = grey+1.5;

filldraw(box((0, 1), (1, 3)), lightgreen, border);
filldraw(box((0, 3), (2, 4)), lightcyan, border);
filldraw(box((0, 4), (2, 5)), lightgreen, border);
filldraw(box((1, 0), (2, 3)), lightred, border);
filldraw(box((1, 2), (3, 3)), lightblue, border);
filldraw(box((2, 3), (3, 5)), palered, border);
filldraw(box((2, 0), (4, 1)), lightcyan, border);
filldraw(box((2, 1), (4, 2)), lightgreen, border);
filldraw(box((4, 0), (5, 2)), lightred, border);
filldraw(box((3, 3), (5, 4)), lightblue, border);
filldraw(box((3, 2), (5, 3)), palered, border);
filldraw(box((3, 4), (5, 5)), lightcyan, border);
filldraw(box((0, 0), (1, 1)), grey, border);

pair A = (0.5, 4.5);
pair B = (2.5, 4.5);
pair C = (4.5, 4.5);
pair D = (0.5, 2.5);
pair E = (2.5, 2.5);
pair F = (4.5, 2.5);
pair X = (0.5, 0.5);
pair Y = (2.5, 0.5);
pair Z = (4.5, 0.5);

dotfactor *= 2;
dot(A); dot(B); dot(C);
dot(D); dot(E); dot(F);
dot(X); dot(Y); dot(Z);

draw(A--B, EndArrow, Margin(3,3));
draw(C--B, EndArrow, Margin(3,3));
draw(B--E, EndArrow, Margin(3,3));
draw(E--D, EndArrow, Margin(3,3));
draw(F--E, EndArrow, Margin(3,3));
draw(Z--F, EndArrow, Margin(3,3));
draw(Y--Z, EndArrow, Margin(3,3));
draw(D--X, EndArrow, Margin(3,3));
\end{asy}
\end{center}

\begin{claim*}
  Any undirected connected component of $G$ is acyclic unless the cycle
  contains the empty square inside it.
\end{claim*}
\begin{proof}
  Consider a cycle of $G$; we are going to prove
  that the number of chessboard cells enclosed is always odd.

  This can be proven directly by induction,
  but for theatrical effect, we use Pick's theorem.
  Mark the center of every chessboard cell on or inside the cycle
  to get a lattice.
  The dominoes of the cycle then enclose a polyominoe which
  actually consists of $2 \times 2$ squares,
  meaning its area is a multiple of $4$.
  \begin{center}
  \begin{asy}
    unitsize(0.9cm);
    filldraw(box((-0.5,1.5), (0.5,3.5)), paleyellow, grey+1.4);
    filldraw(box((-0.5,3.5), (1.5,4.5)), paleyellow, grey+1.4);
    filldraw(box((1.5,3.5), (3.5,4.5)), paleyellow, grey+1.4);
    filldraw(box((3.5,2.5), (4.5,4.5)), paleyellow, grey+1.4);
    filldraw(box((3.5,0.5), (4.5,2.5)), paleyellow, grey+1.4);
    filldraw(box((2.5,-0.5), (4.5,0.5)), paleyellow, grey+1.4);
    filldraw(box((1.5,-0.5), (2.5,1.5)), paleyellow, grey+1.4);
    filldraw(box((0.5,1.5), (2.5,2.5)), paleyellow, grey+1.4);
    draw((0,2)--(0,4), blue+1.2, EndArrow(TeXHead), Margins);
    draw((0,4)--(2,4), blue+1.2, EndArrow(TeXHead), Margins);
    draw((2,4)--(4,4), blue+1.2, EndArrow(TeXHead), Margins);
    draw((4,4)--(4,2), blue+1.2, EndArrow(TeXHead), Margins);
    draw((4,2)--(4,0), blue+1.2, EndArrow(TeXHead), Margins);
    draw((4,0)--(2,0), blue+1.2, EndArrow(TeXHead), Margins);
    draw((2,0)--(2,2), blue+1.2, EndArrow(TeXHead), Margins);
    draw((2,2)--(0,2), blue+1.2, EndArrow(TeXHead), Margins);
    draw((2,2)--(2,4), blue+dashed, Margins);
    draw((2,2)--(4,2), blue+dashed, Margins);
    dotfactor *= 1.8;
    dot((2,0), blue);
    dot((4,0), blue);
    dot((0,2), blue);
    dot((2,2), blue);
    dot((4,2), blue);
    dot((0,4), blue);
    dot((2,4), blue);
    dot((4,4), blue);
    dot((0,3), black);
    dot((1,4), black);
    dot((3,4), black);
    dot((4,3), black);
    dot((4,1), black);
    dot((3,0), black);
    dot((2,1), black);
    dot((1,2), black);
    dot((1,3), red);
    dot((2,3), red);
    dot((3,3), red);
    dot((3,2), red);
    dot((3,1), red);
  \end{asy}
  \end{center}
  Hence $B/2+I-1$ is a multiple of $4$, in the notation of Pick's theorem.
  As $B$ is twice the number of dominoes,
  and a parity argument on the special squares shows that number is even,
  it follows that $B$ is also a multiple of $4$
  (these correspond to blue and black in the figure above).
  This means $I$ is odd (the red dots in the figure above), as desired.
\end{proof}

Consider the connected component $T$ of the graph containing the empty square;
it's acyclic, so it's a tree.
Notice that all the arrows along $T$ point towards the empty cell,
and moving a domino corresponds to flipping an arrow.
Therefore:
\begin{claim*}
$k(C)$ is exactly the number of vertices of $T$.
\end{claim*}
\begin{proof}
Starting with the underlying tree,
the set of possible graphs is described by picking one vertex to be the sink
(the empty cell) and then directing all arrows towards it.
\end{proof}

This implies that $k(C) \le \left( \frac{n+1}{2} \right)^2$,
the total number of vertices of $G$
(this could only occur if the special squares are odd-odd, not even-even).
Equality is achieved as long as $T$ is a spanning tree;
one example of a way to achieve this is using the snake configuration below.

\begin{center}
\begin{asy}
unitsize(0.9cm);
dotfactor *= 2;
pen border = grey+1.5;
pair A = (0.5, 4.5);
pair B = (2.5, 4.5);
pair C = (4.5, 4.5);
pair D = (0.5, 2.5);
pair E = (2.5, 2.5);
pair F = (4.5, 2.5);
pair X = (0.5, 0.5);
pair Y = (2.5, 0.5);
pair Z = (4.5, 0.5);

filldraw(box((3, 4), (5, 5)), palecyan, border);
filldraw(box((1, 4), (3, 5)), palecyan, border);
filldraw(box((0, 3), (1, 5)), palecyan, border);
filldraw(box((0, 2), (2, 3)), palecyan, border);
filldraw(box((2, 2), (4, 3)), palecyan, border);
filldraw(box((4, 1), (5, 3)), palecyan, border);
filldraw(box((3, 0), (5, 1)), palecyan, border);
filldraw(box((1, 0), (3, 1)), palecyan, border);
fill(unitsquare, mediumgrey);
filldraw(box((3, 3), (5, 4)), lightred, border);
filldraw(box((1, 3), (3, 4)), lightred, border);
filldraw(box((0, 1), (2, 2)), lightred, border);
filldraw(box((2, 1), (4, 2)), lightred, border);
dot(A); dot(B); dot(C);
dot(D); dot(E); dot(F);
dot(X); dot(Y); dot(Z);

draw(C--B, EndArrow, Margin(3,3));
draw(B--A, EndArrow, Margin(3,3));
draw(A--D, EndArrow, Margin(3,3));
draw(D--E, EndArrow, Margin(3,3));
draw(E--F, EndArrow, Margin(3,3));
draw(F--Z, EndArrow, Margin(3,3));
draw(Z--Y, EndArrow, Margin(3,3));
draw(Y--X, EndArrow, Margin(3,3));
\end{asy}
\end{center}

\begin{remark*}
  In Russia 1997/11.8 it's shown that as long as the missing square is a corner,
  we have $G = T$.
  The proof is given implicitly from our work here:
  when the empty cell is in a corner, it cannot be surrounded,
  ergo $G$ has no cycles at all.
  Since it has one fewer edge than vertex, it's a tree.
\end{remark*}
\pagebreak

\section{Solutions to Day 2}
\subsection{JMO 2023/4, proposed by David Torres}
\textsl{Available online at \url{https://aops.com/community/p27349414}.}
\begin{mdframed}[style=mdpurplebox,frametitle={Problem statement}]
Two players, Blake and Ruby, play the following game
on an infinite grid of unit squares, all initially colored white.
The players take turns starting with Blake.
On Blake's turn, Blake selects one white unit square and colors it blue.
On Ruby's turn, Ruby selects two white unit squares and colors them red.
The players alternate until Blake decides to end the game.
At this point, Blake gets a score, given by the number of unit squares in the
largest (in terms of area) simple polygon containing only blue unit squares.

What is the largest score Blake can guarantee?
\end{mdframed}
The answer is $4$ squares.

\paragraph{Algorithm for Blake to obtain at least $4$ squares.}
We simply let Blake start with any cell blue,
then always draw adjacent to a previously drawn blue cell
until this is no longer possible.

Note that for $n \le 3$, any connected region of $n$ blue cells
has more than $2n$ liberties (non-blue cells adjacent to a blue cell);
up to translation, rotation, and reflection,
all the cases are shown in the figure below
with liberties being denoted by circles.

\begin{center}
\begin{asy}
  size(8cm);
  path c = circle( (0.5,0.5), 0.4 );
  defaultpen(linewidth(1.5));
  filldraw(shift(0,5)*unitsquare, palecyan, blue);
  draw(shift(-1,5)*c, deepgreen);
  draw(shift(1,5)*c, deepgreen);
  draw(shift(0,4)*c, deepgreen);
  draw(shift(0,6)*c, deepgreen);

  filldraw(shift(5,5)*unitsquare, palecyan, blue);
  filldraw(shift(6,5)*unitsquare, palecyan, blue);
  draw(shift(4,5)*c, deepgreen);
  draw(shift(5,6)*c, deepgreen);
  draw(shift(6,6)*c, deepgreen);
  draw(shift(7,5)*c, deepgreen);
  draw(shift(5,4)*c, deepgreen);
  draw(shift(6,4)*c, deepgreen);

  filldraw(shift(0,1)*unitsquare, palecyan, blue);
  filldraw(shift(1,0)*unitsquare, palecyan, blue);
  filldraw(shift(1,1)*unitsquare, palecyan, blue);
  draw(shift(0,0)*c, deepgreen);
  draw(shift(-1,1)*c, deepgreen);
  draw(shift(1,-1)*c, deepgreen);
  draw(shift(2,0)*c, deepgreen);
  draw(shift(2,1)*c, deepgreen);
  draw(shift(1,2)*c, deepgreen);
  draw(shift(0,2)*c, deepgreen);

  filldraw(shift(5,0)*unitsquare, palecyan, blue);
  filldraw(shift(6,0)*unitsquare, palecyan, blue);
  filldraw(shift(7,0)*unitsquare, palecyan, blue);

  draw(shift(5,1)*c, deepgreen);
  draw(shift(6,1)*c, deepgreen);
  draw(shift(7,1)*c, deepgreen);
  draw(shift(5,-1)*c, deepgreen);
  draw(shift(6,-1)*c, deepgreen);
  draw(shift(7,-1)*c, deepgreen);
  draw(shift(4,0)*c, deepgreen);
  draw(shift(8,0)*c, deepgreen);
\end{asy}
\end{center}

So as long as $n \le 3$, it's impossible that Ruby has blocked every liberty,
since Ruby has colored exactly $2n$ cells red.
Therefore, this algorithm could only terminate once $n \ge 4$.

\paragraph{Algorithm for Ruby to prevent more than $4$ squares.}
Divide the entire grid into $2 \times 2$ squares, which we call \emph{windows}.
Any time Blake makes a move in a cell $c$,
let Ruby mark any orthogonal neighbors of $c$ in its window;
then place any leftover red cells arbitrarily.

\begin{claim*}
  It's impossible for any window to
  contain two orthogonally adjacent blue cells.
\end{claim*}
\begin{proof}
  By construction:
  if there were somehow two adjacent blue cells in the same window,
  whichever one was played first should have caused red cells to be added.
\end{proof}

We show this gives the upper bound of $4$ squares.
Consider a blue cell $w$, and assume WLOG it is in the southeast corner
of a window. Label squares $x$, $y$, $z$ as shown below.
\begin{center}
\begin{asy}
  size(5cm);
  for (int i=0; i<3; ++i) {
    draw((0,2*i+1)--(6,2*i+1), black+1.5 );
    draw((2*i+1,0)--(2*i+1,6), black+1.5 );
  }
  for (int i=1; i<3; ++i) {
    draw((0,2*i)--(6,2*i), grey+dashed );
    draw((2*i,0)--(2*i,6), grey+dashed );
  }
  filldraw(box((2.1,3.1),(2.9,3.9)), palecyan, blue+1.5);
  draw((1.2,3.2)--(1.8,3.8), red+2.5);
  draw((1.8,3.2)--(1.2,3.8), red+2.5);
  draw((2.2,4.2)--(2.8,4.8), red+2.5);
  draw((2.8,4.2)--(2.2,4.8), red+2.5);
  label("$w$", (2.5, 3.5));
  label("$x$", (3.5, 3.5));
  label("$y$", (2.5, 2.5));
  label("$z$", (3.5, 2.5));
\end{asy}
\end{center}
Note that by construction, the blue polygon cannot leave the square
$\{w,x,y,z\}$, since whenever one of these four cells is blue,
its neighbours outside that square are guaranteed to be red.
This implies the bound.

\begin{remark*}
  [For Tetris fans]
  Here is a comedic alternative finish after proving the claim.
  Consider the possible tetrominoes
  (using the notation of
  \url{https://en.wikipedia.org/wiki/Tetromino#One-sided_tetrominoes}).
  We claim that only the square (\texttt{O}) is obtainable; as
  \begin{itemize}
    \ii \texttt{T}, \texttt{J}/\texttt{L}, and \texttt{I}
    all have three cells in a row, so they can't occur;
    \ii \texttt{S} and \texttt{Z} can't occur either;
    if the bottom row of an \texttt{S} crossed a window boundary,
    then the top row doesn't for example.
  \end{itemize}
  Moreover, the only way a blue \texttt{O} could be obtained is if each
  of it cells is in a different window. In that case,
  no additional blue cells can be added: it's fully surrounded by red.

  Finally, for any $k$-omino with $k > 4$,
  one can find a tetromino as a subset.
  (Proof: take the orthogonal adjacency graph of the $k$-omino,
  choose a spanning tree, and delete leaves from the tree
  until there are only four vertices left.)
\end{remark*}


\begin{remark*}
  [Common wrong approach]
  Suppose Ruby employs the following algorithm whenever Blake places a square $x$.
  If either the north and west neighbors of $x$ are unoccupied,
  place red squares on both of them.
  With any leftover red squares, place them at other neighbors of $x$ if possible.
  Finally, place any other red squares arbitrarily.
  (Another variant, the one Evan originally came up with,
  is to place east if possible when west is occupied,
  place south if possible when north is occupied,
  and then place any remaining red squares arbitrarily.)

  As written, this strategy \emph{does not work}.
  The reason is that one can end up in the following situation
  (imagine the blue square in the center is played first;
  moves for Ruby are drawn as red X's):
  \begin{center}
  \begin{asy}
    size(4cm);
    filldraw(unitsquare, palecyan, blue+1.5);
    filldraw(shift(1,1)*unitsquare, palecyan, blue+1.5);
    filldraw(shift(2,2)*unitsquare, palecyan, blue+1.5);
    label("$1$", (1.5,1.5), blue);
    label("$2$", (2.5,2.5), blue);
    label("$3$", (0.5,0.5), blue);
    picture redx;
    real r = 0.2;
    draw(redx, (r,r)--(1-r,1-r), red+2.5);
    draw(redx, (r,1-r)--(1-r,r), red+2.5);
    for (int i=0; i<=3; ++i) {
      add(shift(i-1,i)*redx);
    }
    add(shift(3,2)*redx);
    add(shift(0,-1)*redx);
  \end{asy}
  \end{center}
  In order to prevent Blake from winning, Ruby would need to begin
  playing moves not adjacent to Blake's most recent move.

  Thus in order for this solution to be made correct,
  one needs a careful algorithm for how Ruby should play
  when the north and west neighbors are not available.
  As far as I am aware, there are some specifications that work
  (and some that don't), but every working algorithm I have seen
  seems to involve some amount of casework.

  It is even more difficult to come up with a solution involving
  playing on just ``some'' two neighbors of recently added blue squares
  without the ``prefer north and west'' idea.
\end{remark*}
\pagebreak

\subsection{JMO 2023/5, proposed by Carl Schildkraut}
\textsl{Available online at \url{https://aops.com/community/p27349336}.}
\begin{mdframed}[style=mdpurplebox,frametitle={Problem statement}]
Positive integers $a$ and $N$ are fixed,
and $N$ positive integers are written on a blackboard.
Alice and Bob play the following game.
On Alice's turn, she must replace some integer $n$ on the board with $n+a$,
and on Bob's turn he must replace some even integer $n$ on the board with $n/2$.
Alice goes first and they alternate turns.
If on his turn Bob has no valid moves, the game ends.

After analyzing the $N$ integers on the board, Bob realizes that,
regardless of what moves Alice makes,
he will be able to force the game to end eventually.
Show that, in fact, for this value of $a$ and these $N$ integers on the board,
the game is guaranteed to end regardless of Alice's or Bob's moves.
\end{mdframed}
For $N=1$, there is nothing to prove.
We address $N \ge 2$ only henceforth.
Let $S$ denote the numbers on the board.

\begin{claim*}
  When $N \ge 2$, if $\nu_2(x) < \nu_2(a)$ for all $x \in S$,
  the game must terminate no matter what either player does.
\end{claim*}
\begin{proof}
  The $\nu_2$ of a number is unchanged by Alice's move
  and decreases by one on Bob's move.
  The game ends when every $\nu_2$ is zero.

  Hence, in fact the game will always terminate in exactly
  $\sum_{x \in S} \nu_2(x)$ moves in this case,
  regardless of what either player does.
\end{proof}

\begin{claim*}
  When $N \ge 2$, if there exists a number $x$ on the board such that
  $\nu_2(x) \ge \nu_2(a)$, then Alice can cause the game to go on forever.
\end{claim*}
\begin{proof}
  Denote by $x$ the first entry of the board (its value changes over time).
  Then Alice's strategy is to:
  \begin{itemize}
    \ii Operate on the first entry if $\nu_2(x) = \nu_2(a)$
    (the new entry thus has $\nu_2(x+a) > \nu_2(a)$);
    \ii Operate on any other entry besides the first one, otherwise.
  \end{itemize}
  A double induction then shows that
  \begin{itemize}
    \ii Just before each of Bob's turns, $\nu_2(x) > \nu_2(a)$ always holds; and
    \ii After each of Bob's turns, $\nu_2(x) \ge \nu_2(a)$ always holds.
  \end{itemize}
  In particular Bob will never run out of legal moves,
  since halving $x$ is always legal.
\end{proof}
\pagebreak

\subsection{JMO 2023/6, proposed by Anton Trygub}
\textsl{Available online at \url{https://aops.com/community/p27349508}.}
\begin{mdframed}[style=mdpurplebox,frametitle={Problem statement}]
Isosceles triangle $ABC$, with $AB=AC$, is inscribed in circle $\omega$.
Let $D$ be an arbitrary point inside $BC$ such that $BD \neq DC$.
Ray $AD$ intersects $\omega$ again at $E$ (other than $A$).
Point $F$ (other than $E$) is chosen on $\omega$ such that $\angle DFE = 90\dg$.
Line $FE$ intersects rays $AB$ and $AC$ at points $X$ and $Y$, respectively.
Prove that $\angle XDE = \angle EDY$.
\end{mdframed}
We present three solutions.

\paragraph{Angle chasing solution.}
Note that $(BDA)$ and $(CDA)$ are congruent,
since $BA=CA$ and $\angle BDA + \angle CDA = 180\dg$.
So these two circles are reflections around line $ED$.
Moreover, $(DEF)$ is obviously also symmetric around line $ED$.
\begin{center}
\begin{asy}
size(10cm);

pair E = dir(100);
pair B = dir(160);
pair C = dir(0);
pair A = dir(260);
draw(unitcircle, grey);

draw(E--B--C--cycle, dashed+grey);
pair D = extension(E, A, B, C);
pair W = -E;
pair F = foot(E, W, D);
pair X = extension(A, B, E, F);
pair Y = extension(A, C, E, F);
draw(E--A, blue);
draw(X--Y--A--cycle, dashed+grey);
draw(X--D--Y, red);

pair X_prime = foot(E, D, X);
pair Y_prime = foot(E, D, Y);
filldraw(circumcircle(E, X_prime, Y_prime), opacity(0.1)+cyan, blue);
filldraw(circumcircle(B, D, A), opacity(0.1)+cyan, blue);
filldraw(circumcircle(C, D, A), opacity(0.1)+cyan, blue);

dot("$E$", E, dir(E));
dot("$B$", B, dir(160));
dot("$C$", C, dir(340));
dot("$A$", A, dir(A));
dot("$D$", D, dir(315));
dot("$F$", F, dir(F));
dot("$X$", X, dir(X));
dot("$Y$", Y, dir(Y));
dot(X_prime);
dot(Y_prime);

clip(scale(1.7)*unitcircle);
dot("$Y$", D+2*dir(Y-D), dir(Y-D));

/* -----------------------------------------------------------------+
|                 TSQX: by CJ Quines and Evan Chen                  |
| https://github.com/vEnhance/dotfiles/blob/main/py-scripts/tsqx.py |
+-------------------------------------------------------------------+
!size(14cm);
E = dir 100
B 160 = dir 160
C 340 = dir 0
A = dir 260
unitcircle / grey
E--B--C--cycle / dashed grey
D 315 = extension E A B C
W := -E
F = foot E W D
X = extension A B E F
Y = extension A C E F
E--A blue
X--Y--A--cycle / dashed grey
X--D--Y / red
X' 70 .= foot E D X
Y' 130 .= foot E D Y
circumcircle E X' Y' / 0.1 cyan / blue
circumcircle B D A / 0.1 cyan / blue
circumcircle C D A / 0.1 cyan / blue
*/
\end{asy}
\end{center}
Hence, the radical axis of $(BDA)$ and $(DEF)$,
and the radical axis of $(CDA)$ and $(DEF)$,
should be symmetric about line $DE$.
But these radical axii are exactly lines $XD$ and $YD$, so we're done.

\begin{remark*}
  [Motivation]
  The main idea is that you can replace $DX$ and $DY$ with the radical axii,
  letting $X'$ and $Y'$ be the second intersections of the blue circles.
  Then for the problem to be true, you'd need $X'$ and $Y'$ to be reflections.
  That's equivalent to $(BDA)$ and $(CDA)$ being congruent;
  you check it and it's indeed true.
\end{remark*}

\paragraph{Harmonic solution (mine).}
Let $T$ be the point on line $\ol{XFEY}$ such that $\angle EDT = 90\dg$,
and let $\ol{AT}$ meet $\omega$ again at $K$.
Then
\[ TD^2 = TF \cdot TE = TK \cdot TA \implies \angle DKT = 90\dg \]
so line $DK$ passes through the antipode $M$ of $A$.
\begin{center}
\begin{asy}
size(14cm);

pair E = dir(110);
pair B = dir(160);
pair C = dir(20);
pair A = dir(270);
filldraw(unitcircle, opacity(0.1)+lightcyan, blue);

filldraw(E--B--C--cycle, opacity(0.1)+yellow, blue);
pair D = extension(E, A, B, C);
pair W = -E;
pair F = foot(E, W, D);
pair X = extension(A, B, E, F);
pair Y = extension(A, C, E, F);
draw(E--A, blue);
draw(X--Y--A--cycle, blue);
draw(X--D--Y, deepgreen);
pair M = dir(90);
pair T = extension(E, F, D, D+M-E);
draw(X--T, blue);
draw(D--T, red);
draw(E--M, red);
pair K = extension(D, M, T, A);
draw(T--A, red);
draw(K--M, red);

dot("$E$", E, dir(E));
dot("$B$", B, dir(200));
dot("$C$", C, dir(340));
dot("$A$", A, dir(A));
dot("$D$", D, dir(315));
dot("$F$", F, dir(F));
dot("$X$", X, dir(X));
dot("$Y$", Y, dir(Y));
dot("$M$", M, dir(35));
dot("$T$", T, dir(T));
dot("$K$", K, dir(K));

/* -----------------------------------------------------------------+
|                 TSQX: by CJ Quines and Evan Chen                  |
| https://github.com/vEnhance/dotfiles/blob/main/py-scripts/tsqx.py |
+-------------------------------------------------------------------+
!size(14cm);
E = dir 110
B 200 = dir 160
C 340 = dir 20
A = dir 270
unitcircle / 0.1 lightcyan / blue
E--B--C--cycle / 0.1 yellow / blue
D 315 = extension E A B C
W := -E
F = foot E W D
X = extension A B E F
Y = extension A C E F
E--A blue
X--Y--A--cycle / blue
X--D--Y / deepgreen
M 35 = dir 90
T = extension E F D D+M-E
X--T / blue
D--T / red
E--M / red
K = extension D M T A
T--A / red
K--M / red
*/
\end{asy}
\end{center}

Thus,
\[ -1 = (AM;CB)_\omega \overset{D}{=} (EK;BC)_\omega \overset{A}{=} (TE;XY) \]
and since $\angle EDT = 90\dg$ we're done.

\begin{remark*}
  [Motivation]
  The idea is to kill the points $X$ and $Y$
  by reinterpreting the desired condition as $(TD;XY)=-1$
  and then projecting through $A$ onto $\omega$.
  This eliminates points $X$ and $Y$ altogether
  and reduces the problem to showing that $\ol{TA}$ passes
  through the harmonic conjugate of $E$ with respect to $BC$ on $\omega$.

  The labels on the diagram are slightly misleading in that
  $\triangle EBC$ should probably be thought of as the ``reference'' triangle.
\end{remark*}

\paragraph{Pascal solution (Zuming Feng).}
Extend ray $FD$ to the antipode $T$ of $E$ on $\omega$. Then,
\begin{itemize}
\ii By Pascal's theorem on $EFTABC$,
the points $X$, $D$, and $P \coloneqq \ol{EC} \cap \ol{AT}$ are collinear.
\ii Similarly by Pascal's theorem on $EFTACB$, the points
the points $Y$, $D$, and $Q \coloneqq \ol{EB} \cap \ol{AT}$ are collinear.
\end{itemize}

\begin{center}
\begin{asy}
size(14cm);
pair E = dir(100);
pair B = dir(160);
pair C = dir(0);
pair A = dir(260);
filldraw(unitcircle, opacity(0.1)+yellow, blue);

draw(E--B--C--cycle, dashed+grey);
pair D = extension(E, A, B, C);
pair W = -E;
pair F = foot(E, W, D);
pair X = extension(A, B, E, F);
pair Y = extension(A, C, E, F);
draw(E--A, blue);
draw(X--Y--A--cycle, dashed+grey);
draw(X--D--Y, red);

pair T = -E;
pair P = extension(C, E, T, A);
pair Q = extension(B, E, T, A);

draw(P--D--Q, red);
draw(P--Q--E--cycle, deepgreen);
draw(F--T, blue);
filldraw(circumcircle(E, D, F), opacity(0.1)+cyan, blue);

dot("$E$", E, dir(E));
dot("$B$", B, dir(160));
dot("$C$", C, dir(0));
dot("$A$", A, dir(A));
dot("$D$", D, dir(315));
dot("$F$", F, dir(F));
dot("$X$", X, dir(X));
dot("$Y$", Y, dir(Y));
dot("$T$", T, dir(T));
dot("$P$", P, dir(P));
dot("$Q$", Q, dir(Q));

/* -----------------------------------------------------------------+
|                 TSQX: by CJ Quines and Evan Chen                  |
| https://github.com/vEnhance/dotfiles/blob/main/py-scripts/tsqx.py |
+-------------------------------------------------------------------+
!size(14cm);
E = dir 100
B 160 = dir 160
C 0 = dir 0
A = dir 260
unitcircle / 0.1 yellow / blue
E--B--C--cycle / dashed grey
D 315 = extension E A B C
W := -E
F = foot E W D
X = extension A B E F
Y = extension A C E F
E--A blue
X--Y--A--cycle / dashed grey
X--D--Y / red
T = -E
P = extension C E T A
Q = extension B E T A
P--D--Q / red
P--Q--E--cycle / deepgreen
F--T / blue
circumcircle E D F / 0.1 cyan / blue
*/

\end{asy}
\end{center}

Now it suffices to prove $\ol{ED}$ bisects $\angle QDP$.
However, $\ol{ED}$ is the angle bisector of $\angle QEP = \angle BEC$,
but also $\ol{EA} \perp \ol{QP}$.
Thus triangle $QEP$ is isosceles with $QE=PE$, and $\ol{EA}$ cuts it in half.
Since $D$ is on $\ol{EA}$, the result follows now.
\pagebreak


\end{document}
