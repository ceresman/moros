% © Evan Chen
% Downloaded from https://web.evanchen.cc/

\documentclass[11pt]{scrartcl}
\usepackage[sexy]{evan}
\ihead{\footnotesize\textbf{\thetitle}}
\ohead{\footnotesize\href{http://web.evanchen.cc}{\ttfamily web.evanchen.cc},
    updated \today}
\title{USAMO 2011 Solution Notes}
\date{\today}

\begin{document}

\maketitle

\begin{abstract}
This is a compilation of solutions
for the 2011 USAMO.
Some of the solutions are my own work,
but many are from the official solutions provided by the organizers
(for which they hold any copyrights),
and others were found by users on the Art of Problem Solving forums.

These notes will tend to be a bit more advanced and terse than the ``official''
solutions from the organizers.
In particular, if a theorem or technique is not known to beginners
but is still considered ``standard'', then I often prefer to
use this theory anyways, rather than try to work around or conceal it.
For example, in geometry problems I typically use directed angles
without further comment, rather than awkwardly work around configuration issues.
Similarly, sentences like ``let $\mathbb{R}$ denote the set of real numbers''
are typically omitted entirely.

Corrections and comments are welcome!
\end{abstract}

\tableofcontents
\newpage

\addtocounter{section}{-1}
\section{Problems}
\begin{enumerate}[\bfseries 1.]
\item %% Problem 1
Let $a$, $b$, $c$ be positive real numbers
such that $a^2+b^2+c^2+(a+b+c)^2 \le 4$. Prove that
\[ \frac{ab+1}{(a+b)^2}
  + \frac{bc+1}{(b+c)^2}
  + \frac{ca+1}{(c+a)^2} \ge 3. \]

\item %% Problem 2
An integer is assigned to each vertex of a regular pentagon
so that the sum of the five integers is $2011$.
A turn of a solitaire game consists of subtracting an integer $m$
(not necessarily positive) from each of the integers at two neighboring vertices
and adding $2m$ to the opposite vertex, which is not adjacent
to either of the first two vertices.
(The amount $m$ and the vertices chosen can vary from turn to turn.)
The game is won at a certain vertex if, after some number of turns,
that vertex has the number $2011$ and the other four vertices have the number $0$.
Prove that for any choice of the initial integers,
there is exactly one vertex at which the game can be won.

\item %% Problem 3
In hexagon $ABCDEF$, which is nonconvex but not self-intersecting,
no pair of opposite sides are parallel.
The internal angles satisfy
$\angle A=3\angle D$, $\angle C=3\angle F$, and $\angle E=3\angle B$.
Furthermore $AB=DE$, $BC=EF$, and $CD=FA$.
Prove that diagonals $\ol{AD}$, $\ol{BE}$, and $\ol{CF}$
are concurrent.

\item %% Problem 4
Consider the assertion that for each positive integer $n\geq2$,
the remainder upon dividing $2^{2^n}$ by $2^n-1$ is a power of $4$.
Either prove the assertion or find (with proof) a counterexample.

\item %% Problem 5
Let $P$ be a point inside convex quadrilateral $ABCD$.
Points $Q_1$ and $Q_2$ are located within $ABCD$ such that
\begin{align*}
\angle Q_1BC=\angle ABP, & \qquad \angle Q_1CB=\angle DCP, \\
\angle Q_2AD=\angle BAP, & \qquad \angle Q_2DA=\angle CDP.
\end{align*}
Prove that $\ol{Q_1Q_2} \parallel \ol{AB}$
if and only if $\ol{Q_1Q_2} \parallel \ol{CD}$.

\item %% Problem 6
Let $A$ be a set with $|A|=225$, meaning that $A$ has $225$ elements.
Suppose further that there are eleven subsets $A_1, \dots, A_{11}$ of $A$
such that $|A_i|=45$ for $1\leq i\leq11$ and $|A_i\cap A_j|=9$
for $1\leq i<j\leq11$.
Prove that $|A_1\cup A_2\cup\dots\cup A_{11}|\geq 165$,
and give an example for which equality holds.

\end{enumerate}
\pagebreak

\section{Solutions to Day 1}
\subsection{USAMO 2011/1, proposed by Titu Andreescu}
\textsl{Available online at \url{https://aops.com/community/p2254758}.}
\begin{mdframed}[style=mdpurplebox,frametitle={Problem statement}]
Let $a$, $b$, $c$ be positive real numbers
such that $a^2+b^2+c^2+(a+b+c)^2 \le 4$. Prove that
\[ \frac{ab+1}{(a+b)^2}
  + \frac{bc+1}{(b+c)^2}
  + \frac{ca+1}{(c+a)^2} \ge 3. \]
\end{mdframed}
The condition becomes $2 \ge a^2+b^2+c^2 + ab+bc+ca$.
Therefore,
\begin{align*}
  \sum_{\text{cyc}} \frac{2ab+2}{(a+b)^2}
  &\ge \sum_{\text{cyc}} \frac{2ab+(a^2+b^2+c^2+ab+bc+ca)}{(a+b)^2} \\
  &= \sum_{\text{cyc}} \frac{(a+b)^2+(c+a)(c+b)}{(a+b)^2} \\
  &= 3 + \sum_{\text{cyc}} \frac{(c+a)(c+b)}{(a+b)^2} \\
  &\ge 3 + 3\sqrt[3]{\prod_{\text{cyc}} \frac{(c+a)(c+b)}{(a+b)^2}}
  = 3 + 3 = 6
\end{align*}
with the last line by AM-GM.
This completes the proof.
\pagebreak

\subsection{USAMO 2011/2, proposed by Sam Vandervelde}
\textsl{Available online at \url{https://aops.com/community/p2254765}.}
\begin{mdframed}[style=mdpurplebox,frametitle={Problem statement}]
An integer is assigned to each vertex of a regular pentagon
so that the sum of the five integers is $2011$.
A turn of a solitaire game consists of subtracting an integer $m$
(not necessarily positive) from each of the integers at two neighboring vertices
and adding $2m$ to the opposite vertex, which is not adjacent
to either of the first two vertices.
(The amount $m$ and the vertices chosen can vary from turn to turn.)
The game is won at a certain vertex if, after some number of turns,
that vertex has the number $2011$ and the other four vertices have the number $0$.
Prove that for any choice of the initial integers,
there is exactly one vertex at which the game can be won.
\end{mdframed}
Call the vertices $0$, $1$, $2$, $3$, $4$ in order.
First, notice that the quantity
\[ S \coloneqq N_1 + 2N_2 + 3N_3 + 4N_4 \pmod 5 \]
is invariant, where $N_i$ is the amount at vertex $i$.
This immediately implies that at most one vertex can win,
since in a winning situation all $N_i$ are $0$ except for one, which is $2011$.
(For example, if $S \equiv 3 \pmod 5$, any victory must occur at the
vertex $3$, via $N_3 = 2011$, $N_0 = N_1 = N_2 = N_4 = 0$.)

Now we prove we can win on this unique vertex.
Let $a_i$, $x_i$ denote the number initially at $i$
and $x_i$ denote $\sum m$ over all $m$ where vertex $i$ gains $2m$.
WLOG the possible vertex is $0$,
meaning $a_1 + 2a_2 + 3a_3 + 4a_4 \equiv 0 \pmod 5$.
Moreover we want
\begin{align*}
  2011 &= a_0 + 2x_0 - x_2 - x_3 \\
  0 &= a_1 + 2x_1 - x_3 - x_4 \\
  0 &= a_2 + 2x_2 - x_4 - x_0 \\
  0 &= a_3 + 2x_3 - x_0 - x_1 \\
  0 &= a_4 + 2x_4 - x_1 - x_2.
\end{align*}
We can ignore the first equation since its the sum of the other four,
and we can WLOG shift $x_0 \to 0$ by shifting each $x_i$ by a fixed amount.
We will now solve the resulting system of equations.

First, we have \[ x_4 = 2x_2 + a_2 \text{ and } x_1 = 2x_3 + a_3. \]
Using these to remove all instances of $x_1$ and $x_4$ gives
\[ 2x_2-3x_3 = 2a_3+a_1-a_2 \text{ and } 2x_3-3x_2 = 2a_2+a_4-a_3 \]
whence we have a two-variable system of equations!
To verify its solution is integral, note that
\[ x_2-x_3 = \frac{a_1 - 3a_2 + 3a_3 - a_4}{5} \]
is an integer, since
\[ a_1 - 3a_2 + 3a_3 - a_4 \equiv a_1 + 2a_2 + 3a_3 + 4a_4 \equiv 0 \pmod 5. \]
Abbreviating $\frac{a_1 - 3a_2 + 3a_3 - a_4}{5}$ as $k$, we obtain the desired $x_i$:
\begin{align*}
  x_2 &= 2a_3+a_1-a_2 + 2k \\
  x_3 &= x_2 + k \\
  x_1 &= 2x_3 + a_3 \\
  x_4 &= 2x_2 + a_2 \\
  x_0 &= 0.
\end{align*}
This is the desired integer solution.

\begin{remark*}
  In principle, you could unwind all the definitions above to explicitly
  write every $x_i$ as a function of $a_1$, $a_2$, $a_3$, $a_4$.
  If you did this, you could get the long equations
  \begin{align*}
    x_0 &= 0 \\
    x_1 &= -\tfrac15(6a_1 + 2a_2 + 3a_3 + 4a_4) \\
    x_2 &= -\tfrac15(2a_1 + 4a_2 + a_3 + 3a_4) \\
    x_3 &= -\tfrac15(3a_1 + a_2 + 4a_3 + 2a_4) \\
    x_4 &= -\tfrac15(4a_1 + 3a_2 + 2a_3 + 6a_4)
  \end{align*}
  which indeed are all integers whenever $a_1 + 2a_2 + 3a_3 + 4a_4 \equiv 0 \pmod 5$.

  However, this is quite tedious and also unnecessary to solve the problem.
  That's because we only care that the $x_i$ are integers,
  and do not need to actually know the values.
  This lets us work more indirectly to avoid long calculation, as we did above.
\end{remark*}
\pagebreak

\subsection{USAMO 2011/3, proposed by Gabriel Carroll}
\textsl{Available online at \url{https://aops.com/community/p2254803}.}
\begin{mdframed}[style=mdpurplebox,frametitle={Problem statement}]
In hexagon $ABCDEF$, which is nonconvex but not self-intersecting,
no pair of opposite sides are parallel.
The internal angles satisfy
$\angle A=3\angle D$, $\angle C=3\angle F$, and $\angle E=3\angle B$.
Furthermore $AB=DE$, $BC=EF$, and $CD=FA$.
Prove that diagonals $\ol{AD}$, $\ol{BE}$, and $\ol{CF}$
are concurrent.
\end{mdframed}
We present the official solution.
We say a hexagon is \emph{satisfying}
if it obeys the six conditions;
note that intuitively we expect three degrees of freedom
for satisfying hexagons.

Main idea:
\begin{claim*}
In a satisfying hexagon,
$B$, $D$, $F$ are reflections of $A$, $C$, $E$ across
the sides of $\triangle ACE$.
\end{claim*}
(This claim looks plausible because every excellent hexagon is satisfying,
and both configuration spaces are three-dimensional.)
Call a hexagon of this shape ``excellent'';
in a excellent hexagon the diagonals clearly concur (at the orthocenter).

Set $\beta = \angle B$, $\delta = \angle D$, $\varphi = \angle F$.

Now given a satisfying hexagon $ABCDEF$,
construct a ``phantom hexagon'' $A'B'C'D'E'F'$
with the same angles which is excellent (see figure).
This is possible since $\beta + \delta + \varphi = 180\dg$.

\begin{center}
\begin{asy}
size(12cm);

void MA(string s, pair A, pair O, pair B) {
  markangle(11.0, A,O,B);
  label("$"+s+"$", O, 4*dir((pair)incenter(A,O,B)-O));
}

pair A = dir(110);
pair C = dir(205);
pair E = dir(335);
pair B = -E+2*foot(E, A, C);
pair D = -A+2*foot(A, C, E);
pair F = -C+2*foot(C, E, A);

filldraw(A--B--C--D--E--F--cycle, opacity(0.1)+lightgreen, heavygreen);
draw(A--C--E--cycle, green);
draw(A--D, dashed+green);
draw(B--E, dashed+green);
draw(C--F, dashed+green);

MA("\beta", C,B,A);
MA("\delta", E,D,C);
MA("\varphi", A,F,E);

transform t = shift(-4,0);

add(t*CC());


// Again

pair Ap = A;
pair Cp = C;
pair Ep = E;
pair Bp = B;
pair Dp = D;
pair Fp = F;

filldraw(A--B--C--D--E--F--cycle, opacity(0.1)+lightcyan, heavycyan);
draw(A--C--E--cycle, heavycyan);
draw(A--D, dashed+heavycyan);
draw(B--E, dashed+heavycyan);
draw(C--F, dashed+heavycyan);

MA("\beta", C,B,A);
MA("\delta", E,D,C);
MA("\varphi", A,F,E);

MA("\beta", A,E,C);
MA("\varphi", E,C,A);
MA("\delta", C,A,E);


pair A = t*A;
pair B = t*B;
pair C = t*C;
pair D = t*D;
pair E = t*E;
pair F = t*F;

dot("$A'$", Ap, dir(Ap));
dot("$C'$", Cp, dir(Cp));
dot("$E'$", Ep, dir(Ep));
dot("$B'$", Bp, dir(Bp));
dot("$D'$", Dp, dir(Dp));
dot("$F'$", Fp, dir(Fp));
dot("$A$", A, dir(120));
dot("$B$", B, dir(135));
dot("$C$", C, dir(225));
dot("$D$", D, dir(270));
dot("$E$", E, dir(350));
dot("$F$", F, dir(45));

/* TSQ Source:

!size(12cm);

A := dir 110
C := dir 205
E := dir 335
B := -E+2*foot E A C
D := -A+2*foot A C E
F := -C+2*foot C E A

A--B--C--D--E--F--cycle 0.1 lightgreen / heavygreen
A--C--E--cycle green
A--D dashed green
B--E dashed green
C--F dashed green

!MA("\beta", C,B,A, 0.2);
!MA("\delta", E,D,C, 0.2);
!MA("\varphi", A,F,E, 0.2);

! transform t = shift(-4,0);

!add(t*CC());


// Again

A' = A
C' = C
E' = E
B' = B
D' = D
F' = F

A--B--C--D--E--F--cycle 0.1 lightcyan / heavycyan
A--C--E--cycle heavycyan
A--D dashed heavycyan
B--E dashed heavycyan
C--F dashed heavycyan
! anglepen = heavycyan;

!MA("\beta", C,B,A, 0.2);
!MA("\delta", E,D,C, 0.2);
!MA("\varphi", A,F,E, 0.2);

!MA("\beta", A,E,C, 0.2);
!MA("\varphi", E,C,A, 0.2);
!MA("\delta", C,A,E, 0.2);


A = t*A R120
B = t*B R135
C = t*C R225
D = t*D R270
E = t*E R350
F = t*F R45

*/

\end{asy}
\end{center}


Then it would suffice to prove that:
\begin{lemma*}
  A satisfying hexagon is uniquely determined by its angles
  up to similarity.
  That is, at most one hexagon
  (up to similarity) has angles $\beta$, $\delta$, $\gamma$ as above.
\end{lemma*}
\begin{proof}
  Consider any two satisfying hexagons $ABCDEF$ and $A'B'C'D'E'F'$
  (not necessarily as constructed above!) with the same angles.
  We show they are similar.

  To do this, consider the unit complex numbers
  in the directions $\overrightarrow{BA}$ and $\overrightarrow{DE}$
  respectively and let $\vec x$ denote their sum.
  Define $\vec y$, $\vec z$ similarly.
  Note that the condition $\ol{AB} \not\parallel \ol{DE}$
  implies $\vec x \neq 0$, and similarly.
  Then we have the identities
  \[ AB \cdot \vec x + CD \cdot \vec y + EF \cdot \vec z
    = A'B' \cdot \vec x + C'D' \cdot \vec y + E'F' \cdot \vec z
    = 0. \]

  So we would obtain $AB : CD : EF = A'B' : C'D' : E'F'$
  if only we could show that $\vec x$, $\vec y$, $\vec z$
  are not multiples of each other
  (linear dependency reasons).
  This is a tiresome computation with arguments,
  but here it is.

  First note that none of $\beta$, $\delta$, $\varphi$
  can be $90\dg$,
  since otherwise we get a pair of parallel sides.
  Now work in the complex plane,
  fix a reference such that $\vec A - \vec B$ has argument $0$,
  and assume $ABCDEF$ are labelled counterclockwise.
  Then
  \begin{itemize}
    \ii $\vec B - \vec C$ has argument $\pi-\beta$
    \ii $\vec C - \vec D$ has argument $-(\beta+3\varphi)$
    \ii $\vec D - \vec E$ has argument $\pi-(\beta+3\varphi+\delta)$
    \ii $\vec E - \vec F$ has argument $-(4\beta+3\varphi+\delta)$
    % \ii $\vec F - \vec A$ has argument $\pi-(4\beta+4\varphi+\delta)$
  \end{itemize}
  So the argument of $\vec x$ has argument
  $\frac{\pi-(\beta+3\varphi+\delta)}{2} \pmod \pi$.
  The argument of $\vec y$ has argument
  $\frac{\pi-(5\beta+3\varphi+\delta)}{2} \pmod \pi$.
  Their difference is $2\beta \pmod \pi$, and since $\beta \neq 90\dg$,
  it follows that $\vec x$ and $\vec y$ are not
  multiples of each other; the other cases are similar.
\end{proof}

Then the lemma implies $ABCDEF \sim A'B'C'D'E'F$ and we're done.

\begin{remark*}
  This problem turned out to be known already.
  It appears in this reference:
  \begin{quote}
  Nikolai Beluhov, \emph{Matematika}, 2008, issue 6, problem 3.
  \end{quote}
  It was reprinted as Kvant, 2009, issue 2, problem M2130;
  the reprint is available at
  \url{http://kvant.ras.ru/pdf/2009/2009-02.pdf}.
\end{remark*}

\begin{remark*}
  The vector perspective also shows the condition about parallel
  sides cannot be dropped.
  Here is a counterexample from Ryan Kim
  in the event that it is.
  \begin{center}
  \begin{asy}
  pair A = dir(140);
  pair C_0 = conj(A);
  pair E = -A;
  pair B_0 = 2*C_0-E;
  pair D_0 = 2*C_0-A;
  pair F = extension(A, E*E/C_0, E, A*A/C_0);


  pair C = C_0+2*C_0;
  pair B = B_0+2*C_0;
  pair D = D_0+2*C_0;

  filldraw(A--B--C--D--E--F--cycle, opacity(0.1)+lightcyan, heavycyan);

  draw(A--D, heavygreen);
  draw(B--E, heavygreen);
  draw(C--F, heavygreen);
  draw(A--C_0--E--cycle, blue);

  draw(B_0--C--D_0--C_0--cycle, orange+dashed);

  dot("$A$", A, dir(A));
  dot("$C_0$", C_0, dir(45));
  dot("$E$", E, dir(E));
  dot("$B_0$", B_0, dir(135));
  dot("$D_0$", D_0, dir(315));
  dot("$F$", F, dir(F));
  dot("$C$", C, dir(C));
  dot("$B$", B, dir(B));
  dot("$D$", D, dir(D));

  /* TSQ Source:

  A = dir 140
  C_0 = conj(A) R45
  E = -A
  B_0 = 2*C_0-E R135
  D_0 = 2*C_0-A R315
  F = extension A E*E/C_0 E A*A/C_0


  C = C_0+2*C_0
  B = B_0+2*C_0
  D = D_0+2*C_0

  A--B--C--D--E--F--cycle 0.1 lightcyan / heavycyan

  A--D heavygreen
  B--E heavygreen
  C--F heavygreen
  A--C_0--E--cycle blue

  B_0--C--D_0--C_0--cycle orange dashed

  */
  \end{asy}
  \end{center}
  By adjusting the figure above so that
  the triangles are right isosceles (instead of just right),
  one also finds an example of a hexagon which is satisfying
  and whose diagonals are concurrent,
  but which is \emph{not} excellent.
\end{remark*}
\pagebreak

\section{Solutions to Day 2}
\subsection{USAMO 2011/4, proposed by Sam Vandervelde}
\textsl{Available online at \url{https://aops.com/community/p2254810}.}
\begin{mdframed}[style=mdpurplebox,frametitle={Problem statement}]
Consider the assertion that for each positive integer $n\geq2$,
the remainder upon dividing $2^{2^n}$ by $2^n-1$ is a power of $4$.
Either prove the assertion or find (with proof) a counterexample.
\end{mdframed}
We claim $n = 25$ is a counterexample.
Since $2^{25} \equiv 2^0 \pmod{2^{25}-1}$, we have
\[ 2^{2^{25}} \equiv 2^{2^{25} \bmod{25}}
  \equiv 2^7 \bmod{2^{25}-1} \]
and the right-hand side is actually the remainder,
since $0 < 2^7 < 2^{25}$.
But $2^7$ is not a power of $4$.

\begin{remark*}
  Really, the problem is just equivalent
  for asking $2^n$ to have odd remainder when divided by $n$.
\end{remark*}
\pagebreak

\subsection{USAMO 2011/5, proposed by Zuming Feng, Delong Meng}
\textsl{Available online at \url{https://aops.com/community/p2254841}.}
\begin{mdframed}[style=mdpurplebox,frametitle={Problem statement}]
Let $P$ be a point inside convex quadrilateral $ABCD$.
Points $Q_1$ and $Q_2$ are located within $ABCD$ such that
\begin{align*}
\angle Q_1BC=\angle ABP, & \qquad \angle Q_1CB=\angle DCP, \\
\angle Q_2AD=\angle BAP, & \qquad \angle Q_2DA=\angle CDP.
\end{align*}
Prove that $\ol{Q_1Q_2} \parallel \ol{AB}$
if and only if $\ol{Q_1Q_2} \parallel \ol{CD}$.
\end{mdframed}
If $\ol{AB} \parallel \ol{CD}$ there is nothing to prove.
Otherwise let $X = \ol{AB} \cap \ol{CD}$.
Then the $Q_1$ and $Q_2$ are the isogonal conjugates of $P$
with respect to triangles $XBC$ and $XAD$.
Thus $X$, $Q_1$, $Q_2$ are collinear,
on the isogonal of $\ol{XP}$ with respect to
$\angle DXA = \angle CXB$.
\pagebreak

\subsection{USAMO 2011/6, proposed by Sid Graham}
\textsl{Available online at \url{https://aops.com/community/p2254871}.}
\begin{mdframed}[style=mdpurplebox,frametitle={Problem statement}]
Let $A$ be a set with $|A|=225$, meaning that $A$ has $225$ elements.
Suppose further that there are eleven subsets $A_1, \dots, A_{11}$ of $A$
such that $|A_i|=45$ for $1\leq i\leq11$ and $|A_i\cap A_j|=9$
for $1\leq i<j\leq11$.
Prove that $|A_1\cup A_2\cup\dots\cup A_{11}|\geq 165$,
and give an example for which equality holds.
\end{mdframed}
Ignore the $225$ --- it is irrelevant.

Denote the elements of $A_1 \cup \dots \cup A_{11}$
by $a_1$, \dots, $a_n$,
and suppose that $a_i$ appears $x_i$ times among $A_i$
for each $1 \le i \le n$ (so $1 \le x_i \le 11$).
Then we have
\[ \sum_{i=1}^{11} x_i = \sum_{i=1}^{11} |A_i| = 45 \cdot 11 \]
and
\[ \sum_{i=1}^{11} \binom{x_i}{2} = \sum_{1 \le i < j \le 11}
  \left\lvert A_i \cap A_j \right\rvert
  = \binom{11}{2} \cdot 9. \]
Therefore, we deduce that
$\sum x_i = 495$ and $\sum_i x_i^2 = 1485$.
Now, by Cauchy Schwarz
\[ n \left( \sum_i x_i^2 \right) \ge \left( \sum x_i \right)^2 \]
which implies $n \ge \frac{495^2}{1485} = 165$.

Equality occurs if we let $A$ consist of the $165$
three-element subsets of $\left\{ 1, \dots, 11 \right\}$
(plus $60$ of your favorite reptiles
if you really insist $|A|=225$).
Then we let $A_i$ denote those subsets containing $i$,
of which there are $\binom{10}{2} = 45$,
and so that $|A_i \cap A_j| = \binom 91 = 9$.
\pagebreak


\end{document}
