% © Evan Chen
% Downloaded from https://web.evanchen.cc/

\documentclass[11pt]{scrartcl}
\usepackage[sexy]{evan}
\ihead{\footnotesize\textbf{\thetitle}}
\ohead{\footnotesize\href{http://web.evanchen.cc}{\ttfamily web.evanchen.cc},
    updated \today}
\title{USAMO 2020 Solution Notes}
\date{\today}

\begin{document}

\maketitle

\begin{abstract}
This is a compilation of solutions
for the 2020 USAMO.
Some of the solutions are my own work,
but many are from the official solutions provided by the organizers
(for which they hold any copyrights),
and others were found by users on the Art of Problem Solving forums.

These notes will tend to be a bit more advanced and terse than the ``official''
solutions from the organizers.
In particular, if a theorem or technique is not known to beginners
but is still considered ``standard'', then I often prefer to
use this theory anyways, rather than try to work around or conceal it.
For example, in geometry problems I typically use directed angles
without further comment, rather than awkwardly work around configuration issues.
Similarly, sentences like ``let $\mathbb{R}$ denote the set of real numbers''
are typically omitted entirely.

Corrections and comments are welcome!
\end{abstract}

\tableofcontents
\newpage

\addtocounter{section}{-1}
\section{Problems}
\begin{enumerate}[\bfseries 1.]
\item %% Problem 1
Let $ABC$ be a fixed acute triangle
inscribed in a circle $\omega$ with center $O$.
A variable point $X$ is chosen on minor arc $AB$ of $\omega$,
and segments $CX$ and $AB$ meet at $D$.
Denote by $O_1$ and $O_2$ the circumcenters of
triangles $ADX$ and $BDX$, respectively.
Determine all points $X$ for which
the area of triangle $OO_1O_2$ is minimized.

\item %% Problem 2
An empty $2020 \times 2020 \times 2020$ cube is given,
and a $2020 \times 2020$ grid of square unit cells is drawn on each of its six faces.
A \emph{beam} is a $1 \times 1 \times 2020$ rectangular prism.
Several beams are placed inside the cube subject to the following conditions:
\begin{itemize}
\item The two $1 \times 1$ faces of each beam coincide
  with unit cells lying on opposite faces of the cube.
  (Hence, there are $3 \cdot 2020^2$ possible positions for a beam.)
\item No two beams have intersecting interiors.
\item The interiors of each of the four $1 \times 2020$ faces of each beam touch
  either a face of the cube or the interior of the face of another beam.
\end{itemize}
What is the smallest positive number of beams that can be placed to satisfy these conditions?

\item %% Problem 3
Let $p$ be an odd prime.
An integer $x$ is called a \emph{quadratic non-residue}
if $p$ does not divide $x-t^2$ for any integer $t$.

Denote by $A$ the set of all integers $a$
such that $1 \le a < p$,
and both $a$ and $4-a$ are quadratic non-residues.
Calculate the remainder
when the product of the elements of $A$ is divided by $p$.

\item %% Problem 4
Suppose that $(a_1, b_1)$, $(a_2, b_2)$, \dots, $(a_{100}, b_{100})$
are distinct ordered pairs of nonnegative integers.
Let $N$ denote the number of pairs of integers $(i,j)$ satisfying
$1 \le i < j \le 100$ and $\left\lvert a_ib_j - a_jb_i \right\rvert = 1$.
Determine the largest possible value of $N$
over all possible choices of the $100$ ordered pairs.

\item %% Problem 5
A finite set $S$ of points in the coordinate plane
is called \emph{overdetermined} if $|S| \ge 2$
and there exists a nonzero polynomial $P(t)$,
with real coefficients and of degree at most $|S|-2$,
satisfying $P(x)=y$ for every point $(x,y) \in S$.

For each integer $n \ge 2$,
find the largest integer $k$ (in terms of $n$) such that
there exists a set of $n$ distinct points
that is \emph{not} overdetermined,
but has $k$ overdetermined subsets.

\item %% Problem 6
Let $n \geq 2$ be an integer.
Let $x_1 \ge x_2 \ge \dots \ge x_n$
and $y_1 \ge y_2 \ge \dots \ge y_n$ be $2n$ real numbers
such that
\begin{align*}
  0 &= x_1 + x_2 + \dots + x_n = y_1 + y_2 + \dots + y_n, \\
  \text{and}\quad
  1 &= x_1^2 + x_2^2 + \dots + x_n^2 = y_1^2 + y_2^2 + \dots + y_n^2.
\end{align*}
Prove that
\[ \sum_{i=1}^n (x_i y_i - x_i y_{n+1-i})
  \geq \frac{2}{\sqrt{n-1}}.  \]

\end{enumerate}
\pagebreak

\section{Solutions to Day 1}
\subsection{USAMO 2020/1, proposed by Zuming Feng}
\textsl{Available online at \url{https://aops.com/community/p15952768}.}
\begin{mdframed}[style=mdpurplebox,frametitle={Problem statement}]
Let $ABC$ be a fixed acute triangle
inscribed in a circle $\omega$ with center $O$.
A variable point $X$ is chosen on minor arc $AB$ of $\omega$,
and segments $CX$ and $AB$ meet at $D$.
Denote by $O_1$ and $O_2$ the circumcenters of
triangles $ADX$ and $BDX$, respectively.
Determine all points $X$ for which
the area of triangle $OO_1O_2$ is minimized.
\end{mdframed}
We prove $[OO_1O_2] \ge \frac14 [ABC]$,
with equality if and only if $\ol{CX} \perp \ol{AB}$.

\paragraph{First approach (Bobby Shen).}
We use two simultaneous inequalities:
\begin{itemize}
  \ii Let $M$ and $N$ be the midpoints of $CX$ and $DX$.
  Then $MN$ equals the length of the $O$-altitude of $\triangle OO_1O_2$,
  since $\ol{O_1O_2}$ and $\ol{DX}$ meet at $N$ at a right angle.
  Moreover, we have \[ MN = \half CD \ge \half h_a \]
  where $h_a$ denotes the $A$-altitude.

  \ii The projection of $O_1 O_2$ onto line $AB$ has length exactly $AB/2$.
  Thus \[ O_1 O_2 \ge \half AB. \]
\end{itemize}
So, we find
\[ [OO_1O_2] = \half \cdot MN \cdot O_1 O_2
  \ge \frac 18 h_a \cdot AB = \frac14 [ABC]. \]
Note that equality occurs in both cases if and only if $\ol{CX} \perp \ol{AB}$.
So the area is minimized exactly when this occurs.


\paragraph{Second approach (Evan's solution).}
We need two claims.
\begin{claim*}
  We have $\triangle O O_1 O_2 \sim \triangle CBA$, with opposite orientation.
\end{claim*}
\begin{proof}
  Notice that $\ol{OO_1} \perp \ol{AX}$
  and $\ol{O_1 O_2} \perp \ol{CX}$,
  so $\dang OO_1O_2 = \dang AXC = \dang ABC$.
  Similarly $\dang OO_2O_1 = \dang BAC$.
\end{proof}

Therefore, the problem is equivalent to minimizing $O_1 O_2$.
\begin{center}
\begin{asy}
pair C = dir(130);
pair A = dir(200);
pair B = dir(340);
pair X = dir(280);
pair D = extension(C, X, A, B);
pair O_1 = circumcenter(A, D, X);
pair O_2 = circumcenter(B, D, X);
pair O = origin;
filldraw(O--O_1--O_2--cycle, opacity(0.1)+lightcyan, deepcyan);
filldraw(A--X--B--cycle, opacity(0.1)+lightred, red);
filldraw(A--B--C--cycle, opacity(0.1)+lightcyan, blue);
draw(X--C, red);
filldraw(O_1--X--O_2--cycle, opacity(0.1)+yellow, orange);
draw(unitcircle, grey);

dot("$C$", C, dir(C));
dot("$A$", A, dir(A));
dot("$B$", B, dir(B));
dot("$X$", X, dir(X));
dot("$D$", D, dir(245));
dot("$O_1$", O_1, dir(O_1));
dot("$O_2$", O_2, dir(20));
dot("$O$", O, dir(45));

/* TSQ Source:

C = dir 130
A = dir 200
B = dir 340
X = dir 280
D = extension C X A B R245
O_1 = circumcenter A D X
O_2 = circumcenter B D X R20
O = origin R45
O--O_1--O_2--cycle 0.1 lightcyan / deepcyan
A--X--B--cycle 0.1 lightred / red
A--B--C--cycle 0.1 lightcyan / blue
X--C red
O_1--X--O_2--cycle 0.1 yellow / orange
unitcircle grey

*/
\end{asy}
\end{center}

\begin{claim*}
  [Salmon theorem]
  We have $\triangle X O_1 O_2 \sim \triangle XAB$.
\end{claim*}
\begin{proof}
  It follows from the fact that $\triangle AO_1 X \sim \triangle BO_2 X$
  (since $\dang ADX = \dang XDB \implies \dang XO_1A = \dang XO_2B$)
  and that spiral similarities come in pairs.
\end{proof}

Let $\theta = \angle ADX$.
The ratio of similarity in the previous claim is equal to
$\frac{XO_1}{XA} = \frac{1}{2\sin \theta}$.
In other words,
\[ O_1 O_2 = \frac{AB}{2 \sin \theta}. \]
This is minimized when $\theta = 90\dg$,
in which case $O_1 O_2 = AB/2$
and $[OO_1O_2] = \frac14 [ABC]$.
This completes the solution.
\pagebreak

\subsection{USAMO 2020/2, proposed by Alex Zhai}
\textsl{Available online at \url{https://aops.com/community/p15952773}.}
\begin{mdframed}[style=mdpurplebox,frametitle={Problem statement}]
An empty $2020 \times 2020 \times 2020$ cube is given,
and a $2020 \times 2020$ grid of square unit cells is drawn on each of its six faces.
A \emph{beam} is a $1 \times 1 \times 2020$ rectangular prism.
Several beams are placed inside the cube subject to the following conditions:
\begin{itemize}
\item The two $1 \times 1$ faces of each beam coincide
  with unit cells lying on opposite faces of the cube.
  (Hence, there are $3 \cdot 2020^2$ possible positions for a beam.)
\item No two beams have intersecting interiors.
\item The interiors of each of the four $1 \times 2020$ faces of each beam touch
  either a face of the cube or the interior of the face of another beam.
\end{itemize}
What is the smallest positive number of beams that can be placed to satisfy these conditions?
\end{mdframed}
Answer: $3030$ beams.

\medskip

\textbf{Construction}:
We first give a construction with $3n/2$ beams for any $n \times n \times n$ box,
where $n$ is an even integer.
Shown below is the construction for $n=6$, which generalizes.
(The left figure shows the cube in 3d;
the right figure shows a direct view of the three visible faces.)
\begin{center}
\begin{asy}
import three;
import bsp;

size(8cm);
settings.prc = false;
settings.render = 0;
settings.tex="pdflatex";
settings.outformat="pdf";

triple O = (9,7,5);
currentprojection = orthographic(O);
currentlight = light(gray(2),specular=gray(0.7), specularfactor=3, (7,7,5));

pen finalpen = yellow+2;
pen meshpenlight = rgb(0.95,0.95,0.99);

int M = 6;
int s = 0;
for (int i=0; i<=M; ++i) {
draw( (s,i,0)--(s,i,M), meshpenlight);
draw( (s,0,i)--(s,M,i), meshpenlight);
draw( (i,s,0)--(i,s,M), meshpenlight);
draw( (0,s,i)--(M,s,i), meshpenlight);
draw( (i,0,s)--(i,M,s), meshpenlight);
draw( (0,i,s)--(M,i,s), meshpenlight);
}

draw(box((0,0,0), (M,M,M)), finalpen);

pen penA = red;
pen penB = cyan;
pen penC = green;

draw(shift(0,0,0)*unitcube, penA);
draw(shift(0,0,1)*unitcube, penA);
draw(shift(0,0,2)*unitcube, penA);
draw(shift(0,0,3)*unitcube, penA);
draw(shift(0,0,4)*unitcube, penA);
draw(shift(0,0,5)*unitcube, penA);

draw(shift(1,0,0)*unitcube, penB);
draw(shift(1,1,0)*unitcube, penB);
draw(shift(1,2,0)*unitcube, penB);
draw(shift(1,3,0)*unitcube, penB);
draw(shift(1,4,0)*unitcube, penB);
draw(shift(1,5,0)*unitcube, penB);

draw(shift(0,1,1)*unitcube, penC);
draw(shift(1,1,1)*unitcube, penC);
draw(shift(2,1,1)*unitcube, penC);
draw(shift(3,1,1)*unitcube, penC);
draw(shift(4,1,1)*unitcube, penC);
draw(shift(5,1,1)*unitcube, penC);

draw(shift(2,2,0)*unitcube, penA);
draw(shift(2,2,1)*unitcube, penA);
draw(shift(2,2,2)*unitcube, penA);
draw(shift(2,2,3)*unitcube, penA);
draw(shift(2,2,4)*unitcube, penA);
draw(shift(2,2,5)*unitcube, penA);

draw(shift(3,0,2)*unitcube, penB);
draw(shift(3,1,2)*unitcube, penB);
draw(shift(3,2,2)*unitcube, penB);
draw(shift(3,3,2)*unitcube, penB);
draw(shift(3,4,2)*unitcube, penB);
draw(shift(3,5,2)*unitcube, penB);

draw(shift(0,3,3)*unitcube, penC);
draw(shift(1,3,3)*unitcube, penC);
draw(shift(2,3,3)*unitcube, penC);
draw(shift(3,3,3)*unitcube, penC);
draw(shift(4,3,3)*unitcube, penC);
draw(shift(5,3,3)*unitcube, penC);

draw(shift(4,4,0)*unitcube, penA);
draw(shift(4,4,1)*unitcube, penA);
draw(shift(4,4,2)*unitcube, penA);
draw(shift(4,4,3)*unitcube, penA);
draw(shift(4,4,4)*unitcube, penA);
draw(shift(4,4,5)*unitcube, penA);

draw(shift(5,0,4)*unitcube, penB);
draw(shift(5,1,4)*unitcube, penB);
draw(shift(5,2,4)*unitcube, penB);
draw(shift(5,3,4)*unitcube, penB);
draw(shift(5,4,4)*unitcube, penB);
draw(shift(5,5,4)*unitcube, penB);

draw(shift(0,5,5)*unitcube, penC);
draw(shift(1,5,5)*unitcube, penC);
draw(shift(2,5,5)*unitcube, penC);
draw(shift(3,5,5)*unitcube, penC);
draw(shift(4,5,5)*unitcube, penC);
draw(shift(5,5,5)*unitcube, penC);

pen meshpen = grey;
int s = M;
for (int i=1; i<=M-1; ++i) {
draw( (s,i,0)--(s,i,M), meshpen);
draw( (s,0,i)--(s,M,i), meshpen);
draw( (i,s,0)--(i,s,M), meshpen);
draw( (0,s,i)--(M,s,i), meshpen);
draw( (i,0,s)--(i,M,s), meshpen);
draw( (0,i,s)--(M,i,s), meshpen);
}

draw((M,M,M)--(0,M,M), finalpen);
draw((M,M,M)--(M,0,M), finalpen);
draw((M,M,M)--(M,M,0), finalpen);
\end{asy}
\quad
\begin{asy}
size(8cm);
picture redface, greenface, blueface;

void draw_grid(picture pic) {
  for (int i=1; i<6; ++i) {
    draw(pic, (i,0)--(i,6), grey);
    draw(pic, (0,i)--(6,i), grey);
  }
  draw(pic, scale(6)*unitsquare, yellow+2);
}

fill(redface, shift(0,5)*unitsquare, red);
fill(redface, shift(2,3)*unitsquare, red);
fill(redface, shift(4,1)*unitsquare, red);
draw_grid(redface);

fill(greenface, shift(5,5)*unitsquare, green);
fill(greenface, shift(3,3)*unitsquare, green);
fill(greenface, shift(1,1)*unitsquare, green);
draw_grid(greenface);

fill(blueface, shift(0,4)*unitsquare, cyan);
fill(blueface, shift(2,2)*unitsquare, cyan);
fill(blueface, shift(4,0)*unitsquare, cyan);
draw_grid(blueface);

add(shift(0,8.5)*shift(3,3)*rotate(-45)*shift(-3,-3)*redface);
add(shift(-4,0)*greenface);
add(shift( 4,0)*blueface);
label("Left face", (-1,0), dir(-90));
label("Right face", (7,0), dir(-90));
label("Top face", (3,16), dir(90));
\end{asy}
\end{center}
To be explicit,
impose coordinate axes such that one corner of the cube is the origin.
We specify a beam by two opposite corners.
The $3n/2$ beams come in three directions, $n/2$ in each direction:
\begin{itemize}
  \ii $(0,0,0) \to (1,1,n)$, $(2,2,0) \to (3,3,n)$, $(4,4,0) \to (5,5,n)$, and so on;
  \ii $(1,0,0) \to (2,n,1)$, $(3,0,2) \to (4,n,3)$, $(5,0,4) \to (6,n,5)$, and so on;
  \ii $(0,1,1) \to (n,2,2)$, $(0,3,3) \to (n,4,4)$, $(0,5,5) \to (n,6,6)$, and so on.
\end{itemize}
This gives the figure we drew earlier and shows $3030$ beams is possible.

\bigskip

\textbf{Necessity}:
We now show at least $3n/2$ beams are necessary.
Maintain coordinates,
and call the beams $x$-beams, $y$-beams, $z$-beams
according to which plane their long edges are perpendicular too.
Let $N_x$, $N_y$, $N_z$ be the number of these.

\begin{claim*}
  If $\min(N_x, N_y, N_z) = 0$, then at least $n^2$ beams are needed.
\end{claim*}
\begin{proof}
  Assume WLOG that $N_z = 0$.
  Orient the cube so the $z$-plane touches the ground.
  Then each of the $n$ layers of the cube (from top to bottom)
  must be completely filled, and so at least $n^2$ beams are necessary,
\end{proof}

We henceforth assume $\min(N_x, N_y, N_z) > 0$.

\begin{claim*}
  If $N_z > 0$, then we have $N_x + N_y \ge n$.
\end{claim*}
\begin{proof}
  Again orient the cube so the $z$-plane touches the ground.
  We see that for each of the $n$ layers of the cube (from top to bottom),
  there is at least one $x$-beam or $y$-beam.
  (Pictorially, some of the $x$ and $y$ beams form a ``staircase''.)
  This completes the proof.
\end{proof}
Proceeding in a similar fashion, we arrive at the three relations
\begin{align*}
  N_x + N_y &\ge n \\
  N_y + N_z &\ge n \\
  N_z + N_x &\ge n.
\end{align*}
Summing gives $N_x  + N_y + N_z \ge 3n/2$ too.

\begin{remark*}
  The problem condition has the following ``physics'' interpretation.
  Imagine the cube is a metal box which is sturdy enough that
  all beams must remain orthogonal to the faces of the box
  (i.e.\ the beams cannot spin).
  Then the condition of the problem is exactly what is needed so that,
  if the box is shaken or rotated, the beams will not move.
\end{remark*}

\begin{remark*}
  Walter Stromquist points out that the number of constructions
  with $3030$ beams is actually enormous:
  not dividing out by isometries,
  the number is $(2 \cdot 1010!)^3$.
\end{remark*}
\pagebreak

\subsection{USAMO 2020/3, proposed by Richard Stong, Toni Bluher}
\textsl{Available online at \url{https://aops.com/community/p15952782}.}
\begin{mdframed}[style=mdpurplebox,frametitle={Problem statement}]
Let $p$ be an odd prime.
An integer $x$ is called a \emph{quadratic non-residue}
if $p$ does not divide $x-t^2$ for any integer $t$.

Denote by $A$ the set of all integers $a$
such that $1 \le a < p$,
and both $a$ and $4-a$ are quadratic non-residues.
Calculate the remainder
when the product of the elements of $A$ is divided by $p$.
\end{mdframed}
The answer is that $\prod_{a \in A} a \equiv 2 \pmod p$
regardless of the value of $p$.
In the following solution,
we work in $\FF_p$ always
and abbreviate ``quadratic residue'' and ``non-quadratic residue''
to ``QR'' and ``non-QR'', respectively.

We define
\begin{align*}
  A &= \left\{ a \in \FF_p \mid a, 4-a \text{ non-QR} \right\} \\
  B &= \left\{ b \in \FF_p \mid b, 4-b \text{ QR}, b \neq 0, b \neq 4 \right\}.
\end{align*}
Notice that
\[ A \cup B = \left\{ n \in \FF_p \mid
    n(4-n) \text{ is QR} \;
    n \neq 0, 4 \right\}. \]
We now present two approaches both based on the set $B$.

\paragraph{First approach (based on Holden Mui's presentation in Mathematics Magazine).}
The idea behind this approach is that $n(4-n)$
is itself an element of $B$ for $n \in A \cup B$,
because $4 - n(4-n) = (n-2)^2$.
This motivates the following claim.
\begin{claim*}
  The map
  \[ A \cup B \setminus \{2\} \to B
    \qquad \text{by}\quad n \mapsto n(4-n) \]
  is a well-defined two-to-one map,
  i.e.\ every $b \in B$ has exactly two pre-images.
\end{claim*}
\begin{proof}
  Since $n \notin \{0,2,4\}$,
  we have $n(4-n) \notin \{0,4\}$,
  so as discussed previously, $n(4-n) \in B$.
  Thus this map is well-defined.

  Choose $b \in B$.
  The quadratic equation $n(4-n) = b$ in $n$
  rewrites as $n^2-4n+b=0$,
  and has discriminant $4(4-b)$,
  which is a nonzero QR.
  Hence there are exactly two values of $n$, as desired.
\end{proof}
Therefore, it follows that
\[ \prod_{n \in A \cup B \setminus \{2\}} n
  = \prod_{b \in B} b \]
by pairing $n$ with $4-n$ on the left-hand side.
So, $\prod_{a \in A} a = 2$.

\paragraph{Second calculation approach (along the lines of official solution).}
We now do the following magical calculation in $\FF_p$:
\begin{align*}
  \prod_{b \in B} b
  = \prod_{b \in B} (4-b)
  &= \prod_{\substack{1 \le y \le (p-1)/2 \\ y \neq 2 \\ 4-y^2 \text{ is QR}}} (4-y^2) \\
  &= \prod_{\substack{1 \le y \le (p-1)/2 \\ y \neq 2 \\ 4-y^2 \text{ is QR}}} (2+y)
  \prod_{\substack{1 \le y \le (p-1)/2 \\ y \neq 2 \\ 4-y^2 \text{ is QR}}} (2-y) \\
  &= \prod_{\substack{1 \le y \le (p-1)/2 \\ y \neq 2 \\ 4-y^2 \text{ is QR}}} (2+y)
  \prod_{\substack{(p+1)/2 \le y \le p-1 \\ y \neq p-2 \\ 4-y^2 \text{ is QR}}} (2+y) \\
  &= \prod_{\substack{1 \le y \le p-1 \\ y \neq 2, p-2 \\ 4-y^2 \text{ is QR}}} (2+y) \\
  &= \prod_{\substack{3 \le z \le p+1 \\ z \neq 4,p \\ z(4-z) \text{ is QR}}} z \\
  &= \prod_{\substack{0 \le z \le p-1 \\ z \neq 0,4,2 \\ z(4-z) \text{ is QR}}} z.
\end{align*}
Note $z(4-z)$ is a nonzero QR if and only if $z \in A \cup B$.
So the right-hand side is almost the product over $z \in A \cup B$,
except it is missing the $z=2$ term.
Adding it in gives
\[ \prod_{b \in B} b
  = \half \prod_{\substack{0 \le z \le p-1 \\ z \neq 0,4 \\ z(4-z) \text{ is QR}}} z
  = \half \prod_{a \in A} a \prod_{b \in B} b.  \]
This gives $\prod_{a \in A} a = 2$ as desired.
\pagebreak

\section{Solutions to Day 2}
\subsection{USAMO 2020/4, proposed by Ankan Bhattacharya}
\textsl{Available online at \url{https://aops.com/community/p15952792}.}
\begin{mdframed}[style=mdpurplebox,frametitle={Problem statement}]
Suppose that $(a_1, b_1)$, $(a_2, b_2)$, \dots, $(a_{100}, b_{100})$
are distinct ordered pairs of nonnegative integers.
Let $N$ denote the number of pairs of integers $(i,j)$ satisfying
$1 \le i < j \le 100$ and $\left\lvert a_ib_j - a_jb_i \right\rvert = 1$.
Determine the largest possible value of $N$
over all possible choices of the $100$ ordered pairs.
\end{mdframed}
The answer is $197$.
In general, if $100$ is replaced by $n \ge 2$ the answer is $2n-3$.

The idea is that if we let $P_i = (a_i, b_i)$ be a point
in the coordinate plane, and let $O = (0,0)$
then we wish to maximize the number of triangles
$\triangle O P_i P_j$ which have area $1/2$.
Call such a triangle \emph{good}.

\medskip

\textbf{Construction of $197$ points}:
It suffices to use the points
$(1,0)$, $(1,1)$, $(2,1)$, $(3,1)$, \dots, $(99,1)$ as shown.
Notice that:
\begin{itemize}
  \ii There are $98$ good triangles
  with vertices $(0,0)$, $(k,1)$ and $(k+1,1)$ for $k=1, \dots, 98$.

  \ii There are $99$ good triangles
  with vertices $(0,0)$, $(1,0)$ and $(k,1)$ for $k=1, \dots, 99$.
\end{itemize}
This is a total of $98 + 99 = 197$ triangles.
\begin{center}
\begin{asy}
  draw( (6,0)--(0,0)--(0,2), grey, Arrows );
  dot("$O$", (0,0), dir(225), red );
  dot("$(1,0)$", (1,0), dir(-90), blue);
  dot("$(1,1)$", (1,1), dir(90), blue);
  dot("$(2,1)$", (2,1), dir(90), blue);
  dot("$(3,1)$", (3,1), dir(90), blue);
  dot("$(4,1)$", (4,1), dir(90), blue);
  label("$\dotsb$", (5,1), blue);
\end{asy}
\end{center}

\medskip

\textbf{Proof that $197$ points is optimal}:
We proceed by induction on $n$ to show the bound of $2n-3$.
The base case $n=2$ is evident.

For the inductive step,
suppose (without loss of generality) that the point $P = P_n = (a,b)$
is the farthest away from the point $O$ among all points.
\begin{claim*}
  This farthest point $P = P_n$ is part of at most two good triangles.
\end{claim*}
\begin{proof}
  We must have $\gcd(a,b) = 1$ for $P$ to be in any good triangles at all,
  since otherwise any divisor of $\gcd(a,b)$ also divides $2[OPQ]$.
  Now, we consider the locus of all points $Q$ for which $[OPQ] = 1/2$.
  It consists of two parallel lines passing with slope $OP$, as shown.
  \begin{center}
  \begin{asy}
    size(10cm);
    int M = 6;
    pair P = (3,2);
    pair O = (0,0);
    draw(O--P, blue);
    draw(arc(O,abs(P),0,90), deepcyan);
    for (int i=-M#2; i<=M; ++i) {
      draw( (-M#2,i)--(M,i), grey);
      draw( (i,-M#2)--(i,M), grey);
    }
    draw( (-M#2,0)--(M,0), black+1.4, Arrows(TeXHead) );
    draw( (0,-M#2)--(0,M), black+1.4, Arrows(TeXHead) );

    pair X = (1,1);
    pair Y = O+P-X;
    dot("$(u,v)$", X, dir(135), red);
    dot("$(u',v')$", Y, dir(-45), red);

    pair X1 = X+P;
    pair X2 = X-P;
    draw(X1--X2, red+dashed, Arrows);

    pair Y1 = Y+P;
    pair Y2 = Y-P;
    draw(Y1--Y2, red+dashed, Arrows);

    dot("$O$", O, dir(225), blue);
    dot("$P=(a,b)$", P, dir(30), blue);
  \end{asy}
  \end{center}
  Since $\gcd(a,b)=1$, see that only two lattice points on this locus
  actually lie inside the quarter-circle centered at $O$ with radius $OP$.
  Indeed if one of the points is $(u,v)$
  then the others on the line are $(u \pm a, v \pm b)$ where the signs match.
  This proves the claim.
\end{proof}
This claim allows us to complete the induction by simply deleting $P_n$.
\pagebreak

\subsection{USAMO 2020/5, proposed by Carl Schildkraut}
\textsl{Available online at \url{https://aops.com/community/p15952824}.}
\begin{mdframed}[style=mdpurplebox,frametitle={Problem statement}]
A finite set $S$ of points in the coordinate plane
is called \emph{overdetermined} if $|S| \ge 2$
and there exists a nonzero polynomial $P(t)$,
with real coefficients and of degree at most $|S|-2$,
satisfying $P(x)=y$ for every point $(x,y) \in S$.

For each integer $n \ge 2$,
find the largest integer $k$ (in terms of $n$) such that
there exists a set of $n$ distinct points
that is \emph{not} overdetermined,
but has $k$ overdetermined subsets.
\end{mdframed}
We claim the answer is $k = 2^{n-1}-n$.
We denote the $n$ points by $A$.

Throughout the solution we will repeatedly use the following fact:
\begin{lemma*}
  If $S$ is a finite set of points in the plane
  there is at most one polynomial with real coefficients
  and of degree at most $|S|-1$
  whose graph passes through all points of $S$.
\end{lemma*}
\begin{proof}
  If any two of the points have the same $x$-coordinate
  then obviously no such polynomial may exist at all.

  Otherwise, suppose $f$ and $g$ are two such polynomials.
  Then $f-g$ has degree at most $|S|-1$,
  but it has $|S|$ roots, so is the zero polynomial.
\end{proof}
\begin{remark*}
  Actually Lagrange interpolation implies
  that such a polynomial exists
  as long as all the $x$-coordinates are different!
\end{remark*}

\medskip

\textbf{Construction}:
Consider the set $A = \left\{ (1,a), (2,b), (3,b), (4,b), \dots, (n,b) \right\}$,
where $a$ and $b$ are two distinct nonzero real numbers.
Any subset of the latter $n-1$ points with at least one element
is overdetermined, and there are $2^{n-1}-n$ such sets.

\medskip

\textbf{Bound}:
Say a subset $S$ of $A$ is \emph{flooded}
if it is not overdetermined.
For brevity, an \emph{$m$-set}
refers simply to a subset of $A$ with $m$ elements.

\begin{claim*}
  If $S$ is an flooded $m$-set for $m \ge 3$,
  then at most one $(m-1)$-subset of $S$ is not flooded.
\end{claim*}
\begin{proof}
  Let $S = \left\{ p_1, \dots, p_m \right\}$ be flooded.
  Assume for contradiction that $S - \{p_i\}$
  and $S - \{p_j\}$ are both overdetermined.
  Then we can find polynomials $f$ and $g$ of degree at most $m-3$
  passing through $S - \{p_i\}$ and $S - \{p_j\}$, respectively.

  Since $f$ and $g$ agree on $S - \{p_i, p_j\}$,
  which has $m-2$ elements, by the lemma we have $f = g$.
  Thus this common polynomial (actually of degree at most $m-3$)
  witnesses that $S$ is overdetermined,
  which is a contradiction.
\end{proof}

\begin{claim*}
  For all $m = 2, 3, \dots, n$ there are at least $\binom{n-1}{m-1}$
  flooded $m$-sets of $A$.
\end{claim*}
\begin{proof}
  The proof is by downwards induction on $m$.
  The base case $m = n$ is by assumption.

  For the inductive step, suppose it's true for $m$.
  Each of the $\binom{n-1}{m-1}$ flooded $m$-sets
  has at least $m-1$ flooded $(m-1)$-subsets.
  Meanwhile, each $(m-1)$-set has exactly $n-(m-1)$ parent $m$-sets.
  We conclude the number of flooded sets of size $m-1$ is at least
  \[ \frac{m-1}{n-(m-1)} \binom{n-1}{m-1} = \binom{n-1}{m-2} \]
  as desired.
\end{proof}

This final claim completes the proof,
since it shows there are at most
\[ \sum_{m=2}^n \left( \binom nm - \binom{n-1}{m-1} \right)
  = 2^{n-1} - n \]
overdetermined sets, as desired.

\begin{remark*}
  [On repeated $x$-coordinates]
  Suppose $A$ has two points $p$ and $q$ with repeated $x$-coordinates.
  We can argue directly that $A$ satisfies the bound.
  Indeed, any overdetermined set contains at most one of $p$ and $q$.
  Moreover, given $S \subseteq A - \{p,q\}$,
  if $S \cup \{p\}$ is overdetermined
  then $S \cup \{q\}$ is not, and vice-versa.
  (Recall that overdetermined sets always
  have distinct $x$-coordinates.)
  This gives a bound $\left[ 2^{n-2}-(n-2)-1 \right]
  + \left[ 2^{n-2}-1 \right] = 2^{n-1}-n$ already.
\end{remark*}

\begin{remark*}
  [Alex Zhai]
  An alternative approach to the double-counting
  argument is to show that any overdetermined $m$-set
  has an flooded $m$-superset.
  Together with the first claim,
  this lets us pair overdetermined sets in a way
  that implies the bound.
\end{remark*}
\pagebreak

\subsection{USAMO 2020/6, proposed by David Speyer, Kiran Kedlaya}
\textsl{Available online at \url{https://aops.com/community/p15953303}.}
\begin{mdframed}[style=mdpurplebox,frametitle={Problem statement}]
Let $n \geq 2$ be an integer.
Let $x_1 \ge x_2 \ge \dots \ge x_n$
and $y_1 \ge y_2 \ge \dots \ge y_n$ be $2n$ real numbers
such that
\begin{align*}
  0 &= x_1 + x_2 + \dots + x_n = y_1 + y_2 + \dots + y_n, \\
  \text{and}\quad
  1 &= x_1^2 + x_2^2 + \dots + x_n^2 = y_1^2 + y_2^2 + \dots + y_n^2.
\end{align*}
Prove that
\[ \sum_{i=1}^n (x_i y_i - x_i y_{n+1-i})
  \geq \frac{2}{\sqrt{n-1}}.  \]
\end{mdframed}
We present two approaches.
In both approaches, it's helpful motivation that
for even $n$, equality occurs at
\begin{align*}
  (x_i) &= \Big(
    \underbrace{\frac{1}{\sqrt n}, \dots, \frac{1}{\sqrt n}}_{n/2},
    \underbrace{-\frac{1}{\sqrt n}, \dots, -\frac{1}{\sqrt n}}_{n/2} \Big) \\
  (y_i) &= \Big( \frac{n-1}{\sqrt{n(n-1)}},
      \underbrace{- \frac{1}{\sqrt{n(n-1)}},
      \dots,
      - \frac{1}{\sqrt{n(n-1)}}}_{n-1} \Big)
\end{align*}

\paragraph{First approach (expected value).}
For a permutation $\sigma$ on $\{1, 2, \dots, n\}$ we define
\[ S_\sigma = \sum_{i=1}^n x_i y_{\sigma(i)}. \]

\begin{claim*}
  For random permutations $\sigma$,
  $\mathbb E[S_\sigma] = 0$
  and $\mathbb E[S_\sigma^2] = \frac{1}{n-1}$.
\end{claim*}
\begin{proof}
  The first one is clear.

  Since $\sum_{i < j} 2x_i x_j = -1$,
  it follows that (for fixed $i$ and $j$),
  $\mathbb E[y_{\sigma(i)} y_{\sigma(j)}] = -\frac{1}{n(n-1)}$,
  Thus
  \begin{align*}
    \sum_{i} x_i^2 \cdot \mathbb E \left[ y_{\sigma(i)}^2 \right]
      &= \frac 1n \\
    2\sum_{i < j} x_i x_j \cdot \mathbb E \left[
      y_{\sigma(i)} y_{\sigma(j)}
      \right] &= (-1) \cdot \frac{1}{n(n-1)}
  \end{align*}
  the conclusion follows.
\end{proof}

%\begin{claim*}
%  For two independent random permutations $\sigma$ and $\tau$,
%  \[ \mathbb E \left[ \left( S_\sigma - S_\tau \right)^2
%    \right] = \frac{2}{n-1}.  \]
%\end{claim*}
%\begin{proof}
%  Since
%  \[ \mathbb E[S_\sigma S_\tau]
%    = \sum_i \sum_{i'} x_i x_{i'}
%      \mathbb E \left[ y_{\sigma(i)} \right]
%      \mathbb E \left[ y_{\tau(i')} \right] = 0 \]
%  the conclusion follows.
%\end{proof}

\begin{claim*}
  [A random variable in {$[0,1]$} has variance at most $1/4$]
  If $A$ is a random variable with mean $\mu$
  taking values in the closed interval $[m, M]$ then
  \[ \mathbb E[(A-\mu)^2] \le \frac14 (M-m)^2. \]
\end{claim*}
\begin{proof}
  By shifting and scaling, we may assume $m = 0$ and $M = 1$,
  so $A \in [0,1]$ and hence $A^2 \le A$.
  Then
  \[
    \mathbb E[(A-\mu)^2]
    = \mathbb E[A^2] - \mu^2
    \le \mathbb E[A] - \mu^2 = \mu-\mu^2 \le \frac14.
  \]
  This concludes the proof.
\end{proof}
Thus the previous two claims together give
\[ \max_\sigma S_\sigma - \min_\sigma S_\sigma
  \ge \sqrt{\frac{4}{n-1}} = \frac{2}{\sqrt{n-1}}.
\]
But $\sum_i x_i y_i = \max_\sigma S_\sigma$
and $\sum_i x_i y_{n+1-i} = \min_\sigma S_\sigma$
by rearrangement inequality
and therefore we are done.

\paragraph{Outline of second approach (by convexity, due to Alex Zhai).}
We will instead prove a converse result: given the hypotheses
\begin{itemize}
  \ii $x_1 \ge \dots \ge x_n$
  \ii $y_1 \ge \dots \ge y_n$
  \ii $\sum_i x_i = \sum_i y_i = 0$
  \ii $\sum_i x_i y_i - \sum_i x_i y_{n+1-i} = \frac{2}{\sqrt{n-1}}$
\end{itemize}
we will prove that $\sum x_i^2 \sum y_i^2 \le 1$.

Fix the choice of $y$'s.
We see that we are trying to maximize a convex function in $n$
variables $(x_1, \dots, x_n)$ over a convex domain
(actually the intersection of two planes with several half planes).
So a maximum can only happen at the boundaries:
when at most two of the $x$'s are different.

An analogous argument applies to $y$.
In this way we find that it suffices to consider situations
where $x_\bullet$ takes on at most two different values.
The same argument applies to $y_\bullet$.

At this point the problem can be checked directly.
\pagebreak


\end{document}
