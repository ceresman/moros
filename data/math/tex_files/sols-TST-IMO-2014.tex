\documentclass[11pt]{scrartcl}
\usepackage[sexy]{evan}
\ihead{\footnotesize\textbf{\thetitle}}
\ohead{\footnotesize\theauthor}
\begin{document}

\title{USA TST 2014 Solution Notes}
\date{\today}

\maketitle
\begin{abstract}
This is a compilation of solutions
for the 2014 USA TST.
Some of the solutions are my own work,
but many are from the official solutions provided by the organizers
(for which they hold any copyrights),
and others were found by users on the Art of Problem Solving forums.

These notes will tend to be a bit more advanced and terse than the ``official''
solutions from the organizers.
In particular, if a theorem or technique is not known to beginners
but is still considered ``standard'', then I often prefer to
use this theory anyways, rather than try to work around or conceal it.
For example, in geometry problems I typically use directed angles
without further comment, rather than awkwardly work around configuration issues.
Similarly, sentences like ``let $\mathbb{R}$ denote the set of real numbers''
are typically omitted entirely.

Corrections and comments are welcome!
\end{abstract}
\tableofcontents
\newpage

\addtocounter{section}{-1}
\section{Problems}
\begin{enumerate}[\bfseries 1.]
\item %% Problem 1
Let $ABC$ be an acute triangle, and let $X$ be a variable interior point
on the minor arc $BC$ of its circumcircle.
Let $P$ and $Q$ be the feet of the perpendiculars from $X$ to lines $CA$ and $CB$, respectively.
Let $R$ be the intersection of line $PQ$ and the perpendicular from $B$ to $AC$.
Let $\ell$ be the line through $P$ parallel to $XR$.
Prove that as $X$ varies along minor arc $BC$,
the line $\ell$ always passes through a fixed point.

\item %% Problem 2
Let $a_1$, $a_2$, $a_3$, \dots be a sequence of integers,
with the property that every consecutive group of $a_i$'s
averages to a perfect square.
More precisely, for all positive integers $n$ and $k$, the quantity
 \[ \frac{a_n + a_{n+1} + \dots + a_{n+k-1}}{k} \]
is always the square of an integer.
Prove that the sequence must be constant
(all $a_i$ are equal to the same perfect square).

\item %% Problem 3
Let $n$ be an even positive integer,
and let $G$ be an $n$-vertex (simple) graph
with exactly $\frac{n^2}{4}$ edges.
An unordered pair of distinct vertices $\{x,y\}$
is said to be \emph{amicable} if they have a common neighbor
(there is a vertex $z$ such that $xz$ and $yz$ are both edges).
Prove that $G$ has at least $2\binom{n/2}{2}$
pairs of vertices which are amicable.

\item %% Problem 4
Let $n$ be a positive even integer,
and let $c_1$, $c_2$, \dots, $c_{n-1}$ be real numbers satisfying
\[ \sum_{i=1}^{n-1} \left\lvert c_i-1 \right\rvert < 1. \]
Prove that
\[ 2x^n - c_{n-1}x^{n-1} + c_{n-2}x^{n-2} - \dots - c_1x^1 + 2 \]
has no real roots.

\item %% Problem 5
Let $ABCD$ be a cyclic quadrilateral,
and let $E$, $F$, $G$, and $H$ be the midpoints of $AB$, $BC$, $CD$, and $DA$ respectively.
Let $W$, $X$, $Y$ and $Z$ be the orthocenters of
triangles $AHE$, $BEF$, $CFG$ and $DGH$, respectively.
Prove that the quadrilaterals $ABCD$ and $WXYZ$ have the same area.

\item %% Problem 6
For a prime $p$, a subset $S$ of residues modulo $p$
is called a \emph{sum-free multiplicative subgroup}
of $\FF_p$ if
\begin{itemize}
  \ii there is a nonzero residue $\alpha$ modulo $p$
  such that $S = \left\{ 1, \alpha^1, \alpha^2, \dots \right\}$
  (all considered mod $p$), and
  \ii there are no $a,b,c \in S$
  (not necessarily distinct) such that $a+b \equiv c \pmod p$.
\end{itemize}
Prove that for every integer $N$,
there is a prime $p$ and a sum-free multiplicative subgroup $S$
of $\FF_p$ such that $\left\lvert S \right\rvert \ge N$.

\end{enumerate}
\pagebreak

\section{Solutions to Day 1}
\subsection{USA TST 2014/1}
\textsl{Available online at \url{https://aops.com/community/p3332310}.}
\begin{mdframed}[style=mdpurplebox,frametitle={Problem statement}]
Let $ABC$ be an acute triangle, and let $X$ be a variable interior point
on the minor arc $BC$ of its circumcircle.
Let $P$ and $Q$ be the feet of the perpendiculars from $X$ to lines $CA$ and $CB$, respectively.
Let $R$ be the intersection of line $PQ$ and the perpendicular from $B$ to $AC$.
Let $\ell$ be the line through $P$ parallel to $XR$.
Prove that as $X$ varies along minor arc $BC$,
the line $\ell$ always passes through a fixed point.
\end{mdframed}
The fixed point is the orthocenter,
since $\ell$ is a Simson line.
See Lemma 4.4 of \emph{Euclidean Geometry in Math Olympiads}.
\pagebreak

\subsection{USA TST 2014/2, proposed by Victor Wang}
\textsl{Available online at \url{https://aops.com/community/p3332299}.}
\begin{mdframed}[style=mdpurplebox,frametitle={Problem statement}]
Let $a_1$, $a_2$, $a_3$, \dots be a sequence of integers,
with the property that every consecutive group of $a_i$'s
averages to a perfect square.
More precisely, for all positive integers $n$ and $k$, the quantity
 \[ \frac{a_n + a_{n+1} + \dots + a_{n+k-1}}{k} \]
is always the square of an integer.
Prove that the sequence must be constant
(all $a_i$ are equal to the same perfect square).
\end{mdframed}
Let $\nu_p(n)$ denote the largest exponent of $p$ dividing $n$.
The problem follows from the following proposition.

\begin{proposition*}
  Let $(a_n)$ be a sequence of integers and let $p$ be a prime.
  Suppose that every consecutive group of $a_i$'s
  with length at most $p$ averages to a perfect square.
  Then $\nu_p(a_i)$ is independent of $i$.
\end{proposition*}

We proceed by induction on the smallest value of $\nu_p(a_i)$ as $i$ ranges
(which must be even, as each of the $a_i$ are themselves a square).
First we prove two claims.

\begin{claim*}
  If $j \equiv k \pmod p$ then $a_j \equiv a_k \pmod{p}$.
\end{claim*}
\begin{proof}
  Taking groups of length $p$ in our given,
  we find that $p \mid a_j +\dots + a_{j+p-1}$
  and $p \mid a_{j+1} + \dots + a_{j+p}$ for any $j$.
  So $a_j \equiv a_{j+p} \pmod p$ and the conclusion follows.
\end{proof}

\begin{claim*}
  If some $a_i$ is divisible by $p$ then all of them are.
\end{claim*}
\begin{proof}
  The case $p=2$ is trivial so assume $p \ge 3$.
  Without loss of generality (via shifting indices)
  assume that $a_1 \equiv 0 \pmod{p}$, and define
  \[ S_n = a_1 + a_2 + \dots + a_n \equiv a_2 + \dots + a_n \pmod{p}. \]
  Call an integer $k$ with $2 \le k < p$ a
  \textbf{pivot} if $1-k^{-1}$ is a quadratic nonresidue modulo $p$.

  We claim that for any pivot $k$, $S_k \equiv 0 \pmod{p}$.
  If not, then
  \[ \frac{a_1 + a_2 + \dots + a_k}{k}
    \text{ and } \frac{a_2 + \dots + a_k}{k-1} \]
  are both qudaratic residues.
  Division implies that $\frac{k-1}{k} = 1 - k\inv$ is a
  quadratic residue, contradiction.

  Next we claim that there is an integer $m$
  with $S_m \equiv S_{m+1} \equiv 0 \pmod{p}$,
  which implies $p \mid a_{m+1}$.
  If $2$ is a pivot, then we simply take $m=1$.
  Otherwise, there are $\frac12(p-1)$ pivots,
  one for each nonresidue (which includes neither $0$ nor $1$),
  and all pivots lie in $[3,p-1]$,
  so we can find an $m$ such that $m$ and $m+1$ are both pivots.

  Repeating this procedure starting with $a_{m+1}$
  shows that $a_{2m+1}, a_{3m+1}, \dots$ must all be divisible by $p$.
  Combined with the first claim and the fact that $m < p$,
  we find that all the $a_i$ are divisible by $p$.
\end{proof}

The second claim establishes the base case of our induction.
Now assume all $a_i$ are divisible by $p$ and hence $p^2$.
Then all the averages in our proposition (with length at most $p$)
are divisible by $p$ and hence $p^2$.
Thus the map $a_i \mapsto \frac{1}{p^2} a_i$ gives a new sequence
satisfying the proposition, and our inductive hypothesis completes the proof.

\begin{remark*}
  There is a subtle bug that arises
  if one omits the condition that $k \le p$ in the proposition.
  When $k = p^2$ the average $\frac{a_1 + \dots + a_{p^2}}{p^2}$
  is not necessarily divisible by $p$
  even if all the $a_i$ are.
  Hence it is not valid to divide through by $p$.
  This is why the condition $k \le p$ was added.
\end{remark*}
\pagebreak

\subsection{USA TST 2014/3}
\textsl{Available online at \url{https://aops.com/community/p3332307}.}
\begin{mdframed}[style=mdpurplebox,frametitle={Problem statement}]
Let $n$ be an even positive integer,
and let $G$ be an $n$-vertex (simple) graph
with exactly $\frac{n^2}{4}$ edges.
An unordered pair of distinct vertices $\{x,y\}$
is said to be \emph{amicable} if they have a common neighbor
(there is a vertex $z$ such that $xz$ and $yz$ are both edges).
Prove that $G$ has at least $2\binom{n/2}{2}$
pairs of vertices which are amicable.
\end{mdframed}
First, we prove the following lemma.
(\url{https://en.wikipedia.org/wiki/Friendship_paradox}).
\begin{lemma*}
  [On average, your friends are more popular than you]
  For a vertex $v$, let $a(v)$ denote
  the average degree of the neighbors of $v$
  (setting $a(v) = 0$ if $\deg v = 0$).
  Then
  \[ \sum_v a(v) \ge \sum_v \deg v = 2 \# E. \]
\end{lemma*}
\begin{proof}
  Ignoring isolated vertices, we can write
  \begin{align*}
    \sum_v a(v)
    &= \sum_v \frac{\sum_{w \sim v} \deg w}{\deg v} \\
    &= \sum_{v} \sum_{w \sim v} \frac{\deg w}{\deg v} \\
    &= \sum_{\text{edges } vw} \left(
      \frac{\deg w}{\deg v} + \frac{\deg v}{\deg w} \right) \\
    &\overset{\text{AM-GM}}{\ge} \sum_{\text{edges } vw} 2
      = 2\#E = \sum_{v} \deg v
  \end{align*}
  as desired.
\end{proof}
\begin{corollary*}
  [On average, your most popular friend is more popular than you]
  For a vertex $v$, let $m(v)$ denote
  the maximum degree of the neighbors of $v$
  (setting $m(v) = 0$ if $\deg v = 0$).
  Then
  \[ \sum_v m(v) \ge \sum_v \deg v = 2 \# E. \]
\end{corollary*}

We can use this to count amicable pairs
by noting that any particular vertex $v$ is in
at least $m(v)-1$ amicable pairs.
So, the number of amicable pairs is at least
\[ \half \sum_v (m(v)-1) \ge \# E - \half \# V. \]

Note that up until now we haven't used any information about $G$.
But now if we plug in $\# E = n^2/4$, $\# V = n$,
then we get exactly the desired answer.
(Equality holds for $G = K_{n/2, n/2}$.)
\pagebreak

\section{Solutions to Day 2}
\subsection{USA TST 2014/4}
\textsl{Available online at \url{https://aops.com/community/p3476290}.}
\begin{mdframed}[style=mdpurplebox,frametitle={Problem statement}]
Let $n$ be a positive even integer,
and let $c_1$, $c_2$, \dots, $c_{n-1}$ be real numbers satisfying
\[ \sum_{i=1}^{n-1} \left\lvert c_i-1 \right\rvert < 1. \]
Prove that
\[ 2x^n - c_{n-1}x^{n-1} + c_{n-2}x^{n-2} - \dots - c_1x^1 + 2 \]
has no real roots.
\end{mdframed}
We will prove the polynomial is positive for all $x \in \RR$.
As $c_i > 0$, the result is vacuous for $x \le 0$,
so we restrict attention to $x > 0$.

Then letting $c_i = 1 - d_i$ for each $i$,
the inequality we want to prove becomes
\[ x^n + 1 + \frac{x^{n+1}+1}{x+1} > \sum_1^{n-1} d_i x^i
  \qquad\text{given } \sum |d_i| < 1.  \]
But obviously $x^n+1 > x^i$ for any $1 \le i \le n-1$ and $x > 0$.
So in fact $x^n+1 > \sum_1^{n-1} |d_i| x^i$ holds for $x > 0$,
as needed.
\pagebreak

\subsection{USA TST 2014/5, proposed by Po-Shen Loh}
\textsl{Available online at \url{https://aops.com/community/p3476291}.}
\begin{mdframed}[style=mdpurplebox,frametitle={Problem statement}]
Let $ABCD$ be a cyclic quadrilateral,
and let $E$, $F$, $G$, and $H$ be the midpoints of $AB$, $BC$, $CD$, and $DA$ respectively.
Let $W$, $X$, $Y$ and $Z$ be the orthocenters of
triangles $AHE$, $BEF$, $CFG$ and $DGH$, respectively.
Prove that the quadrilaterals $ABCD$ and $WXYZ$ have the same area.
\end{mdframed}
The following solution is due to Grace Wang.

We begin with:
\begin{claim*}
  Point $W$ has coordinates $\half(2a+b+d)$.
\end{claim*}
\begin{proof}
  The orthocenter of $\triangle DAB$ is $d+a+b$,
  and $\triangle AHE$ is homothetic to $\triangle DAB$
  through $A$ with ratio $1/2$.
  Hence $w = \half(a+(d+a+b))$ as needed.
\end{proof}

By symmetry, we have
\begin{align*}
  w &= \half(2a+b+d) \\
  x &= \half(2b+c+a) \\
  y &= \half(2c+d+b) \\
  z &= \half(2d+a+c).
\end{align*}
We see that $w-y = a-c$, $x-z = b-d$.
So the diagonals of $WXYZ$ have the same length as those
of $ABCD$ as well as the same directed angle between them.
This implies the areas are equal, too.
\pagebreak

\subsection{USA TST 2014/6}
\textsl{Available online at \url{https://aops.com/community/p3476292}.}
\begin{mdframed}[style=mdpurplebox,frametitle={Problem statement}]
For a prime $p$, a subset $S$ of residues modulo $p$
is called a \emph{sum-free multiplicative subgroup}
of $\FF_p$ if
\begin{itemize}
  \ii there is a nonzero residue $\alpha$ modulo $p$
  such that $S = \left\{ 1, \alpha^1, \alpha^2, \dots \right\}$
  (all considered mod $p$), and
  \ii there are no $a,b,c \in S$
  (not necessarily distinct) such that $a+b \equiv c \pmod p$.
\end{itemize}
Prove that for every integer $N$,
there is a prime $p$ and a sum-free multiplicative subgroup $S$
of $\FF_p$ such that $\left\lvert S \right\rvert \ge N$.
\end{mdframed}
We first prove the following general lemma.

\begin{lemma*}
  If $f,g \in \ZZ[X]$ are relatively prime nonconstant polynomials,
  then for sufficiently large primes $p$,
  they have no common root modulo $p$.
\end{lemma*}
\begin{proof}
  By B\'{e}zout Lemma,
  there exist polynomials $a(X)$ and $b(X)$ in $\ZZ[X]$
  and a nonzero constant $c \in \ZZ$ satisfying the identity
  \[ a(X) f(X) + b(X) g(X) \equiv c. \]
  So, plugging in $X = r$ we get $p \mid c$,
  so the set of permissible primes $p$ is finite.
\end{proof}

With this we can give the construction.
\begin{claim*}
  Suppose that
  \begin{itemize}
    \ii $n$ is a positive integer with $n \not\equiv 0 \pmod 3$;
    \ii $p$ is a prime which is $1 \bmod n$; and
    \ii $\alpha$ is a primitive $n$'th root of unity modulo $p$.
  \end{itemize}
  Then $|S| = n$ and, if $p$ is sufficiently large in $n$,
  is also sum-free.
\end{claim*}
\begin{proof}
  The assertion $|S|=n$ is immediate from the choice of $\alpha$.
  As for sum-free, assume for contradiction that
  \[ 1 + \alpha^k \equiv \alpha^m \pmod p \]
  for some integers $k,m \in \ZZ$.
  This means $(X+1)^n-1$ and $X^n-1$ have common root $X=\alpha^k$.

  But
  \[ \gcd_{\ZZ[x]} \Big( (X+1)^n-1, \; X^n-1 \Big)
    = 1 \qquad \forall n \not\equiv 0 \pmod 3 \]
  because when $3 \nmid n$ the two polynomials
  have no common complex roots.
  (Indeed, if $|\omega| = |1+\omega| = 1$
  then $\omega = -\half \pm \frac{\sqrt3}{2}i$.)

  Thus $p$ is bounded by the lemma, as desired.
\end{proof}
\pagebreak


\end{document}
