% © Evan Chen
% Downloaded from https://web.evanchen.cc/

\documentclass[11pt]{scrartcl}
\usepackage[sexy]{evan}
\ihead{\footnotesize\textbf{\thetitle}}
\ohead{\footnotesize\href{http://web.evanchen.cc}{\ttfamily web.evanchen.cc},
    updated \today}
\title{USAMO 1999 Solution Notes}
\date{\today}

\begin{document}

\maketitle

\begin{abstract}
This is a compilation of solutions
for the 1999 USAMO.
Some of the solutions are my own work,
but many are from the official solutions provided by the organizers
(for which they hold any copyrights),
and others were found by users on the Art of Problem Solving forums.

These notes will tend to be a bit more advanced and terse than the ``official''
solutions from the organizers.
In particular, if a theorem or technique is not known to beginners
but is still considered ``standard'', then I often prefer to
use this theory anyways, rather than try to work around or conceal it.
For example, in geometry problems I typically use directed angles
without further comment, rather than awkwardly work around configuration issues.
Similarly, sentences like ``let $\mathbb{R}$ denote the set of real numbers''
are typically omitted entirely.

Corrections and comments are welcome!
\end{abstract}

\tableofcontents
\newpage

\addtocounter{section}{-1}
\section{Problems}
\begin{enumerate}[\bfseries 1.]
\item %% Problem 1
Some checkers placed on an $n \times n$ checkerboard satisfy the following conditions:
\begin{enumerate}
  \ii[(a)] every square that does not contain a checker shares a side with one that does;
  \ii[(b)] given any pair of squares that contain checkers,
  there is a sequence of squares containing checkers,
  starting and ending with the given squares,
  such that every two consecutive squares of the sequence share a side.
\end{enumerate}
Prove that at least $(n^{2}-2)/3$ checkers have been placed on the board.

\item %% Problem 2
Let $ABCD$ be a convex cyclic quadrilateral.
Prove that \[ |AB - CD| + |AD - BC| \geq 2|AC - BD|. \]

\item %% Problem 3
Let $p > 2$ be a prime and let $a$, $b$, $c$, $d$ be integers not divisible by $p$,
such that
\[ \left\{ \dfrac{ra}{p} \right\} + \left\{ \dfrac{rb}{p} \right\}
  + \left\{ \dfrac{rc}{p} \right\} + \left\{ \dfrac{rd}{p} \right\} = 2  \]
for any integer $r$ not divisible by $p$.
(Here, $\{t\} = t - \left\lfloor t \right\rfloor$ is the fractional part.)
Prove that at least two of the numbers
$a+b$, $a+c$, $a+d$, $b+c$, $b+d$, $c+d$ are divisible by $p$.

\item %% Problem 4
Let $a_1$, $a_2$, \dots, $a_n$ be a sequence of $n > 3$ real numbers
such that
\[ a_1 + \dots + a_n \ge n \quad\text{and}\quad
  a_1^2 + \dots + a_n^2 \ge n^2. \]
Prove that $\max(a_1, \dots, a_n) \ge 2$.

\item %% Problem 5
The Y2K Game is played on a $1 \times 2000$ grid as follows.
Two players in turn write either an S or an O in an empty square.
The first player who produces three consecutive boxes that spell SOS wins.
If all boxes are filled without producing SOS then the game is a draw.
Prove that the second player has a winning strategy.

\item %% Problem 6
Let $ABCD$ be an isosceles trapezoid with $AB \parallel CD$.
The inscribed circle $\omega$ of triangle $BCD$ meets $CD$ at $E$.
Let $F$ be a point on the (internal) angle bisector of $\angle DAC$
such that $EF \perp CD$.
Let the circumscribed circle of triangle $ACF$
meet line $CD$ at $C$ and $G$.
Prove that the triangle $AFG$ is isosceles.

\end{enumerate}
\pagebreak

\section{Solutions to Day 1}
\subsection{USAMO 1999/1}
\textsl{Available online at \url{https://aops.com/community/p340035}.}
\begin{mdframed}[style=mdpurplebox,frametitle={Problem statement}]
Some checkers placed on an $n \times n$ checkerboard satisfy the following conditions:
\begin{enumerate}
  \ii[(a)] every square that does not contain a checker shares a side with one that does;
  \ii[(b)] given any pair of squares that contain checkers,
  there is a sequence of squares containing checkers,
  starting and ending with the given squares,
  such that every two consecutive squares of the sequence share a side.
\end{enumerate}
Prove that at least $(n^{2}-2)/3$ checkers have been placed on the board.
\end{mdframed}
Take a spanning tree on the set $V$ of checkers
where the $|V|-1$ edges of the tree are given by orthogonal adjacency.
By condition (a) we have
\[ \sum_{v \in V} (4-\deg v) \ge n^2 - |V| \]
and since $\sum_{v \in V} \deg v = 2(|V|-1)$
we get
\[ 4|V| - \left( 2|V|-2 \right) \ge n^2 - |V| \]
which implies $|V| \ge \frac{n^2-2}{3}$.
\pagebreak

\subsection{USAMO 1999/2}
\textsl{Available online at \url{https://aops.com/community/p340036}.}
\begin{mdframed}[style=mdpurplebox,frametitle={Problem statement}]
Let $ABCD$ be a convex cyclic quadrilateral.
Prove that \[ |AB - CD| + |AD - BC| \geq 2|AC - BD|. \]
\end{mdframed}
Let the diagonals meet at $P$,
and let $AP = pq$, $DP = pr$, $BP = qs$, $CP = rs$.
Then set $AB = qx$, $CD = rx$, $AD = py$, $BC = sy$.


In this way we compute
\[ \left\lvert AC-BD \right\rvert
  = \left\lvert (p-s)(q-r) \right\rvert  \]
and
\[ \left\lvert AB-CD \right\rvert
  = \left\lvert q-r \right\rvert x. \]
By triangle inequality on $\triangle AXB$,
we have $x \ge \left\lvert p-s \right\rvert$.
So $\left\lvert AB-CD \right\rvert \ge \left\lvert AC-BD \right\rvert$.
\pagebreak

\subsection{USAMO 1999/3}
\textsl{Available online at \url{https://aops.com/community/p340038}.}
\begin{mdframed}[style=mdpurplebox,frametitle={Problem statement}]
Let $p > 2$ be a prime and let $a$, $b$, $c$, $d$ be integers not divisible by $p$,
such that
\[ \left\{ \dfrac{ra}{p} \right\} + \left\{ \dfrac{rb}{p} \right\}
  + \left\{ \dfrac{rc}{p} \right\} + \left\{ \dfrac{rd}{p} \right\} = 2  \]
for any integer $r$ not divisible by $p$.
(Here, $\{t\} = t - \left\lfloor t \right\rfloor$ is the fractional part.)
Prove that at least two of the numbers
$a+b$, $a+c$, $a+d$, $b+c$, $b+d$, $c+d$ are divisible by $p$.
\end{mdframed}
First of all, we apparently have $r(a+b+c+d) \equiv 0 \pmod p$
for every prime $p$, so it automatically follows that $a+b+c+d \equiv 0 \pmod p$.
By scaling appropriately, and also replacing each number with its remainder modulo $p$,
we are going to assume that
\[ 1 = a \le b \le c \le d < p. \]
We are going to prove that $d = p-1$, which will solve the problem.

\begin{claim*}
  For each integer $r = 1, 2, \dots, p-1$ we have
  \[ 2(r-1) = \left\lfloor \frac{rb}{p} \right\rfloor
    + \left\lfloor \frac{rc}{p} \right\rfloor
    + \left\lfloor \frac{rd}{p} \right\rfloor. \]
\end{claim*}
\begin{proof}
  By plugging in $r=1$ to the given we have $a+b+c+d=2p$.
  Now, we have
  \[ 2 = \sum_{\text{cyc}} \left( \frac{ra}{p}
    - \left\lfloor \frac{ra}{p} \right\rfloor \right) \]
  and since $a+b+c+d=2p$ the conclusion follows.
\end{proof}

%We now define $x = b/p$, $y = c/p$, $z = d/p$.
%Hence $0 < x \le y \le z < 1$.
%Our goal it show that $z = (p-1)/p$.

We vaguely outline the approach now, before giving a formalization.
Imagine the interval $[0,1]$.
One by one, for each $r = 1, 2, 3, \dots, p-1$,
we mark the fractions with denominator $r$ on this number line;
the resulting pictures may be better known as \emph{Farey fractions}.
At each step, we can place the three numbers $b/p$, $c/p$, $d/p$
into one of the resulting sub-intervals.
Our goal is to show that $d/p$ is always in the rightmost interval,
while $b/p$ and $c/p$ are always to the right of symmetrically mirrored points.
An example of a possible diagram is shown below (not to scale).
\begin{center}
\begin{asy}
  unitsize(11cm);
  real h = 0.05;
  draw( (0,0)--(1,0) );
  void tick(string s, real x, real t, pen p) {
    label(s, (x,-t), dir(-90), p);
    draw((x,-t)--(x,t), p);
  }
  tick("$0$", 0, h, black);
  tick("$1$", 1, h, black);
  tick("$\frac{1}{2}$", 1/2, h, black);
  tick("$\frac{1}{3}$", 1/3, 0.7*h, blue);
  tick("$\frac{2}{3}$", 2/3, 0.7*h, blue);
  tick("$\frac{1}{4}$", 1/4, 0.5*h, deepgreen);
  tick("$\frac{3}{4}$", 3/4, 0.5*h, deepgreen);
  tick("$\frac{1}{5}$", 1/6, 0.3*h, grey); // white lie
  tick("$\frac{2}{5}$", 2/5, 0.3*h, grey);
  tick("$\frac{3}{5}$", 3/5, 0.3*h, grey);
  tick("$\frac{4}{5}$", 5/6, 0.3*h, grey); // white lie
  dot("$\boxed{\frac{b}{p}}$", (0.3,0), dir(90), red);
  dot("$\boxed{\frac{c}{p}}$", (0.72,0), dir(90), red);
  dot("$\boxed{\frac{d}{p}}$", (0.95,0), dir(90), red);
\end{asy}
\end{center}

In symbols, it will be enough to prove the following.
\begin{claim*}
  For each $r = 1, 2, \dots, p-2$ we have
  $\frac{r-1}{r} < \frac dp < 1$.

  Equivalently, for each $r = 1, 2, \dots, p-2$
  we have $\left\lfloor \frac{rb}{p} \right\rfloor
  + \left\lfloor \frac{rc}{p} \right\rfloor = r-1$.
\end{claim*}
\begin{proof}
  Assume this is not true and take the minimal counterexample $r > 1$.
  Then evidently
  \[ r-1 > \left\lfloor \frac{rd}{p} \right\rfloor
    \ge \left\lfloor \frac{(r-1)d}{p} \right\rfloor
    = r-2.  \]
  Now, we have that
  \[ 2(r-1) = \left\lfloor \frac{rb}{p} \right\rfloor
    + \left\lfloor \frac{rc}{p} \right\rfloor
    + \underbrace{\left\lfloor \frac{rd}{p} \right\rfloor}_{= r-2}.  \]
  Thus
  $\left\lfloor \frac{rb}{p} \right\rfloor >
  \left\lfloor \frac{(r-1)b}{p} \right\rfloor$,
  and $\left\lfloor \frac{rc}{p} \right\rfloor >
  \left\lfloor \frac{(r-1)b}{p} \right\rfloor$.
  An example of this situation is illustrated below with $r = 7$
  (not to scale).
  \begin{center}
  \begin{asy}
    unitsize(11cm);
    real h = 0.05;
    draw( (0,0)--(1,0) );
    void tick(string s, real x, real t, pen p) {
      label(s, (x,-t), dir(-90), p);
      draw((x,-t)--(x,t), p);
    }
    tick("$0$", 0, h, black);
    tick("$1$", 1, h, black);
    tick("$\frac{1}{2}$", 1/2, h, black);
    tick("$\frac{1}{3}$", 1/3, 0.7*h, blue);
    tick("$\frac{2}{3}$", 2/3, 0.7*h, blue);
    tick("$\frac{1}{4}$", 1/4, 0.5*h, deepgreen);
    tick("$\frac{3}{4}$", 3/4, 0.5*h, deepgreen);
    tick("$\frac{1}{5}$", 1/6, 0.3*h, grey);
    tick("$\frac{2}{5}$", 2/5, 0.3*h, grey);
    tick("$\frac{3}{5}$", 3/5, 0.3*h, grey);
    tick("$\frac{4}{5}$", 5/6, 0.3*h, grey);
    tick("$\frac{1}{6}$", 1/8, 0.3*h, grey);
    tick("$\frac{5}{6}$", 7/8, 0.3*h, grey);
    dot("$\boxed{\frac{b}{p}}$", (0.31,0), dir(90), red);
    dot("$\boxed{\frac{c}{p}}$", (0.725,0), dir(90), red);
    dot("$\boxed{\frac{d}{p}}$", (0.92,0), dir(90), red);
    tick("$\dfrac{2}{7}$", 2/7, 1.5*h, red+1);
    tick("$\dfrac{5}{7}$", 0.7, 1.5*h, red+1);
    tick("$\dfrac{6}{7}$", 0.9, 1.5*h, red+1);

    tick("$\dfrac{7}{8}$", 0.95, 2*h, orange+1);
    tick("$\dfrac{3}{8}$", 0.375, 2*h, orange+1);
    tick("$\dfrac{6}{8}$", 0.75, 2*h, orange+1);
  \end{asy}
  \end{center}
  Right now, $\frac bp$ and $\frac cp$
  are just to the right of
  $\frac ur$ and $\frac vr$ for some $u$ and $v$ with $u+v=r$.
  The issue is that the there is some fraction just to the right of
  $\frac bp$ and $\frac cp$ from an earlier value of $r$,
  and by hypothesis its denominator is going to be strictly greater than $1$.

  It is at this point we are going to use the properties of Farey sequences.
  When we consider the fractions with denominator $r+1$,
  they are going to lie outside of the interval
  they we have constrained $\frac bp$ and $\frac cp$ to lie in.

  Indeed, our minimality assumption on $r$ guarantees that there is no
  fraction with denominator less than $r$ between $\frac ur$ and $\frac bp$.
  So if $\frac ur < \frac bp < \frac st$
  (where $\frac ur$ and $\frac st$ are the closest fractions
  with denominator at most $r$ to $\frac bp$)
  then Farey theory says the next fraction
  inside the interval $[\frac ur, \frac st]$
  is $\frac{u+s}{r+t}$, and since $t > 1$, we have $r+t > r+1$.
  In other words, we get an inequality of the form
  \[ \frac ur < \frac bp < \underbrace{\text{something}}_{=s/t} \le \frac{u+1}{r+1}. \]

  The same holds for $\frac cp$ as
  \[ \frac vr < \frac cp < \text{something} \le \frac{v+1}{r+1}. \]
  Finally,
  \[ \frac dp < \frac{r-1}{r} < \frac{r}{r+1}. \]
  So now we have that
  \[ \left\lfloor \frac{(r+1)b}{p} \right\rfloor
    + \left\lfloor \frac{(r+1)c}{p} \right\rfloor
    + \left\lfloor \frac{(r+1)d}{p} \right\rfloor
    \le u + v + (r-1) = 2r-1 \]
  which is a contradiction.
\end{proof}
Now, since
\[ \frac{p-3}{p-2} < \frac dp
  \implies d > \frac{p(p-3)}{p-2} = p - 1 - \frac{2}{p-2} \]
which for $p > 2$ gives $d = p-1$.
\pagebreak

\section{Solutions to Day 2}
\subsection{USAMO 1999/4}
\textsl{Available online at \url{https://aops.com/community/p63591}.}
\begin{mdframed}[style=mdpurplebox,frametitle={Problem statement}]
Let $a_1$, $a_2$, \dots, $a_n$ be a sequence of $n > 3$ real numbers
such that
\[ a_1 + \dots + a_n \ge n \quad\text{and}\quad
  a_1^2 + \dots + a_n^2 \ge n^2. \]
Prove that $\max(a_1, \dots, a_n) \ge 2$.
\end{mdframed}
Proceed by contradiction, assuming $a_i < 2$ for all $i$.

If all $a_i \ge 0$, then $ n^2 \le \sum_i a_i^2 < n \cdot 2^2$,
contradiction.

Otherwise, assume at least one $a_i$ is negative.
Note that if $-x$ and $-y$ are both present in the sequence ($x,y>0$),
then we can replace them with $-(x+y)$ and $0$.
So we may assume that there is exactly one negative term, say $a_n = -M$.

Now, smooth all the nonnegative $a_i$ to be $2$,
making all inequalities strict.
Now, we have that
\begin{align*}
  2(n-1) - M &> n \\
  4(n-1) + M^2 &> n^2.
\end{align*}
This gives $n-2 < M < n-2$, contradiction.

Equality in the original occurs when $n-1$ of the $a_i$ are equal to $2$
and the last one is equal to $-(n-2)$.
\pagebreak

\subsection{USAMO 1999/5}
\textsl{Available online at \url{https://aops.com/community/p340040}.}
\begin{mdframed}[style=mdpurplebox,frametitle={Problem statement}]
The Y2K Game is played on a $1 \times 2000$ grid as follows.
Two players in turn write either an S or an O in an empty square.
The first player who produces three consecutive boxes that spell SOS wins.
If all boxes are filled without producing SOS then the game is a draw.
Prove that the second player has a winning strategy.
\end{mdframed}
The main insight is that a construct of the form
\[ S \; \square \; \square \; S \]
(here the $\square$ is blank)
will kill any player which plays inside it.
We call this a \emph{trap} accordingly.

\begin{claim*}
  The second player can force a trap to exist;
  in this case the game will never end in a draw.
\end{claim*}
\begin{proof}
  Actually the second player can construct a trap
  on her second turn by playing an $S$ far enough away
  from the edges of the board and the first player's initial move.
\end{proof}

\begin{claim*}
  The second player always has a move which prevents
  her from losing.
\end{claim*}
\begin{proof}
  Since there are an odd number of empty squares at the start of
  the second player's turn,
  there must be a square which is bordered by either
  two filled or two empty squares.
  The second player can then play $O$ in this square,
  which is always safe.
\end{proof}

Together these two claims finish the problem.

\begin{remark*}
  Actually, one can show that the ``only'' way to lose
  is to be forced to play inside a trap.
  Indeed, suppose playing in a certain cell $c$ loses.
  If we wrote $O$, that means $c$ is bordered by exactly one $S$,
  with a blank cell on the neighbor.
  But we could also write $S$; checking cases we find $c$
  is part of a trap.

  Thus a player can lose only if all blank cells
  are in traps; ergo, the number of blank cells is even.
  This never happens for the second player.
  Thus this gives an alternative solution,
  and moreover a reason to believe that all
  correct solutions must involve traps.

  A similar proof shows that for large $n$, with a $1 \times n$ board,
  the first player has a winning strategy for odd $n$,
  and the second player has a winning strategy for even $n$.
\end{remark*}
\pagebreak

\subsection{USAMO 1999/6}
\textsl{Available online at \url{https://aops.com/community/p340041}.}
\begin{mdframed}[style=mdpurplebox,frametitle={Problem statement}]
Let $ABCD$ be an isosceles trapezoid with $AB \parallel CD$.
The inscribed circle $\omega$ of triangle $BCD$ meets $CD$ at $E$.
Let $F$ be a point on the (internal) angle bisector of $\angle DAC$
such that $EF \perp CD$.
Let the circumscribed circle of triangle $ACF$
meet line $CD$ at $C$ and $G$.
Prove that the triangle $AFG$ is isosceles.
\end{mdframed}
Note $E$ is contact point of $A$-excircle of $\triangle ACD$,
so $F$ is $A$-excenter.
Hence $CF$ is external angle bisector of $\angle ACG$
which implies $FA = FG$
(since $F$ is the arc midpoint on the circumcircle of $AFG$).
\pagebreak


\end{document}
