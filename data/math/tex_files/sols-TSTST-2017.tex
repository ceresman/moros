\documentclass[11pt]{scrartcl}
\usepackage[sexy]{evan}
\ihead{\footnotesize\textbf{\thetitle}}
\ohead{\footnotesize\theauthor}
\begin{document}

\title{USA TSTST 2017 Solutions}
\subtitle{United States of America --- TST Selection Test}
\date{59\ts{th} IMO 2018 Romania and 7\ts{th} EGMO 2018 Italy}

\maketitle

\tableofcontents
\newpage

\addtocounter{section}{-1}
\section{Problems}
\begin{enumerate}[\bfseries 1.]
\item %% Problem 1
Let $ABC$ be a triangle with circumcircle $\Gamma$,
circumcenter $O$, and orthocenter $H$.
Assume that $AB \neq AC$ and $\angle A \neq 90\dg$.
Let $M$ and $N$ be the midpoints of $\ol{AB}$ and $\ol{AC}$,
respectively, and let $E$ and $F$ be the feet of the altitudes
from $B$ and $C$ in $\triangle ABC$, respectively.
Let $P$ be the intersection point of line $MN$
with the tangent line to $\Gamma$ at $A$.
Let $Q$ be the intersection point,
other than $A$, of $\Gamma$ with the circumcircle of $\triangle AEF$.
Let $R$ be the intersection point of lines $AQ$ and $EF$.
Prove that $\ol{PR} \perp \ol{OH}$.

\item %% Problem 2
Ana and Banana are playing a game. First Ana picks a word,
which is defined to be a nonempty sequence of capital English letters.
Then Banana picks a nonnegative integer $k$ and challenges Ana
to supply a word with exactly $k$ subsequences which are equal to Ana's word.
Ana wins if she is able to supply such a word, otherwise she loses.
For example, if Ana picks the word ``TST'', and Banana chooses $k = 4$,
then Ana can supply the word ``TSTST'' which has $4$ subsequences
which are equal to Ana's word.
Which words can Ana pick so that she can win no matter what value of
$k$ Banana chooses?

\item %% Problem 3
Consider solutions to the equation
\[ x^2 - cx + 1 = \frac{f(x)}{g(x)} \]
where $f$ and $g$ are nonzero polynomials with nonnegative real coefficients.
For each $c > 0$, determine the minimum possible degree of $f$, or show
that no such $f$, $g$ exist.

\item %% Problem 4
Find all nonnegative integer solutions to \[ 2^a + 3^b + 5^c = n!. \]

\item %% Problem 5
Let $ABC$ be a triangle with incenter $I$.
Let $D$ be a point on side $BC$ and let $\omega_B$ and $\omega_C$
be the incircles of $\triangle ABD$ and $\triangle ACD$, respectively.
Suppose that $\omega_B$ and $\omega_C$ are tangent to segment $BC$
at points $E$ and $F$, respectively.
Let $P$ be the intersection of segment $AD$ with the
line joining the centers of $\omega_B$ and $\omega_C$.
Let $X$ be the intersection point of lines $BI$ and $CP$
and let $Y$ be the intersection point of lines $CI$ and $BP$.
Prove that lines $EX$ and $FY$ meet on the incircle of $\triangle ABC$.

\item %% Problem 6
A sequence of positive integers $(a_{n})_{n \geq 1}$ is of
\emph{Fibonacci type} if it satisfies the recursive relation
$a_{n+2}=a_{n+1}+a_{n}$ for all $n \geq 1$.
Is it possible to partition the set of positive integers
into an infinite number of Fibonacci type sequences?

\end{enumerate}
\pagebreak

\section{Solutions to Day 1}
\subsection{TSTST 2017/1, proposed by Ray Li}
\textsl{Available online at \url{https://aops.com/community/p8526098}.}
\begin{mdframed}[style=mdpurplebox,frametitle={Problem statement}]
Let $ABC$ be a triangle with circumcircle $\Gamma$,
circumcenter $O$, and orthocenter $H$.
Assume that $AB \neq AC$ and $\angle A \neq 90\dg$.
Let $M$ and $N$ be the midpoints of $\ol{AB}$ and $\ol{AC}$,
respectively, and let $E$ and $F$ be the feet of the altitudes
from $B$ and $C$ in $\triangle ABC$, respectively.
Let $P$ be the intersection point of line $MN$
with the tangent line to $\Gamma$ at $A$.
Let $Q$ be the intersection point,
other than $A$, of $\Gamma$ with the circumcircle of $\triangle AEF$.
Let $R$ be the intersection point of lines $AQ$ and $EF$.
Prove that $\ol{PR} \perp \ol{OH}$.
\end{mdframed}
\paragraph{First solution (power of a point).}
Let $\gamma$ denote the nine-point circle of $ABC$.
\begin{center}
\begin{asy}
pair A = dir(125);
pair B = dir(210);
pair C = dir(330);
pair M = midpoint(A--B);
pair N = midpoint(A--C);
pair O = origin;
pair H = A+B+C;

draw(A--B--C--cycle, blue);
pair E = foot(B, A, C);
pair F = foot(C, A, B);
draw(B--E, lightblue);
draw(C--F, lightblue);
pair R = extension(E, F, B, C);
pair Q = -A+2*foot(O, A, R);

filldraw(unitcircle, opacity(0.1)+lightcyan, blue);
draw(A--R--E, heavygreen);
pair P = extension(M, N, A, A+dir(90)*A);
draw(A--P--N, red);
filldraw(circumcircle(A, M, N), opacity(0.1)+lightred, red);
filldraw(circumcircle(A, E, F), opacity(0.1)+lightgreen, heavygreen);

filldraw(circumcircle(M, N, E), opacity(0.1)+lightcyan, heavycyan);

dot("$A$", A, dir(A));
dot("$B$", B, dir(B));
dot("$C$", C, dir(C));
dot("$M$", M, dir(145));
dot("$N$", N, dir(20));
dot("$O$", O, dir(315));
dot("$H$", H, dir(H));
dot("$E$", E, dir(40));
dot("$F$", F, dir(F));
dot("$R$", R, dir(R));
dot("$Q$", Q, dir(Q));
dot("$P$", P, dir(P));

/* TSQ Source:

A = dir 125
B = dir 210
C = dir 330
M = midpoint A--B R145
N = midpoint A--C R20
O = origin R315
H = A+B+C

A--B--C--cycle blue
E = foot B A C R40
F = foot C A B
B--E lightblue
C--F lightblue
R = extension E F B C
Q = -A+2*foot O A R

unitcircle 0.1 lightcyan / blue
A--R--E heavygreen
P = extension M N A A+dir(90)*A
A--P--N red
circumcircle A M N 0.1 lightred / red
circumcircle A E F 0.1 lightgreen / heavygreen

circumcircle M N E 0.1 lightcyan / heavycyan

*/
\end{asy}
\end{center}
Note that
\begin{itemize}
  \ii $PA^2 = PM \cdot PN$,
  so $P$ lies on the radical axis of $\Gamma$ and $\gamma$.
  \ii $RA \cdot RQ = RE \cdot RF$,
  so $R$ lies on the radical axis of $\Gamma$ and $\gamma$.
\end{itemize}
Thus $\ol{PR}$ is the radical axis of $\Gamma$ and $\gamma$,
which is evidently perpendicular to $\ol{OH}$.

\begin{remark*}
  In fact, by power of a point one may also observe
  that $R$ lies on $\ol{BC}$,
  since it is on the radical axis of $(AQFHE)$, $(BFEC)$, $(ABC)$.
  Ironically, this fact is not used in the solution.
\end{remark*}

\paragraph{Second solution (barycentric coordinates).}
Again note first $R \in \ol{BC}$ (although this can be avoided too).
We compute the points in much the same way as before.
Since $\ol{AP} \cap \ol{BC} = (0 : b^2 : -c^2)$
we have \[ P = \left( b^2-c^2 : b^2 : -c^2  \right) \]
(since $x=y+z$ is the equation of line $\ol{MN}$).
Now in Conway notation we have
\[ R = \ol{EF} \cap \ol{BC} = (0 : S_C : -S_B)
  =\left( 0 : a^2+b^2-c^2 : -a^2+b^2-c^2 \right). \]
Hence
\[ \overrightarrow{PR}
  = \frac{1}{2(b^2-c^2)} \left( b^2-c^2 , c^2-a^2 , a^2-b^2 \right).  \]
On the other hand, we have
$\overrightarrow{OH} = \vec A + \vec B + \vec C$.
So it suffices to check that
\[ \sum_{\text{cyc}} a^2\left( (a^2-b^2) + (c^2-a^2) \right) = 0 \]
which is immediate.

\paragraph{Third solution (complex numbers).}
Let $ABC$ be the unit circle.
We first compute $P$ as the midpoint of $A$ and $\ol{AA} \cap \ol{BC}$:
\begin{align*}
  p &= \half \left( a + \frac{a^2(b+c)-bc\cdot2a}{a^2-bc} \right) \\
  &= \frac{a(a^2-bc)+a^2(b+c)-2abc}{2(a^2-bc)}.
\end{align*}
Using the remark above, $R$ is the inverse of $D$ with
respect to the circle with diameter $\ol{BC}$,
which has radius $\left\lvert \half(b-c) \right\rvert$.
Thus
\begin{align*}
  r - \frac{b+c}{2}
  &= \frac{\frac14(b-c)\left( \frac1b-\frac1c \right)}%
  {\ol{\half\left( a-\frac{bc}{a} \right)}} \\
  r &= \frac{b+c}{2}
  + \frac{-\half \frac{(b-c)^2}{bc}}{\frac1a-\frac{a}{bc}} \\
  &= \frac{b+c}{2} + \frac{a(b-c)^2}{2(a^2-bc)} \\
  &= \frac{a(b-c)^2+(b+c)(a^2-bc)}{2(a^2-bc)}.
\end{align*}
Expanding and subtracting gives
\[ p-r = \frac{a^3-abc-ab^2-ac^2+b^2c+bc^2}{2(a^2-bc)}
  = \frac{(a+b+c)(a-b)(a-c)}{2(a^2-bc)} \]
which is visibly equal to the negation of its conjugate
once the factor of $a+b+c$ is deleted.

(Actually, one can guess this factorization ahead of time
by noting that if $A=B$, then $P=B=R$, so $a-b$ must be a factor;
analogously $a-c$ must be as well.)
\pagebreak

\subsection{TSTST 2017/2, proposed by Kevin Sun}
\textsl{Available online at \url{https://aops.com/community/p8526115}.}
\begin{mdframed}[style=mdpurplebox,frametitle={Problem statement}]
Ana and Banana are playing a game. First Ana picks a word,
which is defined to be a nonempty sequence of capital English letters.
Then Banana picks a nonnegative integer $k$ and challenges Ana
to supply a word with exactly $k$ subsequences which are equal to Ana's word.
Ana wins if she is able to supply such a word, otherwise she loses.
For example, if Ana picks the word ``TST'', and Banana chooses $k = 4$,
then Ana can supply the word ``TSTST'' which has $4$ subsequences
which are equal to Ana's word.
Which words can Ana pick so that she can win no matter what value of
$k$ Banana chooses?
\end{mdframed}
First we introduce some notation.
Define a \emph{block} of letters to be a maximal
contiguous subsequence of consecutive letters.
Throughout the solution, we fix the word $A$ that Ana picks,
and introduce the following notation for its $m$ blocks:
\[ A =
  A_1 A_2 \dots A_m
  = \underbrace{a_1 \dots a_1}_{x_1}
  \underbrace{a_2 \dots a_2}_{x_2}
  \dots
  \underbrace{a_m \dots a_m}_{x_m}.
\]
A \emph{rainbow} will be a subsequence equal to Ana's initial word $A$
(meaning Ana seeks words with exactly $k$ rainbows).
Finally, for brevity, let $A_i = \underbrace{a_i \dots a_i}_{x_i}$,
so $A = A_1 \dots A_m$.


We prove two claims that resolve the problem.
\begin{claim*}
  If $x_i = 1$ for some $i$, then for any $k \ge 1$, the word
  \[ W = A_1 \dots A_{i-1} \underbrace{a_i \dots a_i}_{k}
    A_{i+1} \dots A_m  \]
  obtained by repeating the $i$th letter $k$ times
  has exactly $k$ rainbows.
\end{claim*}
\begin{proof}
  Obviously there are at least $\binom{k}{k-1}=k$ rainbows,
  obtained by deleting $k-1$ choices of the letter $a_i$
  in the repeated block.
  We show they are the only ones.

  Given a rainbow, consider the location of this singleton block in $W$.
  It cannot occur within the first $|A_1| + \dots + |A_{i-1}|$ letters,
  nor can it occur within the final $|A_{i+1}| + \dots + |A_m|$ letters.
  So it must appear in the $i$th block of $W$.
  That implies that all the other $a_i$'s in the
  $i$th block of $W$ must be deleted, as desired.
  (This last argument is actually nontrivial, and has some substance;
  many students failed to realize that the upper bound requires care.)
\end{proof}

\begin{claim*}
  If $x_i \ge 2$ for all $i$, then no word $W$ has exactly two rainbows.
\end{claim*}
\begin{proof}
  We prove if there are two rainbows of $W$, then we can construct
  at least three rainbows.

  Let $W = w_1 \dots w_n$ and consider the two rainbows of $W$.
  Since they are not the same,
  there must be a block $A_p$ of the rainbow,
  of length $\ell \ge 2$,
  which do not occupy the same locations in $W$.

  Assume the first rainbow uses $w_{i_1}$, \dots, $w_{i_\ell}$ for this block
  and the second rainbow uses $w_{j_1}$, \dots, $w_{j_\ell}$ for this block.
  Then among the letters $w_q$ for
  $\min(i_1, j_1) \le q \le \max(i_\ell, j_\ell)$,
  there must be at least $\ell+1$ copies of the letter $a_p$.
  Moreover, given a choice of $\ell$ copies of the letter $a_p$
  in this range, one can complete the subsequence to a rainbow.
  So the number of rainbows is at least $\binom{\ell+1}{\ell} \ge \ell+1$.

  Since $\ell \ge 2$, this proves $W$ has at least three rainbows.
\end{proof}

In summary, Ana wins if and only if $x_i = 1$ for some $i$,
since she can duplicate the isolated letter $k$ times;
but if $x_i \ge 2$ for all $i$ then Banana only needs to supply $k = 2$.
\pagebreak

\subsection{TSTST 2017/3, proposed by Calvin Deng, Linus Hamilton}
\textsl{Available online at \url{https://aops.com/community/p8526130}.}
\begin{mdframed}[style=mdpurplebox,frametitle={Problem statement}]
Consider solutions to the equation
\[ x^2 - cx + 1 = \frac{f(x)}{g(x)} \]
where $f$ and $g$ are nonzero polynomials with nonnegative real coefficients.
For each $c > 0$, determine the minimum possible degree of $f$, or show
that no such $f$, $g$ exist.
\end{mdframed}
First, if $c \ge 2$ then we claim no such $f$ and $g$ exist.
Indeed, one simply takes $x=1$ to get $f(1)/g(1) \le 0$, impossible.

For $c < 2$, let $c = 2 \cos \theta$, where $0 < \theta < \pi$.
We claim that $f$ exists and has minimum degree equal to $n$,
where $n$ is defined as the smallest integer
satisfying $\sin n\theta \le 0$.
In other words
\[ n = \left\lceil \frac{\pi}{\arccos(c/2)} \right\rceil. \]

First we show that this is necessary.
To see it, write explicitly
\[ g(x) = a_0 + a_1 x + a_2 x^2 + \dots + a_{n-2} x^{n-2} \]
with each $a_i \ge 0$, and $a_{n-2} \neq 0$.
Assume that $n$ is such that $\sin(k\theta) \ge 0$ for $k=1,\dots,n-1$.
Then, we have the following system of inequalities:
\begin{align*}
  a_1 &\ge 2 \cos \theta \cdot a_0 \\
  a_0 + a_2 &\ge 2 \cos \theta \cdot a_1 \\
  a_1 + a_3 &\ge 2 \cos \theta \cdot a_2 \\
  &\vdots \\
  a_{n-5} + a_{n-3} &\ge 2 \cos \theta \cdot a_{n-4} \\
  a_{n-4} + a_{n-2} &\ge 2 \cos \theta \cdot a_{n-3} \\
  a_{n-3} &\ge 2 \cos \theta \cdot a_{n-2}.
\end{align*}
Now, multiply the first equation by $\sin\theta$,
the second equation by $\sin2\theta$, et cetera,
up to $\sin\left( (n-1)\theta \right)$.
This choice of weights is selected since we have
\[ \sin \left( k\theta \right) + \sin\left( (k+2)\theta \right)
  = 2 \sin\left( (k+1)\theta \right) \cos \theta \]
so that summing the entire expression
cancels nearly all terms and leaves only
\[ \sin\left( (n-2)\theta \right) a_{n-2}
  \ge \sin\left( (n-1)\theta  \right) \cdot 2\cos\theta \cdot a_{n-2} \]
and so by dividing by $a_{n-2}$ and using the same identity
gives us $\sin(n\theta) \le 0$, as claimed.

This bound is best possible, because the example
\[ a_k = \sin\left( (k+1)\theta \right) \ge 0 \]
makes all inequalities above sharp, hence giving
a working pair $(f,g)$.

\begin{remark*}
Calvin Deng points out that a cleaner proof of the lower bound
is to take $\alpha = \cos \theta + i \sin \theta$.
Then $f(\alpha) = 0$, but by condition the imaginary part of $f(\alpha)$
is apparently strictly positive, contradiction.
\end{remark*}


\begin{remark*}
  Guessing that $c < 2$ works at all (and realizing $c \ge 2$ fails)
  is the first part of the problem.

  The introduction of trigonometry into the solution may seem magical,
  but is motivated in one of two ways:
  \begin{itemize}
    \ii Calvin Deng points out that it's possible to
    guess the answer from small cases:
    For $c \le 1$ we have $n = 3$, tight at $\frac{x^3+1}{x+1} = x^2-x+1$,
    and essentially the ``sharpest $n=3$ example''.
    A similar example exists at $n = 4$ with
    $\frac{x^4+1}{x^2+\sqrt 2 x+1} = x^2-\sqrt2x+1$
    by the Sophie-Germain identity.
    In general, one can do long division to extract
    an optimal value of $c$ for any given $n$,
    although $c$ will be the root of some polynomial.

    The thresholds $c \le 1$ for $n = 3$, $c \le \sqrt 2$ for $n = 4$,
    $c \le \frac{1+\sqrt5}{2}$ for $n = 5$,
    and $c \le 2$ for $n < \infty$ suggest the unusual form of the
    answer via trigonometry.

    \ii One may imagine trying to construct a polynomial
    recursively / greedily by making all inequalities above hold
    (again the ``sharpest situation'' in which $f$ has few coefficients).
    If one sets $c = 2t$, then we have
    \[ a_0 = 1, \quad a_1 = 2t, \quad a_2 = 4t^2-1, \quad
      a_3 = 8t^3-4t, \quad \dots \]
    which are the Chebyshev polynomials of the second type.
    This means that trigonometry is essentially mandatory.
    (One may also run into this when by using standard linear recursion techniques,
    and noting that the characteristic polynomial
    has two conjugate complex roots.)
  \end{itemize}
\end{remark*}

\begin{remark*}
  Mitchell Lee notes that an IMO longlist problem from 1997 shows that
  if $P(x)$ is any polynomial satisfying $P(x) > 0$ for $x > 0$,
  then $(x+1)^n P(x)$ has nonnegative coefficients
  for large enough $n$.
  This show that $f$ and $g$ at least exist for $c \le 2$,
  but provides no way of finding the best possible $\deg f$.

  Meghal Gupta also points out that showing $f$ and $g$ exist
  is possible in the following way:
  \[ \left( x^2-1.99x+1 \right) \left( x^2+1.99x+1 \right)
    = \left( x^4 - 1.9601x^2 + 1 \right) \]
  and so on, repeatedly multiplying by the ``conjugate''
  until all coefficients become positive.
  To my best knowledge, this also does not give any way
  of actually minimizing $\deg f$,
  although Ankan Bhattacharya points out that this construction
  is actually optimal in the case where $n$ is a power of $2$.
\end{remark*}

\begin{remark*}
  It's pointed out that Matematicheskoe Prosveshchenie, issue 1, 1997, page 194
  contains a nearly analogous result,
  available at \url{https://mccme.ru/free-books/matpros/pdf/mp-01.pdf}
  with solutions presented in \url{https://mccme.ru/free-books/matpros/pdf/mp-05.pdf},
  pages 221--223;
  and \url{https://mccme.ru/free-books/matpros/pdf/mp-10.pdf}, page 274.
\end{remark*}
\pagebreak

\section{Solutions to Day 2}
\subsection{TSTST 2017/4, proposed by Mark Sellke}
\textsl{Available online at \url{https://aops.com/community/p8526131}.}
\begin{mdframed}[style=mdpurplebox,frametitle={Problem statement}]
Find all nonnegative integer solutions to \[ 2^a + 3^b + 5^c = n!. \]
\end{mdframed}
For $n \le 4$, one can check the only solutions are:
\begin{align*}
  2^2 + 3^0 + 5^0 &= 3! \\
  2^1 + 3^1 + 5^0 &= 3! \\
  2^4 + 3^1 + 5^1 &= 4!.
\end{align*}
Now we prove there are no solutions for $n \ge 5$.

A tricky way to do this is to take modulo $120$, since
\begin{align*}
  2^a \pmod{120} &\in \{ 1, 2, 4, 8, 16, 32, 64 \} \\
  3^b \pmod{120} &\in \{ 1, 3, 9, 27, 81 \} \\
  5^c \pmod{120} &\in \{ 1, 5, 25 \}
\end{align*}
and by inspection one notes that no three
elements have vanishing sum modulo $120$.

I expect most solutions to instead use casework.
Here is one possible approach with cases (with $n \ge 5$).
First, we analyze the cases where $a < 3$:
\begin{itemize}
  \ii $a=0$: No solutions for parity reasons.
  \ii $a=1$: since $3^b + 5^c \equiv 6 \pmod 8$,
  we find $b$ even and $c$ odd (hence $c \neq 0$).
  Now looking modulo $5$ gives that $3^b + 5^c \equiv 3 \pmod 5$,
  \ii $a=2$: From $3^b + 5^c \equiv 4 \pmod 8$,
  we find $b$ is odd and $c$ is even.
  Now looking modulo $5$ gives a contradiction,
  even if $c = 0$,
  since $3^b \in \{2,3 \pmod 5\}$ but $3^b + 5^c \equiv 1 \pmod 5$.
\end{itemize}
Henceforth assume $a \ge 3$.
Next, by taking modulo $8$ we have $3^b+5^c \equiv 0 \pmod 8$,
which forces both $b$ and $c$ to be odd (in particular, $b,c > 0$).
We now have
\begin{align*}
  2^a + 5^c &\equiv 0 \pmod 3 \\
  2^a + 3^b &\equiv 0 \pmod 5.
\end{align*}
The first equation implies $a$ is even,
but the second equation requires $a$ to be odd, contradiction.
Hence no solutions with $n \ge 5$.
\pagebreak

\subsection{TSTST 2017/5, proposed by Ray Li}
\textsl{Available online at \url{https://aops.com/community/p8526136}.}
\begin{mdframed}[style=mdpurplebox,frametitle={Problem statement}]
Let $ABC$ be a triangle with incenter $I$.
Let $D$ be a point on side $BC$ and let $\omega_B$ and $\omega_C$
be the incircles of $\triangle ABD$ and $\triangle ACD$, respectively.
Suppose that $\omega_B$ and $\omega_C$ are tangent to segment $BC$
at points $E$ and $F$, respectively.
Let $P$ be the intersection of segment $AD$ with the
line joining the centers of $\omega_B$ and $\omega_C$.
Let $X$ be the intersection point of lines $BI$ and $CP$
and let $Y$ be the intersection point of lines $CI$ and $BP$.
Prove that lines $EX$ and $FY$ meet on the incircle of $\triangle ABC$.
\end{mdframed}
\paragraph{First solution (homothety).}
Let $Z$ be the diametrically opposite point on the incircle.
We claim this is the desired intersection.

\begin{center}
\begin{asy}
pair A = dir(110);
pair B = dir(210);
pair C = dir(330);

filldraw(A--B--C--cycle, opacity(0.1)+lightred, red);
filldraw(incircle(A, B, C), opacity(0.1)+lightcyan, lightblue);

pair D = 0.35*B+0.65*C;
draw(A--D, red);

pair I_B = incenter(A, B, D);
pair I_C = incenter(A, C, D);
pair E = foot(I_B, B, C);
pair F = foot(I_C, B, C);

filldraw(incircle(A, B, D), opacity(0.1)+green, heavygreen);
filldraw(incircle(A, C, D), opacity(0.1)+green, heavygreen);
pair I = incenter(A, B, C);

pair P = extension(I_B, I_C, A, D);
draw(I_B--I_C, heavygreen);

pair X = extension(B, I, C, P);
pair Y = extension(C, I, B, P);
pair Z = extension(E, X, F, Y);
draw(B--I--C, red);
draw(X--C, dotted+heavygreen);
draw(Y--B, dotted+heavygreen);
draw(E--Z--F, dashed+heavygreen);

pair T = extension(B, C, I_B, I_C);
draw(I_C--T, dotted+heavygreen);
draw(C--T, dotted+red);
pair W = foot(I, B, C);

dot("$A$", A, dir(A));
dot("$B$", B, dir(270));
dot("$C$", C, dir(270));
dot("$D$", D, dir(D));
dot("$I_B$", I_B, dir(170));
dot("$I_C$", I_C, dir(30));
dot("$E$", E, dir(E));
dot("$F$", F, dir(F));
dot("$I$", I, dir(90));
dot("$P$", P, dir(250));
dot("$X$", X, dir(160));
dot("$Y$", Y, dir(20));
dot("$Z$", Z, dir(140));
dot("$T$", T, dir(T));
dot("$W$", W, dir(W));

/* TSQ Source:

A = dir 110
B = dir 210 R270
C = dir 330 R270

A--B--C--cycle 0.1 lightred / red
incircle A B C 0.1 lightcyan / lightblue

D = 0.35*B+0.65*C
A--D red

I_B = incenter A B D R170
I_C = incenter A C D R30
E = foot I_B B C
F = foot I_C B C

incircle A B D 0.1 green / heavygreen
incircle A C D 0.1 green / heavygreen
I = incenter A B C R90

P = extension I_B I_C A D R250
I_B--I_C heavygreen

X = extension B I C P R160
Y = extension C I B P R20
Z = extension E X F Y R140
B--I--C red
X--C dotted heavygreen
Y--B dotted heavygreen
E--Z--F dashed heavygreen

T = extension B C I_B I_C
I_C--T dotted heavygreen
C--T dotted red
W = foot I B C

*/
\end{asy}
\end{center}

Note that:
\begin{itemize}
  \ii $P$ is the insimilicenter of $\omega_B$ and $\omega_C$
  \ii $C$ is the exsimilicenter of $\omega$ and $\omega_C$.
\end{itemize}
Thus by Monge theorem, the insimilicenter of $\omega_B$ and $\omega$
lies on line $CP$.

This insimilicenter should also lie on the line joining the
centers of $\omega$ and $\omega_B$, which is $\ol{BI}$,
hence it coincides with the point $X$.
So $X \in \ol{EZ}$ as desired.

\paragraph{Second solution (harmonic).}
Let $T = \ol{I_B I_C} \cap \ol{BC}$, and $W$ the foot from $I$ to $\ol{BC}$.
Define $Z = \ol{FY} \cap \ol{IW}$.
Because $\angle I_B D I_C = 90\dg$, we have
\[ -1 = (I_B I_C; PT) \overset{B}{=} (I I_C; YC)
  \overset{F}{=} (I\infty; ZW) \]
So $I$ is the midpoint of $\ol{ZW}$ as desired.

\paragraph{Third solution (outline, barycentric, Andrew Gu).}
Let $AD = t$, $BD = x$, $CD = y$, so $a=x+y$ and by Stewart's theorem we have
\begin{equation}
  (x+y)(xy+t^2) = b^2x+c^2y.
  \label{eq:17TSTST5stewart}
\end{equation}
We then have $D = (0:y:x)$ and so
\[ \ol{AI_B} \cap \ol{BC} = \left( 0 : y + \frac{tx}{c+t} : \frac{cx}{c+t} \right) \]
hence intersection with $BI$ gives
\begin{align*}
  I_B &= (ax : cy+at : cx).
  \intertext{Similarly,}
  I_C &= (ay : by : bx+at).
\end{align*}

Then, we can compute
\[ P = \left( 2axy : y(at+bx+cy) : x(at+bx+cy) \right) \]
since $P \in \ol{I_B I_C}$, and clearly $P \in \ol{AD}$.
Intersection now gives
\begin{align*}
  X &= \left( 2ax : at+bx+cy : 2cx \right) \\
  Y &= \left( 2ay : 2by : at+bx+cy \right).
\end{align*}

Finally, we have $BE = \half(c+x-t)$, and similarly for $CF$.
Now if we reflect
$D = (0, \frac{s-c}{a}, \frac{s-b}{a})$ over
$I = (\frac{a}{2s}, \frac{b}{2s}, \frac{c}{2s})$, we get the antipode
\[ Q \coloneqq \left( 4a^2 : -a^2+2ab-b^2+c^2 : -a^2+2ac-c^2+b^2 \right). \]
We may then check $Q$ lies on each of lines $EX$ and $FY$
(by checking $\det(Q,E,X)=0$ using the equation \eqref{eq:17TSTST5stewart}).
\pagebreak

\subsection{TSTST 2017/6, proposed by Ivan Borsenco}
\textsl{Available online at \url{https://aops.com/community/p8526142}.}
\begin{mdframed}[style=mdpurplebox,frametitle={Problem statement}]
A sequence of positive integers $(a_{n})_{n \geq 1}$ is of
\emph{Fibonacci type} if it satisfies the recursive relation
$a_{n+2}=a_{n+1}+a_{n}$ for all $n \geq 1$.
Is it possible to partition the set of positive integers
into an infinite number of Fibonacci type sequences?
\end{mdframed}
Yes, it is possible.
The following solutions were written for me by Kevin Sun and Mark Sellke.
We let $F_1 = F_2 = 1$, $F_3 = 2$, $F_4 = 3$, $F_5 = 5$, \dots
denote the Fibonacci numbers.

\paragraph{First solution (Kevin Sun).}
We are going to appeal to the so-called Zeckendorf theorem:
\begin{theorem*}
  [Zeckendorf]
  Every positive integer can be uniquely expressed
  as the sum of nonconsecutive Fibonacci numbers.
\end{theorem*}
This means every positive integer has a
Zeckendorf (``Fibonacci-binary'') representation
where we put $1$ in the $i$th digit from the right
if $F_{i+1}$ is used.
The idea is then to take the following so-called \emph{Wythoff array}:
\begin{itemize}
  \ii \textbf{Row 1}: $1$, $2$, $3$, $5$, \dots
  \ii \textbf{Row 101}: $1+3$, $2+5$, $3+8$, \dots
  \ii \textbf{Row 1001}: $1+5$, $2+8$, $3+13$, \dots
  \ii \textbf{Row 10001}: $1+8$, $2+13$, $3+21$, \dots
  \ii \textbf{Row 10101}: $1+3+8$, $2+5+13$, $3+8+21$, \dots
  \ii \dots et cetera.
\end{itemize}
More concretely, the array has the following rows to start:
\[
\begin{matrix}
1&2&3&5&8&13&21&\dotsb\\
4&7&11&18&29&47&76&\dotsb\\
6&10&16&26&42&68&110&\dotsb\\
9&15&24&39&63&102&165&\dotsb\\
12&20&32&52&84&136&220&\dotsb\\
14&23&37&60&97&157&254&\dotsb\\
17&28&45&73&118&191&309&\dotsb\\
\vdots&\vdots&\vdots&\vdots&\vdots&\vdots&\vdots&\ddots\\
\end{matrix}.
\]

Here are the full details.

We begin by outlining a proof of Zeckendorf's theorem,
which implies the representation above is unique.
Note that if $F_k$ is the greatest Fibonacci number at most $n$,
then \[n - F_k < F_{k+1} - F_{k} = F_{k-1}.\]
In particular, repeatedly subtracting off the largest $F_k$
from $n$ will produce one such representation with
no two consecutive Fibonacci numbers.
On the other hand, this $F_k$ must be used, as
\[ n \ge F_k > F_{k-1} + F_{k-3} + F_{k-5} + \dotsb \]
This shows, by a simple inductive argument,
that such a representation exists and unique.

We write $n = \ol{a_k\dotso a_1}_{\text{Fib}}$
for the Zeckendorf representation as we described
(where $a_i = 1$ if $F_{i+1}$ is used).
Now for each $\ol{a_k\dotso a_1}_{\text{Fib}}$
with $a_1 = 1$, consider the sequence
\[ \ol{a_k\dotso a_1}_{\text{Fib}}, \;
  \ol{a_k\dotso a_1 0}_{\text{Fib}}, \;
  \ol{a_k\dotso a_1 00}_{\text{Fib}}, \; \dots \]
These sequences are Fibonacci-type by definition,
and partition the positive integers since
each positive integer has exactly one Fibonacci base representation.

\paragraph{Second solution.} Call an infinite set of integers $S$ \textit{sandwiched} if there exist increasing sequences $\{a_i\}_{i=0}^{\infty}, \{b_i\}_{i=0}^{\infty}$ such that the following are true:
\begin{itemize}
\item $a_i + a_{i+1} = a_{i+2}$ and $b_i + b_{i+1} = b_{i+2}$.
\item The intervals $[a_i+1,b_i-1]$ are disjoint and are nondecreasing in length.
\item $\displaystyle S = \bigcup_{i=0}^{\infty} [a_i+1,b_i-1]$.
\end{itemize}

We claim that if $S$ is any nonempty sandwiched set, then $S$ can be partitioned into a Fibonacci-type sequence (involving the smallest element of $S$) and two smaller sandwiched sets. If this claim is proven, then we can start with $\NN \setminus \{1,2,3,5,\dots\}$, which is a sandwiched set, and repeatedly perform this partition, which will eventually sort each natural number into a Fibonacci-type sequence.

Let $S$ be a sandwiched set given by $\{a_i\}_{i=0}^{\infty}, \{b_i\}_{i=0}^{\infty}$, so the smallest element in $S$ is $x = a_0 + 1$. Note that $y = a_1+1$ is also in $S$ and $x < y$. Then consider the Fibonacci-type sequence given by $f_0 = x, f_1 = y$, and $f_{k+2} = f_{k+1} + f_k$. We can then see that $f_i \in [a_i+1,b_i-1]$, as the sum of numbers in the intervals $[a_{k}+1,b_{k}-1]$, $[a_{k+1}+1,b_{k+1}-1]$ lies in the interval \[[a_{k} + a_{k+1} + 2,b_k + b_{k+1}-2] = [a_{k+2}+2,b_{k+2}-2] \subset [a_{k+2}+1,b_{k+2}-1].\]

Therefore, this gives a natural partition of $S$ into this sequence and two sets: \begin{align*} S_1 &= \bigcup_{i=0}^{\infty} [a_i+1,f_i-1] \\ \text{and} \,\, S_2 &= \bigcup_{i=0}^{\infty} [f_i+1,b_i-1].\end{align*} (For convenience, $[x,x-1]$ will be treated as the empty set.)

We now show that $S_1$ and $S_2$ are sandwiched. Since $\{a_i\}$, $\{f_i\}$, and $\{b_i\}$ satisfy the Fibonacci recurrence, it is enough to check that the intervals have nondecreasing lengths. For $S_1$, that is equivalent to $f_{k+1} - a_{k+1} \ge f_k - a_k$ for each $k$. Fortunately, for $k \ge 1$, the difference is $f_{k-1} - a_{k-1} \ge 0$, and for $k = 0$, $f_1 - a_1 = 1 = f_0 - a_0$. Similarly for $S_2$, checking $b_{k+1} - f_{k+1} \ge b_k - f_k$ is easy for $k \ge 1$ as $b_{k-1} - f_{k-1} \ge 0$, and \[(b_{1} - f_1) - (b_0 - f_0) = (b_1 - a_1) - (b_0 - a_0),\] which is nonnegative since the lengths of intervals in $S$ are nondecreasing.

Therefore we have shown that $S_1$ and $S_2$ are sandwiched. (Note that some of the $[a_i+1,f_i-1]$ may be empty, which would shift some indices back.) Since this gives us a procedure to take a set $S$ and produce a Fibonacci-type sequence with its smallest element, along which two other sandwiched types, we can partition $\NN$ into an infinite number of Fibonacci-type sequences.

\paragraph{Third solution.} We add Fibonacci-type sequences one-by-one. At each step, let $x$ be the smallest number that has not been used in any previous sequence. We generate a new Fibonacci-type sequence as follows. Set $a_0 = x$ and for $i \ge 1$, set \[a_i = \left \lfloor \varphi a_{i-1} + \frac{1}{2} \right \rfloor.\] Equivalently, $a_{i}$ is the closest integer to $\varphi a_{i-1}$. \\
It suffices to show that this sequence is Fibonacci-type and that no two sequences generated in this way overlap.
We first show that for a positive integer $n$, \[\left \lfloor \varphi \left \lfloor \varphi n + \frac{1}{2} \right \rfloor + \frac{1}{2} \right \rfloor = n + \left \lfloor \varphi n + \frac{1}{2} \right \rfloor . \]
Indeed, \begin{align*} \left \lfloor \varphi \left \lfloor \varphi n + \frac{1}{2} \right \rfloor + \frac{1}{2} \right \rfloor &= \left \lfloor (1 + \varphi^{-1}) \left \lfloor \varphi n + \frac{1}{2} \right \rfloor + \frac{1}{2}\right \rfloor \\
&= \left \lfloor \varphi n + \frac{1}{2} \right \rfloor + \left \lfloor \varphi^{-1} \left \lfloor \varphi n + \frac{1}{2}\right \rfloor + \frac{1}{2} \right \rfloor.
\end{align*}
Note that $\left \lfloor \varphi n + \frac{1}{2} \right \rfloor = \varphi n + c$ for some $|c| \le \frac{1}{2}$; this implies that $\varphi^{-1}\left \lfloor \varphi n + \frac{1}{2} \right \rfloor$ is within $\varphi^{-1}\cdot \frac{1}{2} < \frac{1}{2}$ of $n$, so its closest integer is $n$, proving the claim.

Therefore these sequences are Fibonacci-type. Additionally, if $a \neq b$, then $|\varphi a - \varphi b| \ge \varphi > 1$. Then \[a \neq b \implies \left \lfloor \varphi a + \frac{1}{2} \right \rfloor \neq \left \lfloor \varphi b + \frac{1}{2} \right \rfloor,\]
and since the first term of each sequence is chosen to not overlap with any previous sequences, these sequences are disjoint.

\begin{remark*}
  Ankan Bhattacharya points out that the same sequence
  essentially appears in IMO 1993, Problem 5 --- in other words,
  a strictly increasing function $f \colon \ZZ_{>0} \to \ZZ_{>0}$ with $f(1) = 2$,
  and $f(f(n)) = f(n) + n$.

  Nikolai Beluhov sent us an older reference from March 1977,
  where Martin Gardner wrote in his column about Wythoff's Nim.
  The relevant excerpt goes:
  \begin{quote}
    ``Imagine that we go through the infinite sequence of safe pairs (in
    the manner of Eratosthenes' sieve for sifting out primes) and cross
    out the infinite set of all safe pairs that are pairs in the Fibonacci
    sequence. The smallest pair that is not crossed out is 4/7. We can now
    cross out a second infinite set of safe pairs, starting with 4/7, that
    are pairs in the Lucas sequence. An infinite number of safe pairs, of
    which the lowest is now 6/10, remain. This pair too begins another
    infinite Fibonacci sequence, all of whose pairs are safe. The process
    continues forever. Robert Silber, a mathematician at North Carolina
    State University, calls a safe pair ``primitive'' if it is the first
    safe pair that generates a Fibonacci sequence.''
  \end{quote}
  The relevant article by Robert Silber is
  \emph{A Fibonacci Property of Wythoff Pairs},
  from The Fibonacci Quarterly 11/1976.
\end{remark*}


\paragraph{Fourth solution (Mark Sellke).}
For later reference let \[ f_1=0, f_2=1, f_3=1, \dots \]
denote the ordinary Fibonacci numbers.
We will denote the Fibonacci-like sequences by $F^i$ and the elements with subscripts;
hence $F^2_1$ is the first element of the second sequence.
Our construction amounts to just iteratively add new sequences;
hence the following claim is the whole problem.

\begin{lemma*}
  For any disjoint collection of Fibonacci-like sequences
  $F^1,\dots,F^k$ and any integer $m$ contained in none of them,
  there is a new Fibonacci-like sequence $F^{k+1}$ beginning
  with $F^{k+1}_1=m$ which is disjoint from the previous sequences.
\end{lemma*}

Observe first that for each sequence $F^j$ there is $c^j\in \RR^n$ such that
\[ F^j_n=c^j\phi^n+o(1)  \]
where \[\phi=\frac{1+\sqrt{5}}{2}.\]

Collapse the group $(\RR^+,\times)$ into the half-open interval
$J = \left\{ x \mid 1 \le x < \phi \right\}$
by defining $T(x)=y$ for the unique $y\in J$ with $x=y\phi^n$ for some integer $n$.

Fix an interval $I=[a,b]\subseteq [1.2,1.3]$ (the last condition is to
avoid wrap-around issues) which contains none of the $c^j$, and take
$\varepsilon<0.001$ to be small enough that in fact each $c^j$ has
distance at least $10\varepsilon$ from $I$; this means any $c_j$ and
element of $I$ differ by at least a $(1+10\varepsilon)$ factor. The idea
will be to take $F^{k+1}_1=m$ and $F^{k+1}_2$ to be a large such that
the induced values of $F^{k+1}_j$ grow like $k\phi^j$ for $j\in
T^{-1}(I)$, so that $F^{k+1}_n$ is separated from the $c^j$ after
applying $T$. What's left to check is the convergence.

Now let \[c=\lim_{n\to\infty} \frac{f_n}{\phi^n}\] and take
$M$ large enough that for $n>M$ we have
\[\left|\frac{f_n}{c\phi^n}-1\right|<\varepsilon.\]

Now $\frac{T^{-1}(I)}{c}$ contains arbitrarily large integers,
so there are infinitely many $N$ with $cN\in T^{-1}(I)$ with $N>\frac{10m}{\varepsilon}$.
We claim that for any such $N$, the sequence $F^{(N)}$ defined by
\[F^{(N)}_1=m,F^{(N)}_2=N\] will be very multiplicatively similar to the
normal Fibonacci numbers up to rescaling;
indeed for $j=2,j=3$ we have $\frac{F_2^{(N)}}{f_2}=N,\frac{F_3^{(N)}}{f_3}=N+m$
and so by induction we will have
\[ \frac{F^{(N)}_j}{f_j} \in [N,N+m] \subseteq [N,N(1+\varepsilon)] \]
for $j\geq 2$.
Therefore, up to small multiplicative errors, we have
\[F_j^{(N)}\approx Nf_j\approx cN\phi^j.\]
From this we see that for $j>M$ we have
\[T(F_j^{(N)})\in  T(cN)\cdot [1-2\varepsilon,1+2\varepsilon].\]
In particular, since $T(cN)\in I$ and $I$ is separated from each $c_j$
by a factor of $(1+10\varepsilon)$,
we get that $F_j^{(N)}$ is not in any of $F^1,F^2,\dots,F^k$.

Finishing is easy, since we now have a uniform estimate on how many
terms we need to check for a new element before the exponential growth
takes over. We will just use pigeonhole to argue that there are few
possible collisions among those early terms, so we can easily pick a
value of $N$ which avoids them all. We write it out below.

For large $L$, the set \[S_L=(I\cdot \phi^L)\cap\ZZ\] contains at
least $k_I\phi^L$ elements. As $N$ ranges over $S_L$, for each fixed
$j$, the value of $F^{(N)}_j$ varies by at most a factor of $1.1$
because we imposed $I\subseteq [1.2,1.3]$ and so this is true for the
first two terms, hence for all subsequent terms by induction. Now
suppose $L$ is very large, and consider a fixed pair $(i,j)$ with $i\leq
k$ and $j\leq M$. We claim there is at most $1$ possible value $k$ such
that the term $F^i_k$ could equal $F^{(N)}_j$ for some $N\in S_L$;
indeed, the terms of $F^i$ are growing at exponential rate with factor
$\phi>1.1$, so at most one will be in a given interval of multiplicative
width at most $1.1$.

Hence, of these $k_I\phi^L$ values of $N$, at most $kM$ could cause
problems, one for each pair $(i,j)$. However by monotonicity of
$F^{(N)}_j$ in $N$, at most $1$ value of $N$ causes a collision for each
pair $(i,j)$. Hence for large $L$ so that $k_I\phi^L>10kM$ we can find a
suitable $N\in S_L$ by pigeonhole and the sequence $F^{(N)}$ defined by
$(m,N,N+m,\dots)$ works.
\pagebreak


\end{document}
