% © Evan Chen
% Downloaded from https://web.evanchen.cc/

\documentclass[11pt]{scrartcl}
\usepackage[sexy]{evan}
\ihead{\footnotesize\textbf{\thetitle}}
\ohead{\footnotesize\href{http://web.evanchen.cc}{\ttfamily web.evanchen.cc},
    updated \today}
\title{IMO 2001 Solution Notes}
\date{\today}

\begin{document}

\maketitle

\begin{abstract}
This is a compilation of solutions
for the 2001 IMO.
Some of the solutions are my own work,
but many are from the official solutions provided by the organizers
(for which they hold any copyrights),
and others were found by users on the Art of Problem Solving forums.

These notes will tend to be a bit more advanced and terse than the ``official''
solutions from the organizers.
In particular, if a theorem or technique is not known to beginners
but is still considered ``standard'', then I often prefer to
use this theory anyways, rather than try to work around or conceal it.
For example, in geometry problems I typically use directed angles
without further comment, rather than awkwardly work around configuration issues.
Similarly, sentences like ``let $\mathbb{R}$ denote the set of real numbers''
are typically omitted entirely.

Corrections and comments are welcome!
\end{abstract}

\tableofcontents
\newpage

\addtocounter{section}{-1}
\section{Problems}
\begin{enumerate}[\bfseries 1.]
\item %% Problem 1
Let $ABC$ be an acute-angled triangle with $O$ as its circumcenter.
Let $P$ on line $BC$ be the foot of the altitude from $A$.
Assume that $\angle BCA \ge \angle ABC + 30\dg$.
Prove that $\angle CAB + \angle COP < 90\dg$.

\item %% Problem 2
Let $a$, $b$, $c$ be positive reals. Prove that
\[ \frac{a}{\sqrt{a^2+8bc}} + \frac{b}{\sqrt{b^2+8ca}} + \frac{c}{\sqrt{c^2+8ab}} \ge 1. \]

\item %% Problem 3
Twenty-one girls and twenty-one boys
took part in a mathematical competition.
It turned out that each contestant solved at most six problems,
and for each pair of a girl and a boy,
there was at least one problem that was
solved by both the girl and the boy.
Show that there is a problem that was solved by
at least three girls and at least three boys.

\item %% Problem 4
Let $n$ be an odd integer greater than $1$
and let $c_1$, $c_2$, \dots, $c_n$ be integers.
For each permutation $a = (a_1, a_2, \dots, a_n)$
of $\{1,2,\dots,n\}$, define $S(a) = \sum_{i=1}^n c_i a_i$.
Prove that there exist two permutations $a \neq b$
of $\{1,2,\dots,n\}$ such that $n!$
is a divisor of $S(a)-S(b)$.

\item %% Problem 5
Let $ABC$ be a triangle.
Let $\ol{AP}$ bisect $\angle BAC$ and let $\ol{BQ}$ bisect $\angle ABC$,
with $P$ on $\ol{BC}$ and $Q$ on $\ol{AC}$.
If $AB + BP = AQ + QB$ and $\angle BAC = 60\dg$,
what are the angles of the triangle?

\item %% Problem 6
Let $a > b > c > d > 0$ be integers satisfying
\[ ac + bd = (b+d+a-c)(b+d-a+c).  \]
Prove that $ab + cd$ is not prime.

\end{enumerate}
\pagebreak

\section{Solutions to Day 1}
\subsection{IMO 2001/1}
\textsl{Available online at \url{https://aops.com/community/p119192}.}
\begin{mdframed}[style=mdpurplebox,frametitle={Problem statement}]
Let $ABC$ be an acute-angled triangle with $O$ as its circumcenter.
Let $P$ on line $BC$ be the foot of the altitude from $A$.
Assume that $\angle BCA \ge \angle ABC + 30\dg$.
Prove that $\angle CAB + \angle COP < 90\dg$.
\end{mdframed}
The conclusion rewrites as
\begin{align*}
  \angle COP &< 90\dg - \angle A = \angle OCP \\
  \iff PC &< PO \\
  \iff PC^2 &< PO^2 \\
  \iff PC^2 &< R^2 - PB \cdot PC \\
  \iff PC \cdot BC &< R^2 \\
  \iff ab \cos C  &< R^2 \\
  \iff \sin A \sin B \cos C & < \frac14.
\end{align*}
Now
\[ \cos C \sin B
  = \half \left( \sin(C+B)-\sin(C-B) \right)
  \le \half \left( 1 - \half \right) = \frac 14 \]
which finishes when combined with $\sin A < 1$.

\begin{remark*}
  If we allow $ABC$ to be right then
  equality holds when $\angle A = 90\dg$,
  $\angle C = 60\dg$, $\angle B = 30\dg$.
  This motivates the choice of estimates
  after reducing to a trig inequality.
\end{remark*}
\pagebreak

\subsection{IMO 2001/2}
\textsl{Available online at \url{https://aops.com/community/p119168}.}
\begin{mdframed}[style=mdpurplebox,frametitle={Problem statement}]
Let $a$, $b$, $c$ be positive reals. Prove that
\[ \frac{a}{\sqrt{a^2+8bc}} + \frac{b}{\sqrt{b^2+8ca}} + \frac{c}{\sqrt{c^2+8ab}} \ge 1. \]
\end{mdframed}
By Holder, we have
\[
  \left( \sum_{\text{cyc}} \frac{a}{\sqrt{a^2+8bc}} \right)^2
  \left( \sum_{\text{cyc}} a(a^2+8bc) \right)
  \ge (a+b+c)^3.
\]
So it suffices to show $(a+b+c)^3 \ge a^3+b^3+c^3+24abc$ which is clear by expanding.
\pagebreak

\subsection{IMO 2001/3}
\textsl{Available online at \url{https://aops.com/community/p119191}.}
\begin{mdframed}[style=mdpurplebox,frametitle={Problem statement}]
Twenty-one girls and twenty-one boys
took part in a mathematical competition.
It turned out that each contestant solved at most six problems,
and for each pair of a girl and a boy,
there was at least one problem that was
solved by both the girl and the boy.
Show that there is a problem that was solved by
at least three girls and at least three boys.
\end{mdframed}
We will show the contrapositive.
That is, assume that
\begin{itemize}
  \ii For each pair of a girl and a boy,
  there was at least one problem that was
  solved by both the girl and the boy.
  \ii Assume every problem is either solved
  mostly by girls (at most two boys)
  or mostly by boys (at most two girls).
\end{itemize}
Then we will prove that then some contestant
solved more than six problems.

Create a $21 \times 21$ grid with boys as columns
and girls as rows, and in each cell
write the name of a problem solved by the pair.
Color the cell \textbf{green} if at most two girls solved that problem,
and color it \textbf{blue} if at most two boys solved that problem.
(G for girl, B for boy.
It's possible both colors are used for some cell.)

WLOG, there are more green cells than blue,
so (by pigeonhole) take a column (boy) with that property.
That means the boy's column has at least $11$ green squares.
By hypothesis, those corresponds to at least $6$ different problems
solved. Now there are two cases:
\begin{itemize}
  \ii If there are any blue-only squares,
  then that square means a seventh distinct problems.
  \ii If the entire column is green,
  then among the $21$ green squares
  there are at least $11$ distinct problems solved
  in that column.
\end{itemize}

\begin{remark*}
The number $21$ cannot be decreased.
Here is an example of $20$ boys and $20$ girls
who solve problems named $A$-$J$
and $0$-$9$, which motivates the solution to begin with.
\begin{center}
\scriptsize \ttfamily
{\color{green}0000000000}{\color{blue}AABBCCDDEE} \\
{\color{green}0000000000}{\color{blue}AABBCCDDEE} \\
{\color{green}1111111111}{\color{blue}AABBCCDDEE} \\
{\color{green}1111111111}{\color{blue}AABBCCDDEE} \\
{\color{green}2222222222}{\color{blue}AABBCCDDEE} \\
{\color{green}2222222222}{\color{blue}AABBCCDDEE} \\
{\color{green}3333333333}{\color{blue}AABBCCDDEE} \\
{\color{green}3333333333}{\color{blue}AABBCCDDEE} \\
{\color{green}4444444444}{\color{blue}AABBCCDDEE} \\
{\color{green}4444444444}{\color{blue}AABBCCDDEE} \\
{\color{blue}FFGGHHIIJJ}{\color{green}5555555555} \\
{\color{blue}FFGGHHIIJJ}{\color{green}5555555555} \\
{\color{blue}FFGGHHIIJJ}{\color{green}6666666666} \\
{\color{blue}FFGGHHIIJJ}{\color{green}6666666666} \\
{\color{blue}FFGGHHIIJJ}{\color{green}7777777777} \\
{\color{blue}FFGGHHIIJJ}{\color{green}7777777777} \\
{\color{blue}FFGGHHIIJJ}{\color{green}8888888888} \\
{\color{blue}FFGGHHIIJJ}{\color{green}8888888888} \\
{\color{blue}FFGGHHIIJJ}{\color{green}9999999999} \\
{\color{blue}FFGGHHIIJJ}{\color{green}9999999999}
\end{center}
\end{remark*}

\begin{remark*}
  This took me embarrassingly long,
  but part of the work of the problem seemed to be
  finding the right ``data structure'' to get a foothold.
  I think the grid is good because we want to fill each intersection,
  then we consider for each cell a problem to put.

  I initially wanted to capture the full data by writing
  in each green cell the row index of the other girl who solved it,
  and similarly for the blue cells.
  (That led naturally to the colors, there were two types of cells.)
  This was actually helpful for finding the equality case above,
  but once I realized the equality case
  I also realized that I could discard the extra information
  and only remember the colors.
\end{remark*}
\pagebreak

\section{Solutions to Day 2}
\subsection{IMO 2001/4}
\textsl{Available online at \url{https://aops.com/community/p119174}.}
\begin{mdframed}[style=mdpurplebox,frametitle={Problem statement}]
Let $n$ be an odd integer greater than $1$
and let $c_1$, $c_2$, \dots, $c_n$ be integers.
For each permutation $a = (a_1, a_2, \dots, a_n)$
of $\{1,2,\dots,n\}$, define $S(a) = \sum_{i=1}^n c_i a_i$.
Prove that there exist two permutations $a \neq b$
of $\{1,2,\dots,n\}$ such that $n!$
is a divisor of $S(a)-S(b)$.
\end{mdframed}
Assume for contradiction that all the $S(a)$ are distinct modulo $n!$.
Then summing across all permutations gives
\begin{align*}
  1 + 2 + \dots + n!
  &\equiv \sum_a S(a) \\
  &= \sum_a \sum_{i=1}^n c_i a_i \\
  &= \sum_{i=1}^n c_i \sum_a a_i \\
  &= \sum_{i=1}^n c_i \cdot \left( (n-1)! \cdot (1+\dots+n) \right) \\
  &= (n-1)! \cdot \frac{n(n+1)}{2} \sum_{i=1}^n c_i \\
  &= n! \cdot \frac{n+1}{2} \sum_{i=1}^n c_i \\
  &\equiv 0
\end{align*}
since $\half(n+1)$ is an integer.
But on the other hand
$1 + 2 + \dots + n! = \frac{n!(n!+1)}{2}$
which is not divisible by $n!$ if $n > 1$,
as the quotient is the non-integer $\frac{n!+1}{2}$.
This is a contradiction.
\pagebreak

\subsection{IMO 2001/5}
\textsl{Available online at \url{https://aops.com/community/p119207}.}
\begin{mdframed}[style=mdpurplebox,frametitle={Problem statement}]
Let $ABC$ be a triangle.
Let $\ol{AP}$ bisect $\angle BAC$ and let $\ol{BQ}$ bisect $\angle ABC$,
with $P$ on $\ol{BC}$ and $Q$ on $\ol{AC}$.
If $AB + BP = AQ + QB$ and $\angle BAC = 60\dg$,
what are the angles of the triangle?
\end{mdframed}
The answer is $\angle B = 80\dg$ and $\angle C = 40\dg$.
Set $x = \angle ABQ = \angle QBC$, so that $\angle QCB = 120\dg - 2x$.
We observe $\angle AQB = 120\dg - x$ and $\angle APB = 150\dg - 2x$.

\begin{center}
\begin{asy}
  pair A = dir(130);
  pair B = dir(210);
  pair C = dir(330);
  pair I = incenter(A, B, C);
  pair P = extension(A, I, B, C);
  pair Q = extension(B, I, A, C);

  dot("$A$", A, dir(A));
  dot("$B$", B, dir(B));
  dot("$C$", C, dir(C));
  dot("$P$", P, dir(P));
  dot("$Q$", Q, dir(Q));

  draw(A--B--C--cycle);
  draw(A--P);
  draw(B--Q);

  label("$30^\circ$", A+0.3*dir((pair)incenter(B,A,P)-A));
  label("$30^\circ$", A+0.3*dir((pair)incenter(P,A,C)-A));
  label(rotate(-20)*"$120^\circ-2x$", C+0.3*dir((pair)incenter(A,C,B)-C));
  label("$x$", B+0.2*dir((pair)incenter(C,B,Q)-B));
  label("$x$", B+0.2*dir((pair)incenter(Q,B,A)-B));
\end{asy}
\end{center}

Now by the law of sines, we may compute
\begin{align*}
  BP &= AB \cdot \frac{\sin 30\dg}{\sin (150\dg- 2x)}  \\
  AQ &= AB \cdot \frac{\sin x}{\sin (120\dg- x)}  \\
  QB &= AB \cdot \frac{\sin 60\dg}{\sin (120\dg- x)}.
\end{align*}
So, the relation $AB + BP = AQ + QB$ is exactly
\[ 1 + \frac{\sin 30\dg}{\sin (150\dg - 2x)}
= \frac{\sin x + \sin 60\dg}{\sin (120\dg - x)}. \]
This is now a trig problem, and we simply solve for $x$.
There are many possible approaches
and we just present one.

First of all, we can write
\[ \sin x + \sin 60\dg
  = 2\sin \left( \half (x+60\dg) \right)
  \cos \left( \half (x-60\dg) \right). \]
On the other hand, $\sin (120\dg - x) = \sin(x+60\dg)$ and
\[ \sin (x+60\dg)
  = 2 \sin \left( \half (x+60\dg) \right)
  \cos \left( \half(x+60\dg) \right) \]
so
\[ \frac{\sin x + \sin 60\dg}{\sin (120\dg - x)}
  = \frac{\cos \left( \half x - 30\dg \right)}%
  {\cos \left( \half x + 30\dg \right)}. \]
Let $y = \half x$ for brevity now. Then
\begin{align*}
  \frac{\cos(y-30\dg)}{\cos(y+30\dg)} - 1
  &= \frac{\cos(y-30\dg)-\cos(y+30\dg)}{\cos(y+30\dg)} \\
  &= \frac{2 \sin (30\dg) \sin y}{\cos(y+30\dg)} \\
  &= \frac{\sin y}{\cos (y+30\dg)}.
\end{align*}
Hence the problem is just
\begin{align*}
  \frac{\sin 30\dg}{\sin(150\dg - 4y)} &= \frac{\sin y}{\cos(y+30\dg)}. \\
  \intertext{Equivalently,}
  \cos(y+30\dg) &= 2\sin y \sin (150\dg -4y) \\
  &= \cos(5y-150\dg) - \cos(150\dg-3y) \\
  &= -\cos(5y+30\dg) + \cos(3y+30\dg).
\end{align*}
Now we are home free, because $3y+30\dg$
is the average of $y+30\dg$ and $5y+30\dg$.
That means we can write
\[ \frac{\cos(y+30\dg)+\cos(5y+30\dg)}{2} = \cos(3y+30\dg) \cos(2y). \]
Hence
\[ \cos(3y+30\dg) \left( 2\cos(2y)-1 \right) = 0. \]
Recall that
\[ y = \half x = \frac 14 \angle B
  < \frac 14 \left( 180\dg - \angle A \right) = 30\dg. \]
Hence it is not possible that $\cos(2y) = \half$,
since the smallest positive value of $y$
that satisfies this is $y=30\dg$.
So $\cos (3y+30\dg) = 0$.

The only permissible value of $y$ is then $y = 20\dg$,
giving $\angle B = 80\dg$ and $\angle C = 40\dg$.
\pagebreak

\subsection{IMO 2001/6}
\textsl{Available online at \url{https://aops.com/community/p119217}.}
\begin{mdframed}[style=mdpurplebox,frametitle={Problem statement}]
Let $a > b > c > d > 0$ be integers satisfying
\[ ac + bd = (b+d+a-c)(b+d-a+c).  \]
Prove that $ab + cd$ is not prime.
\end{mdframed}
The problem condition is equivalent to
\[ ac + bd = (b+d)^2 - (a-c)^2 \]
or
\[ a^2-ac+c^2 = b^2+bd+d^2. \]

Let us construct a quadrilateral $WXYZ$ such that
$WX = a$, $XY = c$, $YZ = b$, $ZW = d$,
and \[ WY = \sqrt{a^2-ac+c^2} = \sqrt{b^2+bd+d^2}.\]
Then by the law of cosines, we obtain $\angle WXY = 60\dg$
and $\angle WZY = 120\dg$.
Hence this quadrilateral is cyclic.

\begin{center}
  \begin{asy}
    pair X = dir(110);
    pair W = dir(210);
    pair Y = dir(330);
    pair Z = dir(280);
    draw(unitcircle);
    draw(W--X--Y--Z--cycle);
    draw(W--Y);

    pair a = midpoint(W--X);
    pair c = midpoint(X--Y);
    pair b = midpoint(Y--Z);
    pair d = midpoint(Z--W);

    dot("$X$", X, dir(X));
    dot("$W$", W, dir(W));
    dot("$Y$", Y, dir(Y));
    dot("$Z$", Z, dir(Z));

    label("$a$", a, dir(a));
    label("$b$", b, -dir(b));
    label("$c$", c, dir(c));
    label("$d$", d, -dir(d));
    label(minipage("\begin{align*} & \sqrt{a^2-ac+c^2} \\ =& \sqrt{b^2+bd+d^2}\end{align*}"), (W+Y)/2, dir(90));
  \end{asy}
\end{center}

By the more precise version of Ptolemy's theorem,
we find that
\[ WY^2 = \frac{(ab+cd)(ad+bc)}{ac+bd}. \]

Now assume for contradiction that that $ab+cd$ is a prime $p$.
Recall that we assumed $a > b > c > d$.
It follows, for example by rearrangement inequality, that
\[ p = ab+cd > ac+bd > ad+bc. \]
Let $y = ac+bd$ and $x = ad+bc$ now.
The point is that \[ p \cdot \frac xy \]
can never be an integer if $p$ is prime and $x < y < p$.
But $WY^2 = a^2-ac+c^2$ is clearly an integer, and this is a contradiction.

Hence $ab+cd$ cannot be prime.
\pagebreak


\end{document}
