% © Evan Chen
% Downloaded from https://web.evanchen.cc/

\documentclass[11pt]{scrartcl}
\usepackage[sexy]{evan}
\ihead{\footnotesize\textbf{\thetitle}}
\ohead{\footnotesize\href{http://web.evanchen.cc}{\ttfamily web.evanchen.cc},
    updated \today}
\title{USAMO 2023 Solution Notes}
\date{\today}

\begin{document}

\maketitle

\begin{abstract}
This is a compilation of solutions
for the 2023 USAMO.
Some of the solutions are my own work,
but many are from the official solutions provided by the organizers
(for which they hold any copyrights),
and others were found by users on the Art of Problem Solving forums.

These notes will tend to be a bit more advanced and terse than the ``official''
solutions from the organizers.
In particular, if a theorem or technique is not known to beginners
but is still considered ``standard'', then I often prefer to
use this theory anyways, rather than try to work around or conceal it.
For example, in geometry problems I typically use directed angles
without further comment, rather than awkwardly work around configuration issues.
Similarly, sentences like ``let $\mathbb{R}$ denote the set of real numbers''
are typically omitted entirely.

Corrections and comments are welcome!
\end{abstract}

\tableofcontents
\newpage

\addtocounter{section}{-1}
\section{Problems}
\begin{enumerate}[\bfseries 1.]
\item %% Problem 1
In an acute triangle $ABC$, let $M$ be the midpoint of $\ol{BC}$.
Let $P$ be the foot of the perpendicular from $C$ to $AM$.
Suppose that the circumcircle of triangle $ABP$
intersects line $BC$ at two distinct points $B$ and $Q$.
Let $N$ be the midpoint of $\ol{AQ}$.
Prove that $NB = NC$.

\item %% Problem 2
Solve over the positive real numbers the functional equation
\[ f(xy+f(x)) = xf(y) + 2. \]

\item %% Problem 3
Consider an $n$-by-$n$ board of unit squares for some odd positive integer $n$.
We say that a collection $C$ of identical dominoes is a
maximal grid-aligned configuration on the board if $C$ consists of $(n^2-1)/2$
dominoes where each domino covers exactly two neighboring squares
and the dominoes don't overlap: $C$ then covers all but one square on the board.
We are allowed to slide (but not rotate) a domino on the board to
cover the uncovered square, resulting in a new maximal grid-aligned configuration
with another square uncovered. Let $k(C)$ be the number of distinct maximal
grid-aligned configurations obtainable from $C$ by repeatedly sliding dominoes.

Find all possible values of $k(C)$ as a function of $n$.

\item %% Problem 4
Positive integers $a$ and $N$ are fixed,
and $N$ positive integers are written on a blackboard.
Alice and Bob play the following game.
On Alice's turn, she must replace some integer $n$ on the board with $n+a$,
and on Bob's turn he must replace some even integer $n$ on the board with $n/2$.
Alice goes first and they alternate turns.
If on his turn Bob has no valid moves, the game ends.

After analyzing the $N$ integers on the board, Bob realizes that,
regardless of what moves Alice makes,
he will be able to force the game to end eventually.
Show that, in fact, for this value of $a$ and these $N$ integers on the board,
the game is guaranteed to end regardless of Alice's or Bob's moves.

\item %% Problem 5
Let $n\geq3$ be an integer. We say that an arrangement of the numbers
$1$, $2$, \dots, $n^2$ in an $n \times n$ table is \emph{row-valid}
if the numbers in each row can be permuted to form an arithmetic progression,
and \emph{column-valid} if the numbers in each column
can be permuted to form an arithmetic progression.

For what values of $n$ is it possible to transform any row-valid arrangement
into a column-valid arrangement by permuting the numbers in each row?

\item %% Problem 6
Let $ABC$ be a triangle with incenter $I$
and excenters $I_a$, $I_b$, $I_c$ opposite $A$, $B$, and $C$, respectively.
Given an arbitrary point $D$ on the circumcircle of $\triangle ABC$
that does not lie on any of the lines $II_a$, $I_bI_c$, or $BC$,
suppose the circumcircles of $\triangle DII_a$ and $\triangle DI_bI_c$
intersect at two distinct points $D$ and $F$.
If $E$ is the intersection of lines $DF$ and $BC$,
prove that $\angle BAD = \angle EAC$.

\end{enumerate}
\pagebreak

\section{Solutions to Day 1}
\subsection{USAMO 2023/1, proposed by Holden Mui}
\textsl{Available online at \url{https://aops.com/community/p27349297}.}
\begin{mdframed}[style=mdpurplebox,frametitle={Problem statement}]
In an acute triangle $ABC$, let $M$ be the midpoint of $\ol{BC}$.
Let $P$ be the foot of the perpendicular from $C$ to $AM$.
Suppose that the circumcircle of triangle $ABP$
intersects line $BC$ at two distinct points $B$ and $Q$.
Let $N$ be the midpoint of $\ol{AQ}$.
Prove that $NB = NC$.
\end{mdframed}
We show several different approaches.
In all solutions, let $D$ denote the foot of the altitude from $A$.

\begin{center}
\begin{asy}
size(10cm);

pair A = dir(115);
pair B = dir(210);
pair C = dir(330);
pair D = foot(A, B, C);
pair M = midpoint(B--C);
pair P = foot(C, A, M);
pair Q = 2*M-D;

filldraw(A--B--C--cycle, opacity(0.1)+cyan, blue);
filldraw(circumcircle(A, B, P), opacity(0.1)+yellow, orange+dashed);
draw(A--D, blue);
draw(A--P--C, deepgreen);

draw(A--Q, red+dashed);
pair N = midpoint(A--Q);
draw(B--N--C, red+dotted);

pair R = foot(B, A, M);
draw(B--R, dashed);

dot("$A$", A, dir(A));
dot("$B$", B, dir(B));
dot("$C$", C, dir(C));
dot("$D$", D, dir(D));
dot("$M$", M, dir(45));
dot("$P$", P, dir(P));
dot("$Q$", Q, dir(315));
dot("$N$", N, dir(45));
dot("$R$", R, dir(30));

/* -----------------------------------------------------------------+
|                 TSQX: by CJ Quines and Evan Chen                  |
| https://github.com/vEnhance/dotfiles/blob/main/py-scripts/tsqx.py |
+-------------------------------------------------------------------+
!size(10cm);
A = dir 115
B = dir 210
C = dir 330
D = foot A B C
M 45 = midpoint B--C
P = foot C A M
Q 315 = 2*M-D
A--B--C--cycle / 0.1 cyan / blue
circumcircle A B P / 0.1 yellow / orange dashed
A--D blue
A--P--C deepgreen
A--Q / red dashed
N 45 = midpoint A--Q
B--N--C / red dotted
R 30 = foot B A M
B--R / dashed
*/
\end{asy}
\end{center}

\paragraph{Most common synthetic approach.}
The solution hinges on the following claim:
\begin{claim*}
  $Q$ coincides with the reflection of $D$ across $M$.
\end{claim*}
\begin{proof}
  Note that $\dang ADC = \dang APC = 90\dg$, so $ADPC$ is cyclic.
  Then by power of a point (with the lengths directed),
  \[ MB \cdot MQ = MA \cdot MP = MC \cdot MD. \]
  Since $MB = MC$, the claim follows.
\end{proof}
It follows that $\ol{MN} \parallel \ol{AD}$,
as $M$ and $N$ are respectively the midpoints of $\ol{AQ}$ and $\ol{DQ}$.
Thus $\ol{MN} \perp \ol{BC}$,
and so $N$ lies on the perpendicular bisector of $\ol{BC}$, as needed.
\begin{remark*}
  [David Lin]
  One can prove the main claim without power of a point as well, as follows:
  Let $R$ be the foot from $B$ to $\ol{AM}$, so $BRCP$ is a parallelogram.
  Note that $ABDR$ is cyclic, and hence
  \[ \dang DRM = \dang DBA = QBA = \dang QPA = \dang QPM. \]
  Thus, $\ol{DR} \parallel \ol{PQ}$, so $DRQ$ is also a parallelogram.
\end{remark*}

\paragraph{Synthetic approach with no additional points at all.}
\begin{claim*}
  $\triangle BPC \sim \triangle ANM$ (oppositely oriented).
\end{claim*}
\begin{proof}
  We have $\triangle BMP \sim \triangle AMQ$ from the given concyclicity of $ABPQ$.
  Then
  \[ \frac{BM}{BP} = \frac{AM}{AQ} \implies
    \frac{2BM}{BP} = \frac{AM}{AQ/2} \implies
    \frac{BC}{BP} = \frac{AM}{AN} \]
  implying the similarity (since $\dang MAQ = \dang BPM$).
\end{proof}
This similarity gives us the equality of directed angles
\[ \dang \left( BC, MN \right) = -\dang \left( PC, AM  \right) = 90\dg \]
as desired.

\paragraph{Synthetic approach using only the point $R$.}
Again let $R$ be the foot from $B$ to $\ol{AM}$, so $BRCP$ is a parallelogram.
\begin{claim*}
  $ARQC$ is cyclic; equivalently, $\triangle MAQ \sim \triangle MCR$.
\end{claim*}
\begin{proof}
  $MR \cdot MA = MP \cdot MA = MB \cdot MQ = MC \cdot MQ$.
\end{proof}

Note that in $\triangle MCR$, the $M$-median is parallel to $\ol{CP}$
and hence perpendicular to $\ol{RM}$.
The same should be true in $\triangle MAQ$ by the similarity,
so $\ol{MN} \perp \ol{MQ}$ as needed.

\paragraph{Cartesian coordinates approach with power of a point.}
Suppose we set $B = (-1,0)$, $M = (0,0)$, $C = (1,0)$, and $A = (a,b)$.
One may compute:
\begin{align*}
  \overleftrightarrow{AM} : 0 &= bx - ay \iff y = \frac ba x \\
  \overleftrightarrow{CP} : 0 &= a(x-1) + by
    \iff y = -\frac ab (x-1) = -\frac ab x + \frac ab. \\
  P &= \left( \frac{a^2}{a^2+b^2}, \frac{ab}{a^2+b^2} \right) \\
\end{align*}
Now note that
\[ AM = \sqrt{a^2+b^2}, \qquad PM = \frac{a}{\sqrt{a^2+b^2}} \]
together with power of a point
\[ AM \cdot PM = BM \cdot QM \]
to immediately deduce that $Q = (a,0)$.
Hence $N = (0, b/2)$ and we're done.

\paragraph{Cartesian coordinates approach without power of a point (outline).}
After computing $A$ and $P$ as above, one could also directly calculate
\begin{align*}
  \text{Perpendicular bisector of $\ol{AB}$}:
    y &= -\frac{a+1}{b} x + \frac{a^2+b^2-1}{2b} \\
  \text{Perpendicular bisector of $\ol{PB}$}:
    y &= - \left( \frac{2a}{b} + \frac{b}{a} \right) x - \frac{b}{2a} \\
  \text{Perpendicular bisector of $\ol{PA}$}:
    y &= - \frac{a}{b} x + \frac{a+a^2+b^2}{2b}. \\
  \text{Circumcenter of $\triangle PAB$}
    &= \left( -\frac{a+1}{2}, \frac{2a^2+2a+b^2}{2b} \right).
\end{align*}
This is enough to extract the coordinates of $Q = (\bullet, 0)$,
because $B = (-1,0)$ is given, and the $x$-coordinate
of the circumcenter should be the average of the $x$-coordinates of $B$ and $Q$.
In other words, $Q = (-a,0)$.
Hence, $N = \left( 0, \frac b2 \right)$, as needed.

\paragraph{Ill-advised barycentric approach (outline).}
Use reference triangle $ABC$.
The $A$-median is parametrized by $(t:1:1)$ for $t \in \RR$.
So because of $\ol{CP} \perp \ol{AM}$, we are looking for $t$ such that
\[
  \left( \frac{t \vec A + \vec B + \vec C}{t+2} - \vec C \right)
  \perp \left( A - \frac{\vec B + \vec C}{2} \right).
\]
This is equivalent to
\[
  \left( t \vec A + \vec B - (t+1) \vec C \right)
  \perp \left( 2 \vec A - \vec B - \vec C \right).
\]
By the perpendicularity formula for barycentric coordinates (EGMO 7.16),
this is equivalent to
\begin{align*}
  0 &= a^2t - b^2 \cdot (3t+2) + c^2 \cdot (2-t) \\
  &= \left( a^2-3b^2-c^2 \right) t - 2(b^2-c^2) \\
  \implies t &= \frac{2(b^2-c^2)}{a^2-3b^2-c^2}.
\end{align*}
In other words,
\[ P = \left( 2(b^2-c^2) : a^2-3b^2-c^2 : a^2-3b^2-c^2 \right). \]
A long calculation gives
$a^2 y_P z_P + b^2 z_P x_P + c^2 x_P y_P
= (a^2-3b^2-c^2)(a^2-b^2+c^2)(a^2-2b^2-2c^2)$.
Together with $x_P+y_P+z_P=2a^2-4b^2-4c^2$,
this makes the equation of $(ABP)$ as
\[
  0=-a^2yz-b^2zx-c^2xy
  + \frac{a^2-b^2+c^2}{2} z(x+y+z).
\]
To solve for $Q$, set $x=0$ to get to get
\[ a^2yz = \frac{a^2-b^2+c^2}{2} z(y+z)
  \implies \frac yz = \frac{a^2-b^2+c^2}{a^2+b^2-c^2}. \]
In other words,
\[ Q = \left(  0 : a^2-b^2+c^2 : a^2+b^2-c^2 \right). \]
Taking the average with $A = (1,0,0)$ then gives
\[ N = \left(  2a^2 : a^2-b^2+c^2 : a^2+b^2-c^2 \right). \]
The equation for the perpendicular bisector of $\ol{BC}$
is given by (see EGMO 7.19)
\[ 0 = a^2(z-y)+x(c^2-b^2) \]
which contains $N$, as needed.

\paragraph{Extremely ill-advised complex numbers approaches (outline).}
Suppose we pick $a$, $b$, $c$ as the unit circle, and let $m = (b+c)/2$.
Using the fully general ``foot'' formula, one can get
\[
  p = \frac{(a-m) \ol c + (\ol a - \ol m)c
    + \ol a m - a \ol m}{2(\ol a - \ol m)}
    = \frac{a^2 b - a^2 c - a b^2 - 2 a b c - a c^2 + b^2 c + 3 b c^2}
    {4bc-2a(b+c)}
\]
Meanwhile, an extremely ugly calculation will eventually yield
\[ q = \frac{\frac{bc}{a}+b+c-a}{2} \]
so
\[ n = \frac{a+q}{2} = \frac{a+b+c+\frac{bc}{a}}{4}
  = \frac{(a+b)(a+c)}{2a}. \]
There are a few ways to then verify $NB = NC$.
The simplest seems to be to verify that
\[ \frac{n - \frac{b+c}{2}}{b-c} = \frac{a-b-c+\frac{bc}{a}}{4(b-c)}
  = \frac{(a-b)(a-c)}{2a(b-c)} \]
is pure imaginary, which is clear.
\pagebreak

\subsection{USAMO 2023/2, proposed by Carl Schildkraut}
\textsl{Available online at \url{https://aops.com/community/p27349314}.}
\begin{mdframed}[style=mdpurplebox,frametitle={Problem statement}]
Solve over the positive real numbers the functional equation
\[ f(xy+f(x)) = xf(y) + 2. \]
\end{mdframed}
The answer is $f(x) \equiv x+1$,
which is easily verified to be the only linear solution.

We show conversely that $f$ is linear. Let $P(x,y)$ be the assertion.

\begin{claim*}
  $f$ is weakly increasing.
\end{claim*}
\begin{proof}
  Assume for contradiction $a>b$ but $f(a)<f(b)$.
  Choose $y$ such that $ay+f(a)=by+f(b)$, that is $y=\frac{f(b)-f(a)}{a-b}$.
  Then $P(a,y)$ and $P(b,y)$ gives $af(y)+2=bf(y)+2$, which is impossible.
\end{proof}

\begin{claim*}
  [Up to an error of $2$, $f$ is linear]
  We have
  \[ \left\lvert f(x) - (K x + C) \right\rvert \le 2 \]
  where $K \coloneqq \frac{2}{f(1)}$ and $C \coloneqq f(f(1))-2$ are constants.
\end{claim*}
\begin{proof}
  Note $P(1,y)$ gives $\boxed{f(y+f(1)) = f(y)+2}$.
  Hence, $f(nf(1)) = 2(n-1)+f(f(1))$ for $n \ge 1$.
  Combined with weakly increasing, this gives
  \[ 2\left\lfloor \frac{x}{f(1)} \right\rfloor + C
    \le f(x) \le 2\left\lceil \frac{x}{f(1)} \right\rceil + C \]
  which implies the result.
\end{proof}

Rewrite the previous claim to the simpler $f(x) = Kx+O(1)$.
Then for any $x$ and $y$, the above claim gives
\[ K\left( xy + Kx + O(1) \right) + O(1) = x f(y) + 2 \]
which means that
\[ x \cdot \left( Ky + K^2 - f(y) \right) = O(1). \]
If we fix $y$ and consider large $x$,
we see this can only happen if $Ky+K^2-f(y)=0$, i.e.\ $f$ is linear.
\pagebreak

\subsection{USAMO 2023/3, proposed by Holden Mui}
\textsl{Available online at \url{https://aops.com/community/p27349464}.}
\begin{mdframed}[style=mdpurplebox,frametitle={Problem statement}]
Consider an $n$-by-$n$ board of unit squares for some odd positive integer $n$.
We say that a collection $C$ of identical dominoes is a
maximal grid-aligned configuration on the board if $C$ consists of $(n^2-1)/2$
dominoes where each domino covers exactly two neighboring squares
and the dominoes don't overlap: $C$ then covers all but one square on the board.
We are allowed to slide (but not rotate) a domino on the board to
cover the uncovered square, resulting in a new maximal grid-aligned configuration
with another square uncovered. Let $k(C)$ be the number of distinct maximal
grid-aligned configurations obtainable from $C$ by repeatedly sliding dominoes.

Find all possible values of $k(C)$ as a function of $n$.
\end{mdframed}
The answer is that
\[ k(C) \in \left\{ 1, 2, \dots, \left( \frac{n-1}{2} \right)^2 \right\}
  \cup \left\{ \left( \frac{n+1}{2} \right)^2 \right\}. \]

Index the squares by coordinates $(x,y) \in \{1,2,\dots,n\}^2$.
We say a square is \emph{special} if it is empty or
it has the same parity in both coordinates as the empty square.

We now proceed in two cases:

\paragraph{The special squares have both odd coordinates.}
We construct a directed graph $G = G(C)$ whose
vertices are special squares as follows:
for each domino on a special square $s$, we draw a directed edge
from $s$ to the special square that domino points to.
Thus all special squares have an outgoing edge except the empty cell.
\begin{center}
\begin{asy}
size(6cm);

pen border = grey+1.5;

filldraw(box((0, 1), (1, 3)), lightgreen, border);
filldraw(box((0, 3), (2, 4)), lightcyan, border);
filldraw(box((0, 4), (2, 5)), lightgreen, border);
filldraw(box((1, 0), (2, 3)), lightred, border);
filldraw(box((1, 2), (3, 3)), lightblue, border);
filldraw(box((2, 3), (3, 5)), palered, border);
filldraw(box((2, 0), (4, 1)), lightcyan, border);
filldraw(box((2, 1), (4, 2)), lightgreen, border);
filldraw(box((4, 0), (5, 2)), lightred, border);
filldraw(box((3, 3), (5, 4)), lightblue, border);
filldraw(box((3, 2), (5, 3)), palered, border);
filldraw(box((3, 4), (5, 5)), lightcyan, border);
filldraw(box((0, 0), (1, 1)), grey, border);

pair A = (0.5, 4.5);
pair B = (2.5, 4.5);
pair C = (4.5, 4.5);
pair D = (0.5, 2.5);
pair E = (2.5, 2.5);
pair F = (4.5, 2.5);
pair X = (0.5, 0.5);
pair Y = (2.5, 0.5);
pair Z = (4.5, 0.5);

dotfactor *= 2;
dot(A); dot(B); dot(C);
dot(D); dot(E); dot(F);
dot(X); dot(Y); dot(Z);

draw(A--B, EndArrow, Margin(3,3));
draw(C--B, EndArrow, Margin(3,3));
draw(B--E, EndArrow, Margin(3,3));
draw(E--D, EndArrow, Margin(3,3));
draw(F--E, EndArrow, Margin(3,3));
draw(Z--F, EndArrow, Margin(3,3));
draw(Y--Z, EndArrow, Margin(3,3));
draw(D--X, EndArrow, Margin(3,3));
\end{asy}
\end{center}

\begin{claim*}
  Any undirected connected component of $G$ is acyclic unless the cycle
  contains the empty square inside it.
\end{claim*}
\begin{proof}
  Consider a cycle of $G$; we are going to prove
  that the number of chessboard cells enclosed is always odd.

  This can be proven directly by induction,
  but for theatrical effect, we use Pick's theorem.
  Mark the center of every chessboard cell on or inside the cycle
  to get a lattice.
  The dominoes of the cycle then enclose a polyominoe which
  actually consists of $2 \times 2$ squares,
  meaning its area is a multiple of $4$.
  \begin{center}
  \begin{asy}
    unitsize(0.9cm);
    filldraw(box((-0.5,1.5), (0.5,3.5)), paleyellow, grey+1.4);
    filldraw(box((-0.5,3.5), (1.5,4.5)), paleyellow, grey+1.4);
    filldraw(box((1.5,3.5), (3.5,4.5)), paleyellow, grey+1.4);
    filldraw(box((3.5,2.5), (4.5,4.5)), paleyellow, grey+1.4);
    filldraw(box((3.5,0.5), (4.5,2.5)), paleyellow, grey+1.4);
    filldraw(box((2.5,-0.5), (4.5,0.5)), paleyellow, grey+1.4);
    filldraw(box((1.5,-0.5), (2.5,1.5)), paleyellow, grey+1.4);
    filldraw(box((0.5,1.5), (2.5,2.5)), paleyellow, grey+1.4);
    draw((0,2)--(0,4), blue+1.2, EndArrow(TeXHead), Margins);
    draw((0,4)--(2,4), blue+1.2, EndArrow(TeXHead), Margins);
    draw((2,4)--(4,4), blue+1.2, EndArrow(TeXHead), Margins);
    draw((4,4)--(4,2), blue+1.2, EndArrow(TeXHead), Margins);
    draw((4,2)--(4,0), blue+1.2, EndArrow(TeXHead), Margins);
    draw((4,0)--(2,0), blue+1.2, EndArrow(TeXHead), Margins);
    draw((2,0)--(2,2), blue+1.2, EndArrow(TeXHead), Margins);
    draw((2,2)--(0,2), blue+1.2, EndArrow(TeXHead), Margins);
    draw((2,2)--(2,4), blue+dashed, Margins);
    draw((2,2)--(4,2), blue+dashed, Margins);
    dotfactor *= 1.8;
    dot((2,0), blue);
    dot((4,0), blue);
    dot((0,2), blue);
    dot((2,2), blue);
    dot((4,2), blue);
    dot((0,4), blue);
    dot((2,4), blue);
    dot((4,4), blue);
    dot((0,3), black);
    dot((1,4), black);
    dot((3,4), black);
    dot((4,3), black);
    dot((4,1), black);
    dot((3,0), black);
    dot((2,1), black);
    dot((1,2), black);
    dot((1,3), red);
    dot((2,3), red);
    dot((3,3), red);
    dot((3,2), red);
    dot((3,1), red);
  \end{asy}
  \end{center}
  Hence $B/2+I-1$ is a multiple of $4$, in the notation of Pick's theorem.
  As $B$ is twice the number of dominoes,
  and a parity argument on the special squares shows that number is even,
  it follows that $B$ is also a multiple of $4$
  (these correspond to blue and black in the figure above).
  This means $I$ is odd (the red dots in the figure above), as desired.
\end{proof}

Let $T$ be the connected component containing the empty cell.
By the claim, $T$ is acyclic, so it's a tree.
Now, notice that all the arrows point along $T$ towards the empty cell,
and moving a domino corresponds to flipping an arrow.
Therefore:
\begin{claim*}
$k(C)$ is exactly the number of vertices of $T$.
\end{claim*}
\begin{proof}
Starting with the underlying tree,
the set of possible graphs is described by picking one vertex to be the sink
(the empty cell) and then directing all arrows towards it.
\end{proof}

This implies that $k(C) \le \left( \frac{n+1}{2} \right)^2$ in this case.
Equality is achieved if $T$ is a spanning tree of $G$.
One example of a way to achieve this is using the snake configuration below.

\begin{center}
\begin{asy}
unitsize(0.9cm);
dotfactor *= 2;
pen border = grey+1.5;
pair A = (0.5, 4.5);
pair B = (2.5, 4.5);
pair C = (4.5, 4.5);
pair D = (0.5, 2.5);
pair E = (2.5, 2.5);
pair F = (4.5, 2.5);
pair X = (0.5, 0.5);
pair Y = (2.5, 0.5);
pair Z = (4.5, 0.5);

filldraw(box((3, 4), (5, 5)), palecyan, border);
filldraw(box((1, 4), (3, 5)), palecyan, border);
filldraw(box((0, 3), (1, 5)), palecyan, border);
filldraw(box((0, 2), (2, 3)), palecyan, border);
filldraw(box((2, 2), (4, 3)), palecyan, border);
filldraw(box((4, 1), (5, 3)), palecyan, border);
filldraw(box((3, 0), (5, 1)), palecyan, border);
filldraw(box((1, 0), (3, 1)), palecyan, border);
fill(unitsquare, mediumgrey);
filldraw(box((3, 3), (5, 4)), lightred, border);
filldraw(box((1, 3), (3, 4)), lightred, border);
filldraw(box((0, 1), (2, 2)), lightred, border);
filldraw(box((2, 1), (4, 2)), lightred, border);
dot(A); dot(B); dot(C);
dot(D); dot(E); dot(F);
dot(X); dot(Y); dot(Z);

draw(C--B, EndArrow, Margin(3,3));
draw(B--A, EndArrow, Margin(3,3));
draw(A--D, EndArrow, Margin(3,3));
draw(D--E, EndArrow, Margin(3,3));
draw(E--F, EndArrow, Margin(3,3));
draw(F--Z, EndArrow, Margin(3,3));
draw(Z--Y, EndArrow, Margin(3,3));
draw(Y--X, EndArrow, Margin(3,3));
\end{asy}
\end{center}

\begin{remark*}
  In Russia 1997/11.8 it's shown that as long as the missing square is a corner,
  we have $G = T$.
  The proof is given implicitly from our work here:
  when the empty cell is in a corner, it cannot be surrounded,
  ergo the resulting graph has no cycles at all.
  And since the overall graph has one fewer edge than vertex, it's a tree.
\end{remark*}

Conversely, suppose $T$ was \emph{not} a spanning tree, i.e.\ $T \neq G$.
Since in this odd-odd case, $G$ has one fewer edge than vertex,
if $G$ is not a tree, then it must contain at least one cycle.
That cycle encloses every special square of $T$.
In particular, this means that $T$ can't contain any special squares
from the outermost row or column of the $n \times n$ grid.
In this situation, we therefore have $k(C) \le \left( \frac{n-3}{2} \right)^2$.

\paragraph{The special squares have both even coordinates.}
We construct the analogous graph $G$ on the same special squares.
However, in this case, some of the points may not have outgoing edges,
because their domino may ``point'' outside the grid.

\begin{center}
\begin{asy}
size(8cm);

pen border = grey+1.5;

filldraw(box((0, 1), (1, 3)), lightgreen, border);
filldraw(box((0, 3), (2, 4)), lightcyan, border);
filldraw(box((0, 4), (2, 5)), lightgreen, border);
filldraw(box((1, 0), (2, 3)), lightred, border);
filldraw(box((1, 2), (3, 3)), lightblue, border);
filldraw(box((2, 3), (3, 5)), palered, border);
filldraw(box((2, 0), (4, 1)), lightcyan, border);
filldraw(box((2, 1), (4, 2)), lightgreen, border);
filldraw(box((4, 0), (5, 2)), lightred, border);
filldraw(box((0, 0), (1, 1)), grey, border);

filldraw(box((0, -1), (2, 0)), yellow, border);
filldraw(box((2, -1), (4, 0)), brown, border);
filldraw(box((4, -1), (6, 0)), yellow, border);
filldraw(box((5, 0), (6, 2)), brown, border);
filldraw(box((5, 4), (6, 6)), brown, border);
filldraw(box((1, 5), (3, 6)), brown, border);
filldraw(box((-1, 5), (1, 6)), yellow, border);
filldraw(box((-1, -1), (0, 1)), brown, border);
filldraw(box((-1, 1), (0, 3)), yellow, border);
filldraw(box((-1, 3), (0, 5)), brown, border);

filldraw(box((3, 4), (4, 6)), purple, border);
filldraw(box((4, 4), (5, 6)), magenta, border);

filldraw(box((4, 3), (6, 4)), lightblue, border);
filldraw(box((4, 2), (6, 3)), palered, border);
filldraw(box((3, 2), (4, 4)), yellow, border);

pair A = (0.5, 4.5);
pair B = (2.5, 4.5);
pair C = (4.5, 4.5);
pair D = (0.5, 2.5);
pair E = (2.5, 2.5);
pair F = (4.5, 2.5);
pair X = (0.5, 0.5);
pair Y = (2.5, 0.5);
pair Z = (4.5, 0.5);

dotfactor *= 2;
dot(A); dot(B); dot(C);
dot(D); dot(E); dot(F);
dot(X); dot(Y); dot(Z);

draw(A--B, EndArrow, Margin(3,3));
draw(B--E, EndArrow, Margin(3,3));
draw(E--D, EndArrow, Margin(3,3));
draw(Z--F, EndArrow, Margin(3,3));
draw(Y--Z, EndArrow, Margin(3,3));
draw(D--X, EndArrow, Margin(3,3));
\end{asy}
\end{center}

As before, the connected component $T$ containing the empty square is a tree,
and $k(C)$ is exactly the number of vertices of $T$.
Thus to finish the problem we need to give,
for each $k \in \{1,2,\dots,\left( \frac{n-1}{2} \right)^2\}$,
an example of a configuration where $G$ has exactly $k$ vertices.

The construction starts with a ``snake'' picture for
$k = \left( \frac{n-1}{2} \right)^2$,
then decreases $k$ by one by perturbing a suitable set of dominoes.
Rather than write out the procedure in words,
we show the sequence of nine pictures for $n=7$ (where $k=9,8,\dots,1$);
the generalization to larger $n$ is straightforward.

\begin{center}
\begin{asy}
unitsize(0.6cm);
dotfactor *= 2;
pen border = grey+1.5;
pair A = (0.5, 4.5);
pair B = (2.5, 4.5);
pair C = (4.5, 4.5);
pair D = (0.5, 2.5);
pair E = (2.5, 2.5);
pair F = (4.5, 2.5);
pair X = (0.5, 0.5);
pair Y = (2.5, 0.5);
pair Z = (4.5, 0.5);

/* Picture 0 */
filldraw(box((3, 4), (5, 5)), palecyan, border);
filldraw(box((1, 4), (3, 5)), palecyan, border);
filldraw(box((0, 3), (1, 5)), palecyan, border);
filldraw(box((0, 2), (2, 3)), palecyan, border);
filldraw(box((2, 2), (4, 3)), palecyan, border);
filldraw(box((4, 1), (5, 3)), palecyan, border);
filldraw(box((3, 0), (5, 1)), palecyan, border);
filldraw(box((1, 0), (3, 1)), palecyan, border);
fill(unitsquare, mediumgrey);
filldraw(box((3, 3), (5, 4)), lightred, border);
filldraw(box((1, 3), (3, 4)), lightred, border);
filldraw(box((0, 1), (2, 2)), lightred, border);
filldraw(box((2, 1), (4, 2)), lightred, border);
filldraw(box((0, -1), (2, 0)), yellow, border);
filldraw(box((2, -1), (4, 0)), brown, border);
filldraw(box((4, -1), (6, 0)), yellow, border);
filldraw(box((5, 0), (6, 2)), brown, border);
filldraw(box((5, 2), (6, 4)), yellow, border);
filldraw(box((5, 4), (6, 6)), brown, border);
filldraw(box((3, 5), (5, 6)), yellow, border);
filldraw(box((1, 5), (3, 6)), brown, border);
filldraw(box((-1, 5), (1, 6)), yellow, border);
filldraw(box((-1, -1), (0, 1)), brown, border);
filldraw(box((-1, 1), (0, 3)), yellow, border);
filldraw(box((-1, 3), (0, 5)), brown, border);
dot(A); dot(B); dot(C);
dot(D); dot(E); dot(F);
dot(X); dot(Y); dot(Z);

draw(C--B, EndArrow, Margin(3,3));
draw(B--A, EndArrow, Margin(3,3));
draw(A--D, EndArrow, Margin(3,3));
draw(D--E, EndArrow, Margin(3,3));
draw(E--F, EndArrow, Margin(3,3));
draw(F--Z, EndArrow, Margin(3,3));
draw(Z--Y, EndArrow, Margin(3,3));
draw(Y--X, EndArrow, Margin(3,3));
picture p0 = rotate(0)*currentpicture;
currentpicture.erase();

/* Picture 1 */
filldraw(box((4, 4), (6, 5)), palecyan, border);
filldraw(box((1, 4), (3, 5)), palecyan, border);
filldraw(box((0, 3), (1, 5)), palecyan, border);
filldraw(box((0, 2), (2, 3)), palecyan, border);
filldraw(box((2, 2), (4, 3)), palecyan, border);
filldraw(box((4, 1), (5, 3)), palecyan, border);
filldraw(box((3, 0), (5, 1)), palecyan, border);
filldraw(box((1, 0), (3, 1)), palecyan, border);
fill(unitsquare, mediumgrey);
filldraw(box((3, 3), (5, 4)), lightred, border);
filldraw(box((1, 3), (3, 4)), lightred, border);
filldraw(box((0, 1), (2, 2)), lightred, border);
filldraw(box((2, 1), (4, 2)), lightred, border);
filldraw(box((0, -1), (2, 0)), yellow, border);
filldraw(box((2, -1), (4, 0)), brown, border);
filldraw(box((4, -1), (6, 0)), yellow, border);
filldraw(box((5, 0), (6, 2)), brown, border);
filldraw(box((5, 2), (6, 4)), yellow, border);
filldraw(box((3, 4), (4, 6)), brown, border);
filldraw(box((4, 5), (6, 6)), yellow, border);
filldraw(box((1, 5), (3, 6)), brown, border);
filldraw(box((-1, 5), (1, 6)), yellow, border);
filldraw(box((-1, -1), (0, 1)), brown, border);
filldraw(box((-1, 1), (0, 3)), yellow, border);
filldraw(box((-1, 3), (0, 5)), brown, border);
dot(A); dot(B); dot(C);
dot(D); dot(E); dot(F);
dot(X); dot(Y); dot(Z);
draw(B--A, EndArrow, Margin(3,3));
draw(A--D, EndArrow, Margin(3,3));
draw(D--E, EndArrow, Margin(3,3));
draw(E--F, EndArrow, Margin(3,3));
draw(F--Z, EndArrow, Margin(3,3));
draw(Z--Y, EndArrow, Margin(3,3));
draw(Y--X, EndArrow, Margin(3,3));
picture p1 = rotate(0)*currentpicture;
currentpicture.erase();

/* Picture 2 */
filldraw(box((4, 4), (6, 5)), palecyan, border);
filldraw(box((2, 4), (4, 5)), palecyan, border);
filldraw(box((0, 3), (1, 5)), palecyan, border);
filldraw(box((0, 2), (2, 3)), palecyan, border);
filldraw(box((2, 2), (4, 3)), palecyan, border);
filldraw(box((4, 1), (5, 3)), palecyan, border);
filldraw(box((3, 0), (5, 1)), palecyan, border);
filldraw(box((1, 0), (3, 1)), palecyan, border);
fill(unitsquare, mediumgrey);
filldraw(box((3, 3), (5, 4)), lightred, border);
filldraw(box((1, 3), (3, 4)), lightred, border);
filldraw(box((0, 1), (2, 2)), lightred, border);
filldraw(box((2, 1), (4, 2)), lightred, border);
filldraw(box((0, -1), (2, 0)), yellow, border);
filldraw(box((2, -1), (4, 0)), brown, border);
filldraw(box((4, -1), (6, 0)), yellow, border);
filldraw(box((5, 0), (6, 2)), brown, border);
filldraw(box((5, 2), (6, 4)), yellow, border);
filldraw(box((1, 4), (2, 6)), brown, border);
filldraw(box((4, 5), (6, 6)), yellow, border);
filldraw(box((2, 5), (4, 6)), brown, border);
filldraw(box((-1, 5), (1, 6)), yellow, border);
filldraw(box((-1, -1), (0, 1)), brown, border);
filldraw(box((-1, 1), (0, 3)), yellow, border);
filldraw(box((-1, 3), (0, 5)), brown, border);
dot(A); dot(B); dot(C);
dot(D); dot(E); dot(F);
dot(X); dot(Y); dot(Z);
draw(A--D, EndArrow, Margin(3,3));
draw(D--E, EndArrow, Margin(3,3));
draw(E--F, EndArrow, Margin(3,3));
draw(F--Z, EndArrow, Margin(3,3));
draw(Z--Y, EndArrow, Margin(3,3));
draw(Y--X, EndArrow, Margin(3,3));
picture p2 = rotate(0)*currentpicture;
currentpicture.erase();


/* Picture 3 */
filldraw(box((4, 4), (6, 5)), palecyan, border);
filldraw(box((2, 4), (4, 5)), palecyan, border);
filldraw(box((-1, 4), (1, 5)), palecyan, border);
filldraw(box((0, 2), (2, 3)), palecyan, border);
filldraw(box((2, 2), (4, 3)), palecyan, border);
filldraw(box((4, 1), (5, 3)), palecyan, border);
filldraw(box((3, 0), (5, 1)), palecyan, border);
filldraw(box((1, 0), (3, 1)), palecyan, border);
fill(unitsquare, mediumgrey);
filldraw(box((3, 3), (5, 4)), lightred, border);
filldraw(box((1, 3), (3, 4)), lightred, border);
filldraw(box((0, 1), (2, 2)), lightred, border);
filldraw(box((2, 1), (4, 2)), lightred, border);
filldraw(box((0, -1), (2, 0)), yellow, border);
filldraw(box((2, -1), (4, 0)), brown, border);
filldraw(box((4, -1), (6, 0)), yellow, border);
filldraw(box((5, 0), (6, 2)), brown, border);
filldraw(box((5, 2), (6, 4)), yellow, border);
filldraw(box((1, 4), (2, 6)), brown, border);
filldraw(box((4, 5), (6, 6)), yellow, border);
filldraw(box((2, 5), (4, 6)), brown, border);
filldraw(box((-1, 5), (1, 6)), yellow, border);
filldraw(box((-1, -1), (0, 1)), brown, border);
filldraw(box((-1, 1), (0, 3)), yellow, border);
filldraw(box((-1, 3), (1, 4)), brown, border);
dot(A); dot(B); dot(C);
dot(D); dot(E); dot(F);
dot(X); dot(Y); dot(Z);
draw(D--E, EndArrow, Margin(3,3));
draw(E--F, EndArrow, Margin(3,3));
draw(F--Z, EndArrow, Margin(3,3));
draw(Z--Y, EndArrow, Margin(3,3));
draw(Y--X, EndArrow, Margin(3,3));
picture p3 = rotate(0)*currentpicture;
currentpicture.erase();

/* Picture 4 */
filldraw(box((4, 4), (6, 5)), palecyan, border);
filldraw(box((2, 4), (4, 5)), palecyan, border);
filldraw(box((-1, 4), (1, 5)), palecyan, border);
filldraw(box((-1, 2), (1, 3)), palecyan, border);
filldraw(box((2, 2), (4, 3)), palecyan, border);
filldraw(box((4, 1), (5, 3)), palecyan, border);
filldraw(box((3, 0), (5, 1)), palecyan, border);
filldraw(box((1, 0), (3, 1)), palecyan, border);
fill(unitsquare, mediumgrey);
filldraw(box((3, 3), (5, 4)), lightred, border);
filldraw(box((1, 3), (3, 4)), lightred, border);
filldraw(box((-1, 1), (1, 2)), lightred, border);
filldraw(box((2, 1), (4, 2)), lightred, border);
filldraw(box((0, -1), (2, 0)), yellow, border);
filldraw(box((2, -1), (4, 0)), brown, border);
filldraw(box((4, -1), (6, 0)), yellow, border);
filldraw(box((5, 0), (6, 2)), brown, border);
filldraw(box((5, 2), (6, 4)), yellow, border);
filldraw(box((1, 4), (2, 6)), brown, border);
filldraw(box((4, 5), (6, 6)), yellow, border);
filldraw(box((2, 5), (4, 6)), brown, border);
filldraw(box((-1, 5), (1, 6)), yellow, border);
filldraw(box((-1, -1), (0, 1)), brown, border);
filldraw(box((1, 1), (2, 3)), yellow, border);
filldraw(box((-1, 3), (1, 4)), brown, border);
dot(A); dot(B); dot(C);
dot(D); dot(E); dot(F);
dot(X); dot(Y); dot(Z);
draw(E--F, EndArrow, Margin(3,3));
draw(F--Z, EndArrow, Margin(3,3));
draw(Z--Y, EndArrow, Margin(3,3));
draw(Y--X, EndArrow, Margin(3,3));
picture p4 = rotate(0)*currentpicture;
currentpicture.erase();

/* Picture 5 */
filldraw(box((4, 4), (6, 5)), palecyan, border);
filldraw(box((2, 4), (4, 5)), palecyan, border);
filldraw(box((-1, 4), (1, 5)), palecyan, border);
filldraw(box((-1, 2), (1, 3)), palecyan, border);
filldraw(box((1, 2), (3, 3)), palecyan, border);
filldraw(box((4, 1), (5, 3)), palecyan, border);
filldraw(box((3, 0), (5, 1)), palecyan, border);
filldraw(box((1, 0), (3, 1)), palecyan, border);
fill(unitsquare, mediumgrey);
filldraw(box((3, 3), (5, 4)), lightred, border);
filldraw(box((1, 3), (3, 4)), lightred, border);
filldraw(box((-1, 1), (1, 2)), lightred, border);
filldraw(box((1, 1), (3, 2)), lightred, border);
filldraw(box((0, -1), (2, 0)), yellow, border);
filldraw(box((2, -1), (4, 0)), brown, border);
filldraw(box((4, -1), (6, 0)), yellow, border);
filldraw(box((5, 0), (6, 2)), brown, border);
filldraw(box((5, 2), (6, 4)), yellow, border);
filldraw(box((1, 4), (2, 6)), brown, border);
filldraw(box((4, 5), (6, 6)), yellow, border);
filldraw(box((2, 5), (4, 6)), brown, border);
filldraw(box((-1, 5), (1, 6)), yellow, border);
filldraw(box((-1, -1), (0, 1)), brown, border);
filldraw(box((3, 1), (4, 3)), yellow, border);
filldraw(box((-1, 3), (1, 4)), brown, border);
dot(A); dot(B); dot(C);
dot(D); dot(E); dot(F);
dot(X); dot(Y); dot(Z);
draw(F--Z, EndArrow, Margin(3,3));
draw(Z--Y, EndArrow, Margin(3,3));
draw(Y--X, EndArrow, Margin(3,3));
picture p5 = rotate(0)*currentpicture;
currentpicture.erase();

/* Picture 6 */
filldraw(box((4, 4), (6, 5)), palecyan, border);
filldraw(box((2, 4), (4, 5)), palecyan, border);
filldraw(box((-1, 4), (1, 5)), palecyan, border);
filldraw(box((-1, 2), (1, 3)), palecyan, border);
filldraw(box((1, 2), (3, 3)), palecyan, border);
filldraw(box((3, 2), (5, 3)), palecyan, border);
filldraw(box((3, 0), (5, 1)), palecyan, border);
filldraw(box((1, 0), (3, 1)), palecyan, border);
fill(unitsquare, mediumgrey);
filldraw(box((3, 3), (5, 4)), lightred, border);
filldraw(box((1, 3), (3, 4)), lightred, border);
filldraw(box((-1, 1), (1, 2)), lightred, border);
filldraw(box((1, 1), (3, 2)), lightred, border);
filldraw(box((0, -1), (2, 0)), yellow, border);
filldraw(box((2, -1), (4, 0)), brown, border);
filldraw(box((4, -1), (6, 0)), yellow, border);
filldraw(box((5, 0), (6, 2)), brown, border);
filldraw(box((5, 2), (6, 4)), yellow, border);
filldraw(box((1, 4), (2, 6)), brown, border);
filldraw(box((4, 5), (6, 6)), yellow, border);
filldraw(box((2, 5), (4, 6)), brown, border);
filldraw(box((-1, 5), (1, 6)), yellow, border);
filldraw(box((-1, -1), (0, 1)), brown, border);
filldraw(box((3, 1), (5, 2)), yellow, border);
filldraw(box((-1, 3), (1, 4)), brown, border);
dot(A); dot(B); dot(C);
dot(D); dot(E); dot(F);
dot(X); dot(Y); dot(Z);
draw(Z--Y, EndArrow, Margin(3,3));
draw(Y--X, EndArrow, Margin(3,3));
picture p6 = rotate(0)*currentpicture;
currentpicture.erase();

/* Picture 7 */
filldraw(box((4, 4), (6, 5)), palecyan, border);
filldraw(box((2, 4), (4, 5)), palecyan, border);
filldraw(box((-1, 4), (1, 5)), palecyan, border);
filldraw(box((-1, 2), (1, 3)), palecyan, border);
filldraw(box((1, 2), (3, 3)), palecyan, border);
filldraw(box((3, 2), (5, 3)), palecyan, border);
filldraw(box((4, 0), (6, 1)), palecyan, border);
filldraw(box((1, 0), (3, 1)), palecyan, border);
fill(unitsquare, mediumgrey);
filldraw(box((3, 3), (5, 4)), lightred, border);
filldraw(box((1, 3), (3, 4)), lightred, border);
filldraw(box((-1, 1), (1, 2)), lightred, border);
filldraw(box((1, 1), (3, 2)), lightred, border);
filldraw(box((0, -1), (2, 0)), yellow, border);
filldraw(box((2, -1), (4, 0)), brown, border);
filldraw(box((4, -1), (6, 0)), yellow, border);
filldraw(box((3, 0), (4, 2)), brown, border);
filldraw(box((5, 2), (6, 4)), yellow, border);
filldraw(box((1, 4), (2, 6)), brown, border);
filldraw(box((4, 5), (6, 6)), yellow, border);
filldraw(box((2, 5), (4, 6)), brown, border);
filldraw(box((-1, 5), (1, 6)), yellow, border);
filldraw(box((-1, -1), (0, 1)), brown, border);
filldraw(box((4, 1), (6, 2)), yellow, border);
filldraw(box((-1, 3), (1, 4)), brown, border);
dot(A); dot(B); dot(C);
dot(D); dot(E); dot(F);
dot(X); dot(Y); dot(Z);
draw(Y--X, EndArrow, Margin(3,3));
picture p7 = rotate(0)*currentpicture;
currentpicture.erase();

/* Picture 8 */
filldraw(box((4, 4), (6, 5)), palecyan, border);
filldraw(box((2, 4), (4, 5)), palecyan, border);
filldraw(box((-1, 4), (1, 5)), palecyan, border);
filldraw(box((-1, 2), (1, 3)), palecyan, border);
filldraw(box((1, 2), (3, 3)), palecyan, border);
filldraw(box((3, 2), (5, 3)), palecyan, border);
filldraw(box((4, 0), (6, 1)), palecyan, border);
filldraw(box((2, 0), (4, 1)), palecyan, border);
fill(unitsquare, mediumgrey);
filldraw(box((3, 3), (5, 4)), lightred, border);
filldraw(box((1, 3), (3, 4)), lightred, border);
filldraw(box((-1, 1), (1, 2)), lightred, border);
filldraw(box((2, 1), (4, 2)), lightred, border);
filldraw(box((0, -1), (2, 0)), yellow, border);
filldraw(box((2, -1), (4, 0)), brown, border);
filldraw(box((4, -1), (6, 0)), yellow, border);
filldraw(box((1, 0), (2, 2)), brown, border);
filldraw(box((5, 2), (6, 4)), yellow, border);
filldraw(box((1, 4), (2, 6)), brown, border);
filldraw(box((4, 5), (6, 6)), yellow, border);
filldraw(box((2, 5), (4, 6)), brown, border);
filldraw(box((-1, 5), (1, 6)), yellow, border);
filldraw(box((-1, -1), (0, 1)), brown, border);
filldraw(box((4, 1), (6, 2)), yellow, border);
filldraw(box((-1, 3), (1, 4)), brown, border);
dot(A); dot(B); dot(C);
dot(D); dot(E); dot(F);
dot(X); dot(Y); dot(Z);
picture p8 = rotate(0)*currentpicture;
currentpicture.erase();

add(shift(0,0)*p0);
add(shift(8,0)*p1);
add(shift(16,0)*p2);
add(shift(0,-8)*p3);
add(shift(8,-8)*p4);
add(shift(16,-8)*p5);
add(shift(0,-16)*p6);
add(shift(8,-16)*p7);
add(shift(16,-16)*p8);
\end{asy}
\end{center}
\pagebreak

\section{Solutions to Day 2}
\subsection{USAMO 2023/4, proposed by Carl Schildkraut}
\textsl{Available online at \url{https://aops.com/community/p27349336}.}
\begin{mdframed}[style=mdpurplebox,frametitle={Problem statement}]
Positive integers $a$ and $N$ are fixed,
and $N$ positive integers are written on a blackboard.
Alice and Bob play the following game.
On Alice's turn, she must replace some integer $n$ on the board with $n+a$,
and on Bob's turn he must replace some even integer $n$ on the board with $n/2$.
Alice goes first and they alternate turns.
If on his turn Bob has no valid moves, the game ends.

After analyzing the $N$ integers on the board, Bob realizes that,
regardless of what moves Alice makes,
he will be able to force the game to end eventually.
Show that, in fact, for this value of $a$ and these $N$ integers on the board,
the game is guaranteed to end regardless of Alice's or Bob's moves.
\end{mdframed}
For $N=1$, there is nothing to prove.
We address $N \ge 2$ only henceforth.
Let $S$ denote the numbers on the board.

\begin{claim*}
  When $N \ge 2$, if $\nu_2(x) < \nu_2(a)$ for all $x \in S$,
  the game must terminate no matter what either player does.
\end{claim*}
\begin{proof}
  The $\nu_2$ of a number is unchanged by Alice's move
  and decreases by one on Bob's move.
  The game ends when every $\nu_2$ is zero.

  Hence, in fact the game will always terminate in exactly
  $\sum_{x \in S} \nu_2(x)$ moves in this case,
  regardless of what either player does.
\end{proof}

\begin{claim*}
  When $N \ge 2$, if there exists a number $x$ on the board such that
  $\nu_2(x) \ge \nu_2(a)$, then Alice can cause the game to go on forever.
\end{claim*}
\begin{proof}
  Denote by $x$ the first entry of the board (its value changes over time).
  Then Alice's strategy is to:
  \begin{itemize}
    \ii Operate on the first entry if $\nu_2(x) = \nu_2(a)$
    (the new entry thus has $\nu_2(x+a) > \nu_2(a)$);
    \ii Operate on any other entry besides the first one, otherwise.
  \end{itemize}
  A double induction then shows that
  \begin{itemize}
    \ii Just before each of Bob's turns, $\nu_2(x) > \nu_2(a)$ always holds; and
    \ii After each of Bob's turns, $\nu_2(x) \ge \nu_2(a)$ always holds.
  \end{itemize}
  In particular Bob will never run out of legal moves,
  since halving $x$ is always legal.
\end{proof}
\pagebreak

\subsection{USAMO 2023/5, proposed by Ankan Bhattacharya}
\textsl{Available online at \url{https://aops.com/community/p27349487}.}
\begin{mdframed}[style=mdpurplebox,frametitle={Problem statement}]
Let $n\geq3$ be an integer. We say that an arrangement of the numbers
$1$, $2$, \dots, $n^2$ in an $n \times n$ table is \emph{row-valid}
if the numbers in each row can be permuted to form an arithmetic progression,
and \emph{column-valid} if the numbers in each column
can be permuted to form an arithmetic progression.

For what values of $n$ is it possible to transform any row-valid arrangement
into a column-valid arrangement by permuting the numbers in each row?
\end{mdframed}
Answer: $n$ prime only.

\paragraph{Proof for $n$ prime.}
Suppose $n = p$.
In an arithmetic progression with $p$ terms, it's easy to see that either
every term has a different residue modulo $p$ (if the common difference
is not a multiple of $p$), or all of the residues coincide
(when the common difference is a multiple of $p$).

So, look at the multiples of $p$ in a row-valid table;
there is either $1$ or $p$ per row.
As there are $p$ such numbers total, there are two cases:
\begin{itemize}
  \ii If all the multiples of $p$ are in the same row,
  then the common difference in each row is a multiple of $p$.
  In fact, it must be exactly $p$ for size reasons.
  In other words, up to permutation the rows are just
  the $k \pmod p$ numbers in some order, and this is obviously column-valid
  because we can now permute such that the $k$th column
  contains exactly $\{(k-1)p+1, (k-1)p+2, \dots, kp\}$.

  \ii If all the multiples of $p$ are in different rows,
  then it follows each row contains every residue modulo $p$ exactly once.
  So we can permute to a column-valid arrangement by ensuring
  the $k$th column contains all the $k \pmod p$ numbers.
\end{itemize}

\paragraph{Counterexample for $n$ composite (due to Anton Trygub).}
Let $p$ be any prime divisor of $n$.
Construct the table as follows:
\begin{itemize}
  \ii Row $1$ contains $1$ through $n$.
  \ii Rows $2$ through $p+1$ contain the numbers from $p+1$ to $np+n$
  partitioned into arithmetic progressions with common difference $p$.
  \ii The rest of the rows contain the remaining numbers in reading order.
\end{itemize}
For example, when $p=2$ and $n=10$, we get the following table:
\[
  \begin{bmatrix}
    1 & 2 & 3 & 4 & 5 & 6 & 7 & 8 & 9 & 10 \\
    \mathbf{\color{red} 11} &
      \mathbf{\color{red} 13} &
      \mathbf{\color{red} 15} &
      \mathbf{\color{red} 17} &
      \mathbf{\color{red} 19} &
      \mathbf{\color{red} 21} &
      \mathbf{\color{red} 23} &
      \mathbf{\color{red} 25} &
      \mathbf{\color{red} 27} &
      \mathbf{\color{red} 29} \\
    \mathbf{\color{red} 12} &
      \mathbf{\color{red} 14} &
      \mathbf{\color{red} 16} &
      \mathbf{\color{red} 18} &
      \mathbf{\color{red} 20} &
      \mathbf{\color{red} 22} &
      \mathbf{\color{red} 24} &
      \mathbf{\color{red} 26} &
      \mathbf{\color{red} 28} &
      \mathbf{\color{red} 30} \\
    31 & 32 & 33 & 34 & 35 & 36 & 37 & 38 & 39 & 40 \\
    41 & 42 & 43 & 44 & 45 & 46 & 47 & 48 & 49 & 50 \\
    51 & 52 & 53 & 54 & 55 & 56 & 57 & 58 & 59 & 60 \\
    61 & 62 & 63 & 64 & 65 & 66 & 67 & 68 & 69 & 70 \\
    71 & 72 & 73 & 74 & 75 & 76 & 77 & 78 & 79 & 80 \\
    81 & 82 & 83 & 84 & 85 & 86 & 87 & 88 & 89 & 90 \\
    91 & 92 & 93 & 94 & 95 & 96 & 97 & 98 & 99 & 100
  \end{bmatrix}
\]
We claim this works fine.
Assume for contradiction the rows may be permuted to obtain
a column-valid arrangement.
Then the $n$ columns should be arithmetic progressions
whose smallest element is in $[1,n]$
and whose largest element is in $[n^2-n+1, n^2]$.
These two elements must be congruent modulo $n-1$,
so in particular the column containing $2$ must end with $n^2-n+2$.

Hence in that column, the common difference must in fact be exactly $n$.
And yet $n+2$ and $2n+2$ are in the same row, contradiction.
\pagebreak

\subsection{USAMO 2023/6, proposed by Zack Chroman}
\textsl{Available online at \url{https://aops.com/community/p27349354}.}
\begin{mdframed}[style=mdpurplebox,frametitle={Problem statement}]
Let $ABC$ be a triangle with incenter $I$
and excenters $I_a$, $I_b$, $I_c$ opposite $A$, $B$, and $C$, respectively.
Given an arbitrary point $D$ on the circumcircle of $\triangle ABC$
that does not lie on any of the lines $II_a$, $I_bI_c$, or $BC$,
suppose the circumcircles of $\triangle DII_a$ and $\triangle DI_bI_c$
intersect at two distinct points $D$ and $F$.
If $E$ is the intersection of lines $DF$ and $BC$,
prove that $\angle BAD = \angle EAC$.
\end{mdframed}
Here are two approaches.
\begin{center}
\begin{asy}
size(9cm);

pair A = dir(115);
pair B = dir(210);
pair C = dir(330);
pair D = dir(240);
pair E = extension(B, C, A, B*C/D);
pair I = incenter(A, B, C);
pair M = dir(270);
pair I_a = 2*M-I;
pair I_b = extension(B, I, C, I_a);
pair I_c = extension(C, I, B, I_a);
pair F = -D+2*foot(circumcenter(D, I, I_a), D, E);
draw(D--F, deepgreen);

filldraw(unitcircle, opacity(0.1)+lightcyan, blue);
filldraw(A--B--C--cycle, opacity(0.1)+lightcyan, blue);
draw(circumcircle(D, I, I_a), lightred);
draw(circumcircle(D, I_b, I_c), lightred);

draw(A--I_a, blue);
draw(I_a--I_b--I_c--cycle, grey);
draw(D--A--E, deepcyan+dashed);

dot("$A$", A, dir(A));
dot("$B$", B, dir(170));
dot("$C$", C, dir(350));
dot("$D$", D, dir(D));
dot("$E$", E, dir(280));
dot("$I$", I, dir(160));
dot("$M$", M, dir(225));
dot("$I_a$", I_a, dir(I_a));
dot("$I_b$", I_b, dir(I_b));
dot("$I_c$", I_c, dir(I_c));
dot("$F$", F, dir(F));

/* -----------------------------------------------------------------+
|                 TSQX: by CJ Quines and Evan Chen                  |
| https://github.com/vEnhance/dotfiles/blob/main/py-scripts/tsqx.py |
+-------------------------------------------------------------------+
!size(9cm);
A = dir 115
B 170 = dir 210
C 350 = dir 330
D = dir 240
E 280 = extension B C A B*C/D
I 160 = incenter A B C
M 225 = dir 270
I_a = 2*M-I
I_b = extension B I C I_a
I_c = extension C I B I_a
F = -D+2*foot (circumcenter D I I_a) D E
D--F deepgreen
unitcircle / 0.1 lightcyan / blue
A--B--C--cycle / 0.1 lightcyan / blue
circumcircle D I I_a / lightred
circumcircle D I_b I_c / lightred
A--I_a / blue
I_a--I_b--I_c--cycle / grey
D--A--E / deepcyan dashed
*/
\end{asy}
\end{center}

\paragraph{Barycentric coordinates (Carl Schildkraut).}
With reference triangle $\triangle ABC$, set $D = (r:s:t)$.
\begin{claim*}
  The equations of $(DII_a)$ and $(DI_bI_c)$ are, respectively,
  \begin{align*}
    0 &= -a^2yz-b^2zx-c^2xy + (x+y+z)
    \cdot \left( bc x - \frac{bcr}{cs-bt} (cy-bz) \right) \\
    0 &= -a^2yz-b^2zx-c^2xy + (x+y+z)
    \cdot \left( -bc x + \frac{bcr}{cs+bt} (cy+bz) \right).
  \end{align*}
\end{claim*}
\begin{proof}
  Since $D \in (ABC)$, we have $a^2st+b^2tr+c^2rs=0$.
  Now each equation can be verified by direct substitution of three points.
\end{proof}

By EGMO Lemma 7.24, the radical axis is then given by
\[ \ol{DF}: bc x - \frac{bcr}{cs-bt} (cy-bz)
    = -bc x + \frac{bcr}{cs+bt} (cy+bz). \]
Now the point
\[ \left( 0 : \frac{b^2}{s} : \frac{c^2}{t} \right)
  = \left( 0 : b^2t : c^2s \right) \]
lies on line $DF$ by inspection, and is obviously on line $BC$,
hence it coincides with $E$.
This lies on the isogonal of $\ol{AD}$ (by EGMO Lemma 7.6), as needed.

\paragraph{Synthetic approach (Anant Mudgal).}
Focus on just $(DII_a)$.
Let $P$ be the second intersection of $(DII_a)$ with $(ABC)$,
and let $M$ be the midpoint of minor arc $\arc{BC}$.
Then by radical axis, lines $AM$, $DP$, and $BC$ are concurrent at a point $K$.

Let $E' = \ol{PM} \cap \ol{BC}$.

\begin{center}
\begin{asy}
size(9cm);

pair A = dir(140);
pair B = dir(210);
pair C = dir(330);
pair D = dir(240);
pair E_prime = extension(B, C, A, B*C/D);
pair I = incenter(A, B, C);
pair M = dir(270);
pair I_a = 2*M-I;
pair K = extension(A, I, B, C);
pair P = extension(D, K, M, E_prime);
pair X = 2*M-E_prime;

filldraw(unitcircle, opacity(0.1)+lightcyan, blue);
filldraw(A--B--C--cycle, opacity(0.1)+lightcyan, blue);
filldraw(circumcircle(D, I, I_a), opacity(0.1)+yellow, blue);
draw(A--I_a, blue);
draw(D--P--X, orange);
filldraw(X--I--E_prime--I_a--cycle, opacity(0.1)+magenta, red);
filldraw(circumcircle(B, I, C), opacity(0.1)+grey, grey);

draw(D--A--E_prime, deepcyan+dashed);

dot("$A$", A, dir(A));
dot("$B$", B, dir(B));
dot("$C$", C, dir(350));
dot("$D$", D, dir(D));
dot("$E'$", E_prime, dir(325));
dot("$I$", I, dir(160));
dot("$M$", M, dir(M));
dot("$I_a$", I_a, dir(I_a));
dot("$K$", K, dir(325));
dot("$P$", P, dir(P));
dot("$X$", X, dir(X));

/* -----------------------------------------------------------------+
|                 TSQX: by CJ Quines and Evan Chen                  |
| https://github.com/vEnhance/dotfiles/blob/main/py-scripts/tsqx.py |
+-------------------------------------------------------------------+
!size(9cm);
A = dir 140
B = dir 210
C 350 = dir 330
D = dir 240
E' 325 = extension B C A B*C/D
I 160 = incenter A B C
M = dir 270
I_a = 2*M-I
K 325 = extension A I B C
P = extension D K M E'
X = 2*M-E'
unitcircle / 0.1 lightcyan / blue
A--B--C--cycle / 0.1 lightcyan / blue
circumcircle D I I_a / 0.1 yellow / blue
A--I_a / blue
D--P--X / orange
X--I--E'--I_a--cycle / 0.1 magenta / red
circumcircle B I C / 0.1 grey / grey
D--A--E' / deepcyan dashed
*/
\end{asy}
\end{center}

\begin{claim*}
  We have $\dang BAD = \dang E'AC$.
\end{claim*}
\begin{proof}
  By shooting lemma, $AKE'P$ is cyclic, so
  \[ \dang KAE' = \dang KPE' = \dang DPM = \dang DAM. \qedhere \]
\end{proof}

\begin{claim*}
  The power of point $E'$ with respect to $(DII_a)$ is $2 E'B \cdot E'C$.
\end{claim*}
\begin{proof}
  Construct parallelogram $IE'I_aX$.
  Since $MI^2 = ME' \cdot MP$, we can get
  \[ \dang XI_aI = \dang I_aIE' = \dang MIE' = \dang MPI = \dang XPI. \]
  Hence $X$ lies on $(DII_a)$,
  and $E'X \cdot E'P = 2 E'M \cdot E'P = 2 E'B \cdot E'C$.
\end{proof}

Repeat the argument on $(DI_bI_c)$;
the same point $E'$ (because of the first claim)
then has power $2 E'B \cdot E'C$ with respect to $(DI_bI_c)$.
Hence $E'$ lies on the radical axis of $(DII_a)$ and $(DI_bI_c)$, ergo $E'=E$.
The first claim then solves the problem.
\pagebreak


\end{document}
