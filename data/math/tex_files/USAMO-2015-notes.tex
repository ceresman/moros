% © Evan Chen
% Downloaded from https://web.evanchen.cc/

\documentclass[11pt]{scrartcl}
\usepackage[sexy]{evan}
\ihead{\footnotesize\textbf{\thetitle}}
\ohead{\footnotesize\href{http://web.evanchen.cc}{\ttfamily web.evanchen.cc},
    updated \today}
\title{USAMO 2015 Solution Notes}
\date{\today}

\begin{document}

\maketitle

\begin{abstract}
This is a compilation of solutions
for the 2015 USAMO.
Some of the solutions are my own work,
but many are from the official solutions provided by the organizers
(for which they hold any copyrights),
and others were found by users on the Art of Problem Solving forums.

These notes will tend to be a bit more advanced and terse than the ``official''
solutions from the organizers.
In particular, if a theorem or technique is not known to beginners
but is still considered ``standard'', then I often prefer to
use this theory anyways, rather than try to work around or conceal it.
For example, in geometry problems I typically use directed angles
without further comment, rather than awkwardly work around configuration issues.
Similarly, sentences like ``let $\mathbb{R}$ denote the set of real numbers''
are typically omitted entirely.

Corrections and comments are welcome!
\end{abstract}

\tableofcontents
\newpage

\addtocounter{section}{-1}
\section{Problems}
\begin{enumerate}[\bfseries 1.]
\item %% Problem 1
Solve in integers the equation
\[ x^2+xy+y^2 = \left(\frac{x+y}{3}+1\right)^3. \]

\item %% Problem 2
Quadrilateral $APBQ$ is inscribed in circle $\omega$ with
$\angle P = \angle Q = 90^{\circ}$ and $AP = AQ < BP$.
Let $X$ be a variable point on segment $\ol{PQ}$.
Line $AX$ meets $\omega$ again at $S$ (other than $A$).
Point $T$ lies on arc $AQB$ of $\omega$ such that $\ol{XT}$
is perpendicular to $\ol{AX}$.
Let $M$ denote the midpoint of chord $\ol{ST}$.

As $X$ varies on segment $\ol{PQ}$, show that $M$ moves along a circle.

\item %% Problem 3
Let $S = \left\{ 1,2,\dots,n \right\}$, where $n \ge 1$.
Each of the $2^n$ subsets of $S$ is to be colored red or blue.
(The subset itself is assigned a color and not its individual elements.)
For any set $T \subseteq S$,
we then write $f(T)$ for the number of subsets of $T$ that are blue.

Determine the number of colorings that satisfy the following condition:
for any subsets $T_1$ and $T_2$ of $S$,
\[ f(T_1)f(T_2) = f(T_1 \cup T_2)f(T_1 \cap T_2). \]

\item %% Problem 4
Steve is piling $m\geq 1$ indistinguishable stones
on the squares of an $n\times n$ grid.
Each square can have an arbitrarily high pile of stones.
After he finished piling his stones in some manner,
he can then perform \emph{stone moves}, defined as follows.
Consider any four grid squares, which are corners of a rectangle,
i.e.\ in positions $(i, k)$, $(i, l)$, $(j, k)$, $(j, l)$
for some $1\leq i, j, k, l\leq n$, such that $i<j$ and $k<l$.
A stone move consists of either removing one stone from each of
$(i, k)$ and $(j, l)$ and moving them to $(i, l)$ and $(j, k)$ respectively,
or removing one stone from each of $(i, l)$ and $(j, k)$
and moving them to $(i, k)$ and $(j, l)$ respectively.

Two ways of piling the stones are equivalent if they can be obtained
from one another by a sequence of stone moves.
How many different non-equivalent ways can Steve pile the stones on the grid?

\item %% Problem 5
Let $a$, $b$, $c$, $d$, $e$ be distinct positive integers
such that $a^4+b^4=c^4+d^4=e^5$.
Show that $ac+bd$ is a composite number.

\item %% Problem 6
Consider $0<\lambda<1$, and let $A$ be a multiset of positive integers.
Let $A_n=\{a\in A: a\leq n\}$.
Assume that for every $n\in\NN$,
the multiset $A_n$ contains at most $n\lambda$ numbers.
Show that there are infinitely many $n\in\NN$ for
which the sum of the elements in $A_n$ is
at most $\frac{n(n+1)}{2}\lambda$.

\end{enumerate}
\pagebreak

\section{Solutions to Day 1}
\subsection{USAMO 2015/1, proposed by Titu Andreescu}
\textsl{Available online at \url{https://aops.com/community/p4769940}.}
\begin{mdframed}[style=mdpurplebox,frametitle={Problem statement}]
Solve in integers the equation
\[ x^2+xy+y^2 = \left(\frac{x+y}{3}+1\right)^3. \]
\end{mdframed}
We do the trick of setting $a=x+y$ and $b=x-y$.
This rewrites the equation as
\[ \frac14\left( (a+b)^2+(a+b)(a-b)+(a-b)^2 \right)
  = \left( \frac a3 + 1 \right)^3 \]
where $a,b \in \ZZ$ have the same parity.
This becomes
\[ 3a^2+b^2 = 4\left( \frac a3 + 1 \right)^3 \]
which is enough to imply $3 \mid a$, so let $a = 3c$.
Miraculously, this becomes \[ b^2 = (c-2)^2 (4c+1). \]

So a solution must have $4c+1=m^2$, with $m$ odd.
This gives
\[ x = \frac 18 \left( 3 (m^2-1) \pm (m^3-9m)  \right)
  \quad\text{and}\quad
  y = \frac 18 \left( 3 (m^2-1) \mp (m^3-9m) \right).  \]
For mod $8$ reasons,
this always generates a valid integer solution,
so this is the complete curve of solutions.
Actually, putting $m=2n+1$ gives the much nicer curve
\[
 \boxed{x = n^3+3n^2-1
 \quad\text{and}\quad
 y = -n^3+3n+1}
\] and permutations.

For $n=0,1,2,3$ this gives the first few solutions
are $(-1,1)$, $(3,3)$, $(19,-1)$, $(53, -17)$, (and permutations).
% http://aops.com/community/c5h1083093p4769984
\pagebreak

\subsection{USAMO 2015/2, proposed by Zuming Feng, Jacek Fabrykowski}
\textsl{Available online at \url{https://aops.com/community/p4769957}.}
\begin{mdframed}[style=mdpurplebox,frametitle={Problem statement}]
Quadrilateral $APBQ$ is inscribed in circle $\omega$ with
$\angle P = \angle Q = 90^{\circ}$ and $AP = AQ < BP$.
Let $X$ be a variable point on segment $\ol{PQ}$.
Line $AX$ meets $\omega$ again at $S$ (other than $A$).
Point $T$ lies on arc $AQB$ of $\omega$ such that $\ol{XT}$
is perpendicular to $\ol{AX}$.
Let $M$ denote the midpoint of chord $\ol{ST}$.

As $X$ varies on segment $\ol{PQ}$, show that $M$ moves along a circle.
\end{mdframed}
We present three solutions,
one by complex numbers, two more synthetic.
(A fourth solution using median formulas is also possible.)
Most solutions will prove that the center of the fixed circle
is the midpoint of $\ol{AO}$
(with $O$ the center of $\omega$);
this can be recovered empirically by letting
\begin{itemize}
  \ii $X$ approach $P$ (giving the midpoint of $\ol{BP}$)
  \ii $X$ approach $Q$ (giving the point $Q$), and
  \ii $X$ at the midpoint of $\ol{PQ}$ (giving the midpoint of $\ol{BQ}$)
\end{itemize}
which determines the circle;
this circle then passes through $P$ by symmetry
and we can find the center by taking the intersection
of two perpendicular bisectors (which two?).

\paragraph{Complex solution (Evan Chen).}
Toss on the complex unit circle with $a = -1$, $b=1$, $z = -\tfrac12$.
Let $s$ and $t$ be on the unit circle. We claim $Z$ is the center.

It follows from standard formulas that
\[ x = \frac 12 \left( s + t - 1 + s/t \right) \]
thus \[ 4\operatorname{Re} x + 2
  = s + t + \frac 1s + \frac 1t + \frac st + \frac ts \]
which depends only on $P$ and $Q$, and not on $X$.
Thus \[ 4\left\lvert z-\frac{s+t}{2} \right\rvert^2
  = \left\lvert s+t+1 \right\rvert^2 = 3 + (4\operatorname{Re}x+2) \]
does not depend on $X$, done.

\paragraph{Homothety solution (Alex Whatley).}
Let $G$, $N$, $O$ denote the centroid, nine-point center,
and circumcenter of triangle $AST$, respectively.
Let $Y$ denote the midpoint of $\ol{AS}$.
Then the three points $X$, $Y$, $M$
lie on the nine-point circle of triangle $AST$,
which is centered at $N$ and has radius $\frac 12 AO$.

\begin{center}
\begin{asy}
size(9cm);

pair A = dir(90);
pair B = dir(-90);
pair S = dir(-50);
pair T = dir(170);
pair O = midpoint(A--B);
pair X = foot(T, A, S);
pair E = dir(0);

pair P = IP(unitcircle, X--(X-2*E));
pair Q = IP(unitcircle, X--(X+2*E));

filldraw(unitcircle, opacity(0.2)+mediumcyan, mediumblue);

pair M = midpoint(S--T);
pair G = centroid(A, S, T);
pair N = 3/2*G;

filldraw(A--P--B--Q--cycle, opacity(0.1)+lightblue, mediumblue);
draw(A--B, mediumblue+dotted);
draw(P--Q, mediumblue+dotted);

filldraw(A--S--T--cycle, opacity(0.4)+mediumgreen, heavygreen);
filldraw(CP(N, M), opacity(0.6)+lightred, red);

draw(T--X, heavygreen);
draw(A--M, heavygreen);
pair Y = midpoint(A--S);

draw(O--N, red);

dot("$A$", A, dir(A));
dot("$B$", B, dir(B));
dot("$S$", S, dir(S));
dot("$T$", T, dir(T));
dot("$O$", O, dir(-45));
dot("$X$", X, dir(45));
dot("$P$", P, dir(P));
dot("$Q$", Q, dir(Q));
dot("$M$", M, dir(M));
dot("$G$", G, dir(30));
dot("$N$", N, dir(N));
dot("$Y$", Y, dir(Y));

/* Source generated by TSQ

!size(9cm);

A = dir 90
B = dir -90
S = dir -50
T = dir 170
O = midpoint A--B R-45
X = foot T A S R45
E := dir 0

P = IP unitcircle X--(X-2*E)
Q = IP unitcircle X--(X+2*E)

unitcircle 0.2 mediumcyan / mediumblue

M = midpoint S--T
G = centroid A S T R30
N = 3/2*G

A--P--B--Q--cycle 0.1 lightblue / mediumblue
A--B mediumblue dotted
P--Q mediumblue dotted

A--S--T--cycle 0.4 mediumgreen / heavygreen
CP N M 0.6 lightred / red

T--X heavygreen
A--M heavygreen
Y = midpoint A--S

O--N red

*/
\end{asy}
\end{center}

Let $R$ denote the radius of $\omega$.
Note that the nine-point circle of $\triangle AST$
has radius equal to $\half R$, and hence is independent of $S$ and $T$.
Then the power of $A$ with respect to the nine-point circle equals
\[ AN^2 - \left( \half R \right)^2 = AX \cdot AY
  = \frac 12 AX \cdot AS = \frac 12 AQ^2 \]
and hence
\[ AN^2 = \left( \half R \right)^2 + \frac 12 AQ^2 \]
which does not depend on the choice of $X$.
So $N$ moves along a circle centered at $A$.

Since the points $O$, $G$, $N$ are collinear on the Euler line of $\triangle AST$
with \[ GO = \frac 23 NO \] it follows by homothety
that $G$ moves along a circle as well,
whose center is situated one-third of the way from $A$ to $O$.
Finally, since $A$, $G$, $M$ are collinear with \[ AM = \frac 32 AG \]
it follows that $M$ moves along a circle
centered at the midpoint of $\ol{AO}$.

\paragraph{Power of a point solution (Zuming Feng, official solution).}
We complete the picture by letting $\triangle KYX$ be the orthic
triangle of $\triangle AST$;
in that case line $XY$ meets the $\omega$ again at $P$ and $Q$.

\begin{center}
\begin{asy}
size(9cm);

pair A = dir(125);
pair B = -A;
pair S = dir(210);
pair T = dir(330);
pair O = midpoint(A--B);
pair X = foot(T, A, S);
pair E = dir(0);

filldraw(unitcircle, opacity(0.2)+mediumcyan, mediumblue);
pair M = midpoint(S--T);

filldraw(A--S--T--cycle, opacity(0.4)+mediumgreen, heavygreen);

draw(T--X, heavygreen);
draw(A--M, heavygreen);

pair Y = foot(S, A, T);
pair K = foot(A, S, T);
filldraw(circumcircle(X, Y, M), opacity(0.1)+yellow, red);
draw(S--Y, heavygreen);
draw(A--K, heavygreen);

pair P = IP(unitcircle, X--(3*Y-2*X));
pair Q = IP(unitcircle, Y--(3*X-2*Y));

pair V = extension(P, Q, S, T);
draw(P--Q, blue);
draw(A--B, blue);
draw(Q--V, blue);
draw(V--S, heavygreen);

dot("$A$", A, dir(A));
dot("$B$", B, dir(B));
dot("$S$", S, dir(S));
dot("$T$", T, dir(T));
dot("$O$", O, dir(45));
dot("$X$", X, dir(135));
dot("$M$", M, dir(M));
dot("$Y$", Y, dir(70));
dot("$K$", K, dir(K));
dot("$P$", P, dir(P));
dot("$Q$", Q, dir(150));
dot("$V$", V, dir(V));

/* TSQ Source:

!size(9cm);

A = dir 125
B = -A
S = dir 210
T = dir 330
O = midpoint A--B R45
X = foot T A S R135
E := dir 0

unitcircle 0.2 mediumcyan / mediumblue
M = midpoint S--T

A--S--T--cycle 0.4 mediumgreen / heavygreen

T--X heavygreen
A--M heavygreen

Y = foot S A T R70
K = foot A S T
circumcircle X Y M 0.1 yellow / red
S--Y heavygreen
A--K heavygreen

P = IP unitcircle X--(3*Y-2*X)
Q = IP unitcircle Y--(3*X-2*Y) R150

V = extension P Q S T
P--Q blue
A--B blue
Q--V blue
V--S heavygreen

*/
\end{asy}
\end{center}

The main claim is:
\begin{claim*}
  Quadrilateral $PQKM$ is cyclic.
\end{claim*}
\begin{proof}
  To see this, we use power of a point: let $V = \ol{QXYP} \cap \ol{SKMT}$.
  One approach is that since $(VK;ST) = -1$
  we have $VQ \cdot VP = VS \cdot VT = VK \cdot VM$.
  A longer approach is more elementary:
  \[ VQ \cdot VP = VS \cdot VT = VX \cdot VY = VK \cdot VM \]
  using the nine-point circle, and the circle with diameter $\ol{ST}$.
\end{proof}
But the circumcenter of $PQKM$,
is the midpoint of $\ol{AO}$,
since it lies on the perpendicular bisectors of $\ol{KM}$ and $\ol{PQ}$.
So it is fixed, the end.
\pagebreak

\subsection{USAMO 2015/3, proposed by Gabriel Carroll}
\textsl{Available online at \url{https://aops.com/community/p4769949}.}
\begin{mdframed}[style=mdpurplebox,frametitle={Problem statement}]
Let $S = \left\{ 1,2,\dots,n \right\}$, where $n \ge 1$.
Each of the $2^n$ subsets of $S$ is to be colored red or blue.
(The subset itself is assigned a color and not its individual elements.)
For any set $T \subseteq S$,
we then write $f(T)$ for the number of subsets of $T$ that are blue.

Determine the number of colorings that satisfy the following condition:
for any subsets $T_1$ and $T_2$ of $S$,
\[ f(T_1)f(T_2) = f(T_1 \cup T_2)f(T_1 \cap T_2). \]
\end{mdframed}
For an $n$-coloring $\mathcal C$ (by which we mean a coloring of
the subsets of $\{1,\dots,n\}$),
define the \textbf{support} of $\mathcal C$ as
\[ \opname{supp}(\mathcal C) =
  \left\{ T \mid f(T) \neq 0 \right\}. \]
Call a coloring \textbf{nontrivial}
if $\opname{supp}(\mathcal C) \neq \varnothing$ (equivalently, the coloring is not all red).
In that case, notice that
\begin{itemize}
  \ii the support is
  \emph{closed under unions and intersections}:
  since if $f(T_1) f(T_2) \neq 0$ then necessarily
  both $f(T_1 \cap T_2)$ and $f(T_1 \cup T_2)$ are nonzero; and
  \ii the support is obviously \emph{upwards closed}.
\end{itemize}
Thus, the support must take the form
\[ \opname{supp}(\mathcal C)
  = [X,S]
  \overset{\text{def}}{=} \left\{ T \mid X \subseteq T \subseteq S \right\} \]
for some set $X$
(for example by letting $X$ be the minimal (by size) element of the support).

Say $\mathcal C$ has \textbf{full support}
if $X = \varnothing$ (equivalently, $\varnothing$ is blue).

\begin{lemma*}
For a given $n$ and $B \subseteq \{1,\dots,n\}$,
there is exactly one $n$-coloring with full support
in which the singletons colored blue are exactly those in $B$.
Therefore there are exactly $2^n$ $n$-colorings with full support.
\end{lemma*}
\begin{proof}
To see there is at least one coloring, color only the subsets of $B$ blue.
In that case \[ f(T) = 2^{\left\lvert T \cap B \right\rvert} \]
which satisfies the condition by Inclusion-Exclusion.
To see this extension is unique,
note that $f(\{b\})$ is determined for each $b$ and we can then determine $f(T)$
inductively on $\left\lvert T \right\rvert$;
hence there is \emph{at most} one way to complete a coloring of the singletons,
which completes the proof.
\end{proof}

For a general nontrivial $n$-coloring $\mathcal C$,
note that if $\opname{supp}(\mathcal C) = [X,S]$,
then we can think of it as an
$(n-\left\lvert X \right\rvert)$-coloring with full support.
For $\left\lvert X \right\rvert = k$, there are $\binom nk$ possible choices of $X \subseteq S$.
Adding back in the trivial coloring, the final answer is
\[ 1 + \sum_{k=0}^n \binom nk 2^k = \boxed{1 + 3^n}. \]

\begin{remark*}
  To be more explicit,
  the possible nontrivial colorings are exactly described
  by specifying two sets $X$ and $Y$ with $X \subseteq Y$,
  and coloring blue only the sets $T$
  with $X \subseteq T \subseteq Y$.

  In particular, one deduces that in a working coloring,
  $f(T)$ is always either zero or a power of two.
  If one manages to notice this while working on the problem,
  it is quite helpful for motivating the solution,
  as it leads one to suspect that the working colorings
  have good structure.
\end{remark*}
\pagebreak

\section{Solutions to Day 2}
\subsection{USAMO 2015/4, proposed by Maria Monks Gillespie}
\textsl{Available online at \url{https://aops.com/community/p4774079}.}
\begin{mdframed}[style=mdpurplebox,frametitle={Problem statement}]
Steve is piling $m\geq 1$ indistinguishable stones
on the squares of an $n\times n$ grid.
Each square can have an arbitrarily high pile of stones.
After he finished piling his stones in some manner,
he can then perform \emph{stone moves}, defined as follows.
Consider any four grid squares, which are corners of a rectangle,
i.e.\ in positions $(i, k)$, $(i, l)$, $(j, k)$, $(j, l)$
for some $1\leq i, j, k, l\leq n$, such that $i<j$ and $k<l$.
A stone move consists of either removing one stone from each of
$(i, k)$ and $(j, l)$ and moving them to $(i, l)$ and $(j, k)$ respectively,
or removing one stone from each of $(i, l)$ and $(j, k)$
and moving them to $(i, k)$ and $(j, l)$ respectively.

Two ways of piling the stones are equivalent if they can be obtained
from one another by a sequence of stone moves.
How many different non-equivalent ways can Steve pile the stones on the grid?
\end{mdframed}
The answer is $\binom{m+n-1}{n-1}^2$.
The main observation is that the ordered sequence of column counts
(i.e.\ the number of stones in the first, second, etc.\ column)
is invariant under stone moves, as does the analogous sequence of row counts.

\paragraph{Definitions.}
Call these numbers $(c_1, c_2, \dots, c_n)$
and $(r_1, r_2, \dots, r_n)$ respectively,
with $\sum c_i = \sum r_i = m$.
We say that the sequence $(c_1, \dots, c_n, r_1, \dots, r_n)$
is the \emph{signature} of the configuration.
These are the $2m$ blue and red numbers shown in the example below
(in this example we have $m=8$ and $n=3$).
\begin{center}
\begin{asy}
  unitsize(1.5cm);
  for (int i=0; i<=3; ++i) {
    draw( (0,i)--(3,i), grey );
    draw( (i,0)--(i,3), grey );
  }
  label("$c_1=\boxed{5}$", (0.5,3), dir(90), blue);
  label("$c_2=\boxed{2}$", (1.5,3), dir(90), blue);
  label("$c_3=\boxed{1}$", (2.5,3), dir(90), blue);

  label("$r_1=\boxed{3}$", (0,2.5), dir(180), red);
  label("$r_2=\boxed{3}$", (0,1.5), dir(180), red);
  label("$r_3=\boxed{2}$", (0,0.5), dir(180), red);

  real r = 0.1;
  filldraw(circle((0.3,2.7), r), grey, black);
  filldraw(circle((0.6,2.4), r), grey, black);
  filldraw(circle((1.4,2.3), r), grey, black);
  filldraw(circle((0.6,1.7), r), grey, black);
  filldraw(circle((0.4,1.4), r), grey, black);
  filldraw(circle((0.7,0.5), r), grey, black);
  filldraw(circle((2.4,1.6), r), grey, black);
  filldraw(circle((1.6,0.4), r), grey, black);
  label("Signature: $(5,2,1;3,3,2)$", (1.5,0), dir(-90));
\end{asy}
\end{center}
By stars-and-bars, the number of possible values $(c_1, \dots, c_n)$
is $\binom{m+n-1}{n-1}$.
The same is true for $(r_1, \dots, r_m)$.
So if we're just counting \emph{signatures},
the total number of possible signatures is $\binom{m+n-1}{n-1}^2$.

\paragraph{Outline and setup.}
We are far from done.
To show that the number of non-equivalent ways is also this number,
we need to show that signatures correspond to pilings.
In other words, we need to prove:
\begin{enumerate}
  \ii Check that signatures are invariant around moves
  (trivial; we did this already);
  \ii Check conversely that two configurations
  are equivalent if they have the same signatures
  (the hard part of the problem); and
  \ii Show that each signature is realized by at least
  one configuration (not immediate, but pretty easy).
\end{enumerate}

Most procedures to the second step are algorithmic in nature,
but Ankan Bhattacharya gives the following far cleaner approach.
Rather than having a grid of stones,
we simply consider the multiset of ordered pairs $(x,y)$
corresponding to the stones.
Then:
\begin{itemize}
  \ii a stone move corresponds to switching two $y$-coordinates
  in two different pairs.
  \ii we \emph{redefine} the signature to be the multiset $(X,Y)$
  of $x$ and $y$ coordinates which appear.
  Explicitly, $X$ is the multiset that contains $c_i$ copies
  of the number $i$ for each $i$.
\end{itemize}
For example, consider the earlier example which had
\begin{itemize}
  \ii Two stones each at $(1,1)$, $(1,2)$.
  \ii One stone each at $(1,3)$, $(2,1)$, $(2,3)$, $(3,2)$.
\end{itemize}
Its signature can then be reinterpreted as
\[
  (5,2,1; 3,3,2)
  \longleftrightarrow
  \begin{cases}
    X = \{1,1,1,1,1,2,2,3\} \\
    Y = \{1,1,1,2,2,2,3,3\}.
  \end{cases}
\]
In that sense, the entire grid is quite misleading!

\paragraph{Proof that two configurations with the same signature are equivalent.}
The second part is completed just because
transpositions generate any permutation.
To be explicit, given two sets of stones,
we can permute the labels so that the first set is
$(x_1, y_1)$, \dots, $(x_m, y_m)$ and
the second set of stones is
$(x_1, y_1')$, \dots, $(x_m, y_m')$.
Then we just induce the correct permutation on $(y_i)$ to get $(y_i')$.

\paragraph{Proof that any signature has at least one configuration.}
Sort the elements of $X$ and $Y$ arbitrarily (say, in non-decreasing order).
Put a stone whose $x$-coordinate is the $i$th element of $X$,
and whose $y$-coordinate is the $i$th element of $Y$,
for each $i = 1, 2, \dots, m$.
Then this gives a stone placement of $m$ stones with signature $(X,Y)$.

For example, if
\begin{align*}
  X &= \{1,1,1,1,1,2,2,3\} \\
  Y &= \{1,1,1,2,2,2,3,3\}
\end{align*}
then placing stones at
$(1,1)$, $(1,1)$, $(1,1)$, $(1,2)$, $(1,2)$, $(2,2)$, $(2,3)$, $(3,3)$
gives a valid piling with this signature.
\pagebreak

\subsection{USAMO 2015/5, proposed by Mohsen Jamaali}
\textsl{Available online at \url{https://aops.com/community/p4774020}.}
\begin{mdframed}[style=mdpurplebox,frametitle={Problem statement}]
Let $a$, $b$, $c$, $d$, $e$ be distinct positive integers
such that $a^4+b^4=c^4+d^4=e^5$.
Show that $ac+bd$ is a composite number.
\end{mdframed}
Assume to the contrary that $p = ac+bd$, so that
\begin{align*}
  ac &\equiv -bd \pmod p \\
  \implies a^4c^4 &\equiv b^4d^4 \pmod p \\
  \implies a^4 (e^5 - d^4) &\equiv (e^5 - a^4) d^4 \pmod p \\
  \implies a^4 e^5 &\equiv d^4 e^5 \pmod p \\
  \implies e^5(a^4-d^4) &\equiv 0 \pmod p
\end{align*}
and hence \[ p \mid e^5(a-d)(a+d)(a^2+d^2). \]

\begin{claim*}
  We should have $p > e$.
\end{claim*}
\begin{proof}
  We have $e^5 = a^4 + b^4 \le a^5 + b^5 < (ac+bd)^5 = p^5$.
\end{proof}
Thus the above equation implies $p \le \max(a-d, a+d, a^2+d^2) = a^2+d^2$.
Similarly, $p \le b^2+c^2$.
So \[ ac+bd = p \le \min \left\{ a^2+d^2, b^2+c^2 \right\} \]
or by subtraction
\[ 0 \le \min \left\{ a(a-c) + d(d-b),
  b(b-d) + c(c-a) \right\}. \]
But since $a^4+b^4 = c^4+d^4$ the numbers $a-c$ and $d-b$
should have the same sign, and so this is an obvious contradiction.
\pagebreak

\subsection{USAMO 2015/6, proposed by Iurie Boreico}
\textsl{Available online at \url{https://aops.com/community/p4774023}.}
\begin{mdframed}[style=mdpurplebox,frametitle={Problem statement}]
Consider $0<\lambda<1$, and let $A$ be a multiset of positive integers.
Let $A_n=\{a\in A: a\leq n\}$.
Assume that for every $n\in\NN$,
the multiset $A_n$ contains at most $n\lambda$ numbers.
Show that there are infinitely many $n\in\NN$ for
which the sum of the elements in $A_n$ is
at most $\frac{n(n+1)}{2}\lambda$.
\end{mdframed}
For brevity, $\#S$ denotes $|S|$.
Let $x_n = n\lambda - \#A_n \ge 0$.
We now proceed by contradiction by assuming
the conclusion fails for $n$ large enough; that is,
\begin{align*}
  \frac{n(n+1)}{2}\lambda
  &< \sum_{a \in A_n} a \\
  &= 1(\#A_1-\#A_0)
  + 2(\#A_2 - \#A_1)
  + \dots + n(\#A_n - \#A_{n-1}) \\
  &= n \# A_n - (\# A_1 + \dots + \# A_{n-1}) \\
  &= n(n \lambda - x_n) - \left[ (\lambda - x_1)
    + (2\lambda - x_2) + \dots + ((n-1)\lambda - x_{n-1}) \right] \\
  &= \frac{n(n+1)}{2} \lambda - n x_n
  + (x_1 + \dots + x_{n-1}).
\end{align*}
This means that for all sufficiently large $n$, say $n \ge N_0$, we have
\[ x_{n} < \frac{x_1 + \dots + x_{n-1}}{n}
  \qquad \forall n \ge N_0. \]
In particular, each $x_n$ is the less
than the average of all preceding terms.
Intuitively this means $x_n$ should become close to each other,
since they are also nonnegative.

However, we have a second condition we haven't used yet:
the ``integer'' condition implies
\[
  \left\lvert x_{n+1} - x_n \right\rvert
  = \left\lvert \lambda - \#\{n \in A \} \right\rvert
  > \eps
\]
for some fixed $\eps > 0$,
namely $\eps = \min \left\{ \lambda, 1 - \lambda \right\}$.
Using the fact that consecutive terms differ by some fixed $\eps$,
we will derive a contradiction.

If we let $M$ be the average of $x_1$, \dots, $x_{N_0}$,
then we ought to have
\[ x_n < M \qquad \forall n > N_0. \]
Hence for $n > N_0$ we have $x_n + x_{n+1} < 2M - \eps$,
and so for large enough $n$ the average
must drop to just above $M - \half \eps$.
Thus for some large $N_1 > N_0$, we will have
\[ x_n < M - \frac13 \eps \qquad \forall n > N_1. \]
If we repeat this argument then with a large $N_2 > N_1$, we obtain
\[ x_n < M - \frac23 \eps \qquad \forall n > N_2 \]
and so on and so forth.
This is a clear contradiction.

\begin{remark*}
Note that if $A = \{2,2,3,4,5,\dots\}$ and $\lambda = 1$ then contradiction.
So the condition that $0 < \lambda < 1$ cannot be dropped,
and (by scaling) neither can the condition that $A \subseteq \ZZ$.
\end{remark*}

\begin{remark*}
  [Suggested by Zhao Ting-wei]
  Despite the relation
  \[ x_{n} < \frac{x_1 + \dots + x_{n-1}}{n}
    \qquad \forall n \ge N_0 \]
  implying that $x_n$ is bounded,
  it does not alone imply that $x_n$ converges,
  not even to some nonzero value.
  Zhao Ting-Wei showed me that one can have a sequence which is zero
  ``every so often'' yet where the average is nonzero.

  A counterexample is given explicitly by
  \[
    x_n
    = \begin{cases}
      1000 & n = 1 \\
      0 & n \text{ is a power of $10$} \\
      1 + \frac 1n & \text{otherwise}
    \end{cases}
  \]
  which does not have a limit.
  For completeness, let's check this ---
  let $H_n$ denote the $n$'th harmonic number, and compute
  \begin{align*}
    \sum_1^{n-1} x_n
    &= 1000 + (n-1) + H_{n-1} - \sum_{k=1}^{\left\lfloor \log_{10} n \right\rfloor} \left( 1 + \frac{1}{10^k} \right) \\
    &> n + 999 + H_{n-1} - \log_{10} n - \left( 1 + \frac{1}{10} + \dots \right) \\
    &> n + 997 + H_{n-1} - \log_{10} n > n + 1
  \end{align*}
  so $1 + \frac 1n < \frac 1n \sum_1^{n-1} x_n$ as needed.
\end{remark*}
\pagebreak


\end{document}
