% © Evan Chen
% Downloaded from https://web.evanchen.cc/

\documentclass[11pt]{scrartcl}
\usepackage[sexy]{evan}
\ihead{\footnotesize\textbf{\thetitle}}
\ohead{\footnotesize\href{http://web.evanchen.cc}{\ttfamily web.evanchen.cc},
    updated \today}
\title{USAMO 1998 Solution Notes}
\date{\today}

\begin{document}

\maketitle

\begin{abstract}
This is a compilation of solutions
for the 1998 USAMO.
Some of the solutions are my own work,
but many are from the official solutions provided by the organizers
(for which they hold any copyrights),
and others were found by users on the Art of Problem Solving forums.

These notes will tend to be a bit more advanced and terse than the ``official''
solutions from the organizers.
In particular, if a theorem or technique is not known to beginners
but is still considered ``standard'', then I often prefer to
use this theory anyways, rather than try to work around or conceal it.
For example, in geometry problems I typically use directed angles
without further comment, rather than awkwardly work around configuration issues.
Similarly, sentences like ``let $\mathbb{R}$ denote the set of real numbers''
are typically omitted entirely.

Corrections and comments are welcome!
\end{abstract}

\tableofcontents
\newpage

\addtocounter{section}{-1}
\section{Problems}
\begin{enumerate}[\bfseries 1.]
\item %% Problem 1
Suppose that the set $\{1,2,\dots, 1998\}$
has been partitioned into disjoint pairs $\{a_i,b_i\}$ ($1\leq i\leq 999$)
so that for all $i$, $|a_i-b_i|$ equals $1$ or $6$.
Prove that the sum
\[ |a_1-b_1|+|a_2-b_2|+\dotsb +|a_{999}-b_{999}|  \]
ends in the digit $9$.

\item %% Problem 2
Let $\mathcal C_1$ and $\mathcal C_2$ be concentric circles, with
$\mathcal C_2$ in the interior of $\mathcal C_1$.
From a point $A$ on $\mathcal C_1$ one draws the tangent $AB$
to $\mathcal C_2$ ($B\in \mathcal C_2$).
Let $C$ be the second point of intersection of ray $AB$ and
$\mathcal C_1$, and let $D$ be the midpoint of $\ol{AB}$.
A line passing through $A$ intersects $\mathcal C_2$ at $E$ and $F$
in such a way that the perpendicular bisectors of $\ol{DE}$ and $\ol{CF}$
intersect at a point $M$ on line $AB$.
Find, with proof, the ratio $AM/MC$.

\item %% Problem 3
Let $a_0,a_1,\dots ,a_n$ be numbers from the interval $(0,\pi/2)$
such that $\tan (a_0-\frac{\pi}{4})+ \tan (a_1-\frac{\pi}{4})
 + \dotsb +\tan (a_n-\frac{\pi}{4}) \ge  n-1$.
Prove that
\[ \tan a_0\tan a_1 \dotsm \tan a_n \ge n^{n+1}. \]

\item %% Problem 4
A computer screen shows a $98 \times 98$ chessboard, colored in the usual way.
One can select with a mouse any rectangle with sides on the lines
of the chessboard and click the mouse button:
as a result, the colors in the selected rectangle switch
(black becomes white, white becomes black).
Find, with proof, the minimum number of mouse clicks
needed to make the chessboard all one color.

\item %% Problem 5
Prove that for each $n\geq 2$,
there is a set $S$ of $n$ integers
such that $(a-b)^2$ divides $ab$ for every distinct $a,b\in S$.

\item %% Problem 6
Let $n \geq 5$ be an integer.
Find the largest integer $k$ (as a function of $n$)
such that there exists a convex $n$-gon $A_{1}A_{2}\dots A_{n}$
for which exactly $k$ of the quadrilaterals
$A_{i}A_{i+1}A_{i+2}A_{i+3}$ have an inscribed circle,
where indices are taken modulo $n$.

\end{enumerate}
\pagebreak

\section{Solutions to Day 1}
\subsection{USAMO 1998/1}
\textsl{Available online at \url{https://aops.com/community/p343865}.}
\begin{mdframed}[style=mdpurplebox,frametitle={Problem statement}]
Suppose that the set $\{1,2,\dots, 1998\}$
has been partitioned into disjoint pairs $\{a_i,b_i\}$ ($1\leq i\leq 999$)
so that for all $i$, $|a_i-b_i|$ equals $1$ or $6$.
Prove that the sum
\[ |a_1-b_1|+|a_2-b_2|+\dotsb +|a_{999}-b_{999}|  \]
ends in the digit $9$.
\end{mdframed}
Let $S$ be the sum.
Modulo $2$,
\[ S = \sum |a_i-b_i| \equiv \sum (a_i+b_i)
  = 1 + 2 + \dots + 1998 \equiv 1 \pmod 2. \]
Modulo $5$,
\[ S = \sum |a_i-b_i| = 1 \cdot 999 \equiv 4 \pmod 5. \]
So $S \equiv 9 \pmod{10}$.
\pagebreak

\subsection{USAMO 1998/2}
\textsl{Available online at \url{https://aops.com/community/p343866}.}
\begin{mdframed}[style=mdpurplebox,frametitle={Problem statement}]
Let $\mathcal C_1$ and $\mathcal C_2$ be concentric circles, with
$\mathcal C_2$ in the interior of $\mathcal C_1$.
From a point $A$ on $\mathcal C_1$ one draws the tangent $AB$
to $\mathcal C_2$ ($B\in \mathcal C_2$).
Let $C$ be the second point of intersection of ray $AB$ and
$\mathcal C_1$, and let $D$ be the midpoint of $\ol{AB}$.
A line passing through $A$ intersects $\mathcal C_2$ at $E$ and $F$
in such a way that the perpendicular bisectors of $\ol{DE}$ and $\ol{CF}$
intersect at a point $M$ on line $AB$.
Find, with proof, the ratio $AM/MC$.
\end{mdframed}
By power of a point we have
\[ AE \cdot AF = AB^2
  = \left( \half AB \right) \cdot \left( 2AB \right)
  = AD \cdot AC \]
and hence $CDEF$ is cyclic.
Then $M$ is the circumcenter of quadrilateral $CDEF$.

\begin{center}
\begin{asy}
pair A = dir(170);
pair C = dir(40);
pair B = midpoint(A--C);
pair D = midpoint(A--B);
pair M = midpoint(C--D);
pair O = origin;

filldraw(unitcircle, opacity(0.1)+lightcyan, heavycyan);
draw(CP(O, B), heavycyan);
pair E = IP(CP(M, D), CP(O, B));
pair F = OP(CP(M, D), CP(O, B));
filldraw(CP(M, D), opacity(0.1)+lightred, red);
draw(A--C, heavygreen);
draw(A--F, heavygreen);
draw(D--E, red);
draw(C--F, red);
pair U = midpoint(D--E);
pair V = midpoint(C--F);
draw(U--M--V, red+dotted);

dot("$A$", A, dir(A));
dot("$C$", C, dir(C));
dot("$B$", B, dir(B));
dot("$D$", D, dir(D));
dot("$M$", M, dir(M));
dot("$O$", O, dir(45));
dot("$E$", E, dir(220));
dot("$F$", F, dir(F));

/* TSQ Source:

A = dir 170
C = dir 40
B = midpoint A--C
D = midpoint A--B
M = midpoint C--D
O = origin R45

unitcircle 0.1 lightcyan / heavycyan
CP O B heavycyan
E = IP CP M D CP O B R220
F = OP CP M D CP O B
CP M D 0.1 lightred / red
A--C heavygreen
A--F heavygreen
D--E red
C--F red
U := midpoint D--E
V := midpoint C--F
U--M--V red dotted

*/
\end{asy}
\end{center}

Thus $M$ is the midpoint of $\ol{CD}$
(and we are given already that $B$ is the midpoint of $\ol{AC}$,
$D$ is the midpoint of $\ol{AB}$).
Thus a quick computation along $\ol{AC}$ gives $AM/MC = 5/3$.
\pagebreak

\subsection{USAMO 1998/3}
\textsl{Available online at \url{https://aops.com/community/p343867}.}
\begin{mdframed}[style=mdpurplebox,frametitle={Problem statement}]
Let $a_0,a_1,\dots ,a_n$ be numbers from the interval $(0,\pi/2)$
such that $\tan (a_0-\frac{\pi}{4})+ \tan (a_1-\frac{\pi}{4})
 + \dotsb +\tan (a_n-\frac{\pi}{4}) \ge  n-1$.
Prove that
\[ \tan a_0\tan a_1 \dotsm \tan a_n \ge n^{n+1}. \]
\end{mdframed}
Let $x_i = \tan(a_i - \frac{\pi}{4})$.
Then we have that
\[ \tan a_i = \tan(a_i - 45\dg + 45\dg) = \frac{x_i + 1}{1 - x_i}. \]
If we further substitute $y_i = \frac{1 - x_i}{2} \in (0,1)$,
then we have to prove that the following statement:
\begin{claim*}
  If $\sum_0^n y_i \le 1$ and $y_i \ge 0$, we have
  \[ \prod_{i=1}^n \left( \frac{1}{y_i}-1 \right) \ge n^{n+1}. \]
\end{claim*}
\begin{proof}
  Homogenizing, we have to prove that
  \[ \prod_{i=1}^n \left( \frac{y_0+y_1+y_2+\dots+y_n}{y_i}-1 \right) \ge n^{n+1}. \]
  By AM-GM, we have
  \[ \frac{y_1+y_2+y_3+\dots+y_n}{y_0}
    \ge n \sqrt[n]{\frac{y_1y_2y_3\dots y_n}{y_1}}. \]
    Cyclic product works.
\end{proof}
\begin{remark*}
  Alternatively, the function $x \mapsto \log(1/x-1)$
  is a convex function on $(0,1)$ so Jensen inequality should also work.
\end{remark*}
\pagebreak

\section{Solutions to Day 2}
\subsection{USAMO 1998/4}
\textsl{Available online at \url{https://aops.com/community/p343869}.}
\begin{mdframed}[style=mdpurplebox,frametitle={Problem statement}]
A computer screen shows a $98 \times 98$ chessboard, colored in the usual way.
One can select with a mouse any rectangle with sides on the lines
of the chessboard and click the mouse button:
as a result, the colors in the selected rectangle switch
(black becomes white, white becomes black).
Find, with proof, the minimum number of mouse clicks
needed to make the chessboard all one color.
\end{mdframed}
The answer is $98$.
One of several possible constructions is to toggle
all columns and rows with even indices.

In the other direction, let $n = 98$ and
suppose that $k$ rectangles are used,
none of which are $n \times n$ (else we may delete it).
Then, for any two orthogonally adjacent cells,
the edge between them must be contained in the edge of
one of the $k$ rectangles.

We define a \emph{gridline} to be a line segment
that runs in the interior of the board from
one side of the board to the other.
Hence there are $2n-2$ gridlines exactly.
Moreover, we can classify these rectangles into two types:
\begin{itemize}
  \ii \emph{Full length rectangles}:
  these span from one edge of the board to the other.
  The two long sides completely cover two gridlines,
  but the other two sides of the rectangle do not.
  \ii \emph{Partial length rectangles}:
  each of four sides can partially cover ``half a'' gridline.
\end{itemize}
See illustration below for $n = 6$.
\begin{center}
\begin{asy}
  unitsize(0.5cm);
  picture base;
  for (int i=1; i<=5; ++i) {
    draw(base, (0,i)--(6,i), black );
    draw(base, (i,0)--(i,6), black );
  }

  picture full;
  picture partial;

  add(full, base);
  add(partial, base);
  label(full, "Full length", (3,0), dir(-90), red);
  label(partial, "Partial length", (3,0), dir(-90), blue);

  filldraw(full, box( (0,3), (6,1) ), lightred+opacity(0.3), red+1.4 );
  filldraw(partial, box( (2,2), (4,5) ), lightcyan+opacity(0.3), blue+1.4 );
  add(shift(0,0)*base);
  add(shift(7,0)*full);
  add(shift(14,0)*partial);
\end{asy}
\end{center}
Since there are $2n-2$ gridlines;
and each rectangle can cover at most two gridlines in total
(where partial-length rectangles are ``worth $\half$''
on each of the four sides),
we immediately get the bound $2k \ge 2n-2$, or $k \ge n-1$.

To finish, we prove that:
\begin{claim*}
  If equality holds and $k = n-1$, then $n$ is odd.
\end{claim*}
\begin{proof}
  If equality holds, then look at the horizontal gridlines
  and say two gridlines are \emph{related} if some rectangle
  has horizontal edges along both gridlines.
  (Hence, the graph has degree either $1$ or $2$ at each vertex,
  for equality to hold.)
  The reader may verify the resulting graph consists only of
  even length cycles and single edges,
  which would mean $n-1$ is even.
\end{proof}
Hence for $n = 98$ the answer is indeed $98$ as claimed.
\pagebreak

\subsection{USAMO 1998/5}
\textsl{Available online at \url{https://aops.com/community/p343870}.}
\begin{mdframed}[style=mdpurplebox,frametitle={Problem statement}]
Prove that for each $n\geq 2$,
there is a set $S$ of $n$ integers
such that $(a-b)^2$ divides $ab$ for every distinct $a,b\in S$.
\end{mdframed}
This is a direct corollary of the more difficult
USA TST 2015/2, reproduced below.
\begin{quote}
  Prove that for every positive integer $n$,
  there exists a set $S$ of $n$ positive integers
  such that for any two distinct $a,b \in S$,
  $a-b$ divides $a$ and $b$
  but none of the other elements of $S$.
\end{quote}
\pagebreak

\subsection{USAMO 1998/6}
\textsl{Available online at \url{https://aops.com/community/p150148}.}
\begin{mdframed}[style=mdpurplebox,frametitle={Problem statement}]
Let $n \geq 5$ be an integer.
Find the largest integer $k$ (as a function of $n$)
such that there exists a convex $n$-gon $A_{1}A_{2}\dots A_{n}$
for which exactly $k$ of the quadrilaterals
$A_{i}A_{i+1}A_{i+2}A_{i+3}$ have an inscribed circle,
where indices are taken modulo $n$.
\end{mdframed}
The main claim is the following:
\begin{claim*}
  We can't have both $A_1 A_2 A_3 A_4$
  and $A_2 A_3 A_4 A_5$ be circumscribed.
\end{claim*}
\begin{proof}
  If not, then we have the following diagram,
  where $a = A_1A_2$, $b = A-2 A_3$, $c = A_3 A_4$, $d = A_4 A_5$.
  \begin{center}
  \begin{asy}
    size(9cm);
    pair A1 = dir(10);
    pair A2 = dir(40);
    pair A3 = dir(90);
    pair A4 = dir(130);
    pair A5 = dir(200);
    draw(A1--A2--A3--A4--A5, blue);
    label("$a$", midpoint(A1--A2), dir(A1+A2), blue);
    label("$b$", midpoint(A2--A3), dir(A2+A3), blue);
    label("$c$", midpoint(A3--A4), dir(A3+A4), blue);
    label("$d$", midpoint(A4--A5), dir(A4+A5), blue);
    draw(A1--A4, red);
    label(rotate(-20)*"$c+a-b$", 0.6*A4+0.4*A1, dir(A1+A4), red);
    draw(A2--A5, orange);
    label(rotate(30)*"$b+d-c$", 0.5*A5+0.5*A2, dir(A2+A5), orange);
    dot("$A_1$", A1, dir(A1), blue);
    dot("$A_2$", A2, dir(A2), blue);
    dot("$A_3$", A3, dir(A3), blue);
    dot("$A_4$", A4, dir(A4), blue);
    dot("$A_5$", A5, dir(A5), blue);
  \end{asy}
  \end{center}
  Then $A_1 A_4 = c+a-b$ and $A_5A_2 = b+d-c$.
  But now
  \[ A_1 A_4 + A_2 A_5 = (c+a-b) + (b+d-c) = a+d = A_1 A_2 + A_4 A_5 \]
  but in the picture we have an obvious violation
  of the triangle inequality.
\end{proof}

This immediately gives an upper bound of
$\left\lfloor n/2 \right\rfloor$.

For the construction, one can construct a suitable cyclic $n$-gon by
using a continuity argument (details to be added).
\pagebreak


\end{document}
