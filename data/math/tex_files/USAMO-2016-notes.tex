% © Evan Chen
% Downloaded from https://web.evanchen.cc/

\documentclass[11pt]{scrartcl}
\usepackage[sexy]{evan}
\ihead{\footnotesize\textbf{\thetitle}}
\ohead{\footnotesize\href{http://web.evanchen.cc}{\ttfamily web.evanchen.cc},
    updated \today}
\title{USAMO 2016 Solution Notes}
\date{\today}

\begin{document}

\maketitle

\begin{abstract}
This is a compilation of solutions
for the 2016 USAMO.
Some of the solutions are my own work,
but many are from the official solutions provided by the organizers
(for which they hold any copyrights),
and others were found by users on the Art of Problem Solving forums.

These notes will tend to be a bit more advanced and terse than the ``official''
solutions from the organizers.
In particular, if a theorem or technique is not known to beginners
but is still considered ``standard'', then I often prefer to
use this theory anyways, rather than try to work around or conceal it.
For example, in geometry problems I typically use directed angles
without further comment, rather than awkwardly work around configuration issues.
Similarly, sentences like ``let $\mathbb{R}$ denote the set of real numbers''
are typically omitted entirely.

Corrections and comments are welcome!
\end{abstract}

\tableofcontents
\newpage

\addtocounter{section}{-1}
\section{Problems}
\begin{enumerate}[\bfseries 1.]
\item %% Problem 1
Let $X_1, X_2, \dots, X_{100}$ be a sequence of mutually distinct nonempty subsets of a set $S$.
Any two sets $X_i$ and $X_{i+1}$ are disjoint and their
union is not the whole set $S$,
that is, $X_i\cap X_{i+1}=\emptyset$ and $X_i\cup X_{i+1}\neq S$,
for all $i\in\{1, \dots, 99\}$.
Find the smallest possible number of elements in $S$.

\item %% Problem 2
Prove that for any positive integer $k$,
\[ (k^2)!\cdot\displaystyle\prod_{j=0}^{k-1}\frac{j!}{(j+k)!} \]
is an integer.

\item %% Problem 3
Let $ABC$ be an acute triangle
and let $I_B$, $I_C$, and $O$ denote its
$B$-excenter, $C$-excenter, and circumcenter, respectively.
Points $E$ and $Y$ are selected on $\ol{AC}$ such that
$\angle ABY = \angle CBY$ and $\ol{BE} \perp \ol{AC}$.
Similarly, points $F$ and $Z$ are selected on $\ol{AB}$ such that
$\angle ACZ = \angle BCZ$ and $\ol{CF} \perp \ol{AB}$.

Lines $I_B F$ and $I_C E$ meet at $P$.
Prove that $\ol{PO}$ and $\ol{YZ}$ are perpendicular.

\item %% Problem 4
Find all functions $f \colon \RR \to \RR$ such that
for all real numbers $x$ and $y$,
\[ (f(x)+xy) \cdot f(x-3y)
 + (f(y)+xy) \cdot f(3x-y)
 = (f(x+y))^2. \]

\item %% Problem 5
An equilateral pentagon $AMNPQ$ is inscribed in triangle $ABC$
such that $M \in \ol{AB}$, $Q \in \ol {AC}$, and $N,P \in \ol{BC}$.
Let $S$ be the intersection of $\ol{MN}$ and $\ol{PQ}$.
Denote by $\ell$ the angle bisector of $\angle MSQ$.

Prove that $\ol{OI}$ is parallel to $\ell$,
where $O$ is the circumcenter of triangle $ABC$,
and $I$ is the incenter of triangle $ABC$.

\item %% Problem 6
Integers $n$ and $k$ are given, with $n \ge k \ge 2$.
You play the following game against an evil wizard.
The wizard has $2n$ cards; for each $i=1,\dots,n$, there are two cards labeled $i$.
Initially, the wizard places all cards face down in a row, in unknown order.
You may repeatedly make moves of the following form: you point to any $k$ of the cards.
The wizard then turns those cards face up.
If any two of the cards match, the game is over and you win.
Otherwise, you must look away, while the wizard arbitrarily permutes
the $k$ chosen cards and then turns them back face-down. Then, it is your turn again.

We say this game is \emph{winnable} if there exist some positive integer $m$ and
some strategy that is guaranteed to win in at most $m$ moves,
no matter how the wizard responds.
For which values of $n$ and $k$ is the game winnable?

\end{enumerate}
\pagebreak

\section{Solutions to Day 1}
\subsection{USAMO 2016/1, proposed by Iurie Boreico}
\textsl{Available online at \url{https://aops.com/community/p6213589}.}
\begin{mdframed}[style=mdpurplebox,frametitle={Problem statement}]
Let $X_1, X_2, \dots, X_{100}$ be a sequence of mutually distinct nonempty subsets of a set $S$.
Any two sets $X_i$ and $X_{i+1}$ are disjoint and their
union is not the whole set $S$,
that is, $X_i\cap X_{i+1}=\emptyset$ and $X_i\cup X_{i+1}\neq S$,
for all $i\in\{1, \dots, 99\}$.
Find the smallest possible number of elements in $S$.
\end{mdframed}
Solution with Danielle Wang: the answer is that $|S| \ge 8$.

\textbf{Proof of sufficiency}
Since we must have $2^{|S|} \geq 100$, we must have $|S| \geq 7$.

To see that $|S| = 8$ is the minimum possible size,
consider a chain on the set $S = \{1, 2, \dots, 7\}$
satisfying $X_i \cap X_{i+1} = \emptyset$ and $X_i \cup X_{i+1} \neq S$.
Because of these requirements any subset of size $4$ or more
can only be neighbored by sets of size $2$ or less,
of which there are $\binom 71 + \binom 72 = 28$ available.
Thus, the chain can contain no more than $29$ sets of size $4$ or more
and no more than $28$ sets of size $2$ or less.
Finally, since there are only $\binom 73 = 35$ sets of size $3$ available,
the total number of sets in such a chain
can be at most $29 + 28 + 35 = 92 < 100$.

\textbf{Construction}
We will provide an inductive construction for a
chain of subsets $X_1, X_2, \dots, X_{2^{n-1} + 1}$
of  $S = \left\{ 1, \dots, n \right\}$
satisfying $X_i \cap X_{i+1} = \varnothing$
and $X_i \cup X_{i+1} \neq S$ for each $n \geq 4$.

For $S = \{1, 2, 3, 4\}$,
the following chain of length $2^3 + 1 = 9$ will work:
\[
\begin{array}{ccccccccc}
  34 & 1 & 23 & 4 & 12 & 3 & 14 & 2 & 13
\end{array}.
\]
Now, given a chain of subsets of $\{1, 2, \dots, n\}$
the following procedure produces a
chain of subsets of $\{1, 2, \dots, n+1\}$:
\begin{enumerate}
\item take the original chain, delete any element,
  and make two copies of this chain, which now has even length;
\item glue the two copies together,
  joined by $\varnothing$ in between; and then
\item insert the element $n+1$ into the sets in alternating positions
  of the chain starting with the first.
\end{enumerate}
For example, the first iteration of this construction gives:
\[
 \begin{array}{ccccccccc}
 345 & 1 & 235 & 4 & 125 & 3 & 145 & 2 & 5 \\
 34 & 15 & 23 & 45 & 12 & 35 & 14 & 25 &
 \end{array}
\]
It can be easily checked that if the original chain satisfies the requirements,
then so does the new chain, and if the original chain has length $2^{n-1}+1$,
then the new chain has length $2^{n}+1$, as desired.
This construction yields a chain of length $129$
when $S = \{1, 2, \dots, 8\}$.

\begin{remark*}
  Here is the construction for $n=8$ in its full glory.
\[
 \begin{array}{ccccccccc}
 345678 & 1 & 235678 & 4 & 125678 & 3 & 145678 & 2 & 5678 \\
 34 & 15678 & 23 & 45678 & 12 & 35678 & 14 & 678 & \\
 345 & 1678 & 235 & 4678 & 125 & 3678 & 145 & 2678 & 5 \\
 34678 & 15 & 23678 & 45 & 12678 & 35 & 78 & & \\\hline
 3456 & 178 & 2356 & 478 & 1256 & 378 & 1456 & 278 & 56 \\
 3478 & 156 & 2378 & 456 & 1278 & 356 & 1478 & 6 & \\
 34578 & 16 & 23578 & 46 & 12578 & 36 & 14578 & 26 & 578 \\
 346 & 1578 & 236 & 4578 & 126 & 8 & & & \\ \hline\hline
 34567 & 18 & 23567 & 48 & 12567 & 38 & 14567 & 28 & 567 \\
 348 & 1567 & 238 & 4567 & 128 & 3567 & 148 & 67 & \\
 3458 & 167 & 2358 & 467 & 1258 & 367 & 1458 & 267 & 58 \\
 3467 & 158 & 2367 & 458 & 1267 & 358 & 7 & & \\\hline
 34568 & 17 & 23568 & 47 & 12568 & 37 & 14568 & 27 & 568 \\
 347 & 1568 & 237 & 4568 & 127 & 3568 & 147 & 68 & \\
 3457 & 168 & 2357 & 468 & 1257 & 368 & 1457 & 268 & 57 \\
 3468 & 157 & 2368 & 457 & 1268 & & & & \\
 \end{array}
\]
\end{remark*}
\pagebreak

\subsection{USAMO 2016/2, proposed by Kiran Kedlaya}
\textsl{Available online at \url{https://aops.com/community/p6213627}.}
\begin{mdframed}[style=mdpurplebox,frametitle={Problem statement}]
Prove that for any positive integer $k$,
\[ (k^2)!\cdot\displaystyle\prod_{j=0}^{k-1}\frac{j!}{(j+k)!} \]
is an integer.
\end{mdframed}
We show the exponent of any given prime $p$ is nonnegative in the expression.
Recall that the exponent of $p$ in $n!$ is equal to
$\sum_{i \ge 1} \left\lfloor n/p^i \right\rfloor$.
In light of this, it suffices to show
that for any prime power $q$, we have
\[
\left\lfloor \frac{k^2}{q} \right\rfloor
+ \sum_{j=0}^{k-1} \left\lfloor \frac{j}{q} \right\rfloor
\ge \sum_{j=0}^{k-1} \left\lfloor \frac{j+k}{q} \right\rfloor
\]
Since both sides are integers, we show
\[
\left\lfloor \frac{k^2}{q} \right\rfloor
+ \sum_{j=0}^{k-1} \left\lfloor \frac{j}{q} \right\rfloor
> -1 + \sum_{j=0}^{k-1} \left\lfloor \frac{j+k}{q} \right\rfloor.
\]
If we denote by $\left\{ x \right\}$ the fractional part of $x$,
then $\left\lfloor x \right\rfloor = x - \left\{ x \right\}$
so it's equivalent to
\[
\left\{ \frac{k^2}{q} \right\}
+ \sum_{j=0}^{k-1} \left\{ \frac{j}{q} \right\}
< 1 + \sum_{j=0}^{k-1} \left\{ \frac{j+k}{q} \right\}.
\]
However, the sum of remainders when $0, 1, \dots, k-1$ are taken modulo $q$
is easily seen to be less than the sum of remainders
when $k, k+1, \dots, 2k-1$ are taken modulo $q$.
So \[ \sum_{j=0}^{k-1} \left\{ \frac{j}{q} \right\}
\le \sum_{j=0}^{k-1} \left\{ \frac{j+k}{q} \right\} \] follows,
and we are done upon noting $\left\{ k^2/q \right\} < 1$.
\pagebreak

\subsection{USAMO 2016/3, proposed by Evan Chen, Telv Cohl}
\textsl{Available online at \url{https://aops.com/community/p6213572}.}
\begin{mdframed}[style=mdpurplebox,frametitle={Problem statement}]
Let $ABC$ be an acute triangle
and let $I_B$, $I_C$, and $O$ denote its
$B$-excenter, $C$-excenter, and circumcenter, respectively.
Points $E$ and $Y$ are selected on $\ol{AC}$ such that
$\angle ABY = \angle CBY$ and $\ol{BE} \perp \ol{AC}$.
Similarly, points $F$ and $Z$ are selected on $\ol{AB}$ such that
$\angle ACZ = \angle BCZ$ and $\ol{CF} \perp \ol{AB}$.

Lines $I_B F$ and $I_C E$ meet at $P$.
Prove that $\ol{PO}$ and $\ol{YZ}$ are perpendicular.
\end{mdframed}
We present two solutions.

\paragraph{First solution.}
Let $I_A$ denote the $A$-excenter and $I$ the incenter.
Then let $D$ denote the foot of the altitude from $A$.
Suppose the $A$-excircle is tangent to $\ol{BC}$, $\ol{AB}$, $\ol{AC}$
at $A_1$, $B_1$, $C_1$ and let
$A_2$, $B_2$, $C_2$ denote the reflections of $I_A$ across these points.
Let $S$ denote the circumcenter of $\triangle I I_B I_C$.

\begin{center}
\begin{asy}
size(14cm);

pair A = dir(110);
pair B = dir(208);
pair C = dir(332);
draw(A--B--C--cycle, blue+1);
pair I = incenter(A, B, C);
pair O = circumcenter(A, B, C);
pair E = foot(B, C, A);
pair F = foot(C, A, B);
pair M_A = extension(A, I, O, B+C);
pair I_A = 2*M_A-I;
pair M_B = extension(B, I, O, C+A);
pair I_B = 2*M_B-I;
pair M_C = extension(C, I, O, A+B);
pair I_C = 2*M_C-I;

pair P = extension(E, I_C, F, I_B);
draw(A--I_A, blue+dashed);

pair D = foot(A, B, C);
filldraw(I--I_B--I_C--cycle, opacity(0.2)+lightred, red);
filldraw(unitcircle, opacity(0.1)+lightblue, blue);

pair A_1 = foot(I_A, B, C);
pair B_1 = foot(I_A, A, B);
pair C_1 = foot(I_A, C, A);
draw(arc(I_A, abs(I_A-A_1), 15, 165), blue+dotted);
draw(C--C_1, blue+dashed);
draw(B--B_1, blue+dashed);
pair A_2 = 2*A_1-I_A;
pair B_2 = 2*B_1-I_A;
pair C_2 = 2*C_1-I_A;
draw(D--A_2, heavygreen);
draw(I_B--B_2, heavygreen);
draw(I_C--C_2, heavygreen);
draw(I_A--A_2, heavygreen+dashed);
draw(I_A--B_2, heavygreen+dashed);
draw(I_A--C_2, heavygreen+dashed);

draw(I_B--I_A--I_C, red+dashed);
draw(B--I--C, red+dashed);
filldraw(A_2--B_2--C_2--cycle, opacity(0.2)+lightgreen, deepgreen+1);
pair S = circumcenter(I, I_B, I_C);
draw(S--I_A, orange);
draw(arc(S, abs(S-I), 185, 355), red+dotted);

pair Y = extension(B, I, A, C);
pair Z = extension(C, I, A, B);
draw(Y--Z, red);

dot("$A$", A, dir(A));
dot("$B$", B, dir(180));
dot("$C$", C, dir(0));
dot("$I$", I, dir(270));
dot("$O$", O, dir(-45));
dot("$E$", E, dir(30));
dot("$F$", F, dir(F));
dot("$I_A$", I_A, dir(I_A));
dot("$I_B$", I_B, dir(I_B));
dot("$I_C$", I_C, dir(I_C));
dot("$P$", P, dir(135));
dot("$D$", D, dir(-90));
dot("$A_1$", A_1, dir(A_1));
dot("$B_1$", B_1, dir(200));
dot("$C_1$", C_1, dir(20));
dot("$A_2$", A_2, dir(A_2));
dot("$B_2$", B_2, dir(B_2));
dot("$C_2$", C_2, dir(C_2));
dot("$S$", S, dir(S));
dot("$Y$", Y, dir(350));
dot("$Z$", Z, dir(110));

/* Source generated by TSQ

!size(14cm);

A = dir 110
B = dir 208 R180
C = dir 332 R0
A--B--C--cycle blue+1
I = incenter A B C R270
O = circumcenter A B C R-45
E = foot B C A R30
F = foot C A B
M_A := extension A I O B+C
I_A = 2*M_A-I
M_B := extension B I O C+A
I_B = 2*M_B-I
M_C := extension C I O A+B
I_C = 2*M_C-I

P = extension E I_C F I_B R140
A--I_A blue dashed

D = foot A B C R-90
I--I_B--I_C--cycle 0.2 lightred / red
unitcircle 0.1 lightblue / blue

A_1 = foot I_A B C
B_1 = foot I_A A B R200
C_1 = foot I_A C A R20
!draw(arc(I_A, abs(I_A-A_1), 15, 165), blue+dotted);
C--C_1 blue dashed
B--B_1 blue dashed
A_2 = 2*A_1-I_A
B_2 = 2*B_1-I_A
C_2 = 2*C_1-I_A
D--A_2 heavygreen
I_B--B_2 heavygreen
I_C--C_2 heavygreen
I_A--A_2 heavygreen dashed
I_A--B_2 heavygreen dashed
I_A--C_2 heavygreen dashed

I_B--I_A--I_C red dashed
B--I--C red dashed
A_2--B_2--C_2--cycle 0.2 lightgreen / deepgreen+1
S = circumcenter I I_B I_C
S--I_A orange
!draw(arc(S, abs(S-I), 185, 355), red+dotted);

Y = extension B I A C R350
Z = extension C I A B R110
Y--Z red



*/
\end{asy}
\end{center}

We begin with the following observation:
\begin{claim*}
  Points $D$, $I$, $A_2$ are collinear, as are
  points $E$, $I_C$, $C_2$ are collinear
  and points $F$, $I_B$, $B_2$ are collinear.
\end{claim*}

\begin{proof}
  This basically follows from the ``midpoints of altitudes'' lemma.
  To see $D$, $I$, $A_2$ are collinear,
  recall first that $\ol{IA_1}$ passes
  through the midpoint $M$ of $\ol{AD}$.
  \begin{center}
  \begin{asy}
    pair A = dir(125);
    pair B = dir(220);
    pair C = dir(320);
    draw(A--B--C--cycle, blue);
    pair I = incenter(A, B, C);
    pair O = circumcenter(A, B, C);
    pair M_A = extension(A, I, O, B+C);
    pair I_A = 2*M_A-I;
    filldraw(incircle(A, B, C), opacity(0.2)+lightblue, blue);

    pair D = foot(A, B, C);
    filldraw(unitcircle, opacity(0.1)+palecyan, blue);
    draw(A--D, heavygreen);

    pair A_1 = foot(I_A, B, C);
    draw(arc(I_A, abs(I_A-A_1), 0, 180), blue+dashed);
    pair B_1 = foot(I_A, A, B);
    pair C_1 = foot(I_A, C, A);
    draw(C--C_1, blue+dotted);
    draw(B--B_1, blue+dotted);

    pair A_2 = 2*A_1-I_A;
    draw(A--I_A, blue);
    draw(D--A_2, red);
    draw(I_A--A_2, blue+dotted);

    pair P = foot(I, B, C);
    pair Q = 2*I-P;
    draw(A--A_1, heavygreen);
    draw(P--Q, heavygreen);

    pair U = extension(Q, Q+B-C, A, B);
    pair V = extension(Q, Q+C-B, A, C);
    draw(U--V, heavygreen);

    pair M = midpoint(A--D);
    draw(M--A_1, dotted);

    dot(P); dot(Q);


    dot("$A$", A, dir(A));
    dot("$B$", B, dir(180));
    dot("$C$", C, dir(0));
    dot("$I$", I, dir(10));
    dot("$I_A$", I_A, dir(I_A));
    dot("$D$", D, dir(-90));
    dot("$A_1$", A_1, dir(-45));
    dot("$B_1$", B_1, dir(200));
    dot("$C_1$", C_1, dir(20));
    dot("$A_2$", A_2, dir(A_2));
    dot("$M$", M, dir(0));

    /* TSQ Source:

    A = dir 125
    B = dir 220 R180
    C = dir 320 R0
    A--B--C--cycle blue
    I = incenter A B C R10
    O := circumcenter A B C R45
    M_A := extension A I O B+C
    I_A = 2*M_A-I
    incircle A B C 0.2 lightblue / blue

    D = foot A B C R-90
    unitcircle 0.1 palecyan / blue
    A--D heavygreen

    A_1 = foot I_A B C R-45
    !draw(arc(I_A, abs(I_A-A_1), 0, 180), blue+dashed);
    B_1 = foot I_A A B R200
    C_1 = foot I_A C A R20
    C--C_1 blue dotted
    B--B_1 blue dotted

    A_2 = 2*A_1-I_A
    A--I_A blue
    D--A_2 red
    I_A--A_2 blue dotted

    P := foot I B C
    Q := 2*I-P
    A--A_1 heavygreen
    P--Q heavygreen

    U := extension Q Q+B-C A B
    V := extension Q Q+C-B A C
    U--V heavygreen

    M = midpoint A--D R0
    M--A_1 dotted

    !dot(P); dot(Q);

    */

  \end{asy}
  \end{center}
  Now since $\ol{AD} \parallel \ol{I_A A_2}$,
  and $M$ and $A_1$ are the midpoints of $\ol{AD}$ and $\ol{I_A A_2}$,
  it follows from the collinearity of $A$, $I$, $I_A$
  that $D$, $I$, $A_2$ are collinear as well.

  The other two claims follow in a dual fashion.
  For example, using the homothety taking the $A$ to $C$-excircle,
  we find that $\ol{C_1I_C}$ bisects the altitude $\ol{BE}$,
  and since $I_C$, $B$, $I_A$ are collinear the same argument
  now gives $I_C$, $E$, $C_2$ are collinear.
  The fact that $I_B$, $F$, $B_2$ are collinear is symmetric.
\end{proof}

Observe that $\ol{B_2C_2} \parallel \ol{B_1C_1} \parallel \ol{I_B I_C}$.
Proceeding similarly on the other sides,
we discover $\triangle I I_B I_C$ and $\triangle A_2 B_2 C_2$ are homothetic.
Hence $P$ is the center of this homothety (in particular, $D$, $I$, $P$, $A_2$ are collinear).
Moreover, $P$ lies on the line joining $I_A$ to $S$,
which is the Euler line of $\triangle I I_B I_C$,
so it passes through the nine-point center of $\triangle I I_B I_C$,
which is $O$. Consequently, $P$, $O$, $I_A$ are collinear as well.

To finish, we need only prove that $\ol{OS} \perp \ol{YZ}$.
In fact, we claim that $\ol{YZ}$ is the radical axis
of the circumcircles of $\triangle ABC$ and $\triangle I I_B I_C$.
Actually, $Y$ is the radical center of these two circumcircles
and the circle with diameter $\ol{II_B}$
(which passes through $A$ and $C$).
Analogously $Z$ is the radical center of the circumcircles
and the circle with diameter $\ol{II_C}$, and the proof is complete.

\paragraph{Second solution (barycentric, outline, Colin Tang).}
we are going to use barycentric coordinates
to show that the line through $O$ perpendicular to $\ol{YZ}$
is concurrent with $\ol{I_B F}$ and $\ol{I_C E}$.

The displacement vector $\overrightarrow{YZ}$
is proportional to $(a(b-c):-b(a+c):c(a+b))$,
and so by strong perpendicularity criterion
and doing a calculation gives the line
\[ x(b-c)bc(a+b+c) + y(a+c)ac(a+b-c) + z(a+b)ab(-a+b-c) = 0. \]
On the other hand, line $I_C E$ has equation
\[
  0 =
  \det \begin{bmatrix}
    a &  b & -c \\
    S_C & 0 & S_A \\
    x & y & z
  \end{bmatrix}
  = bS_a \cdot x + (-cS_C-aS_A) \cdot y + (-bS_C) \cdot z
\]
and similarly for $I_B F$.
Consequently, concurrence of these lines is equivalent to
\[
  \det \begin{bmatrix}
    bS_A & -cS_C - aS_A & -bS_C \\
    cS_A & -cS_B & -aS_A - bS_B \\
    (b-c)bc(a+b+c) & (a+c)ac(a+b-c) & (a+b)ab(-a+b-c)
  \end{bmatrix}
  = 0
\]
which is a computation.

\paragraph{Authorship comments.}
I was intrigued by a Taiwan TST problem which implied that,
in the configuration above, $\angle I_B D I_C$ was bisected by $\ol{DA}$.
This motivated me to draw all three properties above where $I_A$ and $P$
were isogonal conjugates with respect to $DEF$.
After playing around with this picture for a long time,
I finally noticed that $O$ was on line $PI_A$.
(So the original was to show that $I_B F$, $I_C E$, $DA_2$ concurrent).
Eventually I finally noticed in the picture that $PI_A$
actually passed through the circumcenter of $ABC$ as well.
This took me many hours to prove.

The final restatement (which follows quickly from $P$, $O$, $I_A$ collinear)
was discovered by Telv Cohl when I showed him the problem.
\pagebreak

\section{Solutions to Day 2}
\subsection{USAMO 2016/4, proposed by Titu Andreescu}
\textsl{Available online at \url{https://aops.com/community/p6220308}.}
\begin{mdframed}[style=mdpurplebox,frametitle={Problem statement}]
Find all functions $f \colon \RR \to \RR$ such that
for all real numbers $x$ and $y$,
\[ (f(x)+xy) \cdot f(x-3y)
 + (f(y)+xy) \cdot f(3x-y)
 = (f(x+y))^2. \]
\end{mdframed}
We claim that the only two functions satisfying
the requirements are $f(x) \equiv 0$ and $f(x) \equiv x^2$.
These work.

First, taking $x=y=0$ in the given yields $f(0) = 0$,
and then taking $x=0$ gives $f(y)f(-y) = f(y)^2$.
So also $f(-y)^2 = f(y)f(-y)$, from which we conclude $f$ is even.
Then taking $x = -y$ gives
\[ \forall x \in \RR : \qquad
f(x) = x^2 \qquad\text{or}\qquad f(4x) = 0 \qquad(\bigstar) \]
for all $x$.

\begin{remark*}
  Note that an example of a function satisfying $(\bigstar)$ is
  \[
    f(x)
    =
    \begin{cases}
      x^2 & \text{if } |x| < 1 \\
      \log(x^{42} + 2016^{\cos(x)}) & \text{if } 1 \le |x| < 4 \\
      0 & \text{if } |x| \ge 4.
    \end{cases}
  \]
  So, yes, we are currently in a world of trouble, still.
\end{remark*}

Now we claim
\begin{claim*}
  $f(z) = 0 \iff f(2z) = 0 \qquad (\spadesuit)$.
\end{claim*}
\begin{proof}
  Let $(x,y)=(3t,t)$ in the given to get
  \[ \left( f(t)+3t^2 \right)f(8t) = f(4t)^2. \]
  Now if $f(4t) \neq 0$ (in particular, $t \neq 0$),
  then $f(8t) \neq 0$.
  Thus we have $(\spadesuit)$ in the reverse direction.

  Then $f(4t) \neq 0 \overset{(\bigstar)}{\implies} f(t) = t^2 \neq 0
  \overset{(\spadesuit)}{\implies} f(2t) \neq 0$
  implies the forwards direction,
  the last step being the reverse direction $(\spadesuit)$.
\end{proof}
%so by $(\bigstar)$ we have $f(t) = t^2$, and moreover $f(8t) \neq 0$.
%But we also have
%\[ \left( f(t/4)+\frac{3}{16}t^2 \right)f(2t) = f(t)^2 = t^4 \neq 0 \]
%and hence $f(2t) \neq 0$ as well.
%Thus we have seen that $f(4t) \neq 0 \implies f(8t), f(2t) \neq 0$
%which is logically equivalent to $(\spadesuit)$.

By putting together $(\bigstar)$ and $(\spadesuit)$ we finally get
\[ \forall x \in \RR : \qquad
f(x) = x^2 \qquad\text{or}\qquad f(x) = 0 \qquad(\heartsuit) \]

We are now ready to approach the main problem.
Assume there's an $a \neq 0$ for which $f(a) = 0$;
we show that $f \equiv 0$.

Let $b \in \RR$ be given.
Since $f$ is even, we can assume without loss of generality that $a, b > 0$.
Also, note that $f(x) \ge 0$ for all $x$ by $(\heartsuit)$.
By using $(\spadesuit)$ we can generate $c > b$
such that $f(c) = 0$ by taking $c = 2^n a$ for a large enough integer $n$.
Now, select $x, y > 0$ such that
$x-3y=b$ and $x+y=c$. That is,
\[
  (x,y) = \left( \frac{3c+b}{4}, \frac{c-b}{4} \right).
\]
Substitution into the original equation gives
\[
  0 = \left( f(x) + xy \right) f(b)
  + \left( f(y) + xy \right) f(3x-y)
  \ge \left( f(x) + xy \right) f(b).
\]
But since $f(b) \ge 0$, it follows $f(b) = 0$, as desired.
\pagebreak

\subsection{USAMO 2016/5, proposed by Ivan Borsenco}
\textsl{Available online at \url{https://aops.com/community/p6220306}.}
\begin{mdframed}[style=mdpurplebox,frametitle={Problem statement}]
An equilateral pentagon $AMNPQ$ is inscribed in triangle $ABC$
such that $M \in \ol{AB}$, $Q \in \ol {AC}$, and $N,P \in \ol{BC}$.
Let $S$ be the intersection of $\ol{MN}$ and $\ol{PQ}$.
Denote by $\ell$ the angle bisector of $\angle MSQ$.

Prove that $\ol{OI}$ is parallel to $\ell$,
where $O$ is the circumcenter of triangle $ABC$,
and $I$ is the incenter of triangle $ABC$.
\end{mdframed}
\paragraph{First solution (complex).}
In fact, we only need $AM = AQ = NP$ and $MN = QP$.

We use complex numbers with $ABC$ the unit circle,
assuming WLOG that $A$, $B$, $C$ are labeled counterclockwise.
Let $x$, $y$, $z$ be the complex numbers corresponding to the arc midpoints
of $BC$, $CA$, $AB$, respectively; thus $x+y+z$ is the incenter of $\triangle ABC$.
Finally, let $s > 0$ be the side length of $AM = AQ = NP$.

Then, since $MA = s$ and $MA \perp OZ$, it follows that
\[ m - a = i \cdot sz. \]
Similarly, $p-n = i \cdot sy$ and $a-q = i \cdot sx$, so summing these up gives
\[ i \cdot s(x+y+z) = (p-q) + (m-n) = (m-n) - (q-p). \]
Since $MN = PQ$, the argument of $(m-n) - (q-p)$ is along
the external angle bisector of the angle formed, which is perpendicular to $\ell$.
On the other hand, $x+y+z$ is oriented in the same direction as $OI$, as desired.

\paragraph{Second solution (trig, Danielle Wang).}
Let $\delta$ and $\epsilon$ denote $\angle MNB$ and $\angle CPQ$.
Also, assume $AMNPQ$ has side length $1$.

In what follows, assume $AB < AC$.
First, we note that
\begin{align*}
BN   &=   (c-1) \cos B + \cos \delta \textrm{,}    \\
CP   &=   (b-1) \cos C + \cos \epsilon \textrm{, and}   \\
a   &=   1 + BN + CP
\end{align*}
from which it follows that
\[
\cos \delta + \cos \epsilon = \cos B + \cos C - 1
\]
Also, by the Law of Sines, we have $\frac{c-1}{\sin\delta} = \frac{1}{\sin B}$
and similarly on triangle $CPQ$, and from this we deduce
\[ \sin \epsilon - \sin \delta = \sin B - \sin C. \]
The sum-to-product formulas
\begin{align*}
\sin{\epsilon} - \sin{\delta} &= 2 \cos\left( \frac{\epsilon + \delta}{2} \right) \sin\left( \frac{\epsilon - \delta}{2} \right) \\
\cos{\epsilon} - \cos{\delta} &= 2 \cos\left( \frac{\epsilon + \delta}{2} \right) \cos\left( \frac{\epsilon - \delta}{2} \right)
\end{align*}
give us
\[
\tan \left( \frac{\epsilon-\delta}{2} \right) = \frac{\sin{\epsilon} - \sin{\delta}}{\cos{\epsilon} - \cos{\delta}}
= \frac{\sin B - \sin C}{\cos B + \cos C - 1}. \]
Now note that $\ell$ makes an angle of
$\frac{1}{2}(\pi+\epsilon-\delta)$ with line $BC$.
Moreover, if line $OI$ intersects line $BC$ with angle $\varphi$ then
\[ \tan\varphi = \frac{r - R \cos A}{\frac{1}{2}(b-c)}. \]
So in order to prove the result, we only need to check that
\[ \frac{r - R \cos A}{\frac{1}{2}(b-c)}
  = \frac{\cos B + \cos C + 1}{\sin B - \sin C}. \]
Using the fact that $b = 2R\sin B$, $c = 2R\sin C$, this  reduces to the fact that
$r/R + 1 = \cos A + \cos B + \cos C$, which is the so-called Carnot theorem.
\pagebreak

\subsection{USAMO 2016/6, proposed by Gabriel Carroll}
\textsl{Available online at \url{https://aops.com/community/p6220302}.}
\begin{mdframed}[style=mdpurplebox,frametitle={Problem statement}]
Integers $n$ and $k$ are given, with $n \ge k \ge 2$.
You play the following game against an evil wizard.
The wizard has $2n$ cards; for each $i=1,\dots,n$, there are two cards labeled $i$.
Initially, the wizard places all cards face down in a row, in unknown order.
You may repeatedly make moves of the following form: you point to any $k$ of the cards.
The wizard then turns those cards face up.
If any two of the cards match, the game is over and you win.
Otherwise, you must look away, while the wizard arbitrarily permutes
the $k$ chosen cards and then turns them back face-down. Then, it is your turn again.

We say this game is \emph{winnable} if there exist some positive integer $m$ and
some strategy that is guaranteed to win in at most $m$ moves,
no matter how the wizard responds.
For which values of $n$ and $k$ is the game winnable?
\end{mdframed}
The game is winnable if and only if $k < n$.

First, suppose $2 \le k < n$. Query the cards in positions $\left\{ 1, \dots, k \right\}$,
then $\left\{ 2, \dots, k+1 \right\}$, and so on, up to $\left\{ 2n-k+1, 2n \right\}$.
Indeed, by taking the difference of the $i$th and $(i+1)$st query,
we can deduce the value of the $i$th card, for $1 \le i \le 2n-k$.
(This is possible because the cards are flipped face up before they are re-shuffled,
so even if two adjacent queries return the same set,
one can still determine the $i$th card.
It is possible to solve the problem even without the flipped information, though.)
If $k \le n$, this is more than $n$ cards, so we can find a matching pair.

For $k = n$ we remark the following:
at each turn after the first, assuming one has not won,
there are $n$ cards representing each of the $n$ values exactly once,
such that the player has no information about the order of those $n$ cards.
We claim that consequently the player cannot guarantee victory.
Indeed, let $S$ denote this set of $n$ cards, and $\ol{S}$ the other $n$ cards.
The player will never win by picking only cards in $S$ or $\ol{S}$.
Also, if the player selects some cards in $S$ and some cards in $\ol{S}$,
then it is possible that the choice of cards in $S$ is exactly the complement
of those selected from $\ol{S}$; the strategy cannot prevent this since
the player has no information on $S$. This implies the result.
\pagebreak


\end{document}
