% © Evan Chen
% Downloaded from https://web.evanchen.cc/

\documentclass[11pt]{scrartcl}
\usepackage[sexy]{evan}
\ihead{\footnotesize\textbf{\thetitle}}
\ohead{\footnotesize\href{http://web.evanchen.cc}{\ttfamily web.evanchen.cc},
    updated \today}
\title{IMO 2005 Solution Notes}
\date{\today}

\begin{document}

\maketitle

\begin{abstract}
This is a compilation of solutions
for the 2005 IMO.
Some of the solutions are my own work,
but many are from the official solutions provided by the organizers
(for which they hold any copyrights),
and others were found by users on the Art of Problem Solving forums.

These notes will tend to be a bit more advanced and terse than the ``official''
solutions from the organizers.
In particular, if a theorem or technique is not known to beginners
but is still considered ``standard'', then I often prefer to
use this theory anyways, rather than try to work around or conceal it.
For example, in geometry problems I typically use directed angles
without further comment, rather than awkwardly work around configuration issues.
Similarly, sentences like ``let $\mathbb{R}$ denote the set of real numbers''
are typically omitted entirely.

Corrections and comments are welcome!
\end{abstract}

\tableofcontents
\newpage

\addtocounter{section}{-1}
\section{Problems}
\begin{enumerate}[\bfseries 1.]
\item %% Problem 1
Six points are chosen on the sides of an equilateral triangle $ABC$:
$A_1$, $A_2$ on $BC$, $B_1$, $B_2$ on $CA$ and $C_1$, $C_2$ on $AB$,
such that they are the vertices of a
convex hexagon $A_1A_2B_1B_2C_1C_2$ with equal side lengths.
Prove that the lines $A_1B_2$, $B_1C_2$ and $C_1A_2$ are concurrent.

\item %% Problem 2
Let $a_1$, $a_2$, \dots\ be a sequence of integers
with infinitely many positive and negative terms.
Suppose that for every positive integer $n$
the numbers $a_1$, $a_2$, \dots, $a_n$
leave $n$ different remainders upon division by $n$.
Prove that every integer occurs exactly once in the sequence.

\item %% Problem 3
Let $x,y,z > 0$ satisfy $xyz\geq 1$. Prove that
\[ \frac { x^5-x^2 }{x^5+y^2+z^2}
  + \frac {y^5-y^2}{x^2+y^5+z^2}
  + \frac {z^5-z^2}{x^2+y^2+z^5} \geq 0. \]

\item %% Problem 4
Determine all positive integers relatively
prime to all the terms of the infinite sequence
\[ a_n = 2^n+3^n+6^n-1, \quad n \ge 1. \]

\item %% Problem 5
Let $ABCD$ be a fixed convex quadrilateral
with $BC=DA$ and $\ol{BC} \nparallel \ol{DA}$.
Let two variable points $E$ and $F$ lie on the
sides $BC$ and $DA$, respectively, and satisfy $BE=DF$.
The lines $AC$ and $BD$ meet at $P$,
the lines $BD$ and $EF$ meet at $Q$,
the lines $EF$ and $AC$ meet at $R$.
Prove that the circumcircles of the triangles $PQR$,
as $E$ and $F$ vary, have a common point other than $P$.

\item %% Problem 6
In a mathematical competition $6$ problems were posed to the contestants.
Each pair of problems was solved by more than $\frac{2}{5}$ of the contestants.
Nobody solved all 6 problems.
Show that there were at least $2$ contestants
who each solved exactly $5$ problems.

\end{enumerate}
\pagebreak

\section{Solutions to Day 1}
\subsection{IMO 2005/1}
\textsl{Available online at \url{https://aops.com/community/p281571}.}
\begin{mdframed}[style=mdpurplebox,frametitle={Problem statement}]
Six points are chosen on the sides of an equilateral triangle $ABC$:
$A_1$, $A_2$ on $BC$, $B_1$, $B_2$ on $CA$ and $C_1$, $C_2$ on $AB$,
such that they are the vertices of a
convex hexagon $A_1A_2B_1B_2C_1C_2$ with equal side lengths.
Prove that the lines $A_1B_2$, $B_1C_2$ and $C_1A_2$ are concurrent.
\end{mdframed}
The six sides of the hexagon, when oriented, comprise
six vectors with vanishing sum.
However note that \[ \overrightarrow{A_1A_2}
+ \overrightarrow{B_1B_2}
+ \overrightarrow{C_1C_2} = 0. \]
Thus
\[ \overrightarrow{A_2B_1} + \overrightarrow{B_2C_1} +
\overrightarrow{C_2A_1} = 0 \]
and since three unit vectors with vanishing sum
must be rotations of each other by $120\dg$,
it follows they must also form an equilateral triangle.

\begin{center}
\begin{asy}
pair A_1 = origin;
pair A_2 = A_1+dir(0);
pair B_1 = A_2+dir(37);
pair B_2 = B_1+dir(120);
pair C_1 = B_2+dir(157);
pair C_2 = C_1+dir(240);

pair A = extension(B_1, B_2, C_1, C_2);
pair B = extension(C_1, C_2, A_1, A_2);
pair C = extension(A_1, A_2, B_1, B_2);

filldraw(A--B--C--cycle, opacity(0.1)+lightcyan, lightblue);

draw(A_1--A_2, red+1, EndArrow(TeXHead), Margins);
draw(B_1--B_2, red+1, EndArrow(TeXHead), Margins);
draw(C_1--C_2, red+1, EndArrow(TeXHead), Margins);
draw(A_2--B_1, blue+1, EndArrow(TeXHead), Margins);
draw(B_2--C_1, blue+1, EndArrow(TeXHead), Margins);
draw(C_2--A_1, blue+1, EndArrow(TeXHead), Margins);

filldraw(A_1--B_1--C_1--cycle, opacity(0.1)+lightgreen, heavygreen);
draw(A_1--B_2, dotted+heavycyan);
draw(B_1--C_2, dotted+heavycyan);
draw(C_1--A_2, dotted+heavycyan);

dot("$A_1$", A_1, dir(270));
dot("$A_2$", A_2, dir(270));
dot("$B_1$", B_1, dir(30));
dot("$B_2$", B_2, dir(30));
dot("$C_1$", C_1, dir(150));
dot("$C_2$", C_2, dir(150));
dot("$A$", A, dir(90));
dot("$B$", B, dir(210));
dot("$C$", C, dir(330));

/* TSQ Source:

A_1 = origin R270
A_2 = A_1+dir(0) R270
B_1 = A_2+dir(37) R30
B_2 = B_1+dir(120) R30
C_1 = B_2+dir(157) R150
C_2 = C_1+dir(240) R150

A = extension B_1 B_2 C_1 C_2 R90
B = extension C_1 C_2 A_1 A_2 R210
C = extension A_1 A_2 B_1 B_2 R330

A--B--C--cycle 0.1 lightcyan / lightblue

!draw(A_1--A_2, red+1, EndArrow(TeXHead), Margins);
!draw(B_1--B_2, red+1, EndArrow(TeXHead), Margins);
!draw(C_1--C_2, red+1, EndArrow(TeXHead), Margins);
!draw(A_2--B_1, blue+1, EndArrow(TeXHead), Margins);
!draw(B_2--C_1, blue+1, EndArrow(TeXHead), Margins);
!draw(C_2--A_1, blue+1, EndArrow(TeXHead), Margins);

A_1--B_1--C_1--cycle 0.1 lightgreen / heavygreen
A_1--B_2 dotted heavycyan
B_1--C_2 dotted heavycyan
C_1--A_2 dotted heavycyan

*/
\end{asy}
\end{center}

Consequently, triangles $A_1A_2B_1$, $B_1B_2C_1$, $C_1C_2A_1$
are congruent, as $\angle A_2 = \angle B_2 = \angle C_2$.
So triangle $A_1 B_1 C_1$ is equilateral and the
diagonals are concurrent at the center.
\pagebreak

\subsection{IMO 2005/2}
\textsl{Available online at \url{https://aops.com/community/p281572}.}
\begin{mdframed}[style=mdpurplebox,frametitle={Problem statement}]
Let $a_1$, $a_2$, \dots\ be a sequence of integers
with infinitely many positive and negative terms.
Suppose that for every positive integer $n$
the numbers $a_1$, $a_2$, \dots, $a_n$
leave $n$ different remainders upon division by $n$.
Prove that every integer occurs exactly once in the sequence.
\end{mdframed}
Obviously every integer appears at most once
(otherwise take $n$ much larger).
So we will prove every integer appears at least once.

\begin{claim*}
  For any $i < j$ we have $\left\lvert a_i-a_j \right\rvert < j$.
\end{claim*}
\begin{proof}
  Otherwise, let $n = \left\lvert a_i-a_j \right\rvert \neq 0$.
  Then $i,j \in [1,n]$ and $a_i \equiv a_j \pmod n$,
  contradiction.
\end{proof}

\begin{claim*}
  For any $n$, the set $\{a_1, \dots, a_n\}$
  is of the form $\{k+1, \dots, k+n\}$ for some integer $k$.
\end{claim*}
\begin{proof}
  By induction, with the base case $n=1$ being vacuous.
  For the inductive step,
  suppose $\{a_1, \dots, a_n\} = \{k+1, \dots, k+n\}$ are determined.
  Then
  \[ a_{n+1} \equiv k \pmod{n+1}. \]
  Moreover by the earlier claim we have
  \[ \left\lvert a_{n+1}-a_1 \right\rvert < n+1. \]
  From this we deduce $a_{n+1} \in \{k, k+n+1\}$ as desired.
\end{proof}

This gives us actually a complete description
of all possible sequences satisfying the hypothesis:
choose any value of $a_1$ to start.
Then, for the $n$th term,
the set $S = \{a_1, \dots, a_{n-1}\}$
is (in some order) a set of $n-1$ consecutive integers.
We then let $a_n = \max S + 1$ or $a_n = \min S - 1$.
A picture of six possible starting terms is shown below.

\begin{center}
\begin{asy}
size(6cm);
defaultpen(fontsize(9pt));
int[] a = {0, 6, 5, 7, 4, 3, 8};
for (int i=1; i<=9; ++i) {
  draw((0,i)--(7.2,i), grey+dotted);
}
for (int i=1; i<=7; ++i) {
  draw((i,0)--(i,9.2), grey+dotted);
  if (i <= 5) draw((i,a[i])--(i+1,a[i+1]), red+dashed);
  if (i <= 6) dot("$"+(string) a[i] + "$", (i, a[i]), dir(15), red);
  if (i <= 6) label("$a_{" + (string) i + "}$", (i, 0), dir(-90));
}
draw( (-0.2,0)--(7.2,0), black );
draw( (0,-0.2)--(0,9.2), black );
\end{asy}
\end{center}


Finally, we observe that the condition that
the sequence has infinitely many positive and negative terms
(which we have not used until now)
implies it is unbounded above and below.
Thus it must contain every integer.
\pagebreak

\subsection{IMO 2005/3}
\textsl{Available online at \url{https://aops.com/community/p281573}.}
\begin{mdframed}[style=mdpurplebox,frametitle={Problem statement}]
Let $x,y,z > 0$ satisfy $xyz\geq 1$. Prove that
\[ \frac { x^5-x^2 }{x^5+y^2+z^2}
  + \frac {y^5-y^2}{x^2+y^5+z^2}
  + \frac {z^5-z^2}{x^2+y^2+z^5} \geq 0. \]
\end{mdframed}
Negating both sides and adding $3$ eliminates the minus signs:
\[ \sum_{\text{cyc}} \frac{1}{x^5+y^2+z^2}
  \le \frac{3}{x^2+y^2+z^2}. \]
Thus we only need to consider the case $xyz = 1$.

Direct expansion and Muirhead works now!
As advertised, once we show it suffices to analyze if $xyz=1$
the inequality becomes more economically written as
\[ S = \sum_{\text{cyc}} x^2(x^2-yz)(y^4+x^3z+xz^3)(z^4+x^3y+xy^3)
  \overset{?}{\ge} 0. \]
So, clearing all the denominators gives
\begin{align*}
  S &= \sum_{\text{cyc}} x^2(x^2-yz)
  \left[ y^4z^4 + x^3y^5 + xy^7 + x^3z^5 + x^6yz + x^4y^3z
    + xz^7 + x^4yz^3 + x^2y^3z^3 \right] \\
  &= \sum_{\text{cyc}}
  \left[ x^4y^4z^4 + x^7y^5 + x^5y^7 + x^7z^5 + x^{10}yz + x^8y^3z
    + x^5z^7 + x^8yz^3 + x^6y^3z^3 \right] \\
  &- \sum_{\text{cyc}}
  \left[ x^2y^5z^5 + x^5y^6z + x^3y^8z + x^5yz^6 + x^8y^2z^2 + x^6y^4z^2
    + x^3yz^8 + x^6y^2z^4 + x^4y^4z^4 \right] \\
  &= \sum_{\text{cyc}}
  \left[ x^7y^5 + x^5y^7 + x^7z^5 + x^{10}yz
    + x^5z^7 + x^6y^3z^3 \right] \\
  &- \sum_{\text{cyc}}
  \left[ x^2y^5z^5 + x^5y^6z + x^5yz^6 + x^8y^2z^2 + x^6y^4z^2
    + x^6y^2z^4 \right]
\end{align*}
In other words we need to show
\[
  \sum_{\text{sym}} \left( 2x^7y^5
  +\half x^{10}yz + \half x^6y^3z^3 \right)
  \ge
  \sum_{\text{sym}} \left( \half x^8 y^2 z^2
  + \half x^5 y^5 z^2 + x^6 y^4 z^2 + x^6 y^5 z \right).
\]
which follows by summing
\begin{align*}
  \sum_{\text{sym}} \frac{x^{10} yz + x^6 y^3 z^3}{2}
    &\ge \sum_{\text{sym}} x^8 y^2 z^2 \\
  \half \sum_{\text{sym}} x^8 y^2 z^2
    &\ge \half \sum_{\text{sym}} x^6 y^4 z^2 \\
  \half \sum_{\text{sym}} x^7y^5
    &\ge \half \sum_{\text{sym}} x^5 y^5 z^2 \\
  \half \sum_{\text{sym}} x^7y^5
    &\ge \half \sum_{\text{sym}} x^6 y^4 z^2 \\
  \sum_{\text{sym}} x^7y^5
    &\ge \sum_{\text{sym}} x^6 y^5 z.
\end{align*}
The first line here comes from AM-GM,
the rest come from Muirhead.
\begin{remark*}
More elegant approach is to use Cauchy in the form
\[ \frac{1}{x^5+y^2+z^2} \le \frac{x\inv+y^2+z^2}{(x^2+y^2+z^2)^2}. \]
\end{remark*}
\pagebreak

\section{Solutions to Day 2}
\subsection{IMO 2005/4}
\textsl{Available online at \url{https://aops.com/community/p282138}.}
\begin{mdframed}[style=mdpurplebox,frametitle={Problem statement}]
Determine all positive integers relatively
prime to all the terms of the infinite sequence
\[ a_n = 2^n+3^n+6^n-1, \quad n \ge 1. \]
\end{mdframed}
The answer is $1$ only (which works).

It suffices to show there are no primes.
For the primes $p=2$ and $p=3$, take $a_2=48$.
For any prime $p \ge 5$ notice that
\begin{align*}
  a_{p-2} &= 2^{p-2} + 3^{p-2} + 6^{p-2} - 1 \\
  &\equiv \frac 12 + \frac 13 + \frac 16 - 1 \pmod p \\
  &\equiv 0 \pmod p
\end{align*}
so no other larger prime works.
\pagebreak

\subsection{IMO 2005/5}
\textsl{Available online at \url{https://aops.com/community/p282140}.}
\begin{mdframed}[style=mdpurplebox,frametitle={Problem statement}]
Let $ABCD$ be a fixed convex quadrilateral
with $BC=DA$ and $\ol{BC} \nparallel \ol{DA}$.
Let two variable points $E$ and $F$ lie on the
sides $BC$ and $DA$, respectively, and satisfy $BE=DF$.
The lines $AC$ and $BD$ meet at $P$,
the lines $BD$ and $EF$ meet at $Q$,
the lines $EF$ and $AC$ meet at $R$.
Prove that the circumcircles of the triangles $PQR$,
as $E$ and $F$ vary, have a common point other than $P$.
\end{mdframed}
Let $M$ be the Miquel point of complete quadrilateral $ADBC$;
in other words, let $M$ be the second intersection point
of the circumcircles of $\triangle APD$ and $\triangle BPC$.
(A good diagram should betray this secret;
all the points are given in the picture.)
This makes lots of sense since we know $E$ and $F$
will be sent to each other under the spiral similarity too.

\begin{center}
\begin{asy}
size(10cm);
pair A = dir(120);
pair D = dir(210);
pair C = dir(330);
pair B = dir(80);
filldraw(A--D--B--C--cycle, opacity(0.2)+lightcyan, blue+1.2);

pair E = 0.8*B+0.2*C;
pair F = 0.8*D+0.2*A;

pair P = extension(A, C, B, D);
pair Q = extension(E, F, B, D);
pair R = extension(E, F, A, C);

draw(E--F, lightblue);

pair M = (A*B-C*D)/(A+B-C-D);

filldraw(circumcircle(P, Q, R), opacity(0.2)+yellow, red);
draw(circumcircle(A, P, D), lightblue);
draw(circumcircle(B, P, C), lightblue);
filldraw(circumcircle(F, A, R), opacity(0.1)+lightgreen, dotted+heavygreen);
filldraw(circumcircle(F, D, Q), opacity(0.1)+lightgreen, dotted+heavygreen);
filldraw(circumcircle(E, B, Q), opacity(0.1)+lightgreen, dotted+heavygreen);
filldraw(circumcircle(E, C, R), opacity(0.1)+lightgreen, dotted+heavygreen);

dot("$A$", A, dir(A));
dot("$D$", D, dir(240));
dot("$C$", C, dir(C));
dot("$B$", B, dir(B));
dot("$E$", E, dir(E));
dot("$F$", F, dir(F));
dot("$P$", P, 1.8*dir(95));
dot("$Q$", Q, dir(Q));
dot("$R$", R, dir(R));
dot("$M$", M, 1.4*dir(-30));

/* TSQ Source:

!size(10cm);
A = dir 120
D = dir 210 R240
C = dir 330
B = dir 80
A--D--B--C--cycle 0.2 lightcyan / blue+1.2

E = 0.8*B+0.2*C
F = 0.8*D+0.2*A

P = extension A C B D 1.8R95
Q = extension E F B D
R = extension E F A C

E--F lightblue

M = (A*B-C*D)/(A+B-C-D) 1.4R-30

circumcircle P Q R 0.2 yellow / red
circumcircle A P D lightblue
circumcircle B P C lightblue
circumcircle F A R 0.1 lightgreen / dotted heavygreen
circumcircle F D Q 0.1 lightgreen / dotted heavygreen
circumcircle E B Q 0.1 lightgreen / dotted heavygreen
circumcircle E C R 0.1 lightgreen / dotted heavygreen

*/
\end{asy}
\end{center}

Thus $M$ is the Miquel point of complete quadrilateral $FACE$.
As $R = \ol{FE} \cap \ol{AC}$ we deduce $FARM$ is a cyclic quadrilateral
(among many others, but we'll only need one).

Now look at complete quadrilateral $AFQP$.
Since $M$ lies on $(DFQ)$ and $(RAF)$,
it follows that $M$ is in fact the Miquel point of $AFQP$ as well.
So $M$ lies on $(PQR)$.

Thus $M$ is the fixed point that we wanted.

\begin{remark*}
  Naturally, the congruent length
  condition can be relaxed to $DF/DA = BE/BC$.
\end{remark*}
\pagebreak

\subsection{IMO 2005/6, proposed by Radu Gologan, Dan Schwartz}
\textsl{Available online at \url{https://aops.com/community/p282141}.}
\begin{mdframed}[style=mdpurplebox,frametitle={Problem statement}]
In a mathematical competition $6$ problems were posed to the contestants.
Each pair of problems was solved by more than $\frac{2}{5}$ of the contestants.
Nobody solved all 6 problems.
Show that there were at least $2$ contestants
who each solved exactly $5$ problems.
\end{mdframed}
Assume not and at most one contestant solved five problems.
By adding in solves,
we can assume WLOG that one contestant solved problems one through five,
and every other contestant solved four of the six problems.

We split the remaining contestants based on whether they solved P6.
Let $a_i$ denote the number of contestants who solved
$\{1,2,\dots,5\} \setminus \{i\}$ (and missed P6).
Let $b_{ij}$ denote the number of contestants who solved
$\{1,2,\dots,5,6\} \setminus \{i,j\}$, for $1 \le i < j \le 5$
(thus in particular they solved P6).
Thus
\[ n = 1 + \sum_{1 \le i \le 5} a_i + \sum_{1 \le i < j \le 5} b_{ij} \]
denotes the total number of contestants.

Considering contestants who solved P1/P6 we have
\[ t_1 \coloneqq b_{23} + b_{24} + b_{25} + b_{34} + b_{35} + b_{45} \ge \frac25n + \frac15 \]
and we similarly define $t_2$, $t_3$, $t_4$, $t_5$.
(We have written $\frac25n+\frac15$ since we know
the left-hand side is an integer strictly larger than $\frac25n$.)
Also, by considering contestants who solved P1/P2 we have
\[ t_{12} = 1 + a_{3} + a_{4} + a_{5} + b_{34} + b_{35} + b_{45}
  \ge \frac25n + \frac15 \]
and we similarly define $t_{ij}$ for $1 \le i < j \le 5$.

\begin{claim*}
  The number $\frac{2n+1}{5}$ is equal to some integer $k$,
  fourteen of the $t$'s are equal to $k$,
  and the last one is equal to $k+1$.
\end{claim*}
\begin{proof}
First, summing all fifteen equations gives
  \begin{align*}
    6n+4 = 10 + 6(n-1) &= 10
      + \sum_{1 \le i \le 5} 6a_i + \sum_{1 \le i < j \le 5} 6b_{ij} \\
      &= \sum_{1 \le i \le 5} t_i + \sum_{1 \le i < j \le 5} t_{ij}.
  \end{align*}
  Thus the sum of the $15$ $t$'s is $6n+4$.
  But since all the $t$'s are integers at least
  $\frac{2n+1}{5} = \frac{6n+3}{15}$, the conclusion follows.
\end{proof}

However, we will also manipulate the equations to get the following.
\begin{claim*}
  We have
  \[ t_{45} \equiv 1 + t_1 + t_2 + t_3 + t_{12} + t_{23} + t_{31} \pmod 3. \]
\end{claim*}
\begin{proof}
  This follows directly by computing the coefficient of the $a$'s and $b$'s.
  We will nonetheless write out a derivation of this equation, to
  motivate it, but the proof stands without it.

  Let $B = \sum_{1 \le i < j \le 5} b_{ij}$ be the sum of all $b$'s.
  First, note that
  \begin{align*}
    t_1 + t_2 &= B + b_{34} + b_{45} + b_{35} - b_{12} \\
    &= B + \left( t_{12}-1-a_3-a_4-a_5 \right) - b_{12} \\
    \implies b_{12} &= B - (t_1 + t_2) + t_{12} - 1 - (a_3+a_4+a_5).
  \end{align*}
  This means we have more or less solved for each $b_{ij}$
  in terms of only $t$ and $a$ variables.
  Now
  \begin{align*}
    t_{45} &= 1 + a_1 + a_2 + a_3 + b_{12} + b_{23} + b_{31} \\
    &= 1 + a_1 + a_2 + a_3 \\
      &+ [B - (t_1 + t_2) + t_{12} - 1 - (a_3+a_4+a_5)] \\
      &+ [B - (t_2 + t_3) + t_{23} - 1 - (a_1+a_4+a_5)] \\
      &+ [B - (t_3 + t_1) + t_{13} - 1 - (a_2+a_4+a_5)] \\
    &\equiv 1 + t_1 + t_2 + t_3 + t_{12} + t_{23} + t_{31} \pmod 3
  \end{align*}
  as desired.
\end{proof}

However, we now show the two claims are incompatible
(and this is easy, many ways to do this).
There are two cases.
\begin{itemize}
  \ii Say $t_5 = k+1$ and the others are $k$.
  Then the equation for $t_{45}$ gives that $k \equiv 6k+1 \pmod 3$.
  But now the equation for $t_{12}$ give $k \equiv 6k \pmod 3$.
  \ii Say $t_{45} = k+1$ and the others are $k$.
  Then the equation for $t_{45}$ gives that $k+1 \equiv 6k \pmod 3$.
  But now the equation for $t_{12}$ give $k \equiv 6k+1 \pmod 3$.
\end{itemize}


\begin{remark*}
  It is significantly easier to prove that there
  is at least one contestant who solved five problems.
  One can see it by dropping the $+10$ in the proof of the claim,
  and arrives at a contradiction.
  In this situation it is not even necessary to set up
  the many $a$ and $b$ variables;
  just note that the expected number of contestants
  solving any particular pair of problems is
  $\frac{\binom42n}{\binom62} = \frac25n$.

  The fact that $\frac{2n+1}{5}$
  should be an integer also follows quickly,
  since if not one can improve the bound to $\frac{2n+2}{5}$
  and quickly run into a contradiction.
  Again one can get here without setting up $a$ and $b$.

  The main difficulty seems to be the precision required
  in order to nail down the second $5$-problem solve.
\end{remark*}

\begin{remark*}
  The second claim may look miraculous,
  but the proof shows that it is not too unnatural
  to consider $t_1 + t_2 - t_{12}$ to isolate $b_{12}$
  in terms of $a$'s and $t$'s.
  The main trick is: why mod $3$?

  The reason is that if one looks closely, for a
  fixed $k$ we have a system of $15$ equations in $15$ variables.
  Unless the determinant $D$ of that system happens to be zero,
  this means there will be a rational solution in $a$ and $b$,
  whose denominators are bounded by $D$.
  However if $p \mid D$ then we may conceivably run into mod $p$
  issues.

  This motivates the choice $p = 3$,
  since it is easy to see the determinant is divisible by $3$,
  since constant shifts of $\vec a$ and $\vec b$ are also solutions mod $3$.
  (The choice $p = 2$ is a possible guess as well for this reason,
  but the problem seems to have better $3$-symmetry.)
\end{remark*}
\pagebreak


\end{document}
