% © Evan Chen
% Downloaded from https://web.evanchen.cc/

\documentclass[11pt]{scrartcl}
\usepackage[sexy]{evan}
\ihead{\footnotesize\textbf{\thetitle}}
\ohead{\footnotesize\href{http://web.evanchen.cc}{\ttfamily web.evanchen.cc},
    updated \today}
\title{JMO 2021 Solution Notes}
\date{\today}

\begin{document}

\maketitle

\begin{abstract}
This is a compilation of solutions
for the 2021 JMO.
Some of the solutions are my own work,
but many are from the official solutions provided by the organizers
(for which they hold any copyrights),
and others were found by users on the Art of Problem Solving forums.

These notes will tend to be a bit more advanced and terse than the ``official''
solutions from the organizers.
In particular, if a theorem or technique is not known to beginners
but is still considered ``standard'', then I often prefer to
use this theory anyways, rather than try to work around or conceal it.
For example, in geometry problems I typically use directed angles
without further comment, rather than awkwardly work around configuration issues.
Similarly, sentences like ``let $\mathbb{R}$ denote the set of real numbers''
are typically omitted entirely.

Corrections and comments are welcome!
\end{abstract}

\tableofcontents
\newpage

\addtocounter{section}{-1}
\section{Problems}
\begin{enumerate}[\bfseries 1.]
\item %% Problem 1
Find all functions $f \colon \NN \to \NN$
which satisfy $f(a^2+b^2)=f(a)f(b)$ and
$f(a^2)=f(a)^2$ for all positive integers $a$ and $b$.

\item %% Problem 2
Rectangles $BCC_1B_2$, $CAA_1C_2$, and $ABB_1A_2$ are erected
outside an acute triangle $ABC$. Suppose that
\[ \angle BC_1C + \angle CA_1A + \angle AB_1B = 180^\circ. \]
Prove that lines $B_1C_2$, $C_1A_2$, and $A_1B_2$ are concurrent.

\item %% Problem 3
An equilateral triangle $\Delta$ of side length $L > 0$ is given.
Suppose that $n$ equilateral triangles with side length $1$
and with non-overlapping interiors are drawn inside $\Delta$,
such that each unit equilateral triangle has sides parallel to $\Delta$,
but with opposite orientation.
Prove that \[ n \le \frac{2}{3} L^2. \]

\item %% Problem 4
Carina has three pins, labeled $A$, $B$, and $C$, respectively,
located at the origin of the coordinate plane.
In a \emph{move}, Carina may move a pin to
an adjacent lattice point at distance $1$ away.
What is the least number of moves that Carina can make
in order for triangle $ABC$ to have area $2021$?

\item %% Problem 5
A finite set $S$ of positive integers has the property that,
for each $s\in S$, and each positive integer divisor $d$ of $s$,
there exists a unique element $t\in S$ satisfying $\gcd(s,t) = d$.
(The elements $s$ and $t$ could be equal.)

Given this information, find all possible values for the
number of elements of $S$.

\item %% Problem 6
Let $n \ge 4$ be an integer.
Find all positive real solutions to the following
system of $2n$ equations:
\begin{align*}
  a_1 &= \frac{1}{a_{2n}} + \frac{1}{a_{2}}, & a_2 &= a_1 + a_3, \\[1ex]
  a_3 &= \frac{1}{a_{2}} + \frac{1}{a_{4}}, & a_4 &= a_3 + a_5, \\[1ex]
  a_5 &= \frac{1}{a_{4}} + \frac{1}{a_{6}}, & a_6 &= a_5 + a_7, \\[1ex]
  &\vdotswithin{=} &&\vdotswithin{=} \\
  a_{2n-1} &= \frac{1}{a_{2n-2}} + \frac{1}{a_{2n}}, & a_{2n} &= a_{2n-1} + a_1.
\end{align*}

\end{enumerate}
\pagebreak

\section{Solutions to Day 1}
\subsection{JMO 2021/1, proposed by Vincent Huang}
\textsl{Available online at \url{https://aops.com/community/p21498724}.}
\begin{mdframed}[style=mdpurplebox,frametitle={Problem statement}]
Find all functions $f \colon \NN \to \NN$
which satisfy $f(a^2+b^2)=f(a)f(b)$ and
$f(a^2)=f(a)^2$ for all positive integers $a$ and $b$.
\end{mdframed}
The answer is $f \equiv 1$ only, which works.
We prove it's the only one.

The bulk of the problem is:
\begin{claim*}
  If $f(a)=f(b)=1$ and $a>b$, then $f(a^2-b^2)=f(2ab)=1$.
\end{claim*}
\begin{proof}
  Write
  \begin{align*}
    1 = f(a)f(b) &= f(a^2+b^2) = \sqrt{f\left( (a^2+b^2)^2 \right)} \\
    &= \sqrt{f\left( (a^2-b^2)^2 + (2ab)^2 \right)} \\
    &= \sqrt{f(a^2-b^2) f(2ab)}. \qedhere
  \end{align*}
\end{proof}

By setting $a=b=1$ in the given statement
we get $f(1) = f(2) = 1$.
Now a simple induction on $n$ shows $f(n) = 1$:
\begin{itemize}
  \ii If $n = 2k$ take $(u,v) = (k,1)$ hence $2uv = n$.
  \ii If $n = 2k+1$ take $(u,v) = (k+1,k)$ hence $u^2-v^2=n$.
\end{itemize}
\pagebreak

\subsection{JMO 2021/2, proposed by Ankan Bhattacharya}
\textsl{Available online at \url{https://aops.com/community/p21498558}.}
\begin{mdframed}[style=mdpurplebox,frametitle={Problem statement}]
Rectangles $BCC_1B_2$, $CAA_1C_2$, and $ABB_1A_2$ are erected
outside an acute triangle $ABC$. Suppose that
\[ \angle BC_1C + \angle CA_1A + \angle AB_1B = 180^\circ. \]
Prove that lines $B_1C_2$, $C_1A_2$, and $A_1B_2$ are concurrent.
\end{mdframed}
The angle condition implies the circumcircles of the three
rectangles concur at a single point $P$.
Then $\dang C P B_2 = \dang C P A_1 = 90\dg$,
hence $P$ lies on $A_1 B_2$ etc., so we're done.

\begin{remark*}
  As one might guess from the two-sentence solution,
  the entire difficulty of the problem
  is getting the characterization of the concurrence point.
\end{remark*}
\pagebreak

\subsection{JMO 2021/3, proposed by Alex Zhai}
\textsl{Available online at \url{https://aops.com/community/p21499596}.}
\begin{mdframed}[style=mdpurplebox,frametitle={Problem statement}]
An equilateral triangle $\Delta$ of side length $L > 0$ is given.
Suppose that $n$ equilateral triangles with side length $1$
and with non-overlapping interiors are drawn inside $\Delta$,
such that each unit equilateral triangle has sides parallel to $\Delta$,
but with opposite orientation.
Prove that \[ n \le \frac{2}{3} L^2. \]
\end{mdframed}
We present the approach of Andrew Gu.
For each triangle, we draw a green regular hexagon
of side length $1/2$ as shown below.

\begin{center}
\begin{asy}
  size(2cm);
  filldraw(dir(0)--dir(180)--(dir(240)+dir(300))--cycle,
    opacity(0.2)+lightred, black);
  filldraw(dir(0)--dir(60)--dir(120)--dir(180)--dir(240)--dir(300)--cycle,
  opacity(0.1)+green, deepgreen+1.5);
\end{asy}
\end{center}

\begin{claim*}
  All the hexagons are disjoint and lie inside $\Delta$.
\end{claim*}
\begin{proof}
  Annoying casework.
\end{proof}

Since each hexagon has area $\frac{3\sqrt{3}}{8}$ and
lies inside $\Delta$, we conclude
\[ \frac{3\sqrt3}{8} \cdot n \le \frac{\sqrt3}{4} L^2
  \implies n \le \frac 23 L^2. \]

\begin{remark*}
  The constant $\frac23$ is sharp and cannot be improved.
  The following tessellation shows how to achieve the $\frac23$ density.
  In the figure on the left,
  one of the green hexagons is drawn in for illustration.
  The version on the right has all the hexagons.
  \begin{center}
  \begin{asy}
    size(11cm);
    picture pic;
    filldraw(pic, dir(0)--dir(180)--(dir(240)+dir(300))--cycle,
      opacity(0.2)+lightred, black);
    pair v = dir(180)+dir(240);
    pair w = dir(0)+dir(-60);

    for (int i=0; i<=7; ++i) {
      for (int j=0; j<=7; ++j) {
        if (abs(i-j) <= 4) {
          add(shift(i*v+j*w)*pic);
        }
      }
    }

    filldraw(pic, (dir(0)--dir(60)--dir(120)--dir(180)--dir(240)--dir(300)--cycle),
      opacity(0.1)+lightgreen, deepgreen+1.5);

    for (int i=0; i<=7; ++i) {
      for (int j=0; j<=7; ++j) {
        if (abs(i-j) <= 4) {
          add(shift(15,0)*shift(i*v+j*w)*pic);
        }
      }
    }
    filldraw(shift(0,-4*3**0.5)*(dir(0)--dir(60)--dir(120)--dir(180)--dir(240)--dir(300)--cycle),
      opacity(0.1)+lightgreen, deepgreen+1.5);
  \end{asy}
  \end{center}
\end{remark*}
\pagebreak

\section{Solutions to Day 2}
\subsection{JMO 2021/4, proposed by Brandon Wang}
\textsl{Available online at \url{https://aops.com/community/p21498566}.}
\begin{mdframed}[style=mdpurplebox,frametitle={Problem statement}]
Carina has three pins, labeled $A$, $B$, and $C$, respectively,
located at the origin of the coordinate plane.
In a \emph{move}, Carina may move a pin to
an adjacent lattice point at distance $1$ away.
What is the least number of moves that Carina can make
in order for triangle $ABC$ to have area $2021$?
\end{mdframed}
The answer is $128$.

Define the \textbf{bounding box} of triangle $ABC$
to be the smallest axis-parallel rectangle which
contains all three of the vertices $A$, $B$, $C$.

\begin{center}
\begin{asy}
  pair A = (0,0);
  pair X = (7,0);
  pair Z = (0,3);
  pair Y = X+Z-A;
  pair B = (7,1);
  pair C = (2,3);
  filldraw(A--B--C--cycle, opacity(0.1)+lightcyan, blue);
  filldraw(A--X--Y--Z--cycle, opacity(0.1)+lightred, red);
  dot("$A$", A, dir(225), blue);
  dot("$B$", B, dir(0), blue);
  dot("$C$", C, dir(90), blue);
  dot("$X$", X, dir(315), red);
  dot("$Y$", Y, dir(45), red);
  dot("$Z$", Z, dir(135), red);
\end{asy}
\end{center}

\begin{lemma*}
  The area of a triangle $ABC$ is at
  most half the area of the bounding box.
\end{lemma*}
\begin{proof}
  This can be proven by explicit calculation in coordinates.
  Nonetheless, we outline a geometric approach.
  By considering the smallest/largest $x$ coordinate
  and the smallest/largest $y$ coordinate,
  one can check that some vertex of the triangle
  must coincide with a corner of the bounding box
  (there are four ``extreme'' coordinates
  across the $3\cdot2=6$ coordinates of our three points).

  So, suppose the bounding box is $AXYZ$.
  Imagine fixing $C$ and varying $B$ along
  the perimeter entire rectangle.
  The area is a linear function of $B$,
  so the maximal area should be achieved when $B$
  coincides with one of the vertices $\{A,X,Y,Z\}$.
  But obviously the area of $\triangle ABC$ is
  \begin{itemize}
    \ii exactly $0$ if $B=A$,
    \ii at most half the bounding box if $B\in\{X,Z\}$
    by one-half-base-height,
    \ii at most half the bounding box if $B=Y$,
    since $\triangle ABC$ is contained inside
    either $\triangle AYZ$ or $\triangle AXZ$. \qedhere
  \end{itemize}
\end{proof}

We now proceed to the main part of the proof.
\begin{claim*}
  If $n$ moves are made, the bounding box has area at most $(n/2)^2$.
  (In other words, a bounding box of area $A$ requires
  at least $\left\lceil 2\sqrt{A} \right\rceil$ moves.)
\end{claim*}
\begin{proof}
  The sum of the width and height of the bounding box
  increases by at most $1$ each move,
  hence the width and height have sum at most $n$.
  So, by AM-GM, their product is at most $(n/2)^2$.
\end{proof}

This immediately implies $n \ge 128$,
since the bounding box needs to have area at least $4042 > 63.5^2$.

On the other hand, if we start all the pins at
the point $(3,18)$ then we can reach the
following three points in $128$ moves:
\begin{align*}
  A &= (0,0) \\
  B &= (64,18) \\
  C &= (3,64)
\end{align*}
and indeed triangle $ABC$ has area exactly $2021$.
\pagebreak

\subsection{JMO 2021/5, proposed by Carl Schildkraut}
\textsl{Available online at \url{https://aops.com/community/p21498580}.}
\begin{mdframed}[style=mdpurplebox,frametitle={Problem statement}]
A finite set $S$ of positive integers has the property that,
for each $s\in S$, and each positive integer divisor $d$ of $s$,
there exists a unique element $t\in S$ satisfying $\gcd(s,t) = d$.
(The elements $s$ and $t$ could be equal.)

Given this information, find all possible values for the
number of elements of $S$.
\end{mdframed}
The answer is that $|S|$ must be a power of $2$ (including $1$),
or $|S| = 0$ (a trivial case we do not discuss further).

\medskip

\textbf{Construction}:
For any nonnegative integer $k$,
a construction for $|S| = 2^k$ is given
by
\[ S = \left\{
    (p_1 \text{ or } q_1)
    \times
    (p_2 \text{ or } q_2)
    \times
    \dots
    \times
    (p_k \text{ or } q_k)
  \right\}
\]
for $2k$ distinct primes $p_1$, \dots, $p_k$, $q_1$, \dots, $q_k$.

\medskip

\textbf{Converse}: the main claim is as follows.
\begin{claim*}
  In any valid set $S$,
  for any prime $p$ and $x \in S$, $\nu_p(x) \le 1$.
\end{claim*}
\begin{proof}
  Assume for contradiction $e = \nu_p(x) \ge 2$.
  \begin{itemize}
    \ii On the one hand, by taking $x$ in the statement,
    we see $\frac{e}{e+1}$ of the elements of $S$ are divisible by $p$.
    \ii On the other hand, consider a $y \in S$ such that
    $\nu_p(y)=1$ which must exist (say if $\gcd(x,y) = p$).
    Taking $y$ in the statement,
    we see $\half$ of the elements of $S$ are divisible by $p$.
  \end{itemize}
  So $e=1$, contradiction.
\end{proof}
Now since $|S|$ equals the number of divisors of any element of $S$,
we are done.
\pagebreak

\subsection{JMO 2021/6, proposed by Mohsen Jamaali}
\textsl{Available online at \url{https://aops.com/community/p21498967}.}
\begin{mdframed}[style=mdpurplebox,frametitle={Problem statement}]
Let $n \ge 4$ be an integer.
Find all positive real solutions to the following
system of $2n$ equations:
\begin{align*}
  a_1 &= \frac{1}{a_{2n}} + \frac{1}{a_{2}}, & a_2 &= a_1 + a_3, \\[1ex]
  a_3 &= \frac{1}{a_{2}} + \frac{1}{a_{4}}, & a_4 &= a_3 + a_5, \\[1ex]
  a_5 &= \frac{1}{a_{4}} + \frac{1}{a_{6}}, & a_6 &= a_5 + a_7, \\[1ex]
  &\vdotswithin{=} &&\vdotswithin{=} \\
  a_{2n-1} &= \frac{1}{a_{2n-2}} + \frac{1}{a_{2n}}, & a_{2n} &= a_{2n-1} + a_1.
\end{align*}
\end{mdframed}
The answer is that the only solution is
$(1,2,1,2,\dots,1,2)$ which works.

We will prove $a_{2k}$ is a constant sequence,
at which point the result is obvious.

\paragraph{First approach (Andrew Gu).}
Apparently, with indices modulo $2n$, we should have
\[ a_{2k} = \frac{1}{a_{2k-2}}
  + \frac{2}{a_{2k}} + \frac{1}{a_{2k+2}} \]
for every index $k$ (this eliminates all $a_{\text{odd}}$'s).
Define
\[ m = \min_k a_{2k} \qquad\text{and}\qquad M = \max_k a_{2k}. \]
Look at the indices $i$ and $j$
achieving $m$ and $M$ to respectively get
\begin{align*}
  m &= \frac2m + \frac{1}{a_{2i-2}} + \frac{1}{a_{2i+2}}
  \ge \frac2m + \frac1M + \frac1M  = \frac2m + \frac2M \\[1ex]
  M &= \frac2M + \frac{1}{a_{2j-2}} + \frac{1}{a_{2j+2}}
  \le \frac2M + \frac1m + \frac1m = \frac2m + \frac2M.
\end{align*}
Together this gives $m \ge M$, so $m = M$.
That means $a_{2i}$ is constant as $i$ varies,
solving the problem.

\paragraph{Second approach (author's solution).}
As before, we have
\[ a_{2k} = \frac{1}{a_{2k-2}}
  + \frac{2}{a_{2k}} + \frac{1}{a_{2k+2}} \]
The proof proceeds in three steps.
\begin{itemize}
\ii Define
\[ S = \sum_k a_{2k},
  \quad\text{and}\quad
  T = \sum_k \frac{1}{a_{2k}}. \]
Summing gives $S = 4T$.
On the other hand, Cauchy-Schwarz says $S \cdot T \ge n^2$,
so $T \ge \half n$.

\ii On the other hand,
\[ 1 = \frac{1}{a_{2k-2} a_{2k}}
  + \frac{2}{a_{2k}^2} + \frac{1}{a_{2k} a_{2k+2}} \]
Sum this modified statement to obtain
\[
  n = \sum_k \left( \frac{1}{a_{2k}} + \frac{1}{a_{2k+2}} \right)^2
  \overset{\text{QM-AM}}{\ge}
  \frac 1n \left( \sum_k \frac{1}{a_{2k}} + \frac{1}{a_{2k+2}} \right)^2
  = \frac 1n \left( 2T \right)^2
\]
So $T \le \half n$.

\ii Since $T \le \half n$ and $T \ge \half n$,
we must have equality everywhere above.
This means $a_{2k}$ is a constant sequence.
\end{itemize}

\begin{remark*}
  The problem is likely intractable over $\CC$,
  in the sense that one gets a high-degree polynomial
  which almost certainly has many complex roots.
  So it seems likely that most solutions must involve
  some sort of inequality,
  using the fact we are over $\RR_{>0}$ instead.
\end{remark*}
\pagebreak


\end{document}
