\documentclass[11pt]{scrartcl}
\usepackage[sexy]{evan}
\ihead{\footnotesize\textbf{\thetitle}}
\ohead{\footnotesize\theauthor}
\begin{document}

\title{USA TST 2015 Solution Notes}
\date{\today}

\maketitle
\begin{abstract}
This is a compilation of solutions
for the 2015 USA TST.
Some of the solutions are my own work,
but many are from the official solutions provided by the organizers
(for which they hold any copyrights),
and others were found by users on the Art of Problem Solving forums.

These notes will tend to be a bit more advanced and terse than the ``official''
solutions from the organizers.
In particular, if a theorem or technique is not known to beginners
but is still considered ``standard'', then I often prefer to
use this theory anyways, rather than try to work around or conceal it.
For example, in geometry problems I typically use directed angles
without further comment, rather than awkwardly work around configuration issues.
Similarly, sentences like ``let $\mathbb{R}$ denote the set of real numbers''
are typically omitted entirely.

Corrections and comments are welcome!
\end{abstract}
\tableofcontents
\newpage

\addtocounter{section}{-1}
\section{Problems}
\begin{enumerate}[\bfseries 1.]
\item %% Problem 1
Let $ABC$ be a scalene triangle with incenter $I$ whose incircle is
tangent to $\ol{BC}$, $\ol{CA}$, $\ol{AB}$ at $D$, $E$, $F$,
respectively.  Denote by $M$ the midpoint of $\ol{BC}$ and
let $P$ be a point in the interior of $\triangle ABC$
so that $MD = MP$ and $\angle PAB = \angle PAC$.
Let $Q$ be a point on the incircle such that $\angle AQD = 90^{\circ}$.
Prove that either $\angle PQE = 90^{\circ}$ or $\angle PQF = 90^{\circ}$.

\item %% Problem 2
Prove that for every positive integer $n$, there exists a set $S$ of $n$ positive integers
such that for any two distinct $a,b \in S$, $a-b$ divides $a$ and $b$
but none of the other elements of $S$.

\item %% Problem 3
A physicist encounters $2015$ atoms called usamons.
Each usamon either has one electron or zero electrons,
and the physicist can't tell the difference.
The physicist's only tool is a diode.
The physicist may connect the diode from any usamon $A$
to any other usamon $B$.  (This connection is directed.)
When she does so, if usamon $A$ has an electron and usamon $B$
does not, then the electron jumps from $A$ to $B$.
In any other case, nothing happens.  In addition,
the physicist cannot tell whether an electron jumps during any given step.
The physicist's goal is to isolate two usamons that she is 100\%
sure are currently in the same state.
Is there any series of diode usage that makes this possible?

\item %% Problem 4
Let $f \colon \QQ \to \QQ$ be a function such that for any $x,y \in \QQ$,
the number $f(x+y)-f(x)-f(y)$ is an integer.
Decide whether there must exist a constant $c$
such that $f(x) - cx$ is an integer for every rational number $x$.

\item %% Problem 5
Fix a positive integer $n$.
A tournament on $n$ vertices has all its edges colored by $\chi$ colors,
so that any two directed edges $u \to v$
and $v \to w$ have different colors.
Over all possible tournaments on $n$ vertices,
determine the minimum possible value of $\chi$.

\item %% Problem 6
Let $ABC$ be a non-equilateral triangle
and let $M_a$, $M_b$, $M_c$ be the midpoints
of the sides $BC$, $CA$, $AB$, respectively.
Let $S$ be a point lying on the Euler line.
Denote by $X$, $Y$, $Z$ the second intersections
of $M_a S$, $M_b S$, $M_c S$ with the nine-point circle.
Prove that $AX$, $BY$, $CZ$ are concurrent.

\end{enumerate}
\pagebreak

\section{Solutions to Day 1}
\subsection{USA TST 2015/1, proposed by Evan Chen}
\textsl{Available online at \url{https://aops.com/community/p3683109}.}
\begin{mdframed}[style=mdpurplebox,frametitle={Problem statement}]
Let $ABC$ be a scalene triangle with incenter $I$ whose incircle is
tangent to $\ol{BC}$, $\ol{CA}$, $\ol{AB}$ at $D$, $E$, $F$,
respectively.  Denote by $M$ the midpoint of $\ol{BC}$ and
let $P$ be a point in the interior of $\triangle ABC$
so that $MD = MP$ and $\angle PAB = \angle PAC$.
Let $Q$ be a point on the incircle such that $\angle AQD = 90^{\circ}$.
Prove that either $\angle PQE = 90^{\circ}$ or $\angle PQF = 90^{\circ}$.
\end{mdframed}
We present two solutions.

\paragraph{Official solution.}
Assume without loss of generality that $AB < AC$; we show $\angle PQE=90^{\circ}$.
\begin{center}
  \begin{asy}
    size(9cm);

    pair A = dir(130);
    pair B = dir(210);
    pair C = dir(330);
    filldraw(A--B--C--cycle, opacity(0.2)+palecyan, blue);
    pair I = incenter(A, B, C);
    filldraw(incircle(A, B, C), opacity(0.3)+palered, red);
    pair D = foot(I, B, C);
    pair E = foot(I, C, A);
    pair F = foot(I, A, B);

    pair M = midpoint(B--C);
    pair P = extension(A, I, D, E);
    draw(A--P, orange);

    pair D1 = 2*M-D;
    pair Q = foot(D, A, D1);
    draw(P--Q--E, heavyred+1);
    draw(M--Q, red);
    markrightangle(E,Q,P,heavyred+1);

    pair N = midpoint(A--B);
    draw(N--M, dashed+orange);
    draw(D--E, dashed+orange);

    filldraw(CP(M, P), opacity(0.2)+palegreen, heavygreen+dashed);

    pair T = extension(A, Q, B, C);
    pair S = extension(D, I, A, T);
    draw(A--T, heavygreen);
    draw(D--Q, red);
    draw(D--S, dashed+red);

    dot("$A$", A, dir(A));
    dot("$B$", B, dir(B));
    dot("$C$", C, dir(C));
    dot("$I$", I, dir(I));
    dot("$D$", D, dir(D));
    dot("$E$", E, dir(P-D));
    dot("$F$", F, dir(F));
    dot("$M$", M, dir(M));
    dot("$P$", P, dir(200));
    dot("$Q$", Q, dir(25));
    dot("$N$", N, dir(N));
    dot("$T$", T, dir(T));
    dot("$S$", S, dir(70));

    /* Source generated by TSQ

    A = dir 130
    B = dir 210
    C = dir 330
    A--B--C--cycle
    I = incenter A B C
    incircle A B C
    D = foot I B C
    E = foot I C A RP-D
    F = foot I A B

    M = midpoint B--C
    P = extension A I D E R200
    A--P

    D1 := 2*M-D
    Q = foot D A D1 R25
    E--Q--P--M--Q
    !markrightangle(E,Q,P,heavyred+1);

    N = midpoint A--B
    N--M dotted
    D--E dotted

    CP M P dashed

    T = extension A Q B C
    S = extension D I A T R70
    A--T
    D--Q
    D--S dashed

    */
  \end{asy}
\end{center}

First, we claim that $D$, $P$, $E$ are collinear.
Let $N$ be the midpoint of $\ol{AB}$.
It is well-known that the three lines $MN$, $DE$, $AI$ are concurrent at a point
(see for example problem 6 of USAJMO 2014).
Let $P'$ be this intersection point, noting that $P'$ actually lies on segment $DE$.
Then $P'$ lies inside $\triangle ABC$ and moreover
\[ \triangle DP'M \sim \triangle DEC \]
so $MP' = MD$.
Hence $P'=P$, proving the claim.

Let $S$ be the point diametrically opposite $D$ on the incircle,
which is also the second intersection of $\ol{AQ}$ with the incircle.
Let $T = \ol{AQ} \cap \ol{BC}$.
Then $T$ is the contact point of the $A$-excircle;
consequently, \[ MD = MP = MT \] and we obtain a circle with diameter $\ol{DT}$.
Since $\angle DQT = \angle DQS = 90^{\circ}$ we have $Q$ on this circle as well.

As $\ol{SD}$ is tangent to the circle with diameter $\ol{DT}$,
we obtain \[ \angle PQD = \angle SDP = \angle SDE = \angle SQE. \]
Since $\angle DQS = 90^{\circ}$, $\angle PQE = 90^{\circ}$ too.

\paragraph{Solution using spiral similarity.}
We will ignore for now the point $P$.
As before define $S$, $T$ and note $\ol{AQST}$ collinear,
as well as $DPQT$ cyclic on circle $\omega$ with diameter $\ol{DT}$.

Let $\tau$ be the spiral similarity at $Q$
sending $\omega$ to the incircle.
We have $\tau(T) = D$, $\tau(D) = S$, $\tau(Q) = Q$.
Now
\[ I = \ol{DD} \cap \ol{QQ} \implies
\tau(I) = \ol{SS} \cap \ol{QQ} \]
and hence we conclude $\tau(I)$ is the pole of $\ol{ASQT}$
with respect to the incircle, which lies on line $EF$.

Then since $\ol{AI} \perp \ol{EF}$ too,
we deduce $\tau$ sends line $AI$ to line $EF$,
hence $\tau(P)$ must be either $E$ or $F$ as desired.

\paragraph{Authorship comments.}
Written April 2014.
I found this problem while playing with GeoGebra.
Specifically, I started by drawing in the points $A$, $B$, $C$,
$I$, $D$, $M$, $T$, common points.
I decided to add in the circle with diameter $DT$,
because of the synergy it had with the rest of the picture.
After a while of playing around,
I intersected ray $AI$ with the circle to get $P$,
and was surprised to find that $D$, $P$, $E$ were collinear,
which I thought was impossible since the setup should have been symmetric.
On further reflection, I realized it was because $AI$
intersected the circle twice, and set about trying to prove this.
I noticed the relation $\angle PQE = 90\dg$ in my attempts
to prove the result, even though this ended up being a corollary
rather than a useful lemma.
\pagebreak

\subsection{USA TST 2015/2, proposed by Iurie Boreico}
\textsl{Available online at \url{https://aops.com/community/p3683110}.}
\begin{mdframed}[style=mdpurplebox,frametitle={Problem statement}]
Prove that for every positive integer $n$, there exists a set $S$ of $n$ positive integers
such that for any two distinct $a,b \in S$, $a-b$ divides $a$ and $b$
but none of the other elements of $S$.
\end{mdframed}
The idea is to look for a
sequence $d_1, \dots, d_{n-1}$ of ``differences''
such that the following two conditions hold.
Let $s_i = d_1 + \dots + d_{i-1}$,
and $t_{i,j} = d_i + \dots + d_{j-1}$ for $i \le j$.
\begin{enumerate}
\item[(i)] No two of the $t_{i,j}$ divide each other.
\item[(ii)] There exists an integer $a$
   satisfying the CRT equivalences
   \[ a \equiv -s_i \pmod{t_{i,j}} \quad \forall i \le j \]
\end{enumerate}
Then the sequence $a+s_1$, $a+s_2$, \dots, $a+s_n$ will work.
For example, when $n=3$ we can take $(d_1,d_2)=(2,3)$ giving
\[ 10 \overbrace{\underbrace{\qquad}_2 12 \underbrace{\qquad}_3}^5 15 \]
because the only conditions we need satisfy are
\begin{align*}
  a &\equiv 0 \pmod 2 \\
  a &\equiv 0 \pmod 5 \\
  a &\equiv -2 \pmod 3.
\end{align*}

But with this setup we can just construct the $d_i$ inductively.
To go from $n$ to $n+1$, take a $d_1, \dots, d_{n-1}$
and let $p$ be a prime
not dividing any of the $d_i$.
Moreover, let $M$ be a multiple of $\prod_{i \le j} t_{i,j}$ coprime to $p$.
Then we claim that $d_1M, d_2M, \dots, d_{n-1}M, p$
is such a difference sequence.
For example, the previous example extends as follows with $M = 300$ and $p=7$.
\[
   a \overbrace{\underbrace{\qquad}_{600} b
   \overbrace{\underbrace{\qquad}_{900} c
   \underbrace{\qquad}_{7}}^{907}}^{1507} d
\]
The new numbers $p$, $p+Mt_{n-1,n}$, $p+Mt_{n-2,n}$, \dots\
are all relatively prime to everything else.
Hence (i) still holds.
To see that (ii) still holds,
just note that we can still get a family of solutions
for the first $n$ terms, and then the last $(n+1)$st term
can be made to work by Chinese Remainder Theorem
since all the new $p+Mt_{n-2,n}$ are coprime to everything.
\pagebreak

\subsection{USA TST 2015/3, proposed by Linus Hamilton}
\textsl{Available online at \url{https://aops.com/community/p3683111}.}
\begin{mdframed}[style=mdpurplebox,frametitle={Problem statement}]
A physicist encounters $2015$ atoms called usamons.
Each usamon either has one electron or zero electrons,
and the physicist can't tell the difference.
The physicist's only tool is a diode.
The physicist may connect the diode from any usamon $A$
to any other usamon $B$.  (This connection is directed.)
When she does so, if usamon $A$ has an electron and usamon $B$
does not, then the electron jumps from $A$ to $B$.
In any other case, nothing happens.  In addition,
the physicist cannot tell whether an electron jumps during any given step.
The physicist's goal is to isolate two usamons that she is 100\%
sure are currently in the same state.
Is there any series of diode usage that makes this possible?
\end{mdframed}
The answer is no.
Call the usamons $U_1, \dots, U_m$ (here $m=2015$).
Consider models $M_k$ of the following form:
$U_1, \dots, U_k$ are all charged for some $0 \le k \le m$
and the other usamons are not charged.
Note that for any pair there's a model
where they are different states, by construction.

We can consider the physicist as acting
on these $m+1$ models simultaneously,
and trying to reach a state where there's a pair
in all models which are all the same charge.
(This is a necessary condition for a winning strategy to exist.)

But we claim that any diode operation $U_i \to U_j$
results in the $m+1$ models being an isomorphic copy
of the previous set.
If $i < j$ then the diode operation can be interpreted
as just swapping $U_i$ with $U_j$, which doesn't change anything.
Moreover if $i > j$ the operation never does anything.
The conclusion follows from this.

\begin{remark*}
  This problem is not a ``standard'' olympiad problem,
  so I can't say it's trivial.
  But the idea is pretty natural I think.

  You can motivate it as follows:
  there's a sequence of diode operations
  you can do which forces the situation to be one of the $M_k$ above:
  first, use the diode into $U_1$ for all other $U_i$'s,
  so that either no electrons exist at all or $U_1$ has an electron.
  Repeat with the other $U_i$.
  This leaves us at the situation described at the start of the problem.
  Then you could guess the answer was ``no'' just based on the fact
  that it's impossible for $n=2,3$
  and that there doesn't seem to be a reasonable strategy.

  In this way it's possible to give a pretty good description
  of what it's possible to do.
  One possible phrasing:
  ``the physicist can arrange the usamons in a line such that
  all the charged usamons are to the left of the un-charged usamons,
  but can't determine the \emph{number} of charged usamons''.
\end{remark*}
\pagebreak

\section{Solutions to Day 2}
\subsection{USA TST 2015/4, proposed by Victor Wang}
\textsl{Available online at \url{https://aops.com/community/p4628083}.}
\begin{mdframed}[style=mdpurplebox,frametitle={Problem statement}]
Let $f \colon \QQ \to \QQ$ be a function such that for any $x,y \in \QQ$,
the number $f(x+y)-f(x)-f(y)$ is an integer.
Decide whether there must exist a constant $c$
such that $f(x) - cx$ is an integer for every rational number $x$.
\end{mdframed}
No, such a constant need not exist.

One possible solution is as follows:
define a sequence by $x_0 = 1$ and
\begin{align*}
  2x_1 &= x_0 \\
  2x_2 &= x_1 + 1 \\
  2x_3 &= x_2 \\
  2x_4 &= x_3 + 1 \\
  2x_5 &= x_4 \\
  2x_6 &= x_5 + 1 \\
  &\vdotswithin=
\end{align*}
Set $f(2^{-k}) = x_k$ and $f(2^k)=2^k$ for $k=0,1,\dots$.
Then, let
\[ f\left( a \cdot 2^k + \frac bc \right)
  = a f\left( 2^k \right) + \frac bc \]
for odd integers $a$, $b$, $c$.
One can verify this works.

A second shorter solution (given by the proposer) is to set,
whenever $\gcd(p,q) = 1$ and $q > 0$,
\[ f\left( \frac pq \right)
  = \frac pq \left( 1! + 2! + \dots + q! \right). \]


\begin{remark*}
  Silly note: despite appearances,
  $f(x) = \left\lfloor x \right\rfloor$
  is not a counterexample since one can take $c = 0$.
\end{remark*}
\pagebreak

\subsection{USA TST 2015/5, proposed by Po-Shen Loh}
\textsl{Available online at \url{https://aops.com/community/p4628085}.}
\begin{mdframed}[style=mdpurplebox,frametitle={Problem statement}]
Fix a positive integer $n$.
A tournament on $n$ vertices has all its edges colored by $\chi$ colors,
so that any two directed edges $u \to v$
and $v \to w$ have different colors.
Over all possible tournaments on $n$ vertices,
determine the minimum possible value of $\chi$.
\end{mdframed}
The answer is
\[ \chi = \left\lceil \log_2 n \right\rceil. \]

First, we prove by induction on $n$
that $\chi \ge \log_2 n$ for any coloring and any tournament.
The base case $n = 1$ is obvious.
Now given any tournament, consider any used color $c$.
Then it should be possible to divide the tournament
into two subsets $A$ and $B$ such that all $c$-colored edges
point from $A$ to $B$
(for example by letting $A$ be all vertices
which are the starting point of a $c$-edge).
\begin{center}
\begin{asy}
  size(5cm);
  filldraw(box( (-2,-2), (-0.6,2) ), opacity(0.1)+lightcyan, lightblue );
  filldraw(box( (0.6,-2), (2,2) ), opacity(0.1)+lightcyan, lightblue );
  label("$A$", (-2,0), dir(180), blue);
  label("$B$", ( 2,0), dir(  0), blue);
  label("all edges colored $c$", (0,2), dir(90), deepgreen);
  draw( (-1.3,1.2)--(1.5,0.2), deepgreen, EndArrow);
  draw( (-0.9,-0.2)--(1.1, -0.1), deepgreen, EndArrow);
  draw( (-1.4, 0.4)--(0.9, 0.8), deepgreen, EndArrow);
  draw( (-1.4, -0.8)--(1.3, -0.6), deepgreen, EndArrow);
  draw( (-1.3, -0.5)--(1.2, -1.6), deepgreen, EndArrow);
\end{asy}
\end{center}
One of $A$ and $B$ has size at least $n/2$, say $A$.
Since $A$ has no $c$ edges,
and uses at least $\log_2 |A|$ colors other than $c$, we get
\[ \chi \ge 1 + \log_2 (n/2) = \log_2 n \]
completing the induction.

One can read the construction off from the argument above,
but here is a concrete description.
For each integer $n$,
consider the tournament whose vertices are
the binary representations of $S  =\{0, \dots, n-1\}$.
Instantiate colors $c_1$, $c_2$, \dots.
Then for $v, w \in S$,
we look at the smallest order bit for which they differ;
say the $k$th one.
If $v$ has a zero in the $k$th bit,
and $w$ has a one in the $k$th bit,
we draw $v \to w$.
Moreover we color the edge with color $c_k$.
This works and uses at most $\left\lceil \log_2 n \right\rceil$ colors.

\begin{remark*}
  [Motivation]
  The philosophy ``combinatorial optimization'' applies here.
  The idea is given any color $c$,
  we can find sets $A$ and $B$ such that all $c$-edges point $A$ to $B$.
  Once you realize this, the next insight is to realize
  that you may as well color \emph{all} the edges from $A$ to $B$ by $c$;
  after all, this doesn't hurt the condition and makes your life easier.
  Hence, if $f$ is the answer, we have already a proof that
  $f(n) = 1 + \max \left( f(|A|), f(|B|) \right)$
  and we choose $|A| \approx |B|$.
  This optimization also gives the inductive construction.
\end{remark*}
\pagebreak

\subsection{USA TST 2015/6}
\textsl{Available online at \url{https://aops.com/community/p4628087}.}
\begin{mdframed}[style=mdpurplebox,frametitle={Problem statement}]
Let $ABC$ be a non-equilateral triangle
and let $M_a$, $M_b$, $M_c$ be the midpoints
of the sides $BC$, $CA$, $AB$, respectively.
Let $S$ be a point lying on the Euler line.
Denote by $X$, $Y$, $Z$ the second intersections
of $M_a S$, $M_b S$, $M_c S$ with the nine-point circle.
Prove that $AX$, $BY$, $CZ$ are concurrent.
\end{mdframed}
We assume now and forever that $ABC$ is scalene since the problem
follows by symmetry in the isosceles case.
We present four solutions.

\paragraph{First solution by barycentric coordinates (Evan Chen).}
Let $AX$ meet $M_b M_c$ at $D$,
and let $X$ reflected over $M_b M_c$'s midpoint be $X'$.
Let $Y'$, $Z'$, $E$, $F$ be similarly defined.

\begin{center}
\begin{asy}
pair M_a = dir(110);
pair M_b = dir(210);
pair M_c = dir(330);
pair A = M_b+M_c-M_a;
pair B = M_c+M_a-M_b;
pair C = M_a+M_b-M_c;
pair O = circumcenter(A, B, C);
pair S = 0.65*O;
pair N = origin;
pair X = -M_a+2*foot(N,M_a,S);
pair Y = -M_b+2*foot(N,M_b,S);
pair Z = -M_c+2*foot(N,M_c,S);
filldraw(unitcircle, opacity(0.1)+lightcyan, lightblue);
draw(M_a--M_b--M_c--cycle, lightblue);
draw(A--B--C--cycle, lightblue);
draw(X--M_a, cyan);
draw(Y--M_b, cyan);
draw(Z--M_c, cyan);

pair D = extension(X, A, M_b, M_c);
pair Dp = M_b+M_c-D;

pair E = extension(Y, B, M_c, M_a);
pair Ep = M_c+M_a-E;
pair F = extension(Z, C, M_a, M_b);
pair Fp = M_a+M_b-F;

draw(A--D, deepgreen);
draw(B--E, deepgreen);
draw(C--F, deepgreen);

draw(M_a--Dp, lightred);
draw(M_b--Ep, lightred);
draw(M_c--Fp, lightred);

draw(M_a--D, lightred+dotted);
draw(M_b--E, lightred+dotted);
draw(M_c--F, lightred+dotted);

dot("$X$", X, dir(250));
dot("$Y$", Y, dir(80));
dot("$Z$", Z, dir(200));

dot("$M_a$", M_a, dir(M_a));
dot("$M_b$", M_b, dir(M_b));
dot("$M_c$", M_c, dir(M_c));
dot("$A$", A, dir(A));
dot("$B$", B, dir(B));
dot("$C$", C, dir(C));
dot("$S$", S, dir(310));
dot("$D$", D, dir(D));
dot("$D'$", Dp, dir(Dp));
dot("$E$", E, dir(E));
dot("$E'$", Ep, dir(Ep));
dot("$F$", F, dir(220));
dot("$F'$", Fp, dir(Fp));

/* TSQ Source:

M_a = dir 110
M_b = dir 210
M_c = dir 330
A = M_b+M_c-M_a
B = M_c+M_a-M_b
C = M_a+M_b-M_c
O := circumcenter A B C
S = 0.65*O R310
N := origin R45
X := -M_a+2*foot(N,M_a,S)
Y := -M_b+2*foot(N,M_b,S)
Z := -M_c+2*foot(N,M_c,S)
unitcircle 0.1 lightcyan / lightblue
M_a--M_b--M_c--cycle lightblue
A--B--C--cycle lightblue
X--M_a cyan
Y--M_b cyan
Z--M_c cyan

D = extension X A M_b M_c
D' = M_b+M_c-D

E = extension Y B M_c M_a
E' = M_c+M_a-E
F = extension Z C M_a M_b R220
F' = M_a+M_b-F

A--D deepgreen
B--E deepgreen
C--F deepgreen

M_a--Dp lightred
M_b--Ep lightred
M_c--Fp lightred

M_a--D lightred dotted
M_b--E lightred dotted
M_c--F lightred dotted

!dot("$X$", X, dir(250));
!dot("$Y$", Y, dir(80));
!dot("$Z$", Z, dir(200));

*/
\end{asy}
\end{center}

By Cevian Nest Theorem it suffices to
prove that $M_a D$, $M_b E$, $M_c F$ are concurrent.
Taking the isotomic conjugate and recalling that
$M_a M_b A M_c$ is a parallelogram,
we see that it suffices to prove
$M_a X'$, $M_b Y'$, $M_c Z'$ are concurrent.

We now use barycentric coordinates on $\triangle M_a M_b M_c$.
Let \[ S = \left( a^2S_A + t : b^2S_B + t : c^2S_C + t \right) \]
(possibly $t=\infty$ if $S$ is the centroid).
Let $v = b^2S_B+t$, $w=c^2S_C+t$.
Hence \[ X = \left( -a^2 vw : (b^2w+c^2v)v : (b^2w+c^2v)w \right). \]
Consequently,
\[ X' = \left( a^2vw : -a^2vw + (b^2w+c^2v)w : -a^2vw + (b^2w+c^2v)v \right) \]
We can compute
\[ b^2w+c^2v = (bc)^2(S_B+S_C)+ (b^2+c^2)t = (abc)^2 + (b^2+c^2)t. \]
Thus
\[ -a^2v+b^2w+c^2v
  = (b^2+c^2)t + (abc)^2 - (ab)^2S_B - a^2t =
  S_A( (ab)^2 +t ). \]
Finally
\[ X' = \left( a^2vw :
  S_A (c^2S_C+t) \left( (ab)^2 +2t \right) :
  S_A (b^2S_B+t) \left( (ac)^2 +2t \right)
  \right) \]
and from this it's evident that $AX'$, $BY'$, $CZ'$ are concurrent.

\paragraph{Second solution by moving points (Anant Mudgal).}
Let $H_a$, $H_b$, $H_c$ be feet of altitudes,
and let $\gamma$ denote the nine-point circle.
The main claim is that:
\begin{claim*}
  Lines $XH_a$, $YH_b$, $ZH_c$ are concurrent,
\end{claim*}
\begin{proof}
  In fact, we claim that the concurrence point lies on the Euler line $\ell$.
  This gives us a way to apply the moving points method:
  fix triangle $ABC$ and animate $S \in \ell$;
  then the map
  \begin{align*}
    \ell & \to \gamma \to \ell \\
    S &\mapsto X \mapsto S_a \coloneqq \ell \cap \ol{H_a X}
  \end{align*}
  is projective, because it consists of two perspectivities.
  So we want the analogous maps $S \mapsto S_b$, $S \mapsto S_c$ to coincide.
  For this it suffices to check three positions of $S$;
  since you're such a good customer here are four.
  \begin{itemize}
    \ii If $S$ is the orthocenter of $\triangle M_a M_b M_c$
    (equivalently the circumcenter of $\triangle ABC$)
    then $S_a$ coincides with the circumcenter of $M_a M_b M_c$
    (equivalently the nine-point center of $\triangle ABC$).
    By symmetry $S_b$ and $S_c$ are too.

    \ii If $S$ is the circumcenter of $\triangle M_a M_b M_c$
    (equivalently the nine-point center of $\triangle ABC$)
    then $S_a$ coincides with the de Longchamps point of $\triangle M_a M_b M_c$
    (equivalently orthocenter of $\triangle ABC$).
    By symmetry $S_b$ and $S_c$ are too.

    \ii If $S$ is either of the intersections of the Euler
    line with $\gamma$, then $S = S_a = S_b = S_c$
    (as $S = X = Y = Z$).
  \end{itemize}
  This concludes the proof.
\end{proof}

\begin{center}
\begin{asy}
pair M_a = dir(110);
pair M_b = dir(210);
pair M_c = dir(330);
pair A = M_b+M_c-M_a;
pair B = M_c+M_a-M_b;
pair C = M_a+M_b-M_c;
pair O = circumcenter(A, B, C);
pair S = 0.75*O;
pair N = origin;

pair X = -M_a+2*foot(N,M_a,S);
pair Y = -M_b+2*foot(N,M_b,S);
pair Z = -M_c+2*foot(N,M_c,S);

filldraw(unitcircle, opacity(0.1)+lightcyan, lightblue);
draw(M_a--M_b--M_c--cycle, lightblue);
draw(A--B--C--cycle, lightblue);
draw(X--M_a, cyan);
draw(Y--M_b, cyan);
draw(Z--M_c, cyan);

pair H_a = M_b*M_c/M_a;
pair H_b = M_c*M_a/M_b;
pair H_c = M_a*M_b/M_c;

draw(X--A, deepgreen);
draw(Y--B, deepgreen);
draw(Z--C, deepgreen);

filldraw(H_a--H_b--H_c--cycle, opacity(0.1)+lightred, lightred);
draw(H_a--X, lightred);
draw(H_b--Y, lightred);
draw(H_c--Z, lightred);

dot("$X$", X, dir(250));
dot("$Y$", Y, dir(80));
dot("$Z$", Z, dir(200));

dot("$M_a$", M_a, dir(M_a));
dot("$M_b$", M_b, dir(M_b));
dot("$M_c$", M_c, dir(M_c));
dot("$A$", A, dir(A));
dot("$B$", B, dir(B));
dot("$C$", C, dir(C));
dot("$S$", S, dir(310));
dot("$H_a$", H_a, dir(H_a));
dot("$H_b$", H_b, dir(H_b));
dot("$H_c$", H_c, dir(H_c));

/* TSQ Source:

M_a = dir 110
M_b = dir 210
M_c = dir 330
A = M_b+M_c-M_a
B = M_c+M_a-M_b
C = M_a+M_b-M_c
O := circumcenter A B C
S = 0.75*O R310
N := origin R45

X := -M_a+2*foot(N,M_a,S)
Y := -M_b+2*foot(N,M_b,S)
Z := -M_c+2*foot(N,M_c,S)

unitcircle 0.1 lightcyan / lightblue
M_a--M_b--M_c--cycle lightblue
A--B--C--cycle lightblue
X--M_a cyan
Y--M_b cyan
Z--M_c cyan

H_a = M_b*M_c/M_a
H_b = M_c*M_a/M_b
H_c = M_a*M_b/M_c

X--A deepgreen
Y--B deepgreen
Z--C deepgreen

H_a--H_b--H_c--cycle 0.1 lightred / lightred
H_a--X lightred
H_b--Y lightred
H_c--Z lightred

!dot("$X$", X, dir(250));
!dot("$Y$", Y, dir(80));
!dot("$Z$", Z, dir(200));

*/
\end{asy}
\end{center}

We now use Trig Ceva to carry over the concurrence.
By sine law,
\[ \frac{\sin \angle M_{c}AX}{\sin \angle AM_{c}X}
  =\frac{M_{c}X}{AX} \]
and a similar relation for $M_b$ gives that
\[ \frac{\sin \angle M_{c}AX}{\sin \angle M_{b}AX}
  = \frac{\sin \angle AM_{c}X}{\sin \angle AM_{b}X} \cdot \frac{M_{c}X}{M_{b}X}
  = \frac{\sin \angle AM_{c}X}{\sin \angle AM_{b}X}
  \cdot \frac{\sin \angle XM_{a}M_{c}}{\sin \angle XM_{a}M_{b}}. \]
Thus multiplying cyclically gives
\[ \prod_{\text{cyc}} \frac{\sin \angle M_{c}AX}{\sin \angle M_{b}AX}
  = \prod_{\text{cyc}} \frac{\sin \angle AM_{c}X}{\sin \angle AM_{b}X}
  \prod_{\text{cyc}} \frac{\sin \angle XM_{a}M_{c}}{\sin \angle XM_{a}M_{b}}. \]
The latter product on the right-hand side equals $1$
by Trig Ceva on $\triangle M_a M_b M_c$ with cevians
$\ol{M_a X}$, $\ol{M_b Y}$, $\ol{M_c Z}$.
The former product also equals $1$
by Trig Ceva for the concurrence in the previous claim
(and the fact that $\angle AM_{c}X=\angle H_{c}H_{a}X$).
Hence the left-hand side equals $1$, implying the result.

\paragraph{Third solution by moving points (Gopal Goel).}
In this solution,
we will instead use barycentric coordinates with resect to $\triangle ABC$
to bound the degrees suitably,
and then verify for seven distinct choices of $S$.

We let $R$ denote the radius of $\triangle ABC$,
and $N$ the nine-point center.

First, imagine solving for $X$ in the following way.
Suppose $\vec X = (1-t_a) \vec M_a + t_a \vec S$.
Then, using the dot product
(with $\left\lvert \vec v \right\rvert^2 = \vec v \cdot \vec v$ in general)
\begin{align*}
  \frac14 R^2
  &= \left\lvert \vec X - \vec N \right\rvert^2 \\
  &= \left\lvert t_a(\vec S - \vec M_a) + \vec M_a - \vec N \right\rvert^2 \\
  &= \left\lvert t_a(\vec S - \vec M_a) \right\rvert^2
    + 2t_a \left( \vec S - \vec M_a \right) \cdot \left( \vec M_a - \vec N \right)
    + \left\lvert \vec M_a - \vec N \right\rvert^2 \\
  &= t_a^2 \left\lvert (\vec S - \vec M_a) \right\rvert^2
    + 2t_a \left( \vec S - \vec M_a \right) \cdot \left( \vec M_a - \vec N \right)
    + \frac 14 R^2
\end{align*}
Since $t_a \neq 0$ we may solve to obtain
\[ t_a = -\frac{2(\vec M_a - \vec N) \cdot (\vec S - \vec M_a)}%
  {\left\lvert \vec S - \vec M_a \right\rvert^2}.  \]
Now imagine $S$ varies along the Euler line,
meaning there should exist linear functions
$\alpha, \beta, \gamma \colon \RR \to \RR$ such that
\[ S = \left( \alpha(s) , \beta(s) , \gamma(s) \right)
  \qquad s \in \RR \]
with $\alpha(s) + \beta(s) + \gamma(s) = 1$.
Thus $t_a = \frac{f_a}{g_a} = \frac{f_a(s)}{g_a(s)}$
is the quotient of a linear function
$f_a(s)$ and a quadratic function $g_a(s)$.

So we may write:
\begin{align*}
  X &= (1-t_a) \left( 0, \half, \half \right)
    + t_a \left( \alpha, \beta, \gamma \right) \\
  &= \left( t_a \alpha, \half(1-t_a) + t_a \beta,
    \half(1-t_a) + t_a \gamma \right) \\
  &= \left( 2f_a \alpha : g_a - f_a + 2f_a \beta : g_a - f_a + 2 f_a \gamma \right).
\end{align*}
Thus the coordinates of $X$ are quadratic polynomials in $s$
when written in this way.

In a similar way, the coordinates of $Y$ and $Z$
should be quadratic polynomials in $s$.
The Ceva concurrence condition
\[ \prod_{\text{cyc}} \frac{g_a-f_a+2f_a\beta}{g_a-f_a+2f_a\gamma}
  = 1 \]
is thus a polynomial in $s$ of degree at most six.
Our goal is to verify it is identically zero,
thus it suffices to check seven positions of $S$.

\begin{itemize}
  \ii If $S$ is the circumcenter of $\triangle M_a M_b M_c$
  (equivalently the nine-point center of $\triangle ABC$)
  then $\ol{AX}$, $\ol{BY}$, $\ol{CZ}$
  are altitudes of $\triangle ABC$.

  \ii If $S$ is the centroid of $\triangle M_a M_b M_c$
  (equivalently the centroid of $\triangle ABC$),
  then $\ol{AX}$, $\ol{BY}$, $\ol{CZ}$
  are medians of $\triangle ABC$.

  \ii If $S$ is either of the intersections of the Euler
  line with $\gamma$, then $S = X = Y = Z$
  and all cevians concur at $S$.

  \ii If $S$ lies on the $\ol{M_a M_b}$,
  then $Y = M_a$, $X = M_c$, and thus $\ol{AX} \cap \ol{BY} = C$,
  which is of course concurrent with $\ol{CZ}$
  (regardless of $Z$).
  Similarly if $S$ lies on the other sides of $\triangle M_a M_b M_c$.
\end{itemize}
Thus we are also done.

\paragraph{Fourth solution using Pascal (official one).}
We give a different proof of the claim that
$\ol{XH_a}$, $\ol{YH_b}$, $\ol{ZH_c}$ are concurrent
(and then proceed as in the end of the second solution).

Let $H$ denote the orthocenter, $N$ the nine-point center,
and moreover let $N_a$, $N_b$, $N_c$ denote the midpoints
of $\ol{AH}$, $\ol{BH}$, $\ol{CH}$,
which also lie on the nine-point circle
(and are the antipodes of $M_a$, $M_b$, $M_c$).
\begin{itemize}
  \ii By Pascal's theorem on $M_b N_b H_b M_c N_c H_c$,
  the point $P = \ol{M_c H_b} \cap \ol{M_b H_c}$
  is collinear with $N = \ol{M_b N_b} \cap \ol{M_c N_c}$,
  and $H = \ol{N_b H_b} \cap \ol{N_c H_c}$.
  So $P$ lies on the Euler line.

  \ii By Pascal's theorem on $M_b Y H_b M_c Z H_c$,
  the point $\ol{YH_b} \cap \ol{ZH_c}$ is collinear
  with $S = \ol{M_b Y} \cap \ol{M_c Z}$
  and $P = \ol{M_b H_c} \cap \ol{M_c H_b}$.
  Hence $YH_b$ and $ZH_c$ meet on the Euler line, as needed.
\end{itemize}
\pagebreak


\end{document}
