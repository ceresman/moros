% © Evan Chen
% Downloaded from https://web.evanchen.cc/

\documentclass[11pt]{scrartcl}
\usepackage[sexy]{evan}
\ihead{\footnotesize\textbf{\thetitle}}
\ohead{\footnotesize\href{http://web.evanchen.cc}{\ttfamily web.evanchen.cc},
    updated \today}
\title{USAMO 2021 Solution Notes}
\date{\today}

\begin{document}

\maketitle

\begin{abstract}
This is a compilation of solutions
for the 2021 USAMO.
Some of the solutions are my own work,
but many are from the official solutions provided by the organizers
(for which they hold any copyrights),
and others were found by users on the Art of Problem Solving forums.

These notes will tend to be a bit more advanced and terse than the ``official''
solutions from the organizers.
In particular, if a theorem or technique is not known to beginners
but is still considered ``standard'', then I often prefer to
use this theory anyways, rather than try to work around or conceal it.
For example, in geometry problems I typically use directed angles
without further comment, rather than awkwardly work around configuration issues.
Similarly, sentences like ``let $\mathbb{R}$ denote the set of real numbers''
are typically omitted entirely.

Corrections and comments are welcome!
\end{abstract}

\tableofcontents
\newpage

\addtocounter{section}{-1}
\section{Problems}
\begin{enumerate}[\bfseries 1.]
\item %% Problem 1
Rectangles $BCC_1B_2$, $CAA_1C_2$, and $ABB_1A_2$ are erected
outside an acute triangle $ABC$. Suppose that
\[ \angle BC_1C + \angle CA_1A + \angle AB_1B = 180^\circ. \]
Prove that lines $B_1C_2$, $C_1A_2$, and $A_1B_2$ are concurrent.

\item %% Problem 2
The Planar National Park is a undirected 3-regular planar graph
(i.e.\ all vertices have degree $3$).
A visitor walks through the park as follows:
she begins at a vertex and starts walking along an edge.
When she reaches the other endpoint, she turns left.
On the next vertex she turns right, and so on,
alternating left and right turns at each vertex.
She does this until she gets back to the vertex where she started.
What is the largest possible number of times she could have entered
any vertex during her walk, over all possible layouts of the park?

\item %% Problem 3
Let $n \ge 2$ be an integer.
An $n \times n$ board is initially empty.
Each minute, you may perform one of three moves:
\begin{itemize}
\item If there is an L-shaped tromino region
  of three cells without stones on the board
  (see figure; rotations not allowed),
  you may place a stone in each of those cells.
\begin{center}
\begin{asy}
  size(1.5cm);
  filldraw( (0.1,0.1)--(1.9,0.1)--(1.9,0.9)
    --(0.9,0.9)--(0.9,1.9)--(0.1,1.9)--cycle, rgb(0.7,0.7,0.7), black+2);
  draw( scale(2)*unitsquare, black );
  draw( (1,0)--(1,2), black );
  draw( (0,1)--(2,1), black );
\end{asy}
\end{center}
\item If all cells in a column have a stone, you may remove all stones from that column.
\item If all cells in a row have a stone, you may remove all stones from that row.
\end{itemize}
For which $n$ is it possible that, after some non-zero number of
moves, the board has no stones?

\item %% Problem 4
A finite set $S$ of positive integers has the property that,
for each $s\in S$, and each positive integer divisor $d$ of $s$,
there exists a unique element $t\in S$ satisfying $\gcd(s,t) = d$.
(The elements $s$ and $t$ could be equal.)

Given this information, find all possible values for the
number of elements of $S$.

\item %% Problem 5
Let $n \ge 4$ be an integer.
Find all positive real solutions to the following
system of $2n$ equations:
\begin{align*}
  a_1 &= \frac{1}{a_{2n}} + \frac{1}{a_{2}}, & a_2 &= a_1 + a_3, \\[1ex]
  a_3 &= \frac{1}{a_{2}} + \frac{1}{a_{4}}, & a_4 &= a_3 + a_5, \\[1ex]
  a_5 &= \frac{1}{a_{4}} + \frac{1}{a_{6}}, & a_6 &= a_5 + a_7, \\[1ex]
  &\vdotswithin{=} &&\vdotswithin{=} \\
  a_{2n-1} &= \frac{1}{a_{2n-2}} + \frac{1}{a_{2n}}, & a_{2n} &= a_{2n-1} + a_1.
\end{align*}

\item %% Problem 6
Let $ABCDEF$ be a convex hexagon satisfying
$\ol{AB} \parallel \ol{DE}$,
$\ol{BC} \parallel \ol{EF}$,
$\ol{CD} \parallel \ol{FA}$, and
\[ AB \cdot DE = BC \cdot EF = CD \cdot FA. \]

Let $X$, $Y$, and $Z$ be the midpoints
of $\ol{AD}$, $\ol{BE}$, and $\ol{CF}$.
Prove that
the circumcenter of $\triangle ACE$,
the circumcenter of $\triangle BDF$, and
the orthocenter of $\triangle XYZ$ are collinear.

\end{enumerate}
\pagebreak

\section{Solutions to Day 1}
\subsection{USAMO 2021/1, proposed by Ankan Bhattacharya}
\textsl{Available online at \url{https://aops.com/community/p21498558}.}
\begin{mdframed}[style=mdpurplebox,frametitle={Problem statement}]
Rectangles $BCC_1B_2$, $CAA_1C_2$, and $ABB_1A_2$ are erected
outside an acute triangle $ABC$. Suppose that
\[ \angle BC_1C + \angle CA_1A + \angle AB_1B = 180^\circ. \]
Prove that lines $B_1C_2$, $C_1A_2$, and $A_1B_2$ are concurrent.
\end{mdframed}
The angle condition implies the circumcircles of the three
rectangles concur at a single point $P$.
Then $\dang C P B_2 = \dang C P A_1 = 90\dg$,
hence $P$ lies on $A_1 B_2$ etc., so we're done.

\begin{remark*}
  As one might guess from the two-sentence solution,
  the entire difficulty of the problem
  is getting the characterization of the concurrence point.
\end{remark*}
\pagebreak

\subsection{USAMO 2021/2, proposed by Zoran Sunic}
\textsl{Available online at \url{https://aops.com/community/p21498640}.}
\begin{mdframed}[style=mdpurplebox,frametitle={Problem statement}]
The Planar National Park is a undirected 3-regular planar graph
(i.e.\ all vertices have degree $3$).
A visitor walks through the park as follows:
she begins at a vertex and starts walking along an edge.
When she reaches the other endpoint, she turns left.
On the next vertex she turns right, and so on,
alternating left and right turns at each vertex.
She does this until she gets back to the vertex where she started.
What is the largest possible number of times she could have entered
any vertex during her walk, over all possible layouts of the park?
\end{mdframed}
The answer is $3$.

We consider the trajectory of the visitor
as an ordered sequence of \emph{turns}.
A turn is defined by specifying a vertex,
the incoming edge, and the outgoing edge.
Hence there are six possible turns for each vertex.

\begin{claim*}
  Given one turn in the sequence,
  one can reconstruct the entire sequence of turns.
\end{claim*}
\begin{proof}
  This is clear from the process's definition:
  given a turn $t$, one can compute the turn after it
  and the turn before it.
\end{proof}

This implies already that the trajectory of the visitor,
when extended to an infinite sequence,
is totally periodic (not just eventually periodic),
because there are finitely many possible turns,
so some turn must be repeated.
So, any turn appears at most once in the period of the sequence,
giving a na\"{\i}ve bound of $6$ for the original problem.

However, the following claim improves the bound to $3$.
\begin{claim*}
  It is impossible for both of the turns
  $a \to b \to c$ and $c \to b \to a$ to occur
  in the same trajectory.
\end{claim*}
\begin{proof}
  If so, then extending the path, we get
  $a \to b \to c \to d \to e \to \dotsb$
  and $\dots \to e \to d \to c \to b \to a$, as illustrated below
  in red and blue respectively.
  \begin{center}
  \begin{asy}
    defaultpen(fontsize(11pt));
    dotfactor *= 1.8;
    pair A = (0,0);
    pair B = (1,1);
    pair C = (0,2);
    pair D = (1,3);
    pair E = (0,4);
    draw(A--B--C--D--E--(0.5,4.5), black+1.3);
    draw( (0.5,4.5)--(0.8,4.8), black+1.3+dotted );
    draw((1,-1)--A--(-1,0), grey);
    draw(B--(2,1), grey);
    draw(C--(-1,2), grey);
    draw(D--(2,3), grey);
    draw(E--(-1,4), grey);
    dot("$a$", A, -dir(B-A), deepgreen);
    dot("$b$", B, -dir(C-B), deepgreen);
    dot("$c$", C, -dir(D-C), deepgreen);
    dot("$d$", D, -dir(E-D), deepgreen);
    dot("$e$", E, dir(90), deepgreen);
    draw( (1,0.2)..(1.4,1)..(1,1.8), red+1.8, EndArrow(TeXHead));
    draw( (1,2.2)..(1.4,3)..(1,3.8), lightred+1, EndArrow(TeXHead));
    draw( (0.8,1.6)..(0.4,2)..(0.8,2.4), lightred+1, EndArrow(TeXHead));
    draw( (0.8,3.6)..(0.4,4)..(0.8,4.4), lightred+1, EndArrow(TeXHead));
    draw( (0.2,1.4)..(0.6,1)..(0.2,0.6), blue+1.8, EndArrow(TeXHead));
    draw( (0.2,3.4)..(0.6,3)..(0.2,2.6), lightblue+1, EndArrow(TeXHead));
    draw( (0,2.8)..(-0.4,2)..(0,1.2), lightblue+1, EndArrow(TeXHead));
    draw( (0,4.8)..(-0.4,4)..(0,3.2), lightblue+1, EndArrow(TeXHead));
  \end{asy}
  \end{center}
  However, we assumed for contradiction the red and blue paths
  were part of the same trajectory,
  yet they clearly never meet.
\end{proof}

It remains to give a construction showing $3$ can be achieved.
There are many, many valid constructions.
One construction due to Danielle Wang is given here,
who provided the following motivation:
``I was lying in bed and drew the first thing I could think of''.
The path is $CAHIFGDBAHEFGJBAC$ which visits $A$ three times.
\begin{center}
\begin{asy}
size(10cm);
defaultpen(fontsize(11pt));
pair A = (-2,2);
pair B = (2,2);
pair C = (-1,1);
pair D = (1,1);
pair E = (-1,-1);
pair F = (0,-1);
pair G = (1,-1);
pair H = (-2,-2);
pair I = (0,-2);
pair J = (2,-2);
draw(A--H, blue);
label("$2,9$", A--H, fontsize(8pt)+deepgreen);
draw(A--C, blue);
label("$16$", A--C, fontsize(8pt)+deepgreen);
draw(A--B, blue);
draw(B--D, blue);
draw(B--J, blue);
draw(B--A, blue);
label("$8,15$", B--A, fontsize(8pt)+deepgreen);
draw(C--E, blue);
draw(C--D, blue);
draw(C--A, blue);
label("$1$", C--A, fontsize(8pt)+deepgreen);
draw(D--C, blue);
draw(D--G, blue);
draw(D--B, blue);
label("$7$", D--B, fontsize(8pt)+deepgreen);
draw(E--H, blue);
draw(E--F, blue);
label("$11$", E--F, fontsize(8pt)+deepgreen);
draw(E--C, blue);
draw(F--I, blue);
draw(F--G, blue);
label("$5,12$", F--G, fontsize(8pt)+deepgreen);
draw(F--E, blue);
draw(G--F, blue);
draw(G--J, blue);
label("$13$", G--J, fontsize(8pt)+deepgreen);
draw(G--D, blue);
label("$6$", G--D, fontsize(8pt)+deepgreen);
draw(H--I, blue);
label("$3$", H--I, fontsize(8pt)+deepgreen);
draw(H--E, blue);
label("$10$", H--E, fontsize(8pt)+deepgreen);
draw(H--A, blue);
draw(I--H, blue);
draw(I--J, blue);
draw(I--F, blue);
label("$4$", I--F, fontsize(8pt)+deepgreen);
draw(J--B, blue);
label("$14$", J--B, fontsize(8pt)+deepgreen);
draw(J--G, blue);
draw(J--I, blue);
dot("$A$", A, dir(A));
dot("$B$", B, dir(B));
dot("$C$", C, -dir(C));
dot("$D$", D, -dir(D));
dot("$E$", E, -dir(E));
dot("$F$", F, -dir(F));
dot("$G$", G, -dir(G));
dot("$H$", H, dir(H));
dot("$I$", I, dir(I));
dot("$J$", J, dir(J));
\end{asy}
\end{center}

\begin{remark*}
  As the above example shows it is possible to transverse
  an edge more than once even in the same direction,
  as in edge $AH$ above.
\end{remark*}
\pagebreak

\subsection{USAMO 2021/3, proposed by Alex Zhai, Shaunak Kishore}
\textsl{Available online at \url{https://aops.com/community/p21498538}.}
\begin{mdframed}[style=mdpurplebox,frametitle={Problem statement}]
Let $n \ge 2$ be an integer.
An $n \times n$ board is initially empty.
Each minute, you may perform one of three moves:
\begin{itemize}
\item If there is an L-shaped tromino region
  of three cells without stones on the board
  (see figure; rotations not allowed),
  you may place a stone in each of those cells.
\begin{center}
\begin{asy}
  size(1.5cm);
  filldraw( (0.1,0.1)--(1.9,0.1)--(1.9,0.9)
    --(0.9,0.9)--(0.9,1.9)--(0.1,1.9)--cycle, rgb(0.7,0.7,0.7), black+2);
  draw( scale(2)*unitsquare, black );
  draw( (1,0)--(1,2), black );
  draw( (0,1)--(2,1), black );
\end{asy}
\end{center}
\item If all cells in a column have a stone, you may remove all stones from that column.
\item If all cells in a row have a stone, you may remove all stones from that row.
\end{itemize}
For which $n$ is it possible that, after some non-zero number of
moves, the board has no stones?
\end{mdframed}
The answer is $3 \mid n$.

\medskip

\textbf{Construction}:
For $n=3$, the construction is fairly straightforward,
shown below.
\begin{center}
\begin{asy}
  size(10cm);
  picture g;
  for (int i=0; i<=3; ++i) {
  draw(g, (i-0.5,-0.5)--(i-0.5,2.5), grey);
  draw(g, (-0.5,i-0.5)--(2.5,i-0.5), grey);
  }
  add(g);
  add(shift(4,0)*g);
  add(shift(8,0)*g);
  add(shift(12,0)*g);

  real r = 0.35;
  fill(circle( (0,0), r ), red);
  fill(circle( (1,0), r ), red);
  fill(circle( (0,1), r ), red);
  fill(circle( (1,1), r ), blue);
  fill(circle( (2,1), r ), blue);
  fill(circle( (1,2), r ), blue);

  fill(circle( (4,0), r ), red);
  fill(circle( (4,1), r ), red);
  fill(circle( (6,1), r ), blue);

  fill(circle( (8,0), r ), red);
  fill(circle( (8,1), r ), red);
  fill(circle( (10,1), r ), blue);
  fill(circle( (9,0), r ), deepgreen);
  fill(circle( (9,1), r ), deepgreen);
  fill(circle( (10,0), r ), deepgreen);
\end{asy}
\end{center}
This can be extended to any $3 \mid n$.

\medskip

\textbf{Polynomial-based proof of converse}:
Assume for contradiction $3 \nmid n$.
We will show the task is impossible
even if we allow stones to have real weights in our process.
A valid elimination corresponds to a polynomial $P \in \RR[x,y]$
such that
\begin{align*}
  \deg_x P &\le n-2 \\
  \deg_y P &\le n-2 \\
  (1+x+y) P(x,y)
  &\in \left< 1+x+\dots+x^{n-1}, \; 1+y+\dots+y^{n-1}\right>.
\end{align*}
(Here $\left< \dots\right>$ is an ideal of $\RR[x,y]$.)
In particular, if $S$ is the set of $n$th roots of unity other than $1$,
we should have
\[ (1+z_1+z_2) P\left( z_1, z_2 \right) = 0 \]
for any $z_1 , z_2 \in S$.
Since $3 \nmid n$, it follows that $1+z_1+z_2 \neq 0$ always.

So $P$ vanishes on $S \times S$,
a contradiction to the bounds on $\deg P$
(by, say, combinatorial nullstellensatz on any nonzero term).

\textbf{Linear algebraic proof of converse}
(due to \href{https://aops.com/community/p21499813}{William Wang}):
Suppose there is a valid sequence of moves.
Call $r_j$ the number of operations clearing row $j$,
indexing from bottom-to-top.
The idea behind the solution is that we are going to
calculate, for each cell, the number of times it is operated
on entirely as a function of $r_j$.
For example, a hypothetical illustration with $n=6$ is partially drawn below,
with the number in each cell denoting how many times it was the corner of an $L$.
\[
  \begin{bmatrix}
    0 & 0 & 0 & 0 & 0 & 0 \\
    c_1 & c_2 & {\color{blue}c_3}=r_3 & {\color{blue}c_4}=r_5-r_4 & {\color{blue}c_5}={\color{red}r_5} & 0 \\
    \vdots & \vdots & r_2+r_3-r_5  & r_5-r_3 & {\color{red}r_4} & 0  \\
    \vdots & \vdots & r_1+r_2+r_3-r_4-r_5 & r_5-r_2  & {\color{red}r_3} & 0  \\
    \vdots & \vdots & r_1+r_2+r_4-r_5  & r_5-r_1  & {\color{red}r_2} & 0  \\
    \vdots & \vdots & r_1+r_4-r_5  & r_5  & {\color{red}r_1} & 0  \\
  \end{bmatrix}
\]
Let $a_{i,j}$ be the expression in $(i,j)$.
It will also be helpful to define $c_i$ in the obvious way as well.

\begin{claim*}
  We have $c_n = r_n = 0$, $a_{n-1,j} = r_j$ and $a_{i,n-1} = c_i$.
\end{claim*}
\begin{proof}
  The first statement follows since $(n,n)$ may never obtain a stone.
  The equation $a_{n-1,j} = r_j$ follows as both equal the number of times
  that cell $(n,j)$ obtains a stone.
  The final equation is similar.
\end{proof}

\begin{claim*}
  For $1 \le i,j \le n-1$, the following recursion holds:
  \[ a_{i,j} + a_{i+1,j} + a_{i+1,j-1} = r_j + c_{i+1}. \]
\end{claim*}
\begin{proof}
  Focus on cell $(i+1,j)$.
  The left-hand side counts the number of times that gains a stone
  while the right-hand side counts the number of times it loses a stone;
  they must be equal.
\end{proof}
We can coerce the table above into matrix form now as follows.
Define
\[
  K = \begin{bmatrix}
    -1 & -1 & 0 & 0 & \dots & 0 & 0 & 0 \\
    0 & -1 & -1 & 0 & \dots & 0 & 0 & 0 \\
    0 & 0 & -1 & -1 & \dots & 0 & 0 & 0 \\
    \vdots & \vdots & \vdots & \vdots & \ddots & \vdots & \vdots & \vdots \\
    0 & 0 & 0 & 0 & \dots & -1 & -1 & 0 \\
    0 & 0 & 0 & 0 & \dots & 0 & -1 & -1 \\
    1 & 1 & 1 & 1 & \dots & 1 & 1 & 0 \\
  \end{bmatrix}.
\]
Then define a sequence of matrices $M_i$ recursively by $M_{n-1} = \id$,
and \[ M_i = \id + K M_{i+1} = \id + K + \dots  + K^{n-1-i}. \]
The matrices are chosen so that, by construction,
\[ \left< r_1, \dots, r_{n-1} \right> M_i = \left< a_{i,1}, \dots, a_{i,n-1} \right> \]
for $i=1,2,\dots,n-1$.
On the other hand, we can extend the recursion one level deeper and obtain
\[ \left< r_1, \dots, r_{n-1} \right> M_0 = \left< 0, \dots, 0 \right>. \]
However, the crux of the solution is the following.
\begin{claim*}
  The eigenvalues of $K$ are exactly $-(1+e^{\frac{2\pi i k}{n}})$
  for $k=1,2,\dots,n-1$.
\end{claim*}
\begin{proof}
  The matrix $-(K+\id)$ is fairly known to have roots of unity
  as the coefficients.
\end{proof}
However, we are told that apparently
\[ 0 = \det M_0 = \det\left( \id + K + K^2 + \dots + K^{n-1} \right) \]
which means $\det(K^n - \id) = 0$.
This can only happen if $K^n$ has eigenvalue $1$, meaning that
\[ \left[ -(1+\omega) \right]^n = 1 \]
for $\omega$ some $n$\ts{th} root of unity, not necessarily primitive.
This can only happen if $\left\lvert 1+\omega \right\rvert = 1$, ergo $3 \mid n$.
\pagebreak

\section{Solutions to Day 2}
\subsection{USAMO 2021/4, proposed by Carl Schildkraut}
\textsl{Available online at \url{https://aops.com/community/p21498580}.}
\begin{mdframed}[style=mdpurplebox,frametitle={Problem statement}]
A finite set $S$ of positive integers has the property that,
for each $s\in S$, and each positive integer divisor $d$ of $s$,
there exists a unique element $t\in S$ satisfying $\gcd(s,t) = d$.
(The elements $s$ and $t$ could be equal.)

Given this information, find all possible values for the
number of elements of $S$.
\end{mdframed}
The answer is that $|S|$ must be a power of $2$ (including $1$),
or $|S| = 0$ (a trivial case we do not discuss further).

\medskip

\textbf{Construction}:
For any nonnegative integer $k$,
a construction for $|S| = 2^k$ is given
by
\[ S = \left\{
    (p_1 \text{ or } q_1)
    \times
    (p_2 \text{ or } q_2)
    \times
    \dots
    \times
    (p_k \text{ or } q_k)
  \right\}
\]
for $2k$ distinct primes $p_1$, \dots, $p_k$, $q_1$, \dots, $q_k$.

\medskip

\textbf{Converse}: the main claim is as follows.
\begin{claim*}
  In any valid set $S$,
  for any prime $p$ and $x \in S$, $\nu_p(x) \le 1$.
\end{claim*}
\begin{proof}
  Assume for contradiction $e = \nu_p(x) \ge 2$.
  \begin{itemize}
    \ii On the one hand, by taking $x$ in the statement,
    we see $\frac{e}{e+1}$ of the elements of $S$ are divisible by $p$.
    \ii On the other hand, consider a $y \in S$ such that
    $\nu_p(y)=1$ which must exist (say if $\gcd(x,y) = p$).
    Taking $y$ in the statement,
    we see $\half$ of the elements of $S$ are divisible by $p$.
  \end{itemize}
  So $e=1$, contradiction.
\end{proof}
Now since $|S|$ equals the number of divisors of any element of $S$,
we are done.
\pagebreak

\subsection{USAMO 2021/5, proposed by Mohsen Jamaali}
\textsl{Available online at \url{https://aops.com/community/p21498967}.}
\begin{mdframed}[style=mdpurplebox,frametitle={Problem statement}]
Let $n \ge 4$ be an integer.
Find all positive real solutions to the following
system of $2n$ equations:
\begin{align*}
  a_1 &= \frac{1}{a_{2n}} + \frac{1}{a_{2}}, & a_2 &= a_1 + a_3, \\[1ex]
  a_3 &= \frac{1}{a_{2}} + \frac{1}{a_{4}}, & a_4 &= a_3 + a_5, \\[1ex]
  a_5 &= \frac{1}{a_{4}} + \frac{1}{a_{6}}, & a_6 &= a_5 + a_7, \\[1ex]
  &\vdotswithin{=} &&\vdotswithin{=} \\
  a_{2n-1} &= \frac{1}{a_{2n-2}} + \frac{1}{a_{2n}}, & a_{2n} &= a_{2n-1} + a_1.
\end{align*}
\end{mdframed}
The answer is that the only solution is
$(1,2,1,2,\dots,1,2)$ which works.

We will prove $a_{2k}$ is a constant sequence,
at which point the result is obvious.

\paragraph{First approach (Andrew Gu).}
Apparently, with indices modulo $2n$, we should have
\[ a_{2k} = \frac{1}{a_{2k-2}}
  + \frac{2}{a_{2k}} + \frac{1}{a_{2k+2}} \]
for every index $k$ (this eliminates all $a_{\text{odd}}$'s).
Define
\[ m = \min_k a_{2k} \qquad\text{and}\qquad M = \max_k a_{2k}. \]
Look at the indices $i$ and $j$
achieving $m$ and $M$ to respectively get
\begin{align*}
  m &= \frac2m + \frac{1}{a_{2i-2}} + \frac{1}{a_{2i+2}}
  \ge \frac2m + \frac1M + \frac1M  = \frac2m + \frac2M \\[1ex]
  M &= \frac2M + \frac{1}{a_{2j-2}} + \frac{1}{a_{2j+2}}
  \le \frac2M + \frac1m + \frac1m = \frac2m + \frac2M.
\end{align*}
Together this gives $m \ge M$, so $m = M$.
That means $a_{2i}$ is constant as $i$ varies,
solving the problem.

\paragraph{Second approach (author's solution).}
As before, we have
\[ a_{2k} = \frac{1}{a_{2k-2}}
  + \frac{2}{a_{2k}} + \frac{1}{a_{2k+2}} \]
The proof proceeds in three steps.
\begin{itemize}
\ii Define
\[ S = \sum_k a_{2k},
  \quad\text{and}\quad
  T = \sum_k \frac{1}{a_{2k}}. \]
Summing gives $S = 4T$.
On the other hand, Cauchy-Schwarz says $S \cdot T \ge n^2$,
so $T \ge \half n$.

\ii On the other hand,
\[ 1 = \frac{1}{a_{2k-2} a_{2k}}
  + \frac{2}{a_{2k}^2} + \frac{1}{a_{2k} a_{2k+2}} \]
Sum this modified statement to obtain
\[
  n = \sum_k \left( \frac{1}{a_{2k}} + \frac{1}{a_{2k+2}} \right)^2
  \overset{\text{QM-AM}}{\ge}
  \frac 1n \left( \sum_k \frac{1}{a_{2k}} + \frac{1}{a_{2k+2}} \right)^2
  = \frac 1n \left( 2T \right)^2
\]
So $T \le \half n$.

\ii Since $T \le \half n$ and $T \ge \half n$,
we must have equality everywhere above.
This means $a_{2k}$ is a constant sequence.
\end{itemize}

\begin{remark*}
  The problem is likely intractable over $\CC$,
  in the sense that one gets a high-degree polynomial
  which almost certainly has many complex roots.
  So it seems likely that most solutions must involve
  some sort of inequality,
  using the fact we are over $\RR_{>0}$ instead.
\end{remark*}
\pagebreak

\subsection{USAMO 2021/6, proposed by Ankan Bhattacharya}
\textsl{Available online at \url{https://aops.com/community/p21498548}.}
\begin{mdframed}[style=mdpurplebox,frametitle={Problem statement}]
Let $ABCDEF$ be a convex hexagon satisfying
$\ol{AB} \parallel \ol{DE}$,
$\ol{BC} \parallel \ol{EF}$,
$\ol{CD} \parallel \ol{FA}$, and
\[ AB \cdot DE = BC \cdot EF = CD \cdot FA. \]

Let $X$, $Y$, and $Z$ be the midpoints
of $\ol{AD}$, $\ol{BE}$, and $\ol{CF}$.
Prove that
the circumcenter of $\triangle ACE$,
the circumcenter of $\triangle BDF$, and
the orthocenter of $\triangle XYZ$ are collinear.
\end{mdframed}
We present two solutions.

\paragraph{Parallelogram solution found by contestants.}
Note that the following figure is intentionally
\emph{not} drawn to scale, to aid legibility.
We construct parallelograms $ABCE'$, etc as shown.
Note that this gives two congruent triangles $A'C'E'$ and $B'D'F'$.
(Assuming that triangle $XYZ$ is non-degenerate,
the triangles $A'C'E'$ and $B'D'F'$ will also be non-degenerate.)
\begin{center}
\begin{asy}
pair B = dir(123);
pair C = dir(150);
pair D = dir(250);
pair E = -conj(D);
pair F = B*C/E;
pair A = D*C/F;

pair Ap = C+E-D;
pair Cp = E+A-F;
pair Ep = A+C-B;
pair Bp = D+F-E;
pair Dp = F+B-A;
pair Fp = B+D-C;

filldraw(A--B--C--D--E--F--cycle, opacity(0.1)+lightcyan, blue);
draw(A--Cp, red);
draw(C--Ep, red);
draw(E--Ap, red);
draw(B--Fp, deepgreen);
draw(D--Bp, deepgreen);
draw(F--Dp, deepgreen);
draw(A--D, grey);
pair X = midpoint(A--D);

pair M = midpoint(Cp--Ep);
pair N = midpoint(Bp--Fp);
draw(M--N, grey);

dot("$B$", B, dir(B));
dot("$C$", C, dir(C));
dot("$D$", D, dir(D));
dot("$E$", E, dir(E));
dot("$F$", F, dir(F));
dot("$A$", A, dir(A));
dot("$A'$", Ap, dir(90));
dot("$C'$", Cp, dir(160));
dot("$E'$", Ep, dir(310));
dot("$B'$", Bp, dir(90));
dot("$D'$", Dp, dir(180));
dot("$F'$", Fp, dir(315));
dot("$X$", X, dir(X));
dot("$M$", M, dir(350));
dot("$N$", N, dir(270));

/* TSQ Source:

B = dir 123
C = dir 150
D = dir 250
E = -conj(D)
F = B*C/E
A = D*C/F

A' = C+E-D R90
C' = E+A-F R160
E' = A+C-B R310
B' = D+F-E R90
D' = F+B-A R180
F' = B+D-C R315

A--B--C--D--E--F--cycle 0.1 lightcyan / blue
A--Cp red
C--Ep red
E--Ap red
B--Fp deepgreen
D--Bp deepgreen
F--Dp deepgreen
A--D grey
X = midpoint A--D

M = midpoint Cp--Ep R350
N = midpoint Bp--Fp R270
M--N grey

*/
\end{asy}
\end{center}
\begin{claim*}
  If $AB \cdot DE = BC \cdot EF = CD \cdot FA = k$,
  then the circumcenters of $ACE$ and $A'C'E'$ coincide.
\end{claim*}
\begin{proof}
  The power of $A$ to $(A'C'E')$ is
  $AE' \cdot AC' = BC \cdot EF = k$;
  same for $C$ and $E$.
\end{proof}


\begin{center}
\begin{asy}
size(10cm);
pair B = dir(123);
pair C = dir(150);
pair D = dir(250);
pair E = -conj(D);
pair F = B*C/E;
pair A = D*C/F;

pair Ap = C+E-D;
pair Cp = E+A-F;
pair Ep = A+C-B;
pair Bp = D+F-E;
pair Dp = F+B-A;
pair Fp = B+D-C;

pair Ma = midpoint(Cp--Ep);
pair Mc = midpoint(Ep--Ap);
pair Me = midpoint(Ap--Cp);
pair Mb = midpoint(Dp--Fp);
pair Md = midpoint(Fp--Bp);
pair Mf = midpoint(Bp--Dp);

pair X = midpoint(A--D);
pair Y = midpoint(B--E);
pair Z = midpoint(C--F);

draw(Ap--Cp--Ep--cycle, red);
draw(Bp--Dp--Fp--cycle, deepgreen);
draw(X--Y--Z--cycle, blue);
draw(Ma--Mc--Me--cycle, lightred);
draw(Md--Mf--Mb--cycle, lightgreen);

draw(Mc--Mf, grey);
draw(Me--Mb, grey);
draw(Ma--Md, grey);

dot("$A'$", Ap, dir(180));
dot("$C'$", Cp, dir(275));
dot("$E'$", Ep, dir(0));
dot("$B'$", Bp, dir(0));
dot("$D'$", Dp, dir(180));
dot("$F'$", Fp, dir(275));

dot("$X$", X, dir(340), blue);
dot("$Y$", Y, dir(225), blue);
dot("$Z$", Z, dir(130), blue);
dot(Ma, red);
dot(Mc, red);
dot(Me, red);
dot(Mb, deepgreen);
dot(Md, deepgreen);
dot(Mf, deepgreen);

dot(orthocentercenter(Ma, Mc, Me), red);
dot(orthocentercenter(Mb, Md, Mf), deepgreen);
dot(orthocentercenter(X, Y, Z), blue);
\end{asy}
\end{center}

\begin{claim*}
  Triangle $XYZ$ is the vector average
  of the (congruent) medial triangles of
  triangles $A'C'E'$ and $B'D'F'$.
\end{claim*}
\begin{proof}
  If $M$ and $N$ are the midpoints
  of $\ol{C'E'}$ and $\ol{B'F'}$,
  then $X$ is the midpoint of $\ol{MN}$ by vector calculation:
  \begin{align*}
    \frac{\vec M + \vec N}{2}
    &= \frac{\frac{\vec C' + \vec E'}{2} + \frac{\vec B' + \vec F'}{2}}{2} \\
    &= \frac{\vec C' + \vec E' + \vec B' + \vec F'}{4} \\
    &= \frac{(\vec A + \vec E - \vec F) + (\vec C + \vec A - \vec B)
      + (\vec D + \vec F - \vec E) + (\vec B + \vec D - \vec C)}{4} \\
    &= \frac{\vec A + \vec D}{2} = \vec X. \qedhere
  \end{align*}
\end{proof}
Hence the orthocenter of $XYZ$
is the midpoint of the orthocenters
of the medial triangles of $A'C'E'$ and $B'D'F'$ --- that is,
their circumcenters.

\paragraph{Author's solution.}
Call $MNP$ and $UVW$ the medial triangles
of $ACE$ and $BDF$.

\begin{center}
\begin{asy}
size(11cm);
pair M = dir(253.96);
pair N = dir(14.7);
pair P = dir(133.84);

pair U = dir(283.18);
pair V = dir(44.37);
pair W = dir(163.96);

pair A = N+P-M;
pair C = P+M-N;
pair E = M+N-P;

pair B = V+W-U;
pair D = W+U-V;
pair F = U+V-W;

draw(A--C--E--cycle, lightred);
draw(M--N--P--cycle, lightred);
draw(B--D--F--cycle, lightgreen);
draw(U--V--W--cycle, lightgreen);

pair X = midpoint(A--D);
pair Y = midpoint(B--E);
pair Z = midpoint(C--F);

draw(A--D, grey);
draw(B--E, grey);
draw(C--F, grey);

draw(unitcircle, dotted);

draw(M--V, deepcyan+1);
draw(B--C, deepcyan+1);
draw(E--F, deepcyan+1);

draw(A--B, blue+1);
draw(N--W, blue+1);
draw(D--E, blue+1);

draw(A--F, purple+1);
draw(U--P, purple+1);
draw(D--C, purple+1);

dot("$M$", M, dir(M));
dot("$N$", N, dir(N));
dot("$P$", P, dir(P));
dot("$U$", U, dir(U));
dot("$V$", V, dir(V));
dot("$W$", W, dir(W));
dot("$A$", A, dir(A));
dot("$C$", C, dir(C));
dot("$E$", E, dir(E));
dot("$B$", B, dir(B));
dot("$D$", D, dir(D));
dot("$F$", F, dir(F));
dot("$X$", X, dir(X));
dot("$Y$", Y, dir(Y));
dot("$Z$", Z, dir(Z));

/* TSQ Source:

!size(11cm);
M = dir 253.96
N = dir 14.7
P = dir 133.84

U = dir 283.18
V = dir 44.37
W = dir 163.96

A = N+P-M
C = P+M-N
E = M+N-P

B = V+W-U
D = W+U-V
F = U+V-W

A--C--E--cycle lightred
M--N--P--cycle lightred
B--D--F--cycle lightgreen
U--V--W--cycle lightgreen

X = midpoint A--D
Y = midpoint B--E
Z = midpoint C--F

A--D grey
B--E grey
C--F grey

unitcircle dotted

M--V deepcyan+1
B--C deepcyan+1
E--F deepcyan+1

A--B blue+1
N--W blue+1
D--E blue+1

A--F purple+1
U--P purple+1
D--C purple+1

*/
\end{asy}
\end{center}

\begin{claim*}
  In trapezoid $ABDE$, the perpendicular
  bisector of $\ol{XY}$ is the same as the
  perpendicular bisector of the midline $\ol{WN}$.
\end{claim*}
\begin{proof}
  This is true for any trapezoid:
  because $WX = \half AB = YN$.
\end{proof}

\begin{claim*}
  The points $V$, $W$, $M$, $N$ are cyclic.
\end{claim*}
\begin{proof}
  By power of a point from $Y$, since
  \[ WY \cdot YN = \half DE \cdot \half AB
    = \half EF \cdot \half BC
    = VY \cdot YM. \qedhere \]
\end{proof}

Applying all the cyclic variations of the above two claims,
it follows that all six points $U$, $V$, $W$, $M$, $N$, $P$
are concyclic, and the center of that circle
coincides with the circumcenter of $\triangle XYZ$.

\begin{remark*}
  It is also possible to implement ideas from the first solution here,
  by showing all six midpoints have equal power to $(XYZ)$.
\end{remark*}

\begin{claim*}
  The orthocenter of $XYZ$ is the midpoint
  of the circumcenters of $\triangle ACE$ and $\triangle BDF$.
\end{claim*}
\begin{proof}
  Apply complex numbers with the unit circle coinciding
  with the circumcircle of $NVPWMU$.
  Then
  \begin{align*}
    \opname{orthocenter}(XYZ) &= x+y+z = \frac{a+b+c+d+e+f}{2} \\
    \opname{circumcenter}(ACE) &= \opname{orthocenter}(MNP) \\
    &= m+n+p = \frac{c+e}{2} + \frac{e+a}{2} + \frac{a+c}{2} = a+c+e \\
    \opname{circumcenter}(BDF) &= \opname{orthocenter}(UVW) \\
    &= u+v+w = \frac{d+f}{2} + \frac{f+b}{2} + \frac{b+d}{2} = b+d+f.
    \qedhere
  \end{align*}
\end{proof}
\pagebreak


\end{document}
