% © Evan Chen
% Downloaded from https://web.evanchen.cc/

\documentclass[11pt]{scrartcl}
\usepackage[sexy]{evan}
\ihead{\footnotesize\textbf{\thetitle}}
\ohead{\footnotesize\href{http://web.evanchen.cc}{\ttfamily web.evanchen.cc},
    updated \today}
\title{JMO 2012 Solution Notes}
\date{\today}

\begin{document}

\maketitle

\begin{abstract}
This is a compilation of solutions
for the 2012 JMO.
Some of the solutions are my own work,
but many are from the official solutions provided by the organizers
(for which they hold any copyrights),
and others were found by users on the Art of Problem Solving forums.

These notes will tend to be a bit more advanced and terse than the ``official''
solutions from the organizers.
In particular, if a theorem or technique is not known to beginners
but is still considered ``standard'', then I often prefer to
use this theory anyways, rather than try to work around or conceal it.
For example, in geometry problems I typically use directed angles
without further comment, rather than awkwardly work around configuration issues.
Similarly, sentences like ``let $\mathbb{R}$ denote the set of real numbers''
are typically omitted entirely.

Corrections and comments are welcome!
\end{abstract}

\tableofcontents
\newpage

\addtocounter{section}{-1}
\section{Problems}
\begin{enumerate}[\bfseries 1.]
\item %% Problem 1
Given a triangle $ABC$, let $P$ and $Q$ be points
on segments $\ol{AB}$ and $\ol{AC}$, respectively, such that $AP=AQ$.
Let $S$ and $R$ be distinct points on segment $\ol{BC}$
such that $S$ lies between $B$ and $R$,
$\angle BPS=\angle PRS$, and $\angle CQR=\angle QSR$.
Prove that $P$, $Q$, $R$, $S$ are concyclic.

\item %% Problem 2
Find all integers $n \ge 3$ such that
among any $n$ positive real numbers
$a_1$, $a_2$, \dots, $a_n$ with
\[ \max(a_1,a_2,\dots,a_n)
  \le n \cdot \min(a_1,a_2,\dots,a_n), \]
there exist three that are the side lengths
of an acute triangle.

\item %% Problem 3
For $a,b,c > 0$ prove that
\[ \frac{a^3+3b^3}{5a+b}
  + \frac{b^3+3c^3}{5b+c}
  + \frac{c^3+3a^3}{5c+a}
  \ge \frac23(a^2+b^2+c^2).  \]

\item %% Problem 4
Let $\alpha$ be an irrational number with $0 < \alpha < 1$,
and draw a circle in the plane whose
circumference has length $1$.
Given any integer $n\ge 3$,
define a sequence of points $P_1, P_2, \dots, P_n$ as follows.
First select any point $P_1$ on the circle,
and for $2\le k\le n$ define $P_k$
as the point on the circle for which the length
of arc $P_{k-1}P_k$ is $\alpha$,
when travelling counterclockwise around the circle
from $P_{k-1}$ to $P_k$. Suppose that $P_a$ and $P_b$
are the nearest adjacent points on either side of $P_n$.
Prove that $a+b\le n$.

\item %% Problem 5
For distinct positive integers $a, b<2012$,
define $f(a, b)$ to be the number of integers $k$ with $1\le k<2012$
such that the remainder when $ak$ divided by $2012$ is
greater than that of $bk$ divided by $2012$.
Let $S$ be the minimum value of $f(a, b)$, where $a$ and $b$ range
over all pairs of distinct positive integers less than $2012$.
Determine $S$.

\item %% Problem 6
Let $P$ be a point in the plane of $\triangle ABC$,
and $\gamma$ a line through $P$.
Let $A'$, $B'$, $C'$ be the points where the
reflections of lines $PA$, $PB$, $PC$ with respect to $\gamma$
intersect lines $BC$, $CA$, $AB$ respectively.
Prove that $A'$, $B'$, $C'$ are collinear.

\end{enumerate}
\pagebreak

\section{Solutions to Day 1}
\subsection{JMO 2012/1, proposed by Sungyoon Kim, Inseok Seo}
\textsl{Available online at \url{https://aops.com/community/p2669111}.}
\begin{mdframed}[style=mdpurplebox,frametitle={Problem statement}]
Given a triangle $ABC$, let $P$ and $Q$ be points
on segments $\ol{AB}$ and $\ol{AC}$, respectively, such that $AP=AQ$.
Let $S$ and $R$ be distinct points on segment $\ol{BC}$
such that $S$ lies between $B$ and $R$,
$\angle BPS=\angle PRS$, and $\angle CQR=\angle QSR$.
Prove that $P$, $Q$, $R$, $S$ are concyclic.
\end{mdframed}
Assume for contradiction that $(PRS)$ and $(QRS)$ are distinct.
Then $\ol{RS}$ is the radical axis of these two circles.
However, $\ol{AP}$ is tangent to $(PRS)$
and $\ol{AQ}$ is tangent to $(QRS)$,
so point $A$ has equal power to both circles,
which is impossible since $A$ does not lie on line $BC$.
\pagebreak

\subsection{JMO 2012/2, proposed by Titu Andreescu}
\textsl{Available online at \url{https://aops.com/community/p2669112}.}
\begin{mdframed}[style=mdpurplebox,frametitle={Problem statement}]
Find all integers $n \ge 3$ such that
among any $n$ positive real numbers
$a_1$, $a_2$, \dots, $a_n$ with
\[ \max(a_1,a_2,\dots,a_n)
  \le n \cdot \min(a_1,a_2,\dots,a_n), \]
there exist three that are the side lengths
of an acute triangle.
\end{mdframed}
The answer is all $n \ge 13$.

Define $(F_n)$ as the sequence of Fibonacci numbers,
by $F_1 = F_2 = 1$ and $F_{n+1} = F_n + F_{n-1}$.
We will find that Fibonacci numbers show up naturally
when we work through the main proof,
so we will isolate the following calculation now
to make the subsequent solution easier to read.
\begin{claim*}
  For positive integers $m$, we have $F_m \le m^2$ if and only if $m \le 12$.
\end{claim*}
\begin{proof}
  A table of the first $14$ Fibonacci numbers is given below.
  \[
    \begin{array}{rrrrr rrrrr rrrr}
      F_1 & F_2 & F_3 & F_4 & F_5 & F_6 & F_7 & F_8 & F_9
        & F_{10} & F_{11} & F_{12} & F_{13} & F_{14} \\ \hline
      1 & 1 & 2 & 3 & 5 & 8 & 13 & 21 & 34 & 55 & 89 & 144 & 233 & 377
    \end{array}
  \]
  By examining the table, we see that $F_m \le m^2$ is true for $m = 1, 2, \dots 12$,
  and in fact $F_{12} = 12^2 = 144$.
  However, $F_m > m^2$ for $m = 13$ and $m = 14$.

  Now it remains to prove that $F_m > m^2$ for $m \ge 15$.
  The proof is by induction with base cases
  $m = 13$ and $m = 14$ being checked already.
  For the inductive step, if $m \ge 15$ then we have
  \begin{align*}
    F_m &= F_{m-1} + F_{m-2} > (m-1)^2 + (m-2)^2 \\
    &= 2m^2 - 6m + 5 = m^2 + (m-1)(m-5) > m^2
  \end{align*}
  as desired.
\end{proof}

We now proceed to the main problem.
The hypothesis $\max(a_1,a_2,\dots,a_n)
\le n \cdot \min(a_1,a_2,\dots,a_n)$
will be denoted by $(\dagger)$.

\medskip

\textbf{Proof that all $n \ge 13$ have the property.}
We first show now that every $n \ge 13$ has the desired property.
Suppose for contradiction that no three numbers are the sides of an acute triangle.
Assume without loss of generality (by sorting the numbers)
that $a_1 \le a_2 \le \dots \le a_n$.
Then since $a_{i-1}$, $a_i$, $a_{i+1}$ are not the sides of an acute triangle
for each $i \ge 2$, we have that $a_{i+1}^2 \ge a_i^2 + a_{i-1}^2$;
writing this out gives
\begin{align*}
  a_3^2 &\ge a_2^2 + a_1^2 \ge 2a_1^2 \\
  a_4^2 &\ge a_3^2 + a_2^2 \ge 2a_1^2 + a_1^2 = 3a_1^2 \\
  a_5^2 &\ge a_4^2 + a_3^2 \ge 3a_1^2 + 2a_1^2 = 5a_1^2 \\
  a_6^2 &\ge a_5^2 + a_4^2 \ge 5a_1^2 + 3a_1^2 = 8a_1^2
\end{align*}
and so on.
The Fibonacci numbers appear naturally and by induction,
we conclude that $a_i^2 \ge F_i a_1^2$.
In particular, $a_n^2 \ge F_n a_1^2$.

However, we know $\max(a_1, \dots, a_n) = a_n$
and $\min(a_1, \dots, a_n) = a_1$,
so $(\dagger)$ reads $a_n \le n \cdot a_1$.
Therefore we have $F_n \le n^2$, and so $n \le 12$, contradiction!

\medskip

\textbf{Proof that no $n \le 12$ have the property.}
Assume that $n \le 12$.
The above calculation also suggests a way to pick the counterexample:
we choose $a_i = \sqrt{F_i}$ for every $i$.
Then $\min(a_1, \dots, a_n) = a_1 = 1$
and $\max(a_1, \dots, a_n) = \sqrt{F_n}$,
so $(\dagger)$ is true as long as $n \le 12$.
And indeed no three numbers form the sides of an acute triangle:
if $i < j < k$,
then $a_k^2 = F_k = F_{k-1} + F_{k-2} \ge F_j + F_i = a_j^2 + a_i^2$.
\pagebreak

\subsection{JMO 2012/3, proposed by Titu Andreescu}
\textsl{Available online at \url{https://aops.com/community/p2669114}.}
\begin{mdframed}[style=mdpurplebox,frametitle={Problem statement}]
For $a,b,c > 0$ prove that
\[ \frac{a^3+3b^3}{5a+b}
  + \frac{b^3+3c^3}{5b+c}
  + \frac{c^3+3a^3}{5c+a}
  \ge \frac23(a^2+b^2+c^2).  \]
\end{mdframed}
Here are two possible approaches.

\paragraph{Cauchy-Schwarz approach.}
Apply Titu lemma to get
\[ \sum_{\text{cyc}} \frac{a^3}{5a+b}
  = \sum_{\text{cyc}} \frac{a^4}{5a^2+ab}
  \ge \frac{(a^2+b^2+c^2)^2}{\sum_{\text{cyc}} (5a^2+ab)}
  \ge \frac{a^2+b^2+c^2}{6} \]
where the last step follows
from the identity $\sum_{\text{cyc}} (5a^2+ab) \le 6(a^2+b^2+c^2)$.

Similarly,
\[ \sum_{\text{cyc}} \frac{b^3}{5a+b}
  = \sum_{\text{cyc}} \frac{b^4}{5ab+b^2}
  \ge \frac{(a^2+b^2+c^2)^2}{\sum_{\text{cyc}} (5ab+b^2)}
  \ge \frac{a^2+b^2+c^2}{6} \]
using the fact that $\sum_{\text{cyc}} 5ab+b^2 \le 6(a^2+b^2+c^2)$.

Therefore, adding the first display to
three times the second display implies the result.

\paragraph{Cauchy-Schwarz approach.}
The main magical claim is:
\begin{claim*}
  We have
  \[ \frac{a^3+3b^3}{5a+b} \geq \frac{25}{36}b^2 - \frac{1}{36}a^2. \]
\end{claim*}
\begin{proof}
  Let $x=a/b > 0$. The desired inequality is equivalent to
  \[ \frac{x^3+3}{5x+1} \ge \frac{25-x^2}{36}. \]
  However,
  \begin{align*}
    36(x^3+3) - (5x+1)(25-x^2)
    &= 41x^3 + x^2 - 125x + 83 \\
    &= (x-1)^2(41x+83) \ge 0. \qedhere
  \end{align*}
\end{proof}
Sum the claim cyclically to finish.
\begin{remark*}
  [Derivation of the main claim]
  The overall strategy is to hope for a constant $k$ such that
  \[ \frac{a^3+3b^3}{5a+b} \geq ka^2 + \left(\frac{2}{3}-k\right)b^2. \]
  is true.
  Letting $x=a/b$ as above and expanding,
  we need a value $k$ such that the cubic polynomial
  \[ P(x)
    \coloneqq (x^3+3)-(5x+1)\left( kx^2 + \left( \frac 23 - k \right) \right)
    = (1-5k)x^3 - kx^2 + \left( 5k-\frac{10}{3} \right)x
    + \left( k + \frac{7}{3} \right) \]
  is nonnegative everywhere.
  Since $P(1) = 0$ necessarily, in order for $P(1-\eps)$ and $P(1+\eps)$
  to both be nonnegative (for small $\eps$),
  the polynomial $P$ must have a double root at $1$, meaning
  the first derivative $P'(1) = 0$ needs to vanish.
  In other words, we need
  \[ 3(1-5k) - 2k + \left( 5k - \frac{10}{3} \right) = 0. \]
  Solving gives $k = -1/36$.
  One then factors out the repeated root $(x-1)^2$ from the resulting $P$.
\end{remark*}
\pagebreak

\section{Solutions to Day 2}
\subsection{JMO 2012/4, proposed by Sam Vandervelde}
\textsl{Available online at \url{https://aops.com/community/p2669956}.}
\begin{mdframed}[style=mdpurplebox,frametitle={Problem statement}]
Let $\alpha$ be an irrational number with $0 < \alpha < 1$,
and draw a circle in the plane whose
circumference has length $1$.
Given any integer $n\ge 3$,
define a sequence of points $P_1, P_2, \dots, P_n$ as follows.
First select any point $P_1$ on the circle,
and for $2\le k\le n$ define $P_k$
as the point on the circle for which the length
of arc $P_{k-1}P_k$ is $\alpha$,
when travelling counterclockwise around the circle
from $P_{k-1}$ to $P_k$. Suppose that $P_a$ and $P_b$
are the nearest adjacent points on either side of $P_n$.
Prove that $a+b\le n$.
\end{mdframed}
No points coincide since $\alpha$ is irrational.

Assume for contradiction that $n < a+b < 2n$.
Then \[ \ol{P_n P_{a+b-n}} \parallel \ol{P_a P_b}. \]
This is an obvious contradiction
since then $P_{a+b-n}$ is contained in the arc
$\widehat{P_a P_b}$ of the circle through $P_n$.
\pagebreak

\subsection{JMO 2012/5, proposed by Warut Suksompong}
\textsl{Available online at \url{https://aops.com/community/p2669967}.}
\begin{mdframed}[style=mdpurplebox,frametitle={Problem statement}]
For distinct positive integers $a, b<2012$,
define $f(a, b)$ to be the number of integers $k$ with $1\le k<2012$
such that the remainder when $ak$ divided by $2012$ is
greater than that of $bk$ divided by $2012$.
Let $S$ be the minimum value of $f(a, b)$, where $a$ and $b$ range
over all pairs of distinct positive integers less than $2012$.
Determine $S$.
\end{mdframed}
The answer is $S = 502$ (not $503$!).

\begin{claim*}
  If $\gcd(k, 2012) = 1$,
  then necessarily either $k$ or $2012-k$ will counts towards $S$.
\end{claim*}
\begin{proof}
  First note that both $ak$, $bk$ are nonzero modulo $2012$.
  Note also that $ak \not\equiv bk \pmod{2012}$.

  So if $r_a$ is the remainder of $ak \pmod{2012}$,
  then $2012-r_a$ is the remainder of $a(2012-k) \pmod{2012}$
  Similarly we can consider $r_b$ and $2012-r_b$.
  As mentioned already, we have $r_a \neq r_b$.
  So either $r_a > r_b$ or $2012-r_a > 2012-r_b$.
\end{proof}

This implies $S \ge \half \varphi(2012) = 502$.

But this can actually be achieved by taking $a = 4$ and $b = 1010$, since
\begin{itemize}
  \ii If $k$ is even, then $ak \equiv bk \pmod{2012}$
  so no even $k$ counts towards $S$; and
  \ii If $k \equiv 0 \pmod{503}$, then $ak \equiv 0 \pmod{2012}$
  so no such $k$ counts towards $S$.
\end{itemize}
This gives the final answer $S \ge 502$.

\begin{remark*}
  A similar proof works with $2012$ replaced by any $n$
  and will give an answer of $\half\varphi(n)$.
  For composite $n$,
  one uses the Chinese remainder theorem to pick distinct
  $a$ and $b$ not divisible by $n$
  such that $\operatorname{lcm}(a-b, a) = n$.
\end{remark*}
\pagebreak

\subsection{JMO 2012/6, proposed by Titu Andreescu, Cosmin Pohoata}
\textsl{Available online at \url{https://aops.com/community/p2669960}.}
\begin{mdframed}[style=mdpurplebox,frametitle={Problem statement}]
Let $P$ be a point in the plane of $\triangle ABC$,
and $\gamma$ a line through $P$.
Let $A'$, $B'$, $C'$ be the points where the
reflections of lines $PA$, $PB$, $PC$ with respect to $\gamma$
intersect lines $BC$, $CA$, $AB$ respectively.
Prove that $A'$, $B'$, $C'$ are collinear.
\end{mdframed}
We present three solutions.

\paragraph{First solution (complex numbers).}
Let $p=0$ and set $\gamma$ as the real line.
Then $A'$ is the intersection of $bc$ and $p\ol a$.
So, we get
\[ a' = \frac{\ol a(\ol b c - b \ol c)}{(\ol b - \ol c)\ol a-(b-c) a}. \]

\begin{center}
  \begin{asy}
    size(4cm);
    pair A = dir(110), B = dir(210), C = dir(330);
    Drawing("A", A, A);
    Drawing("B", B, dir(-90));
    Drawing("C", C, dir(-90));
    draw(A--B--C--cycle);
    pair P = Drawing("P", 0.1*dir(40), dir(-35));
    pair X1 = P + .5 * dir(10);
    pair X2 = P - .5 * dir(10);
    Drawing(Line(X1,X2), dashed, Arrows);
    pair T = 2*foot(A,X1,X2)-A;
    pair A1 = Drawing("A'", extension(P,T,B,C), dir(-40));
    draw(A--P--A1);
  \end{asy}
\end{center}

Note that
\[ \ol a' = \frac{a (b \ol c - \ol b c)}{(b-c) a - (\ol b - \ol c)\ol a}. \]
Thus it suffices to prove
\[ 0 = \det
\begin{bmatrix}
  \frac{\ol a(\ol b c-b \ol c)}{(\ol b - \ol c)\ol a - (b-c) a} & \frac{a (b \ol c - \ol b c)}{(b-c) a - (\ol b - \ol c) \ol a} & 1 \\
  \frac{\ol b(\ol c a-c \ol a)}{(\ol c - \ol a)\ol b - (c-a) b} & \frac{b (c \ol a - \ol c a)}{(c-a) b - (\ol c - \ol a) \ol b} & 1 \\
  \frac{\ol c(\ol a b-a \ol b)}{(\ol a - \ol b)\ol c - (a-b) c} & \frac{c (a \ol b - \ol a b)}{(a-b) c - (\ol a - \ol b)\ol c} & 1
\end{bmatrix}.
\]
This is equivalent to
\[ 0 = \det
\begin{bmatrix}
  \ol a(\ol b c -b \ol c) & a (\ol b c - b \ol c) & (\ol b - \ol c) \ol a - (b-c) a \\
  \ol b(\ol c a -c \ol a) & b (\ol c a - c \ol a) & (\ol c - \ol a) \ol b - (c-a) b \\
  \ol c(\ol a b -a \ol b) & c (\ol a b - a \ol b) & (\ol a - \ol b) \ol c - (a-b) c \\
\end{bmatrix}. \]
This determinant has the property that the rows sum to zero, and we're done.
\begin{remark*}
  Alternatively, if you don't notice that you could just blindly expand:
  \begin{align*}
    &\phantom{=} \sum_{\text{cyc}} ((\ol b - \ol c) \ol a - (b-c) a) \cdot
    - \det
    \begin{bmatrix}
      b & \ol b \\
      c & \ol c
    \end{bmatrix}
    \left( \ol c a - c \ol a \right)\left( \ol a b - a \ol b \right) \\
    &= (\ol b c - c \ol b)(\ol c a - c \ol a)(\ol a b - a \ol b)
    \sum_{\text{cyc}} \left( ab - ac + \ol c \ol a - \ol b \ol a \right) = 0.
  \end{align*}
\end{remark*}

\paragraph{Second solution (Desargues involution).}
We let $C'' = \ol{A'B'} \cap \ol{AB}$.
Consider complete quadrilateral $ABCA'B'C''C$.
We see that there is an involutive pairing $\tau$ at $P$
swapping $(PA,PA')$, $(PB,PB')$, $(PC,PC'')$.
From the first two, we see $\tau$ coincides with reflection
about $\ell$, hence conclude $C'' = C$.

\paragraph{Third solution (barycentric), by Catherine Xu.}
We will perform barycentric coordinates on the triangle $PCC'$,
with $P=(1,0,0)$, $C'=(0,1,0)$, and $C=(0,0,1)$.
Set $a = CC'$, $b = CP$, $c = C'P$ as usual.
Since $A$, $B$, $C'$ are collinear,
we will define $A = (p : k : q)$ and $B = (p : \ell : q)$.

\begin{claim*}
  Line $\gamma$ is the angle bisector of
  $\angle APA' $, $\angle BPB'$, and $\angle CPC'$.
\end{claim*}
\begin{proof}
  Since $A'P$ is the reflection of $AP$ across $\gamma$, etc.
\end{proof}

Thus $B'$ is the intersection of the isogonal of $B$ with respect to $\angle P$
with the line $CA$; that is,
\[ B' = \left( \frac pk \frac{b^2}{\ell}
  : \frac{b^2}{\ell} : \frac{c^2}{q} \right). \]
Analogously, $A'$ is the intersection of the isogonal of $A$ with respect to $\angle P$
with the line $CB$; that is,
\[ A' = \left( \frac{p}{\ell} \frac{b^2}{k}
    : \frac{b^2}{k} : \frac{c^2}{q} \right). \]
The ratio of the first to third coordinate in these two points
is both $b^2pq : c^2k\ell$, so it follows $A'$, $B'$, and $C'$ are collinear.

\begin{remark*}
  [Problem reference]
  The converse of this problem appears as problem 1052
  attributed S.\ V.\ Markelov in the book
  \emph{Geometriya: 9--11 Klassy: Ot Uchebnoy Zadachi k Tvorcheskoy, 1996},
  by I.\ F.\ Sharygin.
\end{remark*}
\pagebreak


\end{document}
