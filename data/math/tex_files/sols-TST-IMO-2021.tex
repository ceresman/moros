\documentclass[11pt]{scrartcl}
\usepackage[sexy]{evan}
\ihead{\footnotesize\textbf{\thetitle}}
\ohead{\footnotesize\theauthor}
\begin{document}

\title{USA IMO TST 2021 Solutions}
\subtitle{United States of America --- IMO Team Selection Test}
\author{Andrew Gu, Ankan Bhattacharya and Evan Chen}
\date{62\ts{th} IMO 2021 Russia}

\maketitle

\tableofcontents
\newpage

\addtocounter{section}{-1}
\section{Problems}
\begin{enumerate}[\bfseries 1.]
\item %% Problem 1
Determine all integers $s \ge 4$
for which there exist positive integers $a$, $b$, $c$, $d$
such that $s = a+b+c+d$
and $s$ divides $abc+abd+acd+bcd$.

\item %% Problem 2
Points $A$, $V_1$, $V_2$, $B$, $U_2$, $U_1$
lie fixed on a circle $\Gamma$, in that order,
and such that $BU_2 > AU_1 > BV_2 > AV_1$.

Let $X$ be a variable point on the arc $V_1 V_2$ of $\Gamma$
not containing $A$ or $B$.
Line $XA$ meets line $U_1 V_1$ at $C$,
while line $XB$ meets line $U_2 V_2$ at $D$.

Prove there exists a fixed point $K$, independent of $X$,
such that the power of $K$ to the circumcircle
of $\triangle XCD$ is constant.

\item %% Problem 3
Find all functions $f \colon \RR \to \RR$ that satisfy the inequality
\[
  f(y) -
  \left(\frac{z-y}{z-x} f(x) + \frac{y-x}{z-x}f(z)\right) \leq
  f\left(\frac{x+z}{2}\right) - \frac{f(x)+f(z)}{2}
\]
for all real numbers $x < y < z$.

\end{enumerate}
\pagebreak

\section{Solutions to Day 1}
\subsection{USA TST 2021/1, proposed by Ankan Bhattacharya, Michael Ren}
\textsl{Available online at \url{https://aops.com/community/p20672573}.}
\begin{mdframed}[style=mdpurplebox,frametitle={Problem statement}]
Determine all integers $s \ge 4$
for which there exist positive integers $a$, $b$, $c$, $d$
such that $s = a+b+c+d$
and $s$ divides $abc+abd+acd+bcd$.
\end{mdframed}
The answer is $s$ composite.

\paragraph{Composite construction.}
Write $s = (w+x)(y+z)$,
where $w$, $x$, $y$, $z$ are positive integers.
Let $a=wy$, $b=wz$, $c=xy$, $d=xz$.
Then
\[ abc+abd+acd+bcd = wxyz(w+x)(y+z) \]
so this works.

\paragraph{Prime proof.}
Choose suitable $a$, $b$, $c$, $d$. Then
\[
  (a+b)(a+c)(a+d)
  = (abc+abd+acd+bcd) + a^2(a+b+c+d)
  \equiv 0 \pmod s.
\]
Hence $s$ divides a product of positive integers less than $s$,
so $s$ is composite.

\begin{remark*}
  Here is another proof that $s$ is composite.

  Suppose that $s$ is prime.
  Then the polynomial $(x-a)(x-b)(x-c)(x-d) \in \FF_s[x]$
  is even, so the roots come in two opposite pairs in $\FF_s$.
  Thus the sum of each pair is at least $s$,
  so the sum of all four is at least $2s > s$, contradiction.
\end{remark*}
\pagebreak

\subsection{USA TST 2021/2, proposed by Andrew Gu, Frank Han}
\textsl{Available online at \url{https://aops.com/community/p20672623}.}
\begin{mdframed}[style=mdpurplebox,frametitle={Problem statement}]
Points $A$, $V_1$, $V_2$, $B$, $U_2$, $U_1$
lie fixed on a circle $\Gamma$, in that order,
and such that $BU_2 > AU_1 > BV_2 > AV_1$.

Let $X$ be a variable point on the arc $V_1 V_2$ of $\Gamma$
not containing $A$ or $B$.
Line $XA$ meets line $U_1 V_1$ at $C$,
while line $XB$ meets line $U_2 V_2$ at $D$.

Prove there exists a fixed point $K$, independent of $X$,
such that the power of $K$ to the circumcircle
of $\triangle XCD$ is constant.
\end{mdframed}
For brevity, we let $\ell_i$ denote line $U_iV_i$ for $i=1,2$.

We first give an explicit description of the fixed point $K$.
Let $E$ and $F$ be points on $\Gamma$ such that $\ol{AE} \parallel \ell_1$
and $\ol{BF} \parallel \ell_2$.
The problem conditions imply that $E$ lies between $U_1$ and $A$
while $F$ lies between $U_2$ and $B$.
Then we let \[ K = \ol{AF} \cap \ol{BE}. \]
This point exists because $AEFB$ are the vertices
of a convex quadrilateral.

\begin{remark*}
  [How to identify the fixed point]
  If we drop the condition that $X$ lies on the arc,
  then the choice above is motivated by choosing $X \in \{E,F\}$.
  Essentially, when one chooses $X \to E$,
  the point $C$ approaches an infinity point.
  So in this degenerate case, the only points whose
  power is finite to $(XCD)$ are bounded are those on line $BE$.
  The same logic shows that $K$ must lie on line $AF$.
  Therefore, if the problem is going to work,
  the fixed point must be exactly $\ol{AF} \cap \ol{BE}$.
\end{remark*}

We give two possible approaches for proving the power
of $K$ with respect to $(XCD)$ is fixed.

\paragraph{First approach by Vincent Huang.}
We need the following claim:
\begin{claim*}
  Suppose distinct lines $AC$ and $BD$ meet at $X$.
  Then for any point $K$
  \[ \opname{pow}(K, XAB) + \opname{pow}(K, XCD)
    = \opname{pow}(K, XAD) + \opname{pow}(K, XBC). \]
\end{claim*}
\begin{proof}
  The difference between the left-hand side and right-hand
  side is a linear function in $K$,
  which vanishes at all of $A$, $B$, $C$, $D$.
\end{proof}

Construct the points $P = \ell_1 \cap \ol{BE}$
and $Q = \ell_2 \cap \ol{AF}$, which do not depend on $X$.
\begin{claim*}
  Quadrilaterals $BPCX$ and $AQDX$ are cyclic.
\end{claim*}
\begin{proof}
  By Reim's theorem: $\dang CPB = \dang AEB = \dang AXB = \dang CXB$, etc.
\end{proof}

\begin{center}
\begin{asy}
pair A = dir(190);
pair B = -conj(A);
pair U_1 = dir(125);
pair V_1 = dir(220);
pair U_2 = dir(80);
pair V_2 = dir(310);
filldraw(unitcircle, opacity(0.1)+lightcyan, blue);
draw(A--B, blue);
draw(U_1--V_1, deepgreen);
draw(U_2--V_2, deepgreen);

pair E = V_1*U_1/A;
pair F = V_2*U_2/B;

pair X = dir(265);
pair C = extension(A, X, U_1, V_1);
pair D = extension(B, X, U_2, V_2);
draw(A--X--B, lightred);
draw(A--E, deepcyan);
draw(B--F, deepcyan);
pair K = extension(A, F, B, E);
draw(A--F, blue);
draw(B--E, blue);

pair P = extension(B, E, U_1, V_1);
pair Q = extension(A, F, U_2, V_2);

filldraw(circumcircle(X, C, B), opacity(0.05)+yellow, dotted+orange);
filldraw(circumcircle(X, A, D), opacity(0.05)+yellow, dotted+orange);

dot("$A$", A, dir(A));
dot("$B$", B, dir(B));
dot("$U_1$", U_1, dir(U_1));
dot("$V_1$", V_1, dir(V_1));
dot("$U_2$", U_2, dir(U_2));
dot("$V_2$", V_2, dir(V_2));
dot("$E$", E, dir(E));
dot("$F$", F, dir(F));
dot("$X$", X, dir(X));
dot("$C$", C, dir(C));
dot("$D$", D, dir(D));
dot("$K$", K, dir(270));
dot("$P$", P, dir(290));
dot("$Q$", Q, dir(240));

/* TSQ Source:

A = dir 190
B = -conj(A)
U_1 = dir 125
V_1 = dir 220
U_2 = dir 80
V_2 = dir 310
unitcircle 0.1 lightcyan / blue
A--B blue
U_1--V_1 deepgreen
U_2--V_2 deepgreen

E = V_1*U_1/A
F = V_2*U_2/B

X = dir 265
C = extension A X U_1 V_1
D = extension B X U_2 V_2
A--X--B lightred
A--E deepcyan
B--F deepcyan
K = extension A F B E R270
A--F blue
B--E blue

P = extension B E U_1 V_1 R290
Q = extension A F U_2 V_2 R240

circumcircle X C B 0.05 yellow / dotted orange
circumcircle X A D 0.05 yellow / dotted orange

*/
\end{asy}
\end{center}

Now, for the particular $K$ we choose, we have
\begin{align*}
  \opname{pow}(K, XCD) &=
  \opname{pow}(K, XAD) + \opname{pow}(K, XBC) - \opname{pow}(K, XAB) \\
  &= KA \cdot KQ + KB \cdot KP - \opname{pow}(K, \Gamma).
\end{align*}
This is fixed, so the proof is completed.

\paragraph{Second approach by authors.}
Let $Y$ be the second intersection of $(XCD)$ with $\Gamma$.
Let $S = \ol{EY} \cap \ell_1$ and $T = \ol{FY} \cap \ell_2$.
\begin{claim*}
  Points $S$ and $T$ lies on $(XCD)$ as well.
\end{claim*}
\begin{proof}
  By Reim's theorem: $\dang CSY = \dang AEY = \dang AXY = \dang CXY$, etc.
\end{proof}

Now let $X'$ be any other choice of $X$,
and define $C'$ and $D'$ in the obvious way.
We are going to show that $K$ lies on the radical axis
of $(XCD)$ and $(X'C'D')$.

\begin{center}
\begin{asy}
pair A = dir(190);
pair B = -conj(A);
pair U_1 = dir(125);
pair V_1 = dir(220);
pair U_2 = dir(80);
pair V_2 = dir(310);
filldraw(unitcircle, opacity(0.1)+lightcyan, blue);
draw(A--B, blue);
draw(U_1--V_1, deepgreen);
draw(U_2--V_2, deepgreen);

pair E = V_1*U_1/A;
pair F = V_2*U_2/B;

pair X = dir(265);
pair C = extension(A, X, U_1, V_1);
pair D = extension(B, X, U_2, V_2);
draw(A--X--B, lightred);
draw(A--E, deepcyan);
draw(B--F, deepcyan);
pair K = extension(A, F, B, E);
draw(A--F, blue);
draw(B--E, blue);

pair Y = -X+2*foot(X, origin, circumcenter(X, C, D));

draw(circumcircle(X, C, D), red);
pair S = extension(Y, E, U_1, V_1);
pair T = extension(Y, F, U_2, V_2);
draw(E--Y--F, brown);

pair Xp = dir(300);
pair Cp = extension(Xp, A, U_1, V_1);
pair Dp = extension(Xp, B, U_2, V_2);
draw(A--Xp--B, orange);

pair L = extension(S, Y, Cp, Xp);

dot("$A$", A, dir(A));
dot("$B$", B, dir(B));
dot("$U_1$", U_1, dir(U_1));
dot("$V_1$", V_1, dir(V_1));
dot("$U_2$", U_2, dir(U_2));
dot("$V_2$", V_2, dir(V_2));
dot("$E$", E, dir(E));
dot("$F$", F, dir(F));
dot("$X$", X, dir(X));
dot("$C$", C, dir(C));
dot("$D$", D, dir(D));
dot("$K$", K, dir(270));
dot("$Y$", Y, dir(Y));
dot("$S$", S, dir(S));
dot("$T$", T, dir(T));
dot("$X'$", Xp, dir(Xp));
dot("$C'$", Cp, dir(170));
dot("$D'$", Dp, dir(10));
dot("$L$", L, dir(L));

/* TSQ Source:

A = dir 190
B = -conj(A)
U_1 = dir 125
V_1 = dir 220
U_2 = dir 80
V_2 = dir 310
unitcircle 0.1 lightcyan / blue
A--B blue
U_1--V_1 deepgreen
U_2--V_2 deepgreen

E = V_1*U_1/A
F = V_2*U_2/B

X = dir 265
C = extension A X U_1 V_1
D = extension B X U_2 V_2
A--X--B lightred
A--E deepcyan
B--F deepcyan
K = extension A F B E R270
A--F blue
B--E blue

Y = -X+2*foot X origin circumcenter X C D

circumcircle X C D red
S = extension Y E U_1 V_1
T = extension Y F U_2 V_2
E--Y--F brown

X' = dir 300
C' = extension Xp A U_1 V_1 R170
D' = extension Xp B U_2 V_2 R10
A--Xp--B orange

L = extension S Y Cp Xp

*/
\end{asy}
\end{center}

The main idea is as follows:
\begin{claim*}
  The point $L = \ol{EY} \cap \ol{AX'}$ lies on the radical axis.
  By symmetry, so does the point $M = \ol{FY} \cap \ol{BX'}$ (not pictured).
\end{claim*}
\begin{proof}
  Again by Reim's theorem, $SC'YX'$ is cyclic.
  Hence we have
  \[ \opname{pow}(L, X'C'D') = LC' \cdot LX'
    = LS \cdot LY = \opname{pow}(L, XCD). \qedhere \]
\end{proof}

To conclude, note that by Pascal theorem on
\[ EYFAX'B \]
it follows $K$, $L$, $M$ are collinear,
as needed.

\begin{remark*}
  All the conditions about $U_1$, $V_1$, $U_2$, $V_2$
  at the beginning are there to eliminate configuration issues,
  making the problem less obnoxious to the contestant.

  In particular, without the various assumptions,
  there exist configurations in which the point $K$ is at infinity.
  In these cases, the center of $XCD$ moves along a fixed line.
\end{remark*}
\pagebreak

\subsection{USA TST 2021/3, proposed by Gabriel Carroll}
\textsl{Available online at \url{https://aops.com/community/p20672681}.}
\begin{mdframed}[style=mdpurplebox,frametitle={Problem statement}]
Find all functions $f \colon \RR \to \RR$ that satisfy the inequality
\[
  f(y) -
  \left(\frac{z-y}{z-x} f(x) + \frac{y-x}{z-x}f(z)\right) \leq
  f\left(\frac{x+z}{2}\right) - \frac{f(x)+f(z)}{2}
\]
for all real numbers $x < y < z$.
\end{mdframed}
Answer: all functions of the form $f(y) = a y^2 + by + c$, where
$a, b, c$ are constants with $a \leq 0$.

If $I = (x,z)$ is an interval,
we say that a real number $\alpha$ is a
\emph{supergradient} of $f$ at $y \in I$
if we always have
\[ f(t) \le f(y) + \alpha(t-y) \]
for every $t \in I$.
(This inequality may be familiar as the so-called ``tangent line trick''.
A cartoon of this situation is drawn below for intuition.)
We will also say $\alpha$ is a supergradient of $f$ at $y$,
without reference to the interval,
if $\alpha$ is a supergradient of \emph{some} open interval containing $y$.
\begin{center}
\begin{asy}
  size(4cm);
  pair X = (0,0);
  pair Z = (7,3);
  pair Y = (3,4);
  draw(X..Y..Z, blue);
  pair t = 3*dir(X..Y..Z, 1);
  draw( (Y-t)--(Y+t), red, Arrows);
  label(rotate(15)*"slope $\alpha$", Y+0.45*t, dir(90), red );
  dot("$x$", X, dir(-90), blue);
  dot("$z$", Z, dir(-90), blue);
  dot("$y$", Y, dir(90), blue);
\end{asy}
\qquad
\begin{asy}
  size(5cm);
  pair X = (-3,-9/3);
  pair Z = (2,-4/3);
  draw( X..(-2,-4/3)..(-1,-1/3)..(0,0)..(1,-1/3)..Z, blue );
  pair Y = (-0.5,-0.25/3);
  draw(X--Z, deepgreen);
  draw((Y-1.5*dir(Z-X))--(Y+1.5*dir(Z-X)), red, Arrows );
  label(rotate(18)*"slope $\frac{f(z)-f(x)}{z-x}$",
    midpoint(X--Z), dir(110), deepgreen);
  dot("$x$", X, dir(-90), blue);
  dot("$y = \frac{x+z}{2}$", Y, 1.5*dir(-60), blue);
  dot("$z$", Z, dir(-90), blue);
\end{asy}
\end{center}

\begin{claim*}
  The problem condition is equivalent to asserting
  that $\frac{f(z) - f(x)}{z-x}$ is a supergradient of $f$
  at $\frac{x+z}{2}$ for the interval $(x,z)$, for every $x < z$.
\end{claim*}
\begin{proof}
  This is just manipulation.
\end{proof}

At this point, we may readily verify that all claimed
quadratic functions $f(x) = ax^2+bx+c$ work --- these functions are concave,
so the graphs lie below the tangent line at any point.
Given $x < z$, the tangent at $\frac{x+z}{2}$ has slope
given by the derivative $f'(x)=2ax+b$, that is
\[ f'\left(\frac{x+z}{2}\right) = 2a \cdot \frac{x+z}{2} + b
  = \frac{f(z)-f(x)}{z-x} \]
as claimed.
(Of course, it is also easy to verify the condition directly
by elementary means, since it is just a polynomial inequality.)

Now suppose $f$ satisfies the required condition; we prove that it has
the above form.

\begin{claim*}
  The function $f$ is concave.
\end{claim*}
\begin{proof}
  Choose any $\Delta > \max\{z-y,y-x\}$.
  Since $f$ has a supergradient $\alpha$ at $y$ over the interval
  $(y-\Delta,y+\Delta)$, and this interval includes $x$ and $z$, we have
  \begin{align*}
  \frac{z-y}{z-x}f(x) + \frac{y-x}{z-x}f(z) &\leq
  \frac{z-y}{z-x}(f(y) + \alpha(x-y)) + \frac{y-x}{z-x}(f(y) +
  \alpha(z-y)) \\
  &= f(y).
  \end{align*}
  That is, $f$ is a concave function.
  Continuity follows from the fact that any concave
  function on $\RR$ is automatically continuous.
\end{proof}

\begin{lemma*}
  [see e.g.\ {\url{https://math.stackexchange.com/a/615161}} for picture]
  Any concave function $f$ on $\RR$ is continuous.
\end{lemma*}
\begin{proof}
  Suppose we wish to prove continuity at $p \in \RR$.
  Choose any real numbers $a$ and $b$ with $a < p < b$.
  For any $0 < \eps < \max(b-p,p-a)$ we always have
  \[ f(p) + \frac{f(b)-f(p)}{b-p} \eps \le f(p+\eps) \le f(p) + \frac{f(p)-f(a)}{p-a} \eps \]
  which implies right continuity; the proof for left continuity is the same.
\end{proof}

\begin{claim*}
  The function $f$ cannot have more than one supergradient
  at any given point.
\end{claim*}
\begin{proof}
  Fix $y \in \RR$.
  For $t > 0$, let's define the function
  \[ g(t) = \frac{f(y)-f(y-t)}{t} - \frac{f(y+t)-f(y)}{t}. \]
  We contend that $g(\eps) \le \frac35g(3\eps)$
  for any $\eps > 0$.
  Indeed by the problem condition,

  \noindent
  \begin{minipage}{0.5\textwidth}
  \begin{align*}
    f(y) &\le f(y-\eps) + \frac{f(y+\eps)-f(y-3\eps)}{4} \\
    f(y) &\le f(y+\eps) - \frac{f(y+3\eps)-f(y-\eps)}{4}.
  \end{align*}
  Summing gives the desired conclusion.
  \end{minipage}
  \begin{minipage}{0.4\textwidth}
  \begin{asy}
    size(6cm);
    real f(real x) { return -x*x/6; }
    pair A = (-3, f(3));
    pair B = (-2, f(2));
    pair C = (-1, f(1));
    pair D = (-0, f(0));
    pair E = (1, f(1));
    pair F = (2, f(2));
    pair G = (3, f(3));
    draw(A..B..C..D..E..F..G, blue);
    draw(A--E, lightred);
    draw(C--G, deepgreen);
    draw( (C-2*dir(E-A))--(C+2*dir(E-A)), lightred, Arrows );
    draw( (E-2*dir(G-C))--(E+2*dir(G-C)), deepgreen, Arrows );

    dot("$y-3\varepsilon$", A, dir(-90), blue);
    dot("$y-\varepsilon$",  C, dir(-90), blue);
    dot("$y$",  D, dir(-90), blue);
    dot("$y+\varepsilon$",  E, dir(-90), blue);
    dot("$y+3\varepsilon$", G, dir(-90), blue);
  \end{asy}
  \end{minipage}

  Now suppose that $f$ has two supergradients $\alpha < \alpha'$ at point $y$.
  For small enough $\eps$, we should have
  we have $f(y-\eps) \leq f(y) - \alpha'\eps$
  and $f(y+\eps) \leq f(y) + \alpha\eps$, hence
  \[ g(\eps) = \frac{f(y)-f(y-\eps)}{\eps} - \frac{f(y+\eps)-f(y)}{\eps}
    \geq \alpha' - \alpha. \]
  This is impossible since $g(\eps)$ may be arbitrarily small.
\end{proof}

\begin{claim*}
  The function $f$ is quadratic on the rational numbers.
\end{claim*}
\begin{proof}
Consider any four-term arithmetic progression $x, x+d, x+2d, x+3d$.
Because $(f(x+2d)-f(x+d))/d$ and $(f(x+3d)-f(x))/3d$ are both
supergradients of $f$ at the point $x+3d/2$, they must be equal, hence
\begin{equation} \label{quadratic}
  f(x+3d) - 3f(x+2d) + 3f(x+d) - f(x) = 0.
\end{equation}
If we fix $d = 1/n$, it follows inductively
that $f$ agrees with a quadratic function $\wt f_n$ on the set $\frac1n \ZZ$.
On the other hand, for any $m \neq n$,
we apparently have $\wt f_n = \wt f_{mn} = \wt f_m$,
so the quadratic functions on each ``layer'' are all equal.
\end{proof}
Since $f$ is continuous, it follows $f$ is quadratic, as needed.

\begin{remark*}
  [Alternate finish using differentiability due to Michael Ren]
  In the proof of the main claim (about uniqueness of supergradients),
  we can actually notice the two terms $\frac{f(y)-f(y-t)}{t}$
  and $\frac{f(y+t)-f(y)}{t}$ in the definition of $g(t)$ are both monotonic
  (by concavity).
  Since we supplied a proof that $\lim_{t \to 0} g(t) = 0$,
  we find $f$ is differentiable.

  Now, if the derivative at some point exists,
  it must coincide with all the supergradients;
  (informally, this is why ``tangent line trick'' always has the slope
  as the derivative, and formally, we use the mean value theorem).
  In other words, we must have
  \[ f(x+y) - f(x-y) = 2f'(x) \cdot y \]
  holds for all real numbers $x$ and $y$.
  By choosing $y=1$ we obtain that $f'(x) = f(x+1) - f(x-1)$
  which means $f'$ is also continuous.

  Finally differentiating both sides with respect to $y$ gives
  \[ f'(x+y) - f'(x-y) = 2f'(x) \]
  which means $f'$ obeys Jensen's functional equation.
  Since $f'$ is continuous, this means $f'$ is linear.
  Thus $f$ is quadratic, as needed.
\end{remark*}
\pagebreak


\end{document}
