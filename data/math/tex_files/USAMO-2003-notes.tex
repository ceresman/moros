% © Evan Chen
% Downloaded from https://web.evanchen.cc/

\documentclass[11pt]{scrartcl}
\usepackage[sexy]{evan}
\ihead{\footnotesize\textbf{\thetitle}}
\ohead{\footnotesize\href{http://web.evanchen.cc}{\ttfamily web.evanchen.cc},
    updated \today}
\title{USAMO 2003 Solution Notes}
\date{\today}

\begin{document}

\maketitle

\begin{abstract}
This is a compilation of solutions
for the 2003 USAMO.
Some of the solutions are my own work,
but many are from the official solutions provided by the organizers
(for which they hold any copyrights),
and others were found by users on the Art of Problem Solving forums.

These notes will tend to be a bit more advanced and terse than the ``official''
solutions from the organizers.
In particular, if a theorem or technique is not known to beginners
but is still considered ``standard'', then I often prefer to
use this theory anyways, rather than try to work around or conceal it.
For example, in geometry problems I typically use directed angles
without further comment, rather than awkwardly work around configuration issues.
Similarly, sentences like ``let $\mathbb{R}$ denote the set of real numbers''
are typically omitted entirely.

Corrections and comments are welcome!
\end{abstract}

\tableofcontents
\newpage

\addtocounter{section}{-1}
\section{Problems}
\begin{enumerate}[\bfseries 1.]
\item %% Problem 1
Prove that for every positive integer $n$
there exists an $n$-digit number divisible by $5^n$
all of whose digits are odd.

\item %% Problem 2
A convex polygon $\mathcal{P}$ in the plane
is dissected into smaller convex polygons by
drawing all of its diagonals.
The lengths of all sides and all diagonals of the
polygon $\mathcal{P}$ are rational numbers.
Prove that the lengths of all sides of all polygons
in the dissection are also rational numbers.

\item %% Problem 3
Let $n$ be a positive integer.
For every sequence of integers
\[ A = (a_0, \; a_1, \; a_2, \; \dots, a_n) \]
satisfying $0 \le a_i \le i$, for $i=0,\dots,n$,
we define another sequence
\[ t(A)= (t(a_0), \; t(a_1), \; t(a_2), \; \dots, \; t(a_n)) \]
by setting $t(a_i)$ to be the number of terms in the
sequence $A$ that precede the term $a_i$ and are different from $a_i$.
Show that, starting from any sequence $A$ as above,
fewer than $n$ applications of the  transformation $t$
lead to a sequence $B$ such that $t(B) = B$.

\item %% Problem 4
Let $ABC$ be a triangle.
A circle passing through $A$ and $B$
intersects segments $AC$ and $BC$ at $D$ and $E$, respectively.
Lines $AB$ and $DE$ intersect at $F$,
while lines $BD$ and $CF$ intersect at $M$.
Prove that $MF = MC$ if and only if $MB \cdot MD = MC^2$.

\item %% Problem 5
Let $a$, $b$, $c$ be positive real numbers.
Prove that
\[ \frac{(2a+b+c)^2}{2a^2+(b+c)^2}
+ \frac{(2b+c+a)^2}{2b^2+(c+a)^2}
+ \frac{(2c+a+b)^2}{2c^2+(a+b)^2} \le 8.  \]

\item %% Problem 6
At the vertices of a regular hexagon are written six nonnegative integers
whose sum is $2003^{2003}$.
Bert is allowed to make moves of the following form:
he may pick a vertex and replace the number written
there by the absolute value of the difference between the numbers
written at the two neighboring vertices.
Prove that Bert can make a sequence of moves,
after which the number $0$ appears at all six vertices.

\end{enumerate}
\pagebreak

\section{Solutions to Day 1}
\subsection{USAMO 2003/1, proposed by Titu Andreescu}
\textsl{Available online at \url{https://aops.com/community/p336189}.}
\begin{mdframed}[style=mdpurplebox,frametitle={Problem statement}]
Prove that for every positive integer $n$
there exists an $n$-digit number divisible by $5^n$
all of whose digits are odd.
\end{mdframed}
This is immediate by induction on $n$.
For $n = 1$ we take $5$;
moving forward if $M$ is a working $n$-digit number then exactly one of
\begin{align*}
  N_1 &= 10^n + M \\
  N_3 &= 3 \cdot 10^n + M \\
  N_5 &= 5 \cdot 10^n + M \\
  N_7 &= 7 \cdot 10^n + M \\
  N_9 &= 9 \cdot 10^n + M
\end{align*}
is divisible by $5^{n+1}$;
as they are all divisible by $5^n$
and $N_k/5^n$ are all distinct.
\pagebreak

\subsection{USAMO 2003/2}
\textsl{Available online at \url{https://aops.com/community/p336193}.}
\begin{mdframed}[style=mdpurplebox,frametitle={Problem statement}]
A convex polygon $\mathcal{P}$ in the plane
is dissected into smaller convex polygons by
drawing all of its diagonals.
The lengths of all sides and all diagonals of the
polygon $\mathcal{P}$ are rational numbers.
Prove that the lengths of all sides of all polygons
in the dissection are also rational numbers.
\end{mdframed}
Suppose $AB$ is a side of a polygon in the dissection,
lying on diagonal $XY$, with $X$, $A$, $B$, $Y$ in that order.
Then \[ AB = XY - XA - YB. \]
In this way, we see that it actually just suffices to
prove the result for a quadrilateral.

We present two approaches to this end.

\paragraph{First approach (trig).}
Consider quadrilateral $ABCD$.
There are twelve angles one can obtain using three of its four vertices,
three at each vertex; denote this set of $12$ angles by $S$
Note that:
\begin{itemize}
  \ii The law of cosines implies $\cos \theta \in \QQ$ for each $\theta \in S$.
  \ii Hence, $(\sin \theta)^2 \in \QQ$ for $\theta \in S$.
  (This is because $\sin\theta^2+\cos^2\theta$.)
\end{itemize}
We say two angles $\theta_1$ and $\theta_2$ are
\emph{equivalent} if $\frac{\sin \theta_1}{\sin \theta_2}$
This is the same as saying,
when $\sin\theta_1$ and $\sin\theta_2$ are written in simplest radical form,
the part under the square root is the same.

Now we contend:
\begin{claim*}
  The angles $\angle BAC$, $\angle CAD$, $\angle BAD$ are equivalent.
\end{claim*}
\begin{proof}
  Note that
  \[ \QQ \ni \cos(\angle BAD) = \cos \angle BAC \cos \angle CAD - \sin \angle BAC \sin \angle CAD \]
  so $\angle BAC$ and $\angle CAD$ are equivalent.
  Then
  \[ \sin (\angle BAD) = \sin \angle BAC \cos \angle CAD + \cos \angle BAC \sin \angle CAD \]
  implies $\angle BAD$ is equivalent to those two.
\end{proof}

\begin{claim*}
  The angles $\angle BAD$, $\angle DBA$, $\angle ADB$ are equivalent.
\end{claim*}
\begin{proof}
  Law of sines on $\triangle BAD$.
\end{proof}

Iterating the argument implies that all angles are equivalent.

Now, if $AB$ and $CD$ meet at $E$,
the law of sines on $\triangle AEB$, etc.\ implies the result.

\paragraph{Second approach (barycentric coordinates).}
To do this, we apply barycentric coordinates.
Consider quadrilateral $ABDC$
(note the changed order of vertices),
with $A=(1,0,0)$, $B=(0,1,0)$, $C=(0,0,1)$.
Let $D = (x,y,z)$, with $x+y+z=1$.
By hypothesis, each of the numbers
\begin{align*}
  -a^2yz + b^2(1-x)z + c^2(1-x)y &= AD^2 \\
  a^2(1-y)z + b^2zx + c^2(1-y)x &= BD^2 \\
  -a^2(1-z)y - b^2(1-z)x + c^2xy &= CD^2
\end{align*}
is rational. Let $W = a^2yz + b^2zx + c^2xy$. Then,
\begin{align*}
  b^2z + c^2y &=  AD^2 + W \\
  a^2z + c^2x &= BD^2 + W \\
  a^2y + b^2x &= CD^2 + W.
\end{align*}
This implies that $AD^2 + BD^2 + 2W - c^2 = 2S_C z$ and cyclically
(as usual $2S_C = a^2+b^2-c^2$).
If any of $S_A$, $S_B$, $S_C$ are zero, then we deduce $W$ is rational.
Otherwise, we have that
\[ 1 = x+y+z = \sum_{\text{cyc}} \frac{AD^2 + BD^2 + 2W - c^2}{2S_C} \]
which implies that $W$ is rational,
because it appears with coefficient
$\frac{1}{S_A} + \frac{1}{S_B} + \frac{1}{S_C} \neq 0$
(since $S_{BC} + S_{CA} + S_{AB}$ is actually the area of $ABC$).

Hence from the rationality of $W$,
we deduce that $x$ is rational as long as $S_A \neq 0$,
and similarly for the others.
So at most one of $x$, $y$, $z$ is irrational,
but since $x+y+z=1$ this implies they are all rational.

Finally, if $P = \ol{AD} \cap \ol{BC}$
then $AP = \frac{1}{y+z} AD$, so $AP$ is rational too,
completing the proof.

\begin{remark*}
  After the reduction to quadrilateral,
  a third alternate approach goes by quoting Putnam 2018 A6,
  reproduced below:
  \begin{quote}
  Four points are given in the plane, with no three collinear,
  such that the squares of the $\binom42=6$ pairwise distances are all rational.
  Show that the ratio of the areas between any two of the
  $\binom43=4$ triangles determined by these points is also rational.
  \end{quote}
  If $ABCD$ is the quadrilateral,
  the heights from $C$ and $D$ to $AB$ have rational ratio.
  Letting $P = AC \cap BD$,
  we see $AP/AB$ can be shown as rational via coordinates, as needed.
\end{remark*}
\pagebreak

\subsection{USAMO 2003/3}
\textsl{Available online at \url{https://aops.com/community/p336202}.}
\begin{mdframed}[style=mdpurplebox,frametitle={Problem statement}]
Let $n$ be a positive integer.
For every sequence of integers
\[ A = (a_0, \; a_1, \; a_2, \; \dots, a_n) \]
satisfying $0 \le a_i \le i$, for $i=0,\dots,n$,
we define another sequence
\[ t(A)= (t(a_0), \; t(a_1), \; t(a_2), \; \dots, \; t(a_n)) \]
by setting $t(a_i)$ to be the number of terms in the
sequence $A$ that precede the term $a_i$ and are different from $a_i$.
Show that, starting from any sequence $A$ as above,
fewer than $n$ applications of the  transformation $t$
lead to a sequence $B$ such that $t(B) = B$.
\end{mdframed}
We go by strong induction on $n$
with the base cases $n=1$ and $n=2$ done by hand.
Consider two cases:
\begin{itemize}
\ii If $a_0 = 0$ and $a_1 = 1$,
then $1 \le t(a_i) \le i$ for $i \ge 1$;
now apply induction to
\[ \left(t(a_1)-1, \; t(a_2)-1, \;
  \dots, \; t(a_n)-1\right). \]

\ii Otherwise, assume that $a_0 = a_1 = \dots = a_{k-1} = 0$
but $a_k \neq 0$, where $k \ge 2$.
Assume $k < n$ or it's obvious.
Then $t(a_i) \neq 0$ for $i \ge k$,
thus $t(t(a_i)) \ge k$ for $i \ge k$,
and we can apply induction hypothesis to
\[ \left( t(t(a_k))-k, \; \dots,
  \; t(t(a_n))-k \right). \]
\end{itemize}
\pagebreak

\section{Solutions to Day 2}
\subsection{USAMO 2003/4, proposed by Titu Andreescu, Zuming Feng}
\textsl{Available online at \url{https://aops.com/community/p336205}.}
\begin{mdframed}[style=mdpurplebox,frametitle={Problem statement}]
Let $ABC$ be a triangle.
A circle passing through $A$ and $B$
intersects segments $AC$ and $BC$ at $D$ and $E$, respectively.
Lines $AB$ and $DE$ intersect at $F$,
while lines $BD$ and $CF$ intersect at $M$.
Prove that $MF = MC$ if and only if $MB \cdot MD = MC^2$.
\end{mdframed}
Ceva theorem plus the similar triangles.

\begin{center}
\begin{asy}
  size(7.5cm);
  pair C = Drawing("C", (3,0), dir(-45));
  pair D = Drawing("D", (0,0));
  pair M = Drawing("M", (1,-0.7));
  pair F = Drawing("F", 2*M-C);
  pair B = Drawing("B", conj(C/M) * (M-C)+C, dir(130));
  pair E = Drawing("E", extension(D,F,B,C), dir(40));
  pair A = Drawing("A", extension(D,C,B,F), dir(225));
  draw(A--B--C--cycle);
  draw(circumcircle(B,D,E), dashed);
  draw(A--F, dotted);
  draw(E--F--C, dotted);
  draw(B--M, dotted);
\end{asy}
\end{center}

We know unconditionally that
\[ \dang CBD = \dang EBD = \dang EAD = \dang EAC. \]
Moreover, by Ceva's theorem on $\triangle BCF$,
we have $MF = MC \iff \ol{FC} \parallel \ol{AE}$.
So we have the equivalences
\begin{align*}
  MF = MC &\iff \ol{FC} \parallel \ol{AE} \\
  &\iff \dang FCA = \dang EAC \\
  &\iff \dang MCD = \dang CBD \\
  &\iff MC^2 = MB \cdot MD.
\end{align*}
\pagebreak

\subsection{USAMO 2003/5, proposed by Zuming Feng, Titu Andreescu}
\textsl{Available online at \url{https://aops.com/community/p336208}.}
\begin{mdframed}[style=mdpurplebox,frametitle={Problem statement}]
Let $a$, $b$, $c$ be positive real numbers.
Prove that
\[ \frac{(2a+b+c)^2}{2a^2+(b+c)^2}
+ \frac{(2b+c+a)^2}{2b^2+(c+a)^2}
+ \frac{(2c+a+b)^2}{2c^2+(a+b)^2} \le 8.  \]
\end{mdframed}
This is a canonical example of tangent line trick.
Homogenize so that $a + b + c = 3$.
The desired inequality reads
\[ \sum_{\text{cyc}} \frac{(a+3)^2}{2a^2+(3-a)^2} \le 8. \]
This follows from
\[ f(x) = \frac{(x+3)^2}{2x^2+(3-x)^2}
  \le \frac{1}{3} (4x + 4) \]
which can be checked as
$\frac 13 (4x+4)(2x^2+(3-x)^2) - (x+3)^2
= (x-1)^2 (4x+3) \ge 0$.
\pagebreak

\subsection{USAMO 2003/6}
\textsl{Available online at \url{https://aops.com/community/p336210}.}
\begin{mdframed}[style=mdpurplebox,frametitle={Problem statement}]
At the vertices of a regular hexagon are written six nonnegative integers
whose sum is $2003^{2003}$.
Bert is allowed to make moves of the following form:
he may pick a vertex and replace the number written
there by the absolute value of the difference between the numbers
written at the two neighboring vertices.
Prove that Bert can make a sequence of moves,
after which the number $0$ appears at all six vertices.
\end{mdframed}
If $a \le b \le c$ are \emph{odd} integers,
the configuration which has $(a,b-a,b,c-b,c,c-a)$ around the hexagon
in some order (up to cyclic permutation and reflection)
is said to be \emph{great} (picture below).

\begin{claim*}
  We can reach a great configuration
  from any configuration with odd sum.
\end{claim*}
\begin{proof}
  We should be able to find an equilateral triangle
  whose vertices have odd sum.
  If all three vertices are odd, then we are already done.
  Otherwise, operate as in the following picture (modulo $2$).
  \begin{center}
  \begin{asy}
  size(10cm);
  picture[] h = new picture[4];
  h[0] = new picture;
  h[1] = new picture;
  h[2] = new picture;
  h[3] = new picture;

  pen cu = blue + fontsize(9pt);
  pen cd = deepgreen + fontsize(7pt);
  pen mu = lightred + fontsize(9pt);
  pen md = orange + fontsize(7pt);

  for (int i=0; i<=3; ++i) {
  draw(h[i], dir(30)--dir(150)--dir(270)--cycle, cd);
  draw(h[i], dir(90)--dir(210)--dir(330)--cycle, cu);
  draw(h[i], dir(30)--dir(90)--dir(150)--dir(210)--dir(270)--dir(330)--cycle, lightgrey);
  }

  label(h[0], "$1$", dir(90), dir(90), cu);
  label(h[0], "$\ast$", dir(150), dir(150), cd);
  label(h[0], "$0$", dir(210), dir(210), cu);
  label(h[0], "$\ast$", dir(270), dir(270), cd);
  label(h[0], "$0$", dir(330), dir(330), cu);
  label(h[0], "$\ast$", dir(30), dir(30), cd);

  label(h[1], "$1$", dir(90), dir(90), cu);
  label(h[1], "$1$", dir(150), dir(150), md);
  label(h[1], "$0$", dir(210), dir(210), cu);
  label(h[1], "$0$", dir(270), dir(270), md);
  label(h[1], "$0$", dir(330), dir(330), cu);
  label(h[1], "$1$", dir(30), dir(30), md);

  label(h[2], "$1$", dir(90), dir(90), cu);
  label(h[2], "$1$", dir(150), dir(150), cd);
  label(h[2], "$1$", dir(210), dir(210), mu);
  label(h[2], "$0$", dir(270), dir(270), cd);
  label(h[2], "$1$", dir(330), dir(330), mu);
  label(h[2], "$1$", dir(30), dir(30), cd);

  label(h[3], "$1$", dir(90), dir(90), cu);
  label(h[3], "$0$", dir(150), dir(150), md);
  label(h[3], "$1$", dir(210), dir(210), cu);
  label(h[3], "$0$", dir(270), dir(270), md);
  label(h[3], "$1$", dir(330), dir(330), cu);
  label(h[3], "$0$", dir(30), dir(30), md);

  add(shift(0,0)*h[0]);
  add(shift(3,0)*h[1]);
  add(shift(6,0)*h[2]);
  add(shift(9,0)*h[3]);
  \end{asy}
  \end{center}
  Thus we arrived at a great configuration.
\end{proof}

\begin{claim*}
  Bert's goal is possible for all great configurations.
\end{claim*}
\begin{proof}
  [Proof, suggested by Haoran Chen]
  If $a=b=c$ then we have $(t,0,t,0,t,0)$ which is obviously winnable.

  Otherwise, we can perform the following three operations
  shown in the figure below,
  which yield a great configuration whose odd entries
  are $a$, $b$, $|c-2a|$.

  \begin{center}
  \begin{asy}
  size(12.5cm);
  picture[] h = new picture[5];
  h[0] = new picture;
  h[1] = new picture;
  h[2] = new picture;
  h[3] = new picture;
  h[4] = new picture;

  pen cu = blue + fontsize(9pt);
  pen cd = deepgreen + fontsize(7pt);
  pen mu = red + fontsize(9pt);
  pen md = red + fontsize(9pt);

  for (int i=0; i<=4; ++i) {
  draw(h[i], dir(30)--dir(150)--dir(270)--cycle, cd);
  draw(h[i], dir(90)--dir(210)--dir(330)--cycle, cu);
  draw(h[i], dir(30)--dir(90)--dir(150)--dir(210)--dir(270)--dir(330)--cycle, lightgrey);
  }

  label(h[0], "$a$", dir(90), dir(90), cu);
  label(h[0], "$b-a$", dir(150), dir(150), cd);
  label(h[0], "$b$", dir(210), dir(210), cu);
  label(h[0], "$c-b$", dir(270), dir(270), cd);
  label(h[0], "$c$", dir(330), dir(330), cu);
  label(h[0], "$c-a$", dir(30), dir(30), cd);

  label(h[1], "$a$", dir(90), dir(90), cu);
  label(h[1], "$b-a$", dir(150), dir(150), cd);
  label(h[1], "$b$", dir(210), dir(210), cu);
  label(h[1], "$c-b$", dir(270), dir(270), cd);
  label(h[1], "$b-a$", dir(330), dir(330), mu);
  label(h[1], "$c-a$", dir(30), dir(30), cd);

  label(h[2], "$a$", dir(90), dir(90), cu);
  label(h[2], "$b-a$", dir(150), dir(150), cd);
  label(h[2], "$b$", dir(210), dir(210), cu);
  label(h[2], "$a$", dir(270), dir(270), md);
  label(h[2], "$b-a$", dir(330), dir(330), cu);
  label(h[2], "$c-a$", dir(30), dir(30), cd);

  label(h[3], "$a$", dir(90), dir(90), cu);
  label(h[3], "$b-a$", dir(150), dir(150), cd);
  label(h[3], "$b$", dir(210), dir(210), cu);
  label(h[3], "$a$", dir(270), dir(270), cd);
  label(h[3], "$|c-2a|$", dir(330), dir(330), mu);
  label(h[3], "$c-a$", dir(30), dir(30), cd);

  label(h[4], "$a$", dir(90), dir(90), cu);
  label(h[4], "$b$", dir(210), dir(210), cu);
  label(h[4], "$|c-2a|$", dir(330), dir(330), cu);

  add(shift(0,3)*h[0]);
  add(shift(4,3)*h[1]);
  add(shift(8,3)*h[2]);
  add(shift(2,0)*h[3]);
  add(shift(6,0)*h[4]);
  \end{asy}
  \end{center}

  Since $|c-2a| < c$ unless $a=b=c$,
  this decreases the sum.
  So an induction now completes the problem.
%  This is annoying, but straightforward.
%  Our standing assumption is $a \neq c$ (but possibly $b=c$).
%  It's already obvious that $|c-2a| < c$,
%  so focus on the last term.
%  If $c > 2b$, then $\left\lvert (c-2b)-(c-a) \right\rvert =
%  \left\lvert 2b-a \right\rvert < c$ as well for $a \neq c$.
%  When $c \le 2b$ we instead have
%  $\left\lvert (2b-c)-(c-a) \right\rvert
%  \le \max \left( 2b-c, c-a \right)$
%  with equality if and only if $c-a = 0$; but $2b-c \le c$ as needed.
%  Thus, in all situations we have
%  \[ c \neq a \implies
%    \max\left( \left\lvert \left\lvert c-2b \right\rvert - (c-a)
%    \right\rvert, \left\lvert c-2a \right\rvert \right) < c. \]
%  Now denote the new odd entries by $a' \le b' \le c'$ (in some order).
%  If $b < c$ then $c' < c$,
%  while if $b = c$ then $c' = b$ but $b' < c = b$.
%  Thus $(c', b', a')$ precedes $(c, b, a)$ lexicographically, and we
%  can induct down.
\end{proof}

\begin{remark*}
  One simple idea might be to try to overwrite
  the maximum number at each point, decreasing the sum.
  However, this fails on the arrangement $(t,t,0,t,t,0)$.

  Unfortunately, this issue is actually fatal,
  as the problem has a hidden parity obstruction.
  The configuration $(1,1,0,1,1,0) \bmod 2$ is invariant modulo $2$,
  and so Bert can walk into a ``fatal death-trap'' of this shape
  long before the numbers start becoming equal/zero/etc.
  In other words, you can mess up on the first move!
  This is why the initial sum is given to be odd;
  however, it's not possible for Bert to win
  so one essentially has to ``tip-toe'' around the $110110$ trap
  any time one leaves the space of odd sum.
  That's why the great configurations defined above serve as an anchor,
  making sure we never veer too far from the
  safe $101010$ configuration.
\end{remark*}

\begin{remark*}
  On the other hand, many other approaches are possible
  which anchor around a different parity configuration,
  like $100000$ for example.
  The choice of $101010$ by me is due to symmetry --- ostensibly,
  if it worked, there should be fewer cases.
\end{remark*}
\pagebreak


\end{document}
