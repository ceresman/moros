% © Evan Chen
% Downloaded from https://web.evanchen.cc/

\documentclass[11pt]{scrartcl}
\usepackage[sexy]{evan}
\ihead{\footnotesize\textbf{\thetitle}}
\ohead{\footnotesize\href{http://web.evanchen.cc}{\ttfamily web.evanchen.cc},
    updated \today}
\title{USAMO 2013 Solution Notes}
\date{\today}

\begin{document}

\maketitle

\begin{abstract}
This is a compilation of solutions
for the 2013 USAMO.
Some of the solutions are my own work,
but many are from the official solutions provided by the organizers
(for which they hold any copyrights),
and others were found by users on the Art of Problem Solving forums.

These notes will tend to be a bit more advanced and terse than the ``official''
solutions from the organizers.
In particular, if a theorem or technique is not known to beginners
but is still considered ``standard'', then I often prefer to
use this theory anyways, rather than try to work around or conceal it.
For example, in geometry problems I typically use directed angles
without further comment, rather than awkwardly work around configuration issues.
Similarly, sentences like ``let $\mathbb{R}$ denote the set of real numbers''
are typically omitted entirely.

Corrections and comments are welcome!
\end{abstract}

\tableofcontents
\newpage

\addtocounter{section}{-1}
\section{Problems}
\begin{enumerate}[\bfseries 1.]
\item %% Problem 1
In triangle $ABC$,
points $P$, $Q$, $R$ lie on sides $BC$, $CA$, $AB$, respectively.
Let $\omega_A$, $\omega_B$, $\omega_C$ denote the
circumcircles of triangles $AQR$, $BRP$, $CPQ$, respectively.
Given the fact that segment $AP$ intersects
$\omega_A$, $\omega_B$, $\omega_C$ again at $X$, $Y$, $Z$ respectively,
prove that $YX/XZ=BP/PC$.

\item %% Problem 2
For a positive integer $n\geq 3$ plot $n$
equally spaced points around a circle.
Label one of them $A$, and place a marker at $A$.
One may move the marker forward in a clockwise direction
to either the next point or the point after that.
Hence there are a total of $2n$ distinct moves available;
two from each point.
Let $a_n$ count the number of ways to advance
around the circle exactly twice,
beginning and ending at $A$, without repeating a move.
Prove that $a_{n-1}+a_n=2^n$ for all $n\geq 4$.

\item %% Problem 3
Let $n$ be a positive integer.
There are $\tfrac{n(n+1)}{2}$ tokens,
each with a black side and a white side,
arranged into an equilateral triangle,
with the biggest row containing $n$ tokens.
Initially, each token has the white side up.
An operation is to choose a line parallel to the sides of the triangle,
and flip all the token on that line.
A configuration is called admissible if it can be obtained from the
initial configuration by performing a finite number of operations.
For each admissible configuration $C$,
let $f(C)$ denote the smallest number of operations required to
obtain $C$ from the initial configuration.
Find the maximum value of $f(C)$,
where $C$ varies over all admissible configurations.

\item %% Problem 4
Find all real numbers $x,y,z \ge 1$ satisfying
\[ \min \left( \sqrt{x+xyz}, \sqrt{y+xyz}, \sqrt{z+xyz} \right)
  = \sqrt{x-1} + \sqrt{y-1} + \sqrt{z-1}. \]

\item %% Problem 5
Let $m$ and $n$ be positive integers.
Prove that there exists a positive integer $c$
such that $cm$ and $cn$ have the same nonzero decimal digits.

\item %% Problem 6
Let $ABC$ be a triangle.
Find all points $P$ on segment $BC$ satisfying the following property:
If $X$ and $Y$ are the intersections of line $PA$ with the common external
tangent lines of the circumcircles of triangles $PAB$ and $PAC$, then
\[ \left( \frac{PA}{XY} \right)^2 + \frac{PB\cdot PC}{AB\cdot AC} = 1.\]

\end{enumerate}
\pagebreak

\section{Solutions to Day 1}
\subsection{USAMO 2013/1, proposed by Zuming Feng}
\textsl{Available online at \url{https://aops.com/community/p3041822}.}
\begin{mdframed}[style=mdpurplebox,frametitle={Problem statement}]
In triangle $ABC$,
points $P$, $Q$, $R$ lie on sides $BC$, $CA$, $AB$, respectively.
Let $\omega_A$, $\omega_B$, $\omega_C$ denote the
circumcircles of triangles $AQR$, $BRP$, $CPQ$, respectively.
Given the fact that segment $AP$ intersects
$\omega_A$, $\omega_B$, $\omega_C$ again at $X$, $Y$, $Z$ respectively,
prove that $YX/XZ=BP/PC$.
\end{mdframed}
Let $M$ be the concurrence point of
$\omega_A$, $\omega_B$, $\omega_C$
(by Miquel's theorem).


\begin{center}
  \begin{asy}
    size(8cm);
    defaultpen(fontsize(9pt));
    pair A = dir(110);
    pair B = dir(210);
    pair C = dir(330);
    pair P = 0.4*B+0.6*C;
    pair Q = 0.4*C+0.6*A;
    pair R = 0.7*A+0.3*B;
    draw(B--A--C);

    draw(circumcircle(A, Q, R), blue);
    draw(circumcircle(B, R, P), blue);
    draw(circumcircle(C, P, Q), blue);

    pair O_A = circumcenter(A, Q, R);
    pair O_B = circumcenter(B, R, P);
    pair O_C = circumcenter(C, P, Q);

    pair M = -P+2*foot(P, O_B, O_C);
    pair X = -A+2*foot(O_A, A, P);
    pair Y = -P+2*foot(O_B, A, P);
    pair Z = -P+2*foot(O_C, A, P);

    draw(X--M--P, dotted);
    draw(R--Q);
    draw(A--Y);
    draw(Z--P);

    draw(B--C, red);
    draw(Y--Z, red);

    dot("$A$", A, dir(A));
    dot("$B$", B, dir(B));
    dot("$C$", C, dir(C));
    dot("$P$", P, dir(P));
    dot("$Q$", Q, dir(Q));
    dot("$R$", R, dir(R));
    dot("$M$", M, dir(-90));
    dot("$X$", X, dir(225));
    dot("$Y$", Y, dir(45));
    dot("$Z$", Z, dir(180));

    /* Source generated by TSQ
    A = dir 110
    B = dir 210
    C = dir 330
    P = 0.4*B+0.6*C
    Q = 0.4*C+0.6*A
    R = 0.7*A+0.3*B
    B--A--C

    circumcircle A Q R blue
    circumcircle B R P blue
    circumcircle C P Q blue

    O_A := circumcenter A Q R
    O_B := circumcenter B R P
    O_C := circumcenter C P Q

    M = -P+2*foot P O_B O_C R-90
    X = -A+2*foot O_A A P R225
    Y = -P+2*foot O_B A P R45
    Z = -P+2*foot O_C A P R180

    X--M--P dotted
    R--Q
    A--Y
    Z--P

    B--C red
    Y--Z red
    */
  \end{asy}
\end{center}


Then $M$ is the center of a spiral similarity sending
$\ol{YZ}$ to $\ol{BC}$.
So it suffices to show that this spiral similarity
also sends $X$ to $P$, but
\[ \dang MXY = \dang MXA = \dang MRA = \dang MRB = \dang MPB \]
so this follows.
\pagebreak

\subsection{USAMO 2013/2, proposed by Kiran Kedlaya}
\textsl{Available online at \url{https://aops.com/community/p3041823}.}
\begin{mdframed}[style=mdpurplebox,frametitle={Problem statement}]
For a positive integer $n\geq 3$ plot $n$
equally spaced points around a circle.
Label one of them $A$, and place a marker at $A$.
One may move the marker forward in a clockwise direction
to either the next point or the point after that.
Hence there are a total of $2n$ distinct moves available;
two from each point.
Let $a_n$ count the number of ways to advance
around the circle exactly twice,
beginning and ending at $A$, without repeating a move.
Prove that $a_{n-1}+a_n=2^n$ for all $n\geq 4$.
\end{mdframed}
We present two similar approaches.

\paragraph{First solution.}
Imagine the counter is moving along
the set $S = \{0, 1, \dots, 2n\}$ instead,
starting at $0$ and ending at $2n$, in jumps of length $1$ and $2$.
We can then record the sequence of moves as a matrix of the form
\[
  \begin{bmatrix}
    p_0 & p_1 & p_2 & \dots & p_{n-1} & p_n \\
    p_n & p_{n+1} & p_{n+2} & \dots & p_{2n-1} & p_{2n}
  \end{bmatrix}
\]
where $p_i = 1$ if the point $i$ was visited by the counter,
and $p_i = 0$ if the point was not visited by the counter.
Note that $p_0 = p_{2n} = 1$ and the upper-right and lower-left entries are equal.
Then, the problem amounts to finding the number of such matrices
which avoid the contiguous submatrices
\[
  \begin{bmatrix} 0 & 0 \end{bmatrix}
  \qquad
  \begin{bmatrix} 0 \\ 0 \end{bmatrix}
  \qquad
  \begin{bmatrix} 1 & 1 \\ 1 & 1 \end{bmatrix}
\]
which correspond to forbidding jumps of length greater than $2$,
repeating a length $2$ jump and repeating a length $1$ jump.

We give a nice symmetric phrasing suggested by \texttt{fclvbfm934} at
\url{https://aops.com/community/p27834267}.
If we focus on just the three possible column vectors that appear, say
\[
  \mathbf u \coloneqq \begin{bmatrix} 1 \\ 0 \end{bmatrix}, \qquad
  \mathbf v \coloneqq \begin{bmatrix} 0 \\ 1 \end{bmatrix}, \qquad
  \mathbf w \coloneqq \begin{bmatrix} 1 \\ 1 \end{bmatrix}
\]
then we can instead describe valid matrices
as sequences of $n+1$ such column vectors,
where no two column vectors are adjacent,
and where the boundary condition is that
\begin{itemize}
  \ii either we start with $\mathbf u$ and end with $\mathbf v$, or
  \ii either we start with $\mathbf w$ and end with $\mathbf w$.
\end{itemize}
Let $x_n$ and $y_n$ denote the number of such $2 \times (n+1)$ matrices.
(Hence $a_n = x_n + y_n$.)
But owing to the symmetry of the setup with $\mathbf u$, $\mathbf v$, $\mathbf w$,
we could instead view $x_n$ and $y_n$ as the number of $2 \times (n+1)$ matrices
for a fixed starting first column whose final column is the same/different.
So we have the recursions
\begin{align*}
  x_{n+1} &= x_n + y_n \\
  y_{n+1} &= 2x_n.
\end{align*}
We also have that
\[ 2x_n + y_n = 2^n \]
which may either be proved directly from the recursions (using $x_1 = 1$ and $y_1 = 0$),
or by noting the left-hand side counts the total number of sequences
of $n+1$ column vectors with no restrictions on the final column at all
(in which case there are simply $2$ choices for each of the $n$ columns
after the first one).
Thus,
\begin{align*}
  a_{n+1} + a_n &= (x_{n+1} + y_{n+1}) + (x_n + y_n) \\
  &= \left( (x_n + y_n) + 2x_n \right) + (x_n + y_n) \\
  &= 2(2x_n + y_n) = 2^{n+1}
\end{align*}
as needed.

\paragraph{Second (longer) solution.}
If one does not notice the nice rephrasing with $\mathbf u$, $\mathbf v$,
$\mathbf w$ above, one may still proceed with the following direct calculation.
Retain the notation of
\[
  \begin{bmatrix}
    p_0 & p_1 & p_2 & \dots & p_{n-1} & p_n \\
    p_n & p_{n+1} & p_{n+2} & \dots & p_{2n-1} & p_{2n}
  \end{bmatrix}
\]
described earlier.
We will for now ignore the boundary conditions.
Instead we say a $2 \times m$ matrix is \emph{silver} ($m \ge 2$)
if it avoids the three shapes above.
We consider three types of silver matrices
(essentially doing casework on the last column):
\begin{itemize}
\ii \emph{type B matrices}, of the shape
  $\begin{bmatrix}
    1 & \dotsb & 1 \\
    0 & \dotsb & 0
  \end{bmatrix}$
\ii \emph{type C matrices}, of the shape
  $\begin{bmatrix}
    1 & \dotsb & 0 \\
    0 & \dotsb & 1
  \end{bmatrix}$.
\ii \emph{type D matrices}, of the shape
  $\begin{bmatrix}
    1 & \dotsb & 1 \\
    0 & \dotsb & 1
  \end{bmatrix}$.
\end{itemize}
We let $b_m$, $c_m$, $d_m$
denote matrices of each type, of size $2 \times m$,
and claim the following two recursions for $m \ge 4$:
\begin{align*}
  b_m &= c_{m-1} + d_{m-1} \\
  c_m &= b_{m-1} + d_{m-1} \\
  d_m &= b_{m-1} + c_{m-1}.
\end{align*}
Indeed, if we delete the last column of a type B matrix
and consider what used to be the second-to-last column,
we find that it is either type C or type D.
This establishes the first recursion and the others are analogous.

Note that $b_2 = 0$ and $c_2 = d_2 = 1$.
So using this recursion, the first few values are
\[
  \begin{array}{rrrrrrrr}
    m  & 2 & 3 & 4 & 5 &  6 &  7 &  8 \\ \hline
    b_m & 0 & 2 & 2 & 6 & 10 & 22 & 42 \\
    c_m & 1 & 1 & 3 & 5 & 11 & 21 & 43 \\
    d_m & 1 & 1 & 3 & 5 & 11 & 21 & 43 \\
  \end{array}
\]
and a calculation gives $b_m = \frac{2^{m-1} + 2(-1)^{m-1}}{3}$,
$c_m = d_m = \frac{2^{m-1} - (-1)^{m-1}}{3}$.

We now relate $a_n$ to $b_m$, $c_m$, $d_m$.
Observe that a matrix as described in the problem is a silver
matrix of one of two forms:
\[
  \begin{bmatrix}
    1 & p_1 & p_2 & \dots & p_{n-1} & 0 \\
    0 & p_{n+1} & p_{n+2} & \dots & p_{2n-1} & 1
  \end{bmatrix}
  \qquad
  \text{or}\qquad
  \begin{bmatrix}
    1 & p_1 & p_2 & \dots & p_{n-1} & 1 \\
    1 & p_{n+1} & p_{n+2} & \dots & p_{2n-1} & 1
  \end{bmatrix}.
\]
There are $c_{n+1}$ matrices of the first form.
Moreover, there are $2d_n$ matrices of the second form
(to see this, delete the first column;
we either get a type-D matrix
or an upside-down type-D matrix).
Thus we get
\[ a_n = c_{n+1} + 2d_n
  = \frac{2^{n+1} + (-1)^{n+1}}{3}. \]
This implies the result.

\begin{remark*}
  The two solutions are closely related.
  In fact, $c_n = x_{n-1}$ and $b_n = y_{n-1}$.
  So the second solution is really the same as the first solution,
  except the symmetry of $\mathbf u$, $\mathbf v$, $\mathbf w$ was not noticed,
  thus requiring a third recursion to handle all the cases manually.
\end{remark*}
\pagebreak

\subsection{USAMO 2013/3, proposed by Warut Suksompong}
\textsl{Available online at \url{https://aops.com/community/p3041827}.}
\begin{mdframed}[style=mdpurplebox,frametitle={Problem statement}]
Let $n$ be a positive integer.
There are $\tfrac{n(n+1)}{2}$ tokens,
each with a black side and a white side,
arranged into an equilateral triangle,
with the biggest row containing $n$ tokens.
Initially, each token has the white side up.
An operation is to choose a line parallel to the sides of the triangle,
and flip all the token on that line.
A configuration is called admissible if it can be obtained from the
initial configuration by performing a finite number of operations.
For each admissible configuration $C$,
let $f(C)$ denote the smallest number of operations required to
obtain $C$ from the initial configuration.
Find the maximum value of $f(C)$,
where $C$ varies over all admissible configurations.
\end{mdframed}
The answer is
\[
  \max_C f(C) =
  \begin{cases}
    6k & n = 4k \\
    6k+1 & n = 4k+1 \\
    6k+2 & n = 4k+2 \\
    6k+3 & n = 4k+3.
  \end{cases}
\]
The main point of the problem is actually to determine
all linear dependencies among the $3n$ possible moves
(since the moves commute and applying a move twice
is the same as doing nothing).
In what follows, assume $n > 1$ for convenience.

To this end, we consider sequences of operations as
additive vectors in $v \in \FF_2^{3n}$, with the
linear map $T \colon \FF_2^{3n} \to \FF_2^{\half n(n+1)}$
denoting the result of applying a vector $v$.
We in particular focus on the following four vectors.
\begin{itemize}
  \ii Three vectors $x$, $y$, $z$ are defined by
  choosing all $n$ lines parallel to one axis.
  Note $T(x) = T(y) = T(z) = \mathbf 1$
  (i.e.\ these vectors flip all tokens).
  \ii The vector $\theta$ which toggles
  all lines with an even number of tokens.
  One can check that $T(\theta) = \mathbf 0$.
  (Easiest to guess from $n=2$ and $n=3$ case.)
  One amusing proof that this works is to use Vivani's theorem:
  in an equilateral triangle $ABC$,
  the sum of distances from an interior point $P$
  to the three sides is equal.
\end{itemize}
The main claim is:
\begin{claim*}
  For $n \ge 2$,
  the kernel of $T$ has exactly eight elements,
  namely $\{\mathbf 0, x+y, y+z, z+x,
  \theta, \theta+x+y, \theta+y+z, \theta+z+x \}$.
\end{claim*}
% (This is motivated by the validity of the claim
% when $n = 2$ and $n = 3$.)
\begin{proof}
  Suppose $T(v) = 0$.
  \begin{itemize}
    \ii If $v$ uses the $y$-move of length $n$,
    then we replace $v$ with $v+(x+y)$
    to obtain a vector in the kernel not using the $y$-move of length $n$.
    \ii If $v$ uses the $z$-move of length $n$,
    then we replace $v$ with $v+(x+z)$
    to obtain a vector in the kernel not using the $z$-move of length $n$.
    \ii If $v$ uses the $x$-move of length $2$, then
    \begin{itemize}
      \ii if $n$ is odd, replace $v$ with $v+\theta$;
      \ii if $n$ is even, replace $v$ with $v+(\theta+y+z)$
    \end{itemize}
    to obtain a vector in the kernel not using the $x$-move of length $2$.
  \end{itemize}
  A picture is shown below,
  with the unused rows being dotted.
  \begin{center}
  \begin{asy}
    pair A = dir(60);
    pair B = dir(0);
    real r = 0.3;
    draw( (-A)--(7*A), dashed );
    draw( (7*B-A)--(7*A-B), dashed );
    draw( (5*A-B)--(5*A+2*B), dashed );
    for (int i=0; i<=6; ++i) {
      for (int j=0; i+j<=6; ++j) {
        filldraw(CR(i*A+j*B, r), opacity(0.2)+lightcyan, black);
      }
    }
  \end{asy}
  \end{center}

  Then, it is easy to check inductively that $v$ must now be the
  zero vector, after the replacements.
  The idea is that for each token $t$,
  if two of the moves involving $t$ are unused, so is the third,
  and in this way we can show all rows are unused.
  Thus the original $v$ was in the kernel we described.

  (An alternative proof by induction is feasible too;
  as a sequence of movings which does not affect the top $n$ rows
  also does not affect the to $n-1$ rows.)
\end{proof}
Then problem is a coordinate bash,
since given any $v$ we now know exactly which vectors $w$
have $T(v) = T(w)$, so given any admissible configuration $C$
one can exactly compute $f(C)$ as the minimum of eight values.

To be explicit, we could represent a vector $v$ as
\[ v \longleftrightarrow (a_1, a_2, b_1, b_2, c_1, c_2) \]
where $a_1$ is the number of $1$'s in odd $x$-indices,
$a_2$ number of $1$'s in even $x$-indices.
Then for example
\begin{align*}
  v &\longleftrightarrow (a_1, a_2, b_1, b_2, c_1, c_2) \\
  v+x+y &\longleftrightarrow \left( \left\lceil \frac n2 \right\rceil - a_1,
    \left\lfloor \frac n2 \right\rfloor - a_2,
    \left\lceil \frac n2  \right\rceil - b_1,
    \left\lfloor \frac n2 \right\rfloor - b_2, c_1, c_2 \right) \\
  v+\theta &\longleftrightarrow \left(a_1, \left\lfloor \frac n2  \right\rfloor - a_2, b_1,
    \left\lfloor \frac n2 \right\rfloor -
    b_2, c_1, \left\lfloor \frac n2 \right\rfloor - c_2\right)
  &\vdotswithin{=}
\end{align*}
and $f(T(v))$ is the smallest sum of the six numbers across all eight $6$-tuples.
So you expect to answer about $\frac32 n$
if all things are about $n/4$.
The details are too annoying to reproduce here, so they are omitted.
\pagebreak

\section{Solutions to Day 2}
\subsection{USAMO 2013/4, proposed by Titu Andreescu}
\textsl{Available online at \url{https://aops.com/community/p3043752}.}
\begin{mdframed}[style=mdpurplebox,frametitle={Problem statement}]
Find all real numbers $x,y,z \ge 1$ satisfying
\[ \min \left( \sqrt{x+xyz}, \sqrt{y+xyz}, \sqrt{z+xyz} \right)
  = \sqrt{x-1} + \sqrt{y-1} + \sqrt{z-1}. \]
\end{mdframed}
Set $x = 1+a$, $y = 1+b$, $z = 1+c$
which eliminates the $x,y,z \ge 1$ condition.
Assume without loss of generality that $a \leq b \leq c$.
Then the given equation rewrites as
\[ \sqrt{(1+a)\left( 1+(1+b)(1+c) \right)} = \sqrt a + \sqrt b + \sqrt c. \]

In fact, we are going to prove the left-hand side always exceeds the
right-hand side, and then determine the equality cases.
We have:
\begin{align*}
  (1+a)\left( 1 + (1+b)(1+c) \right)
  &= (a+1)\left( 1 + (b+1)(1+c) \right) \\
  &\le (a+1) \left( 1 + \left( \sqrt b + \sqrt c \right)^2 \right) \\
  &\le \left( \sqrt a + \left( \sqrt b + \sqrt c \right) \right)^2
\end{align*}
by two applications of Cauchy-Schwarz.

Equality holds if $bc = 1$ and $1/a = \sqrt b + \sqrt c$.
Letting $c = t^2$ for $t \ge 1$,
we recover $b = t^{-2} \le t^2$ and $a = \frac{1}{t+1/t} \le t^2$.

Hence the solution set is
\[ (x,y,z) = \left( 1 + \left( \frac{t}{t^2+1} \right)^2,
  1 + \frac{1}{t^2}, 1 + t^2 \right) \]
and permutations, for any $t > 0$.
\pagebreak

\subsection{USAMO 2013/5, proposed by Richard Stong}
\textsl{Available online at \url{https://aops.com/community/p3043754}.}
\begin{mdframed}[style=mdpurplebox,frametitle={Problem statement}]
Let $m$ and $n$ be positive integers.
Prove that there exists a positive integer $c$
such that $cm$ and $cn$ have the same nonzero decimal digits.
\end{mdframed}
One-line spoiler: $142857$.

More verbosely,
the idea is to look at the decimal representation of $1/D$, $m/D$, $n/D$
for a suitable denominator $D$, which have a ``cyclic shift'' property
in which the digits of $n/D$ are the digits of $m/D$ shifted by $3$.

\begin{remark*}
  [An example to follow along]
  Here is an example to follow along in the subsequent proof
  If $m = 4$ and $n = 23$ then the magic numbers $e = 3$ and $D = 41$ obey
  \[ 10^3 \cdot \frac{4}{41} = 97 + \frac{23}{41}. \]
  The idea is that
  \begin{align*}
    \frac{1}{41} &= 0.\ol{02439} \\
    \frac{4}{41} &= 0.\ol{09756} \\
    \frac{23}{41} &= 0.\ol{56097}
  \end{align*}
  and so $c = 2349$ works;
  we get $4c = 9756$ and $23c = 56097$
  which are cyclic shifts of each other by $3$ places
  (with some leading zeros appended).
\end{remark*}

Here is the one to use:
\begin{claim*}
  There exists positive integers $D$ and $e$
  such that $\gcd(D,10)=1$, $D > \max(m,n)$,
  and moreover \[ \frac{ 10^e m - n }{ D } \in \ZZ. \]
\end{claim*}
\begin{proof}
  Suppose we pick some exponent $e$ and define the number
  \[ A = 10^e n - m. \]
  Let $r = \nu_2(m)$ and $s = \nu_5(m)$.
  As long as $e > \max(r,s)$ we have $\nu_2(A) = r$ and $\nu_5(A) = s$, too.
  Now choose any $e > \max(r,s)$ big enough that $A > 2^r 5^s \max(m,n)$ also holds.
  Then the number $D = \frac{A}{2^r 5^s}$ works;
  the first two properties hold by construction and
  \[ 10^e \cdot \frac nD - \frac mD
    = \frac{A}{D} = 2^r 5^s \]
  is an integer.
\end{proof}
\begin{remark*}
  [For people who like obscure theorems]
  Kobayashi's theorem implies we can actually pick $D$ to be prime.
\end{remark*}
Now we take $c$ to be the
number under the bar of $1/D$ (leading zeros removed).
Then the decimal representation of $\frac mD$
is the decimal representation of $cm$ repeated
(possibly including leading zeros).
Similarly, $\frac nD$ has the decimal representation of $cm$ repeated (possibly including leading zeros).
Finally, since
\[ 10^e \cdot \frac mD - \frac nD \text{ is an integer} \]
it follows that these repeating decimal representations are rotations of each other by $e$ places,
so in particular they have the same number of nonzero digits.


\begin{remark*}
  Many students tried to find a $D$ satisfying the stronger hypothesis
  that $1/D$, $2/D$, \dots, $(D-1)/D$
  are cyclic shifts of each other.
  For example, this holds in the famous $D = 7$ case.

  The official USAMO 2013 solutions try to do this by proving
  that $10$ is a primitive root modulo $7^e$
  for each $e \ge 1$, by Hensel lifting lemma.
  I think this argument is actually \emph{incorrect},
  because it breaks if either $m$ or $n$ are divisible by $7$.
  Put bluntly, $\frac{7}{49}$ and $\frac{8}{49}$
  are not shifts of each other.

  One may be tempted to resort to using large primes $D$
  rather than powers of $7$ to deal with this issue.
  However it is an open conjecture
  (a special case of Artin's primitive root conjecture)
  whether or not $10 \pmod p$ is primitive infinitely often,
  which is the necessary conjecture
  so this is harder than it seems.
\end{remark*}
\pagebreak

\subsection{USAMO 2013/6, proposed by Titu Andreescu, Cosmin Pohoata}
\textsl{Available online at \url{https://aops.com/community/p3043749}.}
\begin{mdframed}[style=mdpurplebox,frametitle={Problem statement}]
Let $ABC$ be a triangle.
Find all points $P$ on segment $BC$ satisfying the following property:
If $X$ and $Y$ are the intersections of line $PA$ with the common external
tangent lines of the circumcircles of triangles $PAB$ and $PAC$, then
\[ \left( \frac{PA}{XY} \right)^2 + \frac{PB\cdot PC}{AB\cdot AC} = 1.\]
\end{mdframed}
Let $O_1$ and $O_2$ denote the circumcenters of $PAB$ and $PAC$.
The main idea is to notice that $\triangle ABC$ and $\triangle AO_1O_2$
are spirally similar.

\begin{center}
\begin{asy}
pair A = dir(140);
pair B = dir(210);
pair C = dir(330);
pair P = 0.3*C+0.7*B;
filldraw(A--B--C--cycle, opacity(0.1)+lightcyan, blue);

pair O_1 = circumcenter(A, B, P);
pair O_2 = circumcenter(A, C, P);

real r1 = abs(P-O_1);
real r2 = abs(P-O_2);
pair Se = (r1*O_2-r2*O_1)/(r1-r2);
path w1 = CP(O_1,P);
path w2 = CP(O_2,P);
filldraw(w1, opacity(0.1)+yellow, deepgreen);
filldraw(w2, opacity(0.1)+yellow, deepgreen);

pair X_1 = IP(w1, CP(midpoint(Se--O_1), Se));
pair Y_1 = OP(w1, CP(midpoint(Se--O_1), Se));
pair X_2 = IP(w2, CP(midpoint(Se--O_2), Se));
pair Y_2 = OP(w2, CP(midpoint(Se--O_2), Se));

draw(X_1--X_2, deepgreen);
draw(Y_1--Y_2, deepgreen);
pair X = extension(A, P, X_1, X_2);
pair Y = extension(A, P, Y_1, Y_2);
draw(X--Y, dotted);

filldraw(A--O_1--O_2--cycle, opacity(0.1)+lightred, red);

dot("$A$", A, dir(A));
dot("$B$", B, dir(B));
dot("$C$", C, dir(C));
dot("$P$", P, dir(70));
dot("$O_1$", O_1, dir(-90));
dot("$O_2$", O_2, dir(-90));
dot("$X_1$", X_1, dir(X_1));
dot("$Y_1$", Y_1, dir(Y_1));
dot("$X_2$", X_2, dir(X_2));
dot("$Y_2$", Y_2, dir(Y_2));
dot("$X$", X, dir(X));
dot("$Y$", Y, dir(Y));

/* TSQ Source:

A = dir 140
B = dir 210
C = dir 330
P = 0.3*C+0.7*B R70
A--B--C--cycle 0.1 lightcyan / blue

O_1 = circumcenter A B P R-90
O_2 = circumcenter A C P R-90

!real r1 = abs(P-O_1);
!real r2 = abs(P-O_2);
Se := (r1*O_2-r2*O_1)/(r1-r2)
!path w1 = CP(O_1,P);
!path w2 = CP(O_2,P);
w1 0.1 yellow / deepgreen
w2 0.1 yellow / deepgreen

X_1 = IP w1 CP midpoint Se--O_1 Se
Y_1 = OP w1 CP midpoint Se--O_1 Se
X_2 = IP w2 CP midpoint Se--O_2 Se
Y_2 = OP w2 CP midpoint Se--O_2 Se

X_1--X_2 deepgreen
Y_1--Y_2 deepgreen
X = extension A P X_1 X_2
Y = extension A P Y_1 Y_2
X--Y dotted

A--O_1--O_2--cycle 0.1 lightred / red

*/
\end{asy}
\end{center}

\begin{claim*}
  [Salmon theorem]
  We have $\triangle ABC \overset{+}{\sim} \triangle AO_1O_2$.
\end{claim*}
\begin{proof}
  We first claim $\triangle AO_1B \overset{+}{\sim} \triangle AO_2C$.
  Assume without loss of generality that $\angle APB \le 90\dg$.
  Then \[ \angle AO_1B = 2 \angle APB \]
  but \[ \angle AO_2C = 2 \left( 180 - \angle APC \right) = 2 \angle ABP. \]
  Hence $\angle AO_1B = \angle AO_2C$.
  Moreover, both triangles are isosceles,
  establishing the first similarity.
  The second part follows from spiral similarities coming in pairs.
\end{proof}

\begin{claim*}
  We always have
  \[ \left( \frac{PA}{XY} \right)^2 = 1 - \left( \frac{a}{b+c} \right)^2. \]
  (In particular, this does not depend on $P$.)
\end{claim*}
\begin{proof}
We now delete the points $B$ and $C$
and remember only the fact that $\triangle AO_1O_2$
has angles $A$, $B$, $C$.
The rest is a computation and several approaches are possible.

Without loss of generality $A$ is closer to $X$ than $Y$,
and let the common tangents be $\ol{X_1X_2}$ and $\ol{Y_1Y_2}$.
We'll perform the main calculation
with the convenient scaling $O_B O_C = a$, $AO_C = b$, and $AO_B = c$.
Let $B_1$ and $C_1$ be the tangency points of $X$,
and let $h = AM$ be the height of $\triangle A O_B O_C$.

\begin{center}
\begin{asy}
pair O_1 = dir(180);
pair O_2 = dir(0);
pair A = dir(130);
pair P = conj(A);
pair M = midpoint(A--P);

real r1 = abs(P-O_1);
real r2 = abs(P-O_2);
pair Se = (r1*O_2-r2*O_1)/(r1-r2);
path w1 = CP(O_1,P);
path w2 = CP(O_2,P);
filldraw(w1, opacity(0.1)+yellow, deepgreen);
filldraw(w2, opacity(0.1)+yellow, deepgreen);

pair X_1 = IP(w1, CP(midpoint(Se--O_1), Se));
pair X_2 = IP(w2, CP(midpoint(Se--O_2), Se));

filldraw(A--O_1--O_2--cycle, opacity(0.1)+lightred, red);
draw(O_1--X_1--X_2--O_2, deepgreen);
pair X = extension(A, P, X_1, X_2);
draw(P--X, deepgreen);

dot("$O_1$", O_1, dir(O_1));
dot("$O_2$", O_2, dir(O_2));
dot("$A$", A, dir(30));
dot("$P$", P, dir(330));
dot("$M$", M, dir(315));
dot("$X_1$", X_1, dir(X_1));
dot("$X_2$", X_2, dir(105));
dot("$X$", X, dir(X));

/* TSQ Source:

O_1 = dir 180
O_2 = dir 0
A = dir 130 R30
P = conj A R330
M = midpoint A--P R315

!real r1 = abs(P-O_1);
!real r2 = abs(P-O_2);
Se := (r1*O_2-r2*O_1)/(r1-r2)
!path w1 = CP(O_1,P);
!path w2 = CP(O_2,P);
w1 0.1 yellow / deepgreen
w2 0.1 yellow / deepgreen

X_1 = IP w1 CP midpoint Se--O_1 Se
X_2 = IP w2 CP midpoint Se--O_2 Se R105

A--O_1--O_2--cycle 0.1 lightred / red
O_1--X_1--X_2--O_2 deepgreen
X = extension A P X_1 X_2
P--X deepgreen

*/
\end{asy}
\end{center}

Note that by Power of a Point, we have $XX_1^2 = XX_2^2 = XM^2 - h^2$.
Also, by Pythagorean theorem we easily obtain $X_1X_2 = a^2 - (b-c)^2$.
So putting these together gives
\[ XM^2 - h^2 = \frac{a^2 - (b-c)^2}{4}
  = \frac{(a+b-c)(a-b+c)}{4} = (s-b)(s-c). \]
Therefore, we have

Then
\begin{align*}
  \frac{XM^2}{h^2}
  &= 1 + \frac{(s-b)(s-c)}{h^2}
    = 1 + \frac{a^2(s-b)(s-c)}{a^2h^2} \\
  &= 1 + \frac{a^2(s-b)(s-c)}{4s(s-a)(s-b)(s-c)}
    = 1 + \frac{a^2}{4s(s-a)} \\
  &= 1 + \frac{a^2}{(b+c)^2-a^2}
  = \frac{(b+c)^2}{(b+c)^2-a^2}.
\end{align*}
Thus \[ \left( \frac{PA}{XY} \right)^2 = \left(\frac{h}{XM}\right)^2
  = 1 - \left( \frac{a}{b+c} \right)^2. \qedhere \]
\end{proof}

To finish, note that when $P$ is the foot of the $\angle A$-bisector,
we necessarily have
\[
  \frac{PB \cdot PC}{AB \cdot AC}
  = \frac{\left( \frac{b}{b+c} a \right)\left( \frac{c}{b+c} a \right)}{bc}
  = \left( \frac{a}{b+c} \right)^2.  \]
Since there are clearly at most two solutions as $\tfrac{PA}{XY}$ is fixed,
these are the only two solutions.
\pagebreak


\end{document}
