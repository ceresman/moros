% © Evan Chen
% Downloaded from https://web.evanchen.cc/

\documentclass[11pt]{scrartcl}
\usepackage[sexy]{evan}
\ihead{\footnotesize\textbf{\thetitle}}
\ohead{\footnotesize\href{http://web.evanchen.cc}{\ttfamily web.evanchen.cc},
    updated \today}
\title{USAMO 2007 Solution Notes}
\date{\today}

\begin{document}

\maketitle

\begin{abstract}
This is a compilation of solutions
for the 2007 USAMO.
Some of the solutions are my own work,
but many are from the official solutions provided by the organizers
(for which they hold any copyrights),
and others were found by users on the Art of Problem Solving forums.

These notes will tend to be a bit more advanced and terse than the ``official''
solutions from the organizers.
In particular, if a theorem or technique is not known to beginners
but is still considered ``standard'', then I often prefer to
use this theory anyways, rather than try to work around or conceal it.
For example, in geometry problems I typically use directed angles
without further comment, rather than awkwardly work around configuration issues.
Similarly, sentences like ``let $\mathbb{R}$ denote the set of real numbers''
are typically omitted entirely.

Corrections and comments are welcome!
\end{abstract}

\tableofcontents
\newpage

\addtocounter{section}{-1}
\section{Problems}
\begin{enumerate}[\bfseries 1.]
\item %% Problem 1
Let $n$ be a positive integer.
Define a sequence by setting $a_1 = n$ and, for each $k > 1$,
letting $a_k$ be the unique integer in the range $0 \leq a_k \leq k-1$
for which $a_1 + a_2 + \dots + a_k$ is divisible by $k$.
(For instance, when $n = 9$
the obtained sequence is $9,1,2,0,3,3,3,\dots$.)
Prove that for any $n$ the sequence $a_1$, $a_2$, \dots
eventually becomes constant.

\item %% Problem 2
Decide whether it possible to cover all lattice points in $\RR^2$
by an (infinite) family of disks whose interiors are disjoint
such that the radius of each disk is at least $5$.

\item %% Problem 3
Let $S$ be a set containing $n^2+n-1$ elements.
Suppose that the $n$-element subsets of $S$
are partitioned into two classes.
Prove that there are at least $n$ pairwise disjoint sets
in the same class.

\item %% Problem 4
An \emph{animal} with $n$ \emph{cells}
is a connected figure consisting of $n$ equal-sized square cells
(equivalently, a polyomino with $n$ cells).
A \emph{dinosaur} is an animal with at least $2007$ cells.
It is said to be \emph{primitive}
it its cells cannot be partitioned into two or more dinosaurs.
Find with proof the maximum number of cells in a primitive dinosaur.

\item %% Problem 5
Prove that for every nonnegative integer $n$,
the number $7^{7^n}+1$ is the product
of at least $2n+3$ (not necessarily distinct) primes.

\item %% Problem 6
Let $ABC$ be an acute triangle
with $\omega$, $S$, and $R$ being its incircle,
circumcircle, and circumradius, respectively.
Circle $\omega_{A}$ is tangent internally to $S$
at $A$ and tangent externally to $\omega$.
Circle $S_{A}$ is tangent internally to $S$
at $A$ and tangent internally to $\omega$.

Let $P_A$ and $Q_A$ denote
the centers of $\omega_A$ and $S_A$, respectively.
Define points $P_B$, $Q_B$, $P_C$, $Q_C$ analogously.
Prove that
\[ 8 P_A Q_A \cdot P_B Q_B \cdot P_C Q_C \le R^3 \]
with equality if and only if triangle $ABC$ is equilateral.

\end{enumerate}
\pagebreak

\section{Solutions to Day 1}
\subsection{USAMO 2007/1, proposed by Sam Vandervelde}
\textsl{Available online at \url{https://aops.com/community/p825490}.}
\begin{mdframed}[style=mdpurplebox,frametitle={Problem statement}]
Let $n$ be a positive integer.
Define a sequence by setting $a_1 = n$ and, for each $k > 1$,
letting $a_k$ be the unique integer in the range $0 \leq a_k \leq k-1$
for which $a_1 + a_2 + \dots + a_k$ is divisible by $k$.
(For instance, when $n = 9$
the obtained sequence is $9,1,2,0,3,3,3,\dots$.)
Prove that for any $n$ the sequence $a_1$, $a_2$, \dots
eventually becomes constant.
\end{mdframed}
For each $k$, the number
\[ b_k \coloneqq \frac 1k (a_1 + \dots + a_k) \]
is a nonnegative integer.
\begin{claim*}
  The sequence $(b_k)$ is eventually constant.
\end{claim*}
\begin{proof}
  Since
  \[ b_{k+1}
    = \frac{a_1 + \dots + a_k + a_{k+1}}{k+1}
    \le \frac{k b_k + k}{k+1} < b_k + 1 \]
  and therefore $b_{k+1} \le b_k$ for all $k$.
  Hence the sequence $b_k$ must eventually be constant.
\end{proof}

This can only happen once the sequence is constant:
indeed if $N$ is an index such that $b_N = b_{N+1} = \dots = b$ then
\[ a_k = (k+1) b_{k+1} - k \cdot b_k
  = (k+1) \cdot b - k \cdot b = b \]
for any $k \ge N$.
In other words, $a_N = a_{N+1} = \dots = b$ as well.
\pagebreak

\subsection{USAMO 2007/2, proposed by Gregory Galperin}
\textsl{Available online at \url{https://aops.com/community/p825495}.}
\begin{mdframed}[style=mdpurplebox,frametitle={Problem statement}]
Decide whether it possible to cover all lattice points in $\RR^2$
by an (infinite) family of disks whose interiors are disjoint
such that the radius of each disk is at least $5$.
\end{mdframed}
The answer is no.

Assume not.
Take a disk $\odot O$ not touching any member of the family,
and then enlarge it until it is maximal.
Then, it must be tangent to at least three other disks,
say $\odot A$, $\odot B$, $\odot C$.
Suppose WLOG that $\angle AOB \le 120\dg$.
Denote the radii of $\odot O$, $\odot A$, $\odot B$ by $r$, $a$, $b$.

But the Law of Cosines gives
\[ (a+b)^2 \le (a+r)^2 + (b+r)^2 + (a+r)(b+r) \]
which rewrites as
\[ 12r^2 \ge (a-3r)(b-3r) \ge (5-3r)^2 \]
which one can check is impossible for $r \le 1/\sqrt2$.
Thus $r > 1/\sqrt2$.

In particular $(\odot O)$ must contain a lattice point
as it contains a unit square.

\begin{remark*}
  The order of the argument here matters in subtle ways.
  A common approach is to try and reduce to the ``optimal'' case
  where we have three mutually tangent circles,
  and then apply the Descarte circle theorem.
  There are ways in which this approach can fail
  if the execution is not done with care.
  (In particular, one cannot simply say to reduce
  to this case, without some justification.)

  For example: it is not true that, given an infinite family of disks,
  we can enlarge disks until we get three mutually tangent ones.
  As a counterexample consider the ``square grid''
  in which a circle is centered at $(10m, 10n)$ for each $m,n \in \ZZ$
  and has radius $5$.
  Thus it is also not possible to simply pick three nearby circles
  and construct a circle tangent to all three:
  that newly constructed circle might intersect a fourth disk
  not in the picture.

  Thus, when constructing the small disk $\odot O$
  in the above solution,
  it seems easiest to start with a point not covered and grow $\odot O$
  until it is tangent to \emph{some} three circles,
  and then argue by cosine law.
  Otherwise it not easy to determine which three
  circles to start with.

  In all solutions it seems easier to prove that a disjoint circle
  of radius $1/\sqrt2$ exists,
  and then \emph{finally} deduce it has a lattice point,
  rather than trying to work the lattice point into the existence proof.
\end{remark*}
\pagebreak

\subsection{USAMO 2007/3, proposed by Andras Gyarfas}
\textsl{Available online at \url{https://aops.com/community/p825499}.}
\begin{mdframed}[style=mdpurplebox,frametitle={Problem statement}]
Let $S$ be a set containing $n^2+n-1$ elements.
Suppose that the $n$-element subsets of $S$
are partitioned into two classes.
Prove that there are at least $n$ pairwise disjoint sets
in the same class.
\end{mdframed}
We present two solutions which are really equivalent,
but phrased differently.
We refer to the two classes as ``red'' and ``blue'', respectively.

\paragraph{First solution (Grant Yu).}
We define a set of $n+1$ elements to be \emph{useful}
if it has $n$-element subsets in each class.

Consider a \textbf{maximal collection of disjoint useful sets}
and assume there are $p$ such sets.
Then, let $T$ be the set of elements remaining
(i.e.\ not in one of chosen useful sets).
\begin{claim*}
  All subsets of $T$ of size $n$ are the same color.
\end{claim*}
\begin{proof}
  Assume there was a red set $R$ in $T$.
  Replace the elements of $R$ one by one
  until we get to any other subset $R'$ of $T$.
  At each step, because no sets of $T$ form a useful set,
  the set remains red --- so $R'$ is red too.
  Since $R'$ is arbitrary, this proves the claim.
\end{proof}
We have $|T| = n^2+n-1 - p(n+1)$, and in particular $p < n$.
WLOG all sets in $T$ are red.
We can extract another red set from each of our chosen useful sets.
So we can get at least
\[ p + \left\lfloor \frac{|T|}{n} \right\rfloor
  = p + \left\lfloor n+1-p - \frac{1+p}{n} \right\rfloor
  \ge p + (n-p) = n.  \]


\paragraph{Second solution (by induction).}
We prove more strongly that:
\begin{claim*}
  Let $S$ be a set containing $k \cdot (n+1)-1$ elements.
  Then we can find $k$ pairwise disjoint sets of the same color.
\end{claim*}
The proof is by induction on $k \ge 1$.
The base case $k=1$ this is immediate; $\binom{S}{n}$ is a single set.

For the inductive step, assume for contradiction the problem fails.
Let $T$ be any subset of $S$ of size $(k-1)(n+1)-1$.
By the induction hypothesis, among the subsets of $T$ alone,
we can already find $k-1$ pairwise disjoint sets of the same color.
Now $S \setminus T$ has size $n+1$, and so we would have to have
that all $\binom{n+1}{n}$ subsets of $S \setminus T$ are the same color.

By varying $T$, the set $S \setminus T$ ranges over all of $\binom{S}{k+1}$.
This causes all sets to be the same color, contradiction.

\begin{remark*}
  Victor Wang writes the following:
  \begin{quote}
  I don't really like this problem,
  but I think the main motivation for generalizing the problem
  is that the original problem doesn't allow you to look at small cases.
  (Also, it's not initially clear where the $n^2+n-1$ comes from.)
  And pretty much the simplest way to get lots of similarly-flavored
  small cases is to start with $k=2,3$ in
  ``find the smallest $N(n,k)$ such that when we partition the $n$-subsets
  of a $\ge N(n,k)$-set into $2$ classes,
  we can find some $k$ pairwise disjoint sets in the same class''.
  \end{quote}
\end{remark*}
\pagebreak

\section{Solutions to Day 2}
\subsection{USAMO 2007/4, proposed by Reid Barton}
\textsl{Available online at \url{https://aops.com/community/p825505}.}
\begin{mdframed}[style=mdpurplebox,frametitle={Problem statement}]
An \emph{animal} with $n$ \emph{cells}
is a connected figure consisting of $n$ equal-sized square cells
(equivalently, a polyomino with $n$ cells).
A \emph{dinosaur} is an animal with at least $2007$ cells.
It is said to be \emph{primitive}
it its cells cannot be partitioned into two or more dinosaurs.
Find with proof the maximum number of cells in a primitive dinosaur.
\end{mdframed}
In fact it's true for any tree with maximum degree $\le 4$.
Here is the solution of Andrew Geng.

Let $T$ be such a tree (a spanning tree of the dinosaur graph)
which corresponds to a primitive dinosaur.

\begin{claim*}
  There exists a vertex $v$ such that when $v$ is deleted, no dinosaurs result.
\end{claim*}
\begin{proof}
  Assume for contradiction that all vertices are bad
  (leave a dinosaur when deleted).
  Consider two adjacent vertices $v$, $w$ in $T$.
  By checking possibilities, one sees that, say,
  the dinosaur in $T-v$ contains $w$ and the dinosaur of $T-w$.
  We can repeat in this way; since $T$ is acyclic,
  this eventually becomes a contradiction.
\end{proof}

When this vertex is deleted,
we get at most $4$ components, each with $\le 2006$ vertices,
giving the answer of $4 \cdot 2006 + 1 = 8025$.
The construction is easy (take a ``cross'', for example).
\pagebreak

\subsection{USAMO 2007/5, proposed by Titu Andreescu}
\textsl{Available online at \url{https://aops.com/community/p825508}.}
\begin{mdframed}[style=mdpurplebox,frametitle={Problem statement}]
Prove that for every nonnegative integer $n$,
the number $7^{7^n}+1$ is the product
of at least $2n+3$ (not necessarily distinct) primes.
\end{mdframed}
We prove this by induction on $n$ by showing that
\[ \frac{X^7+1}{X+1} = X^6 - X^5 + \dots + 1 \]
is never prime for $X = 7^{7^n}$,
hence we gain at least two additional prime factors
whenever we increase $n$ by one.

Indeed, the quotient may be written as
\[ \left( X+1 \right)^6 - 7X \cdot (X^2+X+1)^2 \]
which becomes a difference of squares, hence composite.
\pagebreak

\subsection{USAMO 2007/6, proposed by Sung-Yoon Kim}
\textsl{Available online at \url{https://aops.com/community/p825515}.}
\begin{mdframed}[style=mdpurplebox,frametitle={Problem statement}]
Let $ABC$ be an acute triangle
with $\omega$, $S$, and $R$ being its incircle,
circumcircle, and circumradius, respectively.
Circle $\omega_{A}$ is tangent internally to $S$
at $A$ and tangent externally to $\omega$.
Circle $S_{A}$ is tangent internally to $S$
at $A$ and tangent internally to $\omega$.

Let $P_A$ and $Q_A$ denote
the centers of $\omega_A$ and $S_A$, respectively.
Define points $P_B$, $Q_B$, $P_C$, $Q_C$ analogously.
Prove that
\[ 8 P_A Q_A \cdot P_B Q_B \cdot P_C Q_C \le R^3 \]
with equality if and only if triangle $ABC$ is equilateral.
\end{mdframed}
It turns out we can compute $P_AQ_A$ explicitly.
Let us invert around $A$ with radius $s-a$
(hence fixing the incircle) and then compose this
with a reflection around the angle bisector of $\angle BAC$.
We denote the image of the composed map via
\[ \bullet \mapsto \bullet^\ast \mapsto \bullet^+. \]
We overlay this inversion with the original diagram.

Let $P_AQ_A$ meet $\omega_A$ again at $P$ and $S_A$ again at $Q$.
Now observe that $\omega_A^\ast$ is a line parallel to $S^\ast$;
that is, it is perpendicular to $\ol{PQ}$.
Moreover, it is tangent to $\omega^\ast = \omega$.

Now upon the reflection,
we find that $\omega^+ = \omega^\ast = \omega$,
but line $\ol{PQ}$ gets mapped to the altitude from $A$ to $\ol{BC}$,
since $\ol{PQ}$ originally contained the circumcenter $O$
(isogonal to the orthocenter).
But this means that $\omega_A^\ast$ is none other than the $\ol{BC}$!
Hence $P^+$ is actually the foot of the altitude from $A$
onto $\ol{BC}$.

By similar work, we find that $Q^+$
is the point on $\ol{AP^+}$ such that $P^+Q^+ = 2r$.
\begin{center}
\begin{asy}
  size(8.5cm);
  pair A = dir(140);
  pair B = dir(210);
  pair C = dir (330);
  filldraw(A--B--C--cycle, opacity(0.1)+yellow, grey);
  dot("$A$", A, A);
  dot("$B$", B, B);
  dot("$C$", C, C);
  draw(incircle(A,B,C), black+1);
  pair D = foot(A,B,C);
  dot("$P^+$", D, dir(-45), brown);
  draw(A--D, brown);
  pair I = incenter(A,B,C);
  pair O = circumcenter(A,B,C);

  real a = abs(B-C);
  real b = abs(C-A);
  real c = abs(A-B);
  real s = 0.5*(a+b+c);
  real ha = abs(A-D);
  real k = ((s-a)**3) / (s*ha*abs(A-I)**2);

  pair PA = k*b*c/2*dir(O-A)+A;
  dot("$P_A$", PA, dir(255), blue);
  draw(CP(PA,A), lightblue);
  pair P = Drawing("P", 2*PA-A, dir(180));

  pair QA = PA + dir(PA-A)*( (s-a)*a**2 / (a*b*c) );
  dot("$Q_A$", QA, dir(-80)*1.4, blue);
  draw(CP(QA,A), lightblue);
  pair Q = Drawing("Q", 2*QA-A, dir(-70));

  draw(unitcircle, grey);
  draw(A--Q);
  real r = abs(I-foot(I,B,C));
  Drawing("I", I, dir(90));
  pair X = I + dir(O-A)*r;
  pair P1 = foot(X,P,Q);
  draw(Line(X,P1,2.5), red, Arrows(TeXHead));
  dot("$P^\ast$", P1, dir(5), red);
  draw(CP(A, foot(I,A,B)), dashed+deepgreen);

  pair Q2 = D+dir(A-D)*2*r;
  dot("$Q^+$", Q2, dir(145), brown);
  draw(Q2--(I+r*dir(90)), brown);
\end{asy}
\end{center}

Now we can compute all the lengths directly.  We have that
\[ AP_A = \half AP = \frac{(s-a)^2}{2AP^+}
= \half (s-a)^2 \cdot \frac{1}{h_a} \]
and
\[ AQ_A = \half AQ = \frac{(s-a)^2}{2AQ^+}
= \half (s-a)^2 \cdot \frac{1}{h_a-2r} \]
where $h_a = \frac{2K}{a}$ is the length of the $A$-altitude, with $K$ the area of $ABC$ as usual. Now it follows that
\[ P_AQ_A = \half (s-a)^2 \left( \frac{2r}{h_a(h_a-2r)} \right). \]
This can be simplified, as
\[ h_a - 2r = \frac{2K}{a} - \frac{2K}{s} = 2K \cdot \frac{s-a}{as}. \]
Hence
\[ P_AQ_A = \frac{a^2rs(s-a)}{4K^2} = \frac{a^2(s-a)}{4K}. \]
Hence, the problem is just asking us to show that
\[ a^2b^2c^2(s-a)(s-b)(s-c) \le 8 (RK)^3. \]
Using $abc=4RK$ and $(s-a)(s-b)(s-c) = \frac 1s K^2 = rK$, we find that this becomes
\[ 2(s-a)(s-b)(s-c) \le RK \iff 2r \le R \]
which follows immediately from $IO^2=R(R-2r)$.
Alternatively, one may rewrite this as Schur's Inequality in the form
\[ abc \ge (-a+b+c)(a-b+c)(a+b-c). \]
\pagebreak


\end{document}
