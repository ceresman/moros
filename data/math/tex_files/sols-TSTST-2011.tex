\documentclass[11pt]{scrartcl}
\usepackage[sexy]{evan}
\ihead{\footnotesize\textbf{\thetitle}}
\ohead{\footnotesize\theauthor}
\begin{document}

\title{TSTST 2011 Solution Notes}
\subtitle{Lincoln, Nebraska}
\date{\today}

\maketitle
\begin{abstract}
This is a compilation of solutions
for the 2011 TSTST.
Some of the solutions are my own work,
but many are from the official solutions provided by the organizers
(for which they hold any copyrights),
and others were found by users on the Art of Problem Solving forums.

These notes will tend to be a bit more advanced and terse than the ``official''
solutions from the organizers.
In particular, if a theorem or technique is not known to beginners
but is still considered ``standard'', then I often prefer to
use this theory anyways, rather than try to work around or conceal it.
For example, in geometry problems I typically use directed angles
without further comment, rather than awkwardly work around configuration issues.
Similarly, sentences like ``let $\mathbb{R}$ denote the set of real numbers''
are typically omitted entirely.

Corrections and comments are welcome!
\end{abstract}
\tableofcontents
\newpage

\addtocounter{section}{-1}
\section{Problems}
\begin{enumerate}[\bfseries 1.]
\item %% Problem 1
Find all real-valued functions $f$ defined on pairs of real numbers,
having the following property: for all real numbers $a, b, c$,
the median of $f(a,b), f(b,c), f(c,a)$ equals the median of $a, b, c$.

(The \emph{median} of three real numbers, not necessarily distinct,
is the number that is in the middle when the three numbers
are arranged in nondecreasing order.)

\item %% Problem 2
Two circles $\omega_1$ and $\omega_2$ intersect at points $A$ and $B$.
Line $\ell$ is tangent to $\omega_1$ at $P$
and to $\omega_2$ at $Q$ so that $A$ is closer to $\ell$ than $B$.
Let $X$ and $Y$ be points on major arcs $\arc{PA}$
(on $\omega_1$) and $\arc{AQ}$ (on $\omega_2$), respectively,
such that $AX/PX = AY/QY = c$.
Extend segments $PA$ and $QA$ through $A$ to $R$ and $S$,
respectively, such that $AR = AS = c\cdot PQ$.
Given that the circumcenter of triangle $ARS$ lies on line $XY$,
prove that $\angle XPA = \angle AQY$.

\item %% Problem 3
Prove that there exists a real constant $c$
such that for any pair $(x,y)$ of real numbers,
there exist relatively prime integers $m$ and $n$
satisfying the relation
\[ \sqrt{(x-m)^2 + (y-n)^2} < c \log(x^2+y^2+2). \]

\item %% Problem 4
Acute triangle $ABC$ is inscribed in circle $\omega$.
Let $H$ and $O$ denote its orthocenter and circumcenter, respectively.
Let $M$ and $N$ be the midpoints of sides $AB$ and $AC$, respectively.
Rays $MH$ and $NH$ meet $\omega$ at $P$ and $Q$, respectively.
Lines $MN$ and $PQ$ meet at $R$.
Prove that $\ol{OA} \perp \ol{RA}$.

\item %% Problem 5
At a certain orphanage, every pair of orphans are either friends or enemies.
For every three of an orphan's friends,
an even number of pairs of them are enemies.
Prove that it's possible to assign each orphan two parents
such that every pair of friends shares exactly one parent, but no pair of enemies does,
and no three parents are in a love triangle (where each pair of them has a child).

\item %% Problem 6
Let $a,b,c$ be real numbers in the interval $[0,1]$
with $a+b,b+c,c+a \ge 1$.  Prove that
\[ 1 \le (1-a)^2 + (1-b)^2 + (1-c)^2
  + \frac{2\sqrt2 abc}{\sqrt{a^2+b^2+c^2}}. \]

\item %% Problem 7
Let $ABC$ be a triangle.
Its excircles touch sides $BC$, $CA$, $AB$ at $D$, $E$, $F$.
Prove that the perimeter of triangle $ABC$ is
at most twice that of triangle $DEF$.

\item %% Problem 8
Let $x_0$, $x_1$, \dots, $x_{n_0-1}$ be integers,
and let $d_1$, $d_2$, \dots, $d_k$ be positive integers
with $n_0 = d_1 > d_2 > \dotsb > d_k$ and
$\gcd(d_1, d_2, \dots, d_k) = 1$.
For every integer $n \geq n_0$, define
\[ x_n = \left\lfloor \frac{x_{n-d_1} + x_{n-d_2}
  + \dots + x_{n-d_k}}{k} \right\rfloor. \]
Show that the sequence $(x_n)$ is eventually constant.

\item %% Problem 9
Let $n$ be a positive integer.
Suppose we are given $2^n+1$ distinct sets,
each containing finitely many objects.
Place each set into one of two categories, the red sets and the blue sets,
so that there is at least one set in each category.
We define the \textit{symmetric difference} of two sets as
the set of objects belonging to exactly one of the two sets.
Prove that there are at least $2^n$ different sets which
can be obtained as the symmetric difference of a red set and a blue set.

\end{enumerate}
\pagebreak

\section{Solutions to Day 1}
\subsection{TSTST 2011/1}
\textsl{Available online at \url{https://aops.com/community/p2374841}.}
\begin{mdframed}[style=mdpurplebox,frametitle={Problem statement}]
Find all real-valued functions $f$ defined on pairs of real numbers,
having the following property: for all real numbers $a, b, c$,
the median of $f(a,b), f(b,c), f(c,a)$ equals the median of $a, b, c$.

(The \emph{median} of three real numbers, not necessarily distinct,
is the number that is in the middle when the three numbers
are arranged in nondecreasing order.)
\end{mdframed}
The following solution is joint with Andrew He.

We prove the following main claim,
from which repeated applications can deduce the problem.

\begin{claim*}
  Let $a < b < c$ be arbitrary. On $\{a,b,c\}^2$,
  $f$ takes one of the following two forms,
  where the column indicates the $x$-value
  and the row indicates the $y$-value.
  \[
   \begin{array}{c|rrr}
    f & a & b & c \\ \hline
    a & a & b & \ge c \\
    b & \le a & b & \ge c \\
    c & \le a & b & c
   \end{array}
   \qquad
   \text{or}
   \qquad
   \begin{array}{c|rrr}
    f & a & b & c \\ \hline
    a & a & \le a & \le a \\
    b & b & b & b \\
    c & \ge c & \ge c & c
   \end{array}
  \]
\end{claim*}
\begin{proof}
  First, we of course have $f(x,x) = x$ for all $x$.
  Now:
  \begin{itemize}
  \ii By considering the assertion for $(a,a,c)$ and $(a,c,c)$
  we see that one of $f(a,c)$ and $f(c,a)$ is $\ge c$
  and the other is $\le a$.
  \ii Hence, by considering $(a,b,c)$ we find that
  one of $f(a,b)$ and $f(b,c)$ must be $b$,
  and similarly for $f(b,a)$ and $f(c,b)$.
  \ii Now, WLOG $f(b,a) = b$; we prove we get the first case.
  \ii By considering $(a,a,b)$ we deduce $f(a,b) \le a$,
  so $f(b,c) = b$ and then $f(c,b) \ge c$.
  \ii Finally, considering $(c,b,a)$ once again
  in conjunction with the first bullet, we arrive at the conclusion
  that $f(a,c) \le a$; similarly $f(c,a) \ge c$.
  \qedhere
  \end{itemize}
\end{proof}
From this it's easy to obtain that $f(x,y) \equiv x$
or $f(x,y) \equiv y$ are the only solutions.
\pagebreak

\subsection{TSTST 2011/2}
\textsl{Available online at \url{https://aops.com/community/p2374843}.}
\begin{mdframed}[style=mdpurplebox,frametitle={Problem statement}]
Two circles $\omega_1$ and $\omega_2$ intersect at points $A$ and $B$.
Line $\ell$ is tangent to $\omega_1$ at $P$
and to $\omega_2$ at $Q$ so that $A$ is closer to $\ell$ than $B$.
Let $X$ and $Y$ be points on major arcs $\arc{PA}$
(on $\omega_1$) and $\arc{AQ}$ (on $\omega_2$), respectively,
such that $AX/PX = AY/QY = c$.
Extend segments $PA$ and $QA$ through $A$ to $R$ and $S$,
respectively, such that $AR = AS = c\cdot PQ$.
Given that the circumcenter of triangle $ARS$ lies on line $XY$,
prove that $\angle XPA = \angle AQY$.
\end{mdframed}
We begin as follows:
\begin{claim*}
  There is a spiral similarity centered at $X$ mapping $AR$ to $PQ$.
  Similarly there is a spiral similarity centered at $Y$ mapping $SA$ to $PQ$.
\end{claim*}
\begin{proof}
  Since $\dang XAR = \dang XAP = \dang XPQ$,
  and $AR/AX = PQ/PX$ is given.
\end{proof}
Now the composition of the two spiral similarities
\[ AR \xmapsto{X} PQ \xmapsto{Y} SA \]
must be a rotation, since $AR = AS$.
The center of this rotation must coincide with the circumcenter $O$ of $\triangle ARS$,
which is known to lie on line $XY$.

\begin{center}
\begin{asy}
import graph; size(10cm);
pen uququq = rgb(0.25098,0.25098,0.25098); pen zzttqq = rgb(0.6,0.2,0.);
draw((13.97096,-7.13004)--(20.23013,-5.53182)--(27.19232,-12.23004)--cycle, linewidth(0.6) + zzttqq);
draw(circle((25.,-3.), 9.48683), linewidth(0.6) + uququq);
draw(circle((13.,-3.), 4.24264), linewidth(0.6) + uququq);

draw((13.97096,-7.13004)--(16.,0.), linewidth(0.6) + red);
draw((13.97096,-7.13004)--(11.14589,0.81605), linewidth(0.6) + red);
draw((11.14589,0.81605)--(25.35640,-1.57297), linewidth(0.6) + blue);
draw((27.19232,-12.23004)--(20.85410,5.53296), linewidth(0.6) + red);
draw((27.19232,-12.23004)--(16.,0.), linewidth(0.6) + red);
draw((20.85410,5.53296)--(9.74299,-7.13207), linewidth(0.6) + blue);
draw((13.97096,-7.13004)--(20.23013,-5.53182), linewidth(0.6) + zzttqq);
draw((20.23013,-5.53182)--(27.19232,-12.23004), linewidth(0.6) + zzttqq);
draw((27.19232,-12.23004)--(13.97096,-7.13004), linewidth(0.6) + zzttqq);
draw((20.23013,-5.53182)--(19.26755,-9.17733), linewidth(0.6));
draw((11.14589,0.81605)--(20.85410,5.53296), linewidth(0.6));
dot("$A$", (16.,0.), dir(100));
dot("$Q$", (20.85410,5.53296), dir(120));
dot("$P$", (11.14589,0.81605), dir(120));
dot("$X$", (13.97096,-7.13004), dir(280));
dot("$Y$", (27.19232,-12.23004), dir(280));
dot("$R$", (25.35640,-1.57297), dir((8.043, 16.110)));
dot("$S$", (9.74299,-7.13207), dir(225));
dot("$O$", (19.26755,-9.17733), dir(45));
dot("$O'$", (20.23013,-5.53182), dir(135));
\end{asy}
\end{center}

As $O$ is a fixed-point of the composed map above,
we may let $O'$ be the image of $O$ under the rotation at $X$,
so that
\[ \triangle XPA \overset{+}{\sim} \triangle XO'O, \qquad
  \triangle YQA \overset{+}{\sim} \triangle YO'O. \]
Because
\[ \frac{XO}{XO'} = \frac{XA}{XP} = c \frac{YQ}{YA} = \frac{YO}{YO'} \]
it follows $\ol{O'O}$ bisects $\angle XO'Y$.
Finally, we have
\[ \dang XPA = \dang XO'O = \dang OO'Y = \dang AQY. \]

\begin{remark*}
  Indeed, this also shows $\ol{XP} \parallel \ol{YQ}$; so the positive homothety
  from $\omega_1$ to $\omega_2$ maps $P$ to $Q$ and $X$ to $Y$.
\end{remark*}
\pagebreak

\subsection{TSTST 2011/3}
\textsl{Available online at \url{https://aops.com/community/p2374845}.}
\begin{mdframed}[style=mdpurplebox,frametitle={Problem statement}]
Prove that there exists a real constant $c$
such that for any pair $(x,y)$ of real numbers,
there exist relatively prime integers $m$ and $n$
satisfying the relation
\[ \sqrt{(x-m)^2 + (y-n)^2} < c \log(x^2+y^2+2). \]
\end{mdframed}
This is actually the same problem as USAMO 2014/6. Surprise!
\pagebreak

\section{Solutions to Day 2}
\subsection{TSTST 2011/4}
\textsl{Available online at \url{https://aops.com/community/p2374848}.}
\begin{mdframed}[style=mdpurplebox,frametitle={Problem statement}]
Acute triangle $ABC$ is inscribed in circle $\omega$.
Let $H$ and $O$ denote its orthocenter and circumcenter, respectively.
Let $M$ and $N$ be the midpoints of sides $AB$ and $AC$, respectively.
Rays $MH$ and $NH$ meet $\omega$ at $P$ and $Q$, respectively.
Lines $MN$ and $PQ$ meet at $R$.
Prove that $\ol{OA} \perp \ol{RA}$.
\end{mdframed}
Let $MH$ and $NH$ meet the nine-point circle again at $P'$ and $Q'$, respectively.
Recall that $H$ is the center of the homothety between
the circumcircle and the nine-point circle.
From this we can see that $P$ and $Q$ are the images of this homothety, meaning that
\[ HQ = 2HQ' \quad\text{and}\quad HP = 2HP'. \]
Since $M$, $P'$, $Q'$, $N$ are cyclic, Power of a Point gives us
\[ MH \cdot HP' = HN \cdot HQ'. \]
Multiplying both sides by two, we thus derive
\[ HM \cdot HP = HN \cdot HQ. \]
It follows that the points $M$, $N$, $P$, $Q$ are concyclic.

\begin{center}
  \begin{asy}
    size(11cm);
    defaultpen(fontsize(8pt));
    pair A = dir(110);
    pair B = dir(210);
    pair C = dir(330);
    pair H = A+B+C;
    pair O = circumcenter(A, B, C);
    pair M = midpoint(A--B);
    pair N = midpoint(A--C);
    pair U = -C;
    pair V = -B;
    pair P = -U+2*foot(O, U, H);
    pair Q = -V+2*foot(O, V, H);
    pair R = extension(M, N, P, Q);

    draw(A--B--C--cycle, blue);
    draw(unitcircle, blue);
    draw(M--P);
    draw(N--Q);
    draw(circumcircle(M, N, midpoint(B--C)), green);
    draw(circumcircle(P, Q, M), red+dashed);
    draw(P--R--A);
    draw(R--N, dotted);
    draw(circumcircle(A, M, N), dashed);

    pair Qp = midpoint(H--Q);
    pair Pp = midpoint(H--P);

    dot("$A$", A, dir(A));
    dot("$B$", B, dir(250));
    dot("$C$", C, dir(C));
    dot("$H$", H, dir(-90));
    dot("$O$", O, dir(315));
    dot("$M$", M, 1.4*dir(150));
    dot("$N$", N, dir(20));
    dot("$P$", P, dir(P));
    dot("$Q$", Q, dir(Q));
    dot("$R$", R, dir(R));
    dot("$Q'$", Qp, dir(255));
    dot("$P'$", Pp, dir(Pp));

    /* Source generated by TSQ */
  \end{asy}
\end{center}

Let $\omega_1$, $\omega_2$, $\omega_3$ denote the circumcircles of
$MNPQ$, $AMN$, and $ABC$, respectively.
The radical axis of $\omega_1$ and $\omega_2$ is line $MN$, while the
radical axis of $\omega_1$ and $\omega_3$ is line $PQ$.
Hence the line $R$ lies on the radical axis of $\omega_2$ and $\omega_3$.

But we claim that $\omega_2$ and $\omega_3$ are internally tangent at $A$.
This follows by noting the homothety at $A$ with ratio $2$ sends
$M$ to $B$ and $N$ to $C$.
Hence the radical axis of $\omega_2$ and $\omega_3$ is a line tangent to
both circles at $A$.

Hence $\ol{RA}$ is tangent to $\omega_3$.
Therefore, $\ol{RA} \perp \ol{OA}$.
\pagebreak

\subsection{TSTST 2011/5}
\textsl{Available online at \url{https://aops.com/community/p2374849}.}
\begin{mdframed}[style=mdpurplebox,frametitle={Problem statement}]
At a certain orphanage, every pair of orphans are either friends or enemies.
For every three of an orphan's friends,
an even number of pairs of them are enemies.
Prove that it's possible to assign each orphan two parents
such that every pair of friends shares exactly one parent, but no pair of enemies does,
and no three parents are in a love triangle (where each pair of them has a child).
\end{mdframed}
Of course, we consider the graph with vertices as children and edges as friendships.
Consider all the maximal cliques in the graph
(i.e.\ repeatedly remove maximal cliques until no edges remain;
thus all edges are in some clique).

\begin{claim*}
Every vertex is in at most two maximal cliques.
\end{claim*}

\begin{proof}
Indeed, consider a vertex $v$ adjacent to $w_1$ and $w_2$,
but with $w_1$ not adjacent to $w_2$.
Then by condition, any third vertex $u$
must be adjacent to exactly one of $w_1$ and $w_2$.
Moreover, given vertices $u$ and $u'$ adjacent to $w_1$,
vertices $u$ and $u'$ are adjacent too.
This proves the claim.
\end{proof}

Now, for every maximal clique we assign a particular parent
to all vertices in that clique,
adding in additional distinct parents if there are any deficient children.
This satisfies the friendship/enemy condition.
Moreover, one can readily check that there are no love triangles:
given children $a$, $b$, $c$
such that $a$ and $b$ share a parent
while $a$ and $c$ share another parent,
according to the claim $b$ and $c$ can't share a third parent.
This completes the problem.

\begin{remark*}
This solution is highly motivated for the following reason:
by experimenting with small cases,
one quickly finds that given some vertices which form a clique,
one \emph{must} assign some particular parent
to all vertices in that clique.
That is, the requirements of the problem are sufficiently rigid
that there is no room for freedom on our part,
so we know \emph{a priori} that an assignment based on cliques
(as above) must work.
From there we know exactly what to prove,
and everything else follows through.

Ironically, the condition that there be no love triangle
actually makes the problem easier,
because it tells us exactly what to do!
\end{remark*}
\pagebreak

\subsection{TSTST 2011/6}
\textsl{Available online at \url{https://aops.com/community/p2374852}.}
\begin{mdframed}[style=mdpurplebox,frametitle={Problem statement}]
Let $a,b,c$ be real numbers in the interval $[0,1]$
with $a+b,b+c,c+a \ge 1$.  Prove that
\[ 1 \le (1-a)^2 + (1-b)^2 + (1-c)^2
  + \frac{2\sqrt2 abc}{\sqrt{a^2+b^2+c^2}}. \]
\end{mdframed}
The following approach is due to Ashwin Sah.

We will prove the inequality for any $a$, $b$, $c$
the sides of a possibly degenerate triangle
(which is implied by the condition),
ignoring the particular constant $1$.
Homogenizing, we instead prove the problem
in the following form:
\begin{claim*}
  We have
  \[ k^2 \le (k-a)^2 + (k-b)^2 + (k-c)^2
    + \frac{2\sqrt2 abc}{\sqrt{a^2+b^2+c^2}} \]
  for any $a$, $b$, $c$, $k$
  with $(a,b,c)$ the sides of a
  possibly degenerate triangle.
\end{claim*}
\begin{proof}
  For any particular $(a,b,c)$ this is a quadratic in $k$
  of the form $2k^2 - 2(a+b+c)k + C \ge 0$;
  thus we will verify it holds for $k = \half(a+b+c)$.

  Letting $x = \half(b+c-a)$ as is usual,
  the problem rearranges to
  In that case, the problem amounts to
  \[ (x+y+z)^2 \le x^2+y^2+z^2
    + \frac{2(x+y)(y+z)(z+x)}{\sqrt{x^2+y^2+z^2+xy+yz+zx}} \]
  or equivalently
  \[ x^2+y^2+z^2+xy+yz+zx
    \le \left( \frac{(x+y)(y+z)(z+x)}{xy+yz+zx} \right)^2. \]
  To show this, one may let $t = xy+yz+zx$,
  then using $(x+y)(x+z) = x^2+B$ this becomes
  \[ t^2(x^2+y^2+z^2+t)
    \le (x^2+t)(y^2+t)(z^2+t) \]
  which is obvious upon expansion.
\end{proof}

\begin{remark*}
  The inequality holds actually for
  all real numbers $a$, $b$, $c$,
  with very disgusting proofs.
\end{remark*}
\pagebreak

\section{Solutions to Day 3}
\subsection{TSTST 2011/7}
\textsl{Available online at \url{https://aops.com/community/p2374855}.}
\begin{mdframed}[style=mdpurplebox,frametitle={Problem statement}]
Let $ABC$ be a triangle.
Its excircles touch sides $BC$, $CA$, $AB$ at $D$, $E$, $F$.
Prove that the perimeter of triangle $ABC$ is
at most twice that of triangle $DEF$.
\end{mdframed}
Solution by August Chen:
It turns out that it is enough to take the
orthogonal projection of $EF$ onto side $BC$
(which has length $a-(s-a)(\cos B + \cos C)$)
and sum cyclically:
\begin{align*}
  -s + \sum_{\text{cyc}} EF &\ge
  -s +
  \sum_{\text{cyc}}
  \left[ a - (s-a)\left( \cos B + \cos C \right) \right] \\
  &= s - \sum_{\text{cyc}} a \cos A
    = \sum_{\text{cyc}} a \left( \half - \cos A \right) \\
  &= R \sum_{\text{cyc}}
  \sin A (1 - 2 \cos A) \\
  &= R \sum_{\text{cyc}} \left( \sin A - \sin2A \right).
\end{align*}
Thus we're done upon noting that
\[ \frac{\sin2B + \sin2C}{2} = \sin(B+C) \cos(B-C)
  = \sin A \cos(B-C) \le \sin A. \]
(Alternatively, one can avoid trigonometry by substituting
$\cos A = \frac{b^2+c^2-a^2}{2bc}$
and doing some routine but long calculation.)
\pagebreak

\subsection{TSTST 2011/8}
\textsl{Available online at \url{https://aops.com/community/p2374856}.}
\begin{mdframed}[style=mdpurplebox,frametitle={Problem statement}]
Let $x_0$, $x_1$, \dots, $x_{n_0-1}$ be integers,
and let $d_1$, $d_2$, \dots, $d_k$ be positive integers
with $n_0 = d_1 > d_2 > \dotsb > d_k$ and
$\gcd(d_1, d_2, \dots, d_k) = 1$.
For every integer $n \geq n_0$, define
\[ x_n = \left\lfloor \frac{x_{n-d_1} + x_{n-d_2}
  + \dots + x_{n-d_k}}{k} \right\rfloor. \]
Show that the sequence $(x_n)$ is eventually constant.
\end{mdframed}
Note that if the initial terms are contained
in some interval $[A,B]$ then they will remain in that interval.
Thus the sequence is eventually periodic.
Discard initial terms and let the period be $T$;
we will consider all indices modulo $T$ from now on.

Let $M$ be the maximal term in the sequence
(which makes sense since the sequence is periodic).
Note that if $x_n = M$, we must have
$x_{n-d_i} = M$ for all $i$ as well.
By taking a linear combination $\sum c_i d_i \equiv 1 \pmod T$
(possibly be Bezout's theorem, since $\gcd_i(d_i)=1$),
we conclude $x_{n-1} = M$, as desired.
\pagebreak

\subsection{TSTST 2011/9}
\textsl{Available online at \url{https://aops.com/community/p2374857}.}
\begin{mdframed}[style=mdpurplebox,frametitle={Problem statement}]
Let $n$ be a positive integer.
Suppose we are given $2^n+1$ distinct sets,
each containing finitely many objects.
Place each set into one of two categories, the red sets and the blue sets,
so that there is at least one set in each category.
We define the \textit{symmetric difference} of two sets as
the set of objects belonging to exactly one of the two sets.
Prove that there are at least $2^n$ different sets which
can be obtained as the symmetric difference of a red set and a blue set.
\end{mdframed}
We can interpret the problem as working with
binary strings of length $\ell \ge n+1$,
with $\ell$ the number of elements across all sets.

Let $F$ be a field of cardinality $2^\ell$,
hence $F \cong \FF_2^{\oplus \ell}$.

Then, we can think of red/blue as elements of $F$,
so we have some $B \subseteq F$, and an $R \subseteq F$.
We wish to prove that $|B+R| \ge 2^n$.
Want $|B + R| \ge 2^n$.

Equivalently, any element of a set $X$ with $|X| = 2^n - 1$
should omit some element of $|B+R|$.
To prove this: we know $|B| + |R| = 2^n+1$, and define
\[ P(b,r) = \prod_{x \in X} (b+r-x). \]
Consider $b^{|B|-1} r^{|R|-1}$.
The coefficient of is $\binom{2^n-1}{|B|-1}$,
which is odd (say by Lucas theorem),
so the nullstellensatz applies.
\pagebreak


\end{document}
