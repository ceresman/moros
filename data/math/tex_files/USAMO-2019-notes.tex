% © Evan Chen
% Downloaded from https://web.evanchen.cc/

\documentclass[11pt]{scrartcl}
\usepackage[sexy]{evan}
\ihead{\footnotesize\textbf{\thetitle}}
\ohead{\footnotesize\href{http://web.evanchen.cc}{\ttfamily web.evanchen.cc},
    updated \today}
\title{USAMO 2019 Solution Notes}
\date{\today}

\begin{document}

\maketitle

\begin{abstract}
This is a compilation of solutions
for the 2019 USAMO.
Some of the solutions are my own work,
but many are from the official solutions provided by the organizers
(for which they hold any copyrights),
and others were found by users on the Art of Problem Solving forums.

These notes will tend to be a bit more advanced and terse than the ``official''
solutions from the organizers.
In particular, if a theorem or technique is not known to beginners
but is still considered ``standard'', then I often prefer to
use this theory anyways, rather than try to work around or conceal it.
For example, in geometry problems I typically use directed angles
without further comment, rather than awkwardly work around configuration issues.
Similarly, sentences like ``let $\mathbb{R}$ denote the set of real numbers''
are typically omitted entirely.

Corrections and comments are welcome!
\end{abstract}

\tableofcontents
\newpage

\addtocounter{section}{-1}
\section{Problems}
\begin{enumerate}[\bfseries 1.]
\item %% Problem 1
A function $f \colon \NN \to \NN$ satisfies
\[ \underbrace{f(f(\dots f}_{f(n)\text{ times}} (n)\dots)) %chktex 9
  = \frac{n^2}{f(f(n))} \]
for all positive integers $n$.
What are all possible values of $f(1000)$?

\item %% Problem 2
Let $ABCD$ be a cyclic quadrilateral
satisfying $AD^2 + BC^2 = AB^2$.
The diagonals of $ABCD$ intersect at $E$.
Let $P$ be a point on side $\ol{AB}$
satisfying $\angle APD = \angle BPC$.
Show that line $PE$ bisects $\ol{CD}$.

\item %% Problem 3
Let $K$ be the set of positive integers not containing the decimal digit $7$.
Determine all polynomials $f(x)$ with nonnegative coefficients
such that $f(x) \in K$ for all $x \in K$.

\item %% Problem 4
Let $n$ be a nonnegative integer.
Determine the number of ways to choose sets
$S_{ij} \subseteq \{1, 2, \dots, 2n\}$,
for all $0 \le i \le n$ and $0 \le j \le n$
(not necessarily distinct), such that
\begin{itemize}
  \ii $|S_{ij}| = i+j$, and
  \ii $S_{ij} \subseteq S_{kl}$ if $0 \le i \le k \le n$
  and $0 \le j \le l \le n$.
\end{itemize}

\item %% Problem 5
Let $m$ and $n$ be relatively prime positive integers.
The numbers $\frac mn$ and $\frac nm$ are written on a blackboard.
At any point, Evan may pick two of the numbers $x$ and $y$
written on the board and write either their arithmetic mean $\half(x+y)$
or their harmonic mean $\frac{2xy}{x+y}$.
For which $(m,n)$ can Evan write $1$ on the board in finitely many steps?

\item %% Problem 6
Find all polynomials $P$ with real coefficients such that
\[ \frac{P(x)}{yz} + \frac{P(y)}{zx} + \frac{P(z)}{xy}
  = P(x-y) + P(y-z) + P(z-x) \]
for all nonzero real numbers $x$, $y$, $z$
obeying $2xyz = x+y+z$.

\end{enumerate}
\pagebreak

\section{Solutions to Day 1}
\subsection{USAMO 2019/1, proposed by Evan Chen}
\textsl{Available online at \url{https://aops.com/community/p12189527}.}
\begin{mdframed}[style=mdpurplebox,frametitle={Problem statement}]
A function $f \colon \NN \to \NN$ satisfies
\[ \underbrace{f(f(\dots f}_{f(n)\text{ times}} (n)\dots)) %chktex 9
  = \frac{n^2}{f(f(n))} \]
for all positive integers $n$.
What are all possible values of $f(1000)$?
\end{mdframed}
Actually, we classify all such functions:
$f$ can be any function which fixes odd integers
and acts as an involution on the even integers.
In particular, $f(1000)$ may be any even integer.

It's easy to check that these all work,
so now we check they are the only solutions.

\begin{claim*}
  $f$ is injective.
\end{claim*}
\begin{proof}
  If $f(a) = f(b)$,
  then $a^2 = f^{f(a)}(a) f(f(a)) = f^{f(b)}(b) f(f(b)) = b^2$,
  so $a = b$.
\end{proof}

\begin{claim*}
  $f$ fixes the odd integers.
\end{claim*}
\begin{proof}
  We prove this by induction on odd $n \ge 1$.

  Assume $f$ fixes $S = \{1,3,\dots,n-2\}$ now
  (allowing $S = \varnothing$ for $n=1$).
  Now we have that
  \[ f^{f(n)}(n) \cdot f^2(n) = n^2. \]
  However, neither of the two factors on the left-hand
  side can be in $S$ since $f$ was injective.
  Therefore they must both be $n$,
  and we have $f^2(n) = n$.

  Now let $y = f(n)$, so $f(y) = n$.
  Substituting $y$ into the given yields
  \[ y^2 = f^n(y) \cdot y =
    f^{n+1}(n) \cdot y = ny \]
  since $n+1$ is even.
  We conclude $n=y$, as desired.
\end{proof}

Thus, $f$ maps even integers to even integers.
In light of this, we may let $g = f(f(n))$
(which is also injective),
so we conclude that
\[ g^{f(n)/2} (n) g(n) = n^2 \qquad \text{ for } n = 2, 4, \dots. \]
\begin{claim*}
  The function $g$ is the identity function.
\end{claim*}
\begin{proof}
  The proof is similar to the earlier proof of the claim.
  Note that $g$ fixes the odd integers already.
  We proceed by induction to show $g$ fixes the even integers;
  so assume $g$ fixes the set $S = \{1, 2, \dots, n-1\}$,
  for some even integer $n \ge 2$.
  In the equation
  \[ g^{f(n)/2}(n) \cdot g(n) = n^2 \]
  neither of the two factors may be less than $n$.
  So they must both be $n$.
\end{proof}

These three claims imply that the solutions we claimed earlier
are the only ones.

\begin{remark*}
  The last claim is not necessary to solve the problem;
  after realizing $f$ and injective fixes the odd integers,
  this answers the question about the values of $f(1000)$.
  However, we chose to present the ``full'' solution anyways.
\end{remark*}

\begin{remark*}
  After noting $f$ is injective, another approach is outlined below.
  Starting from any $n$, consider the sequence
  \[ n, \; f(n), \; f(f(n)), \; \]
  and so on.
  We may let $m$ be the smallest term of the sequence;
  then $m^2 = f(f(m)) \cdot f^{f(m)}(m)$
  which forces $f(f(m)) = f^{f(m)}(m) = m$ by minimality.
  Thus the sequence is $2$-periodic.
  Therefore, $f(f(n)) = n$ always holds,
  which is enough to finish.
\end{remark*}

\paragraph{Authorship comments.}
I will tell you a great story about this problem.
Two days before the start of grading of USAMO 2017,
I had a dream that I was grading a functional equation.
When I woke up, I wrote it down, and it was
\[ f^{f(n)}(n) = \frac{n^2}{f(f(n))}. \]
You can guess the rest of the story
(and imagine how surprised I was the solution set was interesting).
%Much to my surprise, when I woke up and remembered my dream
%I was actually able to solve the problem
%(and found that the solution set was nontrivial).
I guess some dreams do come true, huh?
\pagebreak

\subsection{USAMO 2019/2, proposed by Ankan Bhattacharya}
\textsl{Available online at \url{https://aops.com/community/p12189455}.}
\begin{mdframed}[style=mdpurplebox,frametitle={Problem statement}]
Let $ABCD$ be a cyclic quadrilateral
satisfying $AD^2 + BC^2 = AB^2$.
The diagonals of $ABCD$ intersect at $E$.
Let $P$ be a point on side $\ol{AB}$
satisfying $\angle APD = \angle BPC$.
Show that line $PE$ bisects $\ol{CD}$.
\end{mdframed}
Here are three solutions.
The first two are similar although the first one makes use of symmedians.
The last solution by inversion is more advanced.

\paragraph{First solution using symmedians.}
We define point $P$ to obey
\[ \frac{AP}{BP} = \frac{AD^2}{BC^2} = \frac{AE^2}{BE^2} \]
so that $\ol{PE}$ is the $E$-symmedian of $\triangle EAB$,
therefore the $E$-median of $\triangle ECD$.

Now, note that
\[ AD^2 = AP \cdot AB \quad\text{and}\quad BC^2 = BP \cdot BA. \]
This implies $\triangle APD \sim \triangle ADB$ and $\triangle BPC \sim \triangle BCA$.
Thus
\[ \dang DPA = \dang ADB = \dang ACB = \dang BCP \]
and so $P$ satisfies the condition as in the statement
(and is the unique point to do so), as needed.

\paragraph{Second solution using only angle chasing (by proposer).}
We again re-define $P$ to obey $AD^2 = AP \cdot AB$ and $BC^2 = BP \cdot BA$.
As before, this gives $\triangle APD \sim \triangle ABD$
and $\triangle BPC \sim \triangle BDP$ and so we let
\[ \theta \coloneqq \dang DPA = \dang ADB = \dang ACB = \dang BCP. \]
Our goal is to now show $\ol{PE}$ bisects $\ol{CD}$.

Let $K = \ol{AC} \cap \ol{PD}$ and $L = \ol{AD} \cap \ol{PC}$.
Since $\dang KPA = \theta = \dang ACB$, quadrilateral $BPKC$ is cyclic.
Similarly, so is $APLD$.

\begin{center}
\begin{asy}
size(8cm);

pair A = dir(200);
pair B = -conj(A);
pair D = dir(140);
pair O = origin;
pair P = extension(D, foot(D, A, O), A, B);
pair Z = 100*foot(P,O,B)-99*P;
pair C = IP(P--Z, unitcircle);

markangle(13.0, A, D, B, red);
markangle(13.0, A, C, B, red);
markangle(13.0, D, P, A, red);
markangle(13.0, B, P, C, red);

filldraw(unitcircle, opacity(0.1)+palecyan, deepcyan);
draw(A--B--C--D--cycle, deepcyan);
draw(A--C, deepgreen);
draw(B--D, deepgreen);
pair K = extension(D, P, A, C);
pair L = extension(C, P, D, B);
draw(D--P--C, blue);
pair E = extension(A, C, B, D);

dot("$A$", A, dir(A));
dot("$B$", B, dir(B));
dot("$D$", D, dir(D));
dot("$P$", P, dir(270));
dot("$C$", C, dir(C));
dot("$K$", K, dir(K));
dot("$L$", L, dir(20));
dot("$E$", E, dir(90));

/* TSQ Source:

!size(8cm);

A = dir 200
B = -conj(A)
D = dir 140
O := origin
P 270 = extension D foot D A O A B
Z := 100*foot(P,O,B)-99*P
C = IP P--Z unitcircle

!markangle(13.0, A, D, B, red);
!markangle(13.0, A, C, B, red);
!markangle(13.0, D, P, A, red);
!markangle(13.0, B, P, C, red);

unitcircle 0.1 palecyan / deepcyan
A--B--C--D--cycle deepcyan
A--C deepgreen
B--D deepgreen
K = extension D P A C
L 20 = extension C P D B
D--P--C blue
E = extension A C B D R90

*/
\end{asy}
\end{center}

Finally $AKLB$ is cyclic since
\[ \dang BKA = \dang BKC = \dang BPC = \theta = \dang DPA = \dang DLA = \dang BLA. \]
This implies $\dang CKL = \dang LBA = \dang DCK$, so $\ol{KL} \parallel \ol{BC}$.
Then $PE$ bisects $\ol{BC}$ by Ceva's theorem on $\triangle PCD$.

\paragraph{Third solution (using inversion).}
By hypothesis, the circle $\omega_a$ centered at $A$ with radius $AD$
is orthogonal to the circle $\omega_b$ centered at $B$ with radius $BC$.
For brevity, we let $\mathbf{I}_a$ and $\mathbf{I}_b$
denote inversion with respect to $\omega_a$ and $\omega_b$.

We let $P$ denote the intersection of $\ol{AB}$ with the radical axis of $\omega_a$ and $\omega_b$;
hence $P = \mathbf{I}_a(B) = \mathbf{I}_b(A)$.
This already implies that
\[ \dang DPA \overset{\mathbf{I}_a}{=} \dang ADB = \dang ACB \overset{\mathbf{I}_b}{=} \dang BPC \]
so $P$ satisfies the angle condition.

\begin{center}
\begin{asy}
size(10cm);

pair A = dir(200);
pair B = -conj(A);
pair D = dir(140);
pair O = origin;
pair P = extension(D, foot(D, A, O), A, B);
pair Z = 100*foot(P,O,B)-99*P;
pair C = IP(P--Z, unitcircle);
filldraw(unitcircle, opacity(0.1)+palecyan, deepcyan);
draw(A--B--C--D--cycle, deepcyan);
filldraw(CP(A, D), opacity(0.1)+lightred, red);
filldraw(CP(B, C), opacity(0.1)+orange, orange);
pair X = IP(CP(A, D), CP(B, C));
pair Y = OP(CP(A, D), CP(B, C));
draw(X--Y, yellow);
draw(A--C, deepcyan+dotted);
draw(B--D, deepcyan+dotted);
pair K = extension(D, P, A, C);
pair L = extension(C, P, D, B);
draw(D--P, red+dashed);
draw(C--P, orange+dashed);

dot("$A$", A, dir(A));
dot("$B$", B, dir(B));
dot("$D$", D, dir(D));
dot("$P$", P, dir(270));
dot("$C$", C, dir(C));
dot(X);
dot(Y);
dot("$K$", K, dir(K));
dot("$L$", L, dir(20));

/* TSQ Source:

!size(12cm);

A = dir 200
B = -conj(A)
D = dir 140
O := origin
P 270 = extension D foot D A O A B
Z := 100*foot(P,O,B)-99*P
C = IP P--Z unitcircle
unitcircle 0.1 palecyan / deepcyan
A--B--C--D--cycle deepcyan
CP A D 0.1 lightred / red
CP B C 0.1 orange / orange
X .= IP CP A D CP B C
Y .= OP CP A D CP B C
X--Y yellow
A--C deepcyan dotted
B--D deepcyan dotted
K = extension D P A C
L 20 = extension C P D B
D--P red dashed
C--P orange dashed

*/
\end{asy}
\end{center}

\begin{claim*}
  The point $K = \mathbf{I}_a(C)$ lies on $\omega_b$ and $\ol{DP}$.
  Similarly $L = \mathbf{I}_b(D)$ lies on  $\omega_a$ and $\ol{CP}$.
\end{claim*}
\begin{proof}
  The first assertion follows from the fact that $\omega_b$ is orthogonal to $\omega_a$.
  For the other, since $(BCD)$ passes through $A$, it follows $P = \mathbf{I}_a(B)$, $K = \mathbf{I}_a(C)$,
  and $D = \mathbf{I}_a(D)$ are collinear.
\end{proof}

Finally, since $C$, $L$, $P$ are collinear,
we get $A$ is concyclic with $K = \mathbf{I}_a(C)$,
$L = \mathbf{I}_a(L)$, $B = \mathbf{I}_a(B)$, i.e.\ that $AKLB$ is cyclic.
So $\ol{KL} \parallel \ol{CD}$ by Reim's theorem,
and hence $\ol{PE}$ bisects $\ol{CD}$ by Ceva's theorem.
\pagebreak

\subsection{USAMO 2019/3, proposed by Titu Andreescu, Vlad Matei, Cosmin Pohoata}
\textsl{Available online at \url{https://aops.com/community/p12189457}.}
\begin{mdframed}[style=mdpurplebox,frametitle={Problem statement}]
Let $K$ be the set of positive integers not containing the decimal digit $7$.
Determine all polynomials $f(x)$ with nonnegative coefficients
such that $f(x) \in K$ for all $x \in K$.
\end{mdframed}
The answer is only the obvious ones:
$f(x) = 10^e x$, $f(x) = k$,
and $f(x) = 10^e x + k$, for any choice of $k \in K$
and $e > \log_{10} k$ (with $e \ge 0$).

Now assume $f$ satisfies $f(K) \subseteq K$;
such polynomials will be called \emph{stable}.
We first prove the following claim
which reduces the problem to the study of monomials.
\begin{lemma*}
  [Reduction to monomials]
  If $f(x) = a_0 + a_1 x + a_2 x^2 + \dots$ is stable,
  then each monomial $a_0$, $a_1 x$, $a_2 x^2$, \dots is stable.
\end{lemma*}
\begin{proof}
  For any $x \in K$, plug in $f(10^e x)$ for large enough $e$:
  the decimal representation of $f$ will contain
  $a_0$, $a_1 x$, $a_2 x^2$ with some zeros padded in between.
\end{proof}

Let's tackle the linear case next.
Here is an ugly but economical proof.
\begin{claim*}
  [Linear classification]
  If $f(x) = cx$ is stable, then $c = 10^e$
  for some nonnegative integer $e$.
\end{claim*}
\begin{proof}
  We will show when $c \neq 10^e$ then we can find $x \in K$
  such that $cx$ starts with the digit $7$.
  This can actually be done with the following explicit cases
  in terms of how $c$ starts in decimal notation:
  \begin{itemize}
    \ii For $9 \cdot 10^e \le c < 10 \cdot 10^e$, pick $x = 8$.
    \ii For $8 \cdot 10^e \le c < 9 \cdot 10^e$, pick $x = 88$.
    \ii For $7 \cdot 10^e \le c < 8 \cdot 10^e$, pick $x = 1$.
    \ii For $4.4 \cdot 10^e \le c < 7 \cdot 10^e$, pick $11 \le x \le 16$.
    \ii For $2.7 \cdot 10^e \le c < 4.4 \cdot 10^e$, pick $18 \le x \le 26$.
    \ii For $2 \cdot 10^e \le c < 2.7 \cdot 10^e$, pick $28 \le x \le 36$.
    \ii For $1.6 \cdot 10^e \le c < 2 \cdot 10^e$, pick $38 \le x \le 46$.
    \ii For $1.3 \cdot 10^e \le c < 1.6 \cdot 10^e$, pick $48 \le x \le 56$.
    \ii For $1.1 \cdot 10^e \le c < 1.3 \cdot 10^e$, pick $58 \le x \le 66$.
    \ii For $1 \cdot 10^e \le c < 1.1 \cdot 10^e$, pick $x = 699\dots9$
    for suitably many $9$'s. \qedhere
  \end{itemize}
\end{proof}

The hardest part of the problem is the case where $\deg f > 1$.
We claim that no solutions exist then:
\begin{claim*}
  [Higher-degree classification]
  No monomial of the form $f(x) = cx^d$ is stable
  for any $d > 1$.
\end{claim*}
\begin{proof}
  Note that $f(10x+3)$ is stable too.
  Thus
  \[ f(10x+3) = 3^d + 10d \cdot 3^{d-1} x
    + 100 \binom d2 \cdot 3^{d-1} x^2 + \dots \]
  is stable.
  By applying the lemma the linear monomial
  $10d \cdot 3^{d-1} x$ is stable,
  so $10d \cdot 3^{d-1}$ is a power of $10$,
  which can only happen if $d = 1$.
\end{proof}

Thus the only nonconstant stable polynomials with nonnegative coefficients
must be of the form $f(x) = 10^e x + k$ for $e \ge 0$.
It is straightforward to show we then need $k < 10^e$
and this finishes the proof.

\begin{remark*}
  The official solution replaces the proof
  for $f(x) = cx$ with Kronecker density.
  From $f(1) = c \in K$, we get $f(c) = c^2 \in K$,
  et cetera and hence $c^n \in K$.
  But it is known that when $c$ is not a power of $10$,
  some power of $c$ starts with any specified prefix.
\end{remark*}
\pagebreak

\section{Solutions to Day 2}
\subsection{USAMO 2019/4, proposed by Ricky Liu}
\textsl{Available online at \url{https://aops.com/community/p12195861}.}
\begin{mdframed}[style=mdpurplebox,frametitle={Problem statement}]
Let $n$ be a nonnegative integer.
Determine the number of ways to choose sets
$S_{ij} \subseteq \{1, 2, \dots, 2n\}$,
for all $0 \le i \le n$ and $0 \le j \le n$
(not necessarily distinct), such that
\begin{itemize}
  \ii $|S_{ij}| = i+j$, and
  \ii $S_{ij} \subseteq S_{kl}$ if $0 \le i \le k \le n$
  and $0 \le j \le l \le n$.
\end{itemize}
\end{mdframed}
The answer is $(2n)! \cdot 2^{n^2}$.
First, we note that
$\varnothing = S_{00} \subsetneq S_{01} \subsetneq \dots \subsetneq S_{nn}
= \left\{ 1, \dots, 2n \right\}$
and thus multiplying by $(2n)!$
we may as well assume $S_{0i} = \left\{ 1, \dots, i \right\}$
and $S_{in} = \left\{ 1, \dots, n+i \right\}$.
We illustrate this situation by placing the sets in a grid,
as below for $n = 4$;
our goal is to fill in the rest of the grid.
\[
  \begin{bmatrix}
    1234 & 12345 & 123456 & 1234567 & 12345678\\
    123 \\
    12 \\
    1 \\
    \varnothing
  \end{bmatrix}
\]
We claim the number of ways to do so is $2^{n^2}$.
In fact, more strongly even the partial fillings
are given exactly by powers of $2$.
\begin{claim*}
  Fix a choice $T$ of cells we wish to fill in,
  such that whenever a cell is in $T$,
  so are all the cells above and left of it.
  (In other words, $T$ is a Young tableau.)
  The number of ways to fill in these cells with sets
  satisfying the inclusion conditions is $2^{|T|}$.
\end{claim*}
An example is shown below, with an indeterminate set marked in red
(and the rest of $T$ marked in blue).
\[
  \begin{bmatrix}
    1234 & 12345 & 123456 & 1234567 & 12345678\\
    123 & {\color{blue}1234} & {\color{blue}12346} & {\color{blue}123467} \\
    12 & {\color{blue}124} & {\color{red}1234 \text{ or } 1246} \\
    1 & {\color{blue}12} \\
    \varnothing & {\color{blue}2}
  \end{bmatrix}
\]
\begin{proof}
  The proof is by induction on $|T|$,
  with $|T| = 0$ being vacuous.

  Now suppose we have a corner $\begin{bmatrix}
  B & C \\ A & {\color{red}S} \end{bmatrix}$
  where $A$, $B$, $C$ are fixed and $S$ is to be chosen.
  Then we may write $B = A \cup \{x\}$ and $C = A \cup \{x,y\}$
  for $x,y \notin A$.
  Then the two choices of $S$ are $A \cup \{x\}$ (i.e.\ $B$)
  and $A \cup \{y\}$, and both of them are seen to be valid.

  In this way, we gain a factor of $2$
  any time we add one cell as above to $T$.
  Since we can achieve any Young tableau in this way,
  the induction is complete.
\end{proof}
\pagebreak

\subsection{USAMO 2019/5, proposed by Yannick Yao}
\textsl{Available online at \url{https://aops.com/community/p12195834}.}
\begin{mdframed}[style=mdpurplebox,frametitle={Problem statement}]
Let $m$ and $n$ be relatively prime positive integers.
The numbers $\frac mn$ and $\frac nm$ are written on a blackboard.
At any point, Evan may pick two of the numbers $x$ and $y$
written on the board and write either their arithmetic mean $\half(x+y)$
or their harmonic mean $\frac{2xy}{x+y}$.
For which $(m,n)$ can Evan write $1$ on the board in finitely many steps?
\end{mdframed}
We claim this is possible if and only $m+n$ is a power of $2$.
Let $q = m/n$, so the numbers on the board
are $q$ and $1/q$.

\textbf{Impossibility}:
The main idea is the following.
\begin{claim*}
  Suppose $p$ is an odd prime.
  Then if the initial numbers on the board are $-1 \pmod p$,
  then all numbers on the board are $-1 \pmod p$.
\end{claim*}
\begin{proof}
  Let $a \equiv b \equiv -1 \pmod p$.
  Note that $2 \not\equiv 0 \pmod p$
  and $a+b \equiv -2 \not\equiv 0 \pmod p$.
  Thus $\frac{a+b}{2}$ and $\frac{2ab}{a+b}$
  both make sense modulo $p$ and are equal to $-1 \pmod p$.
\end{proof}

Thus if there exists \emph{any} odd prime divisor
$p$ of $m+n$
(implying $p \nmid mn$), then
\[ q \equiv \frac1q \equiv -1 \pmod p. \]
and hence all numbers will be $-1 \pmod p$ forever.
This implies that it's impossible to write $1$,
whenever $m+n$ is divisible by some odd prime.

\bigskip

\textbf{Construction}:
Conversely, suppose $m+n$ is a power of $2$.
We will actually construct $1$ without even using the harmonic mean.
\begin{center}
\begin{asy}
  unitsize(1.2cm);
  draw( (-4,0)--(4,0) );
  dot("$q$", (-4,0), dir(-90), blue);
  dot("$q^{-1}$", (4,0), dir(-90), blue);
  dot("$\frac{q+q^{-1}}{2}$", (0,0), dir(-90), blue);
  dot("$\frac{3q+q^{-1}}{4}$", (-2,0), dir(-90), red);
  dot("$\frac{q+3q^{-1}}{4}$", (2,0), dir(-90), red);
  dot("$\frac{7q+q^{-1}}{8}$", (-3,0), dir(90), deepgreen);
  dot("$\frac{5q+3q^{-1}}{8}$", (-1,0), dir(90), deepgreen);
  dot("$\frac{3q+5q^{-1}}{8}$", ( 1,0), dir(90), deepgreen);
  dot("$\frac{q+7q^{-1}}{8}$", ( 3,0), dir(90), deepgreen);
\end{asy}
\end{center}
Note that
\[
  \frac{n}{m+n} \cdot q
  + \frac{m}{m+n} \cdot \frac{1}{q}
  = 1
\]
and obviously by taking appropriate midpoints
(in a binary fashion) we can achieve this using
arithmetic mean alone.
\pagebreak

\subsection{USAMO 2019/6, proposed by Titu Andreescu, Gabriel Dospinescu}
\textsl{Available online at \url{https://aops.com/community/p12195858}.}
\begin{mdframed}[style=mdpurplebox,frametitle={Problem statement}]
Find all polynomials $P$ with real coefficients such that
\[ \frac{P(x)}{yz} + \frac{P(y)}{zx} + \frac{P(z)}{xy}
  = P(x-y) + P(y-z) + P(z-x) \]
for all nonzero real numbers $x$, $y$, $z$
obeying $2xyz = x+y+z$.
\end{mdframed}
The given can be rewritten as saying that
\begin{align*}
  Q(x,y,z) &\coloneqq xP(x) + yP(y) + zP(z) \\
  &- xyz \left( P(x-y) + P(y-z) + P(z-x) \right)
\end{align*}
is a polynomial vanishing whenever $xyz \neq 0$
and $2xyz = x+y+z$,
for real numbers $x$, $y$, $z$.
\begin{claim*}
  This means $Q(x,y,z)$ vanishes also
  for any complex numbers $x$, $y$, $z$
  obeying $2xyz=x+y+z$.
\end{claim*}
\begin{proof}
  Indeed, this means that the rational function
  \[ R(x,y) \coloneqq Q\left( x,y,\frac{x+y}{2xy-1} \right) \]
  vanishes for any real numbers $x$ and $y$
  such that $xy \neq \half$, $x \neq 0$, $y \neq 0$, $x+y \neq 0$.
  This can only occur if $R$ is identically zero
  as a rational function with real coefficients.
  If we then regard $R$ as having complex coefficients,
  the conclusion then follows.
\end{proof}
\begin{remark*}
  [Algebraic geometry digression on real dimension]
  Note here we use in an essential way that $z$
  can be solved for in terms of $x$ and $y$.
  If $s(x,y,z) = 2xyz-(x+y+z)$ is replaced with some general condition,
  the result may become false; e.g.\ we would
  certainly not expect the result to hold when
  $s(x,y,z) = x^2+y^2+z^2-(xy+yz+zx)$
  since for real numbers $s = 0$ only when $x=y=z$!

  The general condition we need here is that $s(x,y,z) = 0$
  should have ``real dimension two''.
  Here is a proof using this language, in our situation.

  Let $M \subset \RR^3$ be the surface $s=0$.
  We first contend $M$ is two-dimensional manifold.
  Indeed, the gradient $\nabla s = \left< 2yz-1, 2zx-1, 2xy-1 \right>$
  vanishes only at the points $(\pm 1/\sqrt2, \pm 1/\sqrt2, \pm 1/\sqrt2)$
  where the $\pm$ signs are all taken to be the same.
  These points do not lie on $M$,
  so the result follows by the \emph{regular value theorem}.
  In particular the topological closure of points on $M$
  with $xyz \neq 0$ is all of $M$ itself;
  so $Q$ vanishes on all of $M$.

  If we now identify $M$ with the semi-algebraic set consisting of
  maximal ideals $(x-a,y-b,z-c)$ in $\opname{Spec} \RR[x,y,z]$
  satisfying $2abc = a+b+c$, then we have
  \href{https://en.wikipedia.org/wiki/Dimension_of_an_algebraic_variety#Real_dimension}{real dimension} two,
  and thus the Zariski closure of $M$
  is a two-dimensional closed subset of $\opname{Spec} \RR[x,y,z]$.
  Thus it must be $Z = \mathcal V(2xyz-(x+y+z))$,
  since this $Z$ is an irreducible two-dimensional closed subset
  (say, by \emph{Krull's principal ideal theorem}) containing $M$.
  Now $Q$ is a global section vanishing on all of $Z$,
  therefore $Q$ is contained in the
  (radical, principal) ideal $(2xyz-(x+y+z))$ as needed.
  So it is actually divisible by $2xyz-(x+y+z)$ as desired.
\end{remark*}

Now we regard $P$ and $Q$ as complex polynomials instead.
First, note that substituting $(x,y,z) = (t,-t,0)$ implies $P$ is even.
We then substitute
\[ (x,y,z) = \left(x, \frac{i}{\sqrt2}, \frac{-i}{\sqrt2}\right) \]
to get
\begin{align*}
  &\phantom= xP(x) + \frac{i}{\sqrt2}
  \left( P\left( \frac{i}{\sqrt2} \right)
    - P\left( \frac{-i}{\sqrt2} \right) \right) \\
  &= \half x\left( P(x-i/\sqrt2) + P(x+i/\sqrt2) + P(\sqrt2i) \right)
\end{align*}
which in particular implies that
\[ P\left( x + \frac{i}{\sqrt2} \right)
  + P\left( x - \frac{i}{\sqrt2} \right)
  - 2P(x) \equiv P(\sqrt 2i) \]
identically in $x$.
The left-hand side is a second-order finite difference in $x$
(up to scaling the argument),
and the right-hand side is constant, so this implies $\deg P \le 2$.

Since $P$ is even and $\deg P \le 2$,
we must have $P(x) = cx^2 + d$ for some real numbers $c$ and $d$.
A quick check now gives the answer $P(x) = c(x^2+3)$
which all work.
\pagebreak


\end{document}
